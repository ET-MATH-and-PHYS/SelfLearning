%%%%%%%%%%%%%%%%%%%%% chapter.tex %%%%%%%%%%%%%%%%%%%%%%%%%%%%%%%%%
%
% sample chapter
%
% Use this file as a template for your own input.
%
%%%%%%%%%%%%%%%%%%%%%%%% Springer-Verlag %%%%%%%%%%%%%%%%%%%%%%%%%%
%\motto{Use the template \emph{chapter.tex} to style the various elements of your chapter content.}
\chapter{Toward Affine Schemes}
\label{AffSchem} % Always give a unique label
% use \chaptermark{}
% to alter or adjust the chapter heading in the running head

\section{Toward Schemes}

A \textbf{scheme} is a type of geometric space which is central to algebraic geometry. We hope to abstract the notion of a manifold to nonsmooth and arithmetic objects. A key insight for this endeavour is that we can understand a geometric space well by understanding the functions on this space. More precisely we will use the sheaf of functions on the space.

Thus, a scheme will consist of the following data: \begin{itemize}
    \item The set: the points of the scheme
    \item The topology: the open sets of the scheme
    \item The structure sheaf: the sheaf of ``algebraic functions" (a sheaf of rings) on the scheme
\end{itemize}

Recall that a topological space with a sheaf of rings is called a \textbf{ringed space}. In the key example of complex affine varieties, the points are the ``traditional points" cut out by polynomials in $\C^n$, plus some extra points. The topology is given by specifying that ``the subset where an algebraic function vanishes must be closed," and require nothing else. For the sheaf of algebraic functions, we will expect that in the complex plane, $(3x^2+y^2)/(2x+4xy+1)$ should be an algebraic function on the open set consisting of points where the denominator doesn't vanish, and this will largely motivate our definition. 

\begin{example}
    As motivation we return to the example of differentiable manifolds. Suppose $X$ is a differentiable manifold. It is a topological space, and has a sheaf of differentiable functions $\mathscr{O}_X$. This gives $X$ the structure of a ringed space. Recall the evaluation at a point $p \in X$ gives a surjective map from the stalk to $\R$, $\mathscr{O}_{X,p}\twoheadrightarrow \R$, so the kernel, the germs of functinos vanishing at $p$, is a maximal ideal $\mathfrak{m}_{X,p}$.

    We could define a differentiable real manifold as a topological space $X$ with a sheaf of rings. We would require that there is a cover of $X$ by open sets such that on each open set the ringed space is isomorphic to a ball around the origin in $\R^n$ (with the sheaf of differentiable functions on that ball). With this definition, the ball is the basic patch, and a general manifold is obtained by gluing these patches together.
\end{example}

In the algebraic setting, the basic patch is the notion of an \textbf{affine scheme}. Note that classically, functions are determined by their values at points. This won't be true for schemes in general.

Now, how can we describe differentiable maps of manifolds $\pi:X\rightarrow Y$? They are certainly continuous maps, but which ones? Recall we can pull back functions along continuous maps. Differentiable functions pull back to differentiable functions. More formally, we have a map $\pi^{-1}\mathscr{O}_Y\rightarrow \mathscr{O}_X$. Inverse image is left-adjoint to pushforward, so we also get a map $\pi^{\#}:\mathscr{O}_Y\rightarrow \pi_*\mathscr{O}_X$.

Note that manifolds are covered by disks that are all isomorphic. This isn't true for schemes in general. Informally, this is because in the Zariski topology on schemes, all nonempty open sets are huge and have more ``structure."

\section{The Underlying Set of Affine Schemes}

For any ring $A$ we are going to define something called $\text{Spec}\,A$, the \textbf{spectrum of $A$}. Together with a suitable topology and sheaf of rings such an object is called an \textbf{affine scheme}. The set $\text{Spec}\,A$ is the set of prime ideals of $A$. The prime ideal $\mathfrak{p}$ of $A$ when considered as an element of $\text{Spec}\,A$ will be denoted $[\mathfrak{p}]$, to avoid confusion. Elements $a \in A$ will be called \textbf{regular functions on $\text{Spec}\,A$}, and their \textbf{value} at the point $[\mathfrak{p}]$ will be $a\mod \mathfrak{p}$.

``An element $a$ of the ring lying in a prime ideal $\mathfrak{p}$" translates to ``a function $a$ that is $0$ at the point $[\mathfrak{p}]$" or ``a function $a$ vanishing at the point $[\mathfrak{p}]$." Note that addition and multiplication of functions correspond to addition and multiplication of their values at points. This is simply a translation of the fact that $A\rightarrow A/\mathfrak{p}$ is a ring morphism. 

If $A$ is generated over a base field (or base ring) by elements $x_1,...,x_r$, the elements $x_1,...,x_r$ are often called \textbf{coordinates}, because we will later be able to reinterpret them as restrictions of ``coordiantes on $r$-space."

\begin{example}[The Complex Affine Line]
    Define $\mathbb{A}_{\C}^1 := \text{Spec} \C[x]$. Let's find the prime ideals of $\C[x]$. As $\C[x]$ is an integral domain, $0$ is prime. Also, $(x-a)$ is prime for any $a \in \C$ as it is a maximal ideal since its quotient is a field: $$0 \rightarrow (x-a)\rightarrow \C[x]\rightarrow \C\rightarrow 0$$

    Now, suppose $\mathfrak{p}$ is a prime ideal. If $\mathfrak{p} \neq (0)$, then $\mathfrak{p} = (f(x))$ for some $f(x) \in \C[x]$ non-zero since $\C[x]$ is a PID. We can write $f(x) = (x-a_1)^{k_1}\cdots(x-a_n)^{k_n}$ as $\C$ is a algebraically closed field. Then as $\mathfrak{p}$ is prime $(x-a_i)$ must be in $\mathfrak{p}$ for some $1\leq i \leq n$. However $f(x)$ is the generator of $\mathfrak{p}$, so this implies $k_j = 0$ for $j \neq i$ and $k_i = 1$, so $f(x) = (x-a_i)$.

    Thus we can make a picture of $\mathbb{A}_{\C}^1 = \text{Spec}\C[x]$. We have one ``traditional" point for each complex number, plus one extra point $[(0)]$. We can mostly picture $\mathbb{A}_{\C}^1$ as $\C$, associating $[(x-a)]$ with $a \in \C$. But, where should $[(0)]$ go? As $(0)$ is contained in all of these prime ideals, we will somehow associate it with this line passing through all the other points. This new point $[(0)]$ is called the ``generic point" of the line. It is generically on the line, but it is not at any particular place on the line.

    The functions on $\mathbb{A}_{\C}^1$ are the polynomials. So $f(x) = x^2-3x+1$ is a function. Its value at a point $[(x-1)]$ is $f(1)$, or equivalently $f(x) \mod (x-1)$. Its value at $[(0)]$ is just $f(x)$ as $\C[x]$ is already an integral domain.
\end{example}

\begin{example}[The affine line over k]
    Let $\mathbb{A}_{k}^1 := \text{Spec}k[x]$ where $k$ is an algebraically closed field. This is called the affine line over $k$. All the arguments in the previous example carry over without change, so we have the same picture.
\end{example}

\begin{example}
    Consider $\text{Spec}\Z$. Amazingly this will look a lot like the affine line over an algebraically closed field. The integers, like $\overline{k}[x]$, form a PID. The prime ideals are $0\Z$ and $p\Z$ where $p$ is prime. 

    Consider the function $100$ on $\text{Spec}\Z$. Its value at $(3)$ is ``$1\mod 3$". Its value at $(2)$ is ``$0\mod 2$".
\end{example}

\begin{example}
    The set $\text{Spec}k$ where $k$ is any field is one point, $[(0)]$. $\text{Spec}0$, where $0$ is the zero-ring, is the empty set, as $0$ has no prime ideals.
\end{example}


\begin{example}
    Consider $\mathbb{A}_{\R}^1 = \text{Spec}\R[x]$. Using the fact that $\R[x]$ is a ED, we have that $(0)$ and $(x-a)$ where $a \in \R$, and $(x^2+ax+b)$, where $x^2+ax+b$ is an irreducible quadratic are the prime ideals of $\R[x]$. The latter two are maximal ideals, i.e. their quotients are fields. For example $\R[x]/(x-3) \cong \R, \R[x]/(x^2+1)\cong \C$.

    Thus the points of $\mathbb{A}_{\R}^1$ can be pictured as the complex plane folded along the real axis (since we identify conjugate pairs), where we ``glue" Galois-conjugate pairs. Now, consider the function $f(x) = x^3-1$. Its value at the point $[(x-2)]$ is $f(x) = 7$, or perhaps more accurately, $7 (\mod x-2)$. At $(x^2+1)$ we have that $f(x)$ s \begin{equation*}
        x^3-1 \equiv -x-1(\mod x^2+1)
    \end{equation*}
    which may be interpreted as $-i-1$.
\end{example}


\begin{example}
    Consider $\mathbb{A}_{\mathbb{F}_p}^1 = \text{Spec}\mathbb{F}_p[x]$. As with the previous examples, $\mathbb{F}_p[x]$ is a Euclidean domain, so the prime ideals are of the form $(0)$ or $(f(x))$ where $f(x) \in \mathbb{F}_p[x]$ is an irreducible polynomial, which can be of any degree. Irreducible polynomials correspond to sets of Galois conjugates in $\overline{\mathbb{F}}_p$.

    Note that $\text{Spec}\mathbb{F}_p[x]$ has $p$ points corresponding elements of $\mathbb{F}_p$, but also infinitely many more. This makes this space much richer than simply $p$ points. A polynomial $f(x)$ is not determined by its values at the $p$ elements of $\mathbb{F}_p$, but it is determined by its values at the points of $\text{Spec}\mathbb{F}_p[x]$. (This is not true for all schemes)
\end{example}


\begin{example}
    Consider $\mathbb{A}_{\C}^2 := \text{Spec}\C[x,y]$. Sadly $\C[x,y]$ is not a PID, for $(x,y)$ is not principal. Some prime ideals of $\C[x,y]$ are $(0)$ and $(x-a,y-b)$ for $(a,b) \in \C^2$ as it is maximal. Also if $f(x,y)$ is an irreducible polynomial, then $(f(x,y))$ is prime. Further, any $\mathfrak{p}$ prime is of these forms.
\end{example}

\begin{example}
    Let $\mathbb{A}_{\C}^n := \text{Spec}\C[x_1,...,x_n]$. More generally, let $\mathbb{A}_{\mathcal{A}}^n$ be defined as $\text{Spec}\mathcal{A}[x_1,...,x_n]$, where $\mathcal{A}$ is an arbitrary ring. Analogous to before, $(x_1-a_1,...,x_n-a_n)$ is a prime ideal, being maximal with residue field $\C$; these are thought of as ``$0$-dimensional points." There are no more maximal ideals by the following result.
\end{example}

\begin{theorem}[Hilbert's Weak Nullstenelsatz]
    If $k$ is an algebraically closed field, then the maximal ideals of $k[x_1,...,x_n]$ are precisely those ideals of the form $(x_1-a_1,...,x_n-a_n)$, where $a_i \in k$.
\end{theorem}


\begin{theorem}[Hilbert's Nullstellensatz]
    If $k$ is any field, every maximal ideal of $k[x_1,...,x_n]$ has residue field a finite extension of $k$.
\end{theorem}

This can be translated as the fact that any field extension of $k$ that is finitely generated as a ring is necessarily also finitely generated as a module.


\subsection{Quotients and Localizations}

We can interpret quotients and localizations of rings in terms of spectra. If $A$ is a ring with ideal $I$, $\text{Spec}A/I$ can be interpreted as a subset of $\text{Spec}A$, as we have a correspondence between prime ideals of $A/I$ and prime ideals of $A$ containing $I$.
