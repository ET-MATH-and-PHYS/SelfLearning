%%%%%%%%%%%%%%%%%%%%% chapter.tex %%%%%%%%%%%%%%%%%%%%%%%%%%%%%%%%%
%
% sample chapter
%
% Use this file as a template for your own input.
%
%%%%%%%%%%%%%%%%%%%%%%%% Springer-Verlag %%%%%%%%%%%%%%%%%%%%%%%%%%
%\motto{Use the template \emph{chapter.tex} to style the various elements of your chapter content.}
\chapter{Chow Ring}
\label{chowring} % Always give a unique label
% use \chaptermark{}
% to alter or adjust the chapter heading in the running head


\section{The Chow Ring}

Throughout we assume that $k$ is an algebraically closed field.


\subsection{Cycles}

Let $X$ be an algebraic.

\begin{definition}
    The \textbf{group of cycles} on $X$, denoted $Z(X)$, is the free abelian group generated by the set of subvarieties (reduced irreducible subschemes) of $X$. The group $Z(X)$ is graded by dimension.

    A cycle $Z = \sum n_iY_i$, where the $Y_i$ are subvarieties, is \textbf{effective} if the coefficients $n_i$ are nonnegative. A \textbf{divisor}, or \textbf{Weil divisor}, is an $(n-1)$-cycle on a pure $n$-dimensional scheme.
\end{definition}


To any closed subscheme $Y \subset X$ we associate an effective cycle $\langle Y\rangle$: if $Y \subset X$ is a subscheme, and $Y_1,...,Y_s$ are the irreducible components of the reduced scheme $Y_{red}$, then, because our schemes are Noetherian, each local ring $\mathcal{O}_{Y,Y_i}$ has a finite composition series. Writing $l_i$ for its length, we define $\langle Y\rangle = \sum l_iY_i$.


\subsection{Rational Equivalence}

A \textbf{Chow group} of $X$ is the group of cycles of $X$ modulo \textbf{rational equivalence}. Informally, two cycles $A_0,A_1 \in Z(X)$ are rationally equivalent if there is a rationally parametrized family of cycles interpolating between them.

\begin{definition}
    Let $\text{Rat}(X) \subset Z(X)$ be the subgroup generated by differences of the form $$\langle \Phi\cap(\{t_0\}\times X)\rangle - \langle \Phi\cap (\{t_1\}\times X)\rangle$$
    where $t_0,t_1 \in \mathcal{P}^1$ and $\Phi$ is a subvariety of $\mathcal{P}^1\times X$ not contained in any fiber $\{t\}\times X$. We say that two cycles are \textbf{rationally equivalent} if their difference is in $\text{Rat}(X)$, and we say that two subschemes are rationally equivalent if their associated cycles are rationally equivalent.
\end{definition}

\begin{definition}
    The \textbf{Chow group} of $X$ is the quotient $$A(X) = Z(X)/\text{Rat}(X)$$
    the \textbf{group of rational equivalence classes of cycles on $X$}. If $Y \in Z(X)$ is a cycle, we write $[Y] \in A(X)$.
\end{definition}

