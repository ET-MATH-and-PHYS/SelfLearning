%%%%%%%%%%%%%%%%%%%%% chapter.tex %%%%%%%%%%%%%%%%%%%%%%%%%%%%%%%%%
%
% sample chapter
%
% Use this file as a template for your own input.
%
%%%%%%%%%%%%%%%%%%%%%%%% Springer-Verlag %%%%%%%%%%%%%%%%%%%%%%%%%%
%\motto{Use the template \emph{chapter.tex} to style the various elements of your chapter content.}
\chapter{Introduction}
\label{intro} % Always give a unique label
% use \chaptermark{}
% to alter or adjust the chapter heading in the running head


Throughout this book a \textbf{scheme} $X$ refers to a separated scheme of finite type over an algebraically closed field $k$ of characteristic $0$.

\begin{definition}
    A scheme $X$ over $S$, $p:X\rightarrow S$, is said to be \textbf{separated} if the diagonal morphism $\Delta:X\rightarrow X\times_SX$, given by the universal property of the fiber product of $p$ along itself and the identity $1_X:X\rightarrow X$, is a \textbf{closed immersion}.

    A \textbf{closed immersion} $f:Z\rightarrow X$ is a map of locally ringed spaces such that $f$ is a homeomorphism, $f^{\#}:\mathcal{O}_X\rightarrow f_*\mathcal{O}_Z$ is surjective with kernel $\mathcal{I}$, and the kernel $\mathcal{I}$ is locally generated by sections as a $\mathcal{O}_X$-module, i.e. for every $x \in X$ there exists an open neighborhood $U$ of $x$ such that $\mathcal{I}\vert_U$ is globally generated as a sheaf of $\mathcal{O}_U$-modules, i.e. there exists a set $I$, and global sections $s_i \in \Gamma(U,\mathcal{I}\vert_U), i \in I$ such that the map $$\bigoplus_{i \in I}\mathcal{O}_X\rightarrow \mathcal{I}$$
    is surjective.
\end{definition}


\begin{definition}
    An $R$-scheme $X$, for $R$ a ring, is called a $R$-scheme of \textbf{finite type} if there exists a finite affine covering $(X_i)_{i \in I}$ of $X$ such that the $R$-algebras $\Gamma(X_i,\mathcal{O}_X)$ is a finite type $R$-algebra (i.e. finitely generated as an $R$-algebra)
\end{definition}


In practice we will only consider quasi-projective schemes.

\begin{definition}
    We say a scheme $X$ over a ring $R$ is \textbf{quasi-projective} if $X$ is a quasicompact open subscheme of a projective $A$-scheme, where a projective $A$ scheme is isomorphic to $\text{Proj}\;S_{\bullet}$, where $S_{\bullet}$ is some finitely generated graded ring over $A$.
\end{definition}

\begin{definition}
    A scheme $X$ is said to be \textbf{integral} if it is non-empty and $\mathcal{O}_X(U)$ is an integral domain for every non-empty open set $U$ - equivalently $X$ is irreducible and reduced, where $X$ is reduced if each $\mathcal{O}_X(U)$ is reduced. In other words, the nilradical of each $\mathcal{O}_X(U)$ is trivial (no nilpotent elements).
\end{definition}

A \textbf{variety} will mean an integral scheme. If $X$ is a variety we write $k(X)$ for the field of rational functions on $X$ (i.e. the field of fractions of $\Gamma(X,\mathcal{O}_X)$).

\begin{definition}
    If $X$ is a scheme, then a $\mathcal{O}_X$-module $\mathcal{F}$ is \textbf{quasicoherent} if for every affine open subset $\text{Spec}\;A\subset X$, $\mathcal{F}\vert_{\text{Spec}\;A} \cong \widetilde{M}$ for some $A$ module $\widetilde{M}$. $\mathcal{F}$ is said to further be \textbf{coherent} if for every affine open $\text{Spec}\;A$, $\Gamma(\text{Spec}\;A,\mathcal{F})$ is a coherent $A$-module (i.e. it is finitely generated and for any map $A^{\oplus p}\rightarrow \Gamma(\text{Spec}\;A,\mathcal{F})$, the kernel is finitely generated).
\end{definition}

\begin{definition}
    If $V$ is a vector space, its \textbf{projectivization} refers to the scheme $\text{Proj}(\text{Sym}\;V^*)$, where $\text{Sym}\;V$ is the symmetric algebra of $V$. The closed points of this space correspond to one-dimensional subspaces of $V$.
\end{definition}

If $X,Y \subset \mathcal{P}^n$ are subvarieties, we define the \textbf{join} of $X$ and $Y$, denoted $\overline{X,Y}$, to be the closure of the union of lines meeting $X$ and $Y$ at distinct points. If $X \subset \mathcal{P}^n$, this is just the cone over $Y$ with vertex $X$; if $X$ and $Y$ are both linear subspaces, this is simply their span.

\begin{remark}
    There is a one-to-one correspondence between vector bundles on a scheme $X$ and locally free sheaves on $X$.
\end{remark}

\begin{definition}
    A sheaf $\mathcal{F}$ of $\mathcal{O}_X$-modules is said to be \textbf{locally free} if for every point $x \in X$, there exists a set $I$ and an open neighborhood $x \in U \subset X$ such that $\mathcal{F}\vert_U$ is isomorphic to $\bigoplus_{i \in I}\mathcal{O}_X\vert_U$ as an $\mathcal{O}_X\vert_U$-module.
\end{definition}


