%%%%%%%%%% Mod Defs %%%%%%%%%%
\chapter{\textsection\textsection General Definitions and Examples: Modules}


\section{\textsection Basic Definitions and Examples: Modules}

Fix a (unital) ring $R$ for this section.

\begin{defn}
    An (left) $R$-module is an abelian group $M$ with an additional structure of a map \begin{equation}
        act_M:R\times M \rightarrow,\;\;\;(r,m)\mapsto r\cdot m
    \end{equation}
    such that the following properties hold:\begin{enumerate}
        \item For every $m \in M$, we have $1 \cdot m = m$.
        \item For every $r_1,r_2 \in R$ and $m \in M$, we have $$r_1\cdot(r_2\cdot m) = (r_1\cdot r_2)\cdot m$$
        \item For every $r_1,r_2 \in R$ and $m \in M$, we have $$(r_1+r_2)\cdot m = r_1\cdot m + r_2\cdot m$$
        For every $r \in R$ and $m_1,m_2 \in M$, we have $$r\cdot (m_1+m_2) = r\cdot m_1 + r\cdot m_2$$
    \end{enumerate}
    Note that the last condition is equivalent to saying that for any fixed $r \in R$, the map $$\map{M\rightarrow M}{m\mapsto r\cdot m}$$
    is a group endomorphism.
\end{defn}

\begin{defn}[(General)]
    Let $R$ be a ring (not necessarily commutative nor unital). A \Emph{left $R$-module} or a \Emph{left module over $R$} is a set $M$ together with \begin{enumerate}
        \item A binary operation $+$ on $M$ under which $M$ is an abelian group, and
        \item An action of $R$ on $M$ (that is, a map $R\times M \rightarrow M$) denoted by $r.m$, for all $r \in R$ and for all $m \in M$ which satisfies \begin{enumerate}
                \item $(r+s).m = r.m + s.m$, for all $r,s \in R$ and all $m \in M$
                \item $r.(m+n) = r.m + r.n$, for all $r \in R$ and all $m,n \in M$
                \item $r.(s.m) = (rs).m$, for all $r,s \in R$ and all $m \in M$

                    \noindent If the ring $R$ has a $1$ we impose the additional axiom:
                \item $1.m = m$ for all $m \in M$
        \end{enumerate}
    \end{enumerate}
\end{defn}

Note that when $R$ is a field $k$, our definition induces the definition of a $k$-vector space.

\begin{defn}
    Let $R$ be a ring and let $M$ be an $R$-module. An \Emph{$R$-submodule} of $M$ is a subgroup $N$ of $M$ which is closed under the action of ring elements, i.e., $r.n \in N$ for all $n \in N$ and all $r \in R$.
\end{defn}

We note that if $R$ is a field, then $R$-submodules correspond to subspaces. Moreover, a submodule of a module $M$ is precisely a subset of $M$ which is itself an $R$-module under the restricted action by ring elements.

Every $R$-module $M$ has the submodules $M$ and $\{0\}$, the second being the \Emph{trivial submodule}.

\begin{lem}
    For any $r \in R$ we have $r \cdot 0_M = 0_M$. For any $m \in M$ we have $0_R\cdot m = 0_M$ and $(-1)\cdot m = -m$.
\end{lem}


\begin{eg}
    \leavevmode
    \begin{enumerate}
        \item The $0$ module is an $R$-module.
        \item Take $M =R$, with the structure of an abelian group the same as that on $R$. We define $act_M := mult_R$. The module axioms follow from the ring axioms on $R$. Moreover, it follows that every field can be considered as a $1$-dimensional vector space over itself. Additionally, when $R$ is considered as a left module over itself in this fashion, its submodules are precisely its left ideals. If $R$ is not commutative it has a left and right module structure over itself, and these may be different. 
        \item Take $M = R^{1\times 2} := R\times R$. The abelian group structure is defined component wise, and so is the action of $R$.
        \item Generalizing the previous example we can take $M = R^{1\times n}$ for any positive integer $n$. In particular, for $n \in \Z^+$ we define \begin{equation*}
                R^n \cong R^{1\times n}  = \{[a_1,a_2,...,a_n]: a_i \in R, \forall i\}
        \end{equation*}
            The module $R^n$ is called the \Emph{free module of rank $n$ over $R$}. A clear submodule of $R^n$ is the $i$th component, in which arbitrary ring elements can exist in the $i$th component while zeros are in the $j$th component for all $j \neq i$.
        \item If we replace the ring in the previous example with a field $F$, we obtain the \Emph{affine $n$-space over $F$}, $F^n$.
    \end{enumerate}
\end{eg}

We note that if $M$ is an $R$-module and $S$ is a subring of $R$ with $1_S = 1_R$ (if identity exists), then $M$ is automatically an $S$-module. 

\begin{defn}
    If $M$ is an $R$-module, and $I$ is a two-sided ideal such that $a.m = 0$ for all $a \in I$ and all $m \in M$, then we say $M$ is \Emph{annihilated} by $I$. In this case we can make $M$ into an $(R/I)$-module by defining an action of the quotient ring $R/I$ on $M$ as follows: for each $m\in M$ and each coset $r+I \in R/I$, let \begin{equation*}
        (r+I).m := r.m
    \end{equation*}
    Since $a.m = 0$ for all $a \in I$ and $m \in M$, this is well-defined. In particular, if $I$ is a maximal ideal and $R$ is a commutative ring, then $M$ is a vector space over the field $R/I$.
\end{defn}


\begin{eg}[(Z-Modules)]
    Let $R=\Z$, let $A$ be an abelian group and write the operation of $A$ as $+$. We can make $A$ into a $\Z$-module as follows: for any $n \in \Z$ and $a \in A$ define \begin{equation*}
        n.a := \left\{\begin{array}{lc} \underbrace{a+a+...+a}_{n-times} & \text{if } n > 0 \\ 0 & \text{if } n = 0 \\ \underbrace{(-a)+(-a)+...+(-a)}_{-n-times} & \text{if } n < 0
        \end{array}\right.
    \end{equation*}
    where $0$ is the identity of the additive group $A$. This definition makes $A$ into a $\Z$-module, and by the module actions this is the only definition which makes $A$ into a (unital) $\Z$-module. Thus, every abelian group is a $\Z$-module. 


    Conversely, if $M$ is any $\Z$-module, a fortiori $M$ is an abelian group, so \begin{equation*}
        \Z-\text{modules are the same as abelian groups}
    \end{equation*}
    Furthermore, from the definition it is clear that \begin{equation*}
        \Z-\text{submodules are the same as subgroups}
    \end{equation*}
\end{eg}


\begin{eg}[(F{[x]}-modules)]
    Let $F$ be a field, let $x$ be an indeterminate and let $R$ be the polynomial ring $F[x]$. Let $V$ be a vector space over $F$ and let $T$ be a linear transformation from $V$ to $V$. We already know that $V$ is an $F$-module; the linear map $G$ will enable us to make $V$ into an $F[x]$-module.

    First, for the nonnegative integer $n$, define \begin{align*}
        T^0 &:= I, \\
        &\vdots \\
        T^n &:= \underbrace{T\circ T\circ ...\circ T}_{n-times}
    \end{align*}
    where $I$ is the identity map from $V$ to $V$ and $\circ$ denotes function composition. Also, for any two linear transformations $A, B$ from $V$ to $V$ and elements $\alpha,\beta \in F$, let $\alpha A+\beta B$ be defined by \begin{equation*}
        (\alpha A + \beta B)(v) := \alpha(A(v)) = \beta(B(v))
    \end{equation*}
    for all $v \in V$. Note that this is again a linear transformation from $V$ to $V$.


    Now let us define the action of any polynomial in $x$ on $V$. Let $p(x) \in F[x]$, $p(x) = a_nx^n+...a_1x+a_0$, where $a_0,...,a_n \in F$. For each $v \in V$ we define an action fo the ring element $p(x)$ on the module element $v$ by \begin{align*}
        p(x).v &= (a_nt^n+a_{n-1}T^{n-1}+...+a_1T+a_0)(v) \\
        &= a_nT^n(v) + a_{n-1}T^{n-1}(v) + ... + a_1T(v) + a_0v
    \end{align*}
    Put another way, $x$ acts on $V$ as the linear transformation $T$, and we extend this to an action of all of $F[x]$ on $V$, satisfying all the module axioms.

    Note that the action of $F[x]$ on $V$ is consistent with the original action of $F$ on the vector space $V$ when restricted to constant polynomials. This construction in fact describes all $F[x]$-modules.

    Moreover, there is a bijection between the collections of $F[x]$-modules and the collection of pairs $V,T$: \begin{equation*}
        \left\{V\text{ an }F[x]\text{-module}\right\} \leftrightarrow \left\{\begin{array}{c} V\text{ a vector space over } F \\ \text{and} \\ T:V\rightarrow V\text{ a linear transformation}\end{array}\right\}
    \end{equation*}
    given by \begin{equation*}
        \text{the element $x$ acts on $V$ as the linear transformation $T$}
    \end{equation*}

    Next, the $F[x]$-submodules $U$ of $V$ are precisely the $T$-stable (or invariant) subspaces of $V$ as seen with $V$ as a vector space over $F$. We obtain a similar bijection as before: \begin{equation*}
        \left\{W\text{ an }F[x]\text{-submodule}\right\} \leftrightarrow \left\{\begin{array}{c} W\text{ a vector subspace of } V \\ \text{and} \\ W\text{ is $T$-stable}\end{array}\right\}
    \end{equation*}
\end{eg}


\begin{prop}[(The Submodule Criterion)]
    Let $R$ be a ring and let $M$ be an $R$-module. A subset $N$ of $M$ is a submodule of $M$ if and only if \begin{enumerate}
        \item $N \neq \emptyset$, and 
        \item $x+r.y \in N$ for all $r \in R$ and for all $x,y \in N$
    \end{enumerate}
\end{prop}
\begin{proof}
    If $N$ is a submodule, then $0 \in N$ so $N \neq \emptyset$. Also $N$ is closed under addition and is stable under the action of elements of $R$, so $x+r.y \in N$ for all $r \in R$ and $x,y \in N$.

    Conversely, suppose the two points hold. Let $m \in N$ since $N$ is non-empty. Then $m+(-1).m = (1+(-1)).m = 0.m = 0$, so $0 \in N$. Moreover, for all $m,n \in N$ we have that $m+(-n) = m+(-1).n \in N$ by hypothesis, so $N$ is a subgroup of $M$. Now, we can take $x = 0$ and observe that for all $y \in N$ and all $r \in R$, $r.y = 0+r.y \in R$, so $N$ is stable under the action. Thus $N$ is indeed a submodule of $M$.
\end{proof}


\begin{defn}
    Let $R$ be a commutative ring with identity. An \Emph{$R$-algebra} is a ring $A$ with identity together with a ring homomorphism $f:R\rightarrow A$ mapping $1_R$ to $1_A$ such that the subring $f(R)$ of $A$ is contained in the center of $A$.
\end{defn}

Observe that if $A$ is an $R$-algebra, then $A$ has a natural left and right (unital) $R$-module structure defined by $r\cdot a = a\cdot r = f(r)a$ where $f(r)a$ is just the multiplication in the ring $A$. Other $R$-module structures are possible on $A$, but this is the standard one.


\begin{defn}
    If $A$ and $B$ are two $R$-algebras, an \Emph{$R$-algebra homomorphism (or isomorphism)} is a ring homomorphism (isomorphism, respectively) $\phi:A\rightarrow B$ mapping $1_A$ to $1_B$ such that $\phi(r.a) = r.\phi(a)$ for all $r \in R$ and $a \in A$.
\end{defn}


\begin{eg}
    Let $R$ be a commutative ring with $1$. \begin{enumerate}
        \item Any ring with identity is a $\Z$-algebra.
        \item For any ring $A$ with identity, if $R$ is a subring of the center of $A$ containing the identity of $A$ then $A$ is an $R$-algebra. 
        \item If $A$ is an $R$-algebra then the $R$-module structure of $A$ depends only on the subring $f(R)$ contained in the center of $A$ as in the previous example. If we identify $R$ by its image $f(R)$ we see that ``up to a ring homomorphism" every algebra $A$ arises from a subring of the center of $A$ that contains $1_A$.
    \end{enumerate}
\end{eg}

If $A$ is an $R$-algebra, then $A$ is a ring with identity that is a (unital) left $R$-module satisfying $r\cdot (ab) = (r\cdot a)b = a(r\cdot b)$ for all $r \in R$ and $a,b \in A$.



\section{\textsection Module Homomorphisms}

\begin{defn}
    Let $M_1$ and $M_2$ be $R$-modules. An $R$-module homomorphism from $M_1$ to $M_2$ is a map of sets $\phi:M_1\rightarrow M_2$ such that \begin{enumerate}
        \item $\phi$ is a group homomorphism
        \item For every $r \in R$ and $m_1 \in M_1$, we have $\phi(r\cdot m_1) = r\cdot \phi(m_1)$.
    \end{enumerate}
    The last condition can be rewritten in terms of the following commutative diagram:
        \begin{center}
            \begin{tikzpicture}[baseline = (a).base]
            \node[scale = 1] (a) at (0,0){
                \begin{tikzcd}
                    R\times M_1  \ar[d, "act_{M_1}", swap] \ar[r, "\id_R\times \phi"] & R\times M_2 \ar[d,"act_{M_2}"] \\
                    M_1 \ar[r, "\phi"] & M_2
                \end{tikzcd}
            };
            \end{tikzpicture}
        \end{center}
    We denote the set of $R$-module homomorphisms $M_1\rightarrow M_2$ by $\Hom_{\Rmod}(M_1,M_2)$.
\end{defn}

\begin{defn}[D\&F]
    Let $R$ be a ring and let $M$ and $N$ be $R$-modules.
    \begin{enumerate}
        \item A map $\varphi:M\rightarrow N$ is an \Emph{$R$-module homomorphism} if it respects the $R$-module structures of $M$ and $N$, i.e., \begin{enumerate}
                \item $\varphi(x+y) = \varphi(x)+\varphi(y)$, for all $x,y  \in M$, and 
                \item $\varphi(r\cdot_Mx) = r\cdot_N\varphi(x)$, for all $r \in R, x \in M$.
        \end{enumerate}
        \item An $R$-module homomorphism is an \Emph{isomorphism of $R$-modules} if it is both injective and surjective. The modules $M$ and $N$ are said to be \Emph{isomorphic}, denoted $M\cong N$, if there is some $R$-module isomorphism $\varphi:M\rightarrow N$.
        \item If $\varphi:M\rightarrow N$ is an $R$-module homomorphism, let $\ker\varphi = \{m \in M\vert \varpi(m) = 0_N\}$ and let $\varphi(M) = \{n \in N\vert \exists m \in M; n = \varphi(m)\}$.
        \item Let $M$ and $N$ be $R$-modules, and define $\Hom_R(M,N)$ to be the set of all $R$-module homomorphisms from $M$ into $N$.
    \end{enumerate}
\end{defn}


\begin{rmk}
    Give $R$-modules $M_1,M_2,M_3$, and $R$-module homomorphisms $\phi:M_1\rightarrow M_2$, $\psi:M_2\rightarrow M_3$, the composed map \begin{equation}
        \psi \circ \phi:M_1\rightarrow M_3
    \end{equation}
    is an $R$-module homomorphism. We can regard the operation of composition as a map of sets \begin{equation}
        \Hom_{\Rmod}(M_2,M_3)\times\Hom_{\Rmod}(M_1,M_2) \rightarrow \Hom_{\Rmod}(M_1,M_3)
    \end{equation}
\end{rmk}

\begin{eg}
    \leavevmode
    \begin{enumerate}
        \item If $R$ is a ring and $M = R$ is a module over itself, then $R$-module homomorphisms need not be ring homomorphisms and ring homomorphisms need not be $R$ module homomorphisms. For example, take $R = \Z$ and the $\Z$-module homomorphism $x\mapsto 2x$ (doesn't send $1$ to $1$). When $R = F[x]$, the ring homomorphism $\varphi:f(x)\mapsto f(x^2)$ is not an $F[x]$-module homomorphism.
        \item Let $R$ be a ring, let $n \in \Z^+$ and let $M = R^n$. It is a straightforward exercise to show that for each $i \in \{1,2,...,n\}$, the canonical projection map \begin{equation*}
                \pi_i:R^n\rightarrow R\;\;by\;\;\pi_i(x_1,...,x_n) = x_i
        \end{equation*}
            is a surjective $R$-module homomorphism with kernel equal to the submodule of $n$-tuples which have a zero in position $i$.
        \item If $R$ is a field, the $R$-module homomorphisms are called \Emph{linear transformations}.
        \item For a ring $R = \Z$ the action of ring elements on any $\Z$-module amounts to adding and subtracting within the abelian group structure of the module, so $\Z$-module homomorphisms are the same as abelian group homomorphisms.
    \end{enumerate}
\end{eg}


\begin{prop}
    Let $M, N,$ and $L$ be $R$-modules. 
    \begin{enumerate}
        \item A map $\varphi:M\rightarrow N$ is an $R$-module homomorphism if and only if $\varphi(rx+y) = r\varphi(x)+\varphi(y)$ for all $x,y \in M$ and all $r \in R$.
        \item Let $\varphi,\psi \in \Hom_R(M,N)$. Define $\varphi+\psi$ by \begin{equation*}
                (\varphi+\psi)(m) = \varphi(m)+\psi(m),\forall m \in M
        \end{equation*}
            Then $\varphi+\psi \in \Hom_R(M,N)$ and with this operation $\Hom_R(M,N)$ is an abelian group. If $R$ is a commutative ring, then for $r \in R$ define $r\varphi$ by\begin{equation*}
                (r\varphi)(m) = r(\varphi(m)),\forall m \in M
            \end{equation*}
            Then $r\varphi \in \Hom_R(M,N)$ and with this action of the commutative rign $R$ the abelian group $\Hom_R(M,N)$ is an $R$-module.
        \item If $\varphi \in \Hom_R(L,M)$ and $\psi\in \Hom_R(M,N)$ then $\psi\circ \varphi \in \Hom_R(L,N)$.
        \item With addition as above and multiplication defined as function composition, $\Hom_R(M,M)$ is a ring with $1$. When $R$ is commutative $\Hom_R(M,M)$ is an $R$-algebra.
    \end{enumerate}
\end{prop}
\begin{proof}
    $(1)$: Let $\varphi:M\rightarrow N$ be a map. If $\varphi$ is an $R$-module homomorphism then for all $x,y \in M$ and all $r \in R$, $$\varphi(rx+y) = \varphi(rx)+\varphi(y) = r\varphi(x)+\varphi(y)$$ Conversely, suppose this holds for all $r \in R$ and $x,y \in M$. First, take $r = 1$, so $$\varphi(x+y) = \varphi(1\cdot x + y) = 1\cdot \varphi(x)+\varphi(y) = \varphi(x)+\varphi(y)$$ Additionally, if we take $y = 0_M$, then \begin{equation*}
        \varphi(rx) = \varphi(rx+0_M) = r\varphi(x)+\varphi(0_M) = r\varphi(x)+0_N = r\varphi(x)
    \end{equation*}
    Thus $\varphi$ is indeed an $R$-module homomorphism.

    $(2)$: Let $\varphi,\psi \in \Hom_R(M,N)$. Then observe that for all $r \in R$, $x,y \in M$, we have \begin{align*}
        (\varphi+\psi)(rx+y) &= \varphi(rx+y)+\psi(rx+y) \\
        &= (r\varphi(x)+\varphi(y))+(r\psi(x)+\psi(y)) \\
        &= r(\varphi(x)+\psi(x))+(\varphi(y)+\psi(y)) \\
        &= r(\varphi+\psi)(x)+(\varphi+\psi)(y)
    \end{align*}
    so by result $(1)$ $\varphi + \psi \in \Hom_R(M,N)$. Since $N$ is an abelian group $\varphi+\psi = \psi + \varphi$, and as $-1$ is in the center of $R$, $-\varphi \in \Hom_R(M,N)$ where $\varphi+(-\varphi) = \mathbf{0}$, where $\mathbf{0}$ is the trivial map $m \mapsto 0_N$ for all $m \in M$. If $R$ is commutative, then for all $r_1, r_2 \in R$, and all $x,y \in M$, \begin{align*}
        (r_1\varphi)(r_2x+y) &= r_1\varphi(r_2x+y) \\
        &= r_1(r_2\varphi(x)+\varphi(y)) \\
        = r_1r_2\varphi(x)+r_1\varphi(y) \\
        &= r_2(r_1\varphi)(x)+(r_1\varphi)(y)
    \end{align*}
    so $r_1\varphi \in \Hom_R(M,N)$, and by the $R$-module structure of $N$ we conclude that $\Hom_R(M,N)$ is an $R$-module under this action.

    $(3)$: Let $\varphi \in \Hom_R(L,M)$ and $\psi \in \Hom_R(M,N)$, and let $r \in R$, $x,y \in L$. Then it follows that \begin{align*}
        (\psi\circ \varphi)(rx+y) &= \psi(\varphi(rx+y)) \\
        &= \psi(r\varphi(x)+\varphi(y)) \\
        &= r\psi(\varphi(x)) + \psi(\varphi(y)) \\
        &= r(\psi\circ \varphi)(x)+(\psi\circ \varphi)(y)
    \end{align*}
    so $\psi \circ \varphi \in \Hom_R(L,N)$.


    $(4)$: Since the domain and codomain of $\Hom_R(M,M)$ are the same, composition is well defined, and by $(3)$ it is a binary operation on $\Hom_R(M,M)$. Moreover, function composition is associative and $\id_M$ the identity map on $M$ acts as a $1$ in $\Hom_R(M,M)$ under composition. Additionally by $(2)$ $\Hom_R(M,M)$ is an abelian group under addition. Then, to prove distributivity let $r,r' \in R$ and $\varphi,\psi \in \Hom_R(M,M)$. Then for all $x \in M$ we have that \begin{align*}
        (r\cdot(\varphi+\psi))(x) &= r(\varphi(x)+\psi(x)) = r\varphi(x)+r\psi(x) = (r\cdot\varphi+r\cdot\psi)(x)
    \end{align*}
    and \begin{align*}
        ((r+r')\cdot\varphi)(x) &= r\varphi(x)+r'\varphi(x) = (r\cdot\varphi+r'\cdot\varphi)(x)
    \end{align*}
    so indeed $r\cdot(\varphi+\psi) = r\cdot\varphi+r\cdot\psi$ and $(r+r')\cdot\varphi = r\cdot\varphi+r'\cdot\varphi$.
    
    Finally, if $R$ is commutative then $\Hom_R(M,M)$ has an $R$-module structure. Moreover, if we define $r\varphi = \varphi r$ for all $\varphi \in \Hom_R(M,M)$ and $r \in R$, then we observe that for all $\varphi,\psi \in \Hom_R(M,M)$ and $x \in M$ \begin{align*}
        (r\cdot(\varphi\circ\psi))(x) &=r(\varphi\circ\psi)(x) &= r(\varphi\circ\psi)(x) \\
        &=r\varphi(\psi(x))  &= r\varphi(\psi(x)) \\ 
        &= \varphi(r\psi(x)) &= (r\cdot\varphi)(\psi(x)) \\
        &= \varphi((r\psi)(x)) &= ((r\cdot\varphi)\circ\psi)(x) \\
        &= (\varphi\circ(r\psi))(x) &= ((r\cdot\varphi)\circ\psi)(x) 
    \end{align*}
    so $\Hom_R(M,M)$ becomes an $R$-algebra.
\end{proof}


\begin{defn}
    The ring $\Hom_R(M,M)$ is called the \Emph{endomorphism ring} of $M$ and will often be denoted by $\catname{End}_R(M)$, or just $\catname{End}(M)$ when the ring $R$ is clear from context. Elements of $\catname{End}(M)$ are called \Emph{endomorphisms}.
\end{defn}

We now wish to show that we can assign an $R$-module structure to quotients of $R$-modules with submodules:

\begin{prop}
    Let $R$ be a ring, let $M$ be an $R$-module and let $N$ be a submodule of $M$. The (additive, abelian) quotient group $M/N$ can be made into an $R$-module by defining an action of elements of $R$ by: \begin{equation*}
        r\cdot(x+N) = (r\cdot x)+N,\forall r \in R, x+N \in M/N
    \end{equation*}
    The natural projection map $\pi:M\rightarrow M/N$ defined by $\pi(x) = x+N$ is an $R$-module homomorphism with kernel $N$.
\end{prop}
\begin{proof}
    Let $R$ be a ring, $M$ and $R$-module, and $N$ a submodule of $M$. Since $M$ is an abelain group $N$ is a normal subgroup of $M$ so the quotient group $M/N$ is well defined. Now, define an action of elements of $R$ on $M$ as above. Let $r,r' \in R$ and $x+N,y+N \in M/N$. Then it follows that \begin{equation*}
        r\cdot(r'\cdot(x+N)) = r\cdot((r'\cdot x)+N) = r\cdot(r'\cdot x)+N = (rr')\cdot x + N = (rr')\cdot(x+N)
    \end{equation*}
    \begin{align*}
        (r+r')\cdot(x+N) &= (r+r')\cdot x+N \\
        &= (r\cdot x + r'\cdot x)+N \\
        &= (r\cdot x + N) + (r'\cdot x + N) \\
        &= r\cdot(x+N)+r'\cdot(x+N)
    \end{align*}
    and \begin{align*}
        r\cdot ((x+N)+(y+N)) &= r\cdot((x+y)+N) \\
        &= r\cdot(x+y)+N \\
        &= (r\cdot x + r\cdot y) + N\\
        &= (r\cdot x + N) + (r\cdot y +N) \\
        &= r\cdot (x+N) +r\cdot (y+N)
    \end{align*}
    Therefore, the action is a well defined module action on $M/N$, so $M/N$ is an $R$-module under this action.

    Consider the natural projection map $\pi:M\rightarrow M/N$. Recall that $\pi$ is an abelian group homomorphism between $M$ and $M/N$. To show that it is indeed an $R$-module homomorphism, let $r \in R$ and $m \in M$. Then \begin{equation*}
        \pi(r\cdot m) = r\cdot m + N = r\cdot(m+N) = r\cdot \pi(m)
    \end{equation*}
    proving that $\pi$ is an $R$-module homomorphism as desired. Finally, $\ker\pi = \{x \in M:x+N = n\} = \{x\in M:x \in N\} = N$.
\end{proof}


\begin{defn}
    Let $A,B$ be submodules of the $R$-module $M$. The sum of $A$ and $B$ is the set \begin{equation*}
        A+B := \{a+b\vert a\in A, b \in B\}
    \end{equation*}
\end{defn}

This is indeed a submodule, and in fact the small submodule containing both $A$ and $B$.

\subsection{\textsection Isomorphism Theorems for Modules}

\begin{namthm}[The First Isomorphism Theorem for Modules]\label{thmname:fisomod}
    Let $M$ and $N$ be $R$-modules and let $\varphi:M\rightarrow N$ be an $R$-module homomorphism. Then $\ker\varphi$ is a submodule of $M$ and $M/\ker\varphi\cong \varphi(M)$.
\end{namthm}
\begin{proof}
    First, observe that $0_M \in \ker \varphi$ as $\varphi(0_M) = 0_N$ since $\varphi$ is an abelian group homomorphism over $M$. Now, let $m,n \in \ker\varphi$. Then \begin{equation*}
        \varphi(m+(-n)) = \varphi(m)+\varphi(-n) = 0_N+(-\varphi(n)) = -0_N = 0_N
    \end{equation*}
    so $m+(-n) \in \ker\varphi$. Hence, $\ker\varphi$ is indeed a subgroup of $M$. Now, let $r \in R$. Then $$\varphi(r\cdot m) = r\cdot \varphi(m) = r\cdot 0_N = 0_N$$
    so $r\cdot m \in \ker\varphi$, and we conclude that $\ker\varphi$ is indeed a submodule of $M$.

    Hence, $M/\ker\varphi$ is an $R$-module, as by our previous results. Now, define a map $\overline{\varphi}:M/\ker\varphi\rightarrow N$ by $\overline{\varphi}(x+\ker\varphi) = \varphi(x)$. Then, if $x+\ker\varphi = y + \ker\varphi$, there exists $m \in \ker\varphi$ such that $x= y + m$. In particular, \begin{equation*}
        \overline{\varphi}(x+N) = \varphi(x) = \varphi(y+m) = \varphi(y)+\varphi(m)= \overline{\varphi}(y+N)
    \end{equation*}
    so the map is indeed well-defined. Furthermore, $\ker(\overline{\varphi}) = \{x+\ker\varphi\in M/\ker\varphi:x \in \ker\varphi\} = \{0_{M/\ker\varphi}\}$, so $\overline{\varphi}$ is injective. Moreover, restricting the codomain to the image $\varphi(M)$, the map is also surjective. Finally, since $\varphi$ is an $R$-module homomorphism and $\overline{\varphi}$ is defined in terms of $\varphi$, it is also an $R$-module homomorphism. Hence, $\overline{\varphi}$ is an $R$-module isomorphism and $M/\ker\varphi\cong\varphi(M)$. 
\end{proof}


\begin{namthm}[The Second Isomorphism Theorem for Modules]\label{thmname:sisomod}
    Let $A$ and $B$ be submodules of the $R$-module $M$. Then $(A+B)/B\cong A/(A\cap B)$. 
\end{namthm}
\begin{proof}
    Suppose $A$ and $B$ are submodules of the $R$-module $M$. Then I claim $A+B$ and $A\cap B$ are submodules of $M$. Indeed, $A+B$ and $A\cap B$ are nonempty, as they contain $0_M$, and for all $a+b,a'+b' \in A+B$, $k,k' \in A\cap B$, and $r \in R$, \begin{equation*}
        r(a+b)+(a'+b') = (ra+rb)+(a'+b') = (ra+a')+(rb+b') \in A+B
    \end{equation*}
    and $kr+k' \in A\cap B$ since $kr+k' \in A$ and $kr+k' \in B$ as they are submodules. Now, define a map $\varphi:A\rightarrow (A+B)/B$ by $\varphi(a) = a+B$. This map is indeed well defined as $a = a+0_M \in A+B$, and for all $x,y \in A$ and $r \in R$, \begin{equation*}
        \varphi(rx+y) = (rx+y)+B = (rx+B)+(y+B) = r(x+B)+(y+B) = r\varphi(x)+\varphi(y)
    \end{equation*}
    so $\varphi$ is an $R$-module homomorphism. Furthermore, $\ker\varphi = \{a\in A: a \in B\} = A\cap B$. Thus, by \ref{thmname:fisomod} we conclude that $A/(A\cap B) \cong (A+B)/B$, as desired.
\end{proof}


\begin{namthm}[The Third Isomorphism Theorem for Modules]\label{thmname:thisomod}
    Let $M$ be an $R$-module, and let $A$ and $B$ be submodules of $M$ with $A \subseteq B$. Then $(M/A)/(B/A) \cong M/B$. 
\end{namthm}
\begin{proof}
    Define a map $f:M/A\rightarrow M/B$ by $f(x+A) = x+B$. Then, if $x+A = y+A$ we have that $x=y+a$ for some $a \in A$, so as $A \subseteq B$ $$f(x+A) = x+B = (y+a) + B = y+B$$
    since $a \in B$. Hence $f$ is well defined. Moreover, for all $x+A,y+A \in M/A$ and $r \in R$, \begin{equation*}
        f(r(x+A)+(y+A)) = f(rx+y+A) = rx+y+B = r(x+B)+(y+B) = rf(x+A) + f(y+A)
    \end{equation*}
    so $f$ is an $R$-module homomorphism. Now, observe that $$\ker f = \{x+A \in M/A: x \in B\} = B/A$$
    so by \ref{thmname:fisomod} we conclude that \begin{equation*}
        (M/A)/(B/A) \cong M/B
    \end{equation*}
\end{proof}


\begin{namthm}[The Fourth of Lattice Isomorphism Theorem]\label{thmname:lattisomod}
    Let $N$ be a submodule of the $R$-module $M$. There is a bijection between the submodules of $M$ which contain $N$ and the submodules of $M/N$. The correspondence is given by $A\leftrightarrow A/N$, for all $A\supseteq N$. This correspondence commutes with the processes of taking sums and intersections.
\end{namthm}
\begin{proof}
    (To be finished)
\end{proof}



\subsection{\textsection Evaluation Bijections}

\begin{defn}
    Let $M$ be an arbitrary $R$-module. Consider the set $\Hom_{\Rmod}(R,M)$, where $R$ is considered as an $R$-module. We define the map of sets \begin{equation}
        \map{\ev:\Hom_{\Rmod}(R,M) \rightarrow M}{\phi\mapsto \phi(1) \in M}
    \end{equation}
    $\ev$ as defined is a bijection of sets. That is, to give a map of modules $R\rightarrow M$ is the same as to give an element of $M$.
\end{defn}

\begin{defn}
    Generalizing the previous definition we obtain the bijection \begin{equation}
        \map{\ev:\Hom_{\Rmod}(R^{1\times n},M) \rightarrow M^{1\times n}}{\phi\mapsto (\phi(1,0,...,0),\phi(0,1,...,0),...,\phi(0,0,...,1)) \in M^{1\times n}}
    \end{equation}
\end{defn}

\begin{rmk}
    In particular, taking $M = R^{1\times m}$, we obtain a bijection \begin{equation}
        \Hom_{\Rmod}(R^{1\times n},R^{1\times m}) \overset{\ev}{\cong} (R^{1\times m})^{1\times n} \cong Mat_{m\times n}(R)
    \end{equation}
    In particular, for the composition map \begin{equation}
        \Hom_{\Rmod}(R^{1\times n_2},R^{1\times n_3}) \times \Hom_{\Rmod}(R^{1\times n_1},R^{1\times n_2}) \xrightarrow{comp} \Hom_{\Rmod}(R^{1\times n_1},R^{1\times n_3})
    \end{equation}
    we obtain the commutative diagram 
        \begin{center}
            \begin{tikzpicture}[baseline = (a).base]
            \node[scale = 1] (a) at (0,0){
                \begin{tikzcd}
                    \Hom_{\Rmod}(R^{1\times n_2},R^{1\times n_3})\times \Hom_{\Rmod}(R^{1\times n_1},R^{1\times n_2}) \ar[d, "\ev\times \ev", swap] \ar[r, "comp"] & \Hom_{\Rmod}(R^{1\times n_1},R^{1\times n_3}) \ar[d,"\ev"] \\
                    Mat_{n_3\times n_2}(R)\times Mat_{n_2\times n_1}(R) \ar[r, "mult_{mat}"] & Mat_{n_3\times n_1}(R)
                \end{tikzcd}
            };
            \end{tikzpicture}
        \end{center}
\end{rmk}


\section{\textsection Submodules}


\section{\textsection Free Modules and Generators}

In section we assume $R$ is a ring with $1$. 

\begin{defn}
    Let $M$ be an $R$-module and let $N_1,...,N_n$ be submodules of $M$.
    \begin{enumerate}
        \item The \Emph{sum} of $N_1,...,N_n$ is the set of all finite sums of elements from the sets $N_i$: $\{\sum_{i=1}^na_i\vert a_i \in N_i,\forall i\}$. Denote this sum by $N_1+\hdots + N_n$.
        \item For any subset $A$ of $M$ let \begin{equation*}
                RA := \{\sum_{i=1}^mr_ia_i\vert r_1,...,r_m \in R,a_1,...,a_m \in A, m \in \Z^+\}
        \end{equation*}
            (where by convention $RA = \{0\}$ if $A = \emptyset$). If $A$ is the finite set $\{a_1,a_2,...,a_n\}$ we shall write $Ra_1+Ra_2+\hdots Ra_n$ for $RA$. Call $RA$ the \Emph{submodule of $M$ generated by $A$}. If $N$ is a submodule of $M$ and $N = RA$ for some subset $A$ of $M$, we call $A$ a \Emph{set of generators} or \Emph{generating set} for $N$, and we say $N$ is \Emph{generated} by $A$.
        \item A submodule $N$ of $M$ is \Emph{finitely generated} if there is some finite subset $A$ of $M$ such that $N = RA$, that is, if $N$ is generated by some finite subset.
        \item A submodule $N$ of $M$ is \Emph{cyclic} if there exists an element $a \in M$ such that $N = Ra$, that is, if $N$ is generated by one element: \begin{equation*}
                N = Ra = \{ra\vert r \in R\}
        \end{equation*}
    \end{enumerate}
\end{defn}

Note that these definitions do not require that the ring $R$ contain a $1$, however this condition ensures that $A$ is contained in $RA$.

\begin{rmk}
    Let $N$ be a submodule of an $R$-module $M$ which is finitely generated. Then there is a smallest nonnegative integer $d$ such that $N$ is generated by $d$ elements. We then call any generating set consisting of $d$ elements a \Emph{minimal set of generators for $N$}.
\end{rmk}

\begin{defn}
    Let $M_1,...,M_k$ be a collection of $R$-modules. The collection of $k$-tuples $(m_1,m_2,...,m_k)$ where $m_i \in M_i$ with addition and action of $R$ defined componentwise is called the \Emph{direct product} of $M_1,...,M_k$, denoted $M_1\times ...\times M_k$.
\end{defn}

The direct product of $M_1,...,M_k$ is also referred to as the (\emph{external}) \Emph{direct sum} of $M_1,...,M_k$ and denoted $M_1\oplus ... \oplus M_k$. 

\begin{prop}
    Let $N_1,...,N_k$ be submodules of the $R$-module $M$. Then the following are equivalent: \begin{enumerate}
        \item The map $\pi:N_1\times N_2\times ... \times N_k\rightarrow N_1+N_2+...+N_k$ defined by \begin{equation*}
                \pi(a_1,a_2,...,a_k) = a_1+a_2+...+a_k
        \end{equation*}
            is an isomorphism of $R$-modules: $N_1+N_2+...+N_k\cong N_1\times N_2\times ...\times N_k$.
        \item $N_j\cap(N_1+...+N_{j-1}+N_{j+1}+...+N_k) = \{0\}$ for all $j \in \{1,2,...,k\}$.
        \item Every $x \in N_1+...+N_k$ can be written uniquely in the form $a_1+a_2+...+a_k$ with $a_i \in N_i$.
    \end{enumerate}
\end{prop}
\begin{proof}
    $(1)\implies (2)$. Let $a_j \in N_j\cap(N_1+...+N_{j-1}+N_{j+1}+...+N_k)$. Then \begin{equation*}
        a_j = a_1+...+a_{j-1}+a_{j+1}+...+a_k
    \end{equation*}
    for some $a_i \in N_i$, so $(a_1,...,a_{j-1},-a_j,a_{j+1},...,a_k) \in \ker \pi$, but $\pi$ is an isomorphism so $\ker\pi = \{(0,...,0)\}$ which implies $a_i = 0$ for each $i$. In particular, $a_j = 0$, so $N_j\cap(N_1+...+N_{j-1}+N_{j+1}+...+N_k) = \{0\}$.

    $(2)\implies (3)$. Suppose $\sum_{i=1}^ka_i = \sum_{i=1}^kb_i$ for $a_i,b_i \in N_i$. Then we have that for each $j$, $$a_j-b_j = \sum_{i=1}^{j-1}(b_i-a_i)+\sum_{i=j+1}^k(b_i-a_i) \in N_j\cap(N_1+...+N_{j-1}+N_{j+1}+...+N_k)$$ so by hypothesis, $a_j -b_j = 0$, so $a_j = b_j$. As this holds for all $j \in \{1,...,k\}$, the expression $\sum_{i=1}^ka_i$ is unique.

    $(3)\implies (1)$. Suppose $(a_1,...,a_k) \in \ker\pi$. Then $\sum_{i=1}^ka_i = 0 = \sum_{i=1}^k0$, where $a_i \in N_i$ and $0 \in N_i$. Thus, by hypothesis on the uniqueness of the expression of elements in terms of sums of elements of the $N_i$, we have that $a_i = 0$ for each $i$. Thus, $\ker\pi = \{(0,...,0)\}$, so $\pi$ is injective. Thus, as $\pi$ is a surjective $R$-module homomorphism by construction, it is also an $R$-module isomorphism, completing the proof.
\end{proof}

If an $R$-module $M = N_1+...+N_k$ is the sum of submodules $N_1,...,N_k$ of $M$ satisfying the equivalent conditions of the proposition above, then $M$ is said to be the (\emph{internal}) \Emph{direct sum} of $N_1,...,N_k$, written \begin{equation*}
    M = N_1\oplus ... \oplus N_k
\end{equation*}


\begin{defn}
    An $R$-module $F$ is said to be \Emph{free} on the subset $A$ of $F$ if for every nonzero element $x$ of $F$, there exists unique nonzero elements $r_1,...,r_n$ of $R$ and unique $a_1,...,a_n$ of $A$ such that $x = \sum_{i=1}^nr_ia_i$, for some $n \in \Z^+$. In this situation, we say $A$ is a \Emph{basis} or \Emph{set of free generators} for $F$. If $R$ is a commutative ring the cardinality of $A$ is called the \Emph{rank} of $F$.
\end{defn}

\begin{thm}
    For any set $A$ there is a free $R$-module $F(A)$ on the set $A$ and $F(A)$ satisfies the following \Emph{universal property}: if $M$ is any $R$-module and $\phi:A\rightarrow M$ is any map of sets, then there is a unique $R$-module homomorphism $\Phi:F(A) \rightarrow M$ such that $\Phi(a) = \phi(a)$ for all $a \in A$, that is, the followign diagram commutes:
    \begin{center}
		\begin{tikzpicture}[baseline= (a).base]
			\node[scale=1] (a) at (0,0){
			  	\begin{tikzcd}
					A && {F(A)} \\
					\\
					&& {\forall M}
					\arrow["\iota", hook', from=1-1, to=1-3]
					\arrow["{\exists!\Phi}", dashed, from=1-3, to=3-3]
					\arrow["\forall\phi"', from=1-1, to=3-3]
				\end{tikzcd}
			};
		\end{tikzpicture}
	\end{center} 
    when $A$ is the finite set $\{a_1,a_2,...,a_n\}$, $F(A) \cong Ra_1\oplus Ra_2\oplus...\oplus Ra_n\cong R^n$.
\end{thm}
\begin{proof}
    Let $F(A) = \{0\}$ if $A = \emptyset$. If $A$ is nonempty, let $F(A)$ be the collection of set functiosn $f:A\rightarrow R$ such that $f(a) = 0$ for all but finitely many $a \in A$ (i.e. finite support). Make $F(A)$ into an $R$-module by pointwise addition of functions and pointwise multiplication of a ring element times a function, i.e., \begin{align*}
        (f+g)(a) &:= f(a)+g(a) \\
        (rf)(a) &:= r(f(a)) 
    \end{align*}
    for all $a \in A$, $r \in R$, and $f,g \in F(A)$. Since $R$ is a ring addition in $F(A)$ is both associative and commutative, with additive identity the $0$ map, and for every $f:A\rightarrow R$, an additive inverse $-f$. Associativity of the $R$ action and distributivity follow from associativity and distributivity of multiplication in the ring $R$. Thus $F(A)$ is indeed a (left) $R$-module. Identify $A$ as a subset of $F(A)$ by $a \mapsto f_a$, where $f_a$ is the function which is $1$ at $a$ and zero elsewhere. We can, in this way, think of $F(A)$ as all finite $R$-linear combinations of elements of $A$ by identifying each function with the sum $\sum_{i=1}^nr_ia_i$, where $f$ takes on the value $r_i$ at $a_i$ and is zero at all other elements of $A$. Moreover, each element of $F(A)$ has a unique expression as such a formal sum. To establish the universal property of $F(A)$ suppose $\phi:A\rightarrow M$ is a map of the set $A$ into the $R$-module $M$. Define $\Phi:F(A)\rightarrow M$ by \begin{equation*}
        \Phi:\sum\limits_{i=1}^nr_ia_i\mapsto\sum\limits_{i=1}^nr_i\phi(a_i)
    \end{equation*}
    By the uniqueness of the expression for the elements of $F(A)$ as linear combinations of the $a_i$, we see that $\Phi$ is a well defined $R$-module homomorphism. By definition, the restriction $\Phi\vert_A = \phi$. Finally, since $F(A)$ is generated by $A$, once we know the values of an $R$-module homomorphism on $A$ its values on every element of $F(A)$ are uniquely determined, so $\Phi$ is the unique extension of $\phi$ to all of $F(A)$.
\end{proof}

\begin{cor}
    \leavevmode
    \begin{enumerate}
        \item If $F_1$ and $F_2$ are free modules on the same set $A$, there is a unique isomorphism between $F_1$ and $F_2$ which is the identity map on $A$.
        \item If $F$ is any free $R$-module with basis $A$, then $F\cong F(A)$. In particular, $F$ enjoys the same universal property with respect to $A$ as $F(A)$.
    \end{enumerate}
\end{cor}
\begin{proof}
    Let $F$ be a free $R$-module with basis $A$. Then I shall show $F$ enjoys the same universal property with respect to $A$ as $F(A)$. Let $M$ be an $R$ module and suppose $\varphi:A\rightarrow M$ is a map of sets. Define $\Phi:F\rightarrow M$ by $\sum_{i=1}^nr_ia_i \mapsto \sum_{i=1}^nr_i\varphi(a_i)$ for all $\sum_{i=1}^nr_ia_i \in F$. Then by construction $\Phi$ is an $R$-linear map satisfing $\Phi\circ \iota = \varphi$. If $f:F\rightarrow M$ is another map satisfying this property, then $f(a) = \varphi(a)$ for all $a \in A$, so by linearity $$f\left(\sum_{i=1}^nr_ia_i\right) = \sum_{i=1}^nr_if(a_i) = \sum_{i=1}^nr_i\varphi(a_i) = \Phi\left(\sum_{i=1}^nr_ia_i\right)$$
    so $f = \Phi$. Hence the map is unique and $F$ satisfies the universal property. 

    Then, If we consider $\iota_F:A\rightarrow F$ the inclusion map, by the universal property there exists a unique $R$-linear map $\Phi_F:F(A) \rightarrow F$ making the diagram commute. Similarly, if $\iota_{F(A)}:A\rightarrow F(A)$ is the other inclusion map, again by the universal property there exists a unique $R$-linear map $\Phi_{F(A)}:F\rightarrow F(A)$ making the diagram commute. Now, it follows that $\Phi_{F(A)}\circ \Phi_F:F(A) \rightarrow F(A)$ makes the diagram for $F(A)$ commute, but so does $\id_{F(A)}$, so by uniqueness $\Phi_{F(A)}\circ\Phi_F = \id_{F(A)}$. Dually, $\Phi_F\circ \Phi_{F(A)} = \id_F$, so $\Phi_F$ and $\Phi_{F(A)}$ are inverse module isomorphisms. Therefore $F\cong F(A)$, and in particular, for any free modules $F_1$ and $F_2$ on $A$, $F_1 \cong F(A) \cong F_2$, so $F_1 \cong F_2$, and the isomorphism $\Phi:F_1 \rightarrow F_2$ is such that $\Phi\circ \iota_{F_1} = \iota_{F_2}$, so $\Phi$ is the identity restricted to $A$, as desired.
\end{proof}


