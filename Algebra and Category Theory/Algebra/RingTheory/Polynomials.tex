%%%%%%%%%% Polynomials %%%%%%%%%%
\chapter{\textsection\textsection Polynomial Rings}

\section{\textsection Basic Definitions and Examples: Polynomial Rings}

\begin{defn}
    Let $R$ be a ring and let $x$ be a formal symbol (not related to $R$). We want to define a ring $R[x]$ such that \begin{enumerate}
        \item $x \in Z(R[x])$
        \item $R \subseteq R[x]$ is a subring
        \item $R[x]$ is generated by $\{x\}\cup R$
    \end{enumerate}
    Then, for $P \in R[x]$ we have that $P = \sum\limits_{j=0}^nb_jx^j$ where $b_j \in R$ for all $j$, and $n \in \Z$, with $n \geq 0$ (note we define $x^0 = 1_R \in R$). We also require that \begin{equation}
        \sum\limits_{j=0}^nb_jx^j = \sum\limits_{i=0}^ma_ix^i, \iff b_j = a_j \forall j\geq 0
    \end{equation}
    where if $j > n$, $b_j = 0$, and if $j > m,$ $a_j = 0$. Next, we define addition and multiplication as 
    \begin{enumerate}
        \item[]\underline{\textbf{Addition}}: $$\sum\limits_{j=0}^nb_jx^j + \sum\limits_{i=0}^ma_ix^i := \sum\limits_{i=0}^{\max(m,n)}(a_i+b_i)x^i$$
        \item[] \underline{\textbf{Multiplication}}: $$\left[\sum\limits_{j=0}^nb_jx^j\right] \left[\sum\limits_{i=0}^ma_ix^i\right] := \sum\limits_{i=0}^{m+n}\left(\sum\limits_{k=0}^ia_{k}b_{i-k})\right)x^i$$
    \end{enumerate}
\end{defn}

\begin{rmk}
    Formally, the polynomial is determined by a sequence of coefficients $a_i$ \begin{equation}
                \mathbf{a} = (a_0,a_1,a_2,...)
        \end{equation}
    where $a_i \in R$ and only a finite number of $a_i$ are not zero. The sequence with $1$ in the ith position and zero everywhere else corresponds to the indeterminate monomial $x^i$, and the monomials form a basis of the space of polynomials.
\end{rmk}

\begin{claim}
    For all rings $R$, $(R[x], +, \cdot)$ is a ring with \begin{equation}
        0_{R[x]} = 0_R + 0_R\cdot x + ...
    \end{equation}
    and \begin{equation}
        1_{R[x]} = 1_R + 0_R\cdot x + ...
    \end{equation}
    Consequently $R$ can naturally be embedded in $R[x]$ by $\map{R \hookrightarrow R[x]}{r \mapsto r\cdot x^0}$. Also, $x \in Z(R[x])$ and $R[x]$ is generated by $R\cup \{x\}$.
    \begin{proof}
        (Left to the reader)
    \end{proof}
\end{claim}

\begin{prop}
    There is a unique commutative ring structure on the set of polynomials $R[x]$ having these properties:\begin{enumerate}
        \item Addition of polynomials is done coefficient wise for equal degree monomials (like vector addition)
        \item Multiplication of monomials is given by the rule above
        \item The ring $R$ is a subring of $R[x]$, when the elements of $R$ are identified with the constant polynomials
    \end{enumerate}
\end{prop}

\begin{rmk}
    \leavevmode
    \begin{enumerate}
        \item $(a+bx+cx^2)(\alpha+\beta x) = a\alpha + (a\beta + b\alpha)x + (b\beta + c\alpha)x^2 + c\beta x^3$
        \item $R = M_2(\R)$, $I_2.x^0 = 1 \in R[x]$, and $P = A_0 + A_1x + A_2x^2 + ... + A_nx^n$, $A_i \in M_2(\R)$.
        \item $R = \Z/2\Z = \F_2$. In $\F_2[x]$, $(x+1)^2 = x^2+2x+1 = x^2+1 \pmod{2}$.
    \end{enumerate}
\end{rmk}

\begin{defn}[Polynomials in Multiple Variables]
    Let $x_1,...,x_n$ be variables (indeterminates). A \Emph{monomial} is a formal product of these variables of the form \begin{equation}
            x_1^{i_1}x_2^{i_2}...x_n^{i_n}
    \end{equation}
    where the exponents $i_v$ are nonnegative numbers. The n-tuple $(i_1,...,i_n)$ of exponents determines the monomial. Such an n-tuple is called a \Emph{multi-index}, and vector notation $\mathbf{i} = (i_1,...,i_n)$ for multi-indices is convenient. Using it, we may write the monomial symbolically as \begin{equation}
            x^{\mathbf{i}}=x_1^{i_1}x_2^{i_2}...x_n^{i_n}
    \end{equation}
    The monomial $x^{\mathbf{0}}$ is denoted by $1$.


    A polynomial with coefficients in a ring $R$ is a finite linear combination of monomials with coefficients in $R$. Using the shorthand, any polynomial $f(x) = f(x_1,...,x_n)$ can be written uniquely in the form \begin{equation}
            f(x) = \sum\limits_{i} a_{i}x^{\mathbf{i}}
    \end{equation}
    And only finitely many of the coefficients $a_i \in R$ are different from zero.


    A polynomial which is the product of a nonzero element $r \in R$ with a monomial is also called a monomial \begin{equation}
            m=rx^{\mathbf{i}}
    \end{equation}


    Using multi-index notation, the addition and multiplication for polynomials in multiple variables is analogous to the case for one variable using the formulas defined above, and the above proposition also holds analogously for polynomials in multiple variables. 

    A ring of polynomials in several variables with coefficients in the ring $R$ is denoted by \begin{equation}
         R[x_1,...,x_n]\;or\;R[x],\;x=(x_1,...,x_n)
    \end{equation}
\end{defn}


\begin{rmk}
    For a general ring $R$, $R[x]$ is not isomorphic to $\mathcal{P}(R,R) \subseteq F(R,R)$ (polynomial functions over $R$).
\end{rmk}

\begin{eg}
    Consider $R = \F_p = \Z/p\Z$ for $p$ a prime. Consider now a polynomial function $f \in F(\Z/p\Z,\Z/p\Z)$ defined by $$f = \sum\limits_{i=0}^nb_i(\hat{x})^i$$ where $(\hat{x})^i$ is the product in $F(\Z/p\Z, \Z/p\Z)$. This gives us the subring $\mathcal{P}$
    \begin{claim}
        $\mathcal{P}(\F_p,\F_p) \cancel{\cong} \F_p[x]$
        \begin{proof}
            First, observe that $\mathcal{P}(\F_p,\F_p)$ is actually finite because $|F(\F_p,\F_p)| = p^p < +\infty$ and $\mathcal{P}(\F_p,\F_p) \subseteq F(\F_p,\F_p)$. However, $|\F_p[x]|$ is infinite. Indeed, for all $i \neq j\geq 0$, $x^i \neq x^j$, so $\{1,x,x^2,...\} = \{x^n:n\geq 0\} \subseteq \F_p[x]$ is a subset of infinite order. Moreover, note that $(\hat{x})^p - \hat{x} = 0 \in F(\F_p,\F_p)$ by Fermat's theorem because $$((\hat{x})^p - \hat{x})(a) = a^p-a = a-a = 0 \pmod{p}$$
            But, $x^p - x \neq 0 \in \F_p[x]$.
        \end{proof}
    \end{claim}
    Thus, it is not sufficient to consider polynomial functions.
\end{eg} 

\begin{claim}
    However, $\R[x] \cong \mathcal{P}(\R,\R)$
\end{claim}
\begin{proof}
    Define a function $$\map{\Phi:\R[x] \rightarrow \mathcal{P}(\R,\R)}{p\mapsto \left(\map{\ev_p:\R\rightarrow \R}{r \mapsto \ev_r(p)}\right)}$$
    First, note that for all $p = \sum_ia_ix^i \in \R[x]$ and all $r \in \R$, $\ev_r(p) = \sum_ia_ir^i$, so we have that $\ev_p = \sum_ia_i(\hat{x})^i \in \mathcal{P}(\R,\R)$, so the function is well-defined. Recall that $\ev_r(p)$ is a ring homomorphism for all $r \in \R$. Now, let $p = \sum_ia_ix^i, q = \sum_ib_ix^i \in \R[x]$. Then, observe that \begin{align*}
        \Phi(p+q)(r) &= \ev_{p+q}(r)  & \Phi(p\cdot q)(r) &= \ev_{p\cdot q}(r)\\
        &= \ev_r(p+q) & &= \ev_r(p\cdot q) \\
        &= \ev_r(p) + \ev_r(q) & &= \ev_r(p) \cdot\ev_r(q) \\
        &= \ev_p(r) + \ev_q(r) &  &= \ev_p(r) \cdot \ev_q(r) \\
        &= \Phi(p)(r) + \Phi(q)(r) & &= \Phi(p)(r) \cdot \Phi(q)(r)\\
        &= (\Phi(p) + \Phi(q))(r) & &= (\Phi(p) \cdot \Phi(q))(r)
    \end{align*}
    and $$\Phi(1)(r) = \ev_{1}(r) = \ev_r(1) = 1$$
    for all $r \in \R$. Thus, we have that $\Phi$ is a homomorphism of rings. Then, let $p \in \ker(\Phi)$ and for the sake of contradiction suppose $\deg(p) = n$ for some $n \geq 0$. Then, we have that $p$ has at most $n$ roots, $\{r_1,r_2,...,r_n\}$. Let $r \in \R$ with $r \notin\{r_1,r_2,...,r_n\}$. Then, by assumption we have $$0 = \Phi(p)(r) = \ev_p(r) = \ev_r(p)$$
    But, $\ev_r(p)$ is the remainder for the division of $p$ by $x - r$, so this implies $x-r$ divides $p$. However, we would then have $r$ as a root of $p$, but by assumption $r$ is not one of the $n$ roots of $p$, so this is a contradiction. Therefore, we must have that $p = 0$, so $\ker(\Phi)$ is trivial. Now, let $f = \sum_ia_i(\hat{x})^i \in \mathcal{P}(\R,\R)$. Observe that for $q = \sum_ia_ix^i$, we have for all $t \in \R$ $$\Phi(q)(t) = \ev_q(t) = \ev_t(q) = \sum_ia_it^i = (\sum_ia_i(\hat{x})^i)(t)$$
    so $\Phi(q) = f$, and in particular $\Phi$ is a surjection. Therefore, we conclude that $\Phi$ is an isomorphism of rings, so \begin{equation}
        \R[x] \cong \mathcal{P}(\R,\R)
    \end{equation}
\end{proof}


\begin{defn}
    Let $R$ be an arbitrary unital ring. $P \in R[x]$ is called a polynomial with coefficients in $R$. For $P = a_0+a_1x+...+a_nx^n$, $a_0$ is called the \Emph{constant coefficient}. Now, assume $P \neq 0$. Then $\max\{i\geq 0:a_i \neq0\}$ is the \Emph{degree of P}, denoted $\deg(P)$. $a_{\deg(P)}$ is called the \Emph{leading coefficient of P}.
\end{defn}

\begin{defn}
    Let $R$ be a ring and let $r \in Z(R)$. Then, the map \begin{equation}
        \map{\ev_r:R[x]\rightarrow R}{P = \sum_ia_ix^i \mapsto P(r) = \sum_ia_ir^i}
    \end{equation}
    is a surjective ring homomorphism, called the evaluation at $r$. Denote $\ev_r(P)$ by $P(r)$.
    \begin{proof}
        (Left to the reader)
    \end{proof}
\end{defn}

\section{\textsection Division Algorithm}

\begin{defn}
    A non-zero polynomial is called \Emph{monic} if its leading coefficient is $1$.
\end{defn}

\begin{eg}
    \leavevmode
    \begin{enumerate}
        \item $\deg(\underbrace{3x^2+x+1}_{\in\Z[x]}) = 2$, leading coefficient $= 3$.
        \item $x^2+7x_1 \in \Z[x]$ is monic of degree $2$
        \item $2$ has degree $0$.
    \end{enumerate}
\end{eg}

\begin{rmk}
    $P,Q \in R[x]$, then if $PQ \neq 0$, then $\deg(PQ) \leq \deg(P) + \deg(Q)$. If $R$ is a domain and $P \neq 0$, $Q \neq 0$, then $\deg(PQ) = \deg(P) + \deg(Q)$.
    \begin{proof}
        (Left to the reader)
    \end{proof}
\end{rmk}

\begin{eg}
    $\Z/6\Z[x]$, $(2x)(3x+1) = 6x^2+2x = 2x$ of degree $1 < \deg(P) + \deg(Q) = 1 + 1 = 2$.
\end{eg}

\begin{cor}
    \leavevmode
    \begin{enumerate}
        \item If $R$ is a domain, then $R[x]$ is a domain
        \item The units of $R[x]$ are the units of $R$.
    \end{enumerate}
    \begin{proof}
        (Left to the reader)
    \end{proof}
\end{cor}

\begin{thm}[Division Algorithm]
    Let $R$ be a ring. Let $P,Q \in R[x]$. Assume that $P \neq 0$ and that the leading coefficient of $P$ is a unit ($\in R$). Then there are unique polynomials $f$ and $g \in R[x]$ such that: \begin{enumerate}
        \item $Q = fP + g$
        \item $g = 0$ or $\deg(g) < \deg(P)$
    \end{enumerate}
    ($g$ is called the remainder of the division of $Q$ by $P$)
\end{thm}
\begin{proof}
    Write $\deg(Q) = m$ and $\deg(P) = n$. If $Q = 0$ or $m < n$ then $Q = 0P + Q$ does it. Hence, suppose $m \geq n$, and proceed by induction on $m$. Write $P = ux^n + a_{n-1}x^{n-1} + ...$ and $Q = b_mx^m + b_{m-1}x^{m-1} + ...$, where $u \in R^{\times}$ by hypothesis. Consider the new polynomial \begin{align*}
        g_1 &= Q - b_mu^{-1}x^{m-n}P \\
        &= (b_{m-1} - b_mu^{-1}a_{n-1})x^{m-1} + ...
    \end{align*}
    where we use the fact that $x$ is central in $R[x]$. Hence, either $g_1 = 0$ or $\deg(g_1) < m$ so, by induction, polynomials $q_1$ and $r$ exist such that $g_1 = Pq_1 + r$ and either $r = 0$ or $\deg(r) < \deg(P)$. But then $$Q = g_1 + b_mu^{-1}x^{m-n}P = (q_1 + b_mu^{-1}x^{m-n})P + r$$
    Hence, the induction is satisfied, so $f$ and $g$ exist satisfying the claim. 
    
    To prove uniqueness, suppose that also $Q = f_1P + g_1$, where either $g_1 = 0$ or $\deg(g_1) < \deg(P)$. Then $(g-g_1) = (f_1-f)P$. If $(f_1-f) \neq 0$, then since the leading coefficient of $P$ is a unit $(f_1-f)P \neq 0$, and that $$\deg(g-g_1) = \deg[(f_1-f)P] = \deg(f_1-f)+\deg(P)$$ But, this implies $\deg(g-g_1) \geq \deg(P)$, but $\deg(g-g_1) \leq \max\{\deg(g),\deg(g_1)\} < \deg(P)$, a contradiction. Thus, we must have that $(f_1-f) = 0$, and whence $(g-g_1) = (f_1 - f)P = 0$, proving uniqueness.
\end{proof}

\begin{cor}
    A non-zero polynomial of degree $n$ over any field has at most $n$ roots.
\end{cor}
\begin{proof}
    (Left to the reader)
\end{proof}


\begin{eg}
    \leavevmode
    \begin{enumerate}
        \item For $\Z[x]$, $P = -x+1$, $Q = x^2$, $Q = P(-x-1) + 1$, where $\deg(1) = 0 < 1 = \deg(P)$
        \item Conversely, $P = 0.Q + (-x+1)$, where $\deg(-x+1) = 1 < 2 = \deg(x^2)$.
    \end{enumerate}
\end{eg}

\begin{cor}
    Let $R$ be a commutative ring with $a \in R$, and $P \in R[x]$. \begin{enumerate}
        \item $\ev_a(P) = 0$ if and only if $P = (x-a)Q$ for some $Q \in R[x]$.
        \item The remainder of the division of $P$ by $x-a$ is $P(a)$.
    \end{enumerate}
    \begin{proof}
        First, observe that if $P = (x-a)Q$ for some $Q \in R[x]$. Then $\ev_a(P) = 0\ev_a(Q) = 0$, satisfying the implication. On the other hand, since the leading coefficient of $x-a$ is a unit, we have by the division algorithm that there exists $f,g \in R[x]$ such that $P = f(x-a) + g$, where $g = 0$ or $\deg(g) < \deg(x-a) = 1$. Thus, $g = r$ for some $r \in R$. Then, observe that $\ev_a(P) = \ev_a(f(x-a) + r) = \ev_a(f)0+r = r$, since $\ev_a$ is a ring homomorphism. Note that this proves the second claim. Now, by assumption $\ev_P = 0$, $r = 0$. Thus, $P = (x-a)f$, completing the proof.
    \end{proof}
\end{cor}


\begin{defn}
    Let $R$ be a commutative ring. We say that a polynomial $f \in R[X]$ \Emph{divides} a polynomial $g \in R[X]$ denoted $f\;\vert\;g$, and that $f$ is a \Emph{divisor} of $g$ if there exists $P \in R[X]$ such that $g = fP$.
\end{defn}

\begin{cor}
    Let $\F$ be a field. Let $f,g \in \F[X]$ not both zero. Then \begin{equation}
        (f,g) := \{fP+gQ: P,Q\in\F[X]\} = (d)
    \end{equation}
    where $d$ is the monic generator of minimal degree, which satisfies \begin{enumerate}
        \item $d$ divides $f$ and $g$
        \item If a polynomial $P$ divides $f$ and $g$, then $P$ divides $d$
        \item There exist $Q_1, Q_2 \in \F[X]$ such that $d = fQ_1 + gQ_2$ (one says that $d$ is a linear combination of $f$ and $g$)
    \end{enumerate}
    Thus, every ideal $I$ in the ring $R = F[x]$ is principal, $I = (f)$, generated by the monic polynomial $f$ in $I$ of least degree.
    \begin{proof}[A]
        Let $I$ be an ideal. If $I \neq (0)$, take $f \in I$ of minimal degree, n. Scale $f$ by $c = a_n^{-1}$ to make $f$ monic. Note that since $c \in F[x]$, $c.f \in I$. Let $h$ be another polynomial in $I$, and write $h(x) = q(x)f(x) + r(x)$, with the degree of $r(x)$ less than $f(x)$. Note that $q(x) \in F[x]$, so $q(x)f(x) \in I$, and $h(x) - q(x)f(x) \in I$. Thus, $r(x) \in I$. But, $r(x)$ has a smaller degree than $f$, so $r(x) = 0$. Thus, $h(x) = q(x)f(x)$, so $f(x)$ divides $h(x)$, and $I = (f)$. 
    \end{proof}
    \begin{proof}[B]
        (Left to the reader)
    \end{proof}
\end{cor}

\begin{rmk}
        Thus, the set of ideals is in a one-to-one correspondence with the set of monic polynomials, and the ideal associated to $f$, $I_f$, contains the ideal generated by the monic polynomial $g$, that is $I_f \supset I_g$, if and only if $f$ divides the polynomial $g$. Namely, $g(x)=f(x)q(x)$ for some $q(x) \in F[x]$. 
\end{rmk}

\begin{eg}[Non-principal Ideals]
        Take the ring $R = F[x,y] = \left\{\sum\limits_{i=1,j=1}^{n,m} a_{ij}x^iy^j:a_{ij} \in F\right\}$. Consider the map $h:R \rightarrow F$ by $f(x,y)\mapsto f(0,0)$. The kernel of $h$ is not generated by one element (so it's not principal). In fact, $\ker h = (x,y) = \{rx+sy:r,s \in R\}$.
\end{eg}




\begin{defn}
    Let $f,g \in \F[x]$ not both zero, then $d$ in the above corollary is called the \Emph{greatest common divisor} of $f$ and $g$, denoted $\gcd(f,g)$.
    \begin{enumerate}
        \item[$\drsh$] If $\gcd(f,g) = 1$, then $f$ and $g$ are said to be \Emph{relatively prime}. 
    \end{enumerate}
\end{defn}


\begin{qest}
    How do we compute the $\gcd$?
\end{qest}

\begin{eg}
    In $\Q[x]$
    \begin{enumerate}
        \item $(x^2+1,x) = (1)$ because $1 = (x^2+1) + (-x)(x) \in (x^2+1,x)$
        \item $\gcd(x^2-1,2x+2) = \gcd((x-1)(x+1),2(x+1)) = (x+1)$
    \end{enumerate}
\end{eg}


\begin{rmk}
    In general, there is the \Emph{euclidean algorithm} for polynomials over a field.
\end{rmk}

\begin{lem}
    Let $f,g \in \F[x]$, $\F$ a field, not both zero. If $g = Pf + Q$ for some $P,Q \in \F[x]$ then $\gcd(f,g) = \gcd(f,Q)$.
\end{lem}
\begin{proof}
    Let $d = \gcd(f,g)$, $d' = \gcd(f,Q)$. Then I claim $(f,Q) = (f,g)$. Note $f \in (f,Q)$ and $g = Pf + Q \in (f,Q)$ so $(f,g) \subseteq (f,Q)$. Similarly, $f \in (f,g)$ and $Q = g - Pf \in (f,g)$, so $(f,Q) \subseteq (f,g)$. Thus, $(f,g) = (f,Q)$ so $(d) = (f,g) = (f,Q) = (d')$. Hence, $d = d'$ and the proof is complete.
\end{proof}

\begin{namthm}[Euclidean Algorithm]
    Let $f,g \in \F[x]$, not both zero. Assume $f \neq 0$, so the leading coefficient of $f$ is $a \neq 0$. Then $a$ is a unit in $\F$ since $\F$ is a field. By the division algorithm there exist unique $P, Q \in \F[x]$ such that \begin{equation}
        g = Pf + Q
    \end{equation}
    with $Q = 0$ or $\deg(Q) < \deg(f)$. By the Lemma $\gcd(f,g) = \gcd(f,Q)$. If $Q = 0$, $\gcd(f,g) = \gcd(f,0) = a^{-1}f$, where $a$ is the leading coefficient of $f$. Otherwise, if $Q \neq 0$ we apply the division algorithm again to obtain $P',Q' \in \F[x]$ satisfying $f = P'Q + Q'$ for $Q' = 0$ or $\deg(Q') < \deg(Q)$. We then have $\gcd(f,g) = \gcd(f,Q) = \gcd(Q, Q')$. This process must terminate eventually because the degree of the remainder strictly decreases at every step and must be non-negative (or the remainder is $0$).
\end{namthm}

\begin{eg}
    The greatest common divisor of $f = x^3 + x + 2,$ $g = x^3 + x^2 + x + 1$ in $\Q[x]$: First $f = g\cdot 1 + (1-x^2)$. Then $g = (1-x^2)(-x-1) + 2x+2$, and finally $(-x^2+1) = (-x/2+1/2)(2x+2) + 0$. Thus, $\gcd(f,g) = \gcd(g,1-x^2) = \gcd(1-x^2,2x+2) = \gcd(2x+2,0) = x+1$, by making the remainder monic. Working backwards we have $$(x+1) = g/2 - (x+1)(x^2-1)/2 = g/2 - (x+1)(g-f)/2 = -gx/2+f(x+1)/2$$
\end{eg}


\subsection{\textsection Solutions to Polynomials}

\begin{prop}
    If $p \in \Z[x]$ and $\ev_c(p) = 0$ for some $c \in \Z$, then $\ev_c(p) = 0$ in $\Z/m\Z$ for all $m \in \Z^{+}$. The contrapositive is important so it shall be stated: If there exists $m \in \Z^{+}$ such that $\ev_c(p) \neq 0$ in $\Z/m\Z$ for a polynomial $p \in \Z[x]$, then $\ev_c(p) \neq 0$ in $\Z$, and in particular $c$ is not a root of $p$ in $\Z$.
    \begin{proof}
        Let $p \in \Z[x]$ and suppose $\ev_c(p) = 0$. Then, let $m \in \Z^{+}$. Write $p = \sum_ia_ix^i$. Then observe that $\pi(p) = \sum_i[a_i]_mx^i = \left[\sum_ia_ix^i\right]$. Then, since $\ev_c$ is a ring homomorphism we have that $$\ev_c(\pi(p)) = \sum_i[a_i]_mc^i = \left[\sum_ia_ic^i\right] = \left[\ev_c(p)\right] = [0]$$ in $\Z/m\Z$, completing the proof.
    \end{proof}
\end{prop}

\section{\textsection Substitution Principle}

\begin{namthm}[Substitution Principle]
        Let $\phi:R\rightarrow R'$ be a ring homomorphism. \begin{enumerate}
                \item Given an element $\alpha \in R'$, there is a unique homomorphism $\Phi: R[x] \rightarrow R'$ which agrees with the map $\phi$ on constant polynomials, and which sends $x \mapsto \alpha$
                \item More generally, given $\alpha_1,...,\alpha_n \in R'$, there is a unique homomorphism $\Phi: R[x_1,...,x_n] \rightarrow R'$ from the polynomial ring in n variables to $R'$, which agrees with $\phi$ on constant polynomials and which sends $x_i\mapsto \alpha_i$, for $i=1,2,...,n$
        \end{enumerate}
\end{namthm}
\begin{proof}
        With vector notation for indices, the proof of (2) is identical to that of (1). Let us denote the image of an element $r \in R$ in $R'$ by $r'$. Using the fact that $\Phi$ is a homomorphism which restricts to $\phi$ on $R$, and sends $x_v \mapsto \alpha_v$, we find that it acts on a polynomial $f(x) = \sum r_ix^i$ by sending \begin{equation}
                \sum r_ix^i \mapsto \sum\phi(r_i)\alpha^i = \sum r_i'\alpha^i
        \end{equation}
        In other words, $\Phi$ acts on the coefficients of a polynomial as $\phi$, and it substitutes $\alpha$ for $x$. Since this formula describes $\Phi$ completely for us, we have proved the uniqueness of the substitution homomorphism. To prove its existence, we take this formula as the definition of $\Phi$, and we show that the map is a ring homomorphism $R[x] \rightarrow R'$. Since $\phi$ is a ring homomorphism, $\Psi$ sends $1$ to $1$, and by the above formula, it is compatible with addition of polynomials. Using the formula we also find that it is compatible with multiplication as \begin{align*}
                \Psi(fg) &= \Psi\left(\sum a_ib_jx^{i+j}\right) \\
                &= \sum \Psi(a_ib_jx^{i+j})\\
                &= \sum\limits_{i,j} a_i'b_j'\alpha^{i+j}\\
                &= \left(\sum\limits_ia_i'\alpha^i\right)\left(\sum\limits_jb_j'\alpha^j\right) \\
                &=\Psi(f)\Psi(g)
        \end{align*}
\end{proof}


\begin{eg}
        We consider the case of a homomorphism $\Z \rightarrow \Z/p\Z$. This map extends to a homomorphism \begin{equation}
                \Z[x] \rightarrow \Z/p\Z[x], f(x) = a_nx^n+\hdots + a_0 \mapsto \overline{a_n}x^n+\hdots + \overline{a_0} = \overline{f}(x)
        \end{equation}
        where $\overline{a_i}$ denotes the \Emph{residue class} of $a_i$ modulo $p$. We call the polynomial $\overline{f}(x)$ the \Emph{residue of $f(x)$ modulo $p$}.
\end{eg}


\begin{cor}
        Let $x = (x_1,...,x_n)$ and $y = (y_1,...,y_m)$ denote set of variables. There is a unique isomorphism $R[x,y] \xrightarrow{\sim}R[x][y]$ which is the identity on $R$ and which sends the variables to themselves.
\end{cor}
\begin{proof}
        Note that $R$ is a subring of $R[x]$, and that $R[x]$ is a subring of $R[x][y]$. So $R$ is also a subring of $R[x][y]$. Consider the inclusion map $\phi:R\hookrightarrow R[x][y]$. The Substitution Principle tells us that there is a unique homomorphism $\Phi:R[x,y] \rightarrow R[x][y]$ which extends the map and sends variables $x_{\mu},y_{\nu}$ wherever we wish. Thus, we can send the variables to themselves. The map $\Phi$ constructed is thus the desired isomorphism. Using the Substitution Principle once more, we note that $R[x]$ is a subring of $R[x,y]$, so we can extend the inclusion map $\psi:R[x] \rightarrow R[x,y]$ to a map $\Psi:R[x][y] \rightarrow R[x,y]$ by sending $y_j$ to itself. The composed homomorphism $\Psi\Phi: R[x,y] \rightarrow R[x,y]$ is the identity on $R$ and on $\{x_{\mu},y_{\nu}\}$. By uniqueness of the Substitution Principle, $\Psi\Phi$ is the identity map. Similarly, $\Phi\Psi$ is the identity on $R[x][y]$. Thus, $\Phi$ is a bijective homomorphism, so it is an isomorphism.
\end{proof}


\begin{prop}
        Let $\mathcal{R}$ denote the ring of continuous real-valued functions on $\R^n$. The map $\phi:\R[x_1,...,x_n]\rightarrow \mathcal{R}$ sending a polynomial to its associated polynomial function is an injective homomorphism.
\end{prop}
\begin{proof}
        The existence of this homomorphism follows from the Substitution Principle. To prove injectivity, it is enough to show that if the function associated to a polynomial $f(x)$ is the zero function, then $f(x)$ is the zero polynomial. Let the associated function be $\widetilde{f}(x)$. If $\widetilde{f}(x)$ is identically zero, then all its derivatives are zero too. On the other hand we can differentiate a formal polynomial by using the power rule and the linearity of the derivative. If some coefficients of $f(x)$ are nonzero, then the constant term of a suitable derivative will be nonzero too. Hence, that derivative will not vanish at the origin. Therefore, $\widetilde{f}(x)$ can't be the zero function.
\end{proof}


\begin{prop}
        There is exactly one ring homomorphism \begin{equation}
                \phi:\Z\rightarrow R
        \end{equation}
        from the ring of integers to an arbitrary ring $R$. It is the map defined by $\phi(n) = 1_R+...+1_R$ n-times if $n > 0$, and $\phi(-n) = -\phi(n)$.
\end{prop}


\begin{rmk}
        This allows us to identify the images of the integers in an arbitrary ring $R$. We can hence interpret the symbol $3$ as $1+1+1$ in $R$.
\end{rmk}


\section{\textsection Roots and Factorization}

For this section let $R$ be a commutative ring and $P \in R[x]$.

\begin{defn}
    $r \in R$ is a \Emph{root} of $P$ if $\ev_r(P) = P(r) = 0$. Note that this happens if and only if $P = (x-r)Q$ for some $Q \in R[x]$.
\end{defn}

\begin{qest}
    What if $Q(r) = 0$ as well?
\end{qest}

\begin{defn}
    Let $\alpha \in R$ be a root of $P$. We say that $\alpha$ is a root of $P$ of \Emph{multiplicity} $n \geq 1$ if $P = (x-\alpha)^nQ'$ for some $Q' \in R[x]$ and $Q'(\alpha) \neq 0$.
\end{defn}

\begin{prop}
    Let $R$ be an integral domain, and let $P \neq 0$ be a polynomial of degree $n$. Then $P$ has at most $n$ roots counted with multiplicities.
\end{prop}
\begin{proof}
    We argue by induction on the degree of $P$. If $n = 0$ then $P = c \neq 0$, so $P$ has no roots. Thus, the base case holds. Then, suppose that there exists $k \geq 0$ such that for all $j \leq k$, if $n = k$ $P$ has at most $k$ roots counting multiplicities. Then, consider $n = k+1$. If $P$ has no roots then we are done. Otherwise, let $\alpha$ be a root of $P$. Then $P = (x-\alpha)Q$ for some $Q \in R[x]$ such that $\deg(Q) = k+1-1 = k$. Thus, by our induction hypothesis $Q$ has at most $k$ roots counting with multiplicities. Thus, $P$ has at most $k + 1$ roots counting with multiplicities, as desired.
\end{proof}

\begin{rmk}
    Note that if $R$ is not an integral domain this is not true.
    \begin{enumerate}
        \item[$\drsh$] If $R = \Z/6\Z$, then observe $$P = (x-2)(x-3) = x^2-5x+6 = x^2-5x = x(x-5)$$ 
        so $\{0,2,3,5\}$ are roots of $P$ and $P$ is of degree $2$.
    \end{enumerate}
    Moreover, even if $R$ is an integral domain, $P \in R$ with $\deg(P) > 1$ may have no roots. For example, $x^2+1 \in \R[x]$ has no roots in $\R$.
\end{rmk}

\begin{cor}
    If $R$ is an integral domain and $0 \neq P \in R[x]$ of degree $n$, then $P$ has at most $n$ distinct roots.
\end{cor}

\begin{defn}[Irreducible]
    Let $R$ be an integral domain.
    \begin{enumerate}
        \item An element $a \in R$ is \Emph{irreducible} if
        \begin{enumerate}
            \item $a \notin R^{\times}$ and $a \neq 0$
            \item If $a =bc$ for some $b,c \in R$, then $b \in R^{\times}$ or $c \in R^{\times}$
        \end{enumerate}
        \item If $P \in R[x]$ is irreducible, we say that $P$ is \Emph{irreducible over $R$}.
    \end{enumerate}
\end{defn}

\begin{eg}
    \leavevmode
    \begin{enumerate}
        \item $x^2+1$ is irreducible over $\R$
        \item $x^2+1 = (x+i)(x-i)$ is not irreducible over $\C$
        \item $2x+2$ is irreducible over $\Q$, but not over $\Z$ as $2 \notin \Z^{\times} = \{1,-1\}$
        \item A linear polynomial (i.e$\rangle$ of degree $1$) over a field is irreducible.
        \begin{proof}
            Let $F$ be a field and let $P \in F[x]$ of degree $1$. Then $P \neq 0$ and $P \notin F[x]^{\times}$. Moreover, if $P = fg$ for some $f,g \in F[x]$ then $\deg(fg) = \deg(f)+\deg(g) = \deg(P) = 1$, so either $\deg(f) = 0$ or $\deg(g) = 0$. Thus, either $g \in F$ or $f \in F$ with $g,f \neq 0$, so in particular either $g \in F[x]^{\times}$ or $f \in F[x]^{\times}$. Therefore, $P$ is irreducible over $F$.
        \end{proof}
    \end{enumerate}
\end{eg}

\begin{prop}
    Let $F$ be a field and $P \in F[x]$ with $\deg(P) \geq 2$.  
    \begin{enumerate}
        \item If $P$ is irreducible over $F$, then $P$ has no roots in $F$
        \item If $\deg(P) \in \{2,3\}$, then $P$ is irreducible over $F$ if and only if $P$ has no roots in $F$.
    \end{enumerate}
\end{prop}
\begin{proof}
    If $P$ has a root $a \in F$, then $P = (x-a)Q$ for some $Q \in F[x]$. But, $\deg(P) \geq 2$ so $\deg(Q) \geq 1$, and hence $Q \notin F[x]^{\times}$. Thus, $P$ is not irreducible over $F$. On the other hand, suppose $\deg(P) \in \{2,3\}$. Suppose $P$ is not irreducible over $F$ so $P = p_1p_2$ for some $p_1, p_2 \in F[x]$ such that $\deg(p_1),\deg(p_2) \geq 1$. But, $\deg(p_1)+\deg(p_2) = \deg(P) \in \{2,3\}$ so either $\deg(p_1) = 1$ or $\deg(p_2) = 1$. Without loss of generality suppose $\deg(p_1) = 1$. Then $p_1 = ax+b$ for $a,b \in F$ and $a \neq 0$. Since $F$ is a field $a^{-1} \in F$ and $\ev_{a^{-1}(-b)}(p_1) = 0$. Hence, as $P = p_1p_2$ $P$ has a root in $F$, completing the proof.
\end{proof}

\begin{rmk}
    \leavevmode
    \begin{enumerate}
        \item In general no roots $\cancel{\implies}$ irreducible
        \begin{enumerate}
            \item[$\drsh$] For example, $(x^2+1)^2 \in \R[x]$ is not irreducible but it has no roots in $\R$. 
        \end{enumerate}
        \item $x^2+x+1$ is irreducible over $\Z/2\Z$ (no roots)
        \item $x^2-2$ is irreducible over $\Q$ (no roots), but it has roots over $\R$.
    \end{enumerate}
\end{rmk}

\begin{defn}[Algebraically Closed]
    A field $F$ is \Emph{algebraically closed} if every non-constant polynomial $P \in F[x]$ has a root.
\end{defn}

\begin{namthm}[Fundamental Theorem of Algebra]
    The field of complex numbers $\C$ is algebraically closed. So, $P = a(x-r_1)(x-r_2)...(x-r_n)$ for $P \in \C[x]$, $\deg(P) = n$, $a \in \C\backslash\{0\}$ the leading coefficient of $P$, and $\{r_1,r_2,..,r_n\}$ the roots of $P$ (not necessarily distinct).
\end{namthm}

\subsection{\textsection \texorpdfstring{Polynomials over $\Q$ and $\Z$}{}}

\begin{namthm}[Rational Roots Theorem]
    Let $P \in \Z[x]$, $P = a_0+a_1x+...+a_nx^n$, $a_n \neq 0 \neq a_0$. Then every root of $P$ in $\Q$ is of the form $\frac{c}{d}$ such that $c\;\vert\;a_0$ and $d\;\vert\;a_n$. In particular, if $P$ is monic, i.e $a_n = 1$, then every rational root of $P$ is in $\Z$.
\end{namthm}
\begin{proof}
    Suppose $\frac{c}{d}$ is a root of $P$ in $\Q$, and assume $\gcd(c,d) = 1$. Then, $$P\left(\frac{c}{d}\right) = a_0+a_1\frac{c}{d}+ ... + a_n\frac{c^n}{d^n} = 0$$
    Multiply by $d^n$ to obtain $$a_0d^n+a_1cd^{n-1}+...+a_{n-1}c^{n-1}d+a_nc^n = 0$$
    Then, since $a_1cd^{n-1},...,a_nc^n$ are divisible by $c$, we must have that $c$ divides $a_0d^n$. But, $\gcd(c,d) = 1$, so as can be shown by induction, $\gcd(c,d^n) = 1$. Hence, $c$ divides $a_0$. Indeed, $cx+d^ny = 1$ for some $x,y \in \Z$, so $a_0 = c(a_0x+ky)$, where $a_0d^n = ck$. Similarly, $a_nc^n$ is divisible by $d$ as $a_0d^n,...,a_{n-1}c^{n-1}d$ are divisible by $d$. Thus, again $\gcd(d,c^n) = 1$ so $d$ divides $a_n$, completing the proof.
\end{proof}

\begin{eg}
    $P = x^3+2x^2+\frac{3}{5}x+2$ is irreducible over $\Q$. Indeed, since $\deg(P) = 3$, $P$ is irreducible over $\Q$ if it has no roots in $\Q$. To put $P$ in the form of Rational Roots Theorem we multiply by $5$:
    $$5P = 5x^3+10x^2+3x+10 \in \Z[x]$$
    By the Rational Roots Theorem, if $5P$ has a root $\frac{c}{d} \in \Q$ then $c\;\vert\;10$ and $d\;\vert\;5$. Thus, $d \in \{\pm 1,\pm 5\}$ and $c \in \{\pm1,\pm2,\pm5,\pm10\}$. In particular $$\frac{c}{d} = \{\pm1,\pm2,\pm5,\pm10,\pm\frac{1}{5},\pm\frac{2}{5}\}$$
    Upon direct computation none of these values are roots of $5P$, so in particular $5P$ has no roots in $\Q$. Thus, $P$ has no roots over $\Q$ and is hence irreducible over $\Q$.
\end{eg}

\begin{namthm}[Gauss' Lemma]
    Let $f,g,h \in \Z[x]$ such that $f = gh$. If a prime $p$ divides every coefficient of $f$, then $p$ divides every coefficient of $g$ or $p$ divides every coefficient of $h$.
\end{namthm}
\begin{proof}
    Consider the surjective ring homomorphism $$\map{\Z[x]\xrightarrow{\alpha}{\Z/p\Z[x]}}{a_0+a_1x+...+a_nx^n\mapsto [a_0]_p+[a_1]_px+...+[a_n]_px^n}$$
    Then $[0]_p = \alpha(f) = \alpha(g)\alpha(h)$. But, since $\Z/p\Z$ is a field, it is an integral domain and we must have that $\alpha(g) = 0$ or $\alpha(h) = 0$. That is, $g \in (p)$ or $h \in (p)$, completing the proof.
\end{proof}

\begin{defn}
    For all $f \in \Z[x]$, $\alpha(f) \in \Z/p\Z[x]$ is called \Emph{reduction modulo p of $f$}.
\end{defn}

\begin{cor}
    Let $f$ be a non-constant polynomial in $\Z[x]$. \begin{enumerate}
        \item If $f =gh$ with $g,h \in \Q[x]$, then there exist $g_0,h_0 \in \Z[x]$ so that $\deg(g_0) = \deg(g)$, $\deg(h_0) = \deg(h)$ and $f = g_0h_0$
        \item $f$ is irreducible over $\Q$ if and only if $f$ cannot be written $f = g_0h_0 \in \Z[x]$, where $g_0,h_0$ are non-constant.
    \end{enumerate}
\end{cor}
\begin{proof}
    Let $f \in \Z[x]$, with $\deg(f) \geq 1$. 
    
    1) Suppose $f = gh$ with $g,h \in \Q[x]$. Let $a$ and $b$ be least common multiples of the denominators of the coefficients of $g$ and $h$, respectively. Then $g' = ag$ and $h' = bh$ are in $\Z[x]$. Moreover, we have the equation $abf = g'h'$ in $\Z[x]$. If $ab = 1$ then we're done, so suppose $ab > 1$ and let $p$ be a prime dividing $ab$. Then by Gauss' Lemma $p$ divides $g'$ or $h'$. Hence, $p$ can be cancelled to give \begin{equation}
        \frac{ab}{p}f = g_2h_2
    \end{equation}
    in $\Z[x]$. Repeat for all prime factors of $ab$ to obtain $f = g_0h_0$ in $\Z[x]$ with $\deg(g_0) = \deg(g)$ and $\deg(h_0) = \deg(h)$.
    
    2) If $f = g_0h_0 \in \Z[x]$ for $g_0,h_0$ non-constant, then $0 \neq g_0,h_0 \notin \Q[x]^{\times}$, so $f$ is not irreducible. Conversely, if $f$ is not irreducible over $\Q$ then $f = gh \in \Q[x]$ and the result follows from 1), completing the proof.
\end{proof}

\begin{defn}
    Let $f \in \Z[x]$. A \Emph{proper factorization} of $f$ is $f = g_0h_0$ where $\deg(g_0) \geq 1$ and $\deg(h_0) \geq 1$.
\end{defn}


\begin{namthm}[Modular Irreducibility Test]
    Let $0 \neq f \in \Z[x]$ such that there exists a prime number $p$ with: \begin{enumerate}
        \item $p$ does not divide the leading coefficient of $f$
        \item The reduction $\alpha(f)$ of $f$ modulo $p$ is irreducible over $\Z/p\Z$
    \end{enumerate}
    Then $f$ is irreducible over $\Q$.
\end{namthm}
\begin{proof}
    Suppose $f \in \Z[x]$ such that $f$ satisfies the conditions of the theorem. Then, for the sake of contradiction suppose $f$ is not irreducible over $\Q$. Then $f = gh$ for some non-constant $g,h \in \Q[x]$. By the corollary to Gauss' Lemma we have that $f = g_0h_0$ for $g_0,h_0 \in \Z[x]$ non-constant polynomials. Then, we have that $\alpha(f) = \alpha(g_0)\alpha(h_0)$. Note that if $a$ and $b$ are the leading coefficients of $g_0$ and $h_0$ respectively, then $ab$ is the leading coefficient of $f$, which by assumption is not divisible by $p$. Thus, $a$ and $b$ are not divisible by $p$, so $\deg(\alpha(g_0)) \geq 1$ and $\deg(\alpha(h_0)) \geq 1$. But, this implies that $\alpha(g_0),\alpha(h_0)$ are not units in $\Z/p\Z[x]$, so $\alpha(f) = \alpha(g_0)\alpha(h_0)$ is not irreducible over $\Z/p\Z$, contradicting the initial assumptions. Therefore, $f$ must be irreducible over $\Q$.
\end{proof}

\begin{eg}
    \leavevmode
    \begin{enumerate}
        \item $f = x^3+4x^2+6x+2 \in \Z[x]$ is irreducible over $\Q$. Indeed, take $p = 3$, then $f\mod 3 = x^3+x^2+2 \in \Z/3\Z[x]$, which has no roots in $\Z/3\Z[x]$, so $f\mod 3$ is irreducible over $\Z/3\Z[x]$. Applying the Modular Irreducibility Test, $f$ is irreducible over $\Q$.
    \end{enumerate}
\end{eg}

\begin{rmk}
    Irreducible over $\Q$ $\cancel{\implies}$ irreducible over $\Z/p\Z$.
    \begin{enumerate}
        \item[$\drsh$] Eg: $x^2 - 2$ is irreducible over $\Q$, but has a root in $\Z/2\Z$. 
        \item[$\drsh$] For $p=2$, $x^4 + 1$ is irreducible over $\Q$, but not over $\Z/p\Z$.
    \end{enumerate}
\end{rmk}

\begin{namthm}[Eisenstein's Criterion]
    Let $f = a_0+a_1x+...+a_nx^n \in \Z[x]$ with $n \geq 1$ and $a_n \neq 0$. Suppose there exists a prime $p$ such that \begin{enumerate}
        \item $p\;\vert\;a_i$ for all $0 \leq i < n$,
        \item $p\;\cancel{\vert}\;a_n$
        \item $p^2\;\cancel{\vert}\;a_0$
    \end{enumerate}
    Then $f$ is irreducible in $\Q[x]$
\end{namthm}
\begin{proof}
    Suppose $f \in \Z[x]$, $\deg(f) \geq 1$ and let $p \in \Z$ a prime satisfying the conditions of the theorem. If $f$ is not irreducible in $\Q[x]$, then there exists a proper factorization $f = gh$, $g,h \in \Z[x]$ by the corollary to Gauss' Lemma. Write $g = b_0+b_1x+...+b_kx^k$ and $h = c_0+c_1x+...+c_lx^l$, so we have $a_0 = b_0c_0$, $a_n = b_mc_l$, and $n = m+l$. Note, since $p$ does not divide $a_n$, $p$ does not divide $b_k$ nor $c_l$. Let $\alpha:\Z[x] \rightarrow \Z/p\Z[x]$ be the reduction modulo $p$. Then $\alpha(f) = [a_n]_px^n = \alpha(g)\alpha(h)$. Since $p$ does not divide the leading coefficients of $g$ nor $h$, this is a proper factorization. Note that since $p^2$ does not divide $a_0$, $p$ divides $b_0$ or $c_0$ but not both. Without loss of generality suppose $p$ divides $b_0$. Then, let $b_m$ be the first element of $b_0,b_1,...,b_k$ for which $p$ does not divide (this is possible as $p$ does not divide $b_k$). Then, note that $$a_m = b_mc_0 + b_{m-1}c_1 + ... + b_1c_{m-1} + b_0c_m$$ Then, since $m \leq k < n$, $p$ divides $a_m$. Moreover, by construction $p$ divides $b_{m-i}c_i$ for all $i \geq 1$. Thus, it follows that $p$ must divide $b_mc_0$, so $p$ divides $b_m$ or $p$ divides $c_0$. But, by assumption $p$ does not divide $b_m$ and $p$ does not divide $c_0$, leading to a contradiction. Therefore, $f$ must be irreducible over $\Q$.
\end{proof}



\begin{namthm}[General Eisenstein Criterion] \label{namthm:eisgen}
    Let $P$ be a prime ideal of an integral domain $R$. Let $f(x) = a_nx^n + a_{n-1}x^{n-1}+...+a_1x+a_0 \in R[x]$, where $n \geq 1$. Also, suppose $a_{n-1},a_{n-2},...,a_0 \in P$, but $a_n \notin P$ and $a_0 \notin P^2 = \{\sum_{i=1}^np_iq_i:n \geq 1,p_i,q_i \in P\}$.
\end{namthm}
\begin{proof}
    Towards a contradiction suppose $f(x) = b(x)c(x)$ for some $b(x),c(x) \in R[x]$ such that $\deg(b(x)),\deg(c(x)) \geq 1$. Then consider the reduction map $\varphi:R[x]\rightarrow (R/P)[x]$. Since $P$ is a prime ideal, $R/P$ is an integral domain. Denote a reduced element with a bar. Then we have that $\overline{f(x)} = \overline{a(x)}\overline{b(x)}$ and $\overline{f(x)} = a_nx^n$. Let $F$ denote the field of fractions for $(R/P)$, and extend the natural injection $\iota:R/P\hookrightarrow F$ to an injection $\iota':(R/P)[x]\hookrightarrow F[x]$. Then, since $F$ is a field $F[x]$ is a unique factorization domain. Hence, the factorization $\overline{a_nx^n} = \overline{\alpha x^j\beta x^i}$ is unique up to associates, i.e. units. Thus, we must have that $\overline{a(x)} = \overline{\alpha_j x^j}$ for some $j \geq 1$, $\alpha_j \in R$ and $\overline{b(x)} = \overline{\beta_i x^i}$ for $i = n-j$, $\beta_i \in R$. Thus, $\overline{\alpha_0},\overline{\beta_0} = \overline{0}$. Hence, $a_0 = \alpha_0\beta_0 \in P^2$, which contradicts the assumption that $a_0 \notin P^2$. Therefore, we conclude that $f(x)$ must be irreducible in $R[x]$.
\end{proof}




\begin{note}
    The reduction modulo $p$ of $f$, $[a_n]_px^n$, is not irreducible over $\Z/p\Z$ if $n > 1$.
\end{note}

\begin{eg}
    \leavevmode
    \begin{enumerate}
        \item $x^{1000}+3x+6$ is irreducible over $\Q$. Apply Eisenstein's Criterion for $p = 3$, where $p\;\vert\;3,6$, $p^2\;\cancel{\vert}\;6$ and $p\;\cancel{\vert}\;1$. Thus, it is irreducible over $\Q$.
        \item If $p$ is a prime, the $p$th \Emph{cyclotomic polynomial} \begin{equation}
            \Phi_p = x^{p-1}+x^{p-2}+...+x+1
        \end{equation}
        is irreducible over $\Q$.
        \begin{proof}
            Replacing $x$ by $x+1$, it suffices to show that $\Phi_p(x+1)$ is irreducible. Observe that $$(x-1)\Phi_p = (x-1)(x^{p-1}+x^{p-2}+...+x+1) = x^p - 1$$
            Replacing $x$ by $x+1$, $x\Phi_p(x+1) = (x+1)^p-1$, so by the binomial theorem $$\Phi_p(x+1) = x^{p-1} + \begin{pmatrix}p \\ 1 \end{pmatrix}x^{p-2} + ... + \begin{pmatrix}p \\ p-2 \end{pmatrix}x + p$$
            But, $p$ divides $\begin{pmatrix}p \\ k \end{pmatrix}$ for all $1 \leq k \leq p-1$, and $p^2\;\cancel{\vert}\;p$, so by Eisenstein's Criterion, $\Phi_p(x+1)$ is irreducible over $\Q$.
        \end{proof}
    \end{enumerate}
\end{eg}


\subsection{\textsection Parallels between the Integers and Polynomials over a Field}


\begin{rmk}[Parallels]
        \leavevmode
        \begin{enumerate}
                \item \begin{enumerate}
                                \item[$\Z$] Integral domain that is not a field
                                \item[${F[x]}$] Same
                \end{enumerate}
                \item \begin{enumerate}
                                \item[$\Z$] Principal ideal domain ($I = n\Z$)
                                \item[${F[x]}$] Same ($I = (d)$ for $d$ monic)
                \end{enumerate}
                \item \begin{enumerate}
                                \item[$\Z$] For $n \in \Z$, $n \neq 0, \pm 1$, $\Z/n\Z$ is an integral domain if and only if $n = \pm p$, $p$ a prime if and only if $\Z/n\Z$ is a field 
                                \item[${F[x]}$] For $P \in F[x]$, $\deg(P) \geq 0$ (i.e. $P \neq 0$ and $P \notin F[x]^{\times}$) $F[x]/(P)$ is an integral domain if and only if $P$ is irreducible over $F$ (i.e. $P = aQ$ for $Q$ monic irreducible and $a \in F^{\times}$) if and only if $F[x]/(P)$ is a field
                \end{enumerate}
                \item \begin{enumerate}
                                \item[$\Z$] Unique Factorization Domain (in terms of unique prime numbers) 
                                \item[${F[x]}$] Same (in terms of unique monic irreducible polynomials)
                \end{enumerate}
                \item \begin{enumerate}
                                \item[$\Z$] Expression of $\gcd$ as largest common factor in the UFD factorization
                                \item[${F[x]}$] Same
                \end{enumerate}
        \end{enumerate}
\end{rmk}








\section{\textsection Polynomials over a Field}

\begin{thm}
    Let $F$ be a field, $P \in F[x]$, $\deg(P) > 0$. Then the following are equivalent:\begin{enumerate}
        \item $F[x]/(P)$ is an integral domain
        \item $P$ is irreducible in $F[x]$
        \item $F[x]/(P)$ is a field
    \end{enumerate}
\end{thm}
\begin{proof}
    Suppose $F$ is a field and $P \in F[x]$, with $\deg(P) > 0$.
    
    [$1 \implies 2$] Suppose $F[x]/(P)$ is an integral domain. Then, $(P)$ is a prime ideal in $F[x]$, so in particular $P$ is a prime element of $F[x]$. Then, suppose $P = fg$ for some $f,g \in F[x]$. Since $P$ is a prime element $P$ divides $f$ or $g$. Without loss of generality suppose $P$ divides $f$ and write $f = Pq$ for some $q \in F[x]$. Then, we have that $P = Pqg$, so $P(1-qg) = 0$. But, $P \neq 0$ and $F[x]$ is an integral domain, so $1-qg = 0$. Hence, $1 = qg$, so $g \in F[x]^{\times}$. Therefore, $P$ is an irreducible element in $F[x]/(P)$.
    
    
    [$2 \implies 3$] Suppose $P$ is irreducible in $F[x]$. Let $I \subset F[x]$ be an ideal containing $(P)$. Then, since $F$ is a field, $F[x]$ is a PID so $I = (f)$ for some $f \in F[x]$. It follows that $(P) \subset (f)$, so $P \in (f)$. Thus, $P = fq$ for some $q \in F[x]$. Then either $f$ or $q$ is a unit. If $f$ is a unit then $(f) = F[x]$. On the other hand, if $q$ is a unit then $f = q^{-1}P \in (P)$, so $(f) = (P)$. Therefore, $(P)$ is a maximal ideal in $F[x]$, so $F[x]/(P)$ is a field, as claimed.
    
    
    [$3 \implies 1$] Suppose $F[x]/(P)$ is a field. Then, in particular it is an integral domain.
    
    Thus, all implications hold so the statements are equivalent.
\end{proof}

\begin{note}
        The element $x+(P) \in \F[x]/(P) = K$ is a root of $P$ treated with coefficients in $K$. Indeed, we have an injective homomorphism \begin{equation}
                \map{F\xhookrightarrow{\iota} F[x]/(P)}{a \mapsto a+(P)}
        \end{equation}
        Then we can consider $P \in F[x] \hookrightarrow K[x]$. We claim that $P$ as a polynomial in $K[x]$ has a root $\alpha = x+(P) \in K$. In particular, denote elements of $K$ by $\overline{p} \in K$ where $p \in F[x]$. Then, observe that if $P = a_nx^n+...+a_1x+a_0$ in $F[x]$, then $P = \overline{a_n}x^n+...+\overline{a_1}x+\overline{a_0} \in K[x]$. It follows that \begin{align*}
                \ev_{\alpha}(P) &= P(\alpha) \\
                &= \overline{a_n}\alpha^n+...+\overline{a_1}\alpha+\overline{a_0}\\
                &= \overline{a_nx^n + ... a_1x+a_0} \\
                &= \overline{P} = \overline{0} \in K = \F[x]/(P)
        \end{align*}
\end{note}

\begin{eg}
        $\R[x]/(x^2+1)$ is a field such that \begin{equation}
                (x+(x^2+1))^2 = x^2+(x^2+1) = -1+(x^2+1)
        \end{equation}
        because $x^2+1 \in (x^2+1)$, so letting $\alpha = x+(x^2+1)$ we have that \begin{equation}
                \alpha^2+1 = 0 \in K = \R[x]/(x^2+1)
        \end{equation}
\end{eg}


\begin{cor}
    Let $P \in F[x]$ be irreducible over $F$, and let $f_1,...,f_n \in F[x]$. If $P\;\vert\;f_1f_2...f_n$, then there is $i$ such that $P\;\vert\;f_i$.
\end{cor}
\begin{proof}
    By the previous theorem $F[x]/(P)$ is a field since $P \in F[x]$ is irreducible over $F$. Hence, $(P)$ is a maximal ideal so in particular $(P)$ is a prime ideal. Then if $fg \in (P)$, $f\in (P)$ or $g \in (P)$. We shall proceed by induction on $n$. For $n = 1$ and $n = 2$ the base case holds trivially by definition of $P$. Hence, suppose there exists $k \geq 2$ such that if $n = k$, $f_1f_2...f_k \in (P)$ implies $f_i \in (P)$ for some $i \in \{1,2,...,k\}$. Then, consider $n = k+1$, so $f_1f_2...f_kf_{k+1} \in (P)$. Since $(P)$ is a prime ideal either $f_1f_2...f_k \in (P)$ or $f_{k+1} \in (P)$. If $f_{k+1} \in (P)$ we're done. Hence, suppose $f_1f_2...f_k \in (P)$. But then, by the induction hypothesis there exists $i \in \{1,2,...,k\}$ such that $f_i \in (P)$. Therefore, by mathematical induction we conclude that if $f_1f_2...f_n \in (P)$, then there exists $i \in \{1,2,...,n\}$ such that $f_i \in (P)$ for all $n \geq 1$.
\end{proof}

\begin{rmk}
    If $P \in F[x]$, $\deg(P) = n\geq 1$, (not necessarily irreducible) then \begin{equation}
        \map{\prod\limits_{i=1}^nF \xrightarrow{\phi} F[x]/(P)}{(a_0,a_1,...,a_{n-1}) \mapsto a_0+a_1x+...+a_{n-1}x^{n-1}+(P)}
    \end{equation}
    is a group isomorphism for $(F[x]/(P),+)$.
    \begin{proof}
        By definition of addition in $F[x]/(P)$ we note that $\phi$ is a group homomorphism. First, let $(a_0,a_1,...,a_{n-1}) \in \ker(\phi)$. Then in particular $a_0+a_1x+...+a_{n-1}x^{n-1} \in (P)$ so there exists $g \in F[x]$ such that $$a_0+a_1x+...+a_{n-1}x^{n-1} = gP$$ But, since $F$ is a field we have that $\deg(gP) = \deg(g) + \deg(P) \geq n$ or $gP = 0$, and $a_0+a_1x+...+a_{n-1}x^{n-1} = 0$ or $\deg(a_0+a_1x+...+a_{n-1}x^{n-1}) = n-1$. Thus, we must have that $a_0+a_1x+...+a_{n-1}x^{n-1} = gP = 0$. Therefore, $a_0 = a_1 =...= a_{n-1} = 0$, so $\ker(\phi)$ is trivial. Hence, $\phi$ is injective. Write $P = b_0 + b_1x + ... + b_nx^n$. Then, for any $c_0+c_1x+...+c_kx^k + (P) \in F[x]/(P)$, for all $m \geq n$ we can replace $x^m$ by $x^{m-n}b_n^{-1}(-b_0-b_1x-...-b_{n-1}x^{n-1})$. Repeat this step until all powers of $x$ are less than or equal to $n-1$, so $c_0+c_1x+...+c_kx^k + (P) = c_0'+c_1'x+...+c_{n-1}'x^{n-1} + (P)$. Then, we have that $\phi(c_0',c_1',...,c_{n-1}') = c_0'+c_1'x+...+c_{n-1}'x^{n-1} + (P)$, so $\phi$ is indeed surjective. Hence, we have that $\phi$ is a group isomorphism.


        [Alternative Surjectivity] Let $Q \in F[x]$. If $\deg(Q) \leq n-1$ then $Q = b_0+b_1x+...+b_{n-1}x^{n-1} + (P) \in F[x]/(P)$ is the image of $(b_0,b_1,...,b_{n-1})$. If $\deg(Q) \geq n$, then $Q = aP+q$ by the division algorithm, with $q = 0$ or $\deg(q) < \deg(P) = n$. So, $Q+(P) = aP+q+(P) = q+(P)$ which is in the image by the first case.
    \end{proof}
\end{rmk}


\begin{cor}
    If $F$ is a finite field of order $q$ and $P$ is an irreducible polynomial over $F$ of degree $n$, then $F[x]/(P)$ is a finite field of order $q^n$.
\end{cor}


\begin{eg}
    $\Z/2\Z[x]/(x^2+x+1)$ is a field of order $2^2 = 4$. Indeed, $x^2+x+1$ has no roots in $\Z/2\Z$, and is consequently irreducible.
\end{eg}


\subsection{\textsection Field Extensions}

\begin{rmk}
        Let $R\xrightarrow{f} S$ be a ring homomorphism for a commutative rings $R,S$ and let \begin{equation}
                P = a_0+a_1x+...+a_nx^n \in R[x]
        \end{equation}
        then for all $\alpha \in R$ \begin{equation}
                f(P(\alpha)) = P'(f(\alpha))
        \end{equation}
        where $P' = f(a_o) + f(a_1)x+....+f(a_n)x^n \in S[x]$. Indeed, \begin{align*}
                f(P(\alpha)) &= f(a_0+a_1\alpha+...+a_n\alpha^n) \\
                &= f(a_0) + f(a_1)f(\alpha)+...+f(a_n)f(\alpha)^n \\
                &= P'(f(\alpha))
        \end{align*}
\end{rmk}

\begin{defn}
        A ring homomorphism $F \xhookrightarrow{\iota} F'$ for $F, F'$ fields is called a \Emph{field extension} and $F'$ is called an \Emph{extension field} of $F$. Note that $\iota$ must be injective since $F$ is a field and it is assumed to be a ring homomorphism, so one often identifies $F$ with the isomorphic subfield $\iota(F)$ to $F$ in $F'$.
\end{defn}


\begin{thm}[Kronecker's Theorem]
        If $F$ is a field and $P \in F[x]$, $\deg(P) > 0$, then there is an extension field of $F$ in which $P$ has a root.
\end{thm}
\begin{proof}
        By the Unique Factorization Theorem we have that $P = aP_1...P_n$ for a constant $a \neq 0$ in $F$ and monic irreducible polynomials $P_i$, for all $i$. Then we know that $F' = F[x]/(P_1)$ is a field since $P_1$ is irreducible. Moreover, $P$ has a root ($\alpha = x+(P_1) \in F'$) in $F'$, where we see $P$ as a polynomial with coefficients in $F'$ via the embedding \begin{equation}
                \map{F\xhookrightarrow{\iota}F'}{a\mapsto a+(P_a)}
        \end{equation}
        which is a field extension. Then, because $P_1(\alpha) = 0$ in $F'[x]$ and $P = P_1Q$ for $Q = aP_2...P_n$, we have that $P(\alpha) = 0$ in $F'[x]$. Hence, $\alpha$ is a root of $P$ in the extension field $F'$.
\end{proof}


\section{\textsection GCD of Polynomials}

\begin{defn}[GCD]
        We have seen that for $f,g \in F[x]$, $F$ a field, if $f \neq 0$ or $g \neq 0$ and $d$ is the monic generator of \begin{equation}
                (f,g) = \{Pf+Qg\vert P,Q\in F[x]\}
        \end{equation}
        Then \begin{enumerate}
                \item $d$ is monic
                \item $d\;\vert\;f$ and $d\;\vert\;g$
                \item If $P\;\vert\;f$ and $P\;\vert\;g$, then $P\;\vert\;d$ ($\forall P \in F[x]$)
        \end{enumerate}
\end{defn}

\begin{rmk}
        If $d' \in F[x]$ satisfies 1.-2.-3. above, then $d'$ is the monic generator of $(f,g)$
\end{rmk}
\begin{proof}
        Let $d = \gcd(f,g)$ be the monic generator of $(f,g)$. By 1. $(d') \supseteq (f,g) = (d)$ since $d'\;\vert\;f$ and $d'\;\vert\;g$. Thus, $d'\;\vert\;d$. By 2., since $d\;\vert\;f$ and $d\;\vert\;g$, $d\;\vert\;d'$. By the lemma below, $d = d'$
\end{proof}

\begin{lem}
        For $F$ a field, $f,g \in F[x]$, and $f,g$ monic, if $f\;\vert\;g$ and $g\;\vert\;f$, then $f = g$.
\end{lem}
\begin{proof}
        If $f\;\vert\;g$ and $g\;\vert\;f$ then $g = Qf$ and $f = Pg$ for some $P,Q \in F[x]$. Thus $f = Pg = PQf$, so $(1-PQ)f = 0$, where $f \neq 0$ and $F[x]$ is an integral domain, so $1 = PQ$. Hence, $P,Q \in F[x]^{\times} = F^{\times} = F\backslash\{0\}$. Then $g = af$ for $a \in F\backslash\{0\}$. Since $f$ is monic, the leading coefficient of $g$ is $a$. But, $g$ is monic as well, so $a = 1$ and $f = g$.
\end{proof}


\begin{claim}[Greatest Common Factor]
        Let $f,g \in F[x]$, $\deg(f),\deg(g) \geq 1$. Let $f = aP_1...P_n$, $g = bQ_1...Q_m$ be their unique factorization into a constant times a product of monic irreducible polynomials. Let $h = P_{j_1}P_{j_2}...P_{j_l}$ be the greatest common factor (set $h = 1$ if they don't have a common monic irreducible factor). Then $\gcd(f,g) = h$, as $h$ satisfies 1.-2.-3. from the definition.
\end{claim}
\begin{proof}
        By definition $h$ is monic and divides both $f$ and $g$. Now, let $P \in F[x]$ such that $P\;\vert\;f$ and $P\;\vert\;g$. Then there exists $Q,H \in F[x]$ such that $f = PQ$, $g = PH$. If $P$ is constant then $P\;\vert\;h$ automatically. If $\deg(P) \geq 1$ then by the unique factorization theorem $P = cU_1...U_t$ for some constant $c$ and irreducible monic polynomials $U_i$, for all $i \in \{1,...,t\}$. Similarly \textbf{To be continued}
\end{proof}

\begin{eg}
        Let $F = \Q$, $f = 10(x-1)(x-2)^2(x-3)^2$, and $g = \frac{1}{11}(x-1)^3(x-2)^2(x-3)$. Then $$\gcd(f,g) = (x-1)(x-2)^2(x-3)$$
\end{eg}
