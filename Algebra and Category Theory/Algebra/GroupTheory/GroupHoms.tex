%%%%%%%%%% Grp Homs %%%%%%%%%%
\chapter{\textsection\textsection Group Homomorphisms}

\section{\textsection Basic Definitions and Examples: Group Homomorphisms}

\begin{defn}[A]
    A \Emph{group homomorphism} from a group $A$ to a group $B$ is a map of sets $\phi:A\rightarrow B$ that satisfies the condition: \begin{equation}
        \forall a_1,a_2\in A,\;\;\;\phi(a_1\cdot a_2) = \phi(a_1)\cdot \phi(a_2)
    \end{equation}
    We can rewrite the condition on $\phi$ in terms of the requirement that the following diagram commute: \begin{center}
            \begin{tikzpicture}[baseline = (a).base]
            \node[scale = 1] (a) at (0,0){
                \begin{tikzcd}
                    A\times A  \ar[d, "\phi\times\phi", swap] \ar[r, "mult_A"] & A \ar[d,"\phi"] \\
                    B\times B \ar[r, "mult_B"] & B
                \end{tikzcd}
            };
            \end{tikzpicture}
        \end{center}
        We denote the set of all group homomorphisms from $A$ to $B$ by $\Hom_{\Grp}(A,B)$.
\end{defn}


\begin{defn}[B]
    Let $(G,\star)$ and $(M,\circ)$ be groups. A map $G\xrightarrow{f} M$ is a \Emph{homomorphism of groups} if \begin{equation}
        f(x\star y) = f(x)\circ f(y),\forall x,y \in G
    \end{equation} 
\end{defn}


\begin{eg}
    \leavevmode
    \begin{enumerate}
        \item $H \leq G$, then $H \xhookrightarrow{\iota} G$ the inclusion map is a \Emph{monomorphism} (an injective homomorphism)
        \item $\map{(\R,+)\rightarrow \R^{\times}}{x\mapsto 2^x}$ is a monomorphism.
        \item $\map{S_3 \hookrightarrow S_4}{\sigma \mapsto \sigma'}$, where $\sigma'(i) = \sigma(i)$ for $i \in \{1,2,3\}$ and $\sigma'(4) = 4$. In particular $$\begin{pmatrix} 1 & 2 & 3 \\ \sigma(1) & \sigma(2) & \sigma(3)  \end{pmatrix}\mapsto \begin{pmatrix} 1 & 2 & 3 & 4 \\ \sigma'(1) & \sigma'(2) & \sigma'(3) & 4  \end{pmatrix}$$
        is a monomorphism.
        \item $\map{\det:\GL_n(\R) \rightarrow \R^{\times}}{A \mapsto \det(A)}$ is an \Emph{epimorphism} (surjective homomorphism)
        \item $(G,\star)$ a group, then for $g \in G$, $$\map{\Z\xrightarrow{\phi_g}G}{x\mapsto g^n}$$ is a group homomorphism. It needs not be injective or surjective.
        \item $\map{|\cdot|:\C^{\times}\rightarrow \R^{\times}}{z\mapsto |Z|}$ is an epimorphism.
    \end{enumerate}
\end{eg}


\begin{props}
    Let $f:(G,\star)\rightarrow (H,\circ)$ be a group homomorphism. Then \begin{enumerate}
        \item $f(e_G) = e_H$
        \item for all $g \in G$, $f(g^{-1}) = (f(g))^{-1}$
        \item The composition of two homomorphisms $G \xrightarrow{f} H \xrightarrow{\phi}K$ is a homomorphism $G\xrightarrow{\phi \circ f}K$.
    \end{enumerate}
\end{props}
\begin{proof}
    (Left to the reader)
\end{proof}

\begin{rmk}
    \leavevmode
    \begin{enumerate}
        \item If $H_1 \leq G\xrightarrow{f} K$ where $f$ is a homomorphism, then the canonical map \begin{equation}
            f\rvert_{H_1}H_1\rightarrow K
        \end{equation}
        is a homomorphism called the \Emph{restriction of $f$ to $H_1$}. Moreover, $f\rvert_{H_1} = f \circ \iota_{H_1}$, where $\iota_{H_1}:H_1\hookrightarrow G$ is the inclusion homomorphism.
        \item Let $f:G\rightarrow K$ be a homomorphism. If the image of $f$, $\ran(f)$, is contained in a subgroup $H_2 \leq K$, then the associated map \begin{equation}
            f':G\rightarrow H_2
        \end{equation}
        is a homomorphism of groups.
    \end{enumerate}
\end{rmk}

\begin{eg}
    For $\det:\GL_n(\R)\rightarrow \R^{\times}$, we have $\SL_n(\R) \leq \GL_n(\R)$, so $\SL_n(\R)\xrightarrow{\det}\R^{\times}$ is a homomorphism, and \begin{equation}
        \ran\left(\det\rvert_{\SL_n(\R)}\right) = \{1\} \leq \R^{\times}
    \end{equation}
    so \begin{equation}
        \det:\SL_2(\R)\rightarrow \{1\}
    \end{equation}
    is a homomorphism.
\end{eg}


\begin{prop}
    Let $G\xrightarrow{f} K$ be a group homomorphism. Then \begin{enumerate}
        \item Let $H_1 \leq G$, then the image $f(H_1) \leq K$ is a subgroup of $K$
        \item Let $H_2 \leq K$, then $f^{-1}(H_2) \leq G$ is a subgroup of $G$, called the inverse image of $H_2$, or the pre-image of $H_2$ by $f$.
    \end{enumerate}
\end{prop}
\begin{proof}
    (Left to the reader)
\end{proof}


\begin{rmk}
        Note that the image of a cyclic subgroup $\langle g \rangle \leq G$ under a group homomorphism $f: G \rightarrow K$ is the cyclic subgroup \begin{equation}
                \langle f(g) \rangle \leq K
        \end{equation}
\end{rmk}
\begin{proof}
        (Left to the reader)
\end{proof}

\begin{cor}
        Let $G \xrightarrow{f} K$ be a group homomorphism. \begin{enumerate}
                \item The \Emph{image} $f(G) = \ran(f)$ is a subgroup of $K$
                \item The \Emph{kernel} $\ker(f) := f^{-1}(\{e_K\})$ is a subgroup of $G$
        \end{enumerate}
\end{cor}

\begin{prop}
        Let $G\xrightarrow{f} K$ be a group homomorphism. Then $f$ is injective (i.e. a \Emph{monomorphism}) if and only if $\ker(f) = \{e_G\}$.
\end{prop}
\begin{proof}
        (Left to the reader)
\end{proof}


\begin{eg}
        \leavevmode
        \begin{enumerate}
                \item $f$ a homomorphism is an isomorphism if and only if $\ker(f) = \{e_G\}$ and $\ran(f) = K$ (for $G\xrightarrow{f} K$)
                \item $H \leq G$, $H \xhookrightarrow{\iota} G$ the inclusion map is a homomorphism with $\ran(\iota) = H$ and $\ker(\iota) = \{e_G\}$
                \item The determinant map $\GL_n(\R)\xrightarrow{\det}\R^{\times}$ is a group homomorphism. Moroever, we obtain the subgroup \begin{equation}
                                \SL_n(\R) := \{A \in \GL_n(\R):\det(A) = 1\} = \ker(\det)
                        \end{equation}
                \item The map $\map{\Z\xrightarrow{\phi_g} G}{n \mapsto g^n}$ for some fixed $g \in G$ is a group homomorphism with image $\ran \phi_g = \langle g \rangle$, and $\ker(\phi_g) = \{n \in \Z:g^n = e_g\} = S_g$. Thus, $\phi_g$ is surjective if and only if $\langle g\rangle = G$ and $\phi_g$ is injective if and only if $o(g) = +\infty$.
                \item The modulus map $\C^{\times} \xrightarrow{|\cdot|}\R^{\times}$ is a group homomorphism with $\ker(|\cdot|) = \{z = \C^{\times}:|z| = 1\} = S^1$ the circle group, and $\ran(|\cdot|) = \R_{>0}$.
                \item The map $\map{\R\xrightarrow{\alpha}\C^{\times}}{\theta\mapsto \exp(2\pi i \theta)}$ is a group homomorphism with $\ker(\alpha) = \Z$ and $\ran(\alpha) = S^1$.
        \end{enumerate}
\end{eg}




\subsection{\textsection Group Isomorphisms}

\begin{eg}[motivating examples]
    \leavevmode
    \begin{enumerate}
        \item $g_1 := s_{\{1,2,3\}}$ and $g_2 := s_{\{a,b,c\}}$ are essentially the ``same" group, but their elements are not the same. indeed ``everything we do" in $g_1$ using the group operation we can do in $g_2$ by renaming $1$ as $a$, $2$ as $b$, and $3$ and $c$: order ($|g_1| = |g_2|$), subgroups, orders of elements, ``equations," etc.
        \item $\Z/2\Z = \{[0],[1]\}$, $g = \{+1,-1\} \leq (\R\backslash\{0\},\cdot) = \R^{\times}$ are the ``same" group. let us consider their \emph{cayley tables}:
        \begin{equation*}
            \begin{array}{c|cc}
                (\Z/2\Z,+) & [0] & {[1]}  \\ \hline
                {[0]} & [0] & {[1]} \\
                {[1]} & [1] & {[0]} \\
            \end{array}
            \hspace{40pt}
            \begin{array}{c|cc}
                (g, \cdot) & +1 & -1  \\ \hline
                +1 & +1 & -1 \\
                -1 & -1 & +1\\
            \end{array}
        \end{equation*}
        $[0]$ plays the role of $+1$ and $[1]$ plays the role of $-1$.
    \end{enumerate} 
\end{eg}



\begin{prop}
    $\Z/n\Z$ is isomorphic to the group of $n$th roots of unity: \begin{equation}
        \{z\in\C:z^n = 1\}
    \end{equation}
\end{prop}

\begin{defn}
    Let $(G,\star)$ and $(M,\circ)$ be groups. A bijective map $G\xrightarrow{f} M$ which is a group homomorphism is called an \Emph{isomorphism} of the groups $G$ and $M$. The groups $(G,\star)$ and $(M,\circ)$ are said to be \Emph{isomorphic}, denoted $(G,\star) \cong (M,\circ)$.
\end{defn}

\begin{eg}
    \leavevmode
    \begin{enumerate}
        \item $\map{\Z\xrightarrow{f}2\Z}{n\mapsto 2n}$ is a group isomorphism.
        \item If $G$ is a cyclic group, $G = \langle g \rangle$, and $|G| = n <+\infty$, then \begin{equation}
            \map{\Z/n\Z\xrightarrow{\phi}G}{{[m]}\mapsto g^m}
        \end{equation}
        is a group isomorphism. Recall that $g^m = g^{m'}$ if and only if $m\equiv m' \mod n$ ($o(g) = n$), so $\phi$ is well-defined and injective. By construction $\phi$ is also surjective. Therefore, $\Z/n\Z \cong G$.
        \item If $G = \langle g \rangle$ and $o(g) = +\infty$, then $G \cong \Z$. Indeed, the map \begin{equation}
            \map{\Z\xrightarrow{\phi} G}{m\mapsto g^m}
        \end{equation}
        is a group isomorphism.
        \item The map \begin{equation}
            \map{(\R,+)\xrightarrow{\exp}(\R_{>0},\cdot)}{x\mapsto e^x}
        \end{equation}
        is a group isomorphism, so $(\R,+)\cong (\R_{>0},\cdot)$. Another isomorphism is \begin{equation}
            \map{(\R,+)\xrightarrow{\phi}(\R_{>0},\cdot)}{x\mapsto 2^x}
        \end{equation}
        \item For all $h \in G$, where $(G,\star)$ is an arbitrary group, \begin{equation*}
            \map{(G,\star) \xrightarrow{\alpha_h}(G,\star)}{a \mapsto h^{-1}\star a \star h}
        \end{equation*}
        is a group isomorphism with inverse $\alpha_{h^{-1}}$.
        \item $\map{\Z\xrightarrow{\beta}\Z}{n\mapsto -n}$ is a group isomorphism.
    \end{enumerate}
\end{eg}

\begin{defn}
    An \Emph{isomorphism} $(G,\star)\rightarrow(G,\star)$ is called an \Emph{automorphism} of $(G,\star)$.
\end{defn}

\begin{prop}
    The set of automorphisms of a group $G$ is a subgroup of the symmetric group $S_G$.\begin{enumerate}
        \item The identity map $\map{\id:G\rightarrow G}{g\mapsto g}$ is an isomorphism (hence an automorphism) which acts as an identity for the group
        \item If $G\xrightarrow{\phi}H$ is an isomorphism, then the inverse $H\xrightarrow{\phi^{-1}}G$ is an isomorphism
        \item If $G\xrightarrow{\phi}H$ and $H\xrightarrow{\psi}K$ are isomorphisms, then the composition $\psi \circ \phi$ is also an isomorphism
    \end{enumerate}
\end{prop}
\begin{proof}
    (Left to the reader)
\end{proof}

\begin{cor}
    The set $\aut(G)$ of automorphisms of $G$ is a group for the composition of maps.
\end{cor}

\begin{eg}[Non-example]
    $\Q\cancel{\cong} \Q^{\times}$ because $-1 \in \Q^{\times}$ has order $2$, but $\Q$ does not have any element of order $2$. Indeed, if $x \in \Q$ such that $x+x=2x = 0$, then $x = 0$ so $o(x) = 1$. But, if $\Q^{\times}\xrightarrow{\phi}\Q$ is an isomorphism, then $o(\phi(-1)) = 2$, which is not possible.
\end{eg}

\begin{thm}[Dihedral Group Isomorphisms]
    Let $x,y \in G$ such that $G = \langle x,y \rangle (:= \langle \{x,y\}\rangle)$ with the relations $x^n = y^2 = e$, $yx = x^{-1}y$, and $n \geq 1$. Then $|G| \leq 2n$. If $|G| = 2n$, then the relation characterizes the group up to isomorphism.
\end{thm}
\begin{proof}
    If $n = 1$, $G \cong \Z/2\Z \cong D_1$. If $n = 2$, $G \cong \Z/2\Z \times \Z/2\Z \cong D_2$, mapping $x \mapsto ([1],[0])$ and $y \mapsto ([0],[1])$. Now, for all $n \geq 1$, let $Y_n := \{e,x,x^2,...,x^{n-1},y,xy,...,x^{n-1}y\}$. We know that $G = Y_n$ as sets from our study of the relations on the dihedral group. But $|Y_n| = 2n$ as a set, so $|G| \leq 2n$ as a group. If $|G| = 2n$, then all elements described in $Y_n$ are distinct in $G$, and the Caley table of the group is fixed by the relations. Thus, the group $G$ is uniquely determined up to isomorphism by the relations and $|G| = 2n$. If $n \geq 3$ we have seen that $\langle x = \phi_{2\pi/n},y = \psi_0\rangle = D_n$, $x,y$ satisfy the relations and $|D_n| = 2n$, so $G \rightarrow D_n$ is an isomorphism (taking $x$ in $G$ to $x$ in $D_n$ and $y$ in $G$ to $y$ in $D_n$).
\end{proof}

\begin{cor}
    $S_3 \cong D_3$ for $x = (1\;2\;3)$, $y = (1\;2)$.
    \begin{proof}
        (Left to the reader)
    \end{proof}
\end{cor}



\section{\textsection Automorphisms}

\begin{defn}
    Let $G$ be a group. An isomorphism from $G$ onto itself is called an \Emph{automorphism} of $G$. The group of all automorphisms on $G$ is denoted by $\aut(G)$.
\end{defn}


\begin{prop}
    Let $H$ be a normal subgroup of the group $G$. Then $G$ acts by conjugation on $H$ as automorphisms of $H$. More specifically, the action of $G$ on $H$ by conjugation is defined for each $g \in G$ by \begin{equation*}
        h\mapsto ghg^{-1}\;\;\;\text{ for each }\;h\in H
    \end{equation*}
    For each $g \in G$, conjugation by $g$ is an automorphism of $H$. The permutation representation afforded by this action is a homomorphisms of $G$ into $\aut(H)$ with kernel $C_G(H) = \{g \in G:\forall h\in H, ghg^{-1} =h\}$ (The centralizer of $H$ with respect to $G$). In particular, $G/C_G(H)$ is isomorphic to a subgroup of $\aut(H)$.
\end{prop}

\begin{rmk}
    This proposition implies that a group acts by conjugation on a normal subgroup as \emph{structure preserving permutations}, i.e. automorphisms.
\end{rmk}

\begin{cor}
    If $K$ is any subgroup of the group $G$ and $g \in G$, then $K \cong gKg^{-1}$. Conjugate elements and conjugate subgroups have the same order (as the induced map is an automorphism).
\end{cor}


\begin{cor}
    For any subgroup $H$ of a group $G$, the quotient group $N_G(H)/C_G(H)$ is isomorphic to a subgroup of $\aut(H)$. In particular, $G/Z(G)$ is isomorphic to a subgroup of $\aut(G)$.
\end{cor}
\begin{proof}
    Since $H$ is a normal subgroup of $N_G(H)$, our previous proposition implies that $N_G(H)$ acts by conjugation on $H$. Moreover, $C_G(H) \subseteq N_G(H)$, so the kernel of the permutation representation of $N_G(H)$ in $\aut(H)$ afforded by this action is $C_G(H)$. Hence by the first isomorphism theorem $N_G(H)/C_G(H)$ is isomorphic to a subgroup of $\aut(H)$.

    The second case follows from taking $H = G$, so $N_G(G) = G$ and $C_G(G) = Z(G)$.
\end{proof}


\begin{defn}
    Let $G$ be a group and let $g \in G$. Conjugation by $g$ is called an \Emph{inner automorphism} of $G$ and the subgroup of $\aut(G)$ consisting of all inner automorphisms is denoted by $\inn(G)$.
\end{defn}


\begin{defn}
    A subgroup $H$ of a group $G$ is called \Emph{characteristic in $G$} if and only if every automorphism of $G$ maps $H$ onto itself, i.e., $\sigma(H) = H$ for all $\sigma \in \aut(G)$.
\end{defn}


\begin{prop}
    Let $H$ be a subgroup of a group $G$: \begin{enumerate}
        \item If $H$ is characteristic in $G$ then $H \vartriangleleft G$,
        \item If $H$ is the unique subgroup of $G$ of a given order, then $H$ is characteristic in $G$,
        \item If $K$ is a characteristic subgroup of $H$ and $H \vartriangleleft G$, then $K\vartriangleleft G$.
    \end{enumerate}
\end{prop}
\begin{proof}
    (To be completed)
\end{proof}


\begin{cor}
    If $C$ is a cyclic group of order $n$, then every subgroup of $C$ is characteristic in $C$.
\end{cor}


\begin{prop}
    The automorphism group of the cyclic group of order $n$ is isomorphic to $(\Z/n\Z)^{\times}$, an abelian group of order $\varphi(n)$ (where $\varphi$ is the Euler-totient function).
\end{prop}
\begin{proof}
    Let $x$ be a generator of the cyclic group $\Z/n\Z$. If $\psi \in \aut(\Z/n\Z)$, then $\psi(x) = x^a$ for some $a \in \Z$, and the integer $a$ uniquely determines $\psi$. Denote this automorphism by $\psi_a$. As usual, since $|x| = n$, the integer $a$ is only defined modulo $n$. Since $\psi_a$ is an automorphism, $x$ and $x^a$ must have the same order, hence $\gcd(a,n) = 1$. Furthermore, for $a$ relatively prime to $n$, the map $x\mapsto x^a$ is an automorphism of $\Z/n\Z$. Hence we have a surjective map \begin{equation*}
        \map{\Psi:\aut(\Z/n\Z)\rightarrow (\Z/n\Z)^{\times}}{\psi_a\mapsto a\; (\mod n)}
    \end{equation*}
    The map $\Psi$ is a homomorphism because \begin{equation*}
        \psi_a\circ\psi_b(x) = \psi_a(x^b) = x^{ab} = \psi_{ab}(x)
    \end{equation*}
    for all $\psi_a,\psi_b \in \aut(\Z/n\Z)$, so that \begin{equation*}
        \Psi(\psi_a\circ\psi_b) = \Psi(\psi_{ab}) = ab\;(\mod n) = \Psi(\psi_a)\Psi(\psi_b)
    \end{equation*}
    Finally, $\Psi$ is injective by construction of the $\psi_a$, and hence is an isomorphism.
\end{proof}


\begin{eg}
    \leavevmode
    \begin{enumerate}
        \item If $p$ is an odd prime and $n \in \Z^+$, then the automorphism group of the cyclic group of order $p$ is cyclic of order $p-1$. More generally, the automorphism group of the cyclic group of order $p^n$ is cyclic of order $p^{n-1}(p-1)$.
        \item Let $p$ be a prime and let $V$ be an abelian group (written additively) with the property that $pv =0$ for all $v \in V$. If $|V| = p^n$, then $V$ is an $n$-dimensional vector space over the field $\F_p = \Z/p\Z$. The automorphisms of $V$ are precisely the non-singular linear transformations from $V$ to itself, that is \begin{equation*}
                \aut(V) \cong \GL(V) \cong \GL_n(\F_p)
        \end{equation*}
    \end{enumerate}
\end{eg}

