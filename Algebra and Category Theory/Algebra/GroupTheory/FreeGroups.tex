%%%%%%%%%% Free Grps %%%%%%%%%%
\chapter{ Free Groups}
\label{FreeGrps}
% use \chaptermark{}
% to alter or adjust the chapter heading in the running head
\section{ Basic Definitions and Examples: Free Groups}

\begin{remark}
        The idea of a free group, $F(S)$, generated by a set $S$ is that there are no relations satisfied by the elements of $S$. ($S$ is ``free of relations")
\end{remark}

\begin{definition}[Universal Property of Free Groups]
        Given any set map $\varphi$ from the set $S$ to the set underlying the group $G$, there is a unique group homomorphism \begin{equation}
                \Phi:F(S) \rightarrow G
        \end{equation}
        such that $\Phi\circ \iota = \varphi$. That is to say, the following diagram commutes: 
        \begin{center}
            \begin{tikzpicture}[baseline = (a).base]
            \node[scale = 1] (a) at (0,0){
                \begin{tikzcd}
                        S \ar[dr, "\forall\varphi", swap] \ar[r, "\iota"] & F(S) \ar[d, dashed, "\exists!\Phi"] \\
                        & \forall G
                \end{tikzcd}
            };
            \end{tikzpicture}
        \end{center}
\end{definition}

\begin{construction}
        Let $S$ be a set and let $S^{-1}$ be any set disjoint from $S$ such that there is a bijection from $S$ to $S^{-1}$. For each $s \in S$ denote its corresponding element in $S^{-1}$ by $s^{-1}$, and for each $t \in S^{-1}$ denote its corresponding element in $S$ by $t^{-1}$ (so $(s^{-1})^{-1} \in S$). 

        Take a singleton set not contained in $S \cup S^{-1}$ and call it $\{1\}$. Let $1^{-1} = 1$, and for any $x \in S\cup S^{-1}\cup\{1\}$, let $x^1 = x$.

        A \Emph{word} on $S$ is defined by a sequence \begin{equation}
                (s_1,s_2,s_3,...)
        \end{equation}
        where $s_i \in S\cup S^{-1} \cup \{1\}$ for all $i$, and there exists $N \in \N$ such that if $i \geq N$, then $s_i = 1$.

        The word $(s_1,s_2,s_3,...)$ is said to be \Emph{reduced} if \begin{enumerate}
                \item $s_{i+1} \neq s_i^{-1}$ for all $i$ with $s_i \neq 1$
                \item if $s_k = 1$ for some $k$, then $s_i = 1$ for all $i \geq k$
        \end{enumerate}
        The reduced word $(1,1,1,...)$ is called the \Emph{empty word} and is denoted by $1$. We write the reduced word $(s_1^{\varepsilon_1},...,s_n^{\varepsilon_n},1,1,...)$ with $\varepsilon_i = \pm 1$ as $s_1^{\varepsilon_1}...s_n^{\varepsilon_n}$. Note by definition reduced words \begin{equation}
                r_1^{\delta_1}...r_m^{\delta_m}\;and\;s_1^{\varepsilon_1}...s_n^{\varepsilon_n}
        \end{equation}
        are equal if and only if $n = m$ and $\delta_i = \varepsilon_i$ for all $1\leq i \leq n$.

        Let $F(S)$ be the set of reduced words on $S$ and embed $S$ into $F(S)$ by \begin{equation}
                s \mapsto (s,1,1,...)
        \end{equation}
        Note if $S = \emptyset$, $F(S) = \{1\}$.
        
        \Emph{Operation:} Let $r_1^{\delta_1}...r_m^{\delta_m}$ and $s_1^{\varepsilon_1}...s_n^{\varepsilon_n}$ be reduced words, and assume without loss of generality that $m \leq n$. Let $k$ be the smallest integer in the range $1 \leq k \leq m+1$ such that \begin{equation}
                s_k^{\varepsilon_k} \neq r_{m-k+1}^{-\delta_{m-k+1}}
        \end{equation}
        Then the product of these reduced words is defined as \begin{equation}
                (r_1^{\delta_1}...r_m^{\delta_m})(s_1^{\varepsilon_1}...s_n^{\varepsilon_n}) := \left\{\begin{array}{ll}
                        r_1^{\delta_1}...r_{m-k+1}^{\delta_{m-k+1}}s_k^{\varepsilon_k}...s_n^{\varepsilon_n}, & \text{if $k \leq m$} \\
                        s_{m+1}^{\varepsilon_{m+1}}...s_n^{\varepsilon_n}, & \text{if $k = m+1\leq n$} \\
                        1, & \text{if $k = m+1$ and $m = n$}
                \end{array}\right.
        \end{equation}
\end{construction}


\begin{theorem}
        $F(S)$ is a group under the above binary operation.
\end{theorem}
\begin{proof}
        By construction we note that $1$ is an identity element of the binary operation, and that the inverse of a reduced word $s_1^{\varepsilon_1}...s_n^{\varepsilon_n}$ is $s_n^{-\varepsilon_n}...s_1^{-\varepsilon_1}$. For each $s \in S\cup S^{-1} \cup\{1\}$ define a map $\sigma_s:F(S)\rightarrow F(S)$ by \begin{equation}
                \sigma_s(s_1^{\varepsilon_1}...s_n^{\varepsilon_n}) = \left\{\begin{array}{ll} 
                        s\cdot s_1^{\varepsilon_1}...s_n^{\varepsilon_n} & \text{if $s_1^{\varepsilon} \neq s^{-1}$} \\
                        s_2^{\varepsilon_2}...s_n^{\varepsilon_n} & \text{if $s_1^{\varepsilon_1} = s^{-1}$}
                \end{array}\right.
        \end{equation}
        Since $\sigma_{s^{-1}}\circ \sigma_s$ is the identity map on $F(S)$, $\sigma_s$ is a permutation of $F(S)$.Let $A(F)$ be the subgroup of the symmetric group on $F(S)$ generated by $\{\sigma_s:s\in S\}$. We observe that the map \begin{equation}
                s_1^{\varepsilon_1}...s_n^{\varepsilon_n} \mapsto \sigma_{s_1}^{\varepsilon_1}\circ ... \circ \sigma_{s_n}^{\varepsilon_n}
        \end{equation}
        is a set bijection between $F(S)$ and $A(F)$ which respects their binary operation. Since $A(F)$ is a group, and hence its operation is associative, so is $F(S)$.
\end{proof}

\begin{theorem}[Universal Property of Free Groups]
        Let $G$ be a group, $S$ a set, and $S\xrightarrow{\varphi}G$ a set map. Then there exists a unique group homomorphism $\Phi:F(S)\rightarrow G$ such that the diagram commutes 
        \begin{center}
            \begin{tikzpicture}[baseline = (a).base]
            \node[scale = 1] (a) at (0,0){
                \begin{tikzcd}
                        S \ar[dr, "\varphi", swap] \ar[r, "\iota"] & F(S) \ar[d, dashed, "\exists!\Phi"] \\
                        & G
                \end{tikzcd}
            };
            \end{tikzpicture}
        \end{center}
\end{theorem}
\begin{proof}
        Choose $\Phi:s_1^{\varepsilon_1}...s_n^{\varepsilon_n}\mapsto \varphi(s_1)^{\varepsilon_1}...\varphi(s_n)^{\varepsilon_n}$
\end{proof}


\begin{corollary}
        The free group $F(S)$ is unique up to an isomorphism which is an identity on the set $S$.
\end{corollary}
\begin{proof}
        Suppose $F(S)$ and $F'(S)$ are two free groups generated by $S$. Since $S$ is contained in both $F(S)$ and $F'(S)$ we have natural injections \begin{equation}
                S\xhookrightarrow{\iota}F(S),S\xhookrightarrow{\iota'}F'(S)
        \end{equation}
        By the universal property of Free groups there exist unique group homomorphisms $\Phi:F(S)\rightarrow F'(S)$ and $\Phi':F'(S)\rightarrow F(S)$ such that $\Phi \circ \iota = \iota'$ and $\Phi'\circ \iota' = \iota$, which are both the identity on $S$. Then, $\Phi'\circ \Phi$ is a map which makes the diagram \begin{center}
            \begin{tikzpicture}[baseline = (a).base]
            \node[scale = 1] (a) at (0,0){
                \begin{tikzcd}
                        S \ar[dr, "\iota", swap] \ar[r, "\iota"] & F(S) \ar[d, dashed, "?"] \\
                        & F(S)
                \end{tikzcd}
            };
            \end{tikzpicture}
        \end{center}
        commute. But, $\id_{F(S)}$ also makes this commute, so by uniqueness $\Phi'\circ \Phi = \id_{F(S)}$. Similarly, $\Phi\circ \Phi' = \id_{F'(S)}$, so $\Phi$ and $\Phi'$ are inverses, and hence bijections. Thus, $\Phi$ and $\Phi'$ are isomorphisms which are the identity on $S$, so $F(S)\cong F'(S)$ as claimed.
\end{proof}

\begin{definition}[Free Group]
        The group $F(S)$ is called the \Emph{free group} on the set $S$. A group $F$ is a free group if there is some set $S$ such that $F \cong F(S)$. In this case we call $S$ a set of \Emph{free generators} or a \Emph{free basis} of $F$. The cardinality of $S$ is called the \Emph{rank} of the free group $F$.
\end{definition}

\begin{theorem}[Schreier]
        Subgroups of a free group are themselves free.
\end{theorem}





        



\section{ Presentations}


\begin{remark}
        For a group $G$, $G$ is a homomorphic image of a free group. Take $S = G$ and $\varphi$ as the identity map from $G$ to $G$. Then by the universal property of free groups there is a surjective group homomorphism from $F(G)$ onto $G$.
        \begin{enumerate}
                \item[$\rightarrow$] In general, if $S \subseteq G$ such that $G = \langle S \rangle$, then there exists a unique group epimorphism $\varphi:F(S)\twoheadrightarrow G$ which is the identity on $S$.
        \end{enumerate}
\end{remark}

\begin{definition}
        Let $S$ be a subset of a group $G$ such that $G = \langle S \rangle$. \begin{enumerate}
                \item A \Emph{presentation} for $G$ is a pair $(S,R)$ where $R$ is a set of words in $F(S)$ such that the \Emph{normal closure} of $\langle R \rangle$ in $F(S)$ (the smallest normal subgroup containing $\langle R\rangle$) equals the kernel of the homomorphism $\pi:F(S)\rightarrow G$, where $\pi$ extends the identity map from $S$ to $S$. The elements of $S$ are called \Emph{generators} and those of $R$ are called \Emph{relations} of $G$.
                \item We say $G$ is \Emph{finitely generated} if there is a presentation $(S,R)$ such that $S$ is a finite set, and we say $G$ is \Emph{finitely presented} if there is a presentation $(S,R)$ were both $S$ and $R$ are finite sets.
        \end{enumerate}
\end{definition}

