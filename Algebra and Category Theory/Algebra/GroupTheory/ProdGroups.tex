%%%%%%%%%% Prod Grps %%%%%%%%%%
\chapter{\textsection\textsection Product Groups}


\section{\textsection Basic Definitions and Examples: Product Groups}

\begin{rec}
        For $G$, $G'$ group, recall that $G \times G'$ with the binary operation $(a,a') \star (b,b') := (a\star_G b, a' \star_{G'} b')$ is a group with identity $(e_G, e_{G'})$ and inverse $(a,a')^{-1} = (a^{-1},{a'}^{-1})$.
\end{rec}


\begin{defn}[Direct Product]
        The group $G \times G'$ is called the \Emph{direct product} of $G$ and $G'$.
\end{defn}

\begin{rmk}
        We have four homomorphisms which characterize the direct product of $G$ and $G'$:
        \begin{center}
            \begin{tikzpicture}[baseline = (a).base]
            \node[scale = 1] (a) at (0,0){
                \begin{tikzcd}
                    G \ar[dr, hookrightarrow, "\iota_G", swap] & & G \\
                        &G \times G' \ar[ur, twoheadrightarrow, "p_G", swap] \ar[dr, twoheadrightarrow, "p_{G'}"] & \\
                        G' \ar[ur, hookrightarrow, "\iota_{G'}"] & & G'
                \end{tikzcd}
            };
            \end{tikzpicture}
        \end{center}
        defined by \begin{align}
                \iota_G(g) = (g,e_{G'}),&\;\iota_{G'}(g') = (e_G,g') \\
                p_G(g,g') = g,&\;p_{G'}(g,g') = g'
        \end{align}
        for all $g \in G$ and $g' \in G'$. Moreover, this map satisfy the following properties: \begin{enumerate}
                \item $\iota_G$ and $\iota_{G'}$ are \Emph{monomorphisms} of homomorphic image $G \times \{e_{G'}\} \leq G\times G'$ and $\{e_G\}\times G' \leq G\times G'$, respectively.
                \item $p_G$ and $p_{G'}$ are \Emph{epimorphisms} called \Emph{projections}, with kernels $\ker(p_G) = \{e_G\}\times G'$ and $\ker(p_{G'}) = G\times \{e_{G'}\}$. Hence, we have that normal subgroups \begin{equation}
                                \{e_G\}\times G' \vartriangleleft G \times G' \vartriangleright G\times \{e_{G'}\}
                \end{equation}
        \end{enumerate}
\end{rmk}

\begin{namthm}[Universal Property of Product Groups]
        Let $G$ and $G'$ be groups. Then the direct product of $G$ and $G'$ is defined uniquely up to isomorphism by the triple \begin{equation}
                (H,\rho_G:H\rightarrow G, \rho_{G'}:H\rightarrow G')
        \end{equation}
        that satisfies the universal property \begin{center}
            \begin{tikzpicture}[baseline = (a).base]
            \node[scale = 1] (a) at (0,0){
                \begin{tikzcd}
                        & \forall K \ar[ddl, bend right, "\forall \sigma_G", swap] \ar[ddr, bend left, "\forall \sigma_{G'}"] \ar[d, dashed, "\exists! \varphi"] & \\
                        & H \ar[dl, twoheadrightarrow, "\rho_G"] \ar[dr, twoheadrightarrow, "\rho_{G'}", swap] & \\
                        G & & G' 
                \end{tikzcd}
            };
            \end{tikzpicture}
        \end{center}
\end{namthm}
\begin{proof}
        (Left to the reader)
\end{proof}


\begin{rmk}
        Given a group $K$, it is desireable to decompose $K$ as a product $K \cong H \times H'$ for $H,H' \leq K$ proper subgroups. Indeed, $H$ and $H'$ are simpler groups, and it is easy to relate properties of $K$ to properties of $H$ and $H'$.
\end{rmk}


\begin{note}
        A group cannot necessarily be written in this way for non-trivial $H$ and $H'$.
\end{note}


\begin{eg}
        Observe that $\Z/6\Z \cong \Z/2\Z \times \Z/3\Z$ as $([1]_2,[1]_3)$ is an element of order $6$ and $|\Z/2\Z\times \Z/3\Z| = 6$.
\end{eg}

\begin{prop}[Cyclic Group Decomposition]
        Let $m,n \in \Z$, for $m,n \geq 1$. Then we have that \begin{equation}
                \Z/mn\Z \cong \Z/m\Z\times \Z/n\Z
        \end{equation}
        if and only if $m$ and $n$ are relatively prime.
\end{prop}
\begin{proof}
        (Left to the reader)
\end{proof}

\begin{eg}[Non-example]
        Observe that $\Z/4\Z \cancel{\cong} \Z/2\Z \times \Z/2\Z$. Indeed, every element of $\Z/2\Z \times \Z/2\Z$ has order $1$ or $2$, whereas $\Z/4\Z$ is generated by an element of order $4$.
\end{eg}


\begin{prop}
        Let $H,H' \leq K$ and let \begin{equation}
                \map{f:H\times H' \rightarrow K}{(h,h') \mapsto hh'}
        \end{equation}
        be the multiplication map (not a homomorphism in general). Then the image of $f$ is \begin{equation}
                HH' = \{hh':h\in H, h' \in H'\}
        \end{equation}
        We then have that \begin{enumerate}
                \item \begin{enumerate}
                                \item $f$ is injective if and only if $H \cap H' = \{e_K\}$
                                \item $f$ is surjective if and only if $K = HH'$
                        \end{enumerate}
                \item $f$ is a group homomorphism from the direct product group $H \times H'$ to $K$ if and only if $hh' = h'h$ for all $h \in H$ and all $h' \in H'$
                \item $f$ is a group isomorphism if and only if $H \cap H' = \{e_K\}$, $HH' = K$, and $H,H' \vartriangleleft K$.
        \end{enumerate}
\end{prop}
\begin{proof}
        [1. a)] First, let $x \in H \cap H'$ for $x \neq e_K$. Then we have that $f(x,e_K) = x = f(e_K, x)$, where $(x,e_K) \neq (e_K, x)$ by assumption, so $f$ is not injective. Conversely, suppose $H \cap H' = \{e_K\}$ and $(a,b) \in \ker(f)$ so $f(a,b) = e_K$. Then we have that $H \ni a = b^{-1} \in H'$ since it is a subgroup, so $a, b^{-1} \in H \cap H'$. In particular, $a, b^{-1} = e_K$, so $b = e_K$ as well. Thus, $\ker(f) = \{(e_K,e_K)\}$, which implies $f$ is injective.

        [1. b)] Note that $f$ is surjective if and only if $K = HH'$ by definition of $f$. 

        [2.] Let $h_1,h_2 \in H$, $h_1',h_2' \in H'$. Then $f$ is a homomorphism if and only if $$h_1h_1'h_2h_2' = f(h_1,h_1')f(h_2,h_2') = f(h_1h_2,h_1'h_2') = h_1h_2h_1'h_2'$$
        which holds if and only if $h_1'h_2 = h_2h_1'$. But, this is true for all $h_2 \in H$ and all $h_1' \in H'$, so the if and only if statement is true.

        [3.] Note $f$ is injective if and only if $H \cap H' = \{e_K\}$ by 1.a), and $f$ is surjective if and only if $HH' = K$, by 1.b). First, suppose $f$ is an isomorphism. Note that from the four fundamental makes of the group direct product we know that $H \times \{e_K\},\{e_K\}\times H' \vartriangleleft H \times H'$. Thus, since $f$ is assumed to be surjective and $f(H\times \{e_K\}) = H$, we have that $H \vartriangleleft K$, and similarly $H' \vartriangleleft K$. Conversely, suppose $H, H' \vartriangleleft K$. Then, let $h \in H$ and $h' \in H'$ and consider the commutator $[h,h'] = hh'h^{-1}{h'}^{-1}$. Since $H$ and $H'$ are normal we have that $H \ni h(h'h^{-1}{h'}^{-1}) = (hh'h^{-1}){h'}^{-1} \in H$. Hence, $[h,h'] \in H \cap H'$, but $H \cap H' = \{e_K\}$ so $[h,h'] = e_K$. Therefore, we have that $hh' = h'h$, so by 2. $f$ is a homomorphism, and since $f$ is shown to be injective and surjective, it is an isomorphism.
\end{proof}

\begin{prop}
        Let $H,H' \leq K$. If $H$ (or $H'$) is a normal subgroup of $K$, then $HH'$ is a subgroup of $K$.
\end{prop}
\begin{proof}
        (Left to the reader)
\end{proof}

\begin{rmk}
        Note that the multiplication map can be bijective without being a homomorphism. For example, if we take $H = \langle x \rangle, H' = \langle y \rangle \in D_3$, and $H \cap H' = \{1\}$, $D_3 = HH'$, but $D_3 \cancel{\cong}\langle x \rangle \times \langle y \rangle$ because $\langle y \rangle $ is not a normal subgroup.
\end{rmk}

\begin{cor}
        Let $G$ be a finite group with $H,H' \leq G$. \begin{enumerate}
                \item If $H \cap H' = \{e_G\}$, then $|H||H'| = |HH'|$
                \item If $H \cap H' = \{e_G\}$, $H,H' \vartriangleleft G$, and $|G| = |H||H'|$, then \begin{equation}
                                G \cong H \times H'
                \end{equation}
        \end{enumerate}
\end{cor}
\begin{proof}
        [1.] Suppose $H \cap H' = \{e_G\}$. Then by 1.a) of the previous proposition the multiplication map $f:H\times H' \rightarrow G$ is injective. Moreover, its image is precisely $HH' \subseteq G$. Thus, the corestriction $f:H\times H' \rightarrow HH'$ is a bijection. THerefore \begin{equation}
                |H||H'| = |H\times H'| = |HH'|
        \end{equation}

        [2.] Suppose $H \cap H' = \{e_G\}$, $H,H' \vartriangleleft G$, and $|G| = |H||H'|$. Since $H \cap H' = \{e_G\}$ we have by 1. that $|H||H'| = |HH'|$, so $|G| = |HH'|$ which implies $G = HH'$ as $HH' \subseteq G$. Thus, by 3. of the previous proposition $G \cong H \times H'$.
\end{proof}


\begin{rmk}[Application]
        Suppose $G$ is abelian and $|G| = p^2$ for a prime $p$. Then either $G$ is cyclic or \begin{equation}
                G \cong \Z/p\Z\times \Z/p\Z
        \end{equation}
\end{rmk}
\begin{proof}
        Assume $G$ is not cyclic. Then for all $g \in G$ with $g \neq e_G$ we have $o(g) = p$ by \ref{thmname:lagrange}. Take $g,g' \in G$ such that $o(g) = o(g') = p$ and $g' \notin \langle g\rangle$, which is possible since there are $p$ elements not in $\langle g \rangle$ of order $p$. Let $H = \langle g \rangle$ and $H' = \langle g' \rangle$. Since $G$ is abelian $H$ and $H'$ are normal subgroups. Moreover, $H \cap H' = \{e_G\}$. Indeed, $H\cap H'$ is a subgroup of $H$ and $H'$, so $|H\cap H'| \in \{1,p\}$. But, if $|H\cap H'| = p$ then $H = H \cap H' = H'$, which implies that $g' \in H$, contradicting our initial assumption. Thus $|H \cap H'| = 1$ so $H\cap H' = \{e_G\}$. Finally, $|G| = p^2 = |H||H'|$. Thus, by 2. of the previous corollary we conclude that \begin{equation}
                G \cong H \times H' \cong \Z/p\Z \times \Z/p\Z
        \end{equation}
\end{proof}

\section{\textsection Semi-Direct Products}

\begin{rec}
    Let $G$ be a group and $H,K$ subgroups such that $H \vartriangleleft G$. If additionally $H\cap K = \{1\}$, then $HK$ is a subgroup of $G$ and every element of $HK$ can be written uniquely as a product $hk$ for some $h \in H$ and $k \in K$.
\end{rec}

Observe that if $H\vartriangleleft G$, $K \leq G$, then for any two elements $h_1k_1,h_2k_2 \in HK$, \begin{align*}
    (h_1k_1)(h_2k_2) &= h_1k_1h_2(k_1^{-1}k_1)k_2 \\
    &= h_1(k_1h_2k_1^{-1})k_1k_2 \\
    &= h_3k_3
\end{align*}
where $h_3 = h_1(k_1h_2k_1^{-1})$ and $k_3 = k_1k_2$. Note since $H$ is normal in $G$, the group $K$ acts on $H$ by conjugation: \begin{equation*}
    k\cdot h = khk^{-1} \in H,\;\;\text{for all } h\in H,k\in K
\end{equation*}
These observations inspire our following construction of a group given two groups $H$ and $K$ and a homomorphism $\phi:K\rightarrow \aut(H)$.

\begin{thm}
    Let $H$ and $K$ be groups and let $\phi$ be a group homomorphism from $K$ into $\aut(H)$. Let $\cdot$ denote the (left) action of $K$ on $H$ determined by $\phi$. Let $G$ be the set of ordered pairs $(h,k)$ with $h \in H$ and $k \in K$, and define the following multiplication on $G$:\begin{equation*}
        (h_1,k_1)\star(h_2,k_2) := (h_1k_1\cdot h_2,k_1k_2)
    \end{equation*}
    for all $(h_1,k_1),(h_2,k_2) \in G$. Then \begin{enumerate}
        \item this multiplication makes $G$ into a group of order $|G| = |H||K|$
        \item the sets $\{(h,1)\vert h \in H\}$ and $\{(1,k)\vert k \in K\}$ are subgroups of $G$ and the maps $h\mapsto (h,1)$ for $h \in H$ and $k \mapsto (1,k)$ for $k \in K$ are isomorphisms of these subgroups with the groups $H$ and $K$ respectively: \begin{equation*}
                H\cong \{(h,1)\vert h\in H\}\;\;\text{ and }\;\;K\cong \{(1,k)\vert k\in K\}
        \end{equation*}
    \end{enumerate}
            Identifying $H$ and $K$ with their isomorphic copies in $G$ we have \begin{enumerate}
                \item $H\vartriangleleft G$
                \item $H\cap K = 1$
                \item for all $h \in H$ and $k \in K$, $khk^{-1} = k\cdot h = \phi(k)(h)$
            \end{enumerate}
\end{thm}
\begin{proof}
    First, observe that that $(1_H,1_K) \in G$ acts as the identity. Indeed given any $(h,k) \in G$, \begin{equation*}
        (1_H,1_K)(h,k) = (1_H1_K\cdot h,1_Kk) = (h,k)
    \end{equation*}
    and \begin{equation*}
        (h,k)(1_H,1_K) = (hk\cdot 1_H, k1_K) = (h,k)
    \end{equation*}
    because $k\cdot1_H$ is the action of an automorphism of $H$, and hence must send the identity to itself. Let $(h_1,k_1) \in G$. Then I claim that $(h_1,k_1)^{-1} = (k_1^{-1}\cdot h_1^{-1},k_1^{-1}) \in G$. Observe that \begin{align*}
        (h_1,k_1)(k_1^{-1}\cdot h_1^{-1},k_1^{-1}) &= (h_1k_1\cdot(k_1^{-1}\cdot h_1^{-1}),k_1k_1^{-1}) \\
        &= (h_1(k_1k_1^{-1})\cdot h_1^{-1}, 1_K) \\
        &= (h_11_K\cdot h_1^{-1},1_K) \\
        &= (h_1h_1^{-1},1_K) \\
        &= (1_H,1_K)
    \end{align*}
    and \begin{align*}
        (k_1^{-1}\cdot h_1^{-1},k_1^{-1})(h_1,k_1) &= (k_1^{-1}\cdot h_1^{-1}k_1^{-1}\cdot h_1,k_1^{-1}k_1) \\
        &= (k_1^{-1}\cdot (h_1^{-1}h_1),1_K) \tag{since the action $k_1^{-1}\rightarrow\aut(H)$} \\
        &= (k_1^{-1}\cdot 1_H,1_K) \\
        &= (1_H,1_K)
    \end{align*}
    as claimed, so $G$ has inverses. Finally, for $(h_1,k_1),(h_2,k_2),(h_3,k_3) \in G$ we have \begin{align*}
        (h_1,k_1)[(h_2,k_2)(h_3,k_3)] &= (h_1,k_1)(h_2k_2\cdot h_3,k_2k_3) \\
        &= (h_1k_1\cdot(h_2k_2\cdot h_3), k_1(k_2k_3)) \\
        &= (h_1(k_1\cdot h_2)(k_1\cdot(k_2\cdot h_3)), (k_1k_2)k_3) \\
        &= (h_1(k_1\cdot h_2)(k_1k_2\cdot h_3),(k_1k_2)k_3) \\
        &= (h_1k_1\cdot h_2, k_1k_2)(h_3, k_3) \\
        &= [(h_1,k_1)(h_2,k_2)](h_3,k_3)
    \end{align*}
    so multiplication in $G$ is associative. Therefore $G$ is indeed a group. For each $(h,k), (h',k') \in G$, $(h,k) = (h',k')$ if and only if $h=h'$ and $k=k'$, so $|G| = |H||K|$, finishing the proof of the first claim.

    Let $~H := \{(h,1_K)\vert h\in H\}$ and $~K := \{(1_H,k)\vert k\in K\}$. Then note $(a,1_K)(b,1_K) = (a1_K\cdot b,1_K) = (ab,1_K)$, and $(1_H,x)(1_H,y) = (1_Hx\cdot 1_H,xy) = (1_H,xy)$, so $~H$ and $~K$ are indeed subgroups of $G$ (as $H$ and $K$ are groups in their own right). Moreover, it follows that the maps defined in the second bullet connotate isomorphisms $H\xrightarrow{\sim}~H$ and $K\xrightarrow{\sim}~K$. Now, observe that for all $(h,1_K) \in ~H$ and all $(h_1,k_1) \in G$, we have \begin{equation*}
        (h_1,k_1)(h,1_K)(k_1^{-1}\cdot h_1^{-1},k_1^{-1}) = (h_1k_1\cdot h,k_1)(k_1^{-1}\cdot h_1^{-1},k_1^{-1}) = (h_1(k_1\cdot h)(k_1\cdot k_1^{-1}\cdot h_1^{-1}),k_1k_1^{-1}) = (h_1(k_1\cdot h)h_1^{-1}, 1_K) \in ~H
    \end{equation*}
    so $~H\vartriangleleft G$. Next, the fourth bullet follows immediately from the definitions of $~H$ and $~K$, so $~H\cap ~K = \{(1_H,1_K)\}$.

    Finally, let $(h,1_K) \in ~H$ and $(1_H,k) \in ~K$. Then \begin{equation*}
        (1_H,k)(h,1_K)(1_H,k^{-1}) = (k\cdot h, k)(1_H,k^{-1}) = (k\cdot h, 1_K)
    \end{equation*}
    so identifying with $H$ and $K$ by the isomorphisms previously, we find $khk^{-1} = k\cdot h = \phi(k)(h)$. Moreover, under this identitification $K \leq N_G(H)$ since the conjugation acts as an automorphism on $H$. Since $G = HK$ and $H\leq N_G(H)$, we have $N_G(H) = G$, i.e., which again proves $H\vartriangleleft G$ under our identification.
\end{proof}


\begin{defn}
    Let $H$ and $K$ be groups and let $\varphi:K\rightarrow \aut(H)$ be a group homomorphism. The group described in the previous theorem is called the \Emph{semidirect product} of $H$ and $K$ with respect to $\varphi$, and will be denoted by $H\rtimes_{\varphi}K$ (signifying that $K$ is the group doing the action, and $H$ is the normal ``factor").
\end{defn}

We can now formalize direct products as special cases of semidirect products:

\begin{prop}
    Let $H$ and $K$ be groups and let $\varphi:K\rightarrow \aut(H)$ be a group homomorphism. Then the following are equivalent: \begin{enumerate}
        \item The identity (set) map between $H\rtimes_{\varphi}K$ and $H\times K$ is a group homomorphism (hence an isomorphism since the underlying set map is a bijection)
        \item $\varphi$ is the trivial homomorphism from $K$ into $\aut(H)$
        \item $K\vartriangleleft H\rtimes_{\varphi}K$
    \end{enumerate}
\end{prop}
\begin{proof}
    ($1.\implies 2.$) By definition of the group operation in $H\rtimes_{\varphi}K$ \begin{equation*}
        (h_1,k_1)(h_2,k_2) = (h_1k_1\cdot h_2,k_1k_2)
    \end{equation*}
    for all $h_1,h_2 \in H$ and $k_1,k_2 \in K$. By assumption $1.$, we need $(h_1,k_1)(h_2,k_2)=(h_1h_2,k_1k_2)$, which is to say $\varphi(k_1)(h_2) = h_2$ for all $h_2 \in H$. In particular $\varphi(k_1) = \id_H$ for all $k_1 \in K$, so $\varphi(K) = \{\id_H\}$.

    ($2.\implies 3.$) If $\varphi$ is trivial, then the action of $K$ on $H$ is trivial, so that the elements of $H$ commute with those of $K$ by bullet $5.$ of our previous theorem. In particular, $H$ normalizes $K$ and $K$ normalizes itself, so as $G = HK$, $G$ normalizes $K$, proving $3.$.
    

    ($3.\implies 1.$) If $K$ is normal in $H \rtimes_{\varphi} K$ then for all $h \in H$ and $k \in K$, $[h,k] \in H\cap K = \{1\}$. Thus $hk = kh$ and the action of $K$ on $H$ is trivial. The multiplication in the semidirect products is then the same as that in the direct product: \begin{equation*}
        (h_1,k_1)(h_2,k_2) = (h_1h_2,k_1k_2)
    \end{equation*}
    for all $h_1,h_2 \in H$ and $k_1,k_2 \in K$. This gives $1.$ and completes the proof.
\end{proof}


\begin{eg}
    In all examples to follow let $H$ and $K$ be groups and $\varphi$ a homomorphism from $K$ into $\aut(H)$ with associated action of $K$ on $H$ denoted by $\cdot$. Let $G = H\rtimes_{\varphi}K$ and as in our previous work we identify $H$ and $K$ as subgroups of $G$. 
    \begin{enumerate}
        \item Let $H$ be any abelian group and let $K = \langle x \rangle \cong \Z/2\Z$ be the group of order $2$. Define $\varphi:K\rightarrow \aut(H)$ by mapping $x$ to the automorphism of inversion on $H$ so that the associated action is $x\cdot h = h^{-1}$, for all $h \in H$. Then $G$ contains the subgroup $H$ of index $2$ and \begin{equation*}
                xhx^{-1} = h^{-1}\;\text{ for all } h \in H
        \end{equation*}
            When $H$ is cyclic, we have the following special cases: if $H = \Z/n\Z$, one recognizes $G$ as $D_{2n}$, and if $H = \Z$ we denote $G$ by $D_{\infty}$.
        \item For $H$ any group let $K = \aut(H)$ with $\varphi$ the identity map from $K$ to $\aut(H)$. The semidirect product $H\rtimes_{\varphi}\aut(H)$ is called the \Emph{holomorph} of $H$ and will be denoted by $\text{Hol}(H)$. Some holomorphs are described below: \begin{enumerate}
                \item $\text{Hol}(\Z/2\Z\times\Z/2\Z)\cong S_4$
                \item If $|G| = n$ and $\pi:G\rightarrow S_n$ is the left regular representation, then $N_{S_n}(\pi(G)) \cong \text{Hol}(G)$. In particular, since the left regular representation of a generator of $\Z/n\Z$ is an $n$-cycle in $S_n$ we obtain that for any $n$-cycle $(1\;2\;...\;n)$: \begin{equation*}
                        N_{S_n}(\langle(1\;2\;...\;n)\rangle) \cong \text{Hol}(\Z/n\Z)=\Z/n\Z\rtimes \aut(\Z/n\Z)
                \end{equation*}
                with the latter group having order $n\varphi(n)$, for $\varphi$ the Euler-toutient function.
        \end{enumerate}
    \end{enumerate}
\end{eg}


\begin{thm}
    Suppose $G$ is a group with subgroups $H$ and $K$ such that \begin{enumerate}
        \item $H \vartriangleleft G$, and 
        \item $H\cap K = \{1\}$
    \end{enumerate}
    Let $\varphi:K\rightarrow \aut(H)$ be the homomorphism defined by mapping $k \in K$ to the automorphism of left conjugation by $k$ on $H$. Then $HK \cong H\rtimes_{\varphi}K$. In particular, if $G = HK$ with $H$ and $K$ satisfying $1.$ and $2.$, then $G$ is the semidirect product of $H$ and $K$.
\end{thm}
\begin{proof}
    Note that since $H\vartriangleleft G$, $HK$ is a subgroup of $G$. Every element of $HK$ can be written uniquely in the form $hk$, for some $h \in H$ and $k \in K$ by properties $1.$ and $2.$. Thus the map $hk\mapsto (h,k)$ is a set bijection from $HK$ onto $H\rtimes_{\varphi}K$. The fact that this map is a homomorphism is given by $$hkh'k' = (h(kh'k^{-1}))kk' \mapsto (hk\cdot h',kk') = (h,k)(h',k')$$
\end{proof}

\begin{defn}
    Let $H$ be a subgroup of the group $G$. A subgroup $K$ of $G$ is called a \Emph{complement} for $H$ in $G$ if $G = HK$ and $H\cap K = \{1\}$.
\end{defn}

\subsection{Classifications of Certain Finite Groups}

We shall apply our results on Semi-Direct product groups to classify certain finite groups. Are argument shall follow the following structure: \begin{enumerate}
    \item show every group of order $n$ has proper subgroups $H$ and $K$ satisfying $H \triangleleft G,K\leq G$, $H\cap K = \{1\}$ with $G = HK$.
    \item find all possible isomorphism types for $H$ and $K$
    \item for each pair $H,K$ found, find all possible homomorphisms $\varphi:K\rightarrow \aut(H)$
    \item for each triple $H,K,\varphi$ found form the semidirect product $H \rtimes_{\varphi}K$ and among all these semidirect products determine which pairs are isomorphic. This results in a list of the distinct isomorphism types of groups of order $n$.
\end{enumerate}
Since $H$ and $K$ are proper subgroups of $G$ one should think of the determination of $H$ and $K$ as being achieved inductively.

\begin{eg}[Groups of order $pq$, $p$ and $q$ primes with $p < q$]
    Let $G$ be any group of order $pq$, let $P \in Syl_p(G)$ and let $Q \in Syl_q(G)$. Note that in the Sylow section we have shown $Q \triangleleft G$ and $P \leq G$ with $P\cap Q = \{1\}$, so $G \cong Q \rtimes_{\varphi}P$ for some $\varphi:P\rightarrow \aut(Q)$. Since $P$ and $Q$ are of prime order, they are cyclic. The group $\aut(Q)$ is cyclic of order $q-1$. If $p$ does not divide $q-1$, the only homomorphism from $P$ to $\aut(Q)$ is the trivial homomorphism, hence the only semidirect product in this case is the direct product, i.e., $G$ is cyclic.

    Consider now the case when $p \vert q- 1$, and let $P = \langle y \rangle$. Since $\aut(Q)$ is cyclic it contains a unique subgroup of order $p$, say $\langle \gamma \rangle$, and any homomorphism $\varphi:P\rightarrow \aut(Q)$ must map $y$ to a power of $\gamma$. There are therefore $p$ homomorphisms $\varphi_i:P\rightarrow \aut(Q)$ given by $\varphi_i(y) = \gamma^i, 0 \leq i \leq p -1$. Since $\varphi_0$ is the trivial homomorphism, $Q \rtimes_{\varphi_0}P \cong Q\times P$ as before. Each $\varphi_i$ for $i\neq 0$ gives rise to a non-abelian group, $G_i$, of order $pq$. These groups are all isomorphic because for each $\varphi_i, i > 0$, there is some generator $y_i$ of $P$ such that $\varphi_i(y_i) = \gamma$. Thus, up to a choice for the generator of $P$, these semidirect products are all the same.
\end{eg}

\begin{eg}[Groups of order $30$]
    From the examples following Sylow's Theorem, every group $G$ of order $30$ contains a subgroup $H$ of order $15$. By the preceding example $H$ is cyclic and $H$ is normal in $G$ (index $2$). By Sylow's Theorem there is a subgroup $K$ of $G$ of order $2$. Thus $G = HK$ and $H\cap K = \{1\}$ so $G \cong H\rtimes_{\varphi}K$, for some $\varphi:K\rightarrow \aut(H)$. Then \begin{equation*}
        \aut(\Z/15\Z) \cong (\Z/15\Z)^{\times} \cong \Z/4\Z\times \Z/2\Z
    \end{equation*}
    where the latter isomorphism follows from writing $H$ as $\langle a \rangle \times \langle b \rangle \cong \Z/5\Z\times \Z/3\Z$, and since these subgroups are characteristic in $H$ we have \begin{equation*}
        \aut(H) \cong \aut(\Z/5\Z)\times \aut(\Z/3\Z)
    \end{equation*}
    In particular, $\aut(H)$ contains precisely three elements of order $2$, whose actions on the group $\langle a \rangle \times \langle b \rangle$ are the following: \begin{equation*}
        \left\{\begin{array}{c}a\mapsto a \\ b \mapsto b^{-1}\end{array}\right\}\hspace{15pt} \left\{\begin{array}{c}a\mapsto a^{-1} \\ b \mapsto b \end{array}\right\}\hspace{15pt}\left\{\begin{array}{c}a\mapsto a^{-1} \\ b \mapsto b^{-1}\end{array}\right\}
    \end{equation*}
    Thus there are three nontrivial homomorphisms from $K$ into $\aut(H)$ given by sending the generator of $K$ into one of these three elements of order $2$ (as usual, the trivial homomorphism gives the direct product: $H\times K \cong \Z/30\Z$).

    Let $K = \langle k \rangle$. If the homomorphism $\varphi_1K\rightarrow \aut(H)$ is defined by mapping $k$ to the first automorphism above, then $G_1 = H\rtimes_{\varphi_1}K$ is seen to be isomorphic to $\Z/5\Z\times D_6$ (in particular it is $\langle a \rangle \times \langle b,k\rangle$).

    If $\varphi_2$ is defined by mapping $k$ to the second automorphism above, then $G_2 = H\rtimes_{\varphi_2}K$ is seen to be isomorphic to $\Z/3\Z\times D_{10}$ (factorization: $\langle b \rangle \times \langle a,k\rangle$).

    If $\varphi_3$ is defined by mappign $k$ to the third automorphism above, then $G_3=H\rtimes_{\varphi_3}K$ is isomorphic to $D_{30}$.
\end{eg}