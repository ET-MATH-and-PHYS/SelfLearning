%%%%%%%%%% Fld Defs %%%%%%%%%%
\chapter{\textsection\textsection Basic Definitions and Examples: Fields}



\begin{defn}
    A \Emph{field} is a commutative division ring.
\end{defn}


\begin{defn}
    The characteristic of a field $F$, denoted $\ch(F)$, is the smallest $p \in \N = \{1,2,...\}$ such that $p\cdot 1_F = 0_F$ if it exists, and $\ch(F) = 0$, otherwise.
\end{defn}

From here on out we will denote the multiplicative identity by $1$ and the additive identity by $0$ for any field.

\begin{rmk}
    If $F$ is a field and $\ch(F) = p$ for $p \neq 0$, then $p$ is a prime.
\end{rmk}
\begin{proof}
    Suppose that $F$. If $\ch(F) = 0$ then we are done, so suppose $\ch(F) = n$ for $n \geq 1$. If $n = 1$ then we have that $1 = 1\cdot 1 = 0$, so $F = \{0\}$. But, since $F$ is a field $1 \neq 0$, so we have a contradiction. Thus, $n > 1$. Towards a contradiction suppose that $n = a\cdot b$ is composite, where $1 < a,b < n$. Then $0 = n\cdot 1 = (a\cdot 1)\cdot (b\cdot 1)$. Thus, since $F$ is a field either $a\cdot 1 = 0$ or $b \cdot 1 = 0$. But, $a,b < n$ and by assumption $\ch(F) = n$ so $n$ is the minimal positive integer for which $n\cdot 1 = 0$, so we obtain a contradiction. Therefore, $n$ must be prime, as claimed. Moreover, suppose $m \cdot 1 = 0$ for some $m \neq 0$. Then let $m = nq + r$ be the division of $m$ by $n$ with remainder $0\leq r<n$. Then we have that $0 = m\cdot 1 = nq\cdot 1 + r\cdot 1 = r\cdot 1$, but since $r < n$ we must have that $r = 0$ by minimality of $n$. Hence, $n \vert m$.
\end{proof}

\begin{eg}
    \leavevmode
    \begin{enumerate}
        \item $\R, \Q, \C$, fields of characteristic zero.
        \item $\F_p = \Z/p\Z$ the finite field of $p$ elements.
        \item For $\F_p[x]$ polynomials, we have $\F_p(x)$ the field of rational functions with coefficients from $\F_p$. This is isomorphic to the quotient or fraction field of the integral domain $\F_p[x]$.
    \end{enumerate}
\end{eg}

\begin{rmk}
    For a field $F$, we have a unique ring homomorphism $\varphi:\Z\rightarrow F$ defined by $\varphi(n) := n\cdot 1 = \underbrace{1+1+...+1}_{\text{$n$ times}}$ for $n \in \N$, and $\varphi(-n) = -\varphi(n)$. Note that $\ker(\varphi) \subseteq \Z$ is an ideal, so $\ker(\varphi) = n\Z$ for $n \in \N\cup \{0\}$. Moreover, $\Z/\ker(\varphi) \cong \varphi(\Z)$, which is a subring of $F$ since $\varphi$ is a ring homomorphism. Moreover, $\varphi(\Z)$ is isomorphic to $\Z$ if $\ker(\varphi) = (0)$, and $\varphi(\Z)$ is isomorphic to $\Z/p\Z$ for $p$ a prime if $\ker(\varphi) = p\Z$. 

    Thus, each field $F$ has a subring which is isomorphic to either $\Z$ or $\Z/p\Z$. By the Field of Fractions technique we have that $F$ has a \Emph{subfield} isomorphic to either $\Q$ or $\F_p$.
\end{rmk}
\begin{proof}
    First, suppose $\ch(F) = 0$. Then since $\varphi(n) = n\cdot \varphi(1) = n\cdot 1$, we have that $n\cdot 1 \neq 0$ for all $n \neq 0$. Hence, $\ker(\varphi) = (0)$, and $\Z$ is isomorphic to a subring of $F$. Now, suppose $\ch(F) = p$ for some prime $p$. Then, we have that $\varphi(pq) = pq\varphi(1) = pq\cdot1 = 0$ for all $pq \in p\Z$, so $p\Z \subseteq \ker(\varphi)$. Next, let $m \in \ker(\varphi)$. Then $m\cdot 1 = 0$, which implies $p\vert m$. Hence, $m \in p\Z$ so $\ker(\varphi) = p\Z$. Thus, we conclude that $F$ has a subring isomorphic to $\Z/p\Z$.
\end{proof}


\begin{rec}
    Reminder that the only ideals in a field $F$ are either $(0)$ or $F$.
\end{rec}


\begin{defn}
    The \Emph{prime subfield} of a field $F$ is \Emph{generated} by $1_F$ and is isomorphic to either $\Q$ or $\F_p$ for some prime $p$.
\end{defn}
