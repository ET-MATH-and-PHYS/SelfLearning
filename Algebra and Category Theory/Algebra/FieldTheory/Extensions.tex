%%%%%%%%%% Extensions %%%%%%%%%%
\chapter{\textsection\textsection Field Extensions}

\section{\textsection Initial Definitions and Examples}

\begin{defn}
    If $K$ is a field containing the subfield $F$, then $K$ is said to be an \Emph{extension field} of $F$ denoted $K/F$.
        \begin{center}
            \begin{tikzpicture}[baseline = (a).base]
            \node[scale = 1] (a) at (0,0){
                \begin{tikzcd}
                    K \ar[d, dash] \\
                    F
                \end{tikzcd}
            };
            \end{tikzpicture}
        \end{center}
    We call $F$, the field being extended, the \Emph{base field}.
\end{defn}

\begin{rmk}
    Suppose $K/F$. Let $x,y \in K$ and $c_1,c_2 \in F$. Then we have \begin{enumerate}
        \item $c_1(x+y) = c_1x+c_1y$
        \item $c_1(c_2x) = (c_1c_2)x$
        \item $(c_1+c_2)x = c_1x+c_2x$
        \item $1\cdot x = x$
    \end{enumerate}
    along with the fact that $K$ is an abelian group over $+$ as it is a field, so $K$ is an $F$-vector space.
\end{rmk}

Note that what the ``base field" or ``extension field" is dependent on the context. For instance $\R$ is the base field in the context of:
\begin{center}
    \begin{tikzpicture}[baseline = (a).base]
    \node[scale = 1] (a) at (0,0){
        \begin{tikzcd}
            \C \ar[d, dash] \\
            \R
        \end{tikzcd}
    };
    \end{tikzpicture}
\end{center}
while $\R$ is the extension field in the context of:
\begin{center}
    \begin{tikzpicture}[baseline = (a).base]
    \node[scale = 1] (a) at (0,0){
        \begin{tikzcd}
            \R \ar[d, dash] \\
            \Q
        \end{tikzcd}
    };
    \end{tikzpicture}
\end{center}



\begin{defn}
    The degree of $K$ over $F$ is $[K:F] = \dim_F(K)$. If $\dim_F(K)$ is finite then $K$ is said to be a \Emph{finite extension} of $F$. If $\dim_F(K)$ is infinite then $K$ is said to not be a finite extension of $F$ (or an \Emph{infinite extension}).
\end{defn}


\begin{thm}\label{thm:existext}
    Let $F$ be a field and $p(x) \in F[x]$ an \Emph{irreducible polynomial}. Then there exists a field $K$ containing $F' \cong F$, in which $p(x)$ has a root; that is $p(\alpha) = 0$ for some $\alpha \in K$, where $p$ is now envisioned as the corresponding polynomial in $F'$.
\end{thm}
\begin{proof}
    Let $F$ be a field and $p(x) \in F[x]$ an irreducible polynomial. Then the ideal generated by $p(x)$, $(p(x))$, is maximal in $F[x]$ so $F[x]/(p(x)) = K$ is a field. Consider $p \in K[X]$ such that for $p(x) = a_nx^n+...+a_0$ in $F[x]$, we have $$p = (a_n+(p(x)))X^n+...+(a_0+(p(x)))$$  Moreover, for $\alpha = x + (p(x))$ in $K$, we have that \begin{align*}
        p(\alpha) &= (a_n+(p(x)))\alpha^n+...+(a_0+(p(x))) \\
        &= (a_nx^n+(p(x))) + ... + (a_0+(p(x))) \\
        &= (a_nx^n+...+a_0) + (p(x)) \\
        &= p(x) + (p(x)) \\
        &= 0 + (p(x))
    \end{align*}
    so $\alpha$ is a root of $p$ in $K[x]$. Finally, note that the map $\varphi:F\rightarrow K$ sending $\varphi(a) = a+(p(x))$ is a ring monomorphism since $F$ is a field, and thus restricting the codomain to $\varphi(F) = F'$, we have that $F \cong F'$, a subfield of $K$, as desired.
\end{proof}


\begin{thm}\label{thm:vectext}
    Let $p(x) \in F[x]$ for $F$ a field, irreducible of degree $n$ over $F$, and $K = F[x]/p(x)$. Let $\theta = x\mod p(x)$. Then $1,\theta,\theta^2,...,\theta^{n-1}$ is a basis for $K$ over $F$. That is, \begin{equation}
        K = \spn(1,\theta,\theta^2,...,\theta^{n-1}) = \{a_0+a_1\theta+...+a_{n-1}\theta^{n-1}:a_0,...,a_{n-1} \in F\}
    \end{equation}
    so $\dim_F(K) = n = [K:F]$.
\end{thm}
\begin{proof}
    Let $f(x) \in F[x]$. Then since $F$ is a field we have by the Division Algorithm that there exists $P,Q \in F[x]$ such that $f = pQ+P$, where either $P = 0$ or $\deg(P) <\deg(p)$. Then $f = P\mod p(x)$, where $P = \sum_{i=0}^{n-1}a_ix^i$ since $\deg(p) = n$. Then $$f = P\mod p(x) = \sum_{i=0}^{n-1}a_i\theta^i \in \spn(1,\theta,\theta^2,...,\theta^{n-1}$$ so we conclude that $$K = \spn(1,\theta,\theta^2,...,\theta^{n-1})$$
    Now, suppose there existed a linear dependence $\sum_{i=0}^{n-1}c_i\theta^i = 0$ in $K$. Then by definition $\sum_{i=0}^{n-1}c_ix^i \in (p(x))$, so $p(x)\vert \sum_{i=0}^{n-1}c_ix^i$. But, $\deg(p(x)) = n$ while $\deg\left(\sum_{i=0}^{n-1}c_ix^i\right) \leq n-1$ or $\sum_{i=0}^{n-1}c_ix^i = 0$, so we must have that that $c_i = 0$ for all $i$. Thus, $\{1,\theta,\theta^2,...,\theta^{n-1}\}$ is indeed a basis for $K$. 
\end{proof}


\begin{eg}
    Consider $\R[x]/(x^2+1) \cong \{a+b\theta:a,b \in \R\}$ by the previous Theorem. Moreover, in $\R[x]/(x^2+1) = \R(\theta)$, for $p = x^2+1$ $p(\theta) = 0$, so $\theta^2 = -1$.
\end{eg}


\begin{eg}
    In general, if $\theta \in K$ is a root of the irreducible polynomial \begin{equation*}
        p(x) = \sum_{i=0}^np_ix^i
    \end{equation*}
    we can compute $\theta^{-1} \in K$ from \begin{equation*}
        \theta(p_n\theta^{n-1}+p_{n-1}\theta^{n-2}+...+p_1) = -p_0
    \end{equation*}
    namely \begin{equation*}
        \theta^{-1} = -p_0^{-1}(p_n\theta^{n-1}+p_{n-1}\theta^{n-2}+...+p_1)
    \end{equation*}
    where $p_0 \neq 0$ since $p(x)$ is irreducible.
\end{eg}


\begin{defn}
    If $K$ is an extension field of $F$ containing $\alpha, \beta,...$ Then the smallest field containing both $\alpha,\beta,...$ and $F$ is denoted $F(\alpha,\beta,...)$. When we just adjoin $\alpha$, then $F(\alpha)$ is said to be a \Emph{simple extension} of $F$ with primitive element $\alpha$.
\end{defn}


\begin{thm}\label{thm:simplextiso}
    Let $F$ be a field and $p(x) \in F[x]$ and irreducble polynomial. Suppose $K$ is an extension field of $F$ containing the root $\alpha$ of $p(x)$. That is $p(\alpha) = 0$ in $K$. Let $F(\alpha)$ denote the subfield of $K$ generated by $F$ and $\alpha$. Then $F(\alpha) \cong F[x]/(p(x))$.
\end{thm}
\begin{proof}
    Let $\varphi:F[x]\rightarrow F(\alpha) \subseteq K$, defined by $\varphi(f(x)) = f(\alpha)$. Indeed, if $f(x) = c_0+c_1x+...+c_nx^n$, then $\varphi(f(x)) = c_0+c_1\alpha+...+c_n\alpha^n \in F(\alpha)$. Moreover, $\varphi$ is ring homomorphism, since the evaluation map is a ring homomorphism. Then we have that $$\ker(\varphi) = \{g(x) \in F[x]:g(\alpha) = 0\}$$
    It follows that for all $h(x)p(x) \in (p(x))$, $\varphi(h(x)p(x)) = h(\alpha)p(\alpha) = 0$ in $F(\alpha)$, so $(p(x)) \subseteq \ker(\varphi)$. Since $(p(x)) \subseteq \ker(\varphi)$ we have by the Factor Theorem a ring homomorphism $$\overline{\varphi}:F[x]/(p(x)) \rightarrow F(\alpha)$$
    But, since $p(x)$ is irreducible in $F[x]$, $F[x]/(p(x))$ so $\overline{\varphi}$ must be injective. Then, we have that $F[x]/(p(x))$ is isomorphic to a subfield of $F(\alpha)$ containing $F$ and $\alpha$. But then by definition $\overline{\varphi}(F[x]/(p(x)))$ contains $F(\alpha)$, so $\varphi$ must be surjective and hence an isomorphism since $F(\alpha)$ is the smallest such field. Thus $F[x]/(p(x)) \cong F(\alpha)$.
\end{proof}


\begin{eg}
    Consider $\Q[x]/(x^2-2)$. Then by our previous theorem $\Q[x]/(x^2-2) \cong \Q(\sqrt{2})$, as $(\sqrt{2})^2 - 2 = 0$, so $p(x) = x^2-2$ has $\alpha = \sqrt{2}$ as a root. Note $\beta = -\sqrt{2}$ is also a root, so $\Q(\sqrt{2}) \cong \Q(-\sqrt{2})$.
\end{eg}

\begin{eg}
    Consider $\Q[x]/(x^3-2)$, so $\Q[x]/(x^3-2) \cong \Q(\sqrt[3]{2})$ has $p(x) = x^3-2$ has $p(\alpha) = 0$ for $\alpha = \sqrt[3]{2}$. Then, for $\omega = e^{2\pi i/3}$, we have roots $\omega\alpha$ and $\omega^2\alpha$ of $p(x)$. Then we have by our previous theorem $\Q(\sqrt[3]{2}) \cong \Q(\omega\sqrt[3]{2}) \cong \Q(\omega^2\sqrt[3]{2})$. By \ref{thm:simplextiso}, these fields are algebraically indistinguishable. 
\end{eg}


\begin{rmk}
    Suppose $\phi:F\xrightarrow{\sim} F'$ is an isomorphism between fields $F$ and $F'$. Then we can extend $\phi$ to the isomorphism \begin{equation}
        \map{\phi':F[x]\xrightarrow{\sim}{F'[x]}}{\phi'(c_0+c_1x+...+c_nx^n) = \phi(c_0)+\phi(c_1)x+...+\phi(c_n)x^n}
    \end{equation}
    If $p(x)$ is irreducible over $F$, then $(p(x))$ is maximal in $F[x]$, so $\phi'((p(x))) = (p'(x))$ is maximal, which implies $p'(x)$ is irreducible in $F'[x]$. Then the theorem follows:
\end{rmk}


\begin{thm}
    Let $\alpha$ be a root of $p(x)$ and $\beta$ a root of $p'(x) = \phi'(p(x))$ in the extension fields $F(\alpha)$ and $F'(\beta)$. Then there is an isomorphism \begin{equation}
        \sigma:F(\alpha)\xrightarrow{\sim}F'(\beta)
    \end{equation}
    where $\sigma\rvert_F = \phi$ and $\alpha \mapsto \beta$. Then the diagram
        \begin{center}
            \begin{tikzpicture}[baseline = (a).base]
            \node[scale = 1] (a) at (0,0){
                \begin{tikzcd}
                    F(\alpha) \ar[r, "\sigma"] \ar[d,twoheadrightarrow]& F(\beta) \ar[d, twoheadrightarrow] \\
                    F \ar[r, "\phi"] & F'
                \end{tikzcd}
            };
            \end{tikzpicture}
        \end{center}
    commutes.
\end{thm}
\begin{proof}
    Let $p(x) \in F[x]$ be irreducible. Then as $\phi$ is an isomorphism, so is $\phi'$ and hence we have that $\phi'((p(x))) = (\phi'(p(x)))$ is a maximal ideal. Indeed, let $(\phi'(p(x))) \subseteq J$ for $J$ an ideal in $F'[x]$. Then ${\phi'}^{-1}(J)$ is an ideal in $F[x]$ containing $(p(x))$ by the correspondence theorem. Thus, $(p(x)) = {\phi'}^{-1}(J)$ or $F[x] = {\phi'}^{-1}(J)$, which implies that $\phi'((p(x))) = J$ or $F'[x] = J$ by bijectivity of the correspondence. Then, I claim that $\phi':F[x]\xrightarrow{\sim}F'[x]$ induces an isomorphism $\Phi:F[x]/(p(x))\xrightarrow{\sim}F'[x]/\phi'((p(x)))$. Indeed, $\Phi(f(x)+(p(x))) := \phi'(f(x))+(\phi'(p(x)))$ is a well defined isomorphism (since $\phi'$ is an isomorphism). Then since $F[x]/(p(x)) \cong F(\alpha)$ and $F'[x]/\phi'((p(x))) \cong F'(\beta)$, by \ref{thm:simplextiso}, we have that $F(\alpha) \cong F'(\beta)$.
\end{proof}



\section{\textsection Algebraic Extensions}

\begin{defn}
    $\alpha \in K$ over $F$ is \Emph{algebraic} over $F$ if $\alpha$ is a root of some nonzero polynomial $f(x) \in F[x]$. If $\alpha$ is not algebraic over $F$, $\alpha$ is said to be \Emph{transcendental} over $F$. The extension $K/F$ is \Emph{algebraic} if and only if every element of $K$ is algebraic.
\end{defn}


\begin{prop}\label{prop:minpoly}
    Let $\alpha$ be algebraic over $F$. Then there exists a unique monic irreducible polynomial $m_{\alpha,F}(x) \in F[x]$ which has $\alpha$ as a root. Furthermore, $m_{\alpha,F}(x)$ divides any $f(x) \in F[x]$ such that $f(\alpha) = 0$.
\end{prop}
\begin{proof}
    First, since $\alpha$ is algebraic over $F$, there exist $f(x) \in F[x]$ such that $f(\alpha) = 0$. Then, by well-ordering of $\N$, let $g(x) \in F[x]$ be a polynomial of minimal degree such that $g(\alpha) = 0$. Multiplying by a constant we may assume that $g(x)$ is monic. Suppose that $g(x)$ were reducible over $F$, then $g(x) = a(x)b(x)$ in $F$ for $g(x) > \deg(a(x)),\deg(b(x)) \geq 1$. But, then $g(\alpha) = a(\alpha)b(\alpha)$, since the evaluation map is a homomorphism, so as $F$ is a field either $a(\alpha) = 0$ or $b(\alpha) = 0$, contradicting the minimality of $g$. Hence, $g$ is irreducible. Now suppose $f(x) \in F[x]$ such that $f(\alpha) = 0$, and by the division algorithm obtain $P,Q \in F[x]$ such that $f = gP+Q$, with $\deg(Q) < \geq (g)$. But then $0=f(\alpha)=g(\alpha)P(\alpha)+Q(\alpha)=Q(\alpha)$, so by minimality of the degree of $g$ we must have that $Q(x) = 0$. Hence, $g(x)\vert f(x)$ as desired. Finally, by this result $g(x)$ would divide any other monic irreducible polynomial having $\alpha$ as a root, so $g(x)$ must be unique.
\end{proof}



\begin{defn}
    $m_{\alpha,F}(x)$ in the Proposition \ref{prop:minpoly} is the \Emph{minimal polynomial} of $\alpha$ over $F$.
\end{defn}


\begin{eg}
    $m_{\sqrt{2},\Q} = x^2-2$, but $m_{\sqrt{2},\R} = x-\sqrt{2}$.
\end{eg}

\begin{cor}
    $\alpha \in F$ if and only if $m_{\alpha,F} = x-\alpha$.
\end{cor}


\begin{cor}
    For a field extension $L/F$ and $\alpha$ is algebraic over both $F$ and $L$, then $m_{\alpha,L}(x)$ divides $m_{\alpha,F}(x)$ in $L[x]$.
\end{cor}
\begin{proof}
    Note that since $F\subseteq L$, we have by Proposition \ref{prop:minpoly} that $m_{\alpha,L}(x) \vert m_{\alpha,F}(x)$ since $\alpha$ is a root of $m_{\alpha,F}(x)$, and $m_{\alpha,L}(x)$ is the minimal such polynomial in $L$.
\end{proof}



\begin{prop}\label{prop:algmin}
    Let $\alpha$ be algebraic over $F$, $F(\alpha)$ generated by $\alpha$ and $F$. Then $F(\alpha) \cong F[x]/(m_{\alpha,F}(x))$, and $[F(\alpha):F] = \deg(m_{\alpha,F}(x)) = \deg(\alpha)$.
\end{prop}
\begin{proof}
    Since $m_{\alpha,F}(x)$ is an irreducible polynomial over $F$ with root $\alpha$, we have by \ref{thm:simplextiso} that $F(\alpha) \cong F[x]/(m_{\alpha,F}(x))$. Then, by Theorem \ref{thm:vectext} we have that $[F(\alpha:F] = [F[x]/(m_{\alpha,F}(x)):F] = \deg(m_{\alpha,F}(x))$, as claimed.
\end{proof}



\begin{prop}
    The element $\alpha$ is algebraic over $F$ if and only if the simple extension $F(\alpha)/F$ is a finite extension.
\end{prop}
\begin{proof}
    [$\implies$] By Proposition \ref{prop:algmin}.
    [$\impliedby$] Suppose $F(\alpha)/F$ is finite, so $[F(\alpha):F] = n$ for some $n \in \N$. Thus $\{1,\alpha,\alpha^2,...,\alpha^n\}$ must be linearly dependent over $F$. Thus, there exist $c_i \in F$, $0 \leq i \leq n$, not all zero such that \begin{equation*}
        c_0 + c_1\alpha + ... + c_n\alpha^n = 0
    \end{equation*}
    Thus, we have that $f(x) = c_0+c_1x+...+c_nx^n$ is a non-zero polynomial in $F[x]$ which takes $\alpha$ as a root, so $\alpha$ is algebraic over $F$ by definition.
\end{proof}

\begin{cor}\label{cor:finalg}
    If the extension $K/F$ is finite, then its algebraic.
\end{cor}
\begin{proof}
    Let $\alpha \in K$ so $F(\alpha)$ is a subfield of $K$, and $[F(\alpha):F] \leq [K:F] = n$ for some $n \in \N$. Thus, by the previous proposition $\alpha$ is algebraic.
\end{proof}



\begin{eg}[Quadratic Extensions over Fields of Characteristic not equal to 2]\label{eg:quadext}
    Let $F$ be a field of characteristic $\neq 2$, and let $K$ be an extension of $F$ of degree $2$, $[K:F] = 2$. Let $\alpha$ be any element of $K$ not contained in $F$. By the previous corollary $\alpha$ satisfies an equation of degree at most $2$ over $F$. Moreover, since $\alpha \notin F$, it cannot be of degree $1$, so the minimal polynomial of $\alpha$ is a monic quadratic polynomial \begin{equation*}
        m_{\alpha,F}(x) = x^2+bx+c, b,c \in F
    \end{equation*}
    Since $F \subset F(\alpha) \subseteq K$ and $F(\alpha)$ is already a vector space over $F$ of dimension $2$, we have $K = F(\alpha)$. By the quadratic formula, which is valid for any field of characteristic $\neq 2$, the roots of this quadratic formula are \begin{equation*}
        \alpha = \frac{-b\pm \sqrt{b^2-4c}}{2}
    \end{equation*}
    Here $b^2-4c$ is not a square in $F$ since $\alpha$ is not an element of $F$, and the symbol $\sqrt{b^2-4c}$ denotes a root of the equation $x^2-(b^2-4c)=0$ in $K$. Note that here there is no natural choice of one of the roots analogous to choosing the positive square root of $2$ in $\R$ - the roots are algebraically indistinguishable.


    Now $F(\alpha) = F(\sqrt{b^2-4c})$ as follows: by the formula above, $\alpha$ is an element of the field on the right, hence $F(\alpha) \subseteq F(\sqrt{b^2-4c})$. Conversely, $\sqrt{b^2-4c} = \mp(b+2\alpha)$ shows that $\sqrt{b^2-4c}$ is an element of $F(\alpha)$, which gives the reverse inclusion $F(\sqrt{b^2-4c}) \subseteq F(\alpha)$.
\end{eg}

\begin{cor}
    It follows that any extension $K$ of $F$ of degree $2$ is of the form $F(\sqrt{D})$ where $D$ is an element of $F$ which is not a square in $F$, and conversely, every such extension is an extension of degree $2$ of $F$. For this reason extions of degree $2$ of a field $F$ are called \Emph{quadratic extensions} of $F$.
\end{cor}


\begin{thm}\label{thm:multext}
    Let $F \subseteq K \subseteq L$ be fields. Then \begin{equation*}
        [L:F] = [L:K][K:F]
    \end{equation*}
    \begin{center}
        \begin{tikzpicture}[baseline = (a).base]
        \node[scale = 1] (a) at (0,0){
            \begin{tikzcd}
                L \\
                K \\
                F
                \arrow["{[L:K]}", no head, from=1-1, to=2-1]
                \arrow["{[K:F]}", no head, from=2-1, to=3-1]
                \arrow["{[L:F]}"', shift right=2, draw=none, from=1-1, to=3-1]
            \end{tikzcd}
        };
        \end{tikzpicture}
    \end{center}
\end{thm}
\begin{proof}
    First, suppose that $[L:K] = m$ and $[K:F] = n$ are finite. Let $\alpha_1,\alpha_2,...,\alpha_m$ be a basis for $L$ over $K$, and let $\beta_1,\beta_2,...,\beta_n$ be a basis for $K$ over $F$. Then every element of $z \in L$ can be written as a linear combination \begin{equation*}
        z = \sum_{i=1}^ma_i\alpha_i
    \end{equation*}
    where $a_i \in K$ for each $i$. Thus, there exist $b_{ij} \in F$ such that \begin{equation*}
        a_i = \sum_{j=1}^nb_{ij}\beta_j
    \end{equation*}
    for all $1 \leq i \leq m$. Substituting we have that \begin{equation*}
        z = \sum_{i=1}^ma_i\alpha_i = \sum_{i=1}^m\sum_{j=1}^nb_{ij}\beta_j\alpha_i
    \end{equation*}
    Hence, $$z \in \spn(\beta_1\alpha_1,...,\beta_1\alpha_m,\beta_2\alpha_1,...,\beta_2\alpha_m,...,\beta_n\alpha_1,...,\beta_n\alpha_m)$$
    so the $\beta_j\alpha_i$ are a spanning set for $K$. Now, suppose $\sum c_{ij}\alpha_i\beta_j = 0$ is a linear dependence over $F$. In particular, we have that \begin{equation*}
        \sum c_{ij}\alpha_i\beta_j = \sum_{i=1}^m\left(\sum_{j=1}^nc_{ij}\beta_j\right)\alpha_j = 0
    \end{equation*}
    where $\sum_{j=1}^nc_{ij}\beta_j \in K$ for each $i$. Hence, by the linear independence of the $\alpha_i$, we have for each $i$ that \begin{equation*}
        \sum_{j=1}^nc_{ij}\beta_j = 0
    \end{equation*}
    But then $c_{ij} \in F$ for all $i,j$, so by the independence of the $\beta_j$ over $F$, we find that $c_{ij} = 0$ for all $i$ and $j$. Hence, the set is linearly independent and consequently a basis for $L$. Therefore, we conclude that $[L:F] = mn = [L:K][K:F]$.


    Now, if $[L:F]$ is infinite, then either $[L:K]$ is infinite or $[K:F]$ is infinite since otherwise by the previous argument $[L:F]$ would be finite, a contradiction.

    If $[L:K]$ is infinite, there are infinitely many elements of $L$ linearly independent over $K$, so as $F \subseteq K$ there are infinitely many elements of $L$ linearly independent over $F$, so $[L:F]$ is infinite. If $[K:F]$ is infinite, there are infinitely many elements of $K$ linearly independent over $F$. But as $L \supseteq K$, there are also infinitely many elements of $L$ that are linearly independent over $F$.
\end{proof}



\begin{cor}\label{cor:extdiv}
    Suppose $L/F$ is a finite extension and let $K$ be a subfield of $L$ containing $F$, $F \subseteq K \subseteq L$. Then $[K:F]$ divides $[L:F]$.
\end{cor}
\begin{proof}
    By Theorem \ref{thm:multext} $[L:F] = [L:K][K:F]$, so indeed $[K:F] \vert [L:F]$.
\end{proof}


\begin{eg}
    Let $\alpha$ be a real root of $x^3-3x-1 = 0$, which exists between $0 < x < 2$ (exists by IVT). Then I claim $\sqrt{2} \notin \Q(\alpha)$. Recall that $[\Q(\sqrt{2}):\Q] = 2$ as it is a quadratic extension. Now, note that by the Rational Roots Theorem $x^3-3x-1$ is indeed irreducible over $\Q$, so $[\Q(\alpha):\Q] = 3$. Hence, since $2\cancel{\vert}3$, by the previous corollary we conclude that $\Q(\sqrt{2})\nsubseteq \Q(\alpha)$.
\end{eg}

\begin{eg}
    Consider $\alpha = \sqrt[6]{2}$ adjoined to $\Q$. Then by Eisenstein $m_{\alpha,\Q}(x) = x^6 - 2$. In particular, $[\Q(\sqrt[6]{2}),\Q] = 6$. Note that $\sqrt[6]{2}^3 = \sqrt{2}$, so $\sqrt{2} \in \Q(\sqrt[6]{2})$, and $\Q(\sqrt{2}) \subseteq \Q(\sqrt[6]{2})$. Then, by the corollary $[\Q(\sqrt[6]{2}),\Q(\sqrt{2})] = 6/2 = 3$, so $\sqrt[6]{2}$ is of degree $3$ over $\Q(\sqrt{2})$. Indeed, $m_{\sqrt[6]{2},\Q(\sqrt{2})}(x) = x^3 - \sqrt{2}$ is a minimal polynomial by the corollary, so it is also irreducible over $\Q(\sqrt{2})$.
\end{eg}

\begin{defn}
    An extension $K/F$ is \Emph{finitely generated} if there are elements $\alpha_1,\alpha_2,...,\alpha_k \in K$ such that $K = F(\alpha_1,\alpha_2,...,\alpha_k)$.
\end{defn}


\begin{lem}
    $F(\alpha,\beta) = (F(\alpha))(\beta)$, that is, the field generated over $F$ by $\alpha$ and $\beta$ is the field generated by $\beta$ over the field $F(\alpha)$ generated by $\alpha$.
\end{lem}
\begin{proof}
    First, we note that $F(\alpha,\beta)$ contains $F$, $\alpha$, and $\beta$. Thus, since $F(\alpha)$ is the smallest field containing $F$ and $\alpha$ it contains $F(\alpha)$ and $\beta$. Thus, as $(F(\alpha))(\beta)$ is the smallest such field, $(F(\alpha))(\beta) \subseteq F(\alpha,\beta)$. Conversely, since $(F(\alpha))(\beta)$ contains $F$, $\alpha$, and $\beta$, by minimality of $F(\alpha,\beta)$ we find that $F(\alpha,\beta) \subseteq (F(\alpha))(\beta)$.
\end{proof}

By the lemma we have that \begin{equation*}
    K = F(\alpha_1,...,\alpha_k) = (F(\alpha_1,...,\alpha_{k-1}))(\alpha_k)
\end{equation*}
and so by iterating, we see that $K$ is obtained by taking the field $F_1$ generated over $F$ by $\alpha_1$, then the field $F_2$ generated over $F_1$ by $\alpha_2$, and so on, with $F_k = K$. This gives a sequence of fields: \begin{equation*}
    F = F_0 \subseteq F_1 \subseteq F_2 \subseteq ...\subseteq F_k = K
\end{equation*}
where $F_{i+1} = F_i(\alpha_{i+1})$ for $i = 0,1,...,k-1$. By the multiplicativity of extension degrees we see that \begin{equation*}
    [K:F] = [F_k:F_{k-1}]...[F_2:F_1][F_1:F_0]
\end{equation*}


\begin{thm}\label{thm:finalgext}
    The extension $K/F$ is finite if and only if $K$ is generated by a finite number of algebraic elements over $F$.
\end{thm}
\begin{proof}
    If $K/F$ is finite of degree $n$, let $\alpha_1,\alpha_2,...,\alpha_n$ be a basis for $K$ as a vector space over $F$. By Corollary \ref{cor:extdiv} we have that $[F(\alpha_i):F]$ divides $[K:F] = n$ for all $i = 1,2,...,n$, so they are finite and hence each $\alpha_i$ is algebraic over $F$. Thus, $K$ is generated by a finite number of algebraic elements over $F$. Conversely, if $K$ was generated by a finite number of algebraic elements over $F$, say $\beta_1,...,\beta_n$ with $[F(\beta_i):F] = m_i$, we have that $$[K:F] = [F_n:F_{n-1}]...[F_2:F_1][F_1:F_0] \leq [F(\beta_n):F]...[F(\beta_2):F][F(\beta_1):F] = m_1m_2...m_n$$ so $K$ is a finite extension of $F$.
\end{proof}


\begin{cor}
    Suppose $\alpha$ and $\beta$ are algebraic over $F$. Then $\alpha\pm\beta,\alpha\beta,\alpha/\beta$ (for $\beta\neq 0$) are all algebraic.
\end{cor}
\begin{proof}
    By the theorem $F(\alpha,\beta)$ is a finite extension over $F$, and it contains all of these elements (if they exist). Moreover, since $F(\alpha,\beta)$ is finite all of its elements are algebraic by Corollary \ref{cor:finalg}, so these elements are algebraic.
\end{proof}


\begin{cor}
    Let $L/F$ be an arbitrary extension. Then the collection of elements of $L$ that are algebraic over $F$ form a subfield of $K$.
\end{cor}
\begin{proof}
    Let $\mathcal{A}$ be such a subset. Then by the previous corollary $\mathcal{A}$ is closed under addition, subtraction, multiplication, and inverses, so it is indeed a subfield of $L$.
\end{proof}


\begin{eg}[Algebraic Numbers]
    Consider the extension $\C/\Q$, and let $\overline{\Q}$ denote the subfield of all elements of $\C$ that are algebraic over $\Q$. In particular, the elements $\sqrt[n]{2}$ (positive $n$th roots of $2$ in $\R$) are all elements of $\overline{\Q}$, so that $[\overline{\Q},\Q]\geq n$ for all integers $n > 1$. Hence $\overline{Q}$ is an infinite algebraic extension of $\Q$, called the \Emph{algebraic numbers}.

    Consider $\overline{\Q}\cap \R$, the subfield of $\R$ consisting of elements algebraic over $\overline{\Q}$. The field $\Q$ is countable. The number of polynomials in $\Q[x]$ of any given degree $n$ is therefore also countable, being the union of a finite number of countable sets. Since these polynomials have at most $n$ roots in $\R$, the number of algebraic elements of $\R$ of degree $n$ is countable, being contained in a finite union of countable sets. Finally, the collection of all algebraic elements in $\R$ is the countable union (indexed by $n$) of countable sets, hence is countable. Since $\R$ is uncountable, it follows that there exist (in fact many) elements of $\R$ which are not algebraic, i.e. are transcendental, over $\Q$.
\end{eg}


\begin{thm}
    If $K$ is algebraic over $F$ and $L$ is algebraic over $K$, then $L$ is algebraic over $F$.
\end{thm}
\begin{proof}
    Let $\alpha \in L$. Then there exists $f(x) = \sum_{i=0}^n a_ix^i \in K[x]$ such that $f(\alpha) = 0$ in $L$. Consider the field $F(\alpha,a_0,a_1,...,a_n)$ generated over $F$ by $\alpha$ and the coefficients of this polynomial. Since $K/F$ is algebraic, the elements $a_0,a_1,...,a_n$ are algebraic over $F$, so the extension $F(a_0,a_1,...,a_n)/F$ is finite by Theorem \ref{thm:finalgext}. By the equation above we know that $\sum_{i=0}^na_i\alpha^i = 0$, so $\alpha$ generates an extension of this field of degree at most $n$, since its minimal polynomial over this field is a divisor of the polynomial above. Therefore \begin{equation*}
        [F(\alpha,a_0,a_1,...,a_n):F] = [F(\alpha,a_0,a_1,...,a_n):F(a_0,a_1,...,a_n)][F(a_0,a_1,...,a_n):F]
    \end{equation*}
    is also finite and by Theorem \ref{thm:finalgext} $F(\alpha,a_0,a_1,...,a_n)/F$ is an algebraic extension. In particular, the element $\alpha$ is algebraic over $F$, which proves that $L$ is algebraic over $F$.
\end{proof}

\begin{defn}
    Let $K_1$ and $K_2$ be two subfields of a field $K$. Then the \Emph{composite field} of $K_1$ and $K_2$, denoted $K_1K_2$, is the smallest subfield of $K$ containing both $K_1$ and $K_2$. Similarly, the composite of any collection of subfields of $K$ is the smallest subfield containing all the subfields.
\end{defn}



\begin{prop}
    Let $K_1$ and $K_2$ be two finite extensions of a field $F$ contained in $K$. Then \begin{equation*}
        [K_1K_2:F]\leq [K_1:F][K_2:F]
    \end{equation*}
    with equality if and aonly if an $F$-basis for one of the fields remains linearly independent over the other field. If $\alpha_1,\alpha_2,...,\alpha_n$ and $\beta_1,\beta_2,...,\beta_m$ are bases for $K_1$ and $K_2$ over $F$, respectively, then the elements $\alpha_i\beta_j$ for $i = 1,2,...,n$ and $j = 1,2,...,m$ span $K_1K_2$ over $F$.
\end{prop}
\begin{proof}
    Note that $K_1=F(\alpha_1,\alpha_2,...,\alpha_n)$ and $K_2 = F(\beta_1,\beta_2,...,\beta_m)$. Then observe that $K_1K_2$ contains $F,\alpha_1,...,\alpha_n,\beta_1,...,\beta_m$ so $F(\alpha_i,\beta_j|1\leq i \leq n,1\leq j \leq m) \subseteq K_1K_2$, and similarly $F(\alpha_i,\beta_j|1\leq i \leq n,1\leq j \leq m)$ contains both $K_1$ and $K_2$ by their minimality, so $K_1K_2 \subseteq F(\alpha_i,\beta_j|1\leq i \leq n,1\leq j \leq m)$, proving equality. Now, this is equal to $K_1(\beta_1,...,\beta_m)$, so $[K_1K_2:K_1]\leq m = [K_2:F]$, with equality if and only if these elements are linearly independent over $K_1$. Since $[K_1K_2:F] = [K_1K_2:K_1][K_1:F] \leq [K_2:F][K_1:F]$, the proposition is proven.
\end{proof}


By this proposition, and its proof, we have the following diagram:
\begin{center}
    \begin{tikzcd}
        & {K_1K_2} \\
        {K_1} && {K_2} \\
        & F
        \arrow["n"', no head, from=2-1, to=3-2]
        \arrow["m"', no head, from=3-2, to=2-3]
        \arrow["{\leq m}", no head, from=2-1, to=1-2]
        \arrow["{\leq n}", no head, from=1-2, to=2-3]
    \end{tikzcd}
\end{center}


\begin{cor}
    Suppose that $[K_1:F]=n, [K_2:F]=m$ in the previous proposition, where $n$ and $m$ are relatively prime: $\gcd(n,m) = 1$. Then $[K_1K_2:F] = [K_1:F][K_2:F] = nm$.
\end{cor}
\begin{proof}
    First observe that $[K_1K_2:F] = [K_1K_2:K_1][K_1:F]$ and $[K_1K_2:F] = [K_1K_2:K_2][K_2:F]$, so $n$ and $m$ divide $[K_1K_2:F]$. Let $l = \text{lcm}(m,n)$. Then note that $l$ must divide $[K_1K_2:F]$. Now, since $\gcd(n,m) = 1$, we have that $nm$ divides $[K_1K_2:F]$, so $nm \leq [K_1K_2:F] \leq nm$. Thus, $[K_1K_2:F] = nm$, as claimed.
\end{proof}





\section{\textsection Classical Straightedge and Compass Constructions}

The following three geometric problems posed by the Greeks can now be shown to not be possible: 
\begin{enumerate}
    \item[I.] (Doubling the Cube) Is it possible using only straightedge and compass to construct a cube with precisely twice the volume of a given cube?
    \item[II.] (Trisecting the Angle) Is it possible using only straightedge and compass to trisect any given angle $\theta$?
    \item[III.] (Squaring the Circle) Is it possible using only straightedge and compass to construct a square whose area is precisely the area of a given circle?
\end{enumerate}

Let $1$ denote some fixed unit distance. Then any distance is determined by its length $a \in \R$. We construct the usual cartesian plane $\R^2$, and view all of our constructions as occuring in $\R^2$. 

\begin{defn}
    A point $(x,y) \in \R^2$ is \Emph{constructible} starting with the given distance $1$ if and only if its coordinates $x$ and $y$ are constructible elements of $\R$. Elements of $\R$ are called \Emph{constructible} starting with the given distance $1$ if and only if they can be obtained by straightedge and compass constructions from $1$.
\end{defn}

Each straightedge and compass construction consists of a series of operations of the following four types: 
\begin{enumerate}
    \item connecting two given points by a straight line
    \item finding a point of intersection of two straight lines
    \item drawing a circle with given radius and center
    \item finding the point(s) of intersection of a straight line and a circle or the intersection of two circles
\end{enumerate}


\begin{thm}
    Given lengths $a$ and $b$, you can construct $a+b,a-b,ab,a/b$, given $b \neq 0$. 
\end{thm}

$a+b$ and $a-b$ can be attained by parallel lines. Then $ab$, $a/b$ can be attained by similar triangles. Moreover, using an upper half circle, with diameter $a+1$, we can attain $\sqrt{a}$. These facts imply that the set of constructible numbers is a subfield of $\R$. Moreover, this field is an extension of the rationals. 

It can be shown that for any field $F$, the straightedge constructions are closed in $F$, while the compass constructions at most bring $F$ to a quadratic extension of itself. Since quadratic extensions have degree $2$, and extension degrees are multiplicative, it follows that if $\alpha \in \R$ is obtained from elements in $F$ by a finite series of straightedge and compass constructions, then $\alpha$ is an element of an extension $K$ of $F$ of degree a power of $2$: $[K:F] 2^m$ for some $m$. Since $[F(\alpha):F]$ divides this extension, it must also be a power of $2$.

\begin{prop}
    If the element $\alpha \in \R$ is obtained from a field $F \subset \R$ by a series of compass and straightedge constructions, then $[F(\alpha):F] = 2^k$ for some integer $k \geq 0$.
\end{prop}

\begin{thm}
    None of the classical Greek problems: (I) Doubling the Cube, (II) Trisecting an Angle, (III) Squaring the Circle, is possible.
\end{thm}
\begin{proof}[Sketch]
    (I) Doubling the cube amounts to constructing $\sqrt[3]{2}$ in the reals starting with the unit $1$. Since $[\Q(\sqrt[3]{2}):\Q] = 3$ is not a power of $2$, this is impossible.

    (II) Trisecting the angle is possible for some angles, but not all. If an angle $\theta$ can be constructed, then so can $\cos\theta$ and $\sin\theta$. In attempt to trisect $60$ degrees, we see that an extension of $\Q$ by $\cos20^{\circ}$ has degree $3$, and is therefore not constructible, and consequently $20$ degrees is not constructible.

    (III) Squaring the circle is equivalent to determining if $\pi$ is constructible. In fact, $\pi$ is transcendental, so $[\Q(\pi):\Q]$ is infinite and hence $\pi$ is not constructible and we cannot square the circle.
\end{proof}


\section{\textsection Splitting Fields}

Let $F$ be a field, and $f(x) \in F[x]$ with $\deg(f(x)) = n$. Then we have shown there exists $E_1/F$ such that there is $\alpha_1 \in E_1$ for which $\alpha_1$ is a root of $f(x)$ and $f(x) = (x-\alpha_1)g_1(x)$ for some $g_1(x) \in E_1[x]$, where $[E_1:F] \leq n$.

\begin{eg}
    $f(x) = x^2-3x+2 \in \Q[x]$. Then $f(x) = (x-1)(x-2)$, so $\Q$ is its own extension field.
\end{eg}


Next, we can consider $g_1(x) \in E_1[x]$, and find an extension $E_2/E_1$ in which there exists $\alpha_2 \in E_2$ such that $g_1(x) = (x-\alpha_1)g_2(x)$ for some $g_2(x) \in E_2[x]$, and $[E_2:E_1] \leq n-1$, as $\deg(g_1(x)) = n-1$. It follows that $f(x) = (x-\alpha_1)(x-\alpha_2)g_2(x)$ in $E_2[x]$. In principle we can proceed inductively and obtain $E/F$ such that $f(x) = A(x-\alpha_1)(x-\alpha_2)...(x-\alpha_n)$, where $\alpha_1,\alpha_2,...,\alpha_n \in E$ and \begin{equation*}
    [E:F] = [E_n:E_{n-1}][E_{n-1}:E_{n-2}]...[E_2:E_1][E_1:F] \leq n!
\end{equation*}


\begin{defn}
    The extension field $K$ of $F$ is called a \Emph{splitting field} for the polynomial $f(x) \in F[x]$ if $f(x)$ factors completely into linear factors (or \Emph{splits completely}) in $K[x]$ and $f(x)$ does not factor completely into linear factors over any proper subfield of $K$ containing $F$.
\end{defn}

Note that if $f(x)$ is of degree $n$, then $f(x)$ has at most $n$ roots in $F$ and has precisely $n$ roots in $F$ if and only if $f(x)$ splits completely in $F[x]$.

\begin{thm}
    For any field $F$, if $f(x) \in F[x]$ then there exists an extension $K$ of $F$ which is a splitting field for $f(x)$.
\end{thm}
\begin{proof}
    We first show that there exists an extension $E$ of $F$ over which $f(x)$ splits completely into linera factors by induction on the degree $n$ of $f(x)$. If $n = 1$, then take $E = F$. Suppose now that $n > 1$. If the irreducible factors of $f(x)$ over $F$ are all of degree $1$, then $F$ is the splitting field for $f(x)$ and we may take $E= F$. Otherwise, at least one of the irreducible factors, say $p(x)$ of $f(x)$ in $F[x]$ is of degree at least $2$. Then by Theorem \ref{thm:existext} there exists an extension $E_1$ of $F$ containing a root $\alpha$ of $p(x)$. Over $E_1[x]$ $f(x)$ has the linear factor $x-\alpha$. The degree of the remaining factor $f_1(x)$ is $n-1$, so by the induction hypothesis there is an extension $E$ of $E_1$ containing all of the roots of $f_1(x)$. Since $\alpha \in E$, $E$ is an extension of $F$ containing all of the roots of $f(x)$. Now, let $K$ be the intersection of all the subfields of $E$ containing $F$ which also contain all roots of $f(x)$. Then $K$ is a field which is a splitting field for $f(x)$.
\end{proof}

\begin{defn}
    If $K$ is an algebraic extension of $F$ which is the splitting field over $F$ for a collection of polynomials $f(x) \in F[x]$ then $K$ is called a \Emph{normal extension} of $F$.
\end{defn}

\begin{eg}
    $x^2 - 2 \in \Q[x]$ splits in $\Q(\sqrt{2})$, with $x^2-2 = (x-\sqrt{2})(x+\sqrt{2})$ and $\pm\sqrt{2}\in \Q(\sqrt{2})$.
\end{eg}


\begin{eg}
    Consider $(x^2-2)(x^2-3) \in \Q$. Then $\Q(\sqrt{2},\sqrt{3})$ is the splitting field, and we have the field diagram:
    \begin{center}
        \begin{tikzcd}
            & {\mathbb{Q}(\sqrt{2},\sqrt{3})} \\
            {\mathbb{Q}(\sqrt{2})} & {\mathbb{Q}(\sqrt{6})} & {\mathbb{Q}(\sqrt{3})} \\
            & {\mathbb{Q}}
            \arrow["2"', no head, from=2-1, to=3-2]
            \arrow["2"', no head, from=2-2, to=3-2]
            \arrow["2"', no head, from=1-2, to=2-2]
            \arrow["2", no head, from=2-3, to=3-2]
            \arrow["2", no head, from=1-2, to=2-3]
            \arrow["2"', no head, from=1-2, to=2-1]
        \end{tikzcd}
    \end{center}
    Moreover, $[\Q(\sqrt{2},\sqrt{3}):\Q] = 4$.
\end{eg}

\begin{eg}
    Consider $x^3 - 2$ over $\Q$. Then for $\alpha = \sqrt[3]{2} \in \R$, $\alpha^3-2 = 0$. Complex roots are $\alpha,\alpha\zeta,\alpha\zeta^2$ for $\zeta = e^{\frac{2\pi i}{3}}$ the third root of unity. Then \begin{equation*}
        x^3-2 = (x-\alpha)(x-\alpha\zeta)(x-\alpha\zeta^2)
    \end{equation*}
    So the splitting field is $K = \Q(\alpha,\alpha\zeta,\alpha\zeta^2)$. Note $\alpha\zeta/\alpha = \zeta \in K$, so as $\zeta = \frac{-1+i\sqrt{3}}{2}$, $i\sqrt{3} = 2\zeta+1 \in K$. Then we can also write $K = \Q(\sqrt[3]{2},\sqrt{-3})$, so 
    \begin{center}
        \begin{tikzcd}
            & {\mathbb{Q}(\sqrt[3]{2},\sqrt{-3})} \\
            {\mathbb{Q}(\zeta_2)=\mathbb{Q}(\sqrt{-3})} && {\mathbb{Q}(\sqrt[3]{2})} & {\mathbb{Q}(\sqrt[3]{2}\zeta_2)} & {\mathbb{Q}(\sqrt[3]{2}\zeta_2^2)} \\
            & {\mathbb{Q}}
            \arrow["2"', no head, from=2-1, to=3-2]
            \arrow["3"', no head, from=2-3, to=3-2]
            \arrow["2"', no head, from=1-2, to=2-3]
            \arrow["3"{description}, no head, from=2-4, to=3-2]
            \arrow["2"{description}, no head, from=1-2, to=2-4]
            \arrow["3"', no head, from=1-2, to=2-1]
            \arrow["3"', no head, from=3-2, to=2-5]
            \arrow["2", no head, from=1-2, to=2-5]
        \end{tikzcd}
    \end{center}
    Adjoining $\sqrt{-3}$ to $\Q(\sqrt[3]{2})$, $[K:\Q(\sqrt[3]{2})] = 2$, with minimal polynomial $x^2+3 \in \Q(\sqrt[3]{2})[x]$. 
\end{eg}


\begin{eg}
    Consider $x^p - 2 \in \Q[x]$. Then for $\zeta_p = e^{\frac{2\pi i}{p}}$, the $p$th root of unity, we have 
    \begin{center}
        \begin{tikzcd}
        & {\mathbb{Q}(\sqrt[p]{2},\zeta_p)} \\
        {\mathbb{Q}(\zeta_p)} && {\mathbb{Q}(\sqrt[p]{2})} \\
        & {\mathbb{Q}}
        \arrow["{p-1}"', no head, from=2-1, to=3-2]
        \arrow["p", no head, from=2-3, to=3-2]
        \arrow["{p-1}", no head, from=1-2, to=2-3]
        \arrow["p"', no head, from=1-2, to=2-1]
    \end{tikzcd}
    \end{center}
\end{eg}

\begin{eg}
    Consider $(x^4 -1) = (x^2+1)(x^2-1) = (x-i)(x+i)(x-1)(x+1)$, for $x^4-1 \in \Q[x]$. Then $\Q(i)$ is a splitting field for $x^4-1$. 

    Conversely, $(x^4+1) = 0$ has roots $e^{\frac{\pi i}{4}}\zeta_4^k$, for $\zeta_4 = e^{\frac{2\pi i}{4}}$ the principal root of unity. Then we have that \begin{equation*}
        (-1)^{1/4} = \{e^{\frac{\pi i}{4}}, e^{\frac{\pi i}{4}}\zeta_4, e^{\frac{\pi i}{4}}\zeta_4^2, e^{\frac{\pi i}{4}}\zeta_4^3\} 
    \end{equation*}
    Then the splitting field is $\Q(i,\sqrt{2})$, and we have $[\Q(i,\sqrt{2}):\Q] = 4$.
\end{eg}



\begin{prop}
    A splitting field of a polynomial of degree $n$ over $F$ is of degree at most $n!$ over $F$.
\end{prop}
\begin{proof}
    (By induction on degree)
\end{proof}


\subsection{Cyclotomic Fields}

Consider the splitting field of $x^n -1$ over $\Q$. The roots of this are the roots of unity $$z \in (1)^{1/n} = \{1, \zeta_n,\zeta_n^2,...,\zeta_n^{n-1}\}$$ for $\zeta_n = e^{\frac{2\pi i}{n}}$, a primitive generator. Note $\zeta_n^k$ also generates $(1)^{1/n}$ if and only if $\gcd(k,n) = 1$, and $o(\zeta_n^k) = \frac{o(\zeta_n)}{\gcd(o(\zeta_n),k)} = \frac{n}{\gcd(n,k)}$, treating $(1)^{1/n}$ as a group. 

\begin{defn}
    A generator of the cyclic group of all $n$th roots of unity is called a \Emph{primitive $n$th root of unity}.
\end{defn}

Note there are $\varphi(n)$ primitive roots of unity, for $\varphi$ the Euler-totient function. 

\begin{defn}
    The splitting field of $x^n-1$ is $\Q(\zeta_n)$, which is called the \Emph{cyclotomic field of $n$th roots of unity}.
\end{defn}


\begin{thm}
    Let $\varphi:F\xrightarrow{\sim}F'$ be an isomorphism of fields. Let $f(x) \in F[x]$ be a polynomial and let $f'(x) \in F'[x]$ be the polynomial obtained by applying $\varphi$ to the coefficients of $f(x)$. Let $E$ be a splitting field for $f(x)$ over $F$ and let $E'$ be a splitting field for $f'(x)$ over $F'$. Then the isomorphism $\varphi$ extends to an isomorphism $\sigma:E\xrightarrow{\sim}E'$, i.e., restricting $\sigma$ to $F$ is the isomorphsim $\varphi$:
    \begin{center}
        \begin{tikzcd}
            E & {E'} \\
            F & {F'}
            \arrow["\varphi"', from=2-1, to=2-2]
            \arrow["\sigma", from=1-1, to=1-2]
            \arrow[no head, from=1-2, to=2-2]
            \arrow[no head, from=1-1, to=2-1]
        \end{tikzcd}
    \end{center}
\end{thm}
\begin{proof}
    We proceed by induction on the degree of $f(x)$. Let $\Phi$ denote the corresponding isomorphism between $F[x]$ and $F'[x]$ induced by $\varphi$. Then if $f(x)$ corresponds to $f'(x)$ under this isomorphism, then so do the irreducible factors of $f(x)$ and $f'(x)$. 

    If $f(x)$ splits completely in $F$, $f'(x)$ will split completely in $F'$, and vice-versa. Hence, $E = F \cong F' = E'$, taking $\sigma = \varphi$. This shows the result holds for $n = 1$, and when all irreducible factors are linear.

    Assume now by induction that the result holds for any field $F$, isomorphism $\varphi$, and polynomial $f(x) \in F[x]$ of degree less than $n$. Let $p(x)$ be an irreducible factor of $f(x)$ of degree at least $2$, and $p'(x)$ the corresponding irreducible factor of $f'(x)$. If $\alpha \in E$ is a root of $p(x)$ and $\beta \in E'$ is a root of $p'(x)$, then by Theorem \ref{thm:simplextiso} we can extend $\varphi$ to an isomorphism $\sigma':F(\alpha)\xrightarrow{\sim}F'(\beta)$. Then $f(x) = (x-\alpha)f_1(x)$ over $F(\alpha)$ and $f'(x) = (x-\beta)f_1'(x)$ over $F'(\beta)$. The field $E$ is a splitting field for $f_1(x)$ over $F(\alpha)$. Similarly, $E'$ is a splitting field for $f_1'(x)$ over $F'(\beta)$. Since the degrees are less than $n$, we have by the induction hypothesis that there exists an isomorphism $\sigma:E\xrightarrow{\sim}E'$ which restricts to $\sigma'$. Thus, $\sigma$ restricted to $F$ is $\varphi$, as desired.
\end{proof}


\begin{cor}[Uniqueness of Splitting Fields]
    Any two splitting fields for a polynomial $f(x) \in F[x]$ over a field $F$ are isomorphic.
\end{cor}


\begin{defn}
    The field $\overline{F}$ is called an \Emph{algebraic closure} of $F$ if $\overline{F}$ is algebraic over $F$ and if every polynomial $f(x) \in F[x]$ splits completely over $\overline{F}$ (so that $\overline{F}$ can be said to contain all the elements algebraic over $F$).
\end{defn}

\begin{defn}
    A field $K$ is said to be \Emph{algebraically closed} if every polynomial with coefficients in $K$ has a root in $K$.
\end{defn}

\begin{eg}
    $\R$ has algebraic closure $\overline{\R} = \C$, which is also the splitting field $\R(i)$ of $x^2+1 \in \R[x]$.
\end{eg}

\begin{eg}
    $\Q$ has $\overline{\Q}$ algebraic closure. But, $\Q(i)$ is the splitting field for $x^2+1 \in \Q[x]$ is not equal to this.
\end{eg}

\begin{prop}
    Let $\overline{F}$ be an algebraic closure of $F$. Then $\overline{F}$ is algebraically closed.
\end{prop}

\begin{prop}
    For any field $F$, there exists an algebraically closed field $K$ containing $F$.
\end{prop}

\begin{prop}
    Let $K$ be an algebraically closed field and let $F$ be a subfield of $K$. Then the collection of elements of $\overline{F}$ of $K$ that are algebraic over $F$ is an algebraic closure of $F$. An algebraic closure of $F$ is unique up to isomorphism.
\end{prop}


\begin{thm}[Fundamental Theorem of Algebra]
    The field of $\C$ is algebraically closed.
\end{thm}

\begin{cor}
    The field $\C$ contains an algebraic closure for any of its subfields. In particular, $\overline{\Q}$, the collection of complex numbers algebraic over $\Q$, is an algebraic closure of $\Q$.
\end{cor}



\section{\textsection Separable and Inseparable Extensions}

Let $F$ be a field and let $f(x) \in F[x]$. Over a splitting field for $f(x)$ we have a factorization \begin{equation*}
    f(x) (x-\alpha_1)^{n_1}...(x-\alpha_k)^{n_k}
\end{equation*}
where $\alpha_1,...,\alpha_k$ are distinct elements of the splitting field, and $n_i \geq 1$ for all $i$. The integer $n_i$ is called the \Emph{multiplicity} of $\alpha_i$. $\alpha_i$ is called a \Emph{multiple root} if $n_i >1$, and a \Emph{simple root} if $n_i =1$.

\begin{defn}
    A polynomial over $F$ is \Emph{separable} if it has no multiple roots (i.e. all roots are distinct). A polynomial which is not separable is called \Emph{inseparable}.
\end{defn}

\begin{eg}
    $x^2-2 = (x-\sqrt{2})(x+\sqrt{2})$ is separable over $\Q$.
\end{eg}

\begin{eg}
    $x^2 - t \in \F_2(t)[x]$, then there exists $\sqrt{t} \in K/\F_2(t)$ for some extension $K$. Then $$x^2-t = (x-\sqrt{t})(x+\sqrt{t} = (x+\sqrt{t})(x+\sqrt{t}) = (x+\sqrt{t})^2$$ so $x^2-t$ is inseparable over $\F_2(t)$.
\end{eg}


\begin{defn}
    The \Emph{derivative} of the polynomial \begin{equation*}
        f(x) = a_nx^n+...+a_1x+a_0 \in F[x]
    \end{equation*}
    is defined to be the polynomial \begin{equation*}
        D_xf(x) = na_nx^{n-1}+(n-1)a_{n-1}x^{n-2}+...+a_1
    \end{equation*}
\end{defn}


Moreover, this satisfies the identities \begin{equation*}
    D_x(f(x)g(x)) = D_x(f(x))g(x)+f(x)D_x(g(x))
\end{equation*}
and \begin{equation*}
    D_x(f(x)+g(x)) = D_xf(x)+D_xg(x)
\end{equation*}


\begin{prop}\label{prop:seprel}
    A polynomial $f(x)$ has a multiple root $\alpha$ if and only if $\alpha$ is also a root of $D_xf(x)$, i.e., $f(x)$ and $D_xf(x)$ are both divisible by the minimal polynomial for $\alpha$. In particular, $f(x)$ is separable if and only if it is relatively prime to its derivative: $\gcd(f(x),D_xf(x)) = 1$, i.e. $(f(x))+(D_xf(x)) = F[x]$.
\end{prop}
\begin{proof}
    Suppose first that $\alpha$ is a multiple root of $f(x)$. Then over a splitting field $f(x) = (x-\alpha)^ng(x)$ for some integer $n \geq 2$ and some polynomial $g(x)$ in the splitting field. Taking derivatives we obtain \begin{equation*}
        D_xf(x) = n(x-\alpha)^{n-1}g(x) + (x-\alpha)^nD_xg(x)
    \end{equation*}
    which shows that $D_xf(x)$ has $\alpha$ as a root since $n \geq 2$.

    Conversely, suppose $\alpha$ is a root of $D_xf(x)$ and $f(x)$. Write $f(x) = (x-\alpha)h(x)$, and observe that since $D_xf(x) = h(x) + (x-\alpha)D_xh(x)$ and $D_xf(\alpha) = 0$, we have that $h(\alpha) = 0$. Hence, $h(x) = (x-\alpha)h'(x)$ for some polynomial $h'(x)$, so $f(x) = (x-\alpha)^2h'(x)$ and $\alpha$ is a multiple root of $f(x)$ as claimed.
\end{proof}

\begin{eg}
    Consider $x^{p^n}-x \in \F_p$, then its deripative is $p^nx^{p^n-1}-1 = -1$ since the field is of characteristic $p$. Then since the derivative has no roots, $x^{p^n}-x$ has no multiple roots and is therefore separable.
\end{eg}

\begin{cor}
    Every irreducible polynomial over a field of characteristic $0$ is separable (i.e. has no multiple roots). A polynomial over such a field is separable if and only if it is the product of distinct irreducible polynomials.
\end{cor}
\begin{proof}
    Suppose $F$ is a field of characteristic $0$ and $p(x) \in F[x]$ is irreducible of degree $n$. Then the derivative $D_xp(x)$ is a polynomial of degree $n-1$. Up to constant factors the only factors of $p(x)$ in $F[x]$ are $1$ and $p(x)$, so $D_xp(x)$ must be relatively prime to $p(x)$. Then $p(x)$ is separable by Proposition \ref{prop:seprel}. Then, by Proposition \ref{prop:minpoly} all distinct irreducible polynomials have distinct roots, which gives the second result.
\end{proof}

In characteristic $p$, the derivative of any polynomial in $x^p$ is $0$, unlike in the characteristic $0$ case. However, if the derivative is nonzero we can again conclude that the polynomial is separable by the same argument as above.

\begin{prop}
    Let $F$ be a field of characteristic $p$. Then for any $a,b \in F$, \begin{equation*}
        (a+b)^p = a^p+b^p,\; \text{ and }\; (ab)^p = a^pb^p
    \end{equation*}
    Put another way, the $p$th power map defined by $\varphi(a) = a^p$ is an injective field homomorphism from $F$ to $F$.
\end{prop}
\begin{proof}
    The binomial Theorem for expanding $(a+b)^n$ for any positive integer $n$ holds over any commutative ring: \begin{equation*}
        (a+b)^n=\sum_{i=0}^n\begin{pmatrix} n \\ i\end{pmatrix}a^{n-i}b^i
    \end{equation*}
    If $p$ is a prime, then for $i = 1,2,...,p-1$, $\begin{pmatrix} p \\ i\end{pmatrix}$ is divisible by $p$ since for these values of $i$ the numbers $i!$ and $(p-i)!$ only involve factors smaller than $p$, hence are relatively prime to $p$ and so cannot cancel the factor of $p$ in the numerator of the expression $\frac{p!}{i!(p-i)!}$. It follows that over a field of characteristic $p$ all the intermediate terms in the expansion of $(a+b)^p$ are $0$, which gives the first equation of the proposition. The second equation is from commutivity of multiplication in the field, and $\varphi$ is injective since it is a field homomorphism.
\end{proof}


\begin{defn}
    The field homomorphism $\varphi:F\rightarrow F$ defined by $\varphi(a) = a^p$, where $F$ is a field of characteristic $p$, is called the \Emph{Frobenius endomorphism} of $F$.
\end{defn}

\begin{cor}
    Suppose that $F$ is a finite field of characteristic $p$. Then every element of $F$ is a $p$th power in $F$ (Notationally, $F = F^p$).
\end{cor}


Let $F$ be a finite field of characteristic $p$, and $f(x) \in F[x]$ an irreducible polynomial. If $f(x)$ were inseparable, then we would need that $f(x) = q(x^p)$ for some $q(x) \in F[x]$, as otherwise by our previous result $f(x)$ would be separable. Let $q(x) = \sum_{i=0}^na_ix^i$, and note that by our previous result we have $a_i = b_i^p$ for each $i$, so we may write \begin{equation*}
    f(x) = \sum_{i=0}^nb_i^px^{ip} = \left(\sum_{i=0}^nb_ix^i\right)^p
\end{equation*}
where the last equality follows from our endomorphism. But, $f(x)$ is assumed irreducible and we've written it as a product of $p$ non-constant polynomials, which is a contradiction. 

\begin{prop}
    Every irreducible polynomial over a finite field $F$ is separable. A polynomial in $F[x]$ is separable if and only if it is the product of distinct irreducible polynomials in $F[x]$.
\end{prop}

\begin{defn}
    A field $K$ is of characteristic $p$ is called \Emph{perfect} if every element of $K$ is a $p$th power in $K$. Any field of characteristic $0$ is also called perfect.
\end{defn}

From this we see that we've shown that every irreducible polynomial over a perfect field is separable.


\begin{eg}[Existence and Uniqueness of Finite Fields]
    Let $n > 0$ be any positive integer and consider the splitting field of the polynomial $x^{p^n}-x$ over $\F_p$. We know that this polynomial is separable, hence has precisely $p^n$ distinct roots. Let $\alpha$ and $\beta$ be any two roots of this polynomial, so $\alpha^{p^n} = \alpha$ and $\beta^{p^n} = \beta$. Then $(\alpha\beta)^{p^n} = \alpha\beta$, $(\alpha^{-1})^{p^n} = \alpha^{-1}$, and by induction on our endomorphism: \begin{equation*}
        (\alpha+\beta)^{p^n} = \alpha^{p^n}+\beta^{p^n} = \alpha+\beta
    \end{equation*}
    Hence, the set, $F$, of roots of the polynomial is a field, and consequently is the splitting field of the polynomial. Since the number of elements is $p^n$, $[F:\F_p] = n$, which gives the existence of finite fields of degree $n$ over $\F_p$, for all $n > 0$.

    Letting $\F$ be another such finite field of characteristic $p$, if $[\F:\F_p] = n$, then $\F$ has $p^n$ elements. Since the multiplicative group $\F^{\times}$ is of order $p^n-1$, we have $\alpha^{p^n-1} = 1$ for all $\alpha \in \F^{\times}$, so that $\alpha^{p^n} = \alpha$ for every $\alpha \in \F$. Hence, $\F$ is contained in a splitting field for $x^{p^n}-x$, but since we know the splitting field for such a polynomial is of order $p^n$ and $|\F| = p^n$, we have that $\F$ is a splitting field for the polynomial. Since splitting fields are unique up to isomorphism, this shows the existence and uniqueness up to isomorphism of finite fields of order $p^n$.
\end{eg}

\begin{prop}
    Let $p(x)$ be an irreducible polynomial over a field $\F$ of characteristic $p$. Then there is a unique integer $k\geq 0$ and a unique irreducible separable polynomial $p_{sep}(x) \in \F[x]$ such that \begin{equation*}
        p(x) = p_{sep}(x^{p^k})
    \end{equation*}
\end{prop}


\begin{defn}
    Let $p(x)$ be an irreducible polynomial over a field of characteristic $p$. The degree of $p_{sep}(x)$ in the last proposition is called the \Emph{separable degree} of $p(x)$, denoted $\text{deg}_sp(x)$. The integer $p^k$ in the proposition is called the \Emph{inseparable degree} of $p(x)$, denoted $\text{deg}_ip(x)$.
\end{defn}

From this we observe that $p(x)$ is separable if and only if its inseparable degree is $1$ if and only if its separable degree is equal to its actual degree. Computing degrees in the equation $p(x) = p_{sep}(x^{p^k})$ gives \begin{equation*}
    \text{deg}p(x) = \text{deg}_sp(x)\text{deg}_ip(x)
\end{equation*}


\begin{defn}
    The field $K$ is said to be \Emph{separable} (or \Emph{separably algebraic}) over $F$ if every element of $K$ is the root of a separable polynomial over $F$ (equivalently, the minimal polynomial over $F$ of every element of $K$ is separable). A field which is not separable is \Emph{inseparable}.
\end{defn}

The separability of finite extensions of perfect fields is immediate since for these fields the minimal polynomial of an algebraic element is irreducible and hence separable.

\begin{cor}
    Every finite extension of a perfect field is separable. In particular, every finite extension of either $\Q$ or a finite field is separable.
\end{cor}



\section{Cyclotomic Polynomials and Extensions}


The purpose of this section is to prove the cyclotomic extension $\Q(\zeta_n) =\Q$ generated by the $n$th root of unity over $\Q$ is of degree $\varphi(n) = |\{1\leq a < n: \gcd(a,n) = 1\}|$. 

\begin{defn}
    Let $\mu_n$ denote the group of $n$th roots of unity over $\Q$.
\end{defn}


Recall that $\Z/n\Z \cong \mu_n$, with the map $a \mapsto \zeta_n^a$ for a fixed primitive $n$th root of unity. Also recall that the primitive $n$th roots of unity are precisely the residue classes prime to $n$, so there are $\varphi(n)$ primitive $n$th roots of unity.

If $d$ is a divisor of $n$, and $\zeta_d$ is a $d$th root of unity, then $\zeta_d$ is also a $n$th root of unity as $\zeta^n = (\zeta^d)^{n/d}$. Thus, we have that \begin{equation*}
    \mu_d \subseteq \mu_n, \;\;\text{ for all } d\vert n
\end{equation*}

\begin{defn}
    Define the $n$th \Emph{cyclotomic polynomial} $\Phi_n(x)$ to be the polynomial whose roots are the primitive $n$th roots of unity: \begin{equation*}
        \Phi_n(x) = \prod_{\zeta \text{ primitive } \in \mu_n}(x-\zeta) = \prod_{\begin{array}{cc} 1\leq a < n \\ \gcd(a,n)=1\end{array}}(x-\zeta_n^a)
    \end{equation*}
    (which is of degree $\varphi(n)$)
\end{defn}

The roots of $x^n - 1$ are precisely the $n$th roots of unity, so we have the factorization \begin{equation*}
    x^n - 1 = \prod_{\zeta \in \mu_n}(x-\zeta)
\end{equation*}

If we group the factors $(x-\zeta)$ where $\zeta$ is an element of order $d$ in $\mu_n$, then we obtain \begin{equation*}
    x^n - 1 = \prod_{d\vert n}\prod_{\zeta \text{ primitive } \in \mu_d}(x-\zeta)
\end{equation*}
Equivalently, we can write \begin{equation*}
    x^n - 1 = \prod_{d\vert n}\Phi_d(x)
\end{equation*}
Incidentally, comparing degrees we have that \begin{equation*}
    n = \sum_{d\vert n}\varphi(d)
\end{equation*}


\begin{lem}
    The cyclotomic polynomial $\Phi_n(x)$ is a monic polynomial in $\Z[x]$ of degree $\varphi(n)$.
\end{lem}
\begin{proof}
    $\Phi_n(x)$ being a product of monic polynomials is consequently monic. Moreover, by construction the degree of $\Phi_n(x)$ is $\varphi(n)$. To show coefficients lie in $\Z$ we proceed by induction on $n$. The result is immediate for $n = 1$, with $\Phi_1(x) = (x-1)$. Assume by induction that $\Phi_d(x) \in \Z[x]$ for all $1 \leq d < n$. Then $x^n - 1 = f(x)\Phi_n(x)$ where $f(x) = \prod_{d\vert n,d <n}\Phi_d(x)$ is monic and has coefficients in $\Z$ by the induction hypothesis. Since $f(x)$ divides $x^n -1$ in $F[x]$ where $F = \Q(\zeta_n)$ is the field of $n$th roots of unity and both $f(x)$ and $x^n-1$ have coefficients in $\Q$, $f(x)$ divides $x^n-1$ in $\Q[x]$ by the division algorithm. By Gauss' Lemma, $f(x)$ divides $x^n-1$ in $\Z[x]$, hence $\Phi_n(x) \in \Z[x]$.
\end{proof}


\begin{thm}
    The cyclotomic polynomial $\Phi_n(x)$ is an irreducible monic polynomial in $\Z[x]$ of degree $\varphi(n)$.
\end{thm}
\begin{proof}
    Suppose towards a contradiction that $\Phi_n(x) = f(x)g(x)$, with $f(x),g(x) \in \Z[x]$ monic, where we take $f(x)$ to be an irreducible factor of $\Phi_n(x)$. Let $\zeta$ be a primitive $n$th root of unity which is a root of $f(x)$ (so then $f(x)$ is the minimal polynomial for $\zeta$ over $\Q$) and let $p$ denote any prime not dividng $n$. Then $\zeta^p$ is again a primitive $n$th root of unity, hence is a root of either $f(x)$ or $g(x)$.


    Suppose $g(\zeta^p) = 0$. Then $\zeta$ is a root of $g(x^p)$ and since $f(x)$ is the minimal polynomial for $\zeta$, $f(x)$ must divide $g(x^p)$ in $\Z[x]$, say $g(x^p) = f(x)h(x)$ for some $h(x) \in \Z[x]$. If we reduce this equation mod $p$, we obtain $\overline{g}(x^p) = \overline{f}(x)\overline{h}(x)$ in $\F_p[x]$. By the Frobenius endomorphism for finite fields we have that $\overline{g}(x^p) = (\overline{g}(x))^p$, so we have the equation $(\overline{g}(x))^p = \overline{f}(x)\overline{h}(x)$ in the U.F.D $\F_p[x]$. It follows that $\overline{f}(x)$ and $\overline{g}(x)$ have a factor in common in $\F_p[x]$. 

    Now, from $\Phi_n(x) = f(x)g(x)$ we see by reducing mod $p$ that $\overline{\Phi}_n(x) = \overline{f}(x)\overline{g}(x)$, and so by the above it follows that $\overline{\Phi}_n(x) \in \F_p[x]$ has a multiple root. But then also $x^n-1$ would have a multiple root over $\F_p$ since it has $\overline{\Phi}_n(x)$ as a factor. This is a contradiction since we have seen in the last section that there are $n$ distinct roots of $x^n-1$ over any field of characteristic not dividing $n$. 

    Hence $\zeta^p$ must be a root of $f(x)$. Since this applies to every root $\zeta$ of $f(x)$, it follows that $\zeta^a$ is a root of $f(x)$ for every integer $a$ relatively prime to $n$: write $a = p_1p_2...p_k$ as a product of not necessarily distinct primes not dividing $n$ so that $\zeta^{p_1}$ is a root of $f(x)$, so also $(\zeta^{p_1})^{p_2}$ is a root of $f(x)$, etc. But this implies that every primitive $n$th root of unity is a root of $f(x)$, which is to say $f(x) = \Phi_n(x)$, showing $\Phi_n(x)$ is irreducible.
\end{proof}



\begin{cor}
    The degree over $\Q$ of the cyclotomic field of $n$th roots of unity is $\varphi(n)$: \begin{equation*}
        [\Q(\zeta_n):\Q] = \varphi(n)
    \end{equation*}
\end{cor}


