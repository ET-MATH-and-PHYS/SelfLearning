%%%%%%%%%% Galois Theory %%%%%%%%%%
\chapter{\textsection\textsection Galois Theory}


\section{\textsection Basics Definitions and Examples: Galois Theory}


\begin{defn}
    An isomorphism of $K\xrightarrow{\sim} K$ is called an \Emph{automorphism} of $K$. The collection of all automorphisms is denoted $\aut(K)$. We say a $\sigma \in \aut(K)$ fixes $a \in K$ if $\sigma(a) = a$. If $F$ is a subset of $K$, then an automorphism $\sigma \in \aut(K)$ is said to fix $F$ if it fixes all the elements of $F$, i.e., $\sigma(a) =a$ for all $a \in F$.
\end{defn}

Note that all fields have at least one automorphism, namely the identity automorphism, or trivial automorphism. The prime subfield of $K$ is generated by $1 \in K$, and since any automorphism $\sigma$ takes $1$ to $1$ (and $0$ to $0$), i.e., $\sigma(1) = 1$, it follows that $\sigma(a) = a$ for all $a$ in the prime subfield. In particular, we see that $\Q$ and $\F_p$ only have the trivial automorphism.

\begin{defn}
    Let $K/F$ be an extension of fields. Let $\aut(K/F)$ denote the collection of automorphisms of $K$ which fix $F$; $\aut(K/F) = \{\sigma \in \aut(K):\sigma(x) = x,\forall x \in F\}$.
\end{defn}

\begin{eg}
    For $K$ a field of characteristic zero, $\aut(K/\Q) = \aut(K)$, and for $K$ a field of prime characteristic $p$, $\aut(K/\F_p) = \aut(K)$.
\end{eg}

\begin{prop}
    $\aut(K)$ is a group under composition. Moreover, $\aut(K/F)$ is a subgroup of that group.
\end{prop}
\begin{proof}
    Note that the composition of bijections is a bijection and the composition of algebraic structure preserving maps still preserves that structure so $\aut(K)$ is indeed closed under the binary operation of composition. Moreover, composition is associative, and the inverse of a field isomorphism is again a field isomorphism. Finally, every field has the identity isomorphism which acts as an identity under composition, so $\aut(K)$ is indeed a group. First, we note that $1_K \in \aut(K/F)$ as it fixes $F$. If $\sigma, \tau \in \aut(K/F)$ and $x \in F$, then $\sigma\circ \tau(x) = \sigma(x) = x$, so $\sigma \circ \tau \in \aut(K/F)$. Moreover, if $\sigma \in \aut(K/F)$, then $\sigma^{-1}(x) = \sigma^{-1}(\sigma(x)) = x$, so $\sigma^{-1} \in \aut(K/F)$. Thus, $\aut(K/F)$ is a subgroup of $\aut(K)$ as desired.
\end{proof}


\begin{prop}
    Let $K/F$ be a field extension and $\alpha \in K$ algebraic over $F$. Then for any $\sigma \in \aut(K/F)$, $\sigma(\alpha)$ is a root of the minimal polynomial for $\alpha$ over $F$ i.e., $\aut(K/F)$ permutes roots of irreducible polynomials in $F$. Equivalently, any polynomial with coefficients in $F$ having $\alpha$ as a root also has $\sigma(\alpha)$ as a root.
\end{prop}
\begin{proof}
    Suppose that $\alpha \in K$ satisfies the equation \begin{equation*}
        \alpha^n+a_{n-1}\alpha^{n-1}+...+a_1\alpha+a_0 = 0
    \end{equation*}
    which is minimal, for $a_i \in F$. Let $\sigma \in \aut(K/F)$, and we act on the above: \begin{equation*}
        \sigma(\alpha)^n + a_{n-1}\sigma(\alpha)^{n-1}+...+a_1\sigma(\alpha)+a_0 = 0
    \end{equation*}
    since elements of $F$ are fixed, and field homomorphisms are additive and multiplicative. But, this precisely says that $\sigma(\alpha)$ is a root of the same polynomial over $F$ in $K$ as $\alpha$.
\end{proof}

\begin{eg}
    Consider $K = \Q(\sqrt{2})$, and let $\tau \in \aut(K/\Q) = \aut(\Q(\sqrt{2}))$. Then $\tau$ is either given by $\tau(\sqrt{2}) = \sqrt{2}$, which gives $\tau = 1_K$, or $\tau(\sqrt{2}) = -\sqrt{2}$, extended algebraically. Thus, $\aut(K/\Q) \cong \Z/2\Z$.
\end{eg}

\begin{eg}
    Consider $K = \Q(\sqrt[3]{2})$. Then, for $\tau \in \aut(K/\Q)$, we have \begin{equation*}
        \tau(a+b\sqrt[3]{2}+c\sqrt[3]{2}^2) = a+b\tau(\sqrt[3]{2})+c\tau(\sqrt[3]{2})^2
    \end{equation*}
    so $\tau$ is determined by where it sends $\sqrt[3]{2}$. This must again be a root of $x^3 - 2$. But, the other roots are in $\C$, and hence not in $K$, so $\tau$ can only send $\sqrt[3]{2}$ to itself and $\aut(K/\Q) = \{1_K\}$.
\end{eg}

If $K$ is generated over $F$ by some collection of elements, then any automorphism $\sigma \in \aut(K/F)$ is completely determined by what it does to the generators. If $K/F$ is finite then $K$ is finitely generated over $F$ by algebraic elements so by the proposition the number of automorphisms of $K$ fixing $F$ is finite. In particular, the automorphisms of a finite extension can be considered as permutations of the roots of a finite number of equations. 


We have associated to each field extension $K/F$ a group, $\aut(K/F)$, the group of automorphisms of $K$ which fix $F$. One can also reverse this process: 

\begin{prop}
    Let $H \leq \aut(K)$ be a subgroup of the group automorphisms of $K$. Then the collection $F$ of elements of $K$ fixed by all elements of $H$ is a subfield of $K$.
\end{prop}
\begin{proof}
    Let $F$ be the fixed collection under $H$. Then since all automorphisms fix $0$ and $1$, we have that $0,1 \in F$. Let $a,b \in F$, $b \neq 0$. Then observe that for any $\sigma \in H$, $\sigma(a-b) = \sigma(a)-\sigma(b) = a-b$ by additivity of automorphisms, so $a-b \in F$. Moroever, $\sigma(ab^{-1}) = \sigma(a)\sigma(b)^{-1} = ab^{-1}$, so $ab^{-1} \in F$. Therefore, $F$ is closed under subtraction and division so by the subfield test $F$ is a subfield of $K$.
\end{proof}

Note that it is not important in this proposition that $H$ be a subgroup. Indeed, the subcollection of $K$ fixed by any subset of $\aut(K)$ is again a subfield of $K$.

\begin{defn}
    If $H$ is a subgroup of the group of automorphisms of $K$, the subfield of $K$ fixed by all elements of $H$ is called the \Emph{fixed field of $H$}.
\end{defn}

\subsection{The Galois Correspondence}

Let $L/K$ be a field extension, and consider $\sigma:L\rightarrow L$ such that $\sigma(k) = k$ for all $k \in K$, so that $\sigma \in \aut(L/K)$. 

\begin{eg}
    Consider $\sigma:\C\rightarrow \C$ by $\sigma(z) = \overline{z}$. Then $\sigma \in \aut(\C)$, and moreover, $\sigma(x+i0) = x-i0 = x$, so $\sigma \in \aut(\C/\R)$. 
\end{eg}

\begin{defn}
    For any subgroup $H$ of $\aut(L/K)$, then the fixed field of $H$ is denoted by $$L^H = \{\alpha \in L:\forall \sigma \in H,\sigma(\alpha) = \alpha\}$$
\end{defn}

We're now studying the following correspondence for a given $L/K$:\begin{equation*}
    \{K\subseteq F\subseteq L:F\rightsquigarrow \aut(L/F)\}\;\;\text{ and }\;\;\{H\leq \aut(L):H\rightsquigarrow L^H\}
\end{equation*}
In particular, we have such diagrams as 
\begin{center}
    \begin{tikzcd}
	L & {\{id\}} \\
	F & {\text{Aut}(L/F)} \\
	{F'} & {\text{Aut}(L/F')} \\
	K & {\text{Aut}(L/K)}
	\arrow[no head, from=1-1, to=2-1]
	\arrow[""{name=0, anchor=center, inner sep=0}, no head, from=2-1, to=3-1]
	\arrow[no head, from=3-1, to=4-1]
	\arrow[""{name=1, anchor=center, inner sep=0}, no head, from=3-2, to=2-2]
	\arrow[no head, from=2-2, to=1-2]
	\arrow[no head, from=4-2, to=3-2]
    \end{tikzcd}
\end{center}

\begin{prop}
    The association of groups to fields and fields to groups defined above is inclusion reversing, namely: \begin{itemize}
        \item if $F_1\subseteq F_2\subseteq L$ are two subfields of $K$ then $\aut(L/F_2) \leq \aut(L/F_1)$, and
        \item if $H_1 \leq H_2 \leq \aut(L)$ are two subgroups of automorphisms with associated fixed fields $L^{H_1}$ and $L^{H_2}$, $L^{H_2} \subseteq L^{H_1}$.
    \end{itemize}
\end{prop}
\begin{proof}
    First suppose $F_1 \subseteq F_2 \subseteq L$ are subfields. Take $\sigma \in \aut(L/F_2)$. Then let $x \in F_1$. It follows that $x \in F_2$, so $\sigma(x) = x$ and hence $\sigma \in \aut(L/F_1)$. Thus, $\aut(L/F_2) \leq \aut(L/F_2)$.

    Next, let $H_1 \leq H_2 \leq \aut(L)$ be subgroups. Let $x \in L^{H_2}$. Then let $\sigma \in H_1$, so $\sigma \in H_2$. Then $\sigma(x) = x$, and consequently we find that $x \in L^{H_1}$ by definiton. Therefore, $L^{H_2} \subseteq L^{H_1}$.
\end{proof}

We observe that $L^{\aut(L/K)} \supseteq K$. Equality does not always hold. Indeed:

\begin{eg}
    $\Q(\sqrt[3]{2})/\Q$ has $[\Q(\sqrt[3]{2}):\Q] = 3$, so there are no proper intermediate fields. Moreover, $\aut(\Q(\sqrt[3]{2})/\Q) = \{\id\}$, so $\aut(\Q(\sqrt[3]{2})/\Q) = \aut(\Q(\sqrt[3]{2}))$, so $\Q(\sqrt[3]{2})^{\aut(\Q(\sqrt[3]{2}))} = \Q(\sqrt[3]{2})$.
\end{eg}

\begin{defn}
    The roots of a common irreducible polynomial in $K[x]$ are called \Emph{K-conjugates}.
\end{defn}

\begin{eg}
    $\pm\sqrt{2}$ are $\Q$-conjugates as they share the minimal polynomial $x^2 - 2$ in $\Q[x]$. But, they are not $\R$-conjugates since $x^2-2$ is not irreducible over $\R$ since $x - \sqrt{2}$ is irreducible with root only $\sqrt{2}$, not $-\sqrt{2}$.
\end{eg}

Thus, for $\alpha$ a root of $f(x) \in K[x]$, we have that for any $\sigma \in \aut(L/K)$, $\sigma(\alpha)$ and $\alpha$ are $K$-conjugates.

The goal of this section is to bound the size of $L/K$:

\begin{thm}
    For any finite extension $L/K$, the group $\aut(L/K)$ is finite.
\end{thm}
\begin{proof}
    Let $L = K(\alpha_1,\alpha_2,...,\alpha_n)$. Then for any $\sigma \in \aut(L/K)$ is given by $\sigma(\alpha_1),\sigma(\alpha_2),...,\sigma(\alpha_n)$, with each root associated to a polynomial of finite degree. Then for each $\alpha_i$, there are a finite number of choices for where it can be sent since it must be sent to another root of its minimal polynomial in $K[x]$. Hence, each $\sigma$ is determined by a finite choice, and hence $\aut(L/K)$ itself is finite.
\end{proof}


\begin{thm}\label{thm:numbfieldExt}
    Let $\sigma:K\rightarrow K'$ be an isomorphism of fields, $f(X) \in K[X]$, $L$ a splitting field of $f(X)$ over $K$ and $L'$ be a splitting field of $(\sigma f)(X)$ over $K'$. Then $[L:K] = [L':K']$, and $\sigma$ extends to an isomorphism $L\rightarrow L'$, and the number of such extensions is at most $[L:K]$.
\end{thm}


\begin{cor}
    If $L$ is a splitting field over $K$ of a polynomial $f(X) \in K[X]$, then $|\aut(L/K)| \leq [L:K]$.
\end{cor}
\begin{proof}
    Apply Theorem \ref{thm:numbfieldExt} with $K' = K$, $L' = L$, and $\sigma = \id:K\rightarrow K$. Extensions of the identity isomorphism on $K$ to automorphisms of $L$ are precisely the elements of $\aut(L/K)$. 
\end{proof}

\begin{thm}
    Let $\sigma:K\rightarrow K'$ be an isomorphism of fields, $f(X) \in K[X]$, $L$ be a splitting field of $f(X)$ over $K$ and $L'$ be a splitting field of $(\sigma f)(x)$ over $K'$. If $f(X)$ is separable then there are $[L:K]$ extensions of $\sigma$ to an isomorphism $L\rightarrow L'$.
\end{thm}
\begin{proof}
    We may assume $[L:K] > 1$. Let $\alpha$ be a root of $f(X)$ that is not in $K$, and let $\pi(X)$ be its minimal polynomial over $K$, and $d = \deg(\pi(X))$. Because $f(X)$ is separable, $(\sigma f)(X)$ is separable too. One way to show this is with the characterization of separability in terms of relative primality to the derivative: we can write \begin{equation*}
        f(X)u(X) + f'(X)v(X) = 1
    \end{equation*}
    for some $u(X)$ and $v(X)$ in $K[X]$. Applying $\sigma$ to coefficients commutes with forming derivatives (i.e. $\sigma(f') = (\sigma f)'$), so if we apply $\sigma$ to coefficients in our expression to get \begin{equation*}
        (\sigma f)(X)(\sigma u)(X)+(\sigma f)'(X)(\sigma v)(X) = 1
    \end{equation*}
    so $(\sigma f)(X)$ and its derivative are relatively prime in $K'[X]$. This last polynomial identity proves $(\sigma f)(X)$ is separable. Every factor of a separable polynomial is separable, so $(\sigma \pi)(X)$ and therefore has $d$ roots in $L'$ since it splits completely over $L'$.

    (To be continued-relies on proof of \ref{thm:numbfieldExt})
\end{proof}



\begin{cor}\label{cor:galsep}
    If $L/K$ is a splitting field of a separable polynomial then $|\aut(L/K)| = [L:K]$.
\end{cor}


\begin{thm}
    For a finite extension $L/K$, the following are equivalent characterizations of Galois extensions: \begin{itemize}
        \item $|\aut(L/K)| = [L:K]$
        \item $L^{\aut(L/K)} = K$
        \item $L/K$ is separable and normal
        \item $L$ is a splitting field over $K$ of a separable polynomial in $K[x]$.
    \end{itemize}
\end{thm}


\begin{defn}
    Let $K/F$ be a finite extension. Then $K$ is said to be \Emph{Galois} over $F$ and $K/F$ is a \Emph{Galois extension} if $|\aut(K/F)| = [K:F]$. If $K/F$ is Galois the group of automorphisms $\aut(K/F)$, is called the \Emph{Galois group} of $K/F$, denoted $\aut(K/F) = \text{Gal}(K/F)$. 
\end{defn}

In this case $\aut(K/F)$ has the maximal number of possible automorphisms. 


\begin{defn}
    If $f(x)$ is a separable polynomial over $F$, then the \Emph{Galois group of $f(x)$ over $F$} is the Galois group of the splitting field of $f(x)$ over $F$.
\end{defn}


\begin{eg}
    Consider $\Q(\sqrt{2})/\Q$, the splitting field of $x^2 - 2 \in \Q[x]$. This is separable. Then $\aut(\Q(\sqrt{2})/\Q)$ is Galois with $|\aut(\Q(\sqrt{2})/\Q)| = 2$, with $\aut(\Q(\sqrt{2})/\Q) = \{\id, \sigma\}$.
\end{eg}

\begin{eg}
    Any quadratic extension of any field $F$ of characteristic different from $2$ is Galois, $K = F(\sqrt{D})$. This holds as it is the splitting field of $x^2-D$ in $F$, for $\sqrt{D}\neq-\sqrt{D} \in K$.
\end{eg}

\begin{eg}
    For $\Q(\sqrt{2},\sqrt{3})$ over $\Q$, it is the splitting field of $(x^2-2)(x^2-3)$, which is separable. Thus, $\Q(\sqrt{2},\sqrt{3})$ is Galois. It's elements are \begin{equation*}
        \sigma: \begin{array}{c} \sqrt{2}\mapsto -\sqrt{2} \\ \sqrt{3}\mapsto \sqrt{3}\end{array},\;\;\tau: \begin{array}{c} \sqrt{2}\mapsto \sqrt{2} \\ \sqrt{3}\mapsto -\sqrt{3}\end{array}
    \end{equation*}
    and \begin{equation*}
        \sigma\tau: \begin{array}{c} \sqrt{2}\mapsto -\sqrt{2} \\ \sqrt{3}\mapsto -\sqrt{3}\end{array},\;\;\id: \begin{array}{c} \sqrt{2}\mapsto \sqrt{2} \\ \sqrt{3}\mapsto \sqrt{3}\end{array}
    \end{equation*}
    This is in fact the Klien-4 group, as indeed since $\Q(\sqrt{2},\sqrt{3})$ is Galois and $[\Q(\sqrt{2},\sqrt{3}):\Q] = 4$, so $\text{Gal}(\Q(\sqrt{2},\sqrt{3})/\Q) = \{\id, \sigma, \tau, \sigma\tau\} \cong K_4 \cong \Z/2\Z\times\Z/2\Z$. 
    \begin{table}[H]
        \centering
        \begin{tabular}{c|c}
            Subgroup & Fixed \\ \hline
            $\{\id\}$ & $\Q(\sqrt{2},\sqrt{3})$ \\ 
            $\{\id,\sigma\}$ & $\Q(\sqrt{3})$ \\ 
            $\{\id,\tau\}$ & $\Q(\sqrt{2})$ \\
            $\{\id,\sigma\tau\}$ & $\Q(\sqrt{6})$ \\
            $\{1,\sigma,\tau,\sigma\tau\}$ & $\Q$ \\
        \end{tabular}
    \end{table}
    So we can draw:
    \begin{center}
        \begin{tikzcd}
	& {\{\text{id}\}} &&& {\mathbb{Q}(\sqrt{2},\sqrt{3})} \\
	{\{\text{id}, \tau\}} & {\{\text{id},\sigma\}} & {\{\text{id},\sigma\tau\}} & {\mathbb{Q}(\sqrt{2})} & {\mathbb{Q}(\sqrt{3})} & {\mathbb{Q}(\sqrt{6})} \\
	& {\{\text{id},\tau,\sigma,\sigma\tau\}} &&& {\mathbb{Q}}
	\arrow[no head, from=2-1, to=1-2]
	\arrow[no head, from=2-2, to=1-2]
	\arrow[no head, from=1-2, to=2-3]
	\arrow[no head, from=2-1, to=3-2]
	\arrow[no head, from=2-2, to=3-2]
	\arrow[no head, from=2-3, to=3-2]
	\arrow[no head, from=2-4, to=3-5]
	\arrow[no head, from=2-5, to=3-5]
	\arrow[no head, from=2-6, to=3-5]
	\arrow[no head, from=2-4, to=1-5]
	\arrow[no head, from=1-5, to=2-5]
	\arrow[no head, from=1-5, to=2-6]
\end{tikzcd}
    \end{center}
\end{eg}


\begin{eg}
    The field $\Q(\sqrt[4]{2})$ is not Galois over $\Q$ since any automorphism is determined by where it sends $\sqrt[4]{2}$ by where it sends $\sqrt[4]{2}$ and of the four possibilities, only two elements are real and hence in the field. Note \begin{equation*}
        \Q \subset \Q(\sqrt{2}) \subset \Q(\sqrt[4]{2})
    \end{equation*}
    So $[\Q(\sqrt[4]{2}):\Q] = 4$, but $\aut(\Q(\sqrt[4]{2})/\Q) = \{\id, \tau\}$. However, both $\Q(\sqrt{2})/\Q$ and $\Q(\sqrt[4]{2})/\Q(\sqrt{2})$ are in fact Galois, so the composition of Galois extensions is not necessarily Galois.
\end{eg}

\begin{eg}
    Consider $\F_{p^n}/\F_p$ for prime $p$. Recall $x^{p^n}-x$ is separable, and $\F_{p^n}$ is the splitting field for $x^{p^n}-x$, so it is a Galois extension. Consider the Frobenius automorphism $\sigma:\F_{p^n}\rightarrow \F_{p^n}$, with $\sigma(\alpha) = \alpha^p$. Then $\sigma^2(\alpha) = \alpha^{p^2}$, and in general $\sigma^j(\alpha) = \alpha^{p^j}$ for $j = 1,2,...,n-1$, and $\sigma^n(\alpha) = \alpha^{p^n} = \alpha$, so $\sigma^i\sigma^j = \sigma^{i+j}$ modulo $n$. Then the Frobenius automorphism is of order $n$, so as $[\F_{p^n}:\F_p] = n$ the Frobenius automorphism is a generator for the Galois group $\text{Gal}(\F_{p^n}/\F_p)$.
\end{eg}







\section{\textsection The Fundamental Theorem of Galois Theory}

\begin{defn}
    A \Emph{character} $\chi$ of a group $G$ with values in a field $L$ is a homomorphism $\chi:G\rightarrow L^{\times}$. In particular, for $g_1,g_2 \in G$, \begin{equation*}
        \chi(g_1g_2) = \chi(g_1)\chi(g_2)
    \end{equation*}
    and $\chi(g) \neq 0$ for all $g \in \G$.
\end{defn}

\begin{defn}
    If $\chi_1,\chi_2,...,\chi_n$ are characters of $G$ with values in $L$, then they're linearly independent over $L$ provided there does not exist $a_1,a_2,...,a_n \in L$, not all zero, such that \begin{equation*}
        a_1\chi_1+a_2\chi_2+...+a_n\chi_n = 0
    \end{equation*}
    (as a function, $a_1\chi_1(g)+...+a_n\chi_n(g) = 0$ for all $g \in G$)
\end{defn}


\begin{thm}
    If $\chi_1,\chi_2,...,\chi_n$ are distinct characters of $G$ with values in $L$, then they are linearly independent over $L$.
\end{thm}
\begin{proof}
    Towards a contradiction suppose the characters were linearly dependent. Among all of the linear dependence relations, choose one with minimal number $m$ of non zero coefficients $a_i$. With possible renumbering we may write $a_1\chi_1+...+a_m\chi_m = 0$. In particular, for any $g \in G$, $a_1\chi_1(g) + ... + a_m\chi_m(g) = 0$. Let $g_0 \in G$ such that $\chi_1(g_0) \neq \chi_m(g_0)$, which is possible as the characters are distinct. Then we have $a_1\chi_1(g_0g)+...+a_m\chi_m(g_0g) = 0$. Multiplying our previous equation by $\chi_m(g_0)$ and subtracting it from this one we obtain \begin{equation*}
        [\chi_1(g_0) - \chi_m(g_0)]a_1\chi_1(g)+[\chi_2(g_0)-\chi_m(g_0)]a_2\chi_2(g)+...+[\chi_{m-1}(g_0)-\chi_m(g_0)]a_{m-1}\chi_{m-1}(g) = 0
    \end{equation*}
    But, since $\chi_1(g_0) - \chi_m(g_0) \neq 0$, this is a linear dependence of length less than or equal to $m-1 < m$, contradicting the minimality of $m$.
\end{proof}
    

Consider an injective homomorphism $\sigma$ of a field $K$ into a field $L$, called an \Emph{embedding} of $K$ into $L$. Then in particular $\sigma$ restricts to a homomorphism of the multiplicative group $K^{\times}$ into $L^{\times}$. Note also that this character contains all of the useful information about the values of $\sigma$ viewed simply as a function on $K$.

\begin{cor}
    If $\sigma_1,\sigma_2,...,\sigma_n$ are distinct embeddings of a field $K$ into a field $L$, then they are linearly independent as functions on $K$. In particular, distinct automorphisms of a field $K$ are linearly independent as functions on $K$.
\end{cor}
This follows from our last theorem, and the discussion above about restricting embeddings to group characters.

\begin{thm}\label{thm:subfixed}
    Let $G=\{\sigma_1 = 1,\sigma_2,...,\sigma_n\}$ be a subgroup of automorphisms of a field $K$ and let $F$ be the associated fixed field. Then \begin{equation*}
        [K:F] = n = |G|
    \end{equation*}
\end{thm}
\begin{proof}
    Suppose first that $n > [K:F]$ and let $\omega_1,...,\omega_m$ be a basis for $K$ over $F$ ($m < n$). Then the system \begin{align*}
        \sigma_1(\omega_1)x_1 + \sigma_2(\omega_1)x_2&+...+\sigma_n(\omega_1)x_n = 0 \\
        \vdots& \\
        \sigma_1(\omega_m)x_1 + \sigma_2(\omega_m)x_2&+...+\sigma_n(\omega_m)x_n = 0 
    \end{align*}
    of $m$ equations in $n$ unknowns has a nontrivial kernel, and hence solution $\beta_1,...,\beta_n$ in $K$ since $m < n$. 

    Let $a_1,...,a_m$ be $m$ arbitrary elements of $F$. The field is by definition fixed by the $\sigma_i$'s, so each of these elements is fixed by every $\sigma_i$, i.e. $\sigma_i(a_j) = a_j$. Multiplying the $i$th equation by $a_i$ we have
    \begin{align*}
        \sigma_1(a_1\omega_1)x_1 + \sigma_2(a_2\omega_1)x_2&+...+\sigma_n(a_1\omega_1)x_n = 0 \\
        \vdots& \\
        \sigma_1(a_m\omega_m)x_1 + \sigma_2(a_m\omega_m)x_2&+...+\sigma_n(a_m\omega_m)x_n = 0 
    \end{align*}
    Adding these equations and substituting $\beta_i$ for $x_i$ we obtain \begin{equation*}
        \sigma_1(a_1\omega_1+...+a_m\omega_m)\beta_1+...+\sigma_n(a_1\omega_1+...+a_m\omega_m)\beta_n = 0
    \end{equation*}
    for all choices of $a_i \in F$. But since the $\omega_i$ form a basis for $K$ over $F$, this consequently holds for all elements of $K$. However, this implies that this is a linear dependence on the set $\sigma_1,...,\sigma_n$ of automorphisms, contradicting the linear independence of the previous corollary. Thus, $n \leq [K:F]$. 

    Now, suppose $n < [K:F]$. Then there are more than $n$ $F$-linearly independent elements of $K$, say $\alpha_1,...,\alpha_{n+1}$. Then the system 
    \begin{align*}
        \sigma_1(\alpha_1)x_1 + \sigma_1(\alpha_2)x_2&+...+\sigma_1(\alpha_{n+1})x_{n+1} = 0 \\
        \vdots& \\
        \sigma_n(\alpha_1)x_1 + \sigma_n(\alpha_2)x_2&+...+\sigma_n(\alpha_{n+1})x_{n+1} = 0 
    \end{align*}
    of $n$ equations in $n+1$ unknowns has a non trivial solution $\beta_1,...,\beta_{n+1}$ in $K$. If all the $\beta_i$ were elements of $F$, then the first equation would contradict the linear independence of the $\alpha_j$, recalling that $\sigma_1 = 1$. Hence, at least $1$ $\beta_i$ is not in $F$.

    Choose a solution with a minimal number of nonzero $\beta_i$, $r$. By renumbering if necessary write $\beta_1,...,\beta_r$ nonzero. Dividing by $\beta_r$ we may also assume that $\beta_r = 1$. We have already seen that at least one of $\beta_1,...,\beta_{r-1},1$ is not in $F$. With possible reordering, suppose $\beta_1 \notin F$. Then our system reads
    \begin{align*}
        \sigma_1(\alpha_1)\beta_1 + \sigma_1(\alpha_2)\beta_2&+...+\sigma_1(\alpha_{r-1})\beta_{r-1}+\sigma_1(\alpha_r) = 0 \\
        \vdots& \\
        \sigma_n(\alpha_1)\beta_1 + \sigma_n(\alpha_2)\beta_2&+...+\sigma_n(\alpha_{r-1})\beta_{r-1}+\sigma_n(\alpha_r)= 0 
    \end{align*}
    so $\sigma_i(\alpha_1)\beta_1 + \sigma_i(\alpha_2)\beta_2+...+\sigma_i(\alpha_{r-1})\beta_{r-1}+\sigma_i(\alpha_r) = 0$ for each $i$. Since $\beta_1 \notin F$, and $F$ is the fixed field of the subgroup of automorphisms, there exists $\sigma_{k_0}$ such that $\sigma_{k_0}\beta_1 \neq \beta_1$ for some $k_0 \in \{1,2,...,n\}$. If we apply the automorphism to the previous equation we obtain \begin{equation*}
        \sigma_{k_0}\sigma_j(\alpha_1)\sigma_{k_0}\beta_1 +...+\sigma_{k_0}\sigma_j(\alpha_{r-1})\sigma_{k_0}\beta_{r-1}+\sigma_{k_0}\sigma_j(\alpha_r) = 0
    \end{equation*}
    But the action of left multiplication of a group element on the group itself simply permutes the elements, so $\sigma_{k_0}\sigma_1,...,\sigma_{k_0}\sigma_n$ equals $\sigma_1,...,\sigma_n$ with some reordering. Let $\sigma_{k_0}\sigma_j = \sigma_i$. Then we have \begin{equation*}
        \sigma_i(\alpha_1)\sigma_{k_0}\beta_1 +...+\sigma_i(\alpha_{r-1})\sigma_{k_0}\beta_{r-1}+\sigma_i(\alpha_r) = 0
    \end{equation*}
    Subtracting our original linear dependence from this equation we have \begin{equation*}
        \sigma_i(\alpha_1)[\sigma_{k_0}\beta_1-\beta_1] +...+\sigma_i(\alpha_{r-1})[\sigma_{k_0}\beta_{r-1}-\beta_{r-1}]= 0
    \end{equation*}
    But $[\sigma_{k_0}\beta_1 - \beta_1] \neq 0$, so this is a linear dependence with fewer than $r$ nonzero elements, contradicting the minimality of $r$.
\end{proof}


\begin{cor}\label{cor:galifffixed}
    Let $K/F$ be any finite extension. Then \begin{equation*}
        |\aut(K/F)| \leq [K:F]
    \end{equation*}
    with equality if and only if $F$ is the fixed field of $\aut(K/F)$. Put another way, $K/F$ is Galois if and only if $F$ is the fixed field of $\aut(K/F)$.
\end{cor}
\begin{proof}
    Let $F_1$ be the fixed field of $\aut(K/F)$, so that $F \subseteq F_1 \subseteq K$. By our previous theorem $[K:F_1] = |\aut(K/F)|$. Hence, $[K:F] = [K:F_1][F_1:F]$, which proves the corollary.
\end{proof}


\begin{cor}\label{cor:subaut}
    Let $G$ be a finite subgroup of automorphisms of a field $K$ and let $F$ be the fixed field. Then every automorphism of $K$ fixing $F$ is contained in $G$, i.e., $\aut(K/F) = G$, so that $K/F$ is Galois, with Galois group $G$.
\end{cor}
\begin{proof}
    By definition $F$ is fixed by all elements of $G$, so we have $G \leq \aut(K/F)$. Hence $|G| \leq |\aut(K/F)|$. By the theorem we have $|G| = [K:F]$ and by the previous corollary we have $|\aut(K/F)| \leq [K:F]$. This gives \begin{equation*}
        |G| \leq |\aut(K/F)| \leq |G|
    \end{equation*}
    so $|\aut(K/F)| = |G|$, and since $G \leq \aut(K/F)$, $G = \aut(K/F)$. 
\end{proof}

\begin{cor}\label{cor:distffs}
    If $G_1\neq G_2$ are distinct finite subgroups of automorphisms of a field $K$, then their fixed fields are distinct.
\end{cor}
\begin{proof}
    Suppose $F_1$ is the fixed field of $G_1$ and $F_2$ is the fixed field of $G_2$. Then suppose $F_1 = F_2$, so by definition $F_1$ is fixed by $G_2$. By the previous corollary, any automorphism fixing $F_1$ is in $G_1$, so $G_2 \leq G_1$. Similarly, $F_2$ is fixed by $G_1$ so $G_1 \leq G_2$ and we have equality, $G_1 = G_2$, which is a contradiction.
\end{proof}

\begin{thm}\label{thm:galiffsplit}
    The extension $K/F$ is Galois if and only if $K$ is the splitting field of some separable polynomial over $F$. Furthermore, if this is the case then every irreducible polynomial with coefficients in $F$ which has a root in $K$ is separable and has all of its roots in $K$ (so in particular $K/F$ is a separable extension).
\end{thm}
\begin{proof}
    By Corollary \ref{cor:galsep} we have that the splitting field of some separable polynomial over $F$ is Galois. 

    We now suppose that $K/F$ is Galois. Set $G = \text{Gal}(K/F)$. Let $p(x) \in F[x]$ be an irreducible polynomial having a root $\alpha \in K$, and consider the elements $\alpha,\sigma_2(\alpha),...,\sigma_n(\alpha)  \in K$ for $\{1,\sigma_2,...,\sigma_n\}$ are the elements of $G$. Let $\alpha,\alpha_2,...,\alpha_r$ denote the distinct roots in the list. If $\tau \in G$, then since $G$ is a group, $\{\tau,\tau\sigma_2,...,\tau\sigma_n\} = \{1,\sigma_1,...,\sigma_n\}$ in some order. It follows that applying $\tau$ to $\alpha,\alpha_2,...,\alpha_r$ also permutes these elements. The polynomial \begin{equation*}
        f(x) = (x-\alpha)(x-\alpha_2)...(x-\alpha_r)
    \end{equation*}
    has coefficients which are all fixed by all elements of $G$ since the elements of $G$ simply permute the factors. Hence the coefficients lie in the fixed field of $G$, so by a preceeding corollary, this field is $F$. Hence $f(x) \in F[x]$.

    Since $p(x)$ is irreducible and has $\alpha$ as a root, $p(x)$ is the minimal polynomial for $\alpha$ over $F$, hence $p(x)$ divides $f(x)$ in $F[x]$. Since $f(x)$ is a product of linear factors, each of which is a root of $p(x)$ (since roots are permuted under fixed field automorphisms), $f(x)$ divides $p(x)$ as well, so $p(x) = f(x)$ since they are both monic. Thus, $p(x)$ is separable and all of its roots lie in $K$.

    To complete the proof let $\omega_1,...,\omega_n$ be a basis for $K/F$. Let $p_i(x)$ be the minimal polynomial for $\omega_i$ for each $i$. Then by what we've just shown $p_i(x)$ is separable and has all of its roots in $K$. Let $g(x)$ be the polynomial obtained by removing any multiple factors from $p_1(x)...p_n(x)$. Then the splitting field of the two polynomials is the same, and this field is $K$. Indeed, all of the roots lie in $K$ so $K$ contains the splitting field, and the generators of $K$ are among the roots so $K$ is also contained in the splitting field. Hence $K$ is the splitting field of the separable polynomial $g(x)$.
\end{proof}

\begin{defn}
    Let $K/F$ be a Galois extension. If $\alpha \in K$ the elements $\sigma \alpha$ for $\sigma$ in $\text{Gal}(K/F)$ are called the \Emph{conjugates} (or \Emph{Galois conjugates}) of $\alpha$ over $F$. If $E$ is a subfield of $K$ containing $F$, the field $\sigma(E)$ is called the \Emph{conjugate field} of $E$ over $F$.
\end{defn}


The proof of the above theorem shows that in a Galois extension $K/F$, the other roots of a minimal polynomial over $F$ which has $\alpha \in K$ are precisely the distinct conjugates of $\alpha$ under the Galois group of $K/F$.

The second statement of the theorem also shows that $K$ is not Galois over $F$ if we can find even one irreducible polynomial over $F$ having a root in $K$ but not having all of its roots in $K$.

\begin{rmk}
    We now have the 4 characterizations of Galois extensions $K/F$: \begin{enumerate}
        \item splitting fields of separable polynomials over $F$
        \item fields where $F$ is precisely the set of elements fixed by $\aut(K/F)$
        \item fields with $[K:F] = |\aut(K/F)|$
        \item finite, normal and separable extensions.
    \end{enumerate}
\end{rmk}


\begin{namthm}[Fundamental Theorem of Galois Theory] \label{namthm:FundGal}
    Let $K/F$ be a Galois extension and set $G = \text{Gal}(K/F)$. Then there is a bijection \begin{equation*}
        \left\{\begin{array}{cc} & K \\ \text{subfields } E & \vert \\ \text{ of } K& E \\ \text{containing } F & \vert \\ & F\end{array}\right\} \leftrightarrow \left\{\begin{array}{cc} & 1 \\ \text{subgroups } H &  \vert \\ \text{ of } G& H \\  & \vert \\ & G\end{array}\right\}
    \end{equation*}
    given by the correspondences \begin{equation*}
        \begin{array}{ccc} E & \rightarrow & \left\{\begin{array}{cc} \text{the elements of } G \\ \text{fixing } E\end{array}\right\} \\ \left\{\begin{array}{cc} \text{the fixed field} \\ \text{of } H\end{array}\right\} & \leftarrow & H \end{array}
    \end{equation*}
    which are inverse under this correspondence ($E\mapsto \aut(K/E)$ and $H\mapsto K^H$).\begin{enumerate}
        \item (Inclusion reversing) If $E_1,E_2$ correspond to $H_1,H_2$, respectively, then $E_1 \subseteq E_2$ if and only if $H_2 \leq H_1$
        \item $[K:E] = |H|$ and $[E:F] = |G:H|$, the index of $H$ in $G$, then \begin{equation*}
                \begin{array}{cc}
                    \left.\begin{array}{c} K \\ \vert \\ E \end{array}\right\} & |H| \\ 
                        \shortparallel & \\
                    \left.\begin{array}{c} E \\ \vert \\ F \end{array}\right\} & |G:H| \\
                \end{array}
            \end{equation*}
        \item $K/E$ is always Galois, with Galois group $\text{Gal}(K/E) = H$: \begin{equation*}
            \begin{array}{cc}
                \left.\begin{array}{c} K \\ \vert \\ E \end{array}\right\} & |H|
            \end{array}
            \end{equation*}
        \item $E$ is Galois over $F$ if and only if $H$ is a normal subgroup in $G$. If this is the case, then the Galois group is isomorphic to the quotient group \begin{equation*}
                \text{Gal}(E/F)\cong G/H
        \end{equation*}
            More generally, if $H$ is not necessarily normal in $G$, the isomorphisms of $E$ into a fixed algebraic closure of $F$ containing $K$ which fix $F$ are in one to one correspondence with the cosets $\sigma H$ of $H$ in $G$.
        \item If $E_1,E_2$ correspond to $H_1,H_2$, respectively, then the intersection $E_1\cap E_2$ corresponds to the group $\langle H_1,H_2\rangle$ generated by $H_1$ and $H_2$ and the composite $E_1E_2$ corresponds to the intersection $H_1\cap H_2$. Hence the lattice of subfields of $K$ containing $F$ and the lattice of subgroups of $G$ are ``dual" (the lattice diagram for one is the lattice diagram for the other turned upside down)
    \end{enumerate}
\end{namthm}
\begin{proof}
    Let $K/F$ be a Galois extension and set $G = \text{Gal}(K/F)$. Given any subgroup $H$ of $G$ we obtain a unique fixed field $E = K_H$ by Corollary \ref{cor:distffs}. This shows that the correspondence above is injective going from subgroups to fixed fields.

    If $K$ is the splitting field of the separable polynomial $f(x) \in F[x]$, then we may also view $f(x)$ as an element of $E[x]$ for any subfield $E$ of $K$ containing $F$. Then $K$ is also a splitting field of $f(x)$ over $E$, so the extension $K/E$ is Galois. By Corollary \ref{cor:galifffixed}, $E$ is the fixed field of $\aut(K/E)\leq G$, showing that every subfield of $K$ containing $F$ arises as the fixed field for some subgroup of $G$. Hence the correspondence is surjective from subgroups to subfields, and hence a bijection. The correspondences are inverse to each other since the automorphisms fixing $E$ are presicely $\aut(K/E)$ by Corollary \ref{cor:galifffixed}.

    If $E_1,E_2$ correspond to $H_1,H_2$, we observe that $E_1 \subseteq E_2$ implies that every element of $H_2$ fixes $E_2$, and consequently $E_1$, so by definition $H_2 \leq H_1$. Conversely, if $H_2 \leq H_1$ then if $e \in E_1$, it follows that $e$ is fixed by all elements of $H_1$ by definition, and hence all elements of $H_2$, so $e$ is in the fixed field of $H_2$ which is $E_2$. Hence $e \in E_2$ and $E_1 \subseteq E_2$, proving $(1)$.

    If $E = K_H$ is the fixed field of $H$, then Theorem \ref{thm:subfixed} gives $[K:E] = |H|$, and $[K:F] = |G|$. Taking the quotient gives $[E:F] = |G|/|H| = |G:H|$, which proves $(2)$.

    If $E$ is a subfield of $K$, then by bijectivity we have that $E$ is the fixed field of some subgroup of $\aut(K/F)$, so by Corollary \ref{cor:subaut} the subgroup is $\aut(K/E)$ and $K/E$ is Galois, proving $(3)$.

    Suppose $E = K_H$ is the fixed field of the subgroup $H$. Every $\sigma \in G = \text{Gal}(K/F)$ when restricted to $E$ is an embedding $\sigma\vert_E$ of $E$ with the subfield $\sigma(E)$ of $K$. Conversely, let $\tau:E\xrightarrow{\sim}\tau{E}\subseteq \overline{F}$ be any embedding of $E$ (into a fixed algebraic closure $\overline{F}$ of $F$ containing $K$) which fixes $F$. Then $\tau(E)$ is in fact contained in $K$: if $\alpha \in E$ has minimal polynomial $m_{\alpha}(x)$ over $F$ then $\tau(\alpha)$ is another root of $m_{\alpha}(x)$ and $K$ contains all these roots by Theorem \ref{thm:galiffsplit}. As above $K$ is the splitting field of $f(x)$ over $E$ and so also the splitting field of $\tau f(x)$ (which is the same as $f(x)$ since $f(x)$ has coefficients in $F$) over $\tau(E)$. We cam then extend $\tau$ to an isomorphism $\sigma$ of $K$. Since $\sigma$ fixes $F$, because $\tau$ does, it follows that every embedding $\tau$ of $E$ fixing $F$ is the restriction to $E$ of some automorphism $\sigma$ of $K$ fixing $F$, in other words, every embedding of $E$ is of the form $\sigma\vert_E$ for some $\sigma \in G$.

    Two automorphisms $\sigma,\sigma' \in G$ restrict to the same embedding of $E$ if and only if $\sigma^{-1}\sigma'$ is the identity map on $E$. But then $\sigma^{-1}\sigma' \in H$, where $H = \aut(K/F)$ is the fixing subgroup of $E$. Hence, the distinct embeddings of $E$ are in bijection with the cosets $\sigma H$ of $H$ in $G$. In particular this gives \begin{equation*}
        |\text{Emb}(E/F)| = |G:H| = [E:F]
    \end{equation*}
    where $\text{Emb}(E/F)$ denotes the set of embeddings of $E$ into a fixed algebraic closure of $F$ which fix $F$. Note that $\text{Emb}(E/F)$ contains the automorphisms $\aut(E/F)$.

    The extension $E/F$ will be Galois if and only if $|\aut(E/F)| = [E:F]$. By the equality above, this will be the case if and only if each of the embeddings of $E$ is actually an automorphism of $E$, i.e., if and only if $\sigma(E) = E$ for every $\sigma \in G$.

    If $\sigma \in G$, then the subgroup of $G$ fixing the field $\sigma(E)$ is the group $\sigma H\sigma^{-1}$, i.e., $\sigma(E) = K_{\sigma H\sigma^{-1}}$. To see this observe that if $\sigma \alpha \in \sigma(E)$, then $$(\sigma h\sigma^{-1})(\sigma\alpha) = \sigma(h\alpha) = \sigma \alpha\;\;\;\;\text{for all } h \in H,$$ since $h$ fixes $\alpha \in R$, which shows that $\sigma H \sigma^{-1}$ fixes $\sigma(E)$. The group fixing $\sigma(E)$ has order equal to the degree of $K$ over $\sigma(E)$. But this is the same as the degree of $K$ over $E$ since the fields are isomorphic, hence the same as the order of $H$. Hence $\sigma H\sigma^{-1}$ is precisely the group fixing $\sigma(E)$ since we hav eshown containment and their orders are equal.

    Because of the bijective nature of the Galois correspondence already proved we know that two subfields of $K$ containing $F$ are equal if and only if their fixing subgroups are equal in $G$. Hence $\sigma(E) = E$ for all $\sigma \in G$ if and only if $\sigma H \sigma^{-1} = H$ for all $\sigma \in G$, in other words $E$ is Galois over $F$ if and only if $H$ is a normal subgroup of $G$.

    We hav ealready identified the embeddings of $E$ over $F$ as the set of cosets of $H$ in $G$ and when $H$ is normal in $G$ seeen that embedding are automorphisms. It follows that in this case the group of cosets $G/H$ is identified with the group of automorphisms of the Galois extension $E/F$ by the definition of the group operation (composition of automorphisms). Hence $G/H\cong \text{Gal}(E/F)$ when $H$ is normal in $G$, which completes the proof of $(4)$.

    Suppose $H_1$ is the subgroup of elements of $G$ fixing the subfield $E_1$ and $H_2$ is the subgroup of elements of $G$ fixing the subfield $E_2$. Any element in $H_1\cap H_2$ fixed both $E_1$ and $E_2$, hence fixes every element in the composite $E_1E_2$, since the elements in this field are algebraic combinations of the elements of $E_1$ and $E_2$. Conversely, if an automorphism $\sigma$ fixes the composite $E_1E_2$ then in particular $\sigma$ fixes $E_1$, i.e. $\sigma \in H_1$, and $\sigma$ fixes $E_2$, i.e., $\sigma \in H_2$, hence $\sigma \in H_1\cap H_2$. In a similar fashion, the intersection $E_1\cap E_2$ corresponds to the group $\langle H_1,H_2\rangle$ generated by $H_1$ and $H_2$, completing the proof of the theorem.
\end{proof}


\begin{eg}
    Consider $\Q(\sqrt[3]{2},\zeta_3)$, for $\zeta_3 = e^{2\pi i/3}$. Then $\Q(\sqrt[3]{2},\zeta_3)$ is the splitting field of $x^3-2$, which is irreducible and separable over $\Q$ so the extension is Galois. Then we can write: 
    \begin{center}
        \begin{tikzcd}
	& {\mathbb{Q}(\sqrt[3]{2},\zeta_3)} \\
	&& {\mathbb{Q}(\sqrt[3]{2})} & {\mathbb{Q}(\zeta_3\sqrt[3]{2})} & {\mathbb{Q}(\zeta_3^2\sqrt[3]{2})} \\
	{\mathbb{Q}(\zeta_3)} \\
	& {\mathbb{Q}}
	\arrow[no head, from=3-1, to=4-2]
	\arrow[no head, from=2-4, to=4-2]
	\arrow[no head, from=3-1, to=1-2]
	\arrow[no head, from=1-2, to=2-4]
	\arrow[no head, from=2-5, to=4-2]
	\arrow[no head, from=1-2, to=2-5]
	\arrow[no head, from=1-2, to=2-3]
	\arrow[no head, from=2-3, to=4-2]
\end{tikzcd}
    \end{center}
    Then $\text{Gal}(\mathbb{Q}(\sqrt[3]{2},\zeta_3)/\Q) = \langle \sigma,\tau\rangle$ for $\sigma,\tau$ defined by \begin{equation*}
        \sigma:\left\{\begin{array}{c} \sqrt[3]{2} \mapsto \zeta_3\sqrt[3]{2} \\ \zeta_3\mapsto \zeta_3 \end{array}\right.
    \end{equation*}
    and \begin{equation*}
        \tau:\left\{\begin{array}{c} \sqrt[3]{2} \mapsto \sqrt[3]{2} \\ \zeta_3\mapsto \zeta_3^2 = -1-\zeta_3 \end{array}\right.
    \end{equation*}
    and we have the relations $\sigma^3 = \tau^2 = 1$, and $\sigma\tau = \tau\sigma^2,$ so $\sigma\tau\sigma = \tau$. (Dihedral group of order $3$). We also obtain the diagram \begin{center}
\begin{tikzcd}
	& {\{1\}} \\
	&& {\langle 1, \tau\rangle} & {\langle 1, \tau\sigma\rangle} & {\langle 1, \tau\sigma^2\rangle} \\
	{\langle 1, \sigma\rangle} \\
	& {\langle \sigma, \tau\rangle}
	\arrow[no head, from=3-1, to=4-2]
	\arrow[no head, from=2-4, to=4-2]
	\arrow[no head, from=3-1, to=1-2]
	\arrow[no head, from=1-2, to=2-4]
	\arrow[no head, from=2-5, to=4-2]
	\arrow[no head, from=1-2, to=2-5]
	\arrow[no head, from=1-2, to=2-3]
	\arrow[no head, from=2-3, to=4-2]
\end{tikzcd}
    \end{center}
\end{eg}


\begin{eg}
    The extension $\Q(\sqrt[4]{2},i)/\Q$ is Galois since it is the splitting field of the seperable polynomial $x^4 - 2 \in \Q[x]$. We note that $[\Q(\sqrt[4]{2},i):\Q] = 8$, so as the extension is Galois there are $8$ automorphisms of $\Q(\sqrt[4]{2},i)$ fixing $\Q$. But, $x^4-2$ has $4$ roots, which can be permuted in $24$ ways, so not all permutations of the roots result in automorphisms of the field. For example, $\sqrt[4]{2}$ and $-\sqrt[4]{2}$ add to zero, so they must be mapped under field automorphisms to two roots which are negatives of each other. 

    To think about the Galois group concretely, we think of an automorphism $\sigma$ and what it does to $\sqrt[4]{2}$ and $i$, rather than what it does to all fourth roots of $2$. Since $\sigma(\sqrt[4]{2})$ must be a root of $X^4-2$, there are four possible values for which we can send it to. On the other hand $\sigma(i)$ must be a root of $x^2+1$, and hence there are two values for which we can send it to, so there are at most $4\cdot 2 = 8$ automorphisms of $\Q(\sqrt[4]{2},i)/\Q$. But, we have shown that there are exactly $8$ automorphisms in the Galois group, so all choices of where to send these two roots result in automorphisms of the extension field. Let $r$ be the automorphism determined by $r(\sqrt[4]{2}) = i\sqrt[4]{2}$ and $r(i) = i$, and $s$ be the automorphism determined by $s(\sqrt[4]{2}) = \sqrt[4]{2}$ and $s(i) = -i$. Taking compositions we obtain the following eight automorphisms:
    \begin{table}[H]
        \centering
        \caption{Galois Group for $\Q(\sqrt[4]{2},i)/\Q$}
        \begin{tabular}{c||c|c|c|c|c|c|c|c}
            \hline
            $\sigma$ & id & $r$ & $r^2$ & $r^3$ & $s$ & $rs$ & $r^2s$ & $r^3s$ \\ \hline
            $\sigma(\sqrt[4]{2})$ & $\sqrt[4]{2}$ & $i\sqrt[4]{2}$ & $-\sqrt[4]{2}$ & $-i\sqrt[4]{2}$ & $\sqrt[4]{2}$ & $i\sqrt[4]{2}$ & $-\sqrt[4]{2}$ & $-i\sqrt[4]{2}$ \\ \hline
            $\sigma(i)$ & $i$ & $i$ & $i$ & $i$ & $-i$ & $-i$ & $-i$ & $-i$ \\ \hline
        \end{tabular}
    \end{table}
    A calculation shows $r^4 = s^2 = id$, and $rs= sr^{-1}$, so $\text{Gal}(\Q(\sqrt[4]{2},i)/\Q) \cong D_4$, where $D_4$ can be viewed as the $8$ symmetries of the square whose vertices are the four complex roots of $X^4-2$; in particular, $r$ is a counterclockwise rotation by $90^{\circ}$, and $s$ is a reflection in the real axis (complex conjugation), which is a diagonal of the square. Note $r$ is not multiplication by $i$ everywhere, so it is not generally rotating all elements of $\Q(\sqrt[4]{2},i)$ by $90^{\circ}$. 
\end{eg}


\section{\textsection Finite Fields}

Recall a finite field $\F$ has characteristic $p$ for some prime, and hence is a finite dimensional vector space over $\F_p = \Z/p\Z$. Moreover, if the dimension is $n$, i.e. $[\F:\F_p] = n$, then $\F$ has precisely $p^n$ elements. Recall that we have seen previously that $\F$ is then isomorphic to the splitting field of the separable polynomial $x^{p^n}-x$ is $\F_p[x]$, and hence is a Galois extension. Let $\F_{p^n}$ denote this field of order $p^n$. 

Further, the Galois group of $\F_{p^n}$ is of cyclic of order $n$, generated by the Frobenius automorphism $\sigma_p$: \begin{equation*}
    \text{Gal}(\F_{p^n}/\F_p) = \langle \sigma_p\rangle \cong \Z/n\Z
\end{equation*}
where $\sigma_p:\F_{p^n}\rightarrow \F_{p^n}$ is given by $\sigma_p(a) = a^p$. By the Fundamental Theorem of Galois Theory we have a bijective correspondence between subfields of $\F_{p^n}$ containing $\F_p$ and subgroups of $\Z/n\Z$. In particular, for every divisor $d$ of $n$ we have a subfield of degree $d$ over $\F_p$, namely the fixed field generated by $\sigma_p^d$ of order $n/d$. Moreover, from a previous discussion we know this field is isomorphic to $\F_{p^d}$, the unique finite field of order $p^d$. Since the Galois group is cyclic, it is abelian and hence all subgroups are normal so $\F_{p^d}/\F$ is Galois with Galois group $\text{Gal}(\F_{p^d},\F)\cong \langle \sigma_p\rangle/\langle \sigma_p^d\rangle$. 

\begin{prop}
    Any finite field is isomorphic to $\F_{p^n}$ for some prime $p$ and some integer $n \geq 1$. The field $\F_{p^n}$ is the splitting field over $\F_p$ of the polynomial $x^{p^n}-x$, with cyclic Galois group of order $n$ generated by the Frobenius automorphism $\sigma_p$. The subfields of $\F_{p^n}$ are all Galois over $\F_p$ and are in one to one correspondence with divisors $d$ of $n$. They are the fields $\F_{p^d}$, the fixed fields of $\sigma_p^d$.
\end{prop}

We observe that the multiplicative group $\F_{p^n}^{\times}$ is finite group composed of field elements, and is hence cyclic. Then, if $\theta$ is any generator of this group we obtain $\F_{p^n} = \F_p(\theta)$. Consequently:

\begin{prop}
    The finite field $\F_{p^n}$ is simple. In particular, there exists an irreducible polynomial of degree $n$ over $\F_p$ for each $n \geq 1$.
\end{prop}

Recall the elements of $\F_{p^n}$ are precisely the roots of $x^{p^n}-x \in \F_p[x]$. Grouping together the factors $x-\alpha$ of this polynomial based on the degree $d$ of their minimal polynomials over $\F_p$, we obtain
\begin{prop}
    The polynomial $x^{p^n}-x$ is precisely the product of all the distinct irreducible polynomials in $\F_p[x]$ of degree $d$ where $d$ runs through all divisors of $n$.
\end{prop}

This result gives a method for determining the product of all the irreducible polynomials over $\F_p$ of a given degree. Note also that since the finite field $\F_{p^n}$ is unique up to isomorphism, the quotients of $\F_p[x]$ by any irreducible polynomial of degree $n$ are all isomorphic. 

Define the M$\ddot{o}$bius $\mu$-functions by \begin{equation*}
    \mu(n) = \left\{\begin{array}{lc} 1 & \text{for } n = 1 \\ 0 & \text{if } n\text{ has a square factor} \\ (-1)^r & \text{if } n \text{ has } r \text{ distinct prime factors} \end{array}\right.
\end{equation*}

If $f(n)$ is a function defined for all nonnegative integers and $F(n)$ is defined by \begin{equation*}
    F(n) = \sum_{d\vert n}f(d)
\end{equation*}
Then the M$\ddot{o}$bius inversion formula states that one can recover the function $f(n)$ from $F(n)$: \begin{equation*}
    f(n) = \sum_{d\vert n}\mu(d)F(n/d)
\end{equation*}
Then define $\psi(n)$ to be the number of irreducible polynomials of degree $n$ in $\F_p[x]$. Counting degrees in our previous proposition \begin{equation*}
    p^n = \sum_{d\vert n}d\psi(d)
\end{equation*}
Applying the inversion formula for $f(n) = n\psi(n)$, we obtain \begin{equation*}
    n\psi(n) = \sum_{d\vert n}\mu(d)p^{n/d}
\end{equation*}
which gives us a formula for the number of irreducible polynomials of degree $n$ over $\F_p[x]$: \begin{equation*}
    \psi(n) = \frac{1}{n}\sum_{d\vert n}\mu(d)p^{n/d}
\end{equation*}


Next, recall that $\F_{p^d} \subseteq \F_{p^n}$ if and only if $d$ divides $n$. In particular, for finite fields $\F_{p^{n_1}}$ and $\F_{p^{n_2}}$, we have a third field containing isomorphic copies of them, $\F_{p^{n_1n_2}}$. This gives us a partial ordering on finite fields of order $p$, and allows us to take their union. Since these give all the finite extensions of $\F_p$, we see that the union of $\F_{p^n}$ for $n \geq 1$ is an algebraic closure of $\F_p$, unique up to isomorphism: \begin{equation*}
    \overline{\F_p} = \bigcup_{n\geq 1}\F_{p^n}
\end{equation*}




\section{\textsection Composite Extensions and Simple Extensions}


\begin{prop}\label{prop:galtrans}
    Suppose $K/F$ is a Galois extension and $F'/F$ is any extension. Then $KF'/F'$ is a Galois extension, with Galois group \begin{equation*}
        \text{Gal}(KF'/F') \cong \text{Gal}(K/K\cap F')
    \end{equation*}
    isomorphic to a subgroup of $\text{Gal}(K/F)$.
    \begin{center}
        \begin{tikzcd}
	& {KF'} \\
	K && F \\
	& {K\cap F'} \\
	& F
	\arrow[no head, from=4-2, to=3-2]
	\arrow[no head, from=3-2, to=2-3]
	\arrow[""{name=0, anchor=center, inner sep=0}, "{//}"{marking}, no head, from=2-1, to=3-2]
	\arrow[no head, from=2-1, to=1-2]
	\arrow[""{name=1, anchor=center, inner sep=0}, "{//  }"{marking}, no head, from=1-2, to=2-3]
	\arrow[shorten <=4pt, shorten >=4pt, Leftrightarrow, from=0, to=1]
\end{tikzcd}
    \end{center}
\end{prop}
\begin{proof}
    If $K/F$ is a Galois extension, then $K$ is a splitting field of a separable polynomial $f(x)$ over $F[x]$. Then observe that $KF'/F'$ is a splitting field for $f(x)$ viewed as a polynomial in $F'[x]$, since $F'/F$.  Hence, $KF'/F'$ is Galois. Since $K/F$ is Galois, every embedding of $K$ fixing $F$ is an automorphism of $K$, so the map \begin{equation*}
        \map{\varphi:\text{Gal}(KF'/F')\rightarrow \text{Gal}(K/F)}{\sigma\mapsto \sigma\vert_K}
    \end{equation*}
    defined by restricting an automorphism $\sigma$ to the subfield $K$ is well defined. Further it is a homomorphism with kernel \begin{equation*}
        \ker \varphi = \{\sigma \in \text{Gal}(KF'/F')\vert \sigma\vert_K = 1\}
    \end{equation*}
    Since an element of $\text{Gal}(KF'/F')$ is the identity on $F'$, and elements of the kernel are identities on both $K$ and $F'$, they are also the identity on their composite. Hence, the kernel is trivial and $\varphi$ is injective. 

    Let $H$ denote the image of $\varphi$, and let $K_H$ denote the corresponding fixed subfield. Since every element in $H$ fixes $F'$, $K_H$ contains $K\cap F'$. On the other hand, the composite $K_HF'$ is fixed by $\text{Gal}(KF'/F')$ (any $\sigma \in \text{Gal}(KF'/F')$ fixes $F'$, and fixes $K_H \subseteq K$ when restricted to $\sigma\vert_K$). By bijectivity in the fundamental theorem of Galois theory, $K_HF' = F'$, so that $K_H \subseteq F'$, which gives the reverse inclusion $K_H \subseteq K\cap F'$. Hence $K_H = K\cap F'$, so again by the Fundamental Theorem of Galois Theory, $H = \text{Gal}(K/K\cap F')$, completing the proof.
\end{proof}

\begin{cor}
    Suppose $K/F$ is a Galois extension and $F'/F$ is any finite extension. Then \begin{equation*}
        [KF':F] = \frac{[K:F][F':F]}{[K\cap F':F]}
    \end{equation*}
\end{cor}
\begin{proof}
    From the proposition we have the equality $[KF':F'] = [K:K\cap F']$. Then from the equality $[K:F] = [K:K\cap F'][K\cap F':F]$ we observe that \begin{equation*}
        [KF':F] = [KF':F'][F':F] = [K:K\cap F'][F':F] = \frac{[K:F][F':F]}{[K\cap F':F]}
    \end{equation*}
    completing the proof.
\end{proof}

\begin{prop}\label{prop:Galprod}
    Let $K_1$ and $K_2$ be Galois extensions of a field $F$. Then \begin{itemize}
        \item The intersection $K_1\cap K_2$ is Galois over $F$.
        \item The composite $K_1K_2$ is Galois over $F$. The Galois group is isomorphic to the subgroup \begin{equation*}
                H = \{(\sigma,\tau)\vert\sigma\vert_{K_1\cap K_2} = \tau\vert_{K_1\cap K_2}\}
        \end{equation*}
            of the direct product $\text{Gal}(K_1/F)\times \text{Gal}(K_2/F)$ consisting of elements whose restrictions to the intersection $K_1\cap K_2$ are equal
    \end{itemize}
    \begin{center}
        \begin{tikzcd}
	& {K_1K_2} \\
	{K_1} && {K_2} \\
	& {K_1\cap K_2} \\
	& F
	\arrow[no head, from=4-2, to=3-2]
	\arrow[no head, from=3-2, to=2-3]
	\arrow[no head, from=2-1, to=3-2]
	\arrow[no head, from=2-1, to=1-2]
	\arrow[no head, from=1-2, to=2-3]
\end{tikzcd}
    \end{center}
\end{prop}
\begin{proof}
    For $(1)$ suppose $p(x)$ is an irreducible polynomial in $F[x]$ with a root $\alpha$ in $K_1\cap K_2$. Since $\alpha \in K_1$ and $K_1/F$ is Galois, all the roots of $p(x)$ lie in $K_1$. Similarly all the roots lie in $K_2$, hence all the roots of $p(x)$ lie in $K_1\cap K_2$. \textbf{TO BE FINISHED}

    For $(2)$ if $K_1$ is the splitting field of the separable polynomial $f_1(x)$ and $K_2$ is the splitting field of the separable polynomial $f_2(x)$ then the composite is the splitting field for the squarefree part of the polynomial $f_1(x)f_2(x)$, hence is Galois over $F$.

    The map \begin{equation*}
        \map{\varphi:\text{Gal}(K_1K_2/F)\rightarrow \text{Gal}(K_1/F)\times \text{Gal}(K_2/F)}{\sigma \mapsto (\sigma\vert_{K_1},\sigma\vert_{K_2})}
    \end{equation*}
    is clearly a homomorphism. The kernel consists of the elements of $\sigma$ which are trivial on both $K_1$ and $K_2$. hence trivial on the composite, so the map is injective. The image lies in the subgroup $H$, since \begin{equation*}
        (\sigma\vert_{K_1})\vert_{K_1\cap K_2} = \sigma\vert_{K_1\cap K_2} = (\sigma\vert_{K_2})\vert_{K_1\cap K_2}
    \end{equation*}
    The order of $H$ can be computed by observing that for every $\sigma \in \text{Gal}(K_1/F)$ there are $|\text{Gal}(K_2/K_1\cap K_2)|$ elements $\tau \in \text{Gal}(K_2/F)$ whose restriction to $K_1\cap K_2$ are $\sigma\vert_{K_1\cap K_2}$.Hence \begin{align*}
        |H| &= |\text{Gal}(K_1/F)|\cdot|\text{Gal}(K_2/K_1\cap K_2)| \\
        &= |\text{Gal}(K_1/F)|\frac{|\text{Gal}(K_2/F)|}{|\text{Gal}(K_1\cap K_2/F)|}
    \end{align*}
    By the previous corollary and the diagram above we see that the orders of $H$ and $\text{Gal}(K_1K_2/F)$ are then both equal to \begin{equation*}
        [K_1K_2:F] = \frac{[K_1:F][K_2:F]}{[K_1\cap K_2:F]}
    \end{equation*}
    Hence the image of $\varphi$ is precisely $H$, completing the proof.
\end{proof}


\begin{cor}
    Let $K_1$ and $K_2$ be Galois extensions of a field $F$ with $K_1\cap K_2 = F$. Then \begin{equation*}
        \text{Gal}(K_1K_2/F)\cong \text{Gal}(K_1/F)\times \text{Gal}(K_2/F)
    \end{equation*}
    Conversely, if $K$ is Galois over $F$ and $G = \text{Gal}(K/F) = G_1\times G_2$ is the direct product of two subgroups $G_1$ and $G_2$, then $K$ is the composite of two Galois extension $K_1$ and $K_2$ of $F$ with $K_1\cap K_2 = F$.
\end{cor}
\begin{proof}
    By the previous proposition we have that $\text{Gal}(K_1K_2/F)$ is isomorphic to a subgroup of $\text{Gal}(K_1/F)\times \text{Gal}(K_2/F)$. Furthermore, since $K_1 \cap K_2 = F$ we have that $[K_1K_2:F] = [K_1:F][K_2:F]$, so we find that $|\text{Gal}(K_1K_2/F)| = |\text{Gal}(K_1/F)||\text{Gal}(K_2/F)|$, which implies by order considerations that $\text{Gal}(K_1K_2/F)\cong \text{Gal}(K_1/F)\times \text{Gal}(K_2/F)$.

    For the second part, let $K_1$ be the fixed of $G_1 \subset G$ and let $K_2$ be the fixed field of $G_2 \subset G$. Then $K_1\cap K_2$ is the field corresponding to the subgroup $G_1G_2 = G$, so $K_1\cap K_2 = F$ by bijectivity of the Galois correspondence. Hence, the composite $K_1K_2$ being the field corresponding to the subgroup $G_1\cap G_2$, which is the identity here, so $K_1K_2 = K$, completing the proof.
\end{proof}


\begin{cor}
    Let $E/F$ be any finite separable extension. Then $E$ is contained in an extension $K$ which is Galois over $F$ and is minimal in the sense that in a fixed algebraic closure of $K$ any other Galois extension of $F$ containing $E$ contains $K$.
\end{cor}
\begin{proof}
    THere exists a Galois extension of $F$ containing $E$, for example the composite of the splitting fields of the minimal polynomials for a basis for $E$ over $F$ (which are all separable since $E$ is separable over $F$). Then the intersection of all the Galois extensions containing $E$ is the field $K$.
\end{proof}

\begin{defn}
    The Galois extension $K$ of $F$ containing $E$ in the previous corollary is callled the \Emph{Galois closure} of $E$ over $F$.
\end{defn}

Recall that an extension $K$ of $F$ is called simple if $K = F(\theta)$ for some element $\theta$, in which case $\theta$ is called a \Emph{primitive element} for $K$.

\begin{prop}
    Let $K/F$ be a finite extension. Then $K = F(\theta)$ if and only if there exist only finitely many subfields of $K$ containing $F$.
\end{prop}
\begin{proof}
    Suppose first that $K = F(\theta)$ is simple. Let $E$ be a subfield of $K$ containing $F$, $F\subseteq E\subseteq K$. Let $f(x) \in F[x]$ be the minimal polynomial for $\theta$ over $F$ and let $g(x) \in E[x]$ be the minimal polynomial for $\theta$ over $E$. Then $g(x)$ divides $f(x)$ in $E[x]$. Let $E'$ be the field generated over $F$ by the coefficients of $g(x)$. Then $E'\subseteq E$ and evidently the minimal polynomial for $\theta$ over $E'$ is also $g(x)$. But then \begin{equation*}
        [K:E] = \deg g(x) = [K:E']
    \end{equation*}
    implies that $E = E'$. It follows that the subfields of $K$ containing $F$ are the subfields generated by the coefficients of the monic factors of $f(x)$, hence there are finitely many such subfields.

    Suppose conversely that there are finitely many subfields of $K$ containing $F$. If $F$ is a finite field, then we have already that $K$ is simple since it is a finite extension of a finite field, and hence also finite. Hence we may suppose $F$ is infinite. It suffices to show that $F(\alpha,\beta)$ is generated by a single element since $K$ is finitely generated over $F$. Consider the subfields \begin{equation*}
        F(\alpha+c\beta),\;\;\;\;c \in F
    \end{equation*}
    Then since there are infinitely many choices for $c \in F$ and only finitely many such subfields, there exist $c,c' \in F$, $c \neq c'$, such that $F(\alpha+c\beta) = F(\alpha+c'\beta)$. Then $\alpha+c\beta$ and $\alpha+c'\beta$ both lie in $F(\alpha+c\beta)$, and taking their difference shows $(c-c')\beta \in F(\alpha+c\beta)$. Hence $\beta \in F(\alpha+c\beta)$ since $c-c' \neq 0$, and consequently $\alpha \in F(\alpha+c\beta)$. Therefore $F(\alpha,\beta) \subseteq F(\alpha+c\beta)$, and since the reverse inclusion is immediate we have $F(\alpha,\beta) = F(\alpha+c\beta)$, completing the proof.
\end{proof}

\begin{thm}[The Primitive Element Theorem]
    If $K/F$ is finite and separable, then $K/F$ is simple. In particular, any finite extension of fields of characteristic $0$ is simple.
\end{thm}
\begin{proof}
    Let $L$ be the Galois closure of $K$ over $F$. Then any subfield of $K$ containing $F$ corresponds to a subgroup of the Galois group $\text{Gal}(L/F)$ by the fundamental theorem of Galois theory. Since there are only finitely many such subgroups, the previous proposition shows that $K/F$ is simple. The last statement follows since any finite extension of fields in characteristic $0$ is separable (it is a perfect field so all irreducible polynomials are separable).
\end{proof}

As the proof of the proposition illustrates, a primitive element for an extension can be obtained as a simple linear combination of the generators for the extension. In the case of Galois extensions it is only necessary to determine a linear combination which is not fixed by any nontrivial element of the Galois group since then by the Fundamental Theorem this linear combination could not lie in any proper subfield. 



\section{\textsection Cyclotomic Extensions and Abelian Extensions}

Recall that the cyclotomic field $\Q(\zeta_n)$ of $n$th roots of unity is a Galois extension of $\Q$, being the splitting field of the separable polynomial $x^n-1 \in \Q[x]$, and has degree $\varphi(n)$. Moreover, the roots of $x^n-1$ form a cyclic group of order $n$, and hence any automorphism of $\Q(\zeta_n)$ is fully determined by its action on $\zeta_n$. 

\begin{thm}
    The Galois group of the cyclotomic field $\Q(\zeta_n)$ of $n$th roots of unity is isomorphic to the multiplicative group $(\Z/n\Z)^{\times}$. The isomorphism is given explicitly by the map \begin{equation*}
        \begin{array}{rcl} (\Z/n\Z)^{\times} & \xrightarrow{\sim} & \text{Gal}(\Q(\zeta_n)) \\ a\mod n& \mapsto &\sigma_a \end{array}
    \end{equation*}
    where $\sigma_a$ is the automorphism defined by $\sigma_a(\zeta_n) = \zeta_n^a$.
\end{thm}


\begin{eg}
    $\Q(\zeta_5)$ is galois over $\Q$ with $(\Z/5\Z)^{\times} \cong \Z/4\Z$. Indeed, $x^5-1$ is separable as $\{1,\zeta_5,\zeta_5^2,\zeta_5^3,\zeta_5^4\}$ are its five distinct roots, and $\Q(\zeta_5)$ is consequently a splitting field for this polynomial. The elements of the Galois group $\text{Gal}(\Q(\zeta_5)/\Q)$ are generated by $\sigma(\zeta_5) = \zeta_5^2$, so $\sigma^2(\zeta_5) = \zeta_5^4$, $\sigma^3(\zeta_5) = \zeta_5^8 = \zeta_5^3$, $\sigma^4(\zeta_5) = \zeta_5^{6} = \zeta_5$, so $\sigma^4 = 1$, and $\text{Gal}(\Q(\zeta_5)/\Q) \cong \Z/4\Z$. Then, we can consider the subgroup $H = \{1,\sigma^2\}$ of order $2$. Then, the fixed field of of $H$ is $\Q(\zeta_5+\zeta_5^{-1})$
\end{eg}

Suppose that $n = p_1^{a_1}p_2^{a_2}...p_k^{a_k}$ is the decomposition of $n$ into distinct prime powers. Since $\zeta_n^{p_2^{a_2}...p_k^{a_k}}$ is a primitive $p_1^{a_1}$th root of unity, the field $K_1 = \Q(\zeta_{p_1^{a_1}})$ is a subfield of $\Q(\zeta_n)$. Similarly, each of the fields $K_i = \Q(\zeta_{p_i^{a_i}})$, $i = 1,2,...,k$ is a subfield of $\Q(\zeta_n)$. The composite of the fields contains the product $\zeta_{p_1^{a_1}}...\zeta_{p_k^{a_k}}$, which is a primitive $n$th root of unity, hence the composite field is $\Q(\zeta_n)$. SInce the extension degrees $[K_i:\Q]$ equal $\varphi(p_i^{a_i})$, $i=1,2,...,k$ and $\varphi(n) = \varphi(p_1^{a_1})\varphi(p_2^{a_2})...\varphi(p_k^{a_k})$, the degree of the composite of the fields $K_i$ is presicely the product of the degrees of the $K_i$. It follows by Proposition \ref{prop:Galprod} that the intersection of all these fields is precisely $\Q$. Then the Galois group for $\Q(\zeta_n)$ is the direct product of the Galois groups over $\Q$ for the subfields $K_i$.


\begin{cor}
    Let $n = p_1^{a_1}...p_k^{a_k}$ be the decomposition of the positive integer $n$ into distinct prime powers. Then the cyclotomic fields $\Q(\zeta_{p_i^{a_i}})$, $i = 1,2,...,k$ intersect in the field $\Q$ and their composite is the cyclotomic field $\Q(\zeta_n)$. Further, we have \begin{equation*}
        \text{Gal}(\Q(\zeta_n)/\Q)\cong \text{Gal}(\Q(\zeta_{p_1^{a_1}})/\Q)\times ...\times \text{Gal}(\Q(\zeta_{p_k^{a_k}})/\Q)
    \end{equation*}
    which under the isomorphism in the previous theorem is the Chinese Remainder Theorem\begin{equation*}
        (\Z/n\Z)^{\times} \cong (\Z/p_1^{a_1}\Z)^{\times}\times...\times (\Z/p_k^{a_k}\Z)^{\times}
    \end{equation*}
\end{cor}

\begin{defn}
    The extension $K/F$ is called an \Emph{abelian extension} if $K/F$ is Galois and $\text{Gal}(K/F)$ is an abelian group.
\end{defn}

Since all the subgroups and quotient groups of abelian groups are abelian, we see by the Fundamental Theorem of Galois Theory that every subfield of an abelian extension of $F$ which contains $F$ is itself an abelian extension of $F$. Based on the results on composites, the composite of abelian extensions is again an abelian extension. 

\begin{cor}
    Let $G$ be any finite abelian group. Then there is a subfield $K$ of a cyclotomic field with $\text{Gal}(K/\Q)\cong G$.
\end{cor}


\begin{thm}[Kronecker-Weber]
    Let $K$ be a finite abelian extension of $\Q$. Then $K$ is contained in a cyclotomic extension of $\Q$.
\end{thm}

\subsection{Constructability of the n-gon}

Recall that $\alpha$ is constructible over $\Q$ if and only if $\Q(\alpha)$ is contained in a field $K$ obtained by a series of quadratic extensions: \begin{equation*}
    \Q = K_0\subset K_1 \subset ... \subset K_m = K
\end{equation*}
with $[K_{i+1}:K_i] = 2$, $i = 0,1,...,m-1$. Note that $n$th roots of unity form the vertices of a regular $n$-gon on the unit circle in $\C$ with one vertex at the point $1$.

(To be continued)



\section{\textsection Galois Group of Polynomials}

Recall that the Galois group of a separable polynomial $f(x) \in F[x]$ is the Galois group of the splitting field of $f(x)$ over $F$. 

Further, if $K/F$ is Galois, then $K$ is the splitting field for some separable polynomial $f(x) \in F[x]$, and $[K:F] = |\aut(K/F)|$. Moreover, any automorphism $\sigma \in \text{Gal}(K/F)$ maps a root of an irreducible factor of $f(x)$ to another root of the irreducible factor, and $\sigma$ is uniquely determined by its action on these roots (since they generate $K$ over $F$). Fix a labeling $\alpha_1,...,\alpha_n$ the roots of $f(x)$, and observe that $\sigma \in \text{Gal}(K/F)$ defines a unique permutation of $\alpha_1,...,\alpha_n$. This gives an injection $\text{Gal}(K/F) \hookrightarrow S_n$. 

Since the degree of the splitting field is the same as the order of the Galois group, this explains from the group-theoretic side why the splitting field for a polynomial of degree $n$ over $F$ is of degree at most $n!$ over $F$. 

In general, if the factorization of $f(x) =f_1(x)...f_k(x)$ into (distinct) irreducibles where $f_i(x)$ has degree $n_i$, then since the Galois group permutes the roots of the irreducible factors among themselves we have $\text{Gal}(K/F) \hookrightarrow S_{n_1}\times ... \times S_{n_k}$. 

If $f(x)$ is itself irreducible, then given any two roots of $f(x)$ there is an automorphism in the Galois group which maps the first root to the second. Such a group is said to be \Emph{transitive} on the roots. Thus, the Galois group must be transitive on blocks of roots.

\begin{eg}
    Consider the biquadratic extension $\Q(\sqrt{2},\sqrt{3})$ over $\Q$, which is the splitting field of $(x^2-2)(x^2-3)$. Label the roots $\alpha_1 = \sqrt{2},\alpha_2 = -\sqrt{2},\alpha_3 = \sqrt{3},\alpha_4 = -\sqrt{3}$. From our previous discussion $\text{Gal}(\Q(\sqrt{2},\sqrt{3})/\Q) \hookrightarrow S_2\times S_2 = \Z/2\Z\times \Z/2\Z = V_4$, the Klien-4 group. But, $|\text{Gal}(\Q(\sqrt{2},\sqrt{3})/\Q)| = 4$, so $\text{Gal}(\Q(\sqrt{2},\sqrt{3})/\Q)\cong V_4$. Viewing it as a subgroup of $S_4$, with have from a previous discussion $\text{Gal}(\Q(\sqrt{2},\sqrt{3})/\Q) = \{1,\sigma,\tau,\sigma\tau\}$ so \begin{equation*}
        \sigma = (12), \tau = (34)
    \end{equation*}
    as viewed in $S_4$ with the above labelling. Further, $\sigma\tau = (12)(34)$. Hence \begin{equation*}
        \text{Gal}(\Q(\sqrt{2},\sqrt{3})/\Q) \cong \{1,(12),(34),(12)(34)\} \subseteq S_4
    \end{equation*}
\end{eg}


\begin{eg}
    Consider the Galois group of $x^3 - 2$, with extension $\Q(\sqrt[3]{2},\zeta_3)$, with roots $\sqrt[3]{2},\zeta_3\sqrt[3]{2},\zeta_3^2\sqrt[3]{2}$. With this ordering the generators $\sigma$ and $\tau$ of the Galois group givethe following permutations \begin{equation*}
        \sigma = (123), \tau = (23)
    \end{equation*}
    which gives \begin{equation*}
        \text{Gal}(\sqrt[3]{2},\zeta_3) \cong \{1,\sigma,\sigma^2,\tau,\tau\sigma,\tau\sigma^2\} = S_3
    \end{equation*}
    in this case the full symmetric group on $3$ letters.
\end{eg}

Recall that every finite group is isomorphic to a subgroup of $S_n$, for some $n$ (by Cayley's Theorem). It is currently an open problem as to whether every finite group appears as the Galois group for some polynomial over $\Q$. 

\begin{defn}
    Let $x_1,x_2,...,x_n$ be indeterminates. The \Emph{elementary symmetric functions} $s_1,s_2,...,s_n$ are defined by \begin{align*}
        s_1 &= x_1+x_2+...+x_n \\
        s_2 &= \sum_{i<j}x_ix_j= x_1x_2+x_1x_3+ ... + x_2x_3 +x_2x_4+...+x_{n-1}x_n \\
        &\vdots 
        s_n = x_1x_2...x_n
    \end{align*}
    i.e., the $i$th symmetric function $s_i$ of $x_1,...,x_n$ is the sum of all products of the $x_j$'s, taken $i$ at a time.
\end{defn}

\begin{defn}
    The \Emph{general polynomial of degree $n$} is the polynomial \begin{equation*}
        (x-x_1)(x-x_2)...(x-x_n) \in (\F[x_1,x_2,...,x_n])[x]
    \end{equation*}
    whose roots are the indeterminates $x_1,x_2,...,x_n$.
\end{defn}

By induction, it can be shown that the coefficients of the general polynomial of degree $n$ are given by the elementary symmetric functions in the roots: \begin{equation*}
    (x-x_1)(x-x_2)...(x-x_n) = x^n - s_1x^{n-1}+s_2x^{n-2}+...+(-1)^ns_n
\end{equation*}

For any field $F$, the extension $F(x_1,x_2,...,x_n)$ is then a Galois extension of the field $F(s_1,s_2,...,s_n)$ since it is a splitting field of the general polynomial of degree $n$. 

If $\sigma \in S_n$ is any permutation of $\{1,2,...,n\}$, then $\sigma$ acts on the rational functions in $F(x_1,x_2,...,x_n)$ by permuting the subscripts of $x_1,x_2,...,x_n$, and this gives an automorphism of $F(x_1,...x_n)$. Identifying $\sigma \in S_n$ with this automorphism, identifies $S_n$ as a subgroup of $\aut(F(x_1,...,x_n))$. The elementary symmetric functions are fixed under any permutation of their subscripts (this is the reason they are called symmetric), which shows that the subfield $F(s_1,...,s_n)$ is contained in the fixed field of $S_n$ as a subgroup of $\aut(F(x_1,...,x_n))$. By the Fundamental Theorem of Galois Theory, the fixed subfield of $S_n$ has index $n!$ in $F(x_1,...,x_n)$. Since $F(x_1,...,x_n)$ is the splitting field over $F(s_1,...,s_n)$ of the polynomial of degree $n$ above, we have \begin{equation*}
    [F(x_1,...,x_n):F(s_1,...,s_n)] \leq n!
\end{equation*}
Since the subfield must contain $F(s_1,...,s_n)$, it follows that we have equality and that $F(s_1,...,s_n)$ is precisely the fixed field of $S_n$.

\begin{prop}
    The fixed field of the symmetric group $S_n$ acting on the field of rational functions in $n$ variables $F(x_1,x_2,...,x_n)$ is the field of the rational functions in the elementary symmetric functions $F(s_1,...,s_n)$.
\end{prop}

\begin{defn}
    A rational function $f(x_1,...,x_n)$ is called \Emph{symmetric} if it is not changed by any permutation of the variables $x_1,x_2,...,x_n$.
\end{defn}

\begin{cor}[Fundamental Theorem of Symmetric Functions]
    Any symmetric function in the variables $x_1,x_2,...,x_n$ is a rational function in the elementary symmetric functions $s_1,s_2,...,s_n$.
\end{cor}
\begin{proof}
    A symmetric function lies in the fixed field of the subgroup isomorphic to $S_n$ in $\text{Gal}(F(x_1,...,x_n)/F(s_1,...,s_n))$, and hence lies in $F(s_1,...,s_n)$ by the previous proposition.
\end{proof}

\begin{rmk}
    If $f(x_1,...,x_n)$ is a polynomial in $x_1,...,x_n$ which is symmetric then it can be seen that $f$ is actually a polynomial in $s_1,...,s_n$. It is in fact true that a symmetric polynomial whose coefficients lie in $R$, where $R$ is any commutative ring with identity, is a polynomial in the elementary symmetric functions with coefficients in $R$.
\end{rmk}

\begin{eg}
    Consider $(x_1-x_2)^2$, which is symmetric in $x_1$ and $x_2$, so we can write $(x_1-x_2)^2 = (x_1+x_2)^2 - 4x_1x_2 = s_1^2-4s_2$ a polynomial in terms of symmetric functions.
\end{eg}

\begin{eg}
    Consider the three hyperbolic numbers $\mathcal{H}_3 = \langle j\rangle$, $j^3 = 1$, $\zeta = x+jy+j^2z$, which in matrix form is \begin{equation*}
        M[\zeta] = \begin{bmatrix} x & z & y \\ y & x & z \\ z & y & x \end{bmatrix} 
    \end{equation*}
    which has determinant $\det(M[\zeta]) = x^3+y^3+z^3 - 3xyz$, which is symmetric in $x,y,z$. This can then be written as $(x+y+z)^3 -2(xy+xz+yz)-3xyz = s_1(s_1^2-3s_2)$. 
\end{eg}

\begin{thm}
    The general polynomial \begin{equation*}
        x^n-s_1x^{n-1}+s_2x^{n-2}+...+(-1)^ns_n
    \end{equation*}
    over the field $F(s_1,...,s_n)$ is separable with Galois group $S_n$.
\end{thm}

This result says that if there are no relations among the coefficients of a polynomial of degree $n$, then the Galois group of this polynomial over the field generated by its coefficients is the full symmetric group $S_n$. To say the $s_i$ are \Emph{indeterminants} in $x_1,...,x_n$ this means, informally, that there does not exist nonzero polynomial relations over $s_1,...,s_n$.


For $n \geq 5$, there is only one normal subgroup of $S_n$, namely $A_n$, hence there is only one normal subfield of $F(x_1,...,x_n)$ containing $F(s_1,...,s_n)$ and it is an extension of degree $2$. 

\begin{defn}
    Define the \Emph{discriminant} $D$ of $x_1,x_2,...,x_n$ by the formula \begin{equation*}
        D = \prod_{i < j}(x_i-x_j)^2
    \end{equation*}
    Define the discriminant of a polynomial to be the discriminant of the roots of the polynomial.
\end{defn}

The discriminant $D$ is a symmetric function in $x_1,...,x_n$, hence is an element of $K = F(s_1,...,s_n)$.

Recall that $\sigma \in S_n$ is an element of $A_n$ if and only if $\sigma$ fixes the product \begin{equation*}
    \sqrt{D} = \prod_{i<j}(x_i - x_j) \in \Z[x_1,x_2,...,x_n]
\end{equation*}
It follows by the fundamental theorem that if $F$ has characteristic different from $2$, then $\sqrt{D}$ generates the fixed field of $A_n$ and generates a quadratic extension of $K = F(s_1,...,s_n)$. 

\begin{prop}
    If $\ch(F) \neq 2$, then the permutation $\sigma \in S_n$ is an element of $A_n$ if and only if it fixes the square root of the discriminant $D$.
\end{prop}

If the roots of the polynomial $f(x) = x^n + a_{n-1}x^{n-1} + ...+a_1x+ a_0$ are $\alpha_1,\alpha_2,...,\alpha_n$, then the discriminant of $f(x)$ is \begin{equation*}
    D = \prod_{i < j}(\alpha_i - \alpha_j)^2
\end{equation*}
Note that $D = 0$ if and only if $f(x)$ is inseparable, so we have nondistinct roots, or multiples. In any perfect field this implies $f(x)$ is reducible. The discriminant $D$ is symmetric in the roots of $f(x)$, and is hence fixed by automorphisms in the Galois group of $f(x)$. Moreover, we have $\sqrt{D}$ is in the splitting field of $f(x)$, being a product of its roots. Then, if the roots of $f(x)$ are distinct, fix some ordering on the roots and view the Galois group of $f(x)$ as a subgroup of $S_n$, as before.

\begin{prop}
    The Galois group of $f(x) \in F[x]$ is a subgroup of $A_n$ if and only if the discriminant $D \in F$ is the square of an element in $F$
\end{prop}
\begin{proof}
    The Galois group is contained in $A_n$ if and only if every element of the Galois group fixes \begin{equation*}
        \sqrt{D} = \prod_{i < j}(\alpha_i - \alpha_j)
    \end{equation*}
    which is true if and only if $\sqrt{D} \in F$, since Galois groups and their associated fixed fields are unique for a Galois extension by the Fundamental Theorem of Galois.
\end{proof}


\subsection{Polynomials of Degree 2}

Consider the polynomial $x^2+ax+b$ with roots $\alpha,\beta$. The discriminant $D$ for this polynomial is $(\alpha-\beta)^2$, which can be written as a polynomial in the elementary symmetric functions of the roots, which gives $$D = s_1^2-4s_2 = (-a)^2-4(b) = a^2-4b$$
which is the usual discriminant for a quadratic. The polynomial is separable if and only if $a^2-4b \neq 0$. The Galois group is a subgroup of $S_2 \cong \Z/2\Z$, the cyclic group of order $2$, and is trivial ($A_2$ in this case) if and only if $a^2-4b$ is a rational square, which completely determines the possible Galois groups. Moreover, as we found before the splitting field for a non trivial Galois group is the quadratic extension $F(\sqrt{D})$.


\subsection{Polynomials of Degree 3}

Suppose the cubic polynomial is $f(x) = x^3+ax^2+bx+c$. If we make the substitution $x = y-a/3$ the polynomial becomes \begin{equation*}
    g(y) = y^3+py+q
\end{equation*}
which is called the \Emph{depressed} polynomial for $f(x)$, where \begin{equation*}
    p = \frac{1}{3}(3b-a^2),\;\;\;\;q = \frac{1}{27}(2a^3-9ab+27c)
\end{equation*}
The splitting fields for these polynomials are the same since their roots differ by the constant $\frac{a}{3} \in F$ and since the formula for the discriminant involves the differences of roots, we see that these two polynomials also have the same discriminant.

Let the roots of $g(y)$ be $\alpha, \beta,$ and $\gamma$. We first compute the discriminant of this polynomial in terms of $p$ and $q$. Note that \begin{equation*}
    g(y) = (y-\alpha)(y-\beta)(y-\gamma)
\end{equation*}
so that if we differentiate we have \begin{equation*}
    D_yg(y) = (y-\alpha)(y-\beta) + (y-\beta)(y-\gamma) + (y-\alpha)(y-\gamma)
\end{equation*}
Then \begin{align*}
    D_yg(\alpha) &= (\alpha-\beta)(\alpha-\gamma) \\
    D_yg(\beta) &= (\beta-\alpha)(\beta -\gamma) \\
    D_yg(\gamma) &= (\gamma - \alpha)(\gamma - \beta)
\end{align*}
Taking the product we see that \begin{equation*}
    D = [(\alpha-\beta)(\alpha-\gamma)(\beta-\gamma)]^2 = -D_yg(\alpha)D_yg(\beta)D_yg(\gamma)
\end{equation*}


\textbf{(Galois Group of the Cubic)}

\begin{enumerate}
    \item[a.] If the cubic $f(x)$ is reducible, then it splits either into three linear factors or into a linear factor and a irreducible quadratic. In the first case the Galois group is trivial, and in the second case the Galois group is isomorphic to $S_2 \cong \Z/2\Z$.
    \item[b.] If the cubic polynomial $f(x)$ is irreducible, then a root of $f(x)$ generates an extension of degree $3$ over $F$, so the degree of the splitting field over $F$ is divisible by $3$. Since the Galois group is a subgroup of $S_3$, there are only two possibilities, namely $A_3$ or $S_3$. The Galois group is $A_3$ (i.e. cyclic of order $3$) if and only if the discriminant $D$ is a square ($\sqrt{D} \in F$).

        Explicitly, if $D$ is the square of an element of $F$, then the splitting field of the irreducible cubic $f(x)$ is obtained by adjoining any single root of $f(x)$ to $F$. If $D$ is not a square of an element of $F$ then the splitting field of $f(x)$ is of degree $6$ over $F$, hence is the field $F(\theta,\sqrt{D})$ for any of the roots $\theta$ of $f(x)$. This extension is Galois over $F$ with Galois group $S_3$.
\end{enumerate}



\subsection{Polynomials of Degree 4}


Let $f(x) = x^4+ax^3+bx^2+cx+d$ which under the substitution $x = y-a/4$ becomes the depressed quartic \begin{equation*}
    g(y) = y^4 + py^2+qy+r
\end{equation*}
with \begin{align*}
    p &= \frac{1}{8}(-3a^2+8b) \\
    q &= \frac{1}{8}(a^3-4ab+8c) \\
    r &= \frac{1}{256}(-3a^4+16a^2b-64ac+256d)
\end{align*}
Suppose first that $g(y)$ is reducible. If it splits into a linear and a cubic, then the Galois group of the cubic is the Galois group of $g(y)$. Suppose then that $g(y)$ splits into to irreducible quadratics. Then the splitting field is the extension $F(\sqrt{D_1},\sqrt{D_2})$ where $D_1$ and $D_2$ are the discriminants of the two quadratics. If $D_1$ and $D_2$ do not differ by a square factor then this extension is a biquadratic extension and the Galois group is isomorphic to the Klien $4$ subgroup of $S_4$, i.e. $S_2 \times S_2$. If $D_1$ is a square times $D_2$, then this extension is a quadratic extension and the Galois group is isomorphic to $\Z/2\Z$.

Now suppose that $g(y)$ is irreducible. Then we have that the Galois group is transitive on the four roots (recall this is to say that for any pair of roots, there exists an automorphism in the Galois group which sends one to the other). Also recall the Galois group must be a subgroup of $S_4$. The only transitive subgroups of $S_4$ (as considered an action on the set of $4$ letters), hence the only possibilities are $S_4, A_4, D_4 = \langle (1324),(13)(24)\rangle$ and its conjugates, $V_4,$ and $C = \langle (1234)\rangle$ and its conjugates. Let $\alpha_1,\alpha_2,\alpha_3,\alpha_4$ be the roots of $g$. 

(To be continued)



\section{\textsection Solvable and Radical Extensions}


\begin{defn}
    A \Emph{simple radical extension} of a field $F$ is obtained by adjoining the $n$th root of an element $a \in F$. The simple radical extension $F(\sqrt[n]{a})$ is Galois over $F$ if and only if $F$ already contains the $n$th roots of unity.
\end{defn}

\begin{eg}
    Consider $f(x) = x^4 - 2 \in \Q[x]$. Then $\Q(\sqrt[4]{2})$ lets us factor $f(x) = (x^2-\sqrt{2})(x^2+\sqrt{2}) = (x-\sqrt[4]{2})(x+\sqrt[4]{2})(x^2+\sqrt{2})$. But we need $i$ to split $x^2+\sqrt{2}$, which is to say $\Q(\sqrt[4]{2},i)$ is the splitting field for $f(x)$ with \begin{equation*}
        f(x) = (x-\sqrt[4]{2})(x+\sqrt[4]{2})(x-i\sqrt[4]{2})(x+i\sqrt[4]{2})
    \end{equation*}
\end{eg}

\begin{defn}
    The extension $K/F$ is said to be \Emph{cyclic} if it is Galois with a cyclic Galois group.
\end{defn}


\begin{prop}
    Let $F$ be a field of characteristic not dividing $n$ which contains the $n$th roots of unity. The extension $F(\sqrt[n]{a})$ for $a \in F$ is cyclic over $F$ of degree dividing $n$.
\end{prop}
\begin{proof}
    First, the extension $F(\sqrt[n]{a}) = K$ is Galois over $F$ if $F$ contains the $n$th roots of unity since $K$ would then be the splitting field of the separable polynomial $x^n - a$. For $\sigma \in \text{Gal}(K/F)$, $\sigma(\sqrt[n]{a})$ is another root of this polynomial, hence $\sigma(\sqrt[n]{a}) = \zeta_{\sigma}\sqrt[n]{a}$ for some $n$th root of unity $\zeta_{\sigma}$. This gives a map \begin{equation*}
        \map{\text{Gal}(K/F)\rightarrow \mu_n}{\sigma\mapsto \zeta_{\sigma}}
    \end{equation*}
    where $\mu_n$ denotes the group of $n$th roots of unity. Since $F$ contains $\mu_n$, every $n$th root of unity is fixed by every element of $\text{Gal}(K/F)$. Hence \begin{align*}
        \sigma\tau(\sqrt[n]{a}) &= \sigma(\zeta_{\tau}\sqrt[n]{a}) \\
        &= \zeta_{\tau}\sigma(\sqrt[n]{a}) \\
        &= \zeta_{\tau}\zeta_{\sigma}\sqrt[n]{a} = \zeta_{\sigma}\zeta_{\tau}\sqrt[n]{a}
    \end{align*}
    which shows that $\zeta_{\sigma\tau} = \zeta_{\sigma}\zeta_{\tau}$, so the map is a homomorphism. The kernel consists precisely of the automorphisms which fix $\sqrt[n]{a}$, namely the identity. This gives an injection of $\text{Gal}(K/F)$ into the cyclic group $\mu_n$ of order $n$, which proves the proposition.
\end{proof}

Let now $K$ be any cyclic extension of degree $n$ over a field $F$ of characteristic not dividing $n$ which contains the $n$th roots of unity. Let $\sigma$ be a generator for the cyclic Galois group $\text{Gal}(K/F)$:

\begin{defn}
    For $\alpha \in K$ and any $n$th root of unity $\zeta$, define the \Emph{Lagrange resolvent} $(\alpha,\zeta) \in K$ by \begin{equation*}
        (\alpha,\zeta) = \alpha+\zeta\sigma(\alpha) +\zeta^2\sigma^2(\alpha)+...+\zeta^{n-1}\sigma^{n-1}(\alpha)
    \end{equation*}
\end{defn}

If we apply the automorphism $\sigma$ to $(\alpha,\zeta)$ we obtain\begin{equation*}
    \sigma(\alpha,\zeta) = \sigma(\alpha)+\zeta\sigma^2(\alpha)+...+\zeta^{n-1}\sigma^n(\alpha)
\end{equation*}
since $\zeta$ is an element of the base field $F$, and consequently is fixed by $\sigma$. Note we have $\zeta^n = 1$ in $\mu_n$ and $\sigma^n = 1$ in $\text{Gal}(K/F)$, so this can be written as \begin{align*}
    \sigma(\alpha,\zeta) &= \sigma(\alpha)+\zeta\sigma^2(\alpha)+...+\zeta^{-1}\alpha \\
    &= \zeta^{-1}(\alpha+\zeta\sigma(\alpha)+...+\zeta^{n-1}\sigma^{n-1}(\alpha)) \\
    &= \zeta^{-1}(\alpha,\zeta)
\end{align*}
It follows that \begin{equation*}
    \sigma(\alpha,\zeta)^n = (\zeta^{-1})^n(\alpha,\zeta)^n = (\alpha,\zeta)^n
\end{equation*}
so that $(\alpha,\zeta)^n$ is fixed by $\text{Gal}(K/F)$, and hence is an element of $F$ for any $\alpha \in K$.

Let $\zeta$ be a primitive $n$th root of unity. By the linear independence of the automorphisms $1,\sigma,...,\sigma^{n-1}$, there is an element $\alpha \in K$ such that $(\alpha,\zeta)\neq 0$. Iterating our previous result we have $\sigma^i(\alpha,\zeta) = \zeta^{-i}(\alpha,\zeta)$, and it follows that $\sigma^i$ does not fix $(\alpha,\zeta)$ for any $i < n$, since $\zeta$ is primitive. Hence, this element cannot lie in any proper subfield of $K$ by the Fundamental Theorem of Galois Theory, so $K = F((\alpha,\zeta))$. Since we proved $(\alpha,\zeta)^n = a \in F$ above, we have $F(\sqrt[n]{a}) = F((\alpha,\zeta)) = K$. This proves the following converse to the previous proposition.

\begin{prop}\label{prop:cyclicimpliesrad}
    Any cyclic extension of degree $n$ over a field $F$ of characteristic not dividing $n$ which contains the $n$th roots of unity is of the form $F(\sqrt[n]{a})$ for some $a \in F$.
\end{prop}

An extension of the form $F(\sqrt[n]{a})$ is called a \Emph{simple radical extension}. 

\begin{rmk}
    The proofs above are part of something known as \Emph{Krummer Theory}. A group $G$ is said to have \Emph{exponent $n$} if $g^n = 1$ for all $g \in G$. Let $F$ be a field of characteristic not dividing $n$, which contains the $n$th roots of unity. If we take $a_1,...,a_k \in F^{\times}$ then we can see that the extension \begin{equation*}
        F(\sqrt[n]{a_1},...,\sqrt[n]{a_k})
    \end{equation*}
    is an abelian extension of $F$ whose Galois group is of exponent $n$.
\end{rmk}


For simplicity we now consider $F$ as a field of characteristic $0$, but the results hold equally well for finite fields of characteristic not dividing the degree of any of the roots to be taken.

\begin{defn}
    An element $\alpha$ which is algebraic over $F$ can be \Emph{expressed by radicals} or \Emph{solved for in terms of radicals} if $\alpha$ is an element of a field $K$ which can be obtained by a succession of simple radical extensions:\begin{equation*}
        F = K_0 \subset K_1 \subset ... \subset K_s = K
    \end{equation*}
    where $K_{i+1} = K_i(\sqrt[n_i]{a_i})$ for some $a_i \in K_i$, $i = 0,1,...,s-1$. Here $\sqrt[n_i]{a_i}$ denotes some root of the polynomial $x^{n_i}-a_i$. Such a field $K$ will be called a \Emph{root extension} of $F$.
\end{defn}

\begin{defn}
    A polynomial $f(x) \in F[x]$ can be \Emph{solved by radicals} if all of its roots can be solved for in terms of radicals. That is, all of its roots are found in some root extension of $F$.
\end{defn}

\begin{eg}
    Consider $K_0 = \Q$, $K_1 = \Q(\sqrt{17})$, $K_2 = K_1(\sqrt{2(17-\sqrt{17})})$, $K_3 = K_2(\sqrt{2(17+\sqrt{17})})$, and finally \begin{equation*}
        K_4 = K_3\left(\sqrt{17+3\sqrt{17}-\sqrt{2(17-\sqrt{17})}-2\sqrt{2(17+\sqrt{17})}}\right)
    \end{equation*}
    Each of these extensions is a radical extension. This shows that the element \begin{equation*}
        -1+\sqrt{17}+\sqrt{2(17-\sqrt{17})} +2\sqrt{17+3\sqrt{17}-\sqrt{2(17-\sqrt{17})}-2\sqrt{2(17+\sqrt{17})}}
    \end{equation*}
    which is used to construct the 17-gon is constructible starting only with the unit length using a straight edge and compass.
\end{eg}

Note that in considering radical extensions we can always adjoin roots of unity as they are radicals. Thus, cyclic extensions become radical extensions and conversely.

\begin{lem}
    If $\alpha$ is contained in a root extension $K$ ($F = K_0 \subset K_1 \subset ... \subset K_s = K$), then $\alpha$ is contained in a root extension which is Galois over $F$ and where each extension $K_{i+1}/K_i$ is cyclic.
\end{lem}
\begin{proof}
    Let $L$ be the Galois closure of $K$ over $F$. For any $\sigma \in \text{Gal}(L/F)$ we have a chain of subfields \begin{equation*}
        F = \sigma K_0 \subset \sigma K_1 \subset ... \subset \sigma K_s = \sigma K
    \end{equation*}
    where $\sigma K_{i+1}/\sigma K_i$ is again a simple radical extension (since it is generated by the element $\sigma(\sqrt[n_i]{a_i})$, which is a root of $x^{n_i}-\sigma(a_i)$ over $\sigma(K_i)$). It can be shown that the composite of any two root extensions is again a root extension. Then the composite of all the conjugate field $\sigma(K)$ for $\sigma \in \text{Gal}(L/F)$ is again a root extension. Since this field is precisely $L$, we see that $\alpha$ is contained in a Galois root extension.

    Now adjoin to $F$ all of the $n_i$th roots of unity for all the roots $\sqrt[n_i]{a_i}$ of the simple radical extensions in the Galois root extension $K/F$, obtaining the field $F'$, say, and then form the composite of $F'$ with the root extensions: \begin{equation*}
        F \subseteq F' = F'K_0 \subseteq F'K_1 \subseteq ... \subseteq  F'K_s = F'K
    \end{equation*}
    The field $F'K$ is a Galois extension of $F$ since it is the composite of two Galois extensions. The extension from $F$ to $F' = F'K_0$ can be given as a chain of subfields with each individual extension cyclic (this is true for any abelian extension). Each extension $F'K_{i+1}/F'K_i$ is a simple radical extension and since we now have the appropriate roots of unity in the base fields, each of these individual extensions from $F'$ to $F'K$ is a cyclic extension by our first proposition in this section. Hence $F'K/F$ is a root extension which is Galois over $F$ with cyclic intermediate extensions, completing the proof.
\end{proof}

Recall that a finite group $G$ is \Emph{solvable} if there exists a subnormal series \begin{equation*}
    1 = G_s \trianglelefteq G_{s-1} \trianglelefteq ... \trianglelefteq G_0 = G
\end{equation*}
such that $G_{i}/G_{i+1}$ is abelian, or equivalently cyclic since $G$ is finite, for $i = 0,1,2,...,s-1$.


\begin{thm}
    The polynomial $f(x)$ can be solved by radicals if and only if its Galois group is a solvable group.
\end{thm}
\begin{proof}
    Suppose first that $f(x)$ can be solved by radicals. Then each root of $f(x)$ is contained in a Galois root extension over $F$. The composite $L$ of such extensions is again Galois, and by the work in the proof, a root extension. Let $G_i$ be the subgroups corresponding to the subfields $K_i, i = 0,1,...,s-1$. Since $\text{Gal}(K_{i+1}/K_i)= G_i/G_{i+1}$ it follows that the Galois group $G = \text{Gal}(L/F)$ is a solvable group, having a cyclic subnormal series. The field $L$ contains the splitting field of $f(x)$ so the Galois group of $f(x)$ is a quotient group of the solvable group $G$, hence is solvable.

    Suppose now that the Galois group $G$ of $f(x)$ is a solvable group and let $K$ be the splitting field for $f(x)$. Taking the fixed fields of the subgroups in a chain $1 = G_s \trianglelefteq G_{s-1} \trianglelefteq ... \trianglelefteq G_0 = G$ gives a chain \begin{equation*}
        F = K_0 \subset K_1 \subset ... \subset K_s =K
    \end{equation*}
    where $K_{i+1}/K_i, i = 0,1,...,s-1$ is a cyclic extension of degree $n_i$. Let $F'$ be the cyclotomic field over $F$ of all roots of unity of order $n_i, i = 0,1,...,s-1$ and form the composite fields $K_i' = F'K_i$. We obtain a sequence of extensions \begin{equation*}
        F \subseteq F' = F'K_0 \subseteq F'K_1\subseteq ...\subseteq F'K_S = F'K
    \end{equation*}
    The extension $F'K_{i+1}/F'K_i$ is cyclic of degree $n_i, i = 0,1,...,s-1$ by Proposition \ref{prop:galtrans}. Since we now have the appropriate roots of unity in the base fields, each of these cyclic extensions is a simple radical extension by Proposition \ref{prop:cyclicimpliesrad}. Each of the roots of $f(x)$ is therefore contained in the root extension $F'K$ so that $f(x)$ can be solved by radicals.
\end{proof}


\begin{cor}
    The general equation of degree $n$ cannot be solved by radicals for $n \geq 5$.
\end{cor}
\begin{proof}
    For $n\geq 5$ the group $S_n$ is not solvable, so by the previous results we conclude that the general equation cannot be solved by radicals.
\end{proof}

