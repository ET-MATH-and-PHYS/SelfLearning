\documentclass[12pt]{report}
\usepackage[utf8]{inputenc}
\usepackage{import}


%%%%%%%%%%%%%%%%% Book Formatting Comments:

%%%%%%%%%%%%%%%%%%%%%%%%%%%%%%%%%%%%% for Part

%%%%%%%%%%%%%%%%%%%%%% for chapter

%%%%%%%%%%%%%%%%%%%% for section




%%%%%% PACKAGES %%%%%%%
\usepackage{hyperref}
\hypersetup{
    colorlinks,
    citecolor=black,
    filecolor=black,
    linkcolor=black,
    urlcolor=black
}
\usepackage{amsmath} % Math display options
\usepackage{amssymb} % Math symbols
%\usepackage{amsfonts} % Math fonts
%\usepackage{amsthm}
\usepackage{mathtools} % General math tools
\usepackage{array} % Allows you to write arrays
\usepackage{empheq} % For boxing equations
% \usepackage{mathabx}
% \usepackage{mathrsfs}
\usepackage{nameref}
\usepackage{wrapfig}

\usepackage{soul}
\usepackage[normalem]{ulem}

\usepackage{txfonts}
\usepackage{cancel}
\usepackage[toc, page]{appendix}
\usepackage{titletoc,tocloft}
\setlength{\cftchapindent}{1em}
\setlength{\cftsecindent}{2em}
\setlength{\cftsubsecindent}{3em}
%\setlength{\cftsubsubsecindent}{4em}
\usepackage{titlesec}

%\titleformat{\section}
%  {\normalfont\fontsize{25}{15}\bfseries}{\thesection}%{1em}{}
%\titleformat{\section}
%  {\normalfont\fontsize{20}{15}\bfseries}%{\thesubsection}{1em}{}
%\setcounter{secnumdepth}{1}  
  
  

%\newcommand\numberthis{\refstepcounter{equation}\tag{\theequation}} % For equation labelling
\usepackage[framemethod=tikz]{mdframed}

\usepackage{tikz} % For drawing commutative diagrams
\usetikzlibrary{cd}
\usetikzlibrary{calc}
\tikzset{every picture/.style={line width=0.75pt}} %set default line width to 0.75p

\usepackage{datetime}
\usepackage[margin=1.5in]{geometry}
\setlength{\parskip}{1em}
\usepackage{makeidx}         % allows index generation
\usepackage{graphicx}       % standard LaTeX graphics tool
\usepackage{multicol}        % used for the two-column index
\usepackage[bottom]{footmisc}% places footnotes at page bottom

\usepackage{newtxtext}       % 
\usepackage{newtxmath}       % selects Times Roman as basic font
\usepackage{float}
\usepackage{fancyhdr}
\setlength{\headheight}{15pt} 
\pagestyle{fancy}
\lhead[\leftmark]{}
\rhead[]{\leftmark}

%\usepackage{enumitem}

\usepackage{url}
\allowdisplaybreaks

%%%%%% ENVIRONMENTS %%%
\definecolor{purp}{rgb}{0.29, 0, 0.51}
\definecolor{bloo}{rgb}{0, 0.13, 0.80}



%%\newtheoremstyle{note}% hnamei
%{3pt}% hSpace above
%{3pt}% hSpace belowi
%{}% hBody fonti
%{}% hIndent amounti
%{\itshape}% hTheorem head fonti
%{:}% hPunctuation after theorem headi
%{.5em}% hSpace after theorem headi
%{}% hTheorem head spec (can be left empty, meaning ‘normal’)i





% %%%%%%%%%%%%% THEOREM DEFINITIONS

\spnewtheorem{axiom}{Axiom}[chapter]{\bfseries}{\itshape}


\spnewtheorem{construction}{Construction}[chapter]{\bfseries}{\itshape}

\spnewtheorem{props}{Properties}[chapter]{\bfseries}{\itshape}


\renewcommand{\qedsymbol}{$\blacksquare$}


\numberwithin{equation}{section}

\newenvironment{qest}{
    \begin{center}
        \em
    }
    {
    \end{center}
    }

%%%%%% MACROS %%%%%%%%%
%% New Commands
\newcommand{\ip}[1]{\langle#1\rangle} %%% Inner product
\newcommand{\abs}[1]{\lvert#1\rvert} %%% Modulus
\newcommand\diag{\operatorname{diag}} %%% diag matrix
\newcommand\tr{\mbox{tr}\.} %%% trace
\newcommand\C{\mathbb C} %%% Complex numbers
\newcommand\R{\mathbb R} %%% Real numbers
\newcommand\Z{\mathbb Z} %%% Integers
\newcommand\Q{\mathbb Q} %%% Rationals
\newcommand\N{\mathbb N} %%% Naturals
\newcommand\F{\mathbb F} %%% An arbitrary field
\newcommand\ste{\operatorname{St}} %%% Steinberg Representation
\newcommand\GL{\mathbf{GL}} %%% General Linear group
\newcommand\SL{\mathbf{SL}} %%% Special linear group
\newcommand\gl{\mathfrak{gl}} %%% General linear algebra
\newcommand\G{\mathbf{G}} %%% connected reductive group
\newcommand\g{\mathfrak{g}} %%% Lie algebra of G
\newcommand\Hbf{\mathbf{H}} %%% Theta fixed points of G
\newcommand\X{\mathbf{X}} %%% Symmetric space X
\newcommand{\catname}[1]{\normalfont\textbf{#1}}
\newcommand{\Set}{\catname{Set}} %%% Category set
\newcommand{\Grp}{\catname{Grp}} %%% Category group
\newcommand{\Rmod}{\catname{R-Mod}} %%% Category r-modules
\newcommand{\Mon}{\catname{Mon}} %%% Category monoid
\newcommand{\Ring}{\catname{Ring}} %%% Category ring
\newcommand{\Topp}{\catname{Top}} %%% Category Topological spaces
\newcommand{\Vect}{\catname{Vect}_{k}} %%% category vector spaces'
\newcommand\Hom{\mathbf{Hom}} %%% Arrows

\newcommand{\map}[2]{\begin{array}{c} #1 \\ #2 \end{array}}

\newcommand{\Emph}[1]{\textbf{\ul{\emph{#1}}}}




%% Math operators
\DeclareMathOperator{\ran}{Im} %%% image
\DeclareMathOperator{\aut}{Aut} %%% Automorphisms
\DeclareMathOperator{\spn}{span} %%% span
\DeclareMathOperator{\ann}{Ann} %%% annihilator
\DeclareMathOperator{\rank}{rank} %%% Rank
\DeclareMathOperator{\ch}{char} %%% characteristic
\DeclareMathOperator{\ev}{\bf{ev}} %%% evaluation
\DeclareMathOperator{\sgn}{sign} %%% sign
\DeclareMathOperator{\id}{Id} %%% identity
\DeclareMathOperator{\supp}{Supp} %%% support
\DeclareMathOperator{\inn}{Inn} %%% Inner aut
\DeclareMathOperator{\en}{End} %%% Endomorphisms
\DeclareMathOperator{\sym}{Sym} %%% Group of symmetries


%% Diagram Environments
\iffalse
\begin{center}
    \begin{tikzpicture}[baseline= (a).base]
        \node[scale=1] (a) at (0,0){
          \begin{tikzcd}
           
          \end{tikzcd}
        };
    \end{tikzpicture}
\end{center}
\fi




\newdateformat{monthdayyeardate}{%
    \monthname[\THEMONTH]~\THEDAY, \THEYEAR}
%%%%%%%%%%%%%%%%%%%%%%%

%%% Specific Macros %%%


%%%%%% BEGIN %%%%%%%%%%


\begin{document}

%%%%%% TITLE PAGE %%%%%

\begin{titlepage}
    \centering
    \scshape
    \vspace*{\baselineskip}
    \rule{\textwidth}{1.6pt}\vspace*{-\baselineskip}\vspace*{2pt}
    \rule{\textwidth}{0.4pt}
    
    \vspace{0.75\baselineskip}
    
    {\LARGE Real Analysis: A Complete Guide}
    
    \vspace{0.75\baselineskip}
    
    \rule{\textwidth}{0.4pt}\vspace*{-\baselineskip}\vspace{3.2pt}
    \rule{\textwidth}{1.6pt}
    
    \vspace{2\baselineskip}
    Real Analysis \\
    \vspace*{3\baselineskip}
    \monthdayyeardate\today \\
    \vspace*{5.0\baselineskip}
    
    {\scshape\Large Elijah Thompson, \\ Physics and Math Honors\\}
    
    \vspace{1.0\baselineskip}
    \textit{Solo Pursuit of Learning}
\end{titlepage}

%%%%%%%%%%%%%%%%%%%%%%%
\tableofcontents

%%%%%%%%%%%%%%%%%%%%%%%%%%%%%%%%%%%%% Part 1
\part{Single Variable Analysis}

%%%%%%%%%%%%%%%%%%%%%% - P1.Chapter 1
\chapter{Topology and Construction of the Real Line}

\section{The Axiom of Completeness}

\begin{defn}{The Reals}{}
    The real number system $\R$ is an \Emph{ordered field} which contains $\Q$ as a subfield, which satisfies the \Emph{axiom of choice}. In particular, the real numbers is a set $\R$ with two binary operations $+$ and $\cdot$, two distinct elements $0$ and $1$, and a subset $\mathbb{P}$ of positive numbers satisfying the following $13$ postulates:
    \begin{enumerate}
        \item Addition is associative: $\forall a,b,c \in \R, a+(b+c) = (a+b)+c$
        \item The number $0$ is an additive identity: $\forall a \in \R, a+0 = 0+a = a$
        \item Additive inverses exist: $\forall a \in \R;\exists (-a) \in \R\;s.t.\;a+(-a) = (-a)+a=0$
        \item Addition is commutative: $\forall a,b \in \R, a+b = b+a$
        \item Multiplication is associative: $\forall a,b,c \in \R, a\cdot (b\cdot c) = (a \cdot b) \cdot c$
        \item The number $1$ is a multiplicative identity: $\forall a\in \R a\cdot 1 = 1\cdot a = a$
        \item Multiplicative inverses exist: $\forall a \neq 0;\exists a^{-1} \in \R\;s.t.\;a\cdot a^{-1} = a^{-1} \cdot a = 1$
        \item Multiplication is commutative: $\forall a,b \in \R, a\cdot b = b \cdot a$
        \item The distributive law: $\forall a,b,c \in \R, a\cdot (b+c) = a\cdot b + a \cdot c$
        \item The trichotomy of $\mathbb{P}$: for every $a \in \R$, exactly one of the following holds: $a = 0, a \in \mathbb{P}, (-a) \in \mathbb{P}$
        \item Closure under addition: if $a \in \mathbb{P}$ and $b \in \mathbb{P}$, then $a+b \in \mathbb{P}$
        \item Closure under multiplication: if $a \in \mathbb{P}$ and $b \in \mathbb{P}$, then $a\cdot b \in \mathbb{P}$
        \item (to be added)
    \end{enumerate}
    From positive postulates we can define the order relations $>, <, \geq, \leq$ on $\R$ for $a,b \in \R$ by \begin{enumerate}
        \item $a > b$ if $a-b \in \mathbb{P}$
        \item $a < b$ if $b > a$ 
        \item $a \geq b$ if $a > b$ or $a = b$
        \item $a \leq b$ if $a < b$ or $a = b$
    \end{enumerate}
    Note in particular $a > 0$ if and only if $a \in \mathbb{P}$.
\end{defn}

\begin{rmk}{}{}
    A few points which follow from the postulates are:\begin{enumerate}
        \item Finite sums such as $a_1+a_2+...+a_n$ are well defined
        \item The additive identity is unique (also multiplicative)
        \item Additive inverses are unique (also multiplicative)
        \item Subtraction can be defined 
        \item $a \cdot b = b \cdot c \iff a = 0\lor b = c$
        \item $a\cdot b = \iff a = 0 \lor b = 0$
        \item $a-b = b-a \iff a = b$
        \item A ``well behaved" order relation can be defined.
        \item The ``absolute value" function $a \mapsto |a|$ can be defined by $$|a| = \left\{\begin{array}{ll} a, & a \geq 0 \\ -a, & a \leq 0 \end{array}\right.$$ and for all $a,b \in \R$, the triangle inequality $$|a+b| \leq |a| + |b|$$ holds
    \end{enumerate}
\end{rmk}


\begin{axi*}{Axiom of Completeness}{}
    Every non-empty subset of the real numbers that is bounded above has a least upper bound.
\end{axi*}

\subsection{Upper and Lower Bounds}

\begin{defn}{Bounds}{}
    A set $A \subseteq \R$ is \Emph{bounded above} if there exists a number $b \in \R$ such that $a \leq b$ for all $a \in A$. The number $b$ is called an \Emph{upper bound} for $A$.


    Similarly, the set $A$ is \Emph{bounded below} if there exists a \Emph{lower bound} $l \in \R$ satisfying $l \leq a$ for every $a \in A$.
\end{defn}

\begin{defn}{Least Upper Bound}{}
    A real number $s$ is the \Emph{least upper bound} for a set $A \subseteq \R$ if it meets the following two criteria: \begin{enumerate}
        \item $s$ is an upper bound for $A$;
        \item if $b$ is any upper bound for $A$, then $s \leq b$.
    \end{enumerate}
    The least upper bound of a set $A$ is also called the \Emph{supremum} of $A$, and denoted by $\sup A$.
\end{defn}

\begin{defn}{Greatest Lower Bound}{}
    A real number $i$ is the \Emph{Greatest Lower bound} for a set $A \subseteq \R$ if it meets the following two criteria: \begin{enumerate}
        \item $i$ is a lower bound for $A$;
        \item if $b$ is any lower bound for $A$, then $b \leq i$.
    \end{enumerate}
    The greatest lower bound of a set $A$ is also called the \Emph{infemum} of $A$, and denoted by $\inf A$.
\end{defn}


\begin{rmk}{}{}
    From the definitions we assert that the least upper bound and greatest lower bound of a set, if they exist, are unique.
\end{rmk}


\begin{eg*}{}{}
    Consider the set $$A = \left\{\frac{1}{n}:n\in\N\right\} = \left\{1,\frac{1}{2},\frac{1}{3},...\right\}$$ The set $A$ is bounded above and below. Moreover, the least upper bound of $A$ is $\sup A = 1$, which is in $A$, while $\inf A = 0$, which is not in $A$.
\end{eg*}


\begin{defn}{Max and Min}{}
    A real number $a_0$ is a \Emph{maximum} of a set $A$ if $a_0$ is an element of $A$ and $a_0 \geq a$ for all $a \in A$. Similarly, a number $a_1$ is a \Emph{minimum} of $A$ if $a_1 \in A$ and $a_1 \leq a$ for all $a \in A$.
\end{defn}

\begin{eg*}{}{}
    Consider the open interval $(0,2)$, and the closed interval $[0,2]$. Note both sets are bounded above and below, and both have the same infimum and supremum, namely $\inf = 0$ and $\sup = 2$. However, $[0,2]$ has both a maximum and a minimum, namely its infimum and supremum, while $(0,2)$ has neither.
\end{eg*}


\begin{eg*}{}{}
    Let $A \subseteq \R$ be a non-empty and bounded above set, and let $c \in \R$. Define the set $c+A$ by $$c+A:=\{c+a:a \in A\}$$ Then I claim that $\sup(c+A) = c+\sup A$.
    \begin{proof*}{}{}
        Let $\alpha = c+\sup A$. First let us show that $\alpha$ is an upper bound of $c+A$. Indeed, for $x \in c+A$ we can write $x = c+a$ for some $a \in A$ by definition. Then, by definition we have that $a \leq \sup A$. Thus, adding $c$ to both sides we obtain $$x = c+a \leq c+\sup A = \alpha$$
        Therefore, as $x$ was arbitrary $\alpha$ is indeed an upper bound of $c+A$. Now, suppose $b$ is an upper bound of $c+A$. Then, $c+a \leq b$ for all $c+a \in c+ A$, so in particular $a \leq b - c$ for all $a \in A$. Then, $b-c$ is an upper bound for $A$, so as $\sup A$ is the least upper bound of $A$ we have that $\sup A \leq b-c$. Hence, we conclude that $\alpha = c+\sup A \leq b$. Thus $\alpha$ satisfies the axioms of a least upper bound for $c+A$, and we conclude that $\sup(c+A) = c+\sup A$.
    \end{proof*}
\end{eg*}


\begin{lem}{}{}
    Assume $s \in \R$ is an upper bound for a set $A \subseteq \R$. Then $s = \sup A$ if and only if, for every choice $\epsilon > 0$, there exists an element $a \in A$ satisfying $s-\epsilon < a$.
\end{lem}
\begin{proof*}{}{}
    Let $s \in \R$ be an upper bound for a set $A \subseteq \R$. 

    ($\implies$) First, suppose that $s = \sup A$, and choose $\epsilon \in \R$ with $\epsilon > 0$. Then $s-\epsilon$ is not an upper bound of $A$. Indeed, if $s-\epsilon$ was an upper bound then $s \leq s - \epsilon$ which implies that $\epsilon \leq 0$, but by assumption $\epsilon > 0$. Thus, there must exist $a \in A$ such that $s - \epsilon < a$, satisfying the implication.

    ($\impliedby$) Conversely, suppose that for all $\epsilon > 0$ there exists $a \in A$ such that $s - \epsilon < a$. Now, suppose that $b$ is an upper bound of $A$, and towards a contradiction suppose $s > b$. Then $s - b > 0$, so there exists $a \in A$ such that $s - (s-b) < a$. In particular, $b < a$. However, $a \in A$ and $b$ is an upper bound of $A$ by assumption, so $b < a$ is a contradiction. Therefore we conclude that $s \leq b$, so $s$ is the supremum of $A$ by definition.
\end{proof*}


\begin{thm}{Archimedean Property for the Reals}{}
    For all $x,y > 0$ in $\R$, there exists $n \in \N$ such that $nx > y$.
\end{thm}
\begin{proof*}{}{}
    Towards a contradiction suppose $nx \leq y$ for all $n \in \N$. Then the set $\{nx: n \in \N\}$ is bounded above by $y$. Thus, by the least upper bound property of $\R$ we have a supremum $\alpha \in \R$. Then for all $n \in \N$ $\alpha \geq nx$. In particular, $\alpha \geq (n+1)x$ for all $n \in \N$, so $\alpha - x \geq nx$ for all $n \in \N$. But this implies that $\alpha - x$ is also an upper bound of the set, which contradicts the fact that $\alpha$ is the least upper bound. Thus, we must have that $nx > y$ for some $n \in \N$, as claimed.
\end{proof*}

\begin{cor}{}{}
    $\N$ is not bounded above.
\end{cor}


\begin{cor}{}{}
    For any $\epsilon > 0$ there is a natural number $n$ with $1/n < \epsilon$.
\end{cor}
\begin{proof*}{}{}
    Consider $0<1/\epsilon \in \R$ and $1 \in \R$. Then by the Archimedean Property of $\R$ there exists $n \in \N$ such that $1\cdot n > 1/\epsilon$. In particular, we have that $n > 0$, so $\epsilon > 1/n$, completing the proof.
\end{proof*}




\section{Limits}

\begin{rmk}{Motivating Definition}{}
    The function $f$ approaches the limit $l \in \R$ near $a \in \R$, if we can make $f(x)$ as ``close as we like" to $l$ by requiring that $x$ be ``sufficiently close to," but unequal to, $a.$
\end{rmk}

\begin{defn}{Limit}{}
    A real valued function $f:\R\rightarrow \R$ \Emph{approaches the limit $l$ near $a$} if for every $\epsilon > 0$ there is some $\delta > 0$ such that, for all $x \in \R$, if $0 < |x-a| < \delta$, then $|f(x) - l| < \epsilon$.
\end{defn}

\begin{nota*}{}{}
    We denote the number $l$ which a function $f$ approaches near $a \in \R$ by $\lim\limits_{x\rightarrow a}f(x)$, read \Emph{the limit of $f(x)$ as $x$ approaches $a$}.
\end{nota*}


\begin{lem}{}{}
    \leavevmode
    \begin{enumerate}
        \item If $$|x-x_0| < \frac{\epsilon}{2}\;and\;|y-y_0| < \frac{\epsilon}{2}$$
            then $$|(x+y) - (x_0+y_0)| < \epsilon$$
        \item If $$|x-x_0| < \min\left(1,\frac{\epsilon}{2(|y_0|+1)}\right)\;and\;|y-y_0| < \frac{\epsilon}{2(|x_0+1)}$$
            then $$|xy-x_0y_0| < \epsilon$$
        \item If $y_0 \neq 0$ and $$|y-y_0| < \min\left(\frac{|y_0|}{2},\frac{\epsilon|y_0|^2}{2}\right)$$
            then $y\neq 0$ and $$\left|\frac{1}{y} - \frac{1}{y_0}\right|$$
    \end{enumerate}
\end{lem}
\begin{proof*}{}{}
    (1) Suppose $|x-x_0| < \frac{\epsilon}{2}\;and\;|y-y_0| < \frac{\epsilon}{2}$. Then it follows that \begin{align*}
        |(x+y) - (x_0+y_0)| &= |(x-x_0)+(y-y_0)| \\
        &\leq |x-x_0| + |y-y_0| \\
        &< \frac{\epsilon}{2} + \frac{\epsilon}{2} \\
        &= \epsilon
    \end{align*}
    as desired.


    (2) Next, suppose $|x-x_0| < \min\left(1,\frac{\epsilon}{2(|y_0|+1)}\right)\;and\;|y-y_0| < \frac{\epsilon}{2(|x_0+1)}$. Note that as $|x-x_0| < 1$ we have that $|x| - |x_0| < 1$ so $|x| < 1+|x_0|$. It follows that \begin{align*}
        |xy-x_0y_0| &= |xy-xy_0+xy_0-x_0y_0| \\
        &\leq |x||y-y_0| + |y_0||x-x_0| \\
        &< (|x_0|+1)\frac{\epsilon}{2(|x_0|+1)} + (|y_0| + 1)\frac{\epsilon}{2(|y_0|+1)} \\
        &= \frac{\epsilon}{2} + \frac{\epsilon}{2} \\
        &= \epsilon
    \end{align*}


    (3) Suppose $y_0 \neq 0$ and $|y-y_0| < \min\left(\frac{|y_0|}{2},\frac{\epsilon|y_0|^2}{2}\right)$. Then note that as $|y-y_0| < \frac{|y_0|}{2}$ $-\frac{|y_0|}{2} < y-y_0 < \frac{|y_0|}{2}$. If $y_0 > 0$ we have that $y_0 = |y_0|$ so $\frac{|y_0|}{2} < y < \frac{3|y_0|}{2}$. On the other hand if $y_0 < 0$ then $y_0 = -|y_0|$ so $-\frac{3|y_0|}{2} < y < -\frac{|y_0|}{2}$. In either case we have that $|y| > \frac{|y_0|}{2} > 0$, so $y \neq 0$. Then it follows that \begin{align*}
        \left|\frac{1}{y} - \frac{1}{y_0}\right| &= \left|\frac{y - y_0}{yy_0}\right| \\
        &< \frac{\epsilon|y_0|^2}{2}\cdot \frac{1}{|y_0|}\cdot \frac{2}{|y_0|} \\
        &= \epsilon
    \end{align*}
    as claimed.
\end{proof*}


\begin{thm}{Limit Laws}{limlaws}
    If $\lim\limits_{x\rightarrow a}f(x) = l$ and $\lim\limits_{x\rightarrow a}g(x) = m$, then \begin{enumerate}
        \item $\lim\limits_{x\rightarrow a}(f+g)(x) = l+m$;
        \item $\lim\limits_{x\rightarrow a}(f\cdot g)(x) = l\cdot m$;
        \item Moreover, if $m \neq 0$, then $\lim\limits_{x\rightarrow a}\left(\frac{1}{g}\right)(x) = \frac{1}{m}$
    \end{enumerate}
\end{thm}
\begin{proof*}{}{}
    Suppose that $\lim\limits_{x\rightarrow a}f(x) = l$ and $\lim\limits_{x\rightarrow a}g(x) = m$. Let $\epsilon > 0$.


    (1) Then since $\lim\limits_{x\rightarrow a}f(x) = l$ and $\lim\limits_{x\rightarrow a}g(x) = m$ there exist $\delta_1,\delta_2 > 0$ such that $|f(x) - l| < \frac{\epsilon}{2}$ if $0 < |x-a| < \delta_1$ and $|g(x) - m| < \frac{\epsilon}{2}$ if $0 < |x-a| < \delta_2$. Choose $\delta = \min(\delta_1,\delta_2)$. Then it follows that for $0 < |x-a| < \delta$: \begin{align*}
        |(f+g)(x) - (l+m)| &= |(f(x) - l) + (g(x) - m)| \\
        &\leq |f(x) - l) + |g(x) - m| \\
        &< \frac{\epsilon}{2} + \frac{\epsilon}{2} \\
        &= \epsilon
    \end{align*}
    Thus, we have that by definition $\lim\limits_{x\rightarrow a}(f+g)(x) = l+m$.


    (2) Now, fix $\epsilon_1 = \min\left(1,\frac{\epsilon}{2(|m|+1)}\right)$ and $\epsilon_2 = \frac{\epsilon}{2(|l| + 1)}$. Then since $\lim\limits_{x\rightarrow a}f(x) = l$ and $\lim\limits_{x\rightarrow a}g(x) = m$ there exist $\delta_1, \delta_2> 0$ such that if $0<|x-a| < \delta_1$ then $|f(x) - l| < \epsilon_1$ and if $0<|x-a| < \delta_2$ then $|g(x) - m| < \epsilon_2$. Choose $\delta = \min(\delta_1,\delta_2)$. It follows that for $0<|x-a| < \delta$, we have $|f(x)g(x) - lm| < \epsilon$, so by definition $\lim\limits_{x\rightarrow a}(f\cdot g)(x) = l\cdot m$.



    (3) Now, suppose $m \neq 0$. Fix $\epsilon_1 = \min\left(\frac{|m|}{2},\frac{\epsilon|m|^2}{2}\right)$. Then as $\lim\limits_{x\rightarrow a}g(x) = m$, there exists $\delta > 0$ such that if $|x-a| < \delta$, $|g(x) - m| < \epsilon_1$. By the previous Lemma we have that if $|x-a| < \delta$, $g(x) \neq 0$ and $\left|\frac{1}{g(x)} - \frac{1}{m}\right| < \epsilon$. Hence, by definition $\lim\limits_{x\rightarrow a}\left(\frac{1}{g}\right)(x) = \frac{1}{m}$, as desired.
\end{proof*}


\begin{defn}{Limits from Above and Below}{}
    The \Emph{limit from above} for a function $f$ as $x$ goes to $a$ is denoted by $\lim\limits_{x\rightarrow a^+}f(x) = l$, which means for every $\epsilon > 0$ there is a $\delta > 0$ such that for all $x$, $$if\;0 < x-a < \delta,\;then\;|f(x) - l|< \epsilon$$
    where $0 < x-a < \delta$ is equivalent to $0 < |x-a| < \delta$ and $x > a$.

    The \Emph{limit from below} for $f$ as $x$ goes to $a$ is denoted by $\lim\limits_{x\rightarrow a^-}f(x) = l$, and means that for every $\epsilon > 0$ there is a $\delta > 0$ such that, for all $x$, $$if\;0 < a-x < \delta,\;then\;|f(x) - l| < \epsilon$$
\end{defn}


\begin{rmk}{}{}
    For a function $f:\R\rightarrow \R$, $\lim_{x\rightarrow a}f(x)$ exists if and only if $\lim\limits_{x\rightarrow a^+}f(x)$ and $\lim\limits_{x\rightarrow a^-}f(x)$ both exist and are equal.
\end{rmk}


\begin{defn}{Limits at Infinity}{}
    A \Emph{limit at infinity} is denoted by $\lim\limits_{x\rightarrow \infty}f(x) = l$, and means that for every $\epsilon > 0$ there is $M \in \R$ such that for all $x$, $$if\;x>M,\;then\;|f(x) - l| < \epsilon$$

    A limit at negative infinity is defined analogously, replacing $x> M$ with $x < M$.
\end{defn}


\begin{defn}{}{}
    We define $\lim\limits_{x\rightarrow a}f(x) = \infty$ to mean that for all $N \in \R$, there exists $\delta > 0$ such that, for all $x \in \R$, if $0 < |x-a| < \delta$, then $f(x) > N$. (the case for $-\infty$ is defined similarly)
\end{defn}



\section{Continuous Functions}


\begin{defn}{Continuity}{}
    Let $f:\R\rightarrow \R$ be a real function. Then $f$ is said to be \Emph{continuous at a point $a$} if \begin{equation}
        \lim\limits_{x\rightarrow a}f(x) = f(a)
    \end{equation}
\end{defn}


\begin{thm}{}{}
    If $f$ and $g$ are continuous at a point $a$, then \begin{enumerate}
        \item $f+g$ is continuous at $a$
        \item $f\cdot g$ is continuous at $a$
        \item Moreover, if $g(a) \neq 0$, then $1/g$ is continuous at $a$.
    \end{enumerate}
\end{thm}
\begin{proof*}{}{}
    Suppose $f$ and $g$ are continuous at a point $a$. Then by the \nameref{thm:limlaws} theorem, we have that as $\lim\limits_{x\rightarrow a}f(x) = f(a)$ and $\lim\limits_{x\rightarrow a}g(x) = g(a)$, $$\lim\limits_{x\rightarrow a}(f+g)(x) = f(a) + g(a) = (f+g)(a)$$
    Hence, $f+g$ is continuous at $a$. Similarly, again by the \nameref{thm:limlaws} theorem, we have that $$\lim\limits_{x\rightarrow a}(f\cdot g)(x) = f(a) \cdot g(a) = (f\cdot g)(a)$$
    Thus, $f\cdot g$ is continuous at $a$.
    Finally, if $g(a) \neq 0$, the $$\lim\limits_{x\rightarrow a}(1/g)(x) = 1/g(a) = (1/g)(a)$$
    so $1/g$ is continuous at $a$.
\end{proof*}


\begin{thm}{}{}
    If $g$ is continuous at $a$, and $f$ is continuous at $g(a)$, then $f\circ g$ is continuous at $a$.
\end{thm}
\begin{proof*}{}{}
    Let $\epsilon > 0$. Then by continuity of $f$ there exists $\delta_1 > 0$ such that if $|g(x) - g(a)| < \delta_1$, then $$|f(g(x)) - f(g(a))| < \epsilon$$
    Then, by the continuity of $g$, there exists $\delta > 0$ such that if $|x-a| < \delta$, then $|g(x) - g(a)| < \delta_1$. Thus, if $|x-a| < \delta$ we have that $$|(f\circ g)(x) - (f\circ g)(a)| = |f(g(x)) - f(g(a))| < \epsilon$$
    proving continuity of $f\circ g$ at $a$.
\end{proof*}


\begin{defn}{}{}
    A function $f$ is called \Emph{continuous on} an open interval $(a,b)$, if $f$ is continuous at $x$ for all $x \in (a,b)$.

    A function $f$ is called \Emph{continuous on} a closed interval $[a,b]$ if \begin{enumerate}
        \item $f$ is continuous at $x$ for all $x \in (a,b)$
        \item $\lim\limits_{x\rightarrow a^+}f(x) = f(a)$ and $\lim\limits_{x\rightarrow b^-}f(x) = f(b)$
    \end{enumerate}

    In general, a function $f$ is \Emph{continuous} if it is continuous at $x$ for all $x$ in its domain.
\end{defn}


\begin{thm}{}{}
    Suppose $f$ is continuous at $a$, and $f(a) > 0$. Then $f(x) > 0$ for all $x$ in some interval containing $a$; more precisely, there is a number $\delta > 0$ such that $f(x) > 0$ for all $x$ satisfying $|x-a| < \delta$. Similarly, if $f(a) < 0$, then there is a number $\delta > 0$ such that $f(x) < 0$ for all $x$ satisfying $|x-a| < \delta$.
\end{thm}
\begin{proof*}{}{}
    Suppose $f$ is continuous at $a$. Let $\epsilon = \frac{f(a)}{2} > 0$. Then by continuity there exists $\delta > 0$ such that if $|x-a| < \delta$, $|f(x) - f(a)| < \epsilon$. Then we have that $-\epsilon < f(x) - f(a) < \epsilon$, so $f(x) > \epsilon > 0$, satisfying the claim. The case for $f(a) < 0$ is proved analogously.
\end{proof*}


\subsection{Important Theorems and Results on Continuity}


\begin{thm}{}{}
    If $f$ is continuous on a closed interval $[a,b]$ (a compact set) and $f(a) < 0 < f(b)$, then there is some $x \in [a,b]$ such that $f(x) = 0$.
\end{thm}
\begin{proof*}{}{}
    Consider an interval $[a,b]$ such that $f(a) < 0 < f(b)$. Define the set $$a := \{x\in \R:a\leq x \leq b,\text{and $f$ is negative on $[a,x]$}\}$$
    Clearly $A \neq \emptyset$ as $a \in A$. In fact, there exists some $\delta > 0$ such that $A$ contains all points $x \in \R$ satisfying $a \leq x < a+\delta$, since $f$ is continuous on $[a,b]$ and $f(a) < 0$. Similarly, $b$ is an upper bound for $A$ and, in fact, there is a $\delta > 0$ such that all points satisfying $b-\delta < x \leq b$ are upper bounds for $A$.

    Thus, applying the Least Upper Bound property of $\R$, $A$ has a least upper bound $\alpha$ and $a<\alpha < b$. We wish to show that $f(\alpha) = 0$. First, if $f(\alpha) < 0$, then by a previous result there is a $\delta > 0$ such that $f(x) < 0$ for $\alpha - \delta < x < \alpha + \delta$. In particular, there is some number $x_0 \in A$ satisfying $\alpha - \delta < x < \alpha$ since $\alpha$ is the supremum of $A$. Thus $f$ is negative on the whole interval $[a,x_0]$. But, if $x_1 \in (\alpha, \alpha+\delta$, then $f$ is also negative on the whole interval $[x_0,x_1]$. Therefore $f$ is negative on the interval $[a,x_1]$ so $x_1 \in A$. But, this contradicts the fact that $\alpha$ is an upper bound for $A$, so $f(\alpha) < 0$ must be false.


    Suppose, on the other hand, that $f(\alpha) > 0$. Then there is a number $\delta > 0$ such that $f(x) > 0$ for all $\alpha - \delta < x < \alpha + \delta$. Now there is some number $x_0 \in A$ such that $\alpha - \delta < x_0 < \alpha$ as $\alpha$ is presumed to be the supremum of $A$. This means that $f$ is negative on the whole interval $[a,x_0]$, which is impossible since $f(x_0) > 0$. Thus, the assumption $f(\alpha) > 0$ leads to a contradiction, leaving $f(\alpha) = 0$ as the only possible alternative. 
\end{proof*}

\begin{lem}{}{}
    If $f$ is continuous at $a$, then there is a number $\delta > 0$ such that $f$ is bounded on the interval $(a-\delta, a+\delta)$.
\end{lem}
\begin{proof*}{}{}
    Since $f$ is continuous at $a$ we have that $\lim\limits_{x\rightarrow a}f(x) = f(a)$. Then, fix $\epsilon = 1$. By continuity it follows that there exists $\delta > 0$ such that for all $x \in (a-\delta, a+\delta)$, $|f(x) - f(a)| < 1$. In particular, we have that $f(x) < f(a) + 1$, so $f$ is bounded above by $f(a)+1$ on $(a-\delta,a+\delta)$. Moreover, $f(x) > f(a) - 1$, so $f$ is bounded below by $f(a) -1$ on $(a-\delta, a+\delta)$. Thus $f$ is bounded on the interval $(a-\delta,a+\delta)$ as claimed.
\end{proof*}


\begin{cor}{}{}
    If $\lim\limits_{x\rightarrow a^+}f(x) = f(a)$ then there exists $\delta > 0$ such that $f$ is bounded on the interval $[a,a+\delta)$. Moreover, if $\lim\limits_{x\rightarrow b^-}f(x) = f(b)$ then there exists $\delta > 0$ such that $f$ is bounded on the interval $(b-\delta, b]$.
\end{cor}


\begin{thm}{}{}
    If $f$ is continuous on a closed interval $[a,b]$ (a compact set), then $f$ is bounded above on $[a,b]$, that is, there is some number $M \in \R$ such that $f(x) \leq M$ for all $x \in [a,b]$ (consequence of the continuous image of a compact set being compact and the Heine-Borel Theorem).
\end{thm}
\begin{proof*}{}{}
    Define the set $$A:= \{x\in [a,b]:\text{$f$ is bounded above on $[a,x]$}\}$$
    Clearly $A \neq \emptyset$ as $a \in A$, and $A$ is bounded above by $B$, so $A$ has a least upper bound $\alpha \in \R$. We wish to show that $\alpha = b$. Suppose towards a contradiction that $\alpha < b$. Then there exists $\delta > 0$ such that $f$ is bounded on $(\alpha-\delta, \alpha + \delta)$ since $f$ is continuous on $[a,b]$, so in particular $f$ is continuous at $\alpha$. Since $\alpha$ is the least upper bound of $A$ there is some $x_0 \in A$ satisfying $\alpha - \delta < x_0 < \alpha$. This implies that $f$ is bounded on $[a,x_0]$. But, if $x_1$ is any number with $\alpha < x_1 < \alpha + \delta$, then $f$ is also bounded on $[x_0,x_1]$. Therefore $f$ is bounded on $[a,x_1]$ so $x_1 \in A$, contradicting the fact that $\alpha$ is an upper bound for $A$. This contradiction shows that $\alpha = b$. Now, there is a $\delta > 0$ such that $f$ is bounded on $(b-\delta, b]$. There is $x_0 \in A$ such that $b - \delta < x_0 < b$, since $\alpha = b$. Thus $f$ is bounded on $[a,x_0]$, and also on $[x_0,b]$, so $f$ is bounded on $[a,b]$, completing the proof.
\end{proof*}




\begin{thm}{}{}
    If $f$ is continuous on a closed interval $[a,b]$, then there is some number $y \in [a,b]$ such that $f(y) \geq f(x)$ for all $x \in [a,b]$.
\end{thm}
\begin{proof*}{}{}
    From the previous theorem we know that $f$ is bounded on $[a,b]$, so the set $\{f(x):x\in[a,b]\}$ is bounded. This set is obviously non-empty, so it has a least upper bound $\alpha \in \R$. Since $\alpha \geq f(x)$ for all $x \in [a,b]$, it suffices to show that $\alpha = f(y)$ for some $y \in [a,b]$. Suppose instead that $\alpha \neq f(y)$ for all $y \in [a,b]$. Then the function $g$ defined by $$g(x) = \frac{1}{\alpha - f(x)}, x \in [a,b]$$ is continuous on $[a,b]$ since the denominator is never zero and is the sum of continuous functions. On the other hand, $\alpha$ is the least upper bound of $\{f(x):x\in [a,b]\}$ so for every $\epsilon > 0$ there exists $x \in [a,b]$ such that $\alpha - \epsilon < f(x)$, so $\alpha - f(x) < \epsilon$. This in turn implies that for every $\epsilon > 0$ there exists $x \in [a,b]$ with $g(x) > 1/\epsilon$. But, this implies that $g$ is not bounded on $[a,b]$, contradicting the previous theorem as $g$ is assumed to be continuous. Hence, $g$ is not continuous, and i particular $\alpha - f(y) = 0$ for some $y \in [a,b]$.
\end{proof*}


\begin{namthm}{Intermediate Value Theorem}{intval}
    If $f$ is continuous on $[a,b]$ and $f(a) < c < f(b)$, then there is some $x \in [a,b]$ such that $f(x) = c$ (continuous image of a connected set is connected).

    Moreover, if $f(a) > c > f(b)$, then there is some $x \in [a,b]$ such that $f(x) = c$.
\end{namthm}

\begin{thm}{}{}
    If $f$ is continuous on $[a,b]$, then $f$ is bounded below on $[a,b]$, that is, there is some number $M \in \R$ such that $f(x) \geq M$ for all $x \in [a,b]$.
\end{thm}

\begin{thm}{}{}
    If $f$ is continuous on $[a,b]$, then there is some $y \in [a,b]$ such that $f(y) \leq f(x)$ for all $x \in [a,b]$.
\end{thm}


\begin{cor}{}{}
    For all $\alpha \in \mathbb{P}$, so $\alpha >0$, there exists $x \in \R$ such that $x^2 = \alpha$.
\end{cor}
\begin{proof*}{}{}
    Consider the function $f(x) = x^2$, which is certainly continuous over $\R$. Consider $\alpha \in \mathbb{P}$. Then there exists $b > 0$ such that $f(b) > \alpha$. Indeed, if $\alpha > 1$ we can take $b = \alpha$, and if $\alpha < 1$ we can take $b = 1$. Then, $f$ is defined on the closed interval $[0,b]$ and $f(0) = 0 < \alpha < f(b)$. Therefore, by the \nameref{thmname:intval} there exists $c \in [0,b]$ such that $f(c) = \alpha$. In particular, $c^2 = \alpha$.
\end{proof*}



\begin{cor}{}{}
    If $n$ is odd, then any equation \begin{equation}
        x^n + a_{n-1}x^{n-1} + ... + a_0 = 0
    \end{equation}
    has a solution, or root.
\end{cor}
\begin{proof*}{}{}
    Consider the function $f(x) = x^n+a_{n-1}x^{n-1} + ... + a_0$. Write $$f(x) = x^n+a_{n-1}x^{n-1} + ... + a_0 = x^n\left(1 + \frac{a_{n-1}}{x} + ... + \frac{a_0}{x^n}\right)$$ 
    Then note that $$\left|\frac{a_{n-1}}{x} + \frac{a_{n-2}}{x^2} + ... + \frac{a_0}{x^n}\right|\leq \frac{|a_{n-1}|}{|x|} + ... + \frac{|a_0|}{|x^n|}$$
    Choose $x$ such that $$|x| > 1,2n|a_{n-1}|,...,2n|a_0|$$
    so $|x^k| > |x|$ for all $k > 1$, and $$\frac{|a_{n-k}|}{|x^k|} < \frac{|a_{n-k}|}{|x|} < \frac{|a_{n-k}|}{2n|a_{n-k}|} < \frac{1}{2n}$$
    Thus, we have that $$\left|\frac{a_{n-1}}{x} + \frac{a_{n-2}}{x^2} + ... + \frac{a_0}{x^n}\right|\leq \frac{|a_{n-1}|}{|x|} + ... + \frac{|a_0|}{|x^n|} < \underbrace{\frac{1}{2n} + ... +\frac{1}{2n}}_{\text{$n$ times}} = \frac{1}{2}$$
    In other words, $$-\frac{1}{2} < \frac{a_{n-1}}{x} + \frac{a_{n-2}}{x^2} + ... + \frac{a_0}{x^n} < \frac{1}{2}$$
    which implies that $$\frac{1}{2} < 1 + \frac{a_{n-1}}{x} + \frac{a_{n-2}}{x^2} + ... + \frac{a_0}{x^n}$$
    Choosing $x_1 > 0$ which satisfies our condition, we have $$\frac{x_1^n}{2} \leq x_1^n\left(1+\frac{a_{n-1}}{x} + \frac{a_{n-2}}{x^2} + ... + \frac{a_0}{x^n}\right) = f(x_1)$$
    so that $f(x_1) > 0$. On the other hand, choosing $x_2 < 0$ satisfying our condition, $x_2^n < 0$ as $n$ is odd and $$\frac{x_2^n}{2} \geq x_2^n\left(1+\frac{a_{n-1}}{x} + \frac{a_{n-2}}{x^2} + ... + \frac{a_0}{x^n}\right) = f(x_2)$$
    so that $f(x_2) < 0$. Applying the \nameref{thmname:intval} to the interval $[x_2,x_1]$ we conclude that there exists $c \in [x_2,x_1]$ such that $f(c) = 0$.
\end{proof*}


\begin{thm}{}{}
    If $n$ is even and $f(x) = x^n+a_{n-1}x^{n-1} + ... + a_0$, then there is a number $y$ such that $f(y) \leq f(x)$ for all $x \in \R$.
\end{thm}
\begin{proof*}{}{}
    Choose $M$ such that $$M = \max(1,2n|a_{n-1}|,...,2n|a_0)$$
    Then for all $x$ with $|x| \geq M$ we have $$\frac{1}{2} \leq 1 + \frac{a_{n-1}}{x} + \frac{a_{n-2}}{x^2} + ... + \frac{a_0}{x^n}$$
    Since $n$ is even, $x^n \geq 0$ for all $x$, so $$\frac{x^n}{2} \leq x^n\left(1 + \frac{a_{n-1}}{x} + \frac{a_{n-2}}{x^2} + ... + \frac{a_0}{x^n}\right) = f(x)$$
    provided that $|x| \geq M$. Now consider the number $f(0)$. Let $b > 0$ be a number such that $b^n \geq 2f(0)$ and also $b > M$. Then if $x \geq b$, we have $$f(x) \geq \frac{x^n}{2} \geq \frac{b^n}{2} \geq f(0)$$
    The same holds for $x \leq -b$. In particular, if $x \geq b$ or $x \leq -b$, then $f(x) \geq f(0)$. Applying the extreme value theorem to $f$ on the interval $[-b,b]$, we conclude that there is a number $y \in [-b,b]$ such that if $-b \leq x \leq b$, then $f(y) \leq f(x)$. In particular, $f(y) \leq f(0)$. Thus, if $x \leq -b$ or $x \geq b$, then $f(x) \geq f(0) \geq f(y)$. Combining these results we find that $f(y) \leq f(x)$ for all $x \in \R$.
\end{proof*}


\begin{cor}{}{}
    Consider the equation \begin{equation}
        x^n +a_{n-1}x^{n-1} + ... + a_0 = c
    \end{equation}
    for $n$ even. Then there is a number $m$ such that the equation has a solution for $c \geq m$ and has no solution for $c < m$.
\end{cor}
\begin{proof*}{}{}
    Let $f(x) = x^n + a_{n-1}x^{n-1} + ...+ a_0$. According to our previous theorem there exists $y \in \R$ such that $f(y) \leq f(x)$ for all $x \in \R$. Let $m = f(y)$. If $c < m$ then the equation above has no solutiion, since the left hand side has a value $\geq m$ always. If $c = m$, then $y$ is a solution of the equation. Finally, for $c > m$, let $b > y$ such that $f(b) > c$. Then the \nameref{thmname:intval} applied to the interval $[y,b]$ states that there exists $x \in [y,b]$ such taht $f(x) = c$, so $x$ is a solution of the equation.
\end{proof*}


\subsection{Uniform Continuity}


\begin{defn}{}{}
    A function $f:\R\rightarrow \R$ is \Emph{uniformly continuous on an interval $I$} if for every $\epsilon > 0$ there is some $\delta > 0$ such that, for all $x,y \in I$, if $|x-y| < \delta$ then $|f(x) - f(y)| < \delta$.
\end{defn}


\begin{lem}{}{}
    Let $a < b < c$ and let $f$ be continuous on the interval $[a,c]$. Let $\epsilon > 0$ and suppose that \begin{enumerate}
        \item if $x,y \in [a,b]$ and $|x-y| < \delta_1$, then $|f(x) - f(y)| < \epsilon$
        \item if $x,y \in [b,c]$ and $|x-y| < \delta_2$, then $|f(x) - f(y)| < \epsilon$
    \end{enumerate}
    Then there is a $\delta > 0$ such that if $x,y \in [a,c]$ and $|x-y| < \delta$, then $|f(x) - f(y)| < \epsilon$.
\end{lem}
\begin{proof*}{}{}
    Fix $\epsilon > 0$. Since $f$ is continuous at $b$ there exists $\delta_3 > 0$ such that if $|x-b| < \delta_3$, then $|f(x) - f(b)| < \frac{\epsilon}{2}$. It follows that if $|x-b| < \delta_3$ and $|y-b| < \delta_3$ then $|f(x) - f(y)| < \epsilon$. Choose $\delta = \min(\delta_1,\delta_2,\delta_3)$. Let $x,y \in [a,c]$ with $|x-y| < \delta$. If $x$ and $y$ are both in $[a,b]$, then $|f(x) - f(y)| < \epsilon$ by assumption. Similarly, if $x,y \in [b,c]$, then again $|f(x) - f(y)| < \epsilon$ by assumption. Finally, without loss of generality suppose $x < b < y$. Since $|x-y| < \delta$ we have that $|x-b| = |b-x| = b-x < y-x = |y-x| < \delta$ and similarly $|y-b| < \delta$. Thus, we have that $|f(x) - f(y)| < \epsilon$, completing the proof.
\end{proof*}


\begin{thm}{Uniform Continuity Theorem}{}
    If $f$ is continuous on $[a,b]$, then $f$ is uniformly continuous on $[a,b]$.
\end{thm}
\begin{proof*}{}{}
    Consider $\epsilon > 0$. Define the set $$A(\epsilon) := \{x \in [a,b]:\exists \delta > 0;\forall y,z \in [a,x];|y-z| < \delta \implies|f(y) - f(z)| < \epsilon\}$$
    Then $A(\epsilon) \neq \emptyset$ since $a \in A(\epsilon)$, and $A(\epsilon)$ is bounded above by $b$, so $A(\epsilon)$ has a least upper bound $\alpha_{\epsilon} \in \R$. Suppose towards a contradiction that $\alpha < b$. Since $f$ is continuous at $\alpha$, there is some $\delta_0$ such that if $|y-\alpha| < \delta_0$, then $|f(y) - f(\alpha)| < \epsilon/2$. Consequently, if $|y-\alpha| < \delta_0$ and $|z-\alpha| < \delta_0$, then $|f(y) - f(z)| < \epsilon$. So, $f$ surely satisfies the condition for containment in $A(\epsilon)$ on the interval $[\alpha - \delta_0, \alpha + \delta_0$. On the other hand, since $\alpha$ is the least upper bound of $A$ it is also clear that the condition is satisfies on $[a,\alpha - \delta_0]$, namely $\alpha - \delta_0 \in A$. Then, the Lemma implies that $f$ satisfies the condition on $[a,\alpha+\delta_0]$ since it satisfies it on $[a,\alpha - \delta_0]$ and $[\alpha - \delta_0, \alpha + \delta_0]$. Hence, $\alpha + \delta_0 \in A$, contradicting the fact that $\alpha$ is an upper bound.


    To complete the proof we must show that $\alpha = b$ is in $A$. Since $f$ is continuous at $b$, there is some $\delta_0 > 0$ such that if $y \in (b-\delta_0, b)$, then $|f(y) - f(b)| < \epsilon/2$. So, for any $x,y \in [b-\delta_0,b]$, $|f(y) - f(x)| < \epsilon$. But, $f$ satisfies the condition for $A(\epsilon)$ on $[a,b-\delta_0]$ since $b$ is the least upper bound of $A(\epsilon)$, so the Lemma implies that $f$ satisfies the condition on $[a,b]$. Therefore, as $\epsilon > 0$ was arbitrary, we conclude that $f$ is uniformly continuous on $[a,b]$, completing the proof.
\end{proof*}







%%%%%%%%%%%%%%%%%%%%%% - P1.Chapter 2
\chapter{Differentiation}

\section{Introduction to Derivatives}


\begin{defn}{Differentiability}{}
    A function $f:\R\rightarrow \R$ is said to be \Emph{differentiable at $a$} if \begin{equation}
        \lim\limits_{h\rightarrow 0}\frac{f(a+h) - f(a)}{h}
    \end{equation}
    exists. In this case the limit is denoted by \Emph{$f'(a)$} and is called the \Emph{derivative of $f$ at $a$}. We also say that $f$ is \Emph{differentiable} if $f$ is differentiable at $a$ for all $a$ in its domain.
\end{defn}

\begin{defn}{}{}
    We define the \Emph{tangent line} to the graph of $f$ at $(a,f(a))$ to be the line through $(a,f(a))$ with slope $f'(a)$. That is, the tangent line at $(a,f(a))$ is well defined if and only if $f$ is differentiable at $a$.
\end{defn}


\begin{rmk}{}{}
    Given a function $f$, we denote by $f'$ the function whose domain is the set of all numbers $a \in \R$ such that $f$ is differentiable at $a$, and whose value at such a number $a$ is \begin{equation}
        \lim\limits_{h\rightarrow 0}\frac{f(a+h) - f(a)}{h}
    \end{equation}
    The function $f'$ is called the \Emph{derivative} of $f$.
\end{rmk}

\begin{nota*}{}{}
    For a given function $f:\R\rightarrow \R$, the derivative $f'$ is often denoted by \begin{equation}
        \frac{df(x)}{dx}
    \end{equation}
    and the number $f'(a)$ is denoted by \begin{equation}
        \left.\frac{df(x)}{dx}\right\vert_{x=a}
    \end{equation}
\end{nota*}


\begin{thm}{}{}
    If $f$ is differentiable at $a$, then $f$ is continuous at $a$.
\end{thm}
\begin{proof*}{}{}
    Suppose $f$ is differentiable at a point $a$. Then we have that the limit $$\lim\limits_{h\rightarrow 0}\frac{f(a+h) - f(a)}{h}$$ exists. It follows by \nameref{thm:limlaws} that \begin{align*}
        \lim\limits_{h\rightarrow 0}f(a+h) - f(a) &= \lim\limits_{h\rightarrow 0}\frac{f(a+h)-f(a)}{h}\cdot h \\
        &= \lim\limits_{h\rightarrow 0}\frac{f(a+h) - f(a)}{h}\cdot \lim\limits_{h\rightarrow 0} h\\
        &= f'(a)\cdot 0\\
        &= 0
    \end{align*}
    Thus, by \nameref{thm:limlaws} the result that $\lim\limits_{h\rightarrow 0}f(a+h) - f(a) = 0$ is equivalent to $\lim\limits_{h\rightarrow 0}f(a+h) = \lim\limits_{h\rightarrow 0} f(a) = f(a)$. Thus, $f$ is continuous at $a$, replacing $a+h$ with $x$ and $h\rightarrow 0$ with $x \rightarrow a$.
\end{proof*}


\begin{defn}{Higher Order Derivatives}{}
    Since the derivative of a function $f$ is also a function, we can take its derivative to obtain the function $(f')' = f''$. In general, we denote the $k+1$-th derivative of $f$ inductively by \begin{align*}
        f^{(1)} &= f' \\
        f^{(k+1)} &= (f^{(k)})'
    \end{align*}
    These are called \Emph{higher order derivatives of $f$}. We also define $f^{(0)} = f$. In Leibnitzian notation we write \begin{equation}
        \frac{d^kf(x)}{dx} = f^{(k)}
    \end{equation}
\end{defn}

\section{Differentiation Results}

\begin{thm}{}{}
    If $f$ is a constant function, $f(x) = c$, then $f'(a) = 0$ for all $a \in \R$.
\end{thm}
\begin{proof*}{}{}
    Observe that for $a \in \R$, $$f'(a) = \lim\limits_{h\rightarrow 0}\frac{f(a+h)-f(a)}{h} = \lim\limits_{h\rightarrow 0}\frac{c-c}{h} = 0$$
    as desired.
\end{proof*}


\begin{thm}{}{}
    If $f$ is the identity function, $f(x) = x$, then $f'(a) = 1$ for all $a \in \R$.
\end{thm}
\begin{proof*}{}{}
    Observe that for $a \in \R$, $$f'(a) = \lim\limits_{h\rightarrow 0}\frac{f(a+h)-f(a)}{h} = \lim\limits_{h\rightarrow 0}\frac{a+h-a}{h} = \lim\limits_{h\rightarrow 0} 1 = 1$$
    as desired.
\end{proof*}

\begin{thm}{Linearity}{}
    If $f$ and $g$ are differentiable at $a$, then $f+cg$ is differentiable for all $c \in \R$
\end{thm}
\begin{proof*}{}{}
    Observe that \begin{align*}
        (f+cg)'(a) &= \lim\limits_{h\rightarrow 0}\frac{(f+cg)(a+h) - (f+cg)(a)}{h} \\
        &= \lim\limits_{h\rightarrow 0}\frac{f(a+h)+cg(a+h)-[f(a)+cg(a)]}{h} \\
        &= \lim\limits_{h\rightarrow 0}\frac{[f(a+h)-f(a)]+c[g(a+h)-g(a)]}{h} \\
        &= \lim\limits_{h\rightarrow 0}\left(\frac{f(a+h)-f(a)}{h}+c\frac{g(a+h)-g(a)}{h}\right) \\
        &= \lim\limits_{h\rightarrow 0}\frac{f(a+h)-f(a)}{h}+\lim\limits_{h\rightarrow 0}c\frac{g(a+h)-g(a)}{h} \\
        &= f'(a) + c\lim\limits_{h\rightarrow 0}\frac{g(a+h)-g(a)}{h} \\
        &= f'(a)+cg'(a) 
    \end{align*}
    as desired.
\end{proof*}

\begin{thm}{Product Rule}{}
    If $f$ and $g$ are differentiable at $a$, then $f\cdot g$ is also differentiable at $a$ and $$(f\cdot g)'(a) = f'(a)\cdot g(a) + f(a) \cdot g'(a)$$
\end{thm}
\begin{proof*}{}{}
    Observe that \begin{align*}
        (f\cdot g)'(a) &= \lim\limits_{h\rightarrow 0}\frac{(f\cdot g)(a+h) - (f\cdot g)(a)}{h} \\
        &= \lim\limits_{h\rightarrow 0}\frac{f(a+h)g(a+h) - f(a+h)g(a) + f(a+h)g(a) - f(a)g(a)}{h} \\
        &= \lim\limits_{h\rightarrow 0}\frac{f(a+h)[g(a+h)-g(a)]}{h} + \lim\limits_{h\rightarrow 0}\frac{g(a)[f(a+h) - f(a)]}{h} \\
        &= \lim\limits_{h\rightarrow 0}f(a+h)\cdot \lim\limits_{h\rightarrow 0}\frac{g(a+h) - g(a)}{h} + \lim\limits_{h\rightarrow 0}\frac{f(a+h)-f(a)}{h}\cdot \lim\limits_{h\rightarrow 0}g(a) \\
        &= f(a)\cdot g'(a) + f'(a)\cdot g(a)
    \end{align*}
    as claimed, where $\lim\limits_{h\rightarrow 0}f(a+h) = f(a)$ since $f$ is differentiable at $a$, which implies it is also continuous at $a$.
\end{proof*}

\begin{thm}{Power Rule}{}
    IF $f(x) = x^n$ for some natural number $n$, then $$f'(a) = na^{n-1}$$ for all $a$.
\end{thm}
\begin{proof*}{}{}
    For the proof we will proceed by induction on $n$. For $n = 1$ we have shown that $f'(a) = 1 = 1\cdot a^0$, satisfying the base case. Assume that there exists $k \in \N$ such that if $n = k$, $f'(a) = ka^{k-1}$. Then, for the case of $n = k+1$ we may write $g(x) = x\cdot x^k = I(x)\cdot f(x)$. Hence, by the product rule we have that for all $a$ \begin{align*}
        g'(a) &= (I\cdot f)'(a) \\
        &= I'(a) \cdot f(a) + I(a) \cdot f'(a) \\
        &= 1\cdot a^k + a\cdot ka^{k-1} \\
        &= (k+1)a^k
    \end{align*}
    as claimed. Hence, by mathematical induction we conclude that if $f(x) = x^n$ for $n \in \N$, then $f'(a) = na^{n-1}$ for all $a \in \R$.
\end{proof*}


\begin{thm}{Derivative of a Quotient}{}
    If $g$ is differentiable at $a$, and $g(a) \neq 0$, then $1/g$ is differentiable at $a$ and $$\left(\frac{1}{g}\right)'(a) = \frac{-g'(a)}{|g(a)|^2}$$
\end{thm}
\begin{proof*}{}{}
    Note that since $g$ is differentiable at $a$ it is continuous at $a$. Moreover, since $g(a) \neq 0$, there exists $\delta > 0$ such that $g(a+h) \neq 0$ for $|h| < \delta$. Therefore, $(1/g)(a+h)$ is well defined for small enough $h$, and we can write \begin{align*}
        \lim\limits_{h\rightarrow 0}\frac{(1/g)(a+h) - (1/g)(a)}{h} &= \lim\limits_{h\rightarrow 0}\frac{1/g(a+h) - 1/g(a)}{h} \\
        &= \lim\limits_{h\rightarrow 0}\frac{g(a) - g(a+h)}{h[g(a)\cdot g(a+h)]} \\
        &= \lim\limits_{h\rightarrow 0}\frac{-[g(a+h)-g(a)]}{h}\cdot \lim\limits_{h\rightarrow 0}\frac{1}{g(a)\cdot g(a+h)} \\
        &= -g'(a)\cdot \frac{1}{|g(a)|^2}
    \end{align*}
    where $\lim\limits_{h\rightarrow 0}1/g(a+h) = 1/g(a)$ by continuity of $g$.
\end{proof*}

\begin{thm}{Quotient Rule}{}
    If $f$ and $g$ are differentiable at $a$ and $g(a) \neq 0$, then $f/g$ is differentiable at $a$ and $$(f/g)'(a) = \frac{g(a)\cdot f'(a) - f(a) \cdot g'(a)}{|g(a)|^2}$$
\end{thm}
\begin{proof*}{}{}
    Note that $f/g = f\cdot (1/g)$, so we have \begin{align*}
        (f/g)'(a) &= (f\cdot 1/g)'(a) \\
        &= f'(a)\cdot(1/g)(a) + f(a)\cdot(1/g)'(a) \tag{Product Rule}\\
        &= \frac{f'(a)}{g(a)} -\frac{f(a)g'(a)}{|g(a)|^2} \tag{Quotient Derivative}\\
        &= \frac{f'(a)g(a) - f(a)g'(a)}{|g(a)|^2}
    \end{align*}
    as claimed.
\end{proof*}

\begin{thm}{General Product Rule}{}
    If $f_1,f_2,...,f_n$ are differentiable at $a$ for some $n \in \N$, then $f_1\cdot f_2\cdot ...\cdot f_n$ is differentiable at $a$ and $$(f_1\cdot...\cdot f_n)'(a) = \sum\limits_{i=1}^nf_1(a)\cdot...\cdot f_{i-1}(a)\cdot f'_i(a)\cdot f_{i+1}(a)\cdot...\cdot f_n(a)$$
\end{thm}
\begin{proof*}{}{}
    We proceed by induction on $n$. If $n = 1$ then $f_1'(a) = f_1'(a)$, so the base case holds. Now, suppose the claim is true for some $k \in \N$. Then it follows that if $n = k+1$ \begin{align*}
        (f_1\cdot ... \cdot f_k\cdot f_{k+1})'(a) &= (f_1\cdot ...\cdot f_k)'(a)f_{k+1}(a) + (f_1\cdot...\cdot f_k)(a)f_{k+1}'(a) \tag{Product Rule} \\
        &= \left[\sum\limits_{i=1}^kf_1(a)\cdot...\cdot f_{i-1}(a)\cdot f'_i(a)\cdot f_{i+1}(a)\cdot...\cdot f_k(a)\right]f_{k+1}(a)\\
        &+ f_1(a)\cdot ... \cdot f_k(a)\cdot f_{k+1}'(a) \tag{by Induction Hypothesis} \\
        &= \sum\limits_{i=1}^{k+1}f_1(a)\cdot...\cdot f_{i-1}(a)\cdot f'_i(a)\cdot f_{i+1}(a)\cdot...\cdot f_{k+1}(a)
    \end{align*}
    as desired. Thus by mathematical induction we conclude that the formula holds for all $n \in \N$.
\end{proof*}

\begin{thm}{Chain Rule}{}
    If $g$ is differentiable at $a$ and $f$ is differentiable at $g(a)$, then $f\circ g$ is differentiable at $a$ and $$(f\circ g)'(a) = f'(g(a))\cdot g'(a)$$
\end{thm}
\begin{proof*}{}{}
    Define a function $\phi$ as follows: \begin{equation}
        \phi(h) = \left\{\begin{array}{ll}
            \frac{f(g(a+h)) - f(g(a))}{g(a+h)-g(a)}, & \text{if } g(a+h)-g(a) \neq 0 \\
            f'(g(a)), & \text{if } g(a+h) - g(a) = 0
        \end{array}\right.
    \end{equation}
    Note that by differentiability of $g$ at $a$, $g$ is continuous at $a$ as well so as $h\rightarrow 0$, $g(a+h)-g(a)\rightarrow 0$, so if $g(a+h)-g(a)$ is not zero, then $\phi(h)$ will approach $f'(g(a))$ as $h$ goes to zero. If it is zero then $\phi(h)$ is exactly $f'(g(a))$. Note that as $f$ is differentiable at $g(a)$ we have $$\lim\limits_{k\rightarrow 0}\frac{f(g(a) + k) - f(g(a))}{k} = f'(g(a))$$
    Thus, if $\epsilon > 0$ there is some number $\delta' > 0$ such that, for all $k$, \begin{equation*}
        (1)\hspace{5pt}\text{if $0 < |k| < \delta'$, then } \left|\frac{f(g(a) + k) - f(g(a))}{k} - f'(g(a))\right| < \epsilon
    \end{equation*}
    Now, $g$ is differentiable at $a$, hence continuous at $a$, so there is $\delta > 0$ such that for all $h$, \begin{equation*}
        (2)\hspace{5pt}\text{if $|h| < \delta$, then } |g(a+h) - g(a)| < \delta'
    \end{equation*}
    Consider now any $h$ with $|h| < \delta$. If $k = g(a+h) - g(a) \neq 0$, then \begin{equation*}
        \phi(h) = \frac{f(g(a+h)) - f(g(a))}{g(a+h) - g(a)} = \frac{f(g(a)+k) - f(g(a))}{k}
    \end{equation*}
    it follows from $(2)$ that $|k| < \delta'$, and hence from $(1)$ that \begin{equation*}
        |\phi(h) - f'(g(a))| < \epsilon
    \end{equation*}
    On the other hand, if $g(a+h) - g(a) = 0$, then $\phi(h) = f'(g(a))$, so it is surely true that \begin{equation*}
        |\phi(h) - f'(g9a))| < \epsilon
    \end{equation*}
    We therefore have proved that \begin{equation*}
        \lim\limits_{h\rightarrow 0}\phi(h) = f'(g(a))
    \end{equation*}
    so $\phi$ is continuous at $0$. If $h \neq 0$, then we have $$\frac{f(g(a+h)) - f(g(a))}{h} = \phi(h)\cdot \frac{g(a+h)-g(a)}{h}$$
    even if $g(a+h)-g(a) = 0$. Therefore, we have that \begin{align*}
        (f\circ g)'(a) &= \lim\limits_{h\rightarrow 0}\frac{f(g(a+h)) - f(g(a))}{h} \\
        &= \lim\limits_{h\rightarrow 0}\phi(h)\cdot \lim\limits_{h\rightarrow 0}\frac{g(a+h)-g(a)}{h} \\
        &= f'(g(a))\cdot g'(a) 
    \end{align*}
    by continuity of $\phi(h)$ at $0$.
\end{proof*}



\section{Applications of Derivatives}

\begin{defn}{Extrema}{}
    Let $f$ be a function and $A$ a set of numbers contained in the domain of $f$. A point $x \in A$ is \Emph{maximum point} for $f$ on $A$ if \begin{equation}
        f(x) \geq f(y) \forall y \in A
    \end{equation}
    The number $f(x)$ is itself called the \Emph{maximum value} of $f$ on $A$.

    A point $x \in A$ is a \Emph{minimum point} for $f$ on $A$ if \begin{equation}
        f(x) \leq f(y) \forall y \in A
    \end{equation}
    The number $f(x)$ is itself called the \Emph{minimum value} of $f$ on $A$.
\end{defn}


\begin{thm}{}{dirext}
    Let $f$ be any function defined on $(a,b)$. If $x$ is an extremum point for $f$ on $(a,b)$, and $f$ is differentiable at $x$, then $f'(x) = 0$.
\end{thm}
\begin{proof*}{}{}
    Consider the case where $f$ has a maximum at $x$. If $h$ is any number such that $x+h \in (a,b)$, then $$f(x) \geq f(x+h)$$
    since $f$ has a maximum on $(a,b)$ at $x$. This implies that $$f(x+h)-f(x) \leq 0$$
    Thus, if $h > 0$ we have that $$\frac{f(x+h) - f(x)}{h} \leq 0$$
    and consequently $$\lim\limits_{h\rightarrow 0^+}\frac{f(x+h)-f(x)}{h} \leq 0$$
    as otherwise $\frac{f(x+h) - f(x)}{h} > 0$ for some $h$, contradicting our initial assumptions. Similarly, if $h < 0$ we have $$\frac{f(x+h)-f(x)}{h} \geq 0$$
    so $$\lim\limits_{h\rightarrow 0^-}\frac{f(x+h)-f(x)}{h} \geq 0$$
    By hypothesis $f$ is differentiable at $x$, so these two limits must be equal, so in fact $f'(x) \leq 0$ and $f'(x) \geq 0$. Thus, $f'(x) = 0$.

    On the other hand, suppose $f$ has a minimum at $x$. Then $-f$ has a maximum at $x$. Indeed, for all $y \in (a,b)$ we have $f(y) \geq f(x)$, so $-f(y) \leq -f(x)$. Then, from our above argument and the differentiability of $f$ at $x$, we have $-f'(x) = 0$, which implies that $f'(x) = 0$.
\end{proof*}


\begin{defn}{Local Extrema}{}
    Let $f$ be a function, and $A$ a set of numbers contained in the domain of $f$. A point $x$ in $A$ is a \Emph{local maximum [minimum] point} for $f$ on $A$ if there is some $\delta > 0$ such that $x$ is a maximum [minimum] point for $f$ on $A \cap(x-\delta,x+\delta)$.
\end{defn}


\begin{defn}{}{}
    A \Emph{critical point} of a function $f$ is a number $x$ such that \begin{equation}
        f'(x) = 0
    \end{equation}
    The number $f(x)$ itself is called a \Emph{critical value} of $f$.
\end{defn}

\begin{rmk}{}{}
    Give a function continuous $f$, if $x$ is an extrumum of $f$ on $[a,b]$, then one of the following must be satisfied: \begin{enumerate}
        \item $x$ is a critical point of $f$ in $[a,b]$
        \item $x = a$ or $x = b$ so $x$ is an endpoint of $[a,b]$
        \item $x$ is a point in $[a,b]$ such that $f$ is not differentiable at $x$
    \end{enumerate}
\end{rmk}


\begin{namthm}{Rolle's Theorem}{rol}
    If $f$ is continuous on $[a,b]$ and differentiable on $(a,b)$, and $f(a) = f(b)$, then there is a number $x \in (a,b)$ such that $f'(x) = 0$.
\end{namthm}
\begin{proof*}{}{}
    It follows from continuity of $f$ on $[a,b]$ that $f$ has a maximum or minimum value on $[a,b]$ (by the Extreme Value Theorem).

    Suppose first that the maximum value occurs at a point $x \in (a,b0$. Then $f'(x) = 0$ by Theorem \ref{thm:dirext}. On the other hand suppose that the minimum value of $f$ occurs at some point $x$ in $(a,b)$. Then, again, $f'(x) = 0$ by Theorem \ref{thm:dirext}.

    Finally, suppose the maximum and minimum values both occur at the end points. Since $f(a) = f(b)$, the maximum and minimum values of $f$ are equal, so $f$ is a constant function, and for a constant function we can choose any $x \in (a,b)$ and have $f'(x) = 0$, completing the proof.
\end{proof*}


\begin{namthm}{The Mean Value Theorem}{meanval}
    If $f$ is continuous on $[a,b]$ and differentiable on $(a,b)$, then there is a number $x \in (a,b)$ such that \begin{equation}
        f'(x) = \frac{f(b)-f(a)}{b-a} 
    \end{equation}
\end{namthm}
\begin{proof*}{}{}
    Let \begin{equation*}
        h(x) = f(x) - \left[\frac{f(b) - f(a)}{b-a}\right](x-a)
    \end{equation*}
    Evidently, $h$ is continuous on $[a,b]$ and differentiable on $(a,b)$ as it is the sum of correspondingly continuous and differentiable functions. Moreover, \begin{align*}
        h(a) &= f(a) \\
        h(b) &= f(b) - \left[\frac{f(b) - f(a)}{b-a}\right](b-a) \\
        &= f(a)
    \end{align*}
    Consequently, we may apply \nameref{thmname:rol} to $h$ and conclude that there exists $x \in (a,b)$ such that \begin{equation*} 
        0 = h'(x) = f'(x) - \frac{f(b)-f(a)}{b-a}
    \end{equation*}
    so that \begin{equation*}
        f'(x) = \frac{f(b) - f(a)}{b-a}
    \end{equation*}
    as desired.
\end{proof*}

\begin{cor}{}{}
    If $f$ is defined on an interval and $f'(x) = 0$ for all $x$ in the interval, then $f$ is constant on the interval.
\end{cor}
\begin{proof*}
    Let $a$ and $b$ be any two points in the interval with $a \neq b$. Then there is some $x \in (a,b)$ such that \begin{equation*}
        0 = f'(x) = \frac{f(b) - f(a)}{b-a}
    \end{equation*}
    so $f(b) - f(a) = 0$ and consequently $f(a) = f(b)$. Thus the value of $f$ at any two points in the interval is the same, so $f$ is constant on the interval.
\end{proof*}

\begin{cor}{}{}
    If $f$ and $g$ are defined on the same interval, and $f'(x) = g'(x)$ for all $x$ in the interval, then there is come number $c$ such that $f = g+c$.
\end{cor}
\begin{proof*}{}{}
    For all $x$ in the interval we have $(f-g)'(x) = f'(x) - g'(x) = 0$, so by the previous corollary there is some number $c$ such that $f-g = c$.
\end{proof*}


\begin{defn}{}{}
    A function is \Emph{increasing} on an interval $I$ if $f(a) < f(b)$ whenever $a,b \in I$ with $a < b$. The function $f$ is \Emph{decreasing} on an interval $I$ if $f(a) > f(b)$ for all $a,b \in I$ with $a < b$.
\end{defn}


\begin{cor}{}{}
    If $f'(x) > 0$ for all $x$ in an interval, then $f$ is increasing on the interval; if $f'(x) < 0$ for all $x$ in the interval, then $f$ is decreasing on the interval.
\end{cor}
\begin{proof*}{}{}
    Consider the case where $f'(x) > 0$. Let $a,b \in I$ with $a < b$. Then by \nameref{thmname:meanval} there exists $x \in (a,b)$ such that \begin{equation*}
        f'(x) = \frac{f(b) - f(a)}{b-a}
    \end{equation*}
    But, $f'(x) > 0$ for all $x \in (a,b)$, so $$\frac{f(b) - f(a)}{b-a} > 0$$
    Since $b-a > 0$ we conclude that $f(b) > f(a)$ so $f$ is increasing.

    Next, consider the case for $f'(x) < 0$. Then $-f'(x) > 0$ for all $x \in I$, so by the first case we have that for all $a,b \in I$ with $a < b$, $-f(a) < -f(b)$. Multiplying both sides by $-1$ we have that $f(a) > f(b)$ for all $a,b \in I$ such that $a < b$, so $f$ is decreasing, as desired.
\end{proof*}


\begin{thm}{Second Derivative Test}{}
    Suppose $f'(a) = 0$. If $f''(a) > 0$, then $f$ has a local minimum at $a$; if $f''(a) < 0$ then $f$ has a local maximum at $a$.
\end{thm}
\begin{proof*}{}{}
    By definition \begin{equation*}
        f''(a) = \lim\limits_{h\rightarrow 0} \frac{f'(a+h) - f'(a)}{h}
    \end{equation*}
    Since $f'(a) = 0$ by assumption, we can write \begin{equation*}
        f''(a) = \lim\limits_{h\rightarrow 0}\frac{f'(a+h)}{h}
    \end{equation*}
    Suppose now that $f''(a) > 0$. Then there exists $\delta >0$ such that if $|h| < \delta$ $f'(a+h)/h > 0$. Thus, for $|h| < \delta$, if $h < 0$ we must have $f'(a+h) < 0$ and if $h > 0$ we must have $f'(a+h) > 0$. This means by our previous corollary that $f$ is increasing in the interval $(a,a+\delta)$, and decreasing in $(a-\delta, a)$. Thus, as $f'(a) = 0$, $f(a)$ must be a local minimum.

    If $f''(a) < 0$, then $-f''(a) > 0$ so $-f(a)$ is must be a local minimum. That is, there exists $\delta > 0$ such that if $x \in (a - \delta, a + \delta)$, then $-f(x) \geq -f(a)$. Hence, it follows that $f(x) \leq f(a)$ for all $x \in (a-\delta,a+\delta)$, so $f(a)$ is a local maximum of $f$.
\end{proof*}


\begin{thm}{}{}
    Suppose $f''(a)$ exists. If $f$ has a local minimum at $a$, then $f''(a) \geq 0$; if $f$ has a local maximum at $a$, then $f''(a) \leq 0$.
\end{thm}
\begin{proof*}{}{}
    Suppose $f$ has a local minimum at $a$. If $f''(a) < 0$ then by our previous result $f$ would have a local maximum at $a$. But, this implies that $f$ would be constant in some interval containing $a$, so that $f''(a) = 0$, which is a contradiction. Thus, we must have that $f''(a) \geq 0$.

    The case for a local maximum is analogous.
\end{proof*}


\begin{thm}{}{}
    Suppose that $f$ is continuous at $a$, and that $f'(x)$ exists for all $x$ in some interval containing $a$, except perhaps for $x = a$. Suppose, moreover, that $\lim\limits_{x\rightarrow a}f'(x)$ exists. Then $f'(a)$ also exists and \begin{equation}
        f'(a) = \lim\limits_{x\rightarrow a}f'(x)
    \end{equation}
\end{thm}
\begin{proof*}{}{}
    By definition \begin{equation*}
        f'(a) = \lim\limits_{h\rightarrow 0}\frac{f(a+h) - f(a)}{h}
    \end{equation*}
    For sufficiently small $h > 0$ the function $f$ will be continuous on $[a,a+h]$, and differentiable on $(a,a+h)$, by assumption (similarly for sufficiently small $h < 0$). By \nameref{thmname:meanval} there is a number $\alpha_h \in (a,a+h)$ such that $$\frac{f(a+h) - f(a)}{h} = f'(\alpha_h)$$
    Now, $\alpha_h$ approaches $a$ as $h$ approaches $0$, because $\alpha_h$ is in $(a,a+h)$. Since $\lim\limits_{x\rightarrow a}f'(x)$ exists, it follows that $$f'(a) = \lim\limits_{h\rightarrow 0}\frac{f(a+h) - f(a)}{h} = \lim\limits_{h\rightarrow 0}f'(\alpha_h) = \lim\limits_{x\rightarrow a}f'(x)$$
    For this last equality write $\lim\limits_{x\rightarrow a}f'(x) = L \in \R$. Fix $\epsilon > 0$. Then there exists $\delta > 0$ such that for all $x \in (a-\delta, a+\delta)$, $|f'(x) - L| < \epsilon$. It follows that for $|h| < \delta$, if $h > 0$ and $\alpha_h \in (a,a+h) \subset (a-\delta,a+\delta)$ we have $|f'(\alpha_h) - L| < \epsilon$ and if $h < 0$ and $\alpha_h \in (a+h, a) \subset (a-\delta,a+\delta)$, then $|f'(\alpha_h) - L| < \epsilon$. Thus, by definition we have that $\lim\limits_{h\rightarrow 0^+}f'(\alpha_h) = \lim\limits_{h\rightarrow 0^-}f'(\alpha_h) = L$, so in particular $\lim\limits_{h\rightarrow 0}f'(\alpha_h) = L = \lim\limits_{x\rightarrow a}f'(x)$, completing the proof.
\end{proof*}


\begin{namthm}{The Cauchy Mean Value Theorem}{caumeanval}
    If $f$ and $g$ are continuous on $[a,b]$ and differentiable on $(a,b)$, then there is a number $x \in (a,b)$ such that \begin{equation}
        [f(b) - f(a)]g'(x) = [g(b) - g(a)]f'(x)
    \end{equation}
\end{namthm}
\begin{proof*}{}{}
    Let $$h(x) = f(x)[g(b) - g(a)] - g(x)[f(b)-f(a)]$$
    Then $h$ is continuous on $[a,b]$, differentiable on $(a,b)$, and $$h(a) = f(a)g(b) - g(a)f(b) = h(b)$$
    It follows by \nameref{thmname:rol} that $h'(x) = 0$ for some $x \in (a,b)$, which implies that \begin{equation*}
        0 = h'(x) = f'(x)[g(b)-g(a)] - g'(x)[f(b) - f(a)]
    \end{equation*}
    completing the proof.
\end{proof*}


\begin{namthm}{L'H\^{o}pital's Rule}{}
    Suppose that \begin{equation}
        \lim\limits_{x\rightarrow a}f(x) = 0\;and\;\lim\limits_{x\rightarrow a}g(x) = 0
    \end{equation}
    and suppose also that $\lim\limits_{x\rightarrow a}f'(x)/g'(x)$ exists. Then $\lim\limits_{x\rightarrow a}f(x)/g(x)$ exists, and \begin{equation}
        \lim\limits_{x\rightarrow a}\frac{f(x)}{g(x)} = \lim\limits_{x\rightarrow a}\frac{f'(x)}{g'(x)}
    \end{equation}
\end{namthm}
\begin{proof*}{}{}
    The hypothesis that $\lim\limits_{x\rightarrow a}f'(x)/g'(x)$ exists contains two implicit assumptions: \begin{enumerate}
        \item there is an interval $(a-\delta,a+\delta)$ such that $f'(x)$ and $g'(x)$ exist for all $x \in (a - \delta, a + \delta)$, except, perhaps, $x = a$,
        \item in this interval $g'(x) \neq 0$, with the possible exception of $x = a$
    \end{enumerate}
    If we define $f(a) = g(a) = 0$, then $f$ and $g$ are continuous at $a$. If $x \in (a,a+\delta)$, then \nameref{thmname:meanval} and \nameref{thmname:caumeanval} apply to $f$ and $g$ on $[a,x]$ (a similar statement holds for $x \in (a-\delta, a)$). First, applying the \nameref{thmname:meanval} to $g$, we see that $g(x) \neq 0$, for if $g(x) = 0$ there would exist $x_1 \in (a,x)$ with $g'(x_1) = 0$, contradicting 2.. Now, applying \nameref{thmname:caumeanval} to $f$ and $g$, we see that there is a number $\alpha_x \in (a,x)$ such that \begin{equation*}
        [f(x)-0]g'(\alpha_x) = [g(x)-0]f'(\alpha_x)
    \end{equation*}
    or \begin{equation*}
        \frac{f(x)}{g(x)} = \frac{f'(\alpha_x)}{g'(\alpha_x)}
    \end{equation*}
    Now, let $\lim_{y\rightarrow a}f'(y)/g'(y) = L \in \R$. Fix $\epsilon > 0$. Then there exists $\delta' > 0$ such that if $y \in (a - \delta', a + \delta')$ then $|f'(y)/g'(y) - L| < \epsilon$. Then, for $x \in (a,a+\delta)$ (or $x \in (a-\delta, a)$) we have $(a,x) \subset (a-\delta, a+\delta)$ (or $(x,a) \subset (a-\delta, a+\delta$). Thus, for $|x-a| < \delta$ we have $\alpha_x \in (a,x) \subset (a -\delta, a+\delta)$ (or $\alpha_x \in (x,a) \subset (a-\delta,a+\delta)$), so $|f'(\alpha_x)/g'(\alpha_x) - L| < \epsilon$. Therefore, we conclude that \begin{equation*}
        \lim\limits_{x\rightarrow a^+} \frac{f'(\alpha_x)}{g'(\alpha_x)} = L = \lim\limits_{x\rightarrow a^-} \frac{f'(\alpha_x)}{g'(\alpha_x)} 
    \end{equation*}
    so in particular \begin{equation*}
        \lim\limits_{x\rightarrow a} \frac{f(x)}{g(x)} = \lim\limits_{x\rightarrow a} \frac{f'(\alpha_x)}{g'(\alpha_x)} =  \lim\limits_{y\rightarrow a} \frac{f'(y)}{g'(y)}
    \end{equation*}
    completing the proof.
\end{proof*}


\subsection{Convexity}


\begin{defn}{}{}
    A function $f$ is \Emph{convex} on an interval $I$, if for all $a,b \in I$, the line segment joining $(a,f(a))$ and $(b,f(b))$ lies above the graph of $f$.

    This is equivalent to stating that for all $x \in (a,b)$, \begin{equation}
        \frac{f(x) - f(a)}{x-a} < \frac{f(b) - f(a)}{b-a}
    \end{equation}
\end{defn}


\begin{defn}{}{}
    A function $f$ is \Emph{concave} on an interval $I$, if for all $a,b \in I$, the line segment joining $(a,f(a))$ and $(b,f(b))$ lies below the graph of $f$.

    This is equivalent to stating that for all $x \in (a,b)$, \begin{equation}
        \frac{f(x) - f(a)}{x-a} > \frac{f(b) - f(a)}{b-a}
    \end{equation}
\end{defn}


\begin{thm}{}{}
    Let $f$ be convex. If $f$ is differentiable at $a$, then the graph of $f$ lies above the tangent line through $(a,f(a))$, except at $(a,f(a))$ itself. If $a < b$ and $f$ is differentiable at $a$ and $b$, then $f'(a) < f'(b)$.
\end{thm}
\begin{proof*}{}{}
    If $0 < h_1 < h_2$, then $a < a+h_1 < a+h_2$, and applying $f$'s convexity we have that \begin{equation*}
        \frac{f(a+h_1) - f(a)}{h_1} < \frac{f(a+h_2)-f(a)}{h_2}
    \end{equation*}
    This implies that the values of $[f(a+h)-f(a)]/h$ decrease as $h\rightarrow 0^+$. Consequently, \begin{equation*}
        f'(a) < \frac{f(a+h)-f(a)}{h},h> 0
    \end{equation*}
    In fact, $f'(a)$ is the infimum of these numbers. Similarly, for $h$ negative, if $h_2 < h_1 < 0$, then \begin{equation*}
        \frac{f(a+h_1)-f(a)}{h_1} > \frac{f(a+h_2)-f(a)}{h_2}
    \end{equation*}
    This shows that the slope of the tangent line is greater that $[f(a+h)-f(a)]/h$ for $h < 0$. In fact, $f'(a)$ is the supremum of all these numbers, so $f(a+h)$ lies above the tangent line if $h < 0$. This satisfies the first part of the theorem. Now, suppose $a < b$. Then we have that \begin{equation*}
        f'(a) < \frac{f(a+(b-a)) - f(a)}{b-a} = \frac{f(b)-f(a)}{b-a}
    \end{equation*}
    since $b - a> 0$ and \begin{equation*}
        f'(b) > \frac{f(b+(a-b))-f(b)}{a-b} = \frac{f(a)-f(b)}{a-b} = \frac{f(b)-f(a)}{b-a}
    \end{equation*}
    since $a-b < 0$. Combining these inequalities we obtain $f'(a) < f'(b)$, as desired.
\end{proof*}


\begin{lem}{}{}
    Suppose $f$ is differentiable and $f'$ is increasing. If $a < b$ and $f(a) = f(b)$, then $f(x) < f(a) = f(b)$ for $a < x < b$.
\end{lem}
\begin{proof*}{}{}
    Suppose towards a contradiction that $f(x) \geq f(a) = f(b)$ for some $x \in (a,b)$. Then the maximum of $f$ on $[a,b]$ occurs at some point $x_0 \in (a,b)$ with $f(x_0) \geq f(a)$ and, of course, $f'(x_0) = 0$. On the other hand, applying \nameref{thmname:meanval} to the interval $[a,x_0]$, we find that there is $x_1$ with $a < x_1 < x_0$ and \begin{equation*}
        f'(x_1) = \frac{f(x_0) - f(a)}{x_0 - a}\geq 0
    \end{equation*}
    contradicting the fact that $f'$ is increasing (since $f'(x_0) = 0$ and $x_1 < x_0$).
\end{proof*}

\begin{thm}{}{}
    If $f$ is differentiable and $f'$ is increasing, then $f$ is convex.
\end{thm}
\begin{proof*}{}{}
    Let $a < b$. Define $g$ by \begin{equation*}
        g(x) = f(x) - \frac{f(b) - f(a)}{b-a}(x-a)
    \end{equation*}
    It is easy to see that $g'$ is also increasing; moreover, $g(a) = g(b) = f(a)$. Applying the lemma to $g$ we conclude that $$a < x < b \implies g(x) < f(a)$$
    In other words, if $a < x < b$, then \begin{equation*}
        f(x) - \frac{f(b) - f(a)}{b-a}(x-a) < f(a)
    \end{equation*}
    or \begin{equation*}
        \frac{f(x) - f(a)}{x-a} < \frac{f(b) - f(a)}{b-a}
    \end{equation*}
    Hence, $f$ is convex.
\end{proof*}


\begin{thm}{}{}
    If $f$ is differentiable and the graph of $f$ lies above each tangent line except at the point of contact, then $f$ is convex.
\end{thm}
\begin{proof*}{}{}
    Let $a < b$. Since the tangent lien at $(a,f(a))$ is the graph of the function \begin{equation*}
        g(x) = f'(a)(x-a) + f(a)
    \end{equation*}
    and since $(b,f(b))$ lies above the tangent line, we have \begin{equation*}
        (1)\hspace{5pt}f(b) > f'(a)(b-a) + f(a)
    \end{equation*}
    Similarly, since the tangent line at $(b,f(b))$ is the graph of $h(x) =f'(b)(x-b) + f(b)$, and $(a,f(a))$ lies above the tangent line at $(b,f(b))$, we have \begin{equation*}
        (2)\hspace{5pt}f(a) > f'(b)(a-b) + f(b)
    \end{equation*}
    It follows from $(1)$ and $(2)$ that $f'(a) < f'(b)$. Then, from our previous theorem we have that $f$ is convex.
\end{proof*}



\section{Inverse Functions}


\begin{defn}{}{}
    For any function $f$. the \Emph{inverse image} of $f$, denoted by $f^{-1}$, is the set of all pairs $(a,b)$ such that $(b,a) \in f$.
\end{defn}

\begin{rmk}{}{}
    $f^{-1}$ is a function if and only if $f$ is one-to-one.
\end{rmk}


\begin{thm}{}{}
    If $f$ is increasing (decreasing) on an interval $I$, then $f$ is injective on $I$ so $f^{-1}$ is a function and in fact $f^{-1}$ is increasing (decreasing).
\end{thm}
\begin{proof*}{}{}
    Consider the case that $f$ is increasing. Then suppose $a,b \in I$ with $a \neq b$. Without loss of generality suppose $a < b$. Then since $f$ is increasing $f(a) < f(b)$ so in particular $f(a) \neq f(b)$. Therefore, $f$ is injective as claimed, so $f^{-1}$ is a well-defined function on $I$. Now, consider $a' < b'$ in $f(I) = I'$. Then there exist $x,y \in I$ such that $f(x) = a'$ and $f(y) = b'$, so in particular $f^{-1}(a') = x$ and $f^{-1}(b') = y$. Since $f$ is increasing and $f(x) = a' < b' = f(y)$ we must have that $x < y$. Thus, $f^{-1}(a') = x < y = f^{-1}(b')$, so $f^{-1}$ is increasing as claimed.

    Consider the case that $f$ is decreasing. Then $-f$ is increasing so it is injective and $-f^{-1}$ is increasing by the first case. Hence, we have that $f^{-1}$ is decreasing as desired.
\end{proof*}


\begin{thm}{}{}
    If $f$ is continuous and one-to-one on an interval $I$, then $f$ is either increasing or decreasing on $I$.
\end{thm}
\begin{proof*}{}{}
    We proceed in three steps:

    (1) If $a < b < c$ are three points in $I$, then I claim either $f(a) < f(b) < f(c)$ or $f(a) > f(b) > f(c)$. Indeed, suppose that $f(a) < f(c)$. If we have $f(b) < f(a)$, then the \nameref{thmname:intval} applied to $[b,c]$ gives an $x \in (b,c)$ such that $f(x) = f(a)$, contradicting the fact that $f$ is injective on $[a,c]$. Similarly, if $f(b) > f(c)$ we would find a contradiction, so $f(a) < f(b) < f(c)$. Similar argumentation leads to the result that $f(a) > f(b) > f(c)$ in the second case.


    (2) If $a < b < c < d$ are four points in $I$, then I claim that either $f(a) < f(b) < f(c) < f(d)$ or $f(a) > f(b) > f(c) > f(d)$. Indeed we can apply (1) to $a<b<c$ and then to $b < c < d$.


    (3) Take any $a < b$ in $I$, and suppose $f(a) < f(b)$. Then $f$ is increasing, for if $c,d \in I$ are any two points, we can apply (2) to the collection $\{a,b,c,d\}$ after arranging them in increasing order.
\end{proof*}


\begin{thm}{}{}
    If $f$ is continuous and one-to-one on an interval, then $f^{-1}$ is also continuous.
\end{thm}
\begin{proof*}{}{}
    Since $f$ is continuous and injective on the interval, it is either increasing or decreasing. Consider the case that $f$ is increasing. We must show that \begin{equation*}
        \lim\limits_{x\rightarrow b}f^{-1}(x) = f^{-1}(b)
    \end{equation*}
    for each $b$ in the domain of $f^{-1}$. Such a number $b$ is of the form $f(a)$ for some $a$ in the domain of $f$. For any $\epsilon > 0$, we want to find a $\delta > 0$ such that for all $x$, if $x \in (f(a) - \delta, f(a) + \delta)$, then $|f^{-1}(x) - a| < \epsilon$, as $a = f^{-1}(b) = f^{-1}(f(a))$. Now, since $a-\epsilon < a <a+\epsilon$ we have that $f(a-\epsilon) < f(a) < f(a+\epsilon)$ since $f$ is presumed increasing. Let $\delta = \min(f(a+\epsilon)-f(a),f(a) - f(a-\epsilon))$. Our choice of $\delta$ ensures that $$f(a-\epsilon) \leq f(a) - \delta\;and\;f(a) + \delta \leq f(a+\epsilon)$$
    Consequently, if $$f(a) - \delta < x < f(a) + \delta$$ then $$f(a-\epsilon) < x < f(a+\epsilon)$$
    SInce $f$ is increasing, $f^{-1}$ is also increasing, and we obtain $$f^{-1}(f(a-\epsilon)) < f^{-1}(x) < f^{-1}(f(a+\epsilon))$$
    so $a-\epsilon < f^{-1}(x) < a+\epsilon$, which is precisely $|f^{-1}(x) - a| < \epsilon$, as desired.
\end{proof*}


\begin{thm}{}{}
    If $f$ is a continuous one-to-one function defined on an interval $I$, and $f'(f^{-1}(a)) = 0$, then $f^{-1}$ is not differentiable at $a$.
\end{thm}
\begin{proof*}{}{}
    We have $f(f^{-1}(x)) = x$. If $f^{-1}$ were differentiable at $a$, then the chain rule would imply that $$f'(f^{-1}(a))\cdot (f^{-1})'(a) = 1$$
    hence $$0\cdot (f^{-1})'(a) = 1$$
    which is impossible.
\end{proof*}


\begin{thm}{}{}
    Let $f$ be a continuous one-to-one function defined on an interval $I$, and suppose that $f$ is differentiable at $f^{-1}(b)$, with derivative $f'(f^{-1}(b)) \neq 0$. Then $f^{-1}$ is differentiable at $b$, and \begin{equation}
        (f^{-1})'(b) = \frac{1}{f'(f^{-1}(b))}
    \end{equation}
\end{thm}
\begin{proof*}{}{}
    Let $b = f(a)$. Then \begin{equation*}
        \lim\limits_{h\rightarrow 0}\frac{f^{-1}(b+h)-f^{-1}(b)}{h} = \lim\limits_{h\rightarrow 0}\frac{f^{-1}(b+h) - a}{h}
    \end{equation*}
    Now, every number $b+h$ in the domain of $f^{-1}$ can be written in the form $b+h = f(a+k)$ for a unique $k(h)$. Then \begin{align*}
        \lim\limits_{h\rightarrow 0}\frac{f^{-1}(b+h) - a}{h} &= \lim\limits_{h\rightarrow 0}\frac{f^{-1}(f(a+k(h)))-a}{f(a+k(h))-b} \\
        &= \lim\limits_{h\rightarrow 0}\frac{k(h)}{f(a+k(h))-f(a)}
    \end{align*}
    Since $b+h = f(a+k(h))$ we have $f^{-1}(b+h) = a+k(h)$, or $k(h) = f^{-1}(b+h)-f^{-1}(b)$. Now, since $f$ is continuous on $I$, $f^{-1}$ is also continuous on its domain, and in particular it is continuous at $b$. This means that $\lim\limits_{h\rightarrow 0}k(h) = 0$, so $k(h)$ goes to zero as $h$ goes to $0$. Hence, as $$\lim\limits_{k\rightarrow 0}\frac{f(a+k)-f(a)}{k} = f'(a) = f'(f^{-1}(b)) \neq 0$$ this implies that $f^{-1}$ is differentiable at $b$ and \begin{equation*}
        (f^{-1})'(b) = \frac{1}{f'(f^{-1}(b))}
    \end{equation*}
\end{proof*}






%%%%%%%%%%%%%%%%%%%%%% - P1.Chapter 3
\chapter{Integration}

\section{Introduction to Definite Integrals}

\begin{defn}{}{}
    Let $a < b$. A \Emph{partition} of the interval $[a,b]$ is a finite collection of points in $[a,b]$, one of which is $a$, and one of        which is $b$.
\end{defn}
The points in a partition can be numbered $t_0,...,t_n$ so that \begin{equation}
    a = t_0 < t_1 < ... < t_{n-1} < t_n = b
\end{equation}
we shall always assume that such a numbering has been assigned.

\begin{defn}{}{}
    Suppose $f$ is bounded on $[a,b]$ and $P = \{t_0,...,t_n\}$ is a partition of $[a,b]$. Let \begin{align}
        m_i &= \inf\{f(x):t_{i-1} \leq x \leq t_i\} \\
        M_i &= \sup\{f(x):t_{i-1}\leq x \leq t_i\}
    \end{align}
    The \Emph{lower sum} of $f$ for $P$, denoted $L(f,P)$, is defined as \begin{equation}
        L(f,P) := \sum\limits_{i=1}^nm_i(t_i-t_{i-1})
    \end{equation}
    The \Emph{upper sum} of $f$ for $P$, denoted $U(f,P)$, is defined as \begin{equation}
        U(f,P) = \sum\limits_{i=1}^nM_i(t_i-t_{i-1})
    \end{equation}
\end{defn}


\begin{rmk}{}{}
    If $P$ is any partition, then \begin{equation}
        L(f,P) \leq U(f,P)
    \end{equation}
    because \begin{align*}
        L(f,P) &= \sum\limits_{i=1}^nm_i(t_i-t_{i-1}) \\
        U(f,P) &= \sum\limits_{i=1}^nM_i(t_i - t_{i-1})
    \end{align*}
    and for each $i$ we have $m_i(t_i-t_{i-1}) \leq M_i(t_i-t_{i-1})$.
\end{rmk}

\begin{lem}{}{}
    If $Q$ is a partition of $[a,b]$ which contains $P$, then \begin{align*}
        L(f,P) &\leq L(f,Q) \\
        U(f,P) &\geq U(f,Q)
    \end{align*}
\end{lem}
\begin{proof*}{}{}
    Consider first the special case in which $Q$ contains just one more point than $P$;\begin{align*}
        P &=\{t_0,...,t_n\} \\
        Q &= \{t_0,...,t_{k-1},u,t_k,...,t_n\}
    \end{align*}
    where $$a= t_0 < t_1 < ... < t_{k-1} < u < t_k < ... < t_n = b$$
    Let \begin{align*}
        m' &= \inf\{f(x):t_{k-1}\leq x \leq u\} \\
        m'' &= \inf\{f(x):u \leq x \leq t_k\}
    \end{align*}
    Then \begin{align*}
        L(f,P) &= \sum\limits_{i=1}^nm_i(t_i - t_{i-1}) \\
        L(f,Q) &= \sum\limits_{i=1}^{k-1}m_i(t_i - t_{i-1}) + m'(u-t_{k-1}) + m''(t_k-u) + \sum\limits_{i=k+1}^nm_i(t_i - t_{i-1})
    \end{align*}
    To prove that $L(f,P) \leq L(f,Q)$ it therefore suffices to show that \begin{equation*}
        m_k(t_k-t_{k-1}) \leq m'(u-t_{k-1}) + m''(t_k-u)
    \end{equation*}
    Now, the set $\{f(x):t_{k-1}\leq x \leq t_k\}$ contains all the numbers in $\{f(x):t_{k-1}\leq x \leq u\}$ and possibly some smaller        ones, so the greatest lower bound of the first set is less than or equal to the greatest lower bound of the second; thus                    \begin{equation*}
        m_k \leq m'
    \end{equation*}
    Similarly, \begin{equation*}
        m_k \leq m''
    \end{equation*}
    Therefore, \begin{equation*}
        m_k(t_k-t_{k-1}) = m_k(t_k-u)+m_k(u-t_{k-1}) \leq m''(t_k-u)+m'(u-t_{k-1})
    \end{equation*}
    This proves, in this special case that $L(f,P) \leq L(f,Q)$. Now, let \begin{align*}
        M' &= \sup\{f(x):t_{k-1} \leq x \leq u\} \\
        M'' &= \sup\{f(x):u \leq x \leq t_k\}
    \end{align*}
    Then \begin{align*}
        U(f,P) &= \sum\limits_{i=1}^nM_i(t_i - t_{i-1}) \\
        U(f,Q) &= \sum\limits_{i=1}^{k-1}M_i(t_i - t_{i-1}) + M'(u-t_{k-1}) + M''(t_k-u) + \sum\limits_{i=k+1}^nM_i(t_i - t_{i-1})
    \end{align*}
    Hence, to prove that $U(f,Q) \leq U(f,P)$ it suffices to show that \begin{equation*}
        M'(u-t_{k-1}) + M''(t_k - u) \leq M_k(t_k-t_{k-1})
    \end{equation*}
    As before, the set $\{f(x):t_{k-1}\leq x \leq t_k\}$ contains all the numbers in $\{f(x):t_{k-1}\leq x \leq u\}$ and possibly some          larger ones, so the smallest upper bound of the first set is greater than or equal to the smallest upper bound of the second; thus          \begin{equation*}
        M_k \geq M'
    \end{equation*}
    Similarly, \begin{equation*}
        M_k \geq M''
    \end{equation*}
    Therefore, \begin{equation*}
        M_k(t_k-t_{k-1}) = M_k(t_k - u) + M_k(u-t_{k-1}) \geq M''(t_k-u) + M'(u-t_{k-1})
    \end{equation*}
    This proves, in this special case that $U(f,P) \geq U(f,Q)$.


    The general case can now be deduced quite easily. The partition $Q$ can be obtained from $P$ by adding one point at a time; in otherwords, there is a sequence of partition \begin{equation*}
        P = P_1\subsetneq P_2 \subsetneq P_3 \subsetneq ... \subsetneq P_{\alpha} = Q
    \end{equation*}
    such that $P_{j+1} = P_j\cup\{u_{j+1}\}$ for some $u_{j+1} \in [a,b]-P_j$. Then \begin{equation*}
        L(f,P) = L(f,P_1) \leq L(f,P_2) \leq ... \leq L(f,P_{\alpha}) = L(f,Q)
    \end{equation*}
    and \begin{equation*}
        U(f,P) = U(f,P_1) \geq U(f,P_2) \geq ... \geq U(f,P_{\alpha}) = U(f,Q)
    \end{equation*}
    completing the proof.
\end{proof*}

\clearpage

\begin{thm}{}{}
    Let $P_1$ and $P_2$ be partitions of $[a,b]$, and let $f$ be a function which is bounded on $[a,b]$. Then \begin{equation}
        L(f,P_1) \leq U(f,P_2)
    \end{equation}
\end{thm}
\begin{proof*}{}{}
    There is a partition $P$ which contains both $P_1$ and $P_2$ (let $P = P_1 \cup P_2$). According to the lemma \begin{equation*}
        L(f,P_1) \leq L(f,P) \leq U(f,P) \leq U(f,P_1)
    \end{equation*}
\end{proof*}

\begin{rmk}{}{}
    It follows that any upper sum $U(f,P')$ is an upper bound for the set of all lower sums $L(f,P)$. Consequently, any upper sum $U(f,P')$     is greater than or equal to the least upper bound of all lower sums:\begin{equation}
        \sup\{L(f,P):P\subset [a,b];\exists n \in \N,|P| = n\} \leq U(f,P')
    \end{equation}
    for every partition $P'$ of $[a,b]$. This, in turn, means that $\sup\{L(f,P)\}$ is a lower bound for the set of all upper sums of $f$.      Consequently, \begin{equation}
        \sup\{L(f,P)\} \leq \inf\{U(f,P)\}
    \end{equation}
    It is clear that for all partitions $P'$, \begin{equation}
        L(f,P') \leq \sup\{L(f,P\} \leq \inf\{U(f,P)\} \leq U(f,P')
    \end{equation}
\end{rmk}

\begin{defn}{Definite Integral}{}
    A function $f$ which is bounded on $[a,b]$ is \Emph{integrable} on $[a,b]$ if $$\sup\{L(f,P):P\text{ a partition of } [a,b]\} =
    \inf\{U(f,P):P\text{ a partition of } [a,b]\}$$
    In this case, this common number is called the \Emph{integral} of $f$ on $[a,b]$ and is denoted by \begin{equation}                             \int_a^bf
    \end{equation}
    The integral $\int_a^bf$ is also called the \Emph{area} of $R(f,a,b)$ when $f(x) \geq 0$ for all $x \in [a,b]$.
\end{defn}


\begin{thm}{}{}
    If $f$ is bounded on $[a,b]$, then $f$ is integrable on $[a,b]$ if and only if for every $\epsilon > 0$ there is a partition $P$ of $[a,    b]$ such that $$U(f,P) - L(f,P) < \epsilon$$
\end{thm}
\begin{proof*}{}{}
    Suppose first that for every $\epsilon > 0$ there is such a partition $P$. Since \begin{align*}
        \inf\{U(f,P')\} &\leq U(f,P) \\
        \sup\{L(f,P')\} &\geq L(f,P)
    \end{align*}
    it follows that \begin{equation*}
        \inf\{U(f,P')\} - \sup\{L(f,P')\} \leq U(f,P) - L(f,P) < \epsilon
    \end{equation*}
    Since this is true for all $\epsilon > 0$, it follows that \begin{equation*}
        \sup\{L(f,P')\} = \inf\{U(f,P')\}
    \end{equation*}
    so by definition, then, $f$ is integrable. Next, if $f$ is integrable then \begin{equation*}
        \sup\{L(f,P)\} = \inf\{U(f,P)\}
    \end{equation*}
    Let $M$ denote the value of this. Then for each $\epsilon > 0$ there exist partitions $P'$ and $P''$ such that $|U(f,P') - M| <\epsilon/    3$ and $|L(f,P'') - M| < \epsilon/2$. Then as $U(f,P') \geq L(f,P'')$ from the previous theorem, we have that \begin{equation*}
        U(f,P') - L(f,P'') = |U(f,P') - L(f,P'')| \leq |U(f,P') - M| + |M - L(f,P'')| < \epsilon
    \end{equation*}
    Let $P = P' \cup P''$ be a partition. Then, according to the lemma $U(f,P) \leq U(f,P')$ and $L(f,P) \geq L(f,P'')$ so \begin{equation*}
		U(f,P) - L(f,P) \leq U(f,P'') - L(f,P') <\epsilon
	\end{equation*}
\end{proof*}


\begin{thm}{}{}
    If $f$ is continuous on $[a,b]$, then $f$ is integrable on $[a,b]$.
\end{thm}
\begin{proof*}{}{}
    Notice, first, that $f$ is bounded on $[a,b]$, because it is continuous on $[a,b]$. To prove that $f$ is integrable on $[a,b]$, we want to use our previous theorem, and show that for every $\epsilon > 0$ there is a partition $P$ of $[a,b]$ such that \begin{equation*}
        U(f,P) - L(f,P) < \epsilon
    \end{equation*}
    Now we know, by our result on uniform continuity, that $f$ is uniformly continuous on $[a,b]$. So there is some $\delta > 0$ such that for all $x,y \in [a,b]$, if $|x-y| < \delta$, then $|f(x) - f(y)| < \epsilon/[2(b-a)]$. We choose a partition $P = \{t_0,...,t_n\}$ such that each $|t_i-t_{i-1}| < \delta$. Then for each $i$ we have \begin{equation*}
        |f(x) - f(y)| < \frac{\epsilon}{2(b-a)}
    \end{equation*}
    for all $x,y \in [t_{i-1},t_i]$. Then, for the sake of contradiction suppose $M_i - m_i > \frac{\epsilon}{2(b-a)} = \epsilon'$, and let $\delta = \frac{M_i-m_i - \epsilon'}{2}$. Then as $M_i - \delta$ and $m_i + \delta$ are not upper and lower bounds of $\{f(x):t_{i-1} \leq x \leq t_i\}$ respectively, there exist $f(u),f(v) \in \{f(x):t_{i-1} \leq x \leq t_i\}$ such that $M_i - \delta < f(u) \leq M_i$ and $m_i \leq f(v) < m_i+\delta$. It follows that \begin{equation*}
        \epsilon'=M_i-m_i-2\delta < f(u) - f(v) \leq |f(u) - f(v)| < \epsilon'
    \end{equation*}
    However, this implies that $\epsilon' < \epsilon'$, a contradiction. Thus, we have that \begin{equation*}
        M_i - m_i \leq \frac{\epsilon}{2(b-a)} < \frac{\epsilon}{b-a}
    \end{equation*}
    Since this is true for all $i$, we have that \begin{align*}
        U(f,P) - L(f,P) &= \sum_{i=1}^n(M_i-m_i)(t_i-t_{i-1}) \\
        &< \frac{\epsilon}{b-a}\sum_{i=1}^n(t_i-t_{i-1}) \\
        &= \frac{\epsilon}{b-a}(b-a) \\
        &= \epsilon
    \end{align*}
    Thus, by our previous theorem $f$ is integrable.
\end{proof*}



\begin{thm}{}{}
    Let $a < c < b$. If $f$ is integrable on $[a,b]$, then $f$ is integrable on $[a,c]$ and one $[c,b]$. Conversely, if $f$ is integrable on $[a,c]$ and on $[c,b]$, then $f$ is integrable on $[a.b]$. Finally, if $f$ is integrable on $[a,b]$, then \begin{equation}
        \int_a^bf = \int_a^cf + \int_c^bf
    \end{equation}
\end{thm}
\begin{proof*}{}{}

    (1) Suppose $f$ is integrable on $[a,b]$. Then $f$ is bounded on $[a,b]$, so it is bounded on $[a,c]$ and $[c,b]$. Indeed, $f$ being bounded implies that there exists $M \in \R$ such that for all $x \in [a,b]$ $|f(x)| \leq M$. Thus, as this applies for all $x \in [a,b]$ and $[a,c],[c,b] \subset [a,b]$, we have that it holds for all $x \in [a,c]$ and all $x \in [c,b]$. Now fix $\epsilon > 0$. Then there exists a partition $P$ of $[a,b]$ such that $$U(f,P) - L(f,P) < \epsilon$$
    Without loss of generality suppose $c=t_j$ for some $t_j \in P = \{t_0,t_1,...,t_n\}$. Then we have partitions $P' = \{t_0,...,t_j\}$ and $P'' = \{t_j,...,t_n\}$ for $[a,c]$ and $[c,b]$ respectively. Moreover, \begin{align*}
        U(f,P) &= U(f,P') + U(f,P'') \\
        L(f,P) &= L(f,P') + L(f,P'')
    \end{align*}
    Hence, we have that \begin{equation*}
        [U(f,P') - L(f,P')] + [U(f,P'') - L(f,P'')] = U(f,P) - L(f,P) < \epsilon
    \end{equation*}
    But $U(f,P') \geq L(f,P')$ and $U(f,P'') \geq L(f,P'')$, so \begin{align*}
        U(f,P') - L(f,P') &\leq U(f,P) - L(f,P) < \epsilon \\
        U(f,P'') - L(f,P'') &\leq U(f,P) - L(f,P) < \epsilon
    \end{align*}
    Therefore, $f$ is integrable on $[a,c]$ and $[c,b]$


    (2) Suppose $f$ is integrable on $[a,c]$ and $[c,b]$. Thus, there exists $M_1, M_2 \in \R$ such that for all $x \in [a,c]$ $|f(x)| \leq M_1$ and for all $x \in [c,b]$ $|f(x)| \leq M_2$. Let $M = \max(M_1,M_2)$. Then for all $x \in [a,b]$ we have $|f(x)| \leq M$, so $f$ is bounded on $[a,b]$. Let $\epsilon > 0$. Then there exist partitions $P_1,P_2$ of $[a,c]$ and $[c,b]$ respectively such that \begin{align*}
        U(f,P_1) - L(f,P_1) &< \epsilon/2 \\
        U(f,P_2) - L(f,P_2) &< \epsilon/2
    \end{align*}
    Let $P = P_1 \cup P_2$, where $P_1 \cap P_2 = \{c\}$. Then we have that \begin{equation*}
        U(f,P) - L(f,P) = [U(f,P_1) - L(f,P_1)] + [U(f,P_2) - L(f,P_2)] < \epsilon/2 + \epsilon/2 = \epsilon
    \end{equation*}
    Therefore, by definition $f$ is integrable on $[a,b]$.
    

    (3) Suppose $f$ is integrable on $[a,b]$, so by the previous results $f$ is integrable on $[a,c]$ and $[c,b]$. Let $\int_a^bf = R$, $\int_a^cf = R_1$, and $\int_c^bf = R_2$. Let $P$ be a partition of $[a,b]$, and without loss of generality suppose $c \in P = \{t_0,...,t_j = c,...,t_n\}$. Then let $P_1 = \{t_0,...,t_j\}$ and $P_2 = \{t_j,...,t_n\}$ be partitions of $[a,c]$ and $[c,b]$. It then follows that \begin{align*}
        L(f,P_1) \leq &R_1 \leq U(f,P_1) \\
        L(f,P_2) \leq &R_2 \leq U(f,P_2)
    \end{align*}
    Hence, we have that \begin{align*}
        L(f,P) &= L(f,P_1) + L(f,P_2) \leq R_1 + R_2 \\
        U(f,P) &= U(f,P_1) + U(f,P_2) \geq R_1 + R_2
    \end{align*}
    Thus $L(f,P) \leq R_1 + R_2\leq U(f,P)$. Note that this holds for all partitions $P$, as if $P'$ is a partition, then considering the partition $P'_{c} = P' \cup \{c\}$ we have that $$L(f,P') \leq L(f,P'_{c}) \leq R_1 + R_2 \leq U(f,P'_{c}) \leq U(f,P')$$
    Therefore, this holds for all partitions of $[a,b]$, but $R$ is the unique number which does this so we must have that $R = R_1 + R_2$. Thus \begin{equation*}
        \int_a^bf = \int_a^cf + \int_c^bf
    \end{equation*}
\end{proof*}


\begin{defn}{}{}
    Using the previous theorem, we defin \begin{equation}
        \int_a^af := 0 \;\;\;and\;\;\;\int_a^bf := -\int_b^af,\;for\;a>b
    \end{equation}
\end{defn}


\begin{thm}{}{}
    If $f$ and $g$ are integrable on $[a,b]$, then $f+g$ is integrable on $[a,b]$ and \begin{equation}
        \int_a^b(f+g) = \int_a^bf+\int_a^bg
    \end{equation}
\end{thm}
\begin{proof*}{}{}
    Let $P = \{t_0,...,t_n\}$ be a partition of $[a,b]$. Let \begin{align*}
        m_i &= \inf\{(f+g)(x):t_{i-1} \leq x \leq t_i\} \\
        m_i'&= \inf\{f(x):t_{i-1} \leq x \leq t_i\} \\
        m_i''&= \inf\{g(x):t_{i-1} \leq x \leq t_i\} 
    \end{align*}
    and define $M_i,M_i',M_i''$ similary. Then it follows that \begin{equation*}
        m_i \geq m_i' + m_i''
    \end{equation*}
    and \begin{equation*}
        M_i \leq M_i' + M_i''
    \end{equation*}
    Therefore, we have that \begin{align*}
        L(f+g,P) &\geq L(f,P) + L(g,P) \\
        U(f+g,P) &\leq U(f,P) + U(g,P)
    \end{align*}
    Thus we have that \begin{equation*}
        L(f,P) + L(g,P) \leq L(f+g,P) \leq U(f+g,P) \leq U(f,P) + U(g,P)
    \end{equation*}
    Fix $\epsilon > 0$. Since $f$ and $g$ are integrable there exist partitions $P'$ and $P''$ of $[a,b]$ such that \begin{align*}
        U(f,P') - L(f,P') &< \epsilon/2 \\
        U(g,P'') - L(g,P'') &< \epsilon/2
    \end{align*}
    Let $P_{\epsilon} = P' \cup P''$. Then we have that \begin{equation*}
        U(f+g,P_{\epsilon}) - L(f+g,P_{\epsilon}) \leq [U(f,P_{\epsilon}) - L(f,P_{\epsilon})] + [U(g,P_{\epsilon}) - L(g,P_{\epsilon})] < \epsilon/2 + \epsilon/2 = \epsilon
    \end{equation*}
    Therefore $f+g$ is integrable on $[a,b]$. Moreover, \begin{align*}
        L(f,P) + L(g,P) \leq L(f+g,P) \leq \int_a^b(f+g) \leq U(f+g,P) \leq U(f,P) + U(g,P)
    \end{align*}
    and also \begin{equation*}
        L(f,P) + L(g,P) \leq \int_a^bf + \int_a^bg \leq U(f,P) + U(g,P)
    \end{equation*}
    Then, observe that \begin{align*}
        -[U(f,P) + U(g,P)] + (L(f,P) +L(g,P)) &\leq \left(\int_a^bf+\int_a^bg\right) - \int_a^b(f+g)\\
        &\leq U(f,P) + U(g,P) - (L(f,P) +L(g,P))
    \end{align*}
    But, this applies for all partitions $P$, so in particular from above we have that for $\epsilon > 0$ there exists a partition $P_{\epsilon}$ such that $$[U(f,P_{\epsilon}) - L(f,P_{\epsilon})] + [U(g,P_{\epsilon}) - L(g,P_{\epsilon})] < \epsilon$$
    Thus, we have for all $\epsilon > 0$ that \begin{equation*}
        \left|\left(\int_a^bf+\int_a^bg\right) - \int_a^b(f+g)\right| < \epsilon
    \end{equation*}
    Consequently, we have that \begin{equation*}
        \int_a^b(f+g) = \int_a^bf + \int_a^bg
    \end{equation*}
\end{proof*}



\begin{thm}{}{}
    If $f$ is integrable on $[a,b]$, then for any $c \in \R$, the function $cf$ is integrable on $[a,b]$ and \begin{equation}
        \int_a^bcf = c\cdot\int_a^bf
    \end{equation}
\end{thm}
\begin{proof*}{}{}
    (1) Consider $c \geq 0$. Let $P = \{t_0,...,t_n\}$ be a partition of $[a,b]$, define \begin{align*}
        m_i &= \inf\{(cf)(x):t_{i-1} \leq x \leq t_i\} \\
        m_i' &= \inf\{f(x):t_{i-1} \leq x \leq t_i\} 
    \end{align*}
    and define $M_i$ and $M_i'$ similarly. I claim that $m_i \geq cm_i'$ and $M_i \leq cM_i'$. Indeed, for all $x \in [t_{i-1},t_i]$ we have that $(cf)(x) \leq cM_i'$, so $M_i \leq cM_i'$. Similarly, $(cf)(x) \geq cm_i'$ so $m_i \geq cm_i'$. Then it follows that $$L(cf,P) \geq cL(f,P)$$ and $$U(cf,P) \leq cU(f,P)$$
    Thus, we have that $$cL(f,P) \leq L(cf,P) \leq U(cf,P) \leq cU(f,P)$$
    If $c = 0$ then we have that $0 \leq L(cf,P) \leq U(cf,P) \leq 0$, so $L(cf,P) = U(cf,P) = 0$ for all partitions $P$, and consequently $cf$ is integrable with $\int_a^bcf = 0$. Otherwise, fix $\epsilon > 0$. Then there exists a partition $P'$ such that $$U(f,P') - L(f,P') < \epsilon/c$$
    Then it follows that \begin{equation*}
        U(cf,P') - L(cf,P') \leq cU(f,P') - cL(f,P') < \epsilon
    \end{equation*}
    Thus $cf$ is integrable on $[a,b]$. Then we have that \begin{equation*}
        cL(f,P) \leq L(cf,P) \leq \int_a^bcf \leq U(cf,P) \leq cU(f,P)
    \end{equation*}
    and also \begin{equation*}
        cL(f,P) \leq c\int_a^bf\leq cU(f,P)
    \end{equation*}
    Then we have that \begin{equation*}
        \left|\int_a^bcf - c\int_a^bf\right| \leq cU(f,P) - cL(f,P) 
    \end{equation*}
    where we can make $cU(f,P) - cL(f,P)$ as small as we want by choosing an appropriate partition. Therefore, \begin{equation*}
        \int_a^bcf = c\int_a^bf
    \end{equation*}


    (2) Consider $c < 0$. Let $P = \{t_0,...,t_n\}$ be a partition of $[a,b]$, define \begin{align*}
        m_i &= \inf\{(cf)(x):t_{i-1} \leq x \leq t_i\} \\
        m_i' &= \inf\{f(x):t_{i-1} \leq x \leq t_i\} 
    \end{align*}
    and define $M_i$ and $M_i'$ similarly. I claim that $m_i \geq cM_i'$ and $M_i \leq cm_i'$. Indeed, for all $x \in [t_{i-1},t_i]$ we have that $(cf)(x) = cf(x) \leq cm_i'$, since $f(x) \geq m_i'$ and $c < 0$, so $M_i \leq cm_i'$. Similarly, $(cf)(x) \geq cM_i'$ so $m_i \geq cM_i'$. Then it follows that $$L(cf,P) \geq cU(f,P)$$ and $$U(cf,P) \leq cL(f,P)$$
    Thus, we have that $$cU(f,P) \leq L(cf,P) \leq U(cf,P) \leq cL(f,P)$$
    Note that $U(f,P) \geq L(f,P)$ so as $c < 0$, $cU(f,P) \leq cL(f,P)$. Fix $\epsilon > 0$. Then there exists a partition $P'$ such that $$U(f,P') - L(f,P') < -\epsilon/c$$
    Then it follows that \begin{equation*}
        U(cf,P') - L(cf,P') \leq cL(f,P') - cU(f,P') < \epsilon
    \end{equation*}
    as $\epsilon/c < L(f,P') - U(f,P')$ and $c < 0$. Thus $cf$ is integrable on $[a,b]$. Then we have that \begin{equation*}
        cU(f,P) \leq L(cf,P) \leq \int_a^bcf \leq U(cf,P) \leq cL(f,P)
    \end{equation*}
    and also \begin{equation*}
        cU(f,P) \leq c\int_a^bf\leq cL(f,P)
    \end{equation*}
    Then we have that \begin{equation*}
        \left|\int_a^bcf - c\int_a^bf\right| \leq cL(f,P) - cU(f,P) 
    \end{equation*}
    where we can make $cL(f,P) - cU(f,P)$ as small as we want by choosing an appropriate partition. Therefore, \begin{equation*}
        \int_a^bcf = c\int_a^bf
    \end{equation*}
    completing the proof.
\end{proof*}

\begin{thm}{}{boundint}
    Suppose $f$ is integrable on $[a,b]$ and that \begin{equation}
        m \leq f(x) \leq M \forall x \in [a,b]
    \end{equation}
    Then \begin{equation}
        m(b-a) \leq \int_a^bf \leq M(b-a)
    \end{equation}
\end{thm}
\begin{proof*}{}{}
    It is clear that $M \geq \sup\{f(x):x \in [a,b]\}$ and $m \leq \inf\{f(x):x\in [a,b]\}$, so \begin{equation*}
        m(b-a) \leq L(f,P)\;and\;M(b-a) \geq U(f,P)
    \end{equation*}
    for every partition $P$. Since $\int_a^bf = \sup\{L(f,P)\} = \inf\{U(f,P)\}$ we have that \begin{equation*}
        m(b-a) \leq \sup\{L(f,P)\} = \int_a^bf = \inf\{U(f,P)\} \leq M(b-a)
    \end{equation*}
\end{proof*}

\begin{rmk}{}{}
    If $f$ is integrable on $[a,b]$, we can define a new function $F$ on $[a,b]$ by \begin{equation}
        F(x) = \int_a^xf = \int_a^xf(t)dt
    \end{equation}
\end{rmk}


\begin{thm}{}{}
    If $f$ is integrable on $[a,b]$ and $F$ is defined on $[a,b]$ by \begin{equation*}
        F(x) = \int_a^xf
    \end{equation*}
    then $F$ is continuous on $[a,b]$.
\end{thm}
\begin{proof*}{}{}
    Suppose $c \in [a,b]$. Since $f$ is integrable on $[a,b]$ it is, by definition, bounded on $[a,b]$; let $M$ be a number such that \begin{equation*}
        |f(x)| \leq M,\forall x \in [a,b]
    \end{equation*}
    If $h > 0$ for $c+h \in [a,b]$, then \begin{equation*}
        F(c+h) - F(c) = \int_a^{c+h}f - \int_a^cf = \int_c^{c+h}f
    \end{equation*}
    Since $-M \leq f(x) \leq M$ for all $x \in [a,b]$ it follows from the Theorem \ref{thm:boundint} that \begin{equation*}
        -M\cdot h \leq \int_c^{c+h}f \leq M\cdot h
    \end{equation*}
    In other words \begin{equation*}
        -M\cdot h \leq F(c+h) - F(c) \leq M\cdot h
    \end{equation*}
    If $h < 0$, with $c+h \in [a,b]$ we find the inequality \begin{equation*}
        -M\cdot h \geq F(c+h) - F(c) \geq M\cdot h
    \end{equation*}
    In either case we have that \begin{equation*}
        |F(c+h) - F(c)| \leq M\cdot |h|
    \end{equation*}
    Therefore, if $\epsilon > 0$, we have \begin{equation*}
        |F(c+h) - F(c)| < \epsilon
    \end{equation*}
    provided that $|h| < \epsilon/M$. This proves that \begin{equation*}
        \lim\limits_{h\rightarrow 0}F(c+h) = F(c)
    \end{equation*}
    so $F$ is continuous at $c$.
\end{proof*}

\section{Reimann Sums}

\begin{defn}{}{}
    Suppose $P = \{t_0,...,t_n\}$ is a partition of $[a,b]$, and for each $i$ choose $x_i \in [t_{i-1},t_i$. Then we clearly have that \begin{equation}
        L(f,P) \leq \sum\limits_{i=1}^nf(x_i)(t_i-t_{i-1}) \leq U(f,P)
    \end{equation}
    Any sum of the form \begin{equation}
        \sum\limits_{i=1}^nf(x_i)(t_i-t_{i-1})
    \end{equation}
    is called a \Emph{Reimann sum} of $f$ for $P$.
\end{defn}


\begin{thm}{}{}
    Suppose that $f$ is integrable on $[a,b]$. Then for every $\epsilon > 0$ there is some $\delta > 0$ such that, if $P = \{t_0,...,t_n\}$ is any partition of $[a,b]$ with $t_i - t_{i-1} < \delta$ for all $i \in \{1,...,n\}$, then \begin{equation*}
        \left|\sum\limits_{i=1}^nf(x_i)(t_i-t_{i-1}) - \int_a^bf(x)dx\right| < \epsilon
    \end{equation*}
    for any Riemann sum formed by choosing $x_i \in [t_{i-1},t_i]$.
\end{thm}
\begin{proof*}{}{}
    Note that for any partition $P$ both the integral and the Riemann sum lie between $L(f,P)$ and $U(f,P)$. Thus, it suffices to show that for any given $\epsilon > 0$, there exists a $\delta > 0$ such that $U(f,P) - L(f,P) < \epsilon$ for any partition $P$ with $t_i - t_{i-1} < \delta$ for all $i \in \{1,2,...,n\}$.

    Since $f$ is integrable on $[a,b]$ it is also bounded, so there exists $M \in \R$ such that $|f(x)| \leq M$ for all $x \in [a,b]$. First choose some particular partition $P^* = \{u_0,...,u_K\}$ for which \begin{equation*}
        U(f,P^*) - L(f,P^*) < \epsilon/2
    \end{equation*}
    and then choose a $\delta$ such that \begin{equation*}
        \delta < \frac{\epsilon}{4MK}
    \end{equation*}
    For any partition $P$ with $t_i - t_{i-1} < \delta$, we can write the sum \begin{equation*}
        U(f,P) - L(f,P) = \sum_{i=1}^n(M_i-m_i)(t_i-t_{i-1})
    \end{equation*}
    as two sums. Let the first consist of those $i's$ forwhcih $[t_{i-1},t_i] \subseteq [u_{j-1},u_j]$ for some $j$. Evidently, this sum is $\leq U(f,P^*) - L(f,P^*) <\epsilon/2$. For all other $i$ we will have $t_{i-1} < u_j < t_i$ for some $j \in \{1,...,K-1\}$, so there are at most $K-1$ of them. Consequently, the sum for these terms is $< (K-1)\cdot 2M\cdot \delta M < \epsilon/2$. Then we have that $U(f,P) - L(f,P) < \epsilon$, completing the proof.
\end{proof*}


\section{The Fundamental Theorem of Calculus}

\begin{namthm}{The Fundamental Theorem of Calculus}{FTC}
    Let $f$ be integrable on $[a,b]$, and define $F$ on $[a,b]$ by \begin{equation}
        F(x) = \int_a^xf = \int_a^xf(t)dt
    \end{equation}
    If $f$ is continuous at $c$ in $[a,b]$, then $F$ is differentiable at $c$, and \begin{equation}
        F'(c) = f(c)
    \end{equation}
    (if $c = a$ or $b$, then $F'(c)$ is understood to mean the right or left hand derivative of $F$).
\end{namthm}
\begin{proof*}{}{}
    First, consider $c \in (a,b)$. By definition,\begin{equation*}
        F'(c) = \lim\limits_{h\rightarrow 0}\frac{F(c+h)-F(c)}{h}
    \end{equation*}
    Suppose first that $h > 0$. Then \begin{equation*}
        F(c+h) - F(c) = \int_c^{c+h}f
    \end{equation*}
    Define $m_h$ and $M_h$ as follows:\begin{align*}
        m_h &= \inf\{f(x):c\leq x \leq c+h\} \\
        M_h &= \sup\{f(x):c\leq x \leq c+h\}
    \end{align*}

    It follows from Theorem \ref{thm:boundint} that \begin{equation*}
        m_h\cdot h\leq \int_c^{c+h}f \leq M_h\cdot h
    \end{equation*}
    Therefore, \begin{equation*}
        m_h\leq \frac{F(c+h) - F(c)}{h} \leq M_h
    \end{equation*}
    If $h < 0$, let \begin{align*}
        m_h &= \inf\{f(x):c+h\leq x \leq c\} \\
        M_h &= \sup\{f(x):c+h\leq x \leq c\}
    \end{align*}

    It follows from Theorem \ref{thm:boundint} that \begin{equation*}
        m_h\cdot (-h)\leq \int_{c+h}^{c}f \leq M_h\cdot (-h)
    \end{equation*}
    Since \begin{equation*}
        F(c+h) - F(c) = \int_c^{c+h}f = -\int_{c+h}^cf
    \end{equation*}
    this yields \begin{equation*}
        m_h\cdot h\geq F(c+h) - F(c) \geq M_h\cdot h
    \end{equation*}
    Since $h < 0$, we have that \begin{equation*}
        m_h \leq \frac{F(c+h) - F(c)}{h} \leq M_h
    \end{equation*}
    This inequality is true for any integrable function, continuous or not. Since $f$ is continuous at $c$, however, \begin{equation*}
        \lim\limits_{h\rightarrow 0}m_h = \lim\limits_{h\rightarrow 0}M_h = f(c)
    \end{equation*}
    and this proves that \begin{equation*}
        F'(c) = \lim\limits_{h\rightarrow 0}\frac{F(c+h) - F(c)}{h} = f(c)
    \end{equation*}


    Now, if $c = a$ we need only look at when $h > 0$, and in this case we still have  \begin{equation*}
        m_h \leq \frac{F(a+h) - F(a)}{h} \leq M_h
    \end{equation*}
    and from our previous limits, \begin{equation*}
        \lim\limits_{h\rightarrow 0^+}m_h = \lim\limits_{h\rightarrow 0^+}m_h = f(a)
    \end{equation*}
    thus we have that $$F'(a) = \lim\limits_{h\rightarrow 0^+}\frac{F(a+h)-F(a)}{h} = f(a)$$
    Similarly, if $c = b$ we need only look at $h < 0$, so we have that \begin{equation*}
        \lim\limits_{h\rightarrow 0^-}m_h = \lim\limits_{h\rightarrow 0^-}m_h = f(b)
    \end{equation*}
    and $$F'(b) = \lim\limits_{h\rightarrow 0^-}\frac{F(b+h)-F(b)}{h} = f(b)$$
    completing the proof.
\end{proof*}


\begin{rmk}{}{}
    We may consider \begin{equation}
        F(x) = \int_a^xf
    \end{equation}
    when $x < a$. In this case we have that \begin{equation}
        F(x) = -\int_x^af = -\left(\int_b^af - \int_b^xf\right)
    \end{equation}
    so for $c \in [a,b]$, \begin{equation}
        F'(c) = -(-f(c)) = f(c)
    \end{equation}
    as before.
\end{rmk}


\begin{cor}{}{}
    If $f$ is continuous on $[a,b]$ and $f = g'$ for some function $g$, then \begin{equation}
        \int_a^bf = g(b) - f(a)
    \end{equation}
\end{cor}
\begin{proof*}{}{}
    Let \begin{equation*}
        F(x) = \int_a^xf
    \end{equation*}
    Then $F' = f = g'$ on $[a,b]$. Consequently there is a number $c$ such that \begin{equation*}
        F = g+c
    \end{equation*}
    Note that $$0 = F(a) = g(a) + c$$ so $c = -g(a)$; thus \begin{equation*}
        F(x) = g(x) - g(a)
    \end{equation*}
    This is true, in particular, for $x = b$. THus \begin{equation*}
        \int_a^bf = F(b) = g(b) - g(a)
    \end{equation*}
\end{proof*}

\begin{rmk}{}{}
    It is important to note that this is merely a useful result for certain functiosn $f$, \Emph{not} a definition.
\end{rmk}


\begin{namthm}{Second Fundamental Theorem of Calculus}{FTC2}
    If $f$ is integrable on $[a,b]$ and $f = g'$ for some function $g$, then \begin{equation*}
        \int_a^bf = g(b) - g(a)
    \end{equation*}
\end{namthm}
\begin{proof*}{}{}
    Let $P = \{t_0,...,t_n\}$ be any partition of $[a,b]$. By \nameref{thmname:meanval} there is a point $x_i \in [t_{i-1},t_i]$ such that \begin{align*}
        g(t_i) - g(t_{i-1}) &= g'(x_i)(t_i-t_{i-1}) \\
        &= f(x_i)(t_i-t_{i-1})
    \end{align*}
    If \begin{align*}
        m_i &= \inf\{f(x):t_{i-1} \leq x \leq t_i\} \\
        M_i &= \sup\{f(x):t_{i-1} \leq x \leq t_i\}
    \end{align*}
    then clearly \begin{equation*}
        m_i(t_i-t_{i-1}) \leq f(x_i)(t_i-t_{i-1}) \leq M_i(t_i-t_{i-1})
    \end{equation*}
    that is \begin{equation*}
        m_i(t_i-t_{i-1}) \leq g(t_i) - g(t_{i-1}) \leq M_i(t_i-t_{i-1})
    \end{equation*}
    Adding these equations for $i \in \{1,2,...,n\}$ we obtain \begin{equation*}
        L(f,P) \leq g(b) - g(a) \leq U(f,P)
    \end{equation*}
    for every partition $P$. This means that \begin{equation*}
        g(b) - g(a) = \int_a^bf
    \end{equation*}
\end{proof*}

\begin{rmk}{}{}
    If $f$ is any bounded function on $[a,b]$, then \begin{equation}
        \sup\{L(f,P)\}\;\;and\;\;\inf\{U(f,P)\}
    \end{equation}
    will both exist. These numbers are called the \Emph{lower integral} of $f$ on $[a,b]$ and the \Emph{upper integral} of $f$ on $[a,b]$, respectively, and will be denoted by \begin{equation}
        \mathbf{L}\int_a^bf\;\;and\;\;\mathbf{U}\int_a^bf
    \end{equation}
    If $a < c < b$, then \begin{equation}
        \mathbf{L}\int_a^bf = \mathbf{L}\int_a^cf + \mathbf{L}\int_c^bf\;\;and\;\;\mathbf{U}\int_a^bf = \mathbf{U}\int_a^cf + \mathbf{U}\int_c^bf
    \end{equation}
    and if $m \leq f(x) \leq M$ for all $x \in [a,b]$, then \begin{equation}
        m(b-a)\leq \mathbf{L}\int_a^bf \leq \mathbf{U}\int_a^bf\leq M(b-a)
    \end{equation}
    $f$ is integrable precisely when \begin{equation}
        \mathbf{L}\int_a^bf = \mathbf{U}\int_a^bf
    \end{equation}
\end{rmk}
\begin{proof*}{}{}
    (Left to the reader)
\end{proof*}


\begin{rmk}{}{}
    We shall now demonstrate an alternate proof for the following theorem stated previously.
\end{rmk}


\begin{thm}{}{}
    If $f$ is continuous on $[a,b]$, then $f$ is integrable on $[a,b]$.
\end{thm}
\begin{proof*}{}{}
    Define function $L$ and $U$ on $[a,b]$ by \begin{equation*}
        L(x) = \mathbf{L}\int_a^xf\;\;and\;\;U(x) = \mathbf{U}\int_a^xf
    \end{equation*}
    Let $x \in (a,b)$. If $h > 0$ and \begin{align*}
        m_h &= \inf\{f(t):x\leq t \leq x+h\} \\
        M_h &= \sup\{f(t): x\leq t\leq x+h\} 
    \end{align*}
    then \begin{equation*}
        m_h\cdot h \leq \mathbf{L}\int_x^{x+h}f \leq \mathbf{U}\int_x^{x+h}f\leq M_h\cdot h
    \end{equation*}
    so \begin{equation*}
        m_h\cdot h \leq L(x+h) - L(x) \leq U(x+h) - U(x) \leq M_h\cdot h
    \end{equation*}
    or \begin{equation*}
        m_h\leq \frac{L(x+h)-L(x)}{h} \leq \frac{U(x+h)-U(x)}{h} \leq M_h
    \end{equation*}
    If $h < 0$ and \begin{align*}
        m_h &= \inf\{f(t):x+h\leq t \leq x\} \\
        M_h &= \sup\{f(t): x+h\leq t\leq x\} 
    \end{align*}
    one obtains the same inequality, precisely as in the proof of \nameref{thmname:FTC}.

    Since $f$ is continuous at $x$, we have \begin{equation*}
        \lim\limits_{h\rightarrow 0}m_h = \lim\limits_{h\rightarrow 0}M_h = f(x)
    \end{equation*}
    and this proves that \begin{equation*}
        L'(x) = U'(x) = f(x),\forall x\in(a,b)
    \end{equation*}
    THis means that there is a number $c$ such that \begin{equation*}
        U(x) = L(x) + c,\forall x \in [a,b]
    \end{equation*}
    Since $U(a) = L(a) = 0$, the number $c$ must be equal to $0$, so \begin{equation*}
        U(x) = L(x) \forall x \in [a,b]
    \end{equation*}
    In particular, \begin{equation*}
        \mathbf{U}\int_a^bf = U(b) = L(b) = \mathbf{L}\int_a^bf
    \end{equation*}
    so $f$ is integrable on $[a,b]$.
\end{proof*}


\begin{subappendices}
    \section{Trigonometric Functions}

    \begin{defn}{}{}
        We define the mathematical constant $\pi$ as the area of the unit circle, or in this case, twice the area of a semi-circle:\begin{equation}
            \pi:=2\cdot \int_{-1}^1\sqrt{1-x^2}dx
        \end{equation}
    \end{defn}


    \begin{defn}{}{}
        If $-1 \leq x \leq 1$, then the area of the sector bounded between the upper unit circle from $[x,1]$ and the $x$-axis and radial arm is \begin{equation}
            A(x) := \frac{x\sqrt{1-x^2}}{2} + \int_x^1\sqrt{1-t^2}dt
        \end{equation}
    \end{defn}

    \begin{rmk}{}{}
        For $-1 < x < 1$, $A$ is differentiable at $x$ and \begin{align*}
            A'(x) &= \frac{1}{2}\left[\sqrt{1-x^2} +x\cdot\frac{-2x}{2\sqrt{1-x^2}}\right] -\sqrt{1-x^2} \\
            &= \frac{1}{2}\frac{1-x^2-x^2}{\sqrt{1-x^2}} - \frac{1-x^2}{\sqrt{1-x^2}} \\
            &= \frac{1}{2}\frac{-1}{\sqrt{1-x^2}} \\
            &= \frac{-1}{2\sqrt{1-x^2}}
        \end{align*}
        Note that on $[-1,1]$, the function $A$ decreases from $A(-1) = \frac{\pi}{2}$ to $A(1) = 0$.
    \end{rmk}

    \begin{defn}{}{}
        If $0 \leq x \leq \pi$, then $\cos x$ is the unique number in $[-1,1]$ such that \begin{equation}
            A(\cos x) = \frac{x}{2}
        \end{equation}
        and \begin{equation}
            \sin x := \sqrt{1-(\cos x)^2}
        \end{equation}
        Note that such a $\cos x$ exists as $A$ is continuous on $[-1,1]$, and $A(-1) = \frac{\pi}{2}$ while $A(1) = 0$. Hence, by \nameref{thmname:intval} there exists $y \in [-1,1]$ such that $A(y) = \frac{x}{2}$ for all $x \in [0,\pi]$.
    \end{defn}


    \begin{thm}{}{}
        If $0 < x < \pi$, then \begin{align*}
            \cos'(x) &= -\sin x \\
            \sin'(x) &= \cos x
        \end{align*}
    \end{thm}
    \begin{proof*}{}{}
        If $B = 2A$, then the definition $A(\cos x) = x/2$ can be written \begin{equation*}
            B(\cos x) = x
        \end{equation*}
        in other words, $\cos$ is just the inverse of $B$. We have already computed taht \begin{equation*}
            A'(x) = -\frac{1}{2\sqrt{1-x^2}}
        \end{equation*}
        from which we conclude \begin{equation*}
            B'(x) = -\frac{1}{\sqrt{1-x^2}}
        \end{equation*}
        Consequently we have that \begin{align*}
            \cos'(x) &= (B^{-1})'(x) \\
            &= \frac{1}{B'(B^{-1}(x))} \\
            &= \frac{1}{-\frac{1}{\sqrt{1-[B^{-1}(x)]^2}}} \\
            &= -\sqrt{1-(\cos x)^2} \\
            &= - \sin x
        \end{align*}
        Then, since $\sin x = \sqrt{1-(\cos x)^2}$ we also obtain \begin{align*}
            \sin'(x) &= \frac{1}{2}\cdot \frac{-2\cos x\cdot \cos'(x)}{\sqrt{1-(\cos x)^2}} \\
            &= \frac{-\cos x\cdot (-\sin x)}{\sin x}\\
            &= \cos x
        \end{align*}
    \end{proof*}
    

    \begin{defn}{}{}
        Now, to define $\sin$ and $\cos$ on $\R$, we proceed as follows \begin{enumerate}
            \item If $\pi \leq x \leq 2\pi$, the \begin{align*}
                    \sin x &= -\sin(2\pi - x) \\
                    \cos x &= \cos(2\pi - x)
                \end{align*}
            \item If $x = 2\pi k+x'$ for some integer $k$ and some $x' \in [0,2\pi]$, then \begin{align*}
                    \sin x &= \sin x' \\
                    \cos x &= \cos x'
            \end{align*}
        \end{enumerate}
    \end{defn}


    \begin{lem}{}{}
        Suppose $f$ has a second derivative everywhere and that \begin{align*}
            f''+f&= 0\\
            f(0) &= 0\\
            f'(0) &= 0\\
        \end{align*}
        Then $f = 0$
    \end{lem}
    \begin{proof*}{}{}
        Multiplying both sides of the first equation by $f'$ yields \begin{equation*}
            f'f'' + ff' = 0
        \end{equation*}
        Thus \begin{equation*}
            [(f')^2+f^2]' = 2(f'f'' + ff') = 0
        \end{equation*}
        so $(f')^2+f^2$ is a constant function. From $f(0) = 0$ and $f'(0) = 0$ it follows that the constant is $0$; thus \begin{equation*}
            [f'(x)]^2+[f(x)]^2=0\forall x
        \end{equation*}
        This implies that \begin{equation*}
            f(x) = 0 \forall x
        \end{equation*}
    \end{proof*}

    \begin{thm}{}{}
        If $f$ has a second derivative everywhere and \begin{align*}
            f'' + f &= 0 \\
            f(0) &= a \\
            f'(0) &= b 
        \end{align*}
        then \begin{equation*}
            f = b\cdot \sin + a \cdot \cos
        \end{equation*}
    \end{thm}
    \begin{proof*}{}{}
        Let \begin{equation*}
            g(x) = f(x) - b\sin x - a \cos x
        \end{equation*}
        Then \begin{align*}
            g'(x) &= f'(x) - b\cos x + a \sin x \\
            g''(x) &= f''(x) + b\sin x + a\cos x
        \end{align*}
        Consequently, \begin{align*}
            g'' + g &= 0 \\
            g(0) &= 0 \\
            g'(0) &= 0
        \end{align*}
        which shows by the previous lemma that \begin{equation*}
            0 = g(x) = f(x) - b\sin x - a\cos x, \forall x
        \end{equation*}
    \end{proof*}


    \begin{thm}{}{}
        If $x$ and $y$ are any two numbers, then \begin{align*}
            \sin(x+y) &= \sin x\cos y + \cos x \sin y \\
            \cos(x+y) &= \cos x\cos y - \sin x \sin y
        \end{align*}
    \end{thm}
    \begin{proof*}{}{}
        For any particular $y \in \R$, we can define a function $f$ by \begin{equation*}
            f(x) = \sin(x+y)
        \end{equation*}
        Then $f'(x) = \cos(x+y)$ and $f''(x) = -\sin(x+y)$. Consequently, \begin{align*}
            f'' + f &= 0 \\
            f(0) &= \sin y \\
            f'(0) &= \cos y
        \end{align*}
        It follows from the previous theorem that \begin{equation*}
            f = (\cos y)\cdot \sin + (\sin y) \cdot \cos
        \end{equation*}
        that is \begin{equation*}
            \sin(x+y) = \cos y\sin x+\sin y \cos x,\forall x
        \end{equation*}
        Since any number $y$ could have been chosen to begin with, this proves the first formula for $x$ and $y$.


        Similarly, for any $y \in \R$ define $f(x) = \cos(x+y)$, so $f'(x) = -\sin(x+y)$ and $f''(x) = -\cos(x+y)$. Then $f'' + f = 0$, $f(0) = \cos y$ and $f'(0) = -\sin y$. Then we have that \begin{equation*}
            \cos(x+y) = \cos y\cos x - \sin y \sin x
        \end{equation*}
        proving the second formula.
    \end{proof*}

    \begin{rmk}{}{}
        Since \begin{equation*}
            \arcsin'(x) = \frac{1}{\sqrt{1-x^2}}, -1 < x < 1
        \end{equation*}
        it follows from \nameref{thmname:FTC2} that \begin{equation*}
            \arcsin x = \arcsin x - \arcsin 0 = \int_0^x\frac{1}{\sqrt{1-t^2}}dt
        \end{equation*}
        Using this definition of $\arcsin$ we could define $\sin$ as $\arcsin^{-1}$, and the formula for the derivative of an inverse function would show that \begin{equation*}
            \sin'(x) = \sqrt{1-\sin^2 x}
        \end{equation*}
        which could be defined as $\cos x$.
    \end{rmk}


    \section{The Logarithm and Exponential Functions}
    
    \begin{defn}{}{}
        If $x > 0$, then define \begin{equation}
            \log x := \int_1^x\frac{1}{t}dt
        \end{equation}
    \end{defn}

    \begin{thm}{}{}
        If $x.y > 0$, then \begin{equation}
            \log(xy) = \log x+ \log y
        \end{equation}
    \end{thm}
    \begin{proof*}{}{}
        Notice first that $\log'(x) = 1/x$, by \nameref{thmname:FTC}. Now, choose a number $y > 0$ and let \begin{equation*}
            f(x) = \log(xy)
        \end{equation*}
        Then\begin{equation*}
            f'(x) = \log'(xy)\cdot y = \frac{1}{xy}\cdot y = \frac{1}{x}
        \end{equation*}
        Thus, $f' = \log'$. This means that there is a number $c$ such that $f(x) = \log(x) + c$ for all $x > 0$, that is, \begin{equation*}
            \log(xy) = \log x+c,\;\forall x > 0
        \end{equation*}
        The number $c$ can be evaluated by noting that $\log(1) = 0$, so $\log(1\cdot y) = c$. Thus \begin{equation*}
            \log(xy) = \log x + \log y
        \end{equation*}
        for all $x$. Since this is true for all $y > 0$, the theorem is proved.
    \end{proof*}

    \begin{cor}{}{}
        If $n$ is a natural number and $x > 0$, then \begin{equation}
            \log(x^n) = n\log x
        \end{equation}
    \end{cor}
    \begin{proof*}{}{}
        We proceed by induction on $n \in \N$. If $n = 1$ we simply have $\log(x^1) = 1\cdot \log x$, so the base case holds. Now suppose inductively that there exists $k \geq 1$ such that if $n = k$, \begin{equation*}
            \log(x^k) = k\log x
        \end{equation*}
        Then, observe that by the previous theorem \begin{align*}
            \log(x^{k+1}) &= \log(x^kx) \\
            &= \log(x^k) + \log x \\
            &= k\log x + \log x \tag{by the Induction Hypothesis} \\
            &= (k+1)\log x
        \end{align*}
        as desired. Thus by mathematical induction we conclude that for all $n \geq 1$, $\log(x^n) = n\log x$.
    \end{proof*}


    \begin{cor}{}{}
        If $x,y > 0$, then \begin{equation*}
            \log\left(\frac{x}{y}\right) = \log x - \log y
        \end{equation*}
    \end{cor}
    \begin{proof*}{}{}
        This result follows from the equation \begin{equation*}
            \log x = \log \left(\frac{x}{y}\cdot y\right) = \log\left(\frac{x}{y}\right) + \log y
        \end{equation*}
    \end{proof*}

    \begin{defn}{}{}
        The \Emph{exponential function}, $\exp$, is defined as $\log^{-1}$.
    \end{defn}

    \begin{thm}{}{}
        For all numbers $x$,\begin{equation*}
            \exp'(x) = \exp(x)
        \end{equation*}
    \end{thm}
    \begin{proof*}{}{}
        Observe that \begin{align*}
            \exp'(x) &= (\log^{-1})'(x) \\
            &= \frac{1}{\log'(\log^{-1}(x))} \\
            &= \frac{1}{\frac{1}{\log^{-1}(x)}} \\
            &= \log^{-1}(x) = \exp(x)
        \end{align*}
    \end{proof*}

    \begin{thm}{}{}
        If $x$ and $y$ are any two numbers, then \begin{equation*}
            \exp(x+y) = \exp(x)\cdot \exp(y)
        \end{equation*}
    \end{thm}
    \begin{proof*}{}{}
        Let $x' = \exp(x)$ and $y' = \exp(y)$, so that $x = \log x'$ and $y = \log y'$. Then \begin{equation*}
            x+y = \log x' + \log y' = \log(x'y')
        \end{equation*}
        This means that \begin{equation*}
            \exp(x+y) = x'y' = \exp(x) \cdot \exp(y)
        \end{equation*}
    \end{proof*}


    \begin{defn}{}{}
        We define \begin{equation}
            e := \exp(1)
        \end{equation}
        and this is equivalent to the equation \begin{equation}
            1 = \log e = \int_1^e\frac{1}{t}dt
        \end{equation}
        Then, we note that $\exp(x) = [\exp(1)]^x = e^x$ for rational $x$, so we define for any $x \in \R$, \begin{equation}
            e^x =\exp(x)
        \end{equation}
    \end{defn}


    \begin{defn}{}{}
        If $a > 0$, then, for any real number $x$, \begin{equation}
            a^x := e^{x\log a}
        \end{equation}
        If $a = e$ this definition agrees with our previous one.
    \end{defn}


    \begin{thm}{}{}
        If $a > 0$, then \begin{equation*}
            (1)\;\;(a^b)^c = a^{bc},\;\forall a,b\in\R
        \end{equation*}
        and \begin{equation*}
            (2)\;\;a^1=a\;and\;a^{x+y}=a^x\cdot a^y,\;\forall x,y\in\R
        \end{equation*}
    \end{thm}
    \begin{proof*}{}{}
        First, observe that \begin{align*}
            (a^b)^c &= e^{c\log a^b} \\
            &= e^{c\log e^{b\log a}} \\
            &= e^{cb\log a} \\
            &= a^{bc}
        \end{align*}
        Next, observe that \begin{equation*}
            a^1 = e^{1\log a} = e^{\log a} = a
        \end{equation*}
        and \begin{align*}
            a^{x+y} &= e^{(x+y)\log a} \\
            &= e^{x\log a + y\log a} \\
            &= e^{x\log a} \cdot e^{y\log a} \\
            &= a^x \cdot a^y
        \end{align*}
    \end{proof*}

    \begin{thm}{}{}
        If $f$ is differentiable and \begin{equation*}
            f'(x) = f(x),\;\forall x\in\R
        \end{equation*}
        then there is a number $c$ such that \begin{equation*}
            f(x) = ce^x,\;\forall x\in\R
        \end{equation*}
    \end{thm}
    \begin{proof*}{}{}
        Let $g(x) = f(x)/e^x$, which is possible as $e^x \neq 0$ for all $x$. Then \begin{equation*}
            g'(x) = \frac{e^xf'(x) - f(x)e^x}{(e^x)^2} = 0
        \end{equation*}
        THerefore, there is a number $c$ such that \begin{equation*}
            g(x) = \frac{f(x)}{e^x} = c,\;\forall x
        \end{equation*}
    \end{proof*}
    

    \begin{thm}{}{}
        For any natural number $n$, \begin{equation}
            \lim\limits_{x\rightarrow \infty}\frac{e^x}{x^n} = \infty
        \end{equation}
    \end{thm}
    \begin{proof*}{}{}
        Step 1. We claim that $e^x > x$ for all $x$, and consequently $\lim\limits_{x\rightarrow \infty}e^x = \infty$.

        For $x \leq 0$ this is immediate. Now, it suffices to show $x > \log x$ for all $x > 0$. If $x < 1$ this clearly holds since $\log x < 0$. If $x > 1$, then $x-1$ is an upper sum for $f(t) = \frac{1}{t}$ on $[1,x]$, so $\log x < x-1 < x$.


        Step 2. We claim $\lim\limits_{x\rightarrow \infty}\frac{e^x}{x} = \infty$. First, note that \begin{equation*}
            \frac{e^x}{x} = \frac{e^{x/2}\cdot e^{x/2}}{\frac{x}{2}\cdot 2} = \frac{1}{2}\left(\frac{e^{x/2}}{\frac{x}{2}}\right)\cdot e^{x/2}
        \end{equation*}
        By Step 1. the expression in parentheses is greater than $1$, and $\lim\limits_{x\rightarrow \infty}e^{x/2} = \infty$; this shows that $\lim\limits_{x\rightarrow \infty}e^x/x = \infty$.


        Step 3. To prove the main claim note that \begin{equation*}
            \frac{e^x}{x^n} = \frac{(e^{x/n})^x}{\left(\frac{x}{n}\right)^n\cdot n^n} = \frac{1}{n^n}\cdot \left(\frac{e^{x/n}}{\frac{x}{n}}\right)^n
        \end{equation*}
        The expression in parentheses becomes arbitrarily large, by Step 2., so the $n$th power certainly becomes arbitrarily large.
    \end{proof*}




\end{subappendices}



%%%%%%%%%%%%%%%%%%%%%% - P1.Chapter 4
\chapter{Sequences and Series}

\section{Approximation by Polynomial Functions}

\begin{defn}{}{}
    Given a function $f$ that is $n$ times differentiable in a neighborhood of a point $a$, let \begin{equation*}
        a_k = \frac{f^{(k)}(a)}{k!},\;\;0\leq k \leq n,
    \end{equation*}
    and define \begin{equation*}{}{}
        P_{n,a}(x) = a_0 + a_1(x-a) + \hdots + a_x(x-a)^n
    \end{equation*}
    The polynomial $P_{n,a}$ is called the \Emph{Taylor polynomial of degree $n$ for $f$ at $a$}. 
\end{defn}

\begin{rmk}{}{}
    The Taylor polynomial has been defined so that \begin{equation*}{}{}
        P_{n,a}^{(k)}(a) = f^{(k)}(a)\;\;\text{ for }\;0\leq k \leq n
    \end{equation*}
    in fact, it is evidently the only polynomial of degree $\leq n$ with this property.
\end{rmk}

\begin{eg*}{}{}
    Consider the $\sin$ function. We have \begin{align*}
        \sin(0) &= 0 \\
        \sin'(0) &= \cos 0 = 1 \\
        \sin''(0) &= -\sin 0 = 0 \\
        \sin'''(0) &= -\cos 0 = -1 \\
        \sin^{(4)}(0) &= \sin 0 = 0
    \end{align*}
    From this point on, the derivatives repeat modulo $4$. The coefficients become \begin{equation*}
        a_k = \frac{\sin^{(k)}(0)}{k!} = \left\{\begin{array}{lc} 0 & \text{if } \exists l \in \N; k = 2l \\
            \frac{(-1)^{l}}{(2l+1)!} & \text{if } \exists l \in \N; k = 2l+1
        \end{array}\right.
    \end{equation*}
    Therefore, the Taylor polynomial $P_{2n+1,0}$ of degree $2n+1$ for $\sin$ at $0$ is \begin{equation*}
        P_{2n+1,0}(x) = x-\frac{x^3}{3!}+\frac{x^5}{5!} - \frac{x^7}{7!}+\hdots + (-1)^n\frac{x^{2n+1}}{(2n+1)!}
    \end{equation*}
    Via a similar derivation we find that the Taylor polynomial $P_{2n,0}$ of degree $2n$ for $\cos$ at $0$ is \begin{equation*}
        P_{2n,0} = 1 - \frac{x^2}{2!} + \frac{x^4}{4!} - \frac{x^6}{6!} + \hdots + (-1)^n\frac{x^{2n}}{(2n)!}
    \end{equation*}
\end{eg*}


\begin{eg*}{}{}
    Note that for all $k \geq 0$, $\exp^{(k)}(0) = \exp(0) = 1$, so the Taylor polynomial of degree $n$ for $\exp$ at $0$ is \begin{equation*}
        P_{n,0}(x) = 1 + \frac{x}{1!} + \frac{x^2}{2!} + \frac{x^3}{3!} + \hdots + \frac{x^n}{n!}
    \end{equation*}
    For $\log$, observe that \begin{align*}
        \log'(x) &= \frac{1}{x}, &\log'(1) = 1 \\
        \log''(x) &= -\frac{1}{x^2}, &\log''(1) = -1 \\
        \log'''(x) &= \frac{2}{x^3}, &\log'''(1) = 2
    \end{align*}
    in general \begin{equation*}
        \log^{(k)}(x) = \frac{(-1)^{k-1}(k-1)!}{x^k}, \;\;\log^{(k)}(1) = (-1)^{k-1}(k-1)!
    \end{equation*}
    for $k \geq 1$, and $\log(1) =0$. Therefore, the Taylor polynomial of degree $n$ for $\log$ at $1$ is \begin{equation*}
        P_{n,1}(x) = (x-1) -\frac{(x-1)^2}{2} +\frac{(x-1)^3}{3} - \hdots + \frac{(-1)^{n-1}(x-1)^n}{n}
    \end{equation*}
    If we consider the function $f(x) = \log(1+x)$, then the Taylor polynomial of degree $n$ of $f$ at $0$ is \begin{equation*}
        P_{n,0}(x) = x - \frac{x^2}{2} + \frac{x^3}{3} - \frac{x^4}{4} + \hdots + \frac{(-1)^{n-1}x^n}{n}
    \end{equation*}
\end{eg*}


\begin{thm}{}{smoldiff}
    Suppose that $f$ is a function and $a \in \R$ such that \begin{equation*}
        f'(a),...,f^{(n)}(a)
    \end{equation*}
    all exist. Let \begin{equation*}
        a_k = \frac{f^{(k)}}{k!}, 0\leq k \leq n
    \end{equation*}
    and define \begin{equation*}
        P_{n,a}(x) = a_0 + a_1(x-a) + \hdots + a_n(x-a)^n
    \end{equation*}
    Then \begin{equation*}
        \lim\limits_{x\rightarrow a}\frac{f(x)-P_{n,a}(x)}{(x-a)^n} = 0
    \end{equation*}
\end{thm}
\begin{proof*}{}{}
    Weiting out $P_{n,a}(x)$ expliticly we obtain \begin{equation*}
        \frac{f(x) - P_{n,a}(x)}{(x-a)^n} = \frac{f(x) - \sum\limits_{i=0}^{n-1}\frac{f^{(i)}(a)}{i!}(x-a)^i}{(x-a)^n} - \frac{f^{(n)}}{n!}
    \end{equation*}
    Let us introduce the new functions \begin{equation*}
        Q(x) = \sum\limits_{i=0}^{n-1}\frac{f^{(i)}(a)}{i!}(x-a)^i\;\text{ and }\; g(x) = (x-a)^n;
    \end{equation*}
    now we must prove that \begin{equation*}
        \lim\limits_{x\rightarrow a}\frac{f(x) - Q(x)}{g(x)} = \frac{f^{(n)}(a)}{n!}
    \end{equation*}
    Note that $Q^{(k)}(a) = f^{(k)}(a)$ for $k \leq n-1$, and $g^{(k)}(x) =n!\frac{(x-a)^{n-k}}{(n-k)!}$. By the continuity of $f^{(k)}$ and $Q^{(k)}$ for $k \leq n-1$, we have that \begin{equation*}
        \lim\limits_{x\rightarrow a}\left[f^{(k)}(x) - Q^{(k)}(x)\right] = f^{(k)}(a) - Q^{(k)}(a) = 0
    \end{equation*}
    and \begin{equation*}
        \lim\limits_{x\rightarrow a}g^{(k)}(x) = 0
    \end{equation*}
    Then applying l'Hopital's Rule $n-1$ times, we obtain \begin{equation*}
        \lim\limits_{x\rightarrow a}\frac{f(x)-Q(x)}{(x-a)^n} = \lim\limits_{x\rightarrow a}\frac{f^{(n-1)}(x) - Q^{(n-1)}(x)}{n!(x-a)}
    \end{equation*}
    But the $n-1$st derivative of $Q$ is constant, and in fact $Q^{(n-1)}(x) = f^{(n-1)}(a)$, so \begin{equation*}
        \lim\limits_{x\rightarrow a}\frac{f(x)-Q(x)}{(x-a)^n} = \lim\limits_{x\rightarrow a}\frac{f^{(n-1)}(x)-f^{(n-1)}(a)}{n!(x-a)} = \frac{f^{(n)}(a)}{n!}
    \end{equation*}
    by definition of $f^{(n)}(a)$, as desired.
\end{proof*}


\begin{thm}{}{}
    Suppose that \begin{equation*}
        f'(a) = ... = f^{(n-1)}(a) = 0,\;\text{ and }\;f^{(n)}(a) \neq 0
    \end{equation*}
    \begin{enumerate}
        \item If $n$ is even and $f^{(n)}(a) > 0$, then $f$ has a local minimum at $a$.
        \item If $n$ is even and $f^{(n)}(a) < 0$, then $f$ has a local maximum at $a$.
        \item If $n$ is odd, then $f$ has neither a local maximum nor a local minimum at $a$.
    \end{enumerate}
\end{thm}
\begin{proof*}{}{}
    Let $f$ be as in the hypothesis, and without loss of generality let $f(a) = 0$, as otherwise one can replace $f$ with $f-f(a)$ without affecting the hypothesis. THen, since the first $n-1$ derivatives of $f$ at $a$ are $0$, the Taylor polynomial$ P_{n,a}$ of $f$ is \begin{align*}
        P_{n,a}(x) &= \sum\limits_{i=0}^n\frac{f^{(i)}(a)}{i!}(x-a)^i \\
        &= \frac{f^{(n)}(a)}{n!}(x-a)^n 
    \end{align*}
    Thus, Theorem \ref{thm:smoldiff} states that \begin{equation*}
        0 = \lim\limits_{x\rightarrow a} \frac{f(x) - P_{n,a}(x)}{(x-a)^n} = \lim\limits_{x\rightarrow a} \left[\frac{f(x)}{(x-a)^n} - \frac{f^{(n)}(a)}{n!}\right]
    \end{equation*}
    Consequently, if $x$ is sufficiently close to $a$, then $f(x)/(x-a)^n$ has the same sign as $f^{(n)}(a)/n!$. Suppose now that $n$ is even. In this case $(x-a)^n > 0$ for all $x \neq a$. Since $f(x)/(x-a)^n$ has the same sign as $f^{(n)}(a)/n!$ for $x$ sufficeintly close to $a$, it follows that $f(x)$ itself has the same sign as $f^{(n)}(a)/n!$ for $x$ sufficiently close to $a$. If $f^{(n)}(a) > 0$, this means that \begin{equation*}
        f(x) > 0 = f(a)
    \end{equation*}
    for $x$ close to $a$. Consequently, $f$ has a local minimum at $a$. If $f^{(n)}(a) < 0$, this means that \begin{equation*}
        f(x) < 0 = f(a)
    \end{equation*}
    for $x$ close to $a$, so $f$ has a local minimum at $a$.

    Conversely, suppose that $n$ is odd. Then if $x$ is sufficiently close to $a$ $f(x)/(x-a)^n$ always has the same sign, since it has the same sign as $f^{(n)}(a)/n!$ which is constant. But $(x-a)^n >0$ for $x >a$ and $(x-a)^n < 0$ for $x < a$. Therefore $f(x)$ has different signs for $x > a$ and $x <a$. Hence, $f$ has neither a local maximum nor a local minimum at $a$.
\end{proof*}

\begin{defn}{}{}
    Two functions $f$ and $g$ are \Emph{equal up to order $n$ at $a$} if \begin{equation*}
        \lim\limits_{x\rightarrow a}\frac{f(x) - g(x)}{(x-a)^n} = 0
    \end{equation*}
\end{defn}


\begin{thm}{}{}
    Let $P$ and $Q$ be two polynomials in $(x-a)$, of degree $\leq n$, and suppose that $P$ and $Q$ are equal up to order $n$ at $a$. Then $P = Q$.
\end{thm}
\begin{proof*}{}{}
    Let $R= P-Q$. Since $R$ is a polynomial of degree $\leq n$, it is only necessariy to prove that if $R(x) = b_0+\hdots +b_n(x-a)^n$ satisfies \begin{equation*}
        \lim\limits_{x\rightarrow a}\frac{R(x)}{(x-a)^n} = 0
    \end{equation*}
    then $R = 0$. Now the hypothesis on $R$ surely implies that \begin{equation*}
        \lim\limits_{x\rightarrow a}\frac{R(x)}{(x-a)^i} = 0\;\;\text{ for } 0\leq i \leq n
    \end{equation*}
    For $i = 0$ this condition reads that $\lim\limits_{x\rightarrow a}R(x) = 0$. On the other hand $\lim\limits_{x\rightarrow a}R(x) = b_0$. Thus $b_0 = 0$. Similarly, we find that \begin{equation*}
        \frac{R(x)}{x-a} = b_1+b_2(x-a)+\hdots + b_n(x-a)^{n-1}
    \end{equation*}
    and \begin{equation*}
        \lim\limits_{x\rightarrow a}\frac{R(x)}{x-a} = b_1
    \end{equation*}
    so $b_1 = 0$. Continuing in this way, by induction we find that \begin{equation*}
        b_0 = b_1 = ... = b_n = 0
    \end{equation*}
    so $R(x) = 0$ and $P = Q$.
\end{proof*}


\begin{cor}{}{}
    Let $f$ be $n$-times differentiable at $a$, and suppose that $P$ is a polynomial in $(x-a)$ of degree $\leq n$, which equals $f$ up to order $n$ at $a$. Then $P = P_{n,a,f}$ (The Taylor polynomial of $f$ of degree $n$ at $a$).
\end{cor}
\begin{proof*}{}{}
    Since $P$ and $P_{n,a,f}$ both equal $f$ up to order $n$ at $a$, using the triangle inequality and an epsilon-delta proof it can be shown that $P$ equals $P_{n,a,f}$ up to order $n$ at $a$. Consequently, $P = P_{n,a,f}$ by the preceding Theorem.
\end{proof*}

\begin{defn}{}{}
    If $f$ is a function for which $P_{n,a}(x)$ exists, we define the \Emph{remainder term} $R_{n,a}(x)$ by \begin{equation*}
        R_{n,a}(x) = f(x) - P_{n,a}(x)
    \end{equation*}
    If $f^{(n+a)}$ is continuous on $[a,x]$, then \begin{equation*}
        R_{n,a}(x) = \int_a^x\frac{f^{(n+1)}(t)}{n!}(x-t)^ndt
    \end{equation*}
\end{defn}

\begin{rmk}{}{}
    If $m$ and $M$ are the minimum and maximum of $f^{(n+1)}{n!}$ on $[a,x]$, then $R_{n,a}(x)$ satisfies \begin{equation*}
        m\in_a^x(x-t)^ndt\leq R_{n,a}(x)\leq M\int_a^x(x-t)^ndt
    \end{equation*}
    so we can write \begin{equation*}
        R_{n,a}(x) = \alpha\cdot \frac{(x-a)^{n+1}}{n+1}
    \end{equation*}
    for some number $\alpha$ between $m$ and $M$. Since we have assumed that $f^{(n+1)}$ is continuous, for some $t \in (a,x)$ we can also write \begin{equation*}
        R_{n,a}(x) = \frac{f^{(n+1)}(t)}{n!}\frac{(x-a)^{n+1}}{n+1} = \frac{f^{(n+1)}(t)}{(n+1)!}(x-a)^{n+1}
    \end{equation*}
    This is known as the \Emph{Lagrange form of the remainder}.
\end{rmk}


\begin{lem}{}{}
    Suppose that the function $R$ is $(n+1)$-times differentiable on $[a,b]$, and \begin{equation*}
        R^{(k)}(a) = 0\;\;\text{ for } k =0,1,2,...,n
    \end{equation*}
    Then for any $x \in (a,b]$ we have \begin{equation*}
        \frac{R(x)}{(x-a)^{n+1}} = \frac{R^{(n+1)}(t)}{(n+1)!}\;\text{ for some $t$ in } (a,x)
    \end{equation*}
\end{lem}
\begin{proof*}{}{}
    For $n = 0$, this is simply \nameref{thmname:meanval}, and we will proceed for the remaining $n$ by induction on $n$. To do this we use \nameref{thmname:caumeanval} to write \begin{equation*}
        \frac{R(x)}{(x-a)^{n+2}} = \frac{R'(z)}{(n+2)(z-a)^{n+1}} = \frac{1}{n+2}\frac{R'(z)}{(z-a)^{n+1}}\;\;\text{ for some $z$ in } (a,x),
    \end{equation*}
    and then apply the induction hypothesis to $R'$ on the interval $[a,z]$ to get \begin{align*}
        \frac{R(x)}{(x-a)^{n+2}} &= \frac{1}{n+2}\frac{(R')^{(n+1)}(t)}{(n+1)!}\;\text{ for some $z$ in } (a,x), \\
        &= \frac{R^{(n+2)}(t)}{(n+2)!}
    \end{align*}
    as desired.
\end{proof*}

\begin{namthm}{Taylor's Theorem}{taythm}
    Suppose $f',...,f^{(n+1)}$ are defined on $[a,x]$, and that $R_{n,a}(x)$ is defined by \begin{equation*}
        R_{n,a}(x) = f(x) - P_{n,a}(x)
    \end{equation*}
    Then \begin{equation*}
        R_{n,a}(x) = \frac{f^{(n+1)}(t)}{(n+1)!}(x-a)^{n+1}\;\;\text{ for some $t$ in } (a,x)
    \end{equation*}
\end{namthm}
\begin{proof*}{}{}
    The function $R_{n,a}$ satisfies the conditions of the preceding Lemma by the definition of the Taylor polynomial, so \begin{equation*}
        \frac{R_{n,a}(x)}{(x-a)^{n+1}} = \frac{R_{n,a}^{(n+1)}(t)}{(n+1)!}
    \end{equation*}
    for some $t$ in $(a,x)$. But, \begin{equation*}
        R_{n,a}^{(n+1)} = f^{(n+1)}
    \end{equation*}
    since $R_{n,a} - f$ is a polynomial of degree $n$. Substituting gives the Lagrange form for the remainder, as desired.
\end{proof*}

\begin{eg*}{}{}
    Applying \nameref{thmname:taythm} to the functions $\sin$, $\cos$, and $\exp$, with $a = 0$, we obtain the following formulas: \begin{align*}
        \sin x &= \sum\limits_{i=0}^n\frac{(-1)^ix^{2i+1}}{(2i+1)!} + \frac{\sin^{(2n+2)}(t)}{(2n+2)!}x^{2n+2} \\
        \cos x &= \sum\limits_{i=0}^n\frac{(-1)^ix^{2i}}{(2i)!} + \frac{\cos^{(2n+1)}(t)}{(2n+1)!}x^{2n+1} \\
        \exp x &= \sum\limits_{i=0}^n\frac{x^n}{n!} + \frac{\exp t}{(n+1)!}x^{n+1}
    \end{align*}
    (of course, we could go one power higher in the remainder terms for $\sin$ and $\cos$)
\end{eg*}


\section{Infinite Sequences}

\begin{defn}{}{}
    An \Emph{infinite sequence} of real numbers is a real valued function whose domain is $\N$.
\end{defn}


\begin{defn}{}{}
    A sequence $\{a_n\}$ \Emph{converges to $\ell \in \R$}, denoted by $\lim\limits_{n\rightarrow \infty}a_n = \ell$, if for every $\epsilon > 0$ there exists $N \in \N$ such that for all $n \in \N$, if $n \geq N$ then \begin{equation*}
        |a_n - \ell| < \epsilon
    \end{equation*}
    A sequence $\{a_n\}$ is said to \Emph{converge} if such an $\ell$ exists, and to \Emph{diverge} otherwise.
\end{defn}

\begin{thm}{Limit Laws}{}
    If $\lim\limits_{n\rightarrow \infty}a_n$ and $\lim\limits_{n\rightarrow \infty}b_n$ both exist, then \begin{align*}
        \lim\limits_{n\rightarrow \infty}(a_n+b_n) &= \lim\limits_{n\rightarrow \infty}a_n + \lim\limits_{n\rightarrow \infty}b_n \\
        \lim\limits_{n\rightarrow \infty}a_n\cdot b_n &= \lim\limits_{n\rightarrow \infty}a_n\cdot \lim\limits_{n\rightarrow \infty} b_n 
    \end{align*}
    moreover, if $\lim\limits_{n\rightarrow \infty}b_n\neq 0$, then $b_n \neq 0$ for all $n$ greater than some $N$, and \begin{equation*}
        \lim\limits_{n\rightarrow \infty}a_n/b_n = \lim\limits_{n\rightarrow \infty}a_n/\lim\limits_{n\rightarrow \infty}b_n
    \end{equation*}
\end{thm}
\begin{proof*}{}{}
    (To be completed)
\end{proof*}


\begin{thm}{}{}
    Let $f$ be a function defined in an open interval containing $c$, except perhaps at $c$ itself, with \begin{equation*}
        \lim\limits_{x\rightarrow c}f(x) = l
    \end{equation*}
    Suppose that $\{a_n\}$ is a sequence such that \begin{enumerate}
        \item each $a_n$ is in the domain of $f$,
        \item each $a_n \neq c$,
        \item $\lim\limits_{n\rightarrow \infty}a_n = c$
    \end{enumerate}
    Then the sequence $\{f(a_n)\}$ satisfies \begin{equation*}
        \lim\limits_{n\rightarrow \infty}f(a_n) = l
    \end{equation*}
    Conversely, if this is true for every sequence $\{a_n\}$ satisfying the above conditions, then $\lim\limits_{x\rightarrow c}f(x) = l$.
\end{thm}
\begin{proof*}{}{}
    Suppose first that $\lim\limits_{x\rightarrow c}f(x) = l$. Then for every $\varepsilon > 0$ there is a $\delta > 0$ such that, for all $x$, \begin{equation*}
        \text{if } 0 < |x-c| < \delta, \text{ then } |f(x) - l| < \varepsilon
    \end{equation*}
    If the sequence $\{a_n\}$ satisfies $\lim\limits_{n\rightarrow \infty}a_n = c$, then there is a natural number $N$ such that for all $n \in \N$, \begin{equation*}
        \text{if } n \geq N, \text{ then } 0 < |a_n - c| < \delta
    \end{equation*}
    By our choice of $\delta$ this implies that \begin{equation*}
        |f(a_n) - l| <\varepsilon
    \end{equation*}
    showing that $\lim\limits_{n\rightarrow \infty} f(a_n) = l$.


    Suppose, conversely, that $\lim\limits_{n\rightarrow \infty}f(a_n) = l$ for every sequence $\{a_n\}$ satisfying our three conditions. If $\lim\limits_{x\rightarrow c}f(x) = l$ were not true, there would be some $\varepsilon > 0$ such that for all $\delta > 0$ there exists $x$ such that $0 < |x-c| < \delta$ but $|f(x) - l| \geq \varepsilon$. In particular, for each $n$ there would exist $x_n$ such that $0<|x_n - c| < 1/n$, but $|f(x_n) - l| \geq \varepsilon$. Define a sequence $\{x_n\}$ using these $x_n$. Then $x_n$ is in the domain of $f$ for each $n$, and as $0 < |x_n-c|$ for each $n$, $x_n \neq c$ for all $n$. Moreover, for all $\varepsilon' > 0$ there exists $N \in \N$ such that $\varepsilon' > 1/N > 0$ (by the Archimedean Property of $\R$), so for all $n \geq N$, $0 < |x_n - c| < 1/n < \varepsilon'$. Consequently, $\lim\limits_{n\rightarrow \infty}x_n = c$, so the sequences satisfies all of our initial conditions. However, then by assumption $\lim\limits_{n\rightarrow \infty}f(x_n) = l$. But, by construction, for $\varepsilon$ we have that for all $n \in \N$, $0 < |x_n - c| < 1/n$ but $|f(x_n) - l| \geq \varepsilon$, so $f(x_n)$ does not converge to $l$, contradicting our hypothesis. Thus $\lim\limits_{x\rightarrow c}f(x) = l$ must be true.
\end{proof*}


\begin{defn}{}{}
    A sequence $\{a_n\}$ is \Emph{increasing} if $a_{n+1} > a_n$ for all $n$, \Emph{nondecreasing} if $a_{n+1} \geq a_n$ for all $n$, and \Emph{bounded above} if there is a number $M$ such that $a_n \leq M$ for all $n$. Similarly, a sequence $\{a_n\}$ is \Emph{decreasing} if $a_{n+1} < a_n$ for all $n$, \Emph{nonincreasing} if $a_{n+1} \leq a_n$ for all $n$, and \Emph{bounded below} if there is a number $m$ such that $a_n \geq m$ for all $n$.
\end{defn}

\begin{thm}{}{}
    If $\{a_n\}$ is nondecreasing and bounded above, then $\{a_n\}$ converges.
\end{thm}
\begin{proof*}{}{}
    The set $A := \{a_n:n\in\N\}$ is, by assumption, bounded above, so $A$ has a least upper bound $\alpha \in \R$. We claim that $\lim\limits_{n\rightarrow \infty}a_n = \alpha$. If $\varepsilon > 0$, then $\alpha - \varepsilon$ is not an upper bound for $A$ so there exists $a_N$ in $A$ such that $a_N > \alpha - \varepsilon$, so $\alpha - a_N < \varepsilon$. Then for all $n \geq N$, $a_n \geq a_N$ since $\{a_n\}$ is nondecreasing so \begin{equation*}
        |\alpha - a_n| = \alpha - a_n \leq \alpha - a_N < \varepsilon
    \end{equation*}
    Consequently, we conclude that $\lim\limits_{n\rightarrow \infty}a_n = \alpha$.
\end{proof*}


\begin{defn}{}{}
    A \Emph{subsequence} of a sequence $\{a_n\}$ is a sequence \begin{equation*}
        a_{n_1},a_{n_2},a_{n_3},...
    \end{equation*}
    where the $n_j$ are natural numbers with $n_1 < n_2 < n_3 < ...$.
\end{defn}

\begin{lem}{}{}
    Any sequence $\{a_n\}$ contains a subsequence which is either nondecreasing or nonincreasing.
\end{lem}
\begin{proof*}{}{}
    Call a natural number $n$ a ``peak point" of a sequence $\{a_n\}$ if $a_m < a_n$ for all $m > n$.\begin{itemize}[leftmargin=+1in]
        \item[Case 1.] The sequence has infinitely many peak points. In this case, if $n_1 < n_2 < n_3 < ...$ are the peak points, then $a_{n_1} > a_{n_2} > a_{n_3} > ...$, so $\{a_{n_j}\}$ is a nonincreasing subsequence of $\{a_n\}$.
        \item[Case 2.] THe sequence has only finitely many peak points. In this case, let $n_1$ be greater than all peak points. Since $n_1$ is not a peak point, there is some $n_2 > n_1$ such that $a_{n_2} \geq a_{n_1}$. Since $n_2$ is not a peak point, there is some $n_3 > n_2$ such that $a_{n_3} > a_{n_2}$. Suppose there exists $k \geq 3$ such that for all $1 \leq m < k$, $a_{n_m} \leq a_{n_{m+1}}$ and $n_m < n_{m+1}$. Then since $n_k$ is not a peak point, there is some $n_{k+1}$ such that $n_{k+1} > n_k$ and $a_{n_{k+1}} \geq a_{n_k}$. Thus, by recursive definition we have constructed a nondecreasing subsequence $\{a_{n_k}\}$ of $\{a_n\}$.
    \end{itemize}
\end{proof*}


\begin{cor}{The Bolzano-Weierstrass Theorem}{bolz}
    Every bounded sequence has a convergent subsequence.
\end{cor}


\begin{defn}{}{}
    A sequence $\{a_n\}$ is a \Emph{Cauchy sequence} if for every $\varepsilon > 0$ there is a natural number $N$ such that, for all $m,n \in \N$, if $m,n \geq N$, then \begin{equation*}
        |a_n-a_m| <\varepsilon
    \end{equation*}
    (This can be written as $\lim\limits_{m,n\rightarrow \infty}|a_m-a_n| = 0$)
\end{defn}


\begin{thm}{}{}
    A sequence $\{a_n\}$ converges if and only if it is a Cauchy sequence.
\end{thm}
\begin{proof*}{}{}
    First assume that $\lim\limits_{n\rightarrow \infty}a_n = l$ for some $l \in \R$. Then given $\varepsilon > 0$, there exists $N \in \N$ such that if $n \geq N$, then $|a_n - l| < \varepsilon/2$. Hence, if $m,n \geq N$, then $$|a_n - a_m| \leq |a_n - l| + |a_m-l| < \varepsilon/2+\varepsilon/2 = \varepsilon$$
    Thus, $\{a_n\}$ is Cauchy.

    Conversely, suppose that $\{a_n\}$ is a Cauchy sequence. I claim that this implies $\{a_n\}$ is bounded. First, for $\varepsilon = 1 > 0$, there exists $N \in \N$ such that if $m,n \geq N$, then $|a_m-a_n| < \varepsilon = 1$. In particular, for all $n \geq N$, $|a_N - a_n| < 1$. Take $M = \max(a_N + 1, a_{N-1},...,a_1)$, and $m = \min(a_N - 1, a_{N-1},...,a_1)$. Then for all $k \leq N$ we have that $m \leq a_k \leq M$. On the other hand, for $k \geq N$, we have that $a_N - 1 < a_k < a_N + 1$, so $m < a_k < M$. Thus $\{a_n:n \in \N\}$ is bounded. 

    Then, by \nameref{thmname:bolz} $\{a_n\}$ has a convergent subsequence $\{a_{n_k}\}$. Let $\lim\limits_{k\rightarrow \infty}a_{n_k} = l$, for some $l \in \R$. Then, fix $\varepsilon > 0$. It follows that there exist $K, K' \in \N$ such that for $m,n \geq K$ and $j \geq K'$, $|a_m - a_n| < \varepsilon/2$, while $|a_{n_j} - l| < \varepsilon/2$. Note that $n_j \geq j$, since the sequence $\{n_k\}$ is increasing. Then, for all $i \geq \max(K,K')$ we have that \begin{equation*}
        |a_i - l| \leq |a_i - a_{n_i}| + |a_{n_i} - l| < \varepsilon/2 + \varepsilon/2 = \varepsilon
    \end{equation*}
    Therefore, by definition $\{a_n\}$ converges to $l$ as well.
\end{proof*}


\section{Infinite Series}

\begin{defn}{}{}
    The sequence $\{a_n\}$ is \Emph{summable} if the sequence $\{s_n\}$ converges, where \begin{equation*}
        s_n = \sum\limits_{i=1}^na_i
    \end{equation*}
    is the $n$-th \Emph{partial sum}. In this case, $\lim_{n\rightarrow \infty}s_n$ is denoted by \begin{equation*}
        \sum\limits_{n=1}^{\infty}a_n
    \end{equation*}
    and is called the \Emph{sum} of the sequence $\{a_n\}$.
\end{defn}

\begin{rmk}{}{}
    If $\{a_n\}$ and $\{b_n\}$ are summable, then \begin{align*}
        \sum\limits_{n=1}^{\infty}(a_n+b_n) &= \sum\limits_{i=1}^{\infty}a_n + \sum\limits_{i=1}^{\infty}b_n \\
        \sum\limits_{n=1}^{\infty}c\cdot a_n &= c\cdot\sum\limits_{i=1}^{\infty}a_n
    \end{align*}
    for all $c \in \R$. 
\end{rmk}

\begin{namthm}{The Cauchy Criterion}{cauchcrit}
    The sequence $\{a_n\}$ is summable if and only if for all $\varepsilon > 0$ there exists $N \in \N$ such that for all $m \geq n \in \N$, if $m,n \geq N$, then \begin{equation*}
        \left|\sum\limits_{i=1}^ma_i - \sum\limits_{i=1}^na_i\right| = \left|\sum\limits_{i=n+1}^ma_i\right| <\varepsilon
    \end{equation*}
\end{namthm}

\begin{rmk}{}{}
    This result is a direct conseqeunce of the fact that a sequence in $\R$ is Cauchy if and only if it converges applied to the sequence of partial sums for $\{a_n\}$.
\end{rmk}


\begin{namthm}{The Vanishing Condition}{}
    If $\{a_n\}$ is summable, then \begin{equation*}
        \lim\limits_{n\rightarrow \infty}a_n = 0
    \end{equation*}
\end{namthm}
\begin{proof*}{}{}
    If $\lim\limits_{n\rightarrow \infty}s_n = l$, then \begin{align*}
        \lim\limits_{n\rightarrow \infty}a_n &= \lim\limits_{n\rightarrow \infty}(s_n - s_{n-1}) = \lim\limits_{n\rightarrow \infty}s_n - \lim\limits_{n\rightarrow \infty}s_{n-1} \\
        &= l - l = 0
    \end{align*}
\end{proof*}

\begin{eg*}{}{}
    The \Emph{geometric series} are of the form \begin{equation*}
        \sum\limits_{n=0}^{\infty}r^n = 1+r+r^2+r^3+\hdots
    \end{equation*}
    For $|r| \geq 1$, $\lim\limits_{n\rightarrow \infty}r^n \neq 0$, so $\{r^n\}$ is not summable. On the other hand, if $|r| < 1$ the sequence is summable. First write $$s_n = 1+r+r^2+...+r^n,\;\text{ and }\;rs_n = r+r^2+r^3+...+r^{n+1}$$
    It follows that $$s_n(1-r) = 1 - r^{n+1}$$ so $$s_n = \frac{1-r^{n+1}}{1-r}$$ since $r \neq 1$. Finally, it follows that \begin{align*}
        \sum\limits_{n=0}^{\infty}r^n = \lim\limits_{n\rightarrow \infty}s_n = \lim\limits_{n\rightarrow \infty}\frac{1-r^{n+1}}{1-r} = \frac{1}{1-r}
    \end{align*}
    as $|r| < 1$.
\end{eg*}

\begin{defn}{}{}
    A sequence $\{a_n\}$ such that $a_n\geq 0$ for all $n \in \N$ is said to be \Emph{nonnegative}.
\end{defn}

\begin{namthm}{The Boundedness Criterion}{boundcrit}
    A nonnegative sequence $\{a_n\}$ is summable if and only if the set of partial sums $s_n$ is bounded.
\end{namthm}
\begin{proof*}{}{}
    Since $\{a_n\}$ is nonnegative, $\{s_n\}$ is nondecreasing. Hence from our previous results on monotone sequences, $\{s_n\}$ converges if and only if $\{s_n\}$ is bounded.
\end{proof*}


\begin{namthm}{The Comparison Test}{comptest}
    Suppose that $\{a_n\}$ and $\{b_n\}$ are sequences such that $0 \leq a_n \leq b_n$ for all $n \in \N$. Then if $\sum\limits_{n=1}^{\infty}b_n$ converges, so does $\sum\limits_{n=1}^{\infty}a_n$.
\end{namthm}
\begin{proof*}{}{}
    Let $s_n$ denote the $n$-th partial sum of $\{a_n\}$, and let $t_n$ denote the $n$-th partial sum of $\{b_n\}$. Then $0 \leq s_n \leq t_n$ for all $n \in \N$. Now $\{t_n\}$ converges by assumption, so it is bounded. Hence, there exists $M \in \R$ such that $0 \leq s_n \leq t_n \leq M$ for all $n\in \N$, so $\{s_n\}$ is also bounded. Thus by \nameref{thmname:boundcrit} $\{a_n\}$ is summable, so by definition $\sum\limits_{n=1}^{\infty}a_n$ converges.
\end{proof*}


\begin{namthm}{The Limit Comparison Test}{limcomptest}
    If $a_n, b_n > 0$ for convergent sequences $\{a_n\}$ and $\{b_n\}$, and $\lim\limits_{n\rightarrow \infty}a_n/b_n = c \neq 0$, then $\sum\limits_{n=1}^{\infty}a_n$ converges if and only if $\sum\limits_{n=1}^{\infty}b_n$ converges.
\end{namthm}
\begin{proof*}{}{}
    Suppose $\sum\limits_{n=1}^{\infty}b_n$ converges. Since $\lim\limits_{n\rightarrow \infty}a_n/b_n = c$, there is some $N$ such that \begin{equation*}
        a_n/b_n - c \leq c \implies a_n \leq 2cb_n,\;\;\text{ for } n\geq N
    \end{equation*}
    But the sequence $2c\sum\limits_{n=N}^{\infty}b_n$ certainly converges. Then by \nameref{thmname:comptest} we have that $\sum\limits_{n=N}^{\infty}a_n$ converges, and this implies convergence of the whole series $\sum\limits_{n=1}^{\infty}a_n$, which only has finitely many additional terms.
    
    Note that \begin{equation*}
        \lim\limits_{n\rightarrow \infty}b_n/a_n = \frac{1}{\lim\limits_{n\rightarrow\infty}a_n/b_n} = 1/c \neq 0
    \end{equation*}
    so the converse follows immediately.
\end{proof*}


\begin{namthm}{The Ratio Test}{ratio}
    Let $\{a_n\}$ be a positive sequence, and suppose that \begin{equation*}
        \lim\limits_{n\rightarrow \infty}\frac{a_{n+1}}{a_n} = r
    \end{equation*}
    for some $r \geq 0$. Then $\sum\limits_{n=1}^{\infty}a_n$ converges if $r < 1$. On the other hand, if $r > 1$, then the terms $a_n$ are unbounded, so $\sum\limits_{n=1}^{\infty}a_n$ diverges.
\end{namthm}
\begin{proof*}{}{}
    Suppose first that $r<1$. Choose any number $s$ with $r < s < 1$. The hypothesis $\lim\limits_{n\rightarrow \infty}a_{n+1}/a_n = r < 1$ implies that there is some $N \in \N$ such that if $n \geq N$, \begin{equation*}
        a_{n+1}/a_n - r < s-r \implies a_{n+1}/a_n < s
    \end{equation*}
    This can be written as $a_{n+1} < sa_n$. Thus, \begin{align*}
        a_{N+1} &< sa_N \\
        a_{N+2} &< sa_{N+1} < s^2a_N \\
        &\vdots \\
        a_{N+k} &< s^ka_N
    \end{align*}
    Since $\sum\limits_{k=0}^{\infty}a_Ns^k = a_N\sum\limits_{k=0}^{\infty}s^k$ converges, since $|s| < 1$, \nameref{thmname:comptest} shows that \begin{equation*}
        \sum\limits_{n=N}^{\infty}a_n = \sum\limits_{k=0}^{\infty}a_{N+k}
    \end{equation*}
    converges as $a_{N+k} < a_Ns^k$ for all $k \geq 0$. This implies that $\sum\limits_{n=0}^{\infty}a_n$ as a whole converges.

    If $r > 1$, choose some $s \in \R$ such that $1 < s < r$. Then there is a number $N \in \N$ such that \begin{equation*}
        r - a_{n+1}/a_n < r-s \implies s < a_{n+1}/a_n
    \end{equation*}
    for all $n \geq N$. This implies that \begin{equation*}
        a_{N+k} > a_Ns^k,
    \end{equation*}
    for all $k \in \N$, so the terms are unbounded.
\end{proof*}


\begin{namthm}{The Integral Test}{inttest}
    Suppose that $f$ is positive and decreasing on $[1,\infty)$, and that $f(n) = a_n$ for all $n \in \N$. Then $\sum\limits_{n=1}^{\infty}a_n$ converges if and only if the limit \begin{equation*}
        \int_1^{\infty}f = \lim\limits_{A\rightarrow \infty}\int_1^Af
    \end{equation*}
    exists.
\end{namthm}
\begin{proof*}{}{}
    The existence of $\lim\limits_{A\rightarrow \infty}\int_1^Af$ is equivalent to the convergence of the series \begin{equation*}
        \int_1^2f + \int_2^3f + \int_3^4f + ...
    \end{equation*}
    Since $f$ is decreasing, we have \begin{equation*}
        f(n+1) < \int_n^{n+1}f < f(n) 
    \end{equation*}
    The first half of this double inequality shows that the series $\sum\limits_{n=1}^{\infty}a_{n+1}$ may be compared to the series $\sum\limits_{n=1}^{\infty}\int_n^{n+1}$, proving that $\sum\limits_{n=1}^{\infty}a_{n+1}$ (and hence $\sum\limits_{n=1}^{\infty}a_n$) converges if $\lim\limits_{A\rightarrow \infty}\int_1^Af$ exists.

    The second half of the inequality shows that the series $\sum\limits_{n=1}^{\infty}\int_n^{n+1}f$ may be compared to the series $\sum\limits_{n=1}^{\infty}a_n$, proving that $\lim\limits_{A\rightarrow \infty}\int_1^Af$ must exist if $\sum\limits_{n=1}^{\infty}a_n$ converges.
\end{proof*}


\begin{cor}{}{}
    The series $\sum\limits_{n=1}^{\infty}1/n^p$ converges if and only if $p > 1$.
\end{cor}
\begin{proof*}{}{}
    If $p < 0$ then $\lim\limits_{n\rightarrow \infty}1/n^p \neq 0$. If $p > 0$, the convergence of $\sum\limits_{n=1}^{\infty}1/n^p$ is equivalent, by \nameref{thmname:inttest}, to the existence of \begin{equation*}
        \lim\limits_{A\rightarrow\infty}\int_1^A\frac{1}{x^p}dx
    \end{equation*}
    Now, observe that \begin{equation*}
        \int_1^A\frac{1}{x^p}dx = \left\{\begin{array}{lc} -\frac{1}{p-1}\cdot\frac{1}{A^{p-1}} + \frac{1}{p-1}, & p \neq 1 \\ \log A, & p = 1\end{array}\right.
    \end{equation*}
    This shows that $\lim\limits_{A\rightarrow \infty}\int_1^A1/x^pdx$ exists if $p > 1$, but not if $p \leq 1$. Thus, $\sum\limits_{n=1}^{\infty}1/n^p$ converges precisely for $p > 1$. 
\end{proof*}

\begin{defn}{}{}
    The series $\sum\limits_{n=1}^{\infty}a_n$ is \Emph{absolutely convergent} if the series $\sum\limits_{n=1}^{\infty}|a_n|$ is convergent. (In formal language, the sequence $\{a_n\}$ is said to be \Emph{absolutely summable} if the sequence $\{|a_n\}$ is summable.)
\end{defn}

\begin{thm}{}{absconv}
    Every absolutely convergent series is convergent. Moreover, a series is absolutely convergent if and only if the series formed from its positive terms and the series formed from its negative terms both converge.
\end{thm}
\begin{proof*}{}{}
    If $\sum\limits_{n=1}^{\infty}|a_n|$ converges, then, by \nameref{thmname:cauchcrit}, \begin{equation*}
        \lim\limits_{m,n\rightarrow \infty}\sum\limits_{i=n+1}^m|a_i| = 0
    \end{equation*}
    Since \begin{equation*}
        \left|\sum\limits_{i=n+1}^ma_i\right| \leq \sum\limits_{i=n+1}^m|a_i|
    \end{equation*}
    it follows that  \begin{equation*}
        \lim\limits_{m,n\rightarrow \infty}\sum\limits_{i=n+1}^ma_i = 0
    \end{equation*}
    which shows that $\sum\limits_{n=1}^{\infty}a_n$ converges.

    To prove the second portion of the theorem, let $\{a_n^+\}$ and $\{a_n^-\}$ be sequences defined by \begin{align*}
        a_n^+ &= \left\{\begin{array}{lc} a_n, & \text{if } a_n\geq 0 \\ 0, & \text{if } a_n \leq 0 \end{array}\right. \\
            a_n^- &= \left\{\begin{array}{lc} a_n, & \text{if } a_n\leq 0 \\ 0, & \text{if } a_n \geq 0 \end{array}\right.
    \end{align*}
    It follows that \begin{equation*}
        \sum\limits_{n=1}^{\infty}|a_n| = \sum\limits_{n=1}^{\infty}[a_n^+-a_n^-] = \sum\limits_{n=1}^{\infty}a_n^+ - \sum\limits_{n=1}^{\infty}a_n^-
    \end{equation*}
    so if $\sum\limits_{n=1}^{\infty}a_n^+$ and $\sum\limits_{n=1}^{\infty}a_n^-$ are convergent, so is $\sum\limits_{n=1}^{\infty}|a_n|$.


    Conversely, suppose $\sum\limits_{n=1}^{\infty}|a_n|$ converges. Then by our initial argument $\sum\limits_{n=1}^{\infty}a_n$ converges also. Therefore, \begin{equation*}
        \sum\limits_{n=1}^{\infty}a_n^+ = \sum\limits_{n=1}^{\infty}\frac{1}{2}[a_n + |a_n|] = \frac{1}{2}\left(\sum\limits_{n=1}^{\infty}a_n + \sum\limits_{n=1}^{\infty}|a_n|\right)
    \end{equation*}
    and  \begin{equation*}
        \sum\limits_{n=1}^{\infty}a_n^- = \sum\limits_{n=1}^{\infty}\frac{1}{2}[a_n - |a_n|] = \frac{1}{2}\left(\sum\limits_{n=1}^{\infty}a_n - \sum\limits_{n=1}^{\infty}|a_n|\right)
    \end{equation*}
    both converge.
\end{proof*}



\begin{rmk}{}{}
    A consequent of this result is that every convergent series with positive terms can be used to obtain infinitely many other convergent series simply by putting minus sings at random.
\end{rmk}

\begin{defn}{}{}
    A convergent series which is not absolutely convergent is said to be \Emph{conditionally convergent}.
\end{defn}


\begin{namthm}{Leibniz's Theorem}{alttest}
    Suppose that $\{a_n\}$ is a non-decreasing non-negative sequence such that \begin{equation*}
        \lim\limits_{n\rightarrow \infty} a_n = 0
    \end{equation*}
    Then the series \begin{equation*}
        \sum\limits_{n=1}^{\infty}(-1)^{n+1}a_n = a_1-a_2+a_3-a_4+...
    \end{equation*}
    converges.
\end{namthm}
\begin{proof*}{}{}
    First, observe that \begin{enumerate}
        \item $s_2 \leq s_4 \leq s_6 \leq ...$
        \item $s_1 \geq s_3 \geq s_5 \geq ...$ 
        \item $s_k\leq s_l$ if $k$ is even and $l$ is odd.
    \end{enumerate}
    Indeed, note that for all $n$, $a_{2n+1} \geq a_{2n+2}$, so $a_{2n+1} - a_{2n+2} \geq 0$, so \begin{equation*}
        s_{2n+2} = s_{2n} + a_{2n+1} - a_{2n+2} \geq s_{2n}
    \end{equation*}
    and similarly as $a_{2n+2} \geq a_{2n+3}$, we have \begin{equation*}
        s_{2n+3} = s_{2n+1} -a_{2n+2}+a_{2n+3} \leq s_{2n+1}
    \end{equation*}
    To prove the third inequality, first notice that \begin{equation*}
        s_{2n} = s_{2n-1} - a_{2n} \leq s_{2n-1}
    \end{equation*}
    since $a_{2n} \geq 0$. Now, if $k$ is even and $l$ is odd, choose $n$ such that $k\leq 2n$ and $l \leq 2n-1$. Then \begin{equation*}
        s_k \leq s_{2n} \leq s_{2n-1} \leq s_l
    \end{equation*}
    which proves the third inequality.

    Now, the sequence $\{s_{2n}\}$ converges, because it is nondecreasing and is bounded above (by $s_l$ for any odd $l$). Let $\alpha = \sup\{s_{2n}\} = \lim\limits_{n\rightarrow\infty} s_{2n}$. Similarly, let $\beta = \inf\{s_{2n+1}\} = \lim\limits_{n\rightarrow \infty}s_{2n+1}$. It follows from our third inequality that $\alpha \leq \beta$; since \begin{equation*}
        s_{2n+1}-s_{2n} = a_{2n+1}\;\;\text{ and }\;\;\lim\limits_{n\rightarrow \infty}a_n = 0
    \end{equation*}
    it is actually the case that $\alpha = \beta$. This proves that $\alpha = \beta = \lim\limits_{n\rightarrow \infty} s_n$.
\end{proof*}


\begin{defn}{}{}
    A sequence $\{a_n\}$ is a \Emph{rearrangement} of a sequence $\{a_n\}$ if each $b_n = a_{f(n)}$ where $f$ is a certain permutation on the natural numbers.
\end{defn}



\begin{thm}{}{}
    If $\sum\limits_{n=1}^{\infty}a_n$ converges, but does not converge absolutely; then for any number $\alpha$ there is a rearrangement $\{b_n\}$ of $\{a_n\}$ such that $\sum\limits_{n=1}^{\infty}b_n = \alpha$.
\end{thm}
\begin{proof*}{}{}
    Let $\sum\limits_{n=1}^{\infty}p_n$ denote the series formed from the positive terms of $\{a_n\}$ and let $\sum\limits_{n=1}^{\infty}q_n$ denote the series of negative terms. It follows from Theorem \ref{thm:absconv} that at least one of these series does not converge. As a matter of fact, both must fail to converge, for if one had bounded partial sums, and the other had unbounded partial sums, then the original series $\sum\limits_{n=1}^{\infty}a_n$ would also have unbounded partial sums, contradicting the assumption that it converges.

    Let $\alpha$ be any number. Assume, for simplicity, that $\alpha > 0$. Since the series $\sum\limits_{n=1}^{\infty}p_n$ there is a number $N$ such that \begin{equation*}
        \sum\limits_{n=1}^Np_n > \alpha
    \end{equation*}
    We will choose $N_1$ to be the smallest $N$ with this property. This means that \begin{equation*}
        \sum\limits_{n=1}^{N_1-1}p_n \leq \alpha\;\;\text{ and }\;\;\sum\limits_{n=1}^{N_1}p_n > \alpha
    \end{equation*}
    Then if $S_1 = \sum\limits_{n=1}^{N_1}p_n$, we have $S_1 - \alpha \leq p_{N_1}$. Next, choose the smallest integer $M_1$ for which \begin{equation*}
        T_1 = S_1 + \sum\limits_{n=1}^{M_1}q_n < \alpha
    \end{equation*}
    As before, we have $\alpha - T_1 \leq -q_{M_1}$. We continue this process indefinitely. The sequence \begin{equation*}
        p_1,...,p_{N_1},q_1,...,q_{M_1},q_{N_1+1},...,p_{N_2},...
    \end{equation*}
    is a rearrangement of $\{a_n\}$. The partial sums of this rearrangement increase to $S_1$, then decrease to $T_1$, then increase to $S_2$, etc. To complete the proof we note that $|S_k -\alpha|$ and $|T_k - \alpha|$ are less that or equal to $p_{N_k}$ or $-q_{M_k}$, respectively, and that these terms, being numbers of the original sequence $\{a_n\}$, must decrease to $0$, since $\sum\limits_{n=1}^{\infty}a_n$ converges.
\end{proof*}



\begin{thm}{}{}
    If $\sum\limits_{n=1}^{\infty}a_n$ converges absolutely, and $\{b_n\}$ is any rearrangement of $\{a_n\}$, then $\sum\limits_{n=1}^{\infty}b_n$ also converges (absolutely), and in particular \begin{equation*}
        \sum\limits_{n=1}^{\infty}a_n = \sum\limits_{n=1}^{\infty}b_n
    \end{equation*}
\end{thm}
\begin{proof*}{}{}
    Denote the partial sums of $\{a_n\}$ by $s_n$, and the partial sums of $\{b_n\}$ by $t_n$. Suppose that $\varepsilon > 0$. Since $\sum\limits_{n=1}^{\infty}a_n$ converges, there is some $N$ such that \begin{equation*}
        \left|\sum\limits_{n=1}^{\infty}a_n - s_N\right| < \varepsilon
    \end{equation*}
    Moreover, since $\sum\limits_{n=1}^{\infty}|a_n|$ converges, we can also choose $N'$ so that \begin{equation*}
        \sum\limits_{n=1}^{\infty}|a_n| - (|a_1|+\hdots + |a_{N'}|) < \varepsilon
    \end{equation*}
    so that \begin{equation*}
        \sum\limits_{n=N+1}^{\infty}|a_n| < \varepsilon
    \end{equation*}
    Choose $M$ such that each $a_1,...,a_N$ appear among $b_1,...,b_M$. Then whenever $m > M$, the difference $t_m - s)N$ is the sum of certain $a_i$, where $i > N$. Consequently, \begin{equation*}
        |t_m-s_N| \leq \sum\limits_{n=N+1}^{\infty}|a_n| < \varepsilon
    \end{equation*}
    Thus, if $m > $, then \begin{align*}
        \left|\sum\limits_{n=1}^{\infty}a_n - t_m \right|&= \left|\sum\limits_{n=1}^{\infty}a_n - s_N - (t_m-s_N)\right| \\
        &\leq \left|\sum\limits_{n=1}^{\infty}a_n - s_N\right| +|(t_m-s_N)| \\
        &< \varepsilon + \varepsilon
    \end{align*}
    Since this is true for every $\varepsilon > 0$, the series $\sum\limits_{n=1}^{\infty}b_n$ converges to $\sum\limits_{n=1}^{\infty}a_n$.

    To show that $\sum\limits_{n=1}^{\infty}b_n$ converges absolutely, note that $\{|b_n|\}$ is a rearrangement of $\{|a_n|\}$; since $\sum\limits_{n=1}^{\infty}|a_n|$ converges absolutely, $\sum\limits_{n=1}^{\infty}|b_n|$ converges by the first part of the theorem.
\end{proof*}


\begin{thm}{}{}
    If $\sum\limits_{n=1}^{\infty}a_n$ and $\sum\limits_{n=1}^{\infty}b_n$ converge absolutely, and $\{c_n\}$ is any sequence containing the products $a_ib_j$ for each pair $(i,j) \in \N\times \N$, then \begin{equation*}
        \sum\limits_{n=1}^{\infty}c_n = \sum\limits_{n=1}^{\infty}a_n\cdot\sum\limits_{n=1}^{\infty}b_n
    \end{equation*}
\end{thm}
\begin{proof*}{}{}
    Notice first that the sequence \begin{equation*}
        p_L = \sum\limits_{i=1}^L|a_i|\cdot \sum\limits_{j=1}^L|b_j|
    \end{equation*}
    converges since $\{a_n\}$ and $\{b_n\}$ are absolutely convergent, and since the limit of a product is the product of the limits. So $\{p_L\}$ is a Cauchy sequence, which means that for any $\varepsilon > 0$, if there exists $N$ such that for all $L,L' \geq N$, \begin{equation*}
        \left|\sum\limits_{i=1}^{L'}|a_i|\cdot \sum\limits_{j=1}^{L'}|b_j| - \sum\limits_{i=1}^L|a_i|\cdot \sum\limits_{j=1}^L|b_j|\right| < \varepsilon/2
    \end{equation*}
    It follows that \begin{equation*}
        \sum\limits_{i\;or\;j > L} |a_i|\cdot|b_i| \leq \varepsilon/2 < \varepsilon \tag{1}
    \end{equation*}
    Now suppose that $N$ is such that the terms $c_n$ for $n \leq N$ include all $a_ib_j$ for $i,j \leq L$. Then the difference \begin{equation*}
        \sum\limits_{n=1}^Nc_n - \sum\limits_{i=1}^La_i\cdot \sum\limits_{j=1}^Lb_j
    \end{equation*}
    consists of terms $a_ib_j$ with $i > L$ of $j > L$, so \begin{equation*}
        \left|\sum\limits_{n=1}^Nc_n - \sum\limits_{i=1}^La_i\cdot \sum\limits_{j=1}^Lb_j\right| \leq \sum\limits_{i\;or\;j>L}|a_i|\cdot|b_j| < \varepsilon \tag{2}
    \end{equation*}
    But since the limit of a product is the product of the limits we also have \begin{equation*}
        \left|\sum\limits_{i=1}^{\infty}a_i\cdot \sum\limits_{j=1}^{\infty}b_j - \sum\limits_{i=1}^La_i\cdot \sum\limits_{j=1}^Lb_j\right| < \varepsilon \tag{3}
    \end{equation*}
    for sufficiently large $L$. Consequently, if we choose $L$ and then $N$ large enough, we will have \begin{align*}
        \left|\sum\limits_{i=1}^{\infty}a_i\cdot \sum\limits_{j=1}^{\infty}b_j - \sum\limits_{i=1}^Nc_i\right| \leq \left|\sum\limits_{i=1}^{\infty}a_i\cdot \sum\limits_{j=1}^{\infty}b_j - \sum\limits_{i=1}^La_i\cdot \sum\limits_{j=1}^Lb_j\right| \\
        &+ \left|\sum\limits_{n=1}^Nc_n - \sum\limits_{i=1}^La_i\cdot \sum\limits_{j=1}^Lb_j\right| \\
        &< 2\varepsilon
    \end{align*}
    which proves the theorem.
\end{proof*}




\section{Uniform Convergence and Power Series}


\begin{rmk}{}{}
    We are now interested in the study of series of functions, or in other words functions of the form \begin{equation*}
        f(x) = f_1(x) + f_2(x) + f_3(x) + \hdots
    \end{equation*}
    In such a situation $\{f_n\}$ will be some sequence of functions; for each $x$ we obtain a sequence of numbers $\{f_n(x)\}$, and $f(x)$ is the sum of this sequence. Recall that each sum $f_1(x)+f_2(x)+f_3(x) + \hdots$ is, by definition, the limit of the sequence $f_1(x),f_1(x)+f_2(x),f_1(x)+f_2(x)+f_3(x),...$. If we define a new sequence of functions $\{s_n\}$ by \begin{equation*}
        s_n = f_1 + \hdots + f_n
    \end{equation*}
    then we can express this fact more succinctly by writing \begin{equation*}
        f(x) = \lim\limits_{n\rightarrow \infty}s_n(x)
    \end{equation*}
    for some $x \in \R$.
\end{rmk}


\begin{rmk}{}{}
    First let us consider functions of the form \begin{equation*}
        f(x) = \lim\limits_{n\rightarrow\infty}f_n(x)
    \end{equation*}
    All this form may seem simple, it is very important to note that \Emph{nothing one would hope to be true actually is}. Instead we have a flurry of lovely counter-examples.
\end{rmk}

\begin{eg*}{Counter-Example 1}{}
    Even if each $f_n$ is continuous, the function $f$ may not be! Indeed, consider the sequence of functions \begin{equation*}
        f_n(x) = \left\{\begin{array}{lc} x^n, & 0\leq x \leq 1 \\ 1, & x \geq 1 \end{array}\right.
    \end{equation*}
    These functions are all continuous, but the function $f(x) = \lim\limits_{n\rightarrow \infty}f_n(x)$ is not continuous; in fact, \begin{equation*}
        \lim\limits_{n\rightarrow \infty}f_n(x) = \left\{\begin{array}{lc} 0, & 0\leq x < 1 \\ 1, & x \geq 1 \end{array}\right.
    \end{equation*}
    Another example of this phenomenon is illustrated by the family of functions \begin{equation*}
        f_n(x) = \left\{\begin{array}{lc} -1, &  x \leq -\frac{1}{n} \\ nx, & -\frac{1}{n} \leq x \leq \frac{1}{n} \\ 1, & \frac{1}{n} \leq x \end{array}\right.
    \end{equation*}
    In this case, if $x < 0$ $f_n(x)$ is eventually $-1$, and if $x > 0$, then $f_n(x)$ is eventually $1$, while $f_n(0) = 0$ for all $n$. Thus, \begin{equation*}
        \lim\limits_{n\rightarrow \infty}f_n(x) = \left\{\begin{array}{lc} -1, & x < 0 \\ 0, & x = 0 \\ 1, & x > 0 \end{array}\right.
    \end{equation*}
    so once again $f(x) = \lim\limits_{n\rightarrow \infty}f_n(x)$ is not continuous.
\end{eg*}

\begin{eg*}{Counter-Example 2}{}
    It is even possible to produce a sequence of differentiable functions $\{f_n\}$ for which the function $f(x) = \lim\limits_{n\rightarrow \infty}f_n(x)$ is not continuous. One such sequence is \begin{equation*}
        f_n(x) = \left\{\begin{array}{lc} -1, &  x \leq -\frac{1}{n} \\ \sin\left(\frac{n\pi x}{2}\right), & -\frac{1}{n} \leq x \leq \frac{1}{n} \\ 1, & \frac{1}{n} \leq x \end{array}\right.
    \end{equation*}
    These functions are differentiable, but we still have  \begin{equation*}
        \lim\limits_{n\rightarrow \infty}f_n(x) = \left\{\begin{array}{lc} -1, & x < 0 \\ 0, & x = 0 \\ 1, & x > 0 \end{array}\right.
    \end{equation*}
\end{eg*}

\begin{defn}{}{}
    If $f$ is a function defined on some set $A$, and a sequence of functions $\{f_n\}$, all defined on the same set $A$, are such that only \begin{equation*}
        f(x) = \lim\limits_{n\rightarrow \infty}f_n(x)
    \end{equation*}
    for all $x \in A$. Precisely, $\{f_n\}$ is said to \Emph{converge pointwise to $f$ on $A$} if for all $\varepsilon > 0$, and for all $x \in A$, there is some $N$ such that if $n \geq N$, then $|f(x) - f_n(x)| < \varepsilon$.
\end{defn}


\begin{defn}{}{}
    Let $\{f_n\}$ be a sequence of functions defined on $A$, and let $f$ be a function which is also defined on $A$. Then $f$ is called the \Emph{uniform limit of $\{f_n\}$ on $A$} if for every $\varepsilon > 0$ there is some $N$ such that for all $x \in A$, \begin{equation*}
        \text{if } n> N, \text{ then } |f(x) - f_n(x)| < \varepsilon
    \end{equation*}
    We also say that $\{f_n\}$ \Emph{converges uniformly to $f$ on $A$}, or that $f_n$ \Emph{approaches $f$ uniformly on $A$}.
\end{defn}



\begin{rmk}{}{}
    Note that uniform convergence implies pointwise convergence, but the converse is not true.
\end{rmk}

\begin{thm}{}{}
    Suppose that $\{f_n\}$ is a sequence of functions which are integrable on $[a,b]$, and that $\{f_n\}$ converges uniformly on $[a,b]$ to a function $f$ which is also integrable on $[a,b]$. Then \begin{equation*}
        \int_a^bf = \lim\limits_{n\rightarrow \infty}\int_a^bf_n
    \end{equation*}
\end{thm}
\begin{proof*}{}{}
    Let $\varepsilon > 0$. Then since $\{f_n\}$ converges uniformly to $f$, there exists $N \in \N$ such that for all $x \in [a,b]$, if $n \geq N$ \begin{equation*}
        |f(x) f_n(x)| < \varepsilon
    \end{equation*}
    Then for all $n \geq N$, we have that \begin{align*}
        \left|\int_a^bf(x)dx - \int_a^bf_n(x)dx\right| &= \left|\int_a^b[f(x) - f_n(x)]dx\right| \\
        &\leq \int_a^b |f(x) - f_n(x)| dx \\
        &\leq \int_a^b\varepsilon dx \\
        &= \varepsilon (b-a)
    \end{align*}
    Since this is true for all $\varepsilon > 0$, it follows that \begin{equation*}
        \int_a^bf = \lim\limits_{n\rightarrow \infty}\int_a^bf_n
    \end{equation*}
\end{proof*}



\begin{thm}{}{}
    Suppose that $\{f_n\}$ is a sequence of functions which are continuous on $[a,b]$, and that $\{f_n\}$ converges uniformly on $[a,b]$ to $f$. Then $f$ is also continuous on $[a,b]$.
\end{thm}
\begin{proof*}{}{}
    For each $x \in [a,b]$ we must prove that $f$ is continuous at $x$. We first deal with $x \in (a,b)$. Fix $\varepsilon > 0$. Since $\{f_n\}$ converges uniformly to $f$ on $[a,b]$, there is some $N$ such that for all $n \geq N$ and all $y \in [a,b]$, \begin{equation*}
        |f(y) - f_n(y)| < \varepsilon/3
    \end{equation*}
    In particular, for all $h$ such that $x+ h \in [a,b]$, we have \begin{align*}
        |f(x) - f_n(x)| &< \varepsilon/3, \\
        |f(x+h) - f_n(x+h)| &< \varepsilon/3
    \end{align*}
    Now $f_n$ is continuous, so there is some $\delta > 0$ such that for $|h| < \delta$ we have \begin{equation*}
        |f_n(x) - f_n(x+h)| < \varepsilon/3
    \end{equation*}
    Thus, if $|h| < \delta$, then \begin{align*}
        |f(x+h)-f(x)| &= |f(x+h)-f_n(x+h)+f_n(x+h) - f_n(x)+f_n(x)-f(x)| \\
        &\leq |f(x+h) - f_n(x+h)| + |f_n(x+h)-f_n(x)| + |f_n(x) - f(x)| \\
        &<\varepsilon/3 + \varepsilon/3 + \varepsilon/3 \\
        &= \varepsilon
    \end{align*}
    This proves that $f$ is continuous at $x$.
\end{proof*}

\begin{rmk}{}{}
    Allow this last two theorems are great successes, differentiability sadly fails. Even if each $f_n$ is differentiable and $\{f_n\}$ converges uniformly to $f$, it need not be the case that $f$ is differentiable. Moreover, even if $f$ is itself differentiable, it need not be the case that \begin{equation*}
        f'(x) = \lim\limits_{n\rightarrow \infty}f_n'(x)
    \end{equation*}
\end{rmk}

\begin{eg*}{Counter Example 3}{}
    Consider the family of functions \begin{equation*}
        f_n(x) = \frac{1}{n}\sin(n^2x)
    \end{equation*}
    then $\{f_n\}$ converges uniformly to the function $f(x) = 0$, but \begin{equation*}
        f_n'(x) = n\cos(n^2x)
    \end{equation*}
    and $\lim\limits_{n\rightarrow\infty}n\cos(n^2x)$ does not even always exist (for example if $x =0$).
\end{eg*}


\begin{thm}{}{}
    Suppose that $\{f_n\}$ is a sequence of functions which are differentiable on $[a,b]$, with integrable derivatives $f'_n$, and that $\{f_n\}$ converges (pointwise) to $f$. Suppose, moreover, that $\{f_n'\}$ converges uniformly on $[a,b]$ to some continuous function $g$. Then $f$ is differentiable and \begin{equation*}
        f'(x) = \lim\limits_{n\rightarrow \infty}f_n'(x)
    \end{equation*}
\end{thm}
\begin{proof*}{}{}
    Applying Theorem $1$ to the interval $[a,x]$, we see that for each $x$ we have \begin{align*}
        \int_a^xg &= \lim\limits_{n\rightarrow \infty}\int_a^xf'_n \\
        &= \lim\limits_{n\rightarrow \infty}[f_n(x) - f_n(a)] \tag{by \nameref{thmname:FTC2}} \\
        &= f(x) - f(a)
    \end{align*}
    Since $g$ is continuous, it follows that $f'(x) = g(x) = \lim\limits_{n\rightarrow \infty}f_n'(x)$ for all $x$ in the interval $[a,b]$, by \nameref{thmname:FTC}.
\end{proof*}

\begin{defn}{}{}
    The series $\sum\limits_{n=1}^{\infty}f_n$ \Emph{converges uniformly} (more formally, the sequence $\{f_n\}$ is \Emph{uniformly summable}) \Emph{to $f$ on $A$}, if the sequence \begin{equation*}
        f_1, f_1+f_2,f_1+f_2+f_3,...
    \end{equation*}
    converges uniformly to $f$ on $A$.
\end{defn}



\begin{cor}{}{}
    Let $\sum\limits_{n=1}^{\infty}f_n$ converge uniformly to $f$ on $[a,b]$. \begin{enumerate}
        \item If each $f_n$ is continuous on $[a,b]$, then $f$ is continuous on $[a,b]$.
        \item If $f$ and each $f_n$ is integrable on $[a,b]$, then \begin{equation*}
                \int_a^bf = \sum\limits_{n=1}^{\infty}\int_a^bf_n
        \end{equation*}
    \end{enumerate}
    Moreover, if $\sum\limits_{n=1}^{\infty}f_n$ converges (pointwise) to $f$ on $[a,b]$, each $f_n$ has an integrable derivative $f_n'$ and $\sum\limits_{n=1}^{\infty}f'_n$ converges uniformly on $[a,b]$ to some continuous function, then \begin{enumerate}
        \item[3.] $f'(x) = \sum\limits_{n=1}^{\infty}f_n'(x)$   for all $x \in [a,b]$
    \end{enumerate}
\end{cor}
\begin{proof*}{}{}
    Let $\{s_n\}$ be the sequence of partial sums of the $\{f_n\}$. Then since each $f_n$ is continuous, so is each $s_n$. Then as $\{s_n\}$ converges uniformly to $f$ we have by a previous theorem that $f$ is also continuous on $[a,b]$. Next, since each $f_n$ is integrable on $[a,b]$, so is each $s_n$. Then as $\{s_n\}$ converges uniformly to $f$ we have that \begin{align*}
        \int_a^bf &= \lim\limits_{n\rightarrow \infty}\int_a^bs_n \\
        &= \lim\limits_{n\rightarrow\infty}t_n \\
        &= \sum\limits_{n=1}^{\infty}\int_a^bf_n
    \end{align*}
    where $\{t_n\}$ is the sequence such that \begin{equation*}
        t_n = \sum\limits_{i=1}^n\int_a^bf_n = \int_a^bs_n
    \end{equation*}

    Finally, suppose $s_n$ converges (pointwise) to $f$ on $[a,b]$, and each $f_n$ has an integrable derivative $f_n'$. Then each $s_n$ has an integrable derivative $s_n'$ on $[a,b]$ by the linearity of the derivative and integral operators. Moreover, suppose $s_n'$ converges uniformly on $[a,b]$ to some continuous function $g$. Then it follows that for all $x \in [a,b]$ \begin{equation*}
        f'(x) = \lim\limits_{n\rightarrow \infty}s_n'(x) = \sum\limits_{n=1}^{\infty}f_n'(x)
    \end{equation*}
\end{proof*}


\begin{namthm}{The Weierstrass M-Test}{mtest}
    Let $\{f_n\}$ be a sequence of functions defined on $A$, and suppose that $\{M_n\}$ is a sequence of numbers such that \begin{equation*}
        |f_n(x)|\leq M_n,\forall x \in A
    \end{equation*}
    Suppose moreover that $\sum\limits_{n=1}^{\infty}M_n$ converges. Then for each $x$ in $A$ the series $\sum\limits_{n=1}^{\infty}f_n(x)$ converges absolutely, and $\sum\limits_{n=1}^{\infty}f_n$ converges uniformly on $A$ to the function \begin{equation*}
        f(x) = \sum\limits_{n=1}^{\infty}f_n(x)
    \end{equation*}
\end{namthm}
\begin{proof*}{}{}
    For each $x \in A$, the series $\sum\limits_{n=1}^{\infty}|f_n(x)|$ converges by \nameref{thmname:comptest}; consequently $\sum\limits_{n=1}^{\infty}f_n(x)$ converges absolutely. Moreover, for all $x \in A$ we have \begin{align*}
        \left|f(x) - \sum\limits_{i=1}^{N}f(x)\right| &= \left|\sum\limits_{n=N+1}^{\infty}f_n(x)\right| \\
        &\leq \sum\limits_{n=N+1}^{\infty}|f_n(x)| \\
        &\leq \sum\limits_{n=N+1}^{\infty}M_n
    \end{align*}
    Since $\sum\limits_{n=1}^{\infty}M_n$ converges, the number $\sum\limits_{n=N+1}^{\infty}M_n$ can be made as small as desired (by \nameref{thmname:cauchcrit}), by choosing $N$ sufficiently large.
\end{proof*}




\begin{defn}{}{}
    An infinite sum of functions of the form \begin{equation*}
        f(x) = \sum\limits_{n=0}^{\infty}a_n(x-a)^n
    \end{equation*}
    is called a \Emph{power series centered at $a$}. One especially important family of power series are those of the form \begin{equation*}
        \sum\limits_{n=0}^{\infty}\frac{f^{(n)}(a)}{n!}(x-a)^n
    \end{equation*}
    where $f$ is some infinitely differentiable function at $a$; this series is called the \Emph{Taylor series for $f$ at $a$}.
\end{defn}


\begin{rmk}{}{}
    Given a function $f$ infinitely differentiable at $a$, we have for $x \in \R$ that \begin{equation*}
        f(x) = \sum\limits_{n=0}^{\infty}\frac{f^{(n)}(a)}{n!}(x-a)^n
    \end{equation*}
    if and only if the remainder terms satisfy $\lim\limits_{n\rightarrow \infty}R_{n,a}(x) = 0$.
\end{rmk}


\begin{thm}{}{}
    Suppose that the series \begin{equation*}
        f(x_0) = \sum\limits_{n=0}^{\infty}a_nx_0^n
    \end{equation*}
    converges, and let $a$ be any number with $0 < a < |x_0|$. Then on $[-a,a]$ the series \begin{equation*}
        f(x) = \sum\limits_{n=0}^{\infty}a_nx^n
    \end{equation*}
    converges uniformly (and absolutely). Moreover, the same is true for the series \begin{equation*}
        g(x) = \sum\limits_{n=1}^{\infty}na_nx^{n-1}
    \end{equation*}
    Finally, $f$ is differentiable and \begin{equation*}
        f'(x) = \sum\limits_{n=1}^{\infty}na_nx^{n-1}
    \end{equation*}
    for all $x$ with $|x| < |x_0|$.
\end{thm}
\begin{proof*}{}{}
    First, since $\sum\limits_{n=0}^{\infty}a_nx_0^n$ converges, $\lim\limits_{n\rightarrow \infty}a_nx_0^n = 0$. Hence, the sequence $\{a_nx_0^n\}$ is surely bounded: there is some number $M$ such that \begin{equation*}
        |a_nx_0|^n = |a_n|\cdot|x_0|^n \leq M
    \end{equation*}
    for all $n$. Now if $x$ is in $[-a,a]$, then $|x| \leq |a|$, so \begin{align*}
        |a_nx^n| &= |a_n|\cdot |x|^n \\
        &\leq |a_n|\cdot|a|^n \\
        &= |a_n|\cdot|x_0|^n\cdot\left|\frac{a}{x_0}\right|^n \\
        &\leq M\left|\frac{a}{x_0}\right|^n
    \end{align*}
    But $|a/x_0| < 1$, so the (geometric) series \begin{equation*}
        \sum\limits_{n=0}^{\infty}M\left|\frac{a}{x_0}\right|^n = M\sum\limits_{n=0}^{\infty}\left|\frac{a}{x_0}\right|^n
    \end{equation*}
    converges. Choosing $M\cdot|a/x_0|^n$ as the number $M_n$ in \nameref{thmname:mtest}, it follows that $\sum\limits_{n=0}^{\infty}a_nx^n$ converges uniformly on $[-a,a]$.


    To prove the same assertion for $g(x) = \sum\limits_{n=1}^{\infty}na_nx^{n-1}$ notice that \begin{align*}
        |na_nx^{n-1}| &= n|a_n|\cdot |x^{n-1}| \\
        &\leq n|a_n|\cdot|a^{n-1}| \\
        &= \frac{|a_n|}{|a|}\cdot|x_0|^nn\left|\frac{a}{x_0}\right|^n \\
        &\leq \frac{M}{|a|}n\left|\frac{a}{x_0}\right|^n
    \end{align*}
    Since $|a/x_0| < 1$, the series \begin{equation*}
        \sum\limits_{n=1}^{\infty} \frac{M}{|a|}n\left|\frac{a}{x_0}\right|^n = \frac{M}{|a|}\sum\limits_{n=1}^{\infty}n\left|\frac{a}{x_0}\right|^n
    \end{equation*}
    converges (by an application of the Ratio Tes). Another appeal to \nameref{thmname:mtest} proves that $\sum\limits_{n=1}^{\infty}na_nx^{n-1}$ converges uniformly on $[-a,a]$.

    Finally, our corollary proves, first that $g$ is continuous, and then that \begin{equation*}
        f'(x) = g(x) = \sum\limits_{n=1}^{\infty}na_nx^{n-1}
    \end{equation*}
    for all $x \in [-a,a]$. Since we could have chosen any $a$ with $0 < a < |x_0|$, this result holds for all $x$ with $|x| < |x_0|$.
\end{proof*}






%%%%%%%%%%%%%%%%%%%%%%%%%%%%%%%%%%%%% Part 2
\part{Higher-Dimesional Analysis}


%%%%%%%%%%%%%%%%%%%%%% - P2.Chapter 1
\chapter{Metric Spaces}

%%%%%%%%%%%%%%%%%%%%%% - P2.Chapter 2
\chapter{Higher-Dimensional Differentiation}


%%%%%%%%%%%%%%%%%%%%%% - P2.Chapter 3
\chapter{Higher-Dimensional Integration}



%%%%%%%%%%%%%%%%%%%%%% - P2.Chapter 4
\chapter{Manifolds}



%%%%%%%%%%%%%%%%%%%%%% - P2.Chapter 5
\chapter{Differential Forms}


%%%%%%%%%%%%%%%%%%%%%% - P2.Chapter 6
\chapter{Integration on Chains}


%%%%%%%%%%%%%%%%%%%%%% - P2.Chapter 7
\chapter{Integration on Manifolds}


%%%%%%%%%%%%%%%%%%%%%%%%%%%%%%%%%%%%% Part 3
\part{Function Spaces}


%%%%%%%%%%%%%%%%%%%%%% - P3.Chapter 1
\chapter{Normed Spaces}


%%%%%%%%%%%%%%%%%%%%%% - P3.Chapter 2
\chapter{Hilbert Spaces}


%%%%%%%%%%%%%%%%%%%%%% - P3.Chapter 3
\chapter{Banach Spaces}



%%%%%%%%%%%%%%%%%%%%%% - P3.Chapter 4
\chapter{Differentiation and Integration}


%%%%%%%%%%%%%%%%%%%%%% - P3.Chapter 5
\chapter{Banach Algebras}


%%%%%%%%%%%%%%%%%%%%%%%%%%%%%%%%%%%%% Part 4
\part{Measure Theory}

%%%%%%%%%%%%%%%%%%%%%% - P4.Chapter 1
\chapter{Measures}

%%%%%%%%%%%%%%%%%%%%%% - P4.Chapter 2
\chapter{\texorpdfstring{$L^p$ Spaces}{}}

%%%%%%%%%%%%%%%%%%%%%% - P4.Chapter 3
\chapter{Radon Measures}





%%%%%%%%%%%%%%%%%%%%%% - Appendices
\begin{appendices}
    \section{Multivariate Calculus - with Applications}
    
    \subsection{Vector Functions and Derivatives}
    
    \begin{defn}{}{}
        A \Emph{vector function} of one variable is a function $\vec{f}:J\subseteq \R\rightarrow \R^n$ defined by $t \mapsto \vec{f}(t)$, where $\vec{f}(t) \in \R^n$ is unique. 
    \end{defn}
    
    \begin{defn}{}{}
        Let $\vec{f}:\R\rightarrow \R^n$ be a vector valued function. Then we define the derivative of $\vec{f}$ at $t$ by \begin{equation}
            \frac{d\vec{f}(t)}{dt} = \lim\limits_{\Delta t\rightarrow 0} \frac{\vec{f}(t+\Delta t) - \vec{f}(t)}{\Delta t}
        \end{equation}
    \end{defn}
    
    \begin{rmk}{Properties}{}
        Let $\vec{f}:J\subseteq \R\rightarrow \R^n$ and $\vec{g}:I\subseteq \R\rightarrow \R^n$ be vector functions such that \begin{equation}
            \vec{f} = \langle f_1,...,f_n\rangle \; and\;\vec{g} = \langle g_1,...,g_n\rangle 
        \end{equation}
        The for all $t \in J \cap I$ and all $\lambda: D_{\lambda}\subseteq \R\rightarrow \R$ we have \begin{enumerate}
            \item $(\vec{f}+\vec{g})(t) := \vec{f}(t) + \vec{g}(t)$
            \item $(\lambda\vec{f})(t) := \lambda(t)\vec{f}(t)$
            \item $(\vec{f}\cdot\vec{g})(t) = \vec{f}(t)\cdot \vec{g}(t)$
            \item For $n = 3$, $(\vec{f} \times \vec{g})(t) = \vec{f}(t) \times \vec{g}(t)$
        \end{enumerate}
    \end{rmk}
        
    \begin{rec}{}{}
        Given $A \in \R^{n\times n}$, the lagrange expansion is \begin{equation}
            \det(A) = \sum\limits_{i=1}^n(-1)^{i+j}\det(A_{ij})\;(\text{along $j$-th column})
        \end{equation}
        \begin{equation}
            \det(A) = \sum\limits_{j=1}^n(-1)^{i+j}\det(A_{ij})\;(\text{along $i$-th row})
        \end{equation}
        where $A_{ij}$ is the minor matrix of $A$ obtained by deleting the ith ro and jth column of $A$.
    \end{rec}

    \begin{defn}{}{}
        Suppose $\vec{f}(t) = \langle f_1(t),...,f_n(t)\rangle$ and $\vec{L} = \langle L_1,...,L_n\rangle$ Then \begin{equation}
            \lim\limits_{t\rightarrow t_0}\vec{f}(t) = \vec{L} \implies \lim\limits_{t\rightarrow t_0}f_i(t) = L_i, \forall 1\leq i \leq n
        \end{equation}
    \end{defn}

    \begin{rmk}{}{}
        The right hand limit $\lim_{t\rightarrow t_0^+}\vec{f}(t)$ and the left hand limit $\lim_{t\rightarrow t_0^-}\vec{f}(t)$ are defined in the same way.
    \end{rmk}

    \begin{rmk}{Limit Rules}{}
        If $\lim_{t\rightarrow t_0}$ for $\vec{f}(t)$, $\vec{g}(t)$, and $\lambda(t)$ exist and $k \in \R$, then \begin{enumerate}
            \item $\lim\limits_{t\rightarrow t_0}(\vec{f}(t)+\vec{g}(t)) = \lim\limits_{t\rightarrow t_0}\vec{f}(t) + \lim\limits_{t\rightarrow t_0}\vec{g}(t)$
            \item $\lim\limits_{t\rightarrow t_0}k\vec{f}(t) = k\lim\limits_{t\rightarrow t_0}\vec{f}(t)$
            \item $\lim\limits_{t\rightarrow t_0}\lambda(t)\vec{f}(t) = (\lim\limits_{t\rightarrow t_0}\lambda(t))(\lim\limits_{t\rightarrow t_0}\vec{f}(t))$
            \item $\lim\limits_{t\rightarrow t_0}\vec{f}(t)\cdot \vec{g}(t) = (\lim\limits_{t\rightarrow t_0}\vec{f}(t))\cdot(\lim\limits_{t\rightarrow t_0}\vec{g}(t))$
            \item $\lim\limits_{t\rightarrow t_0}\vec{f}(t)\times \vec{g}(t) = (\lim\limits_{t\rightarrow t_0}\vec{f}(t))\times(\lim\limits_{t\rightarrow t_0}\vec{g}(t))$, for $n = 3$.
        \end{enumerate}
    \end{rmk}


    \begin{defn}{Continuity}{}
        A vector function $\vec{f}(t) = \langle f_1(t),...,f_n(t)\rangle$ is said to be continuous at $t = t_0$ if \begin{equation}
            \lim\limits_{t\rightarrow t_0}\vec{f}(t) = \vec{f}(t_0)
        \end{equation}
        In other words, each component function is continuous at $t = t_0$.
    \end{defn}

    \begin{defn}{Differentiability}{}
        A vector function $\vec{f}(t) = \langle f_1(t),...,f_n(t)\rangle$ is said to be differentiable at $t = t_0$ if $\vec{f}(t)$ is defined at and around $t$ and \begin{equation}
            \lim\limits_{t\rightarrow t_0}\frac{\vec{f}(t) - \vec{f}(t_0)}{t - t_0}
        \end{equation}
        exists. We call this limit the derivative of $\vec{f}(t)$ at $t = t_0$ and is denoted by $\vec{f}'(t_0)$ if it exists.
    \end{defn}

    \begin{thm}{}{}
        We say that the vector function is differentiable at $t = t_0$ if and only if its component functions are differentiable at $t = t_0$, and \begin{equation}
            \vec{f}'(t_0) = \langle f_1'(t_0),...,f_n'(t_0)\rangle
        \end{equation}
    \end{thm}



    \begin{rmk}{Differentiation Rules}{}
        Let $\vec{f}(t), \vec{g}(t)$ and $\lambda(t)$ be differentiable and $k \in \R$. Then \begin{enumerate}
            \item $(\vec{f}+\vec{g})'(t) = \vec{f}'(t) + \vec{g}'(t)$
            \item $(k\vec{f})'(t) = k\vec{f}'(t)$
            \item $(\lambda\vec{f})'(t) = \lambda'(t)\vec{f}(t) + \lambda(t)\vec{f}'(t)$
            \item $(\vec{f}\cdot\vec{g})'(t) = \vec{f}'(t)\cdot \vec{g}(t) + \vec{f}(t)\cdot \vec{g}'(t)$
            \item $(\vec{f}\times \vec{g})'(t) = \vec{f}'(t)\times \vec{g}(t) + \vec{f}(t)\times \vec{g}'(t)$ for $n = 3$
            \item $(\vec{f}(\lambda(t)))' = \vec{f}'(\lambda(t))\lambda'(t)$
        \end{enumerate}
    \end{rmk}


    \begin{defn}{}{}
        Let $\vec{f}(t) = \langle f_1(t),...,f_n(t)\rangle$ be a vector function defined on a closed interval $[a,b]$. We say that $\vec{f}(t)$ is \Emph{integrable} on $[a,b]$ if each $f_i(t)$ is integrable on $[a,b]$. When that is the case we define \begin{equation}
            \int_a^b\vec{f}(t)dt := \left\langle \int_a^bf_1(t)dt,...,\int_a^bf_n(t)dt\right\rangle
        \end{equation}
        the definite integral of $\vec{f}(t)$ on $[a,b]$
    \end{defn}


    \begin{rmk}{Properties}{}
        Let $\vec{f}(t),\vec{g}(t)$ be integrable on $[a,b]$ and $k \in \R$. Then \begin{enumerate}
            \item $\int\limits_a^bk\vec{f}(t)dt = k\int\limits_a^b\vec{f}(t)dt$
            \item $\int\limits_a^b(\vec{f}(t)+\vec{g}(t))dt = \int\limits_a^b\vec{f}(t)dt + \int\limits_a^b\vec{g}(t)dt$
            \item $\int\limits_a^b\vec{f}(t)dt = \int\limits_a^c\vec{f}(t)dt + \int\limits_c^b\vec{f}(t)dt$, $a \leq c \leq b$.
            \item $\left|\int\limits_a^b\vec{f}(t)dt\right| \leq \int\limits_a^b\left|\vec{f}(t)\right|dt$ (\Emph{The triangle inequality})
        \end{enumerate}
    \end{rmk}


    \subsection{Parametric Curves and Paths}

    \begin{defn}{}{}
        Let $\vec{f}:J\subseteq \R\rightarrow \R^n$ given by $t \mapsto \langle f_1(t),..., f_n(t)\rangle$, and $f_j: J\subseteq \R\rightarrow \R$ be the $j$th component function of $\vec{f}$. We define the image \begin{equation}
            \vec{f}(J) = \mathscr{C}
        \end{equation}
        and call $\mathscr{C}$ a \Emph{curve parametrized} by $\vec{f}$. If $J = [a,b]$ for $a \leq b \in \R$ and $\vec{f}(a) = \vec{f}(b)$, then $\mathscr{C}$ is a \Emph{closed curve}. If there exists a parametrization $\vec{g}:I \subseteq \R \rightarrow \R^n$ of $\mathscr{C}$ such that $\vec{g}$ is injective (except maybe at end points), then $\mathscr{C}$ is said to be a \Emph{non-self intersecting} curve. If such a curve is closed it is called a \Emph{simple closed curve}.
    \end{defn}

    \begin{rmk}{}{}
        The pair $(\vec{f}(t), J)$ is called a \Emph{parametrization} of the curve $\mathscr{C}$. The triple $(\vec{f}(t), J, \mathscr{C})$ is called a \Emph{path with curve $\mathscr{C}$}.
    \end{rmk}

    \begin{defn}{}{}
        If $\vec{f}:J \subseteq \R\rightarrow \R^n$ is a one-to-one function, then the image $\vec{f}(J) = \mathscr{C}$ is an \Emph{oriented curve} and $\vec{f}$ is a \Emph{consistently oriented path} which covers $\mathscr{C}$.
    \end{defn}

    \begin{rmk}{Ellipse}{}
        The parametrization of an ellipse with equation $\frac{(x-x_0)^2}{a^2} + \frac{(y-y_0)^2}{b^2} = 1$, $a,b > 0$ are constants, is $\vec{f}(t) = \langle x_0 + a\cos(t), y_0 + b\sin(t)\rangle$, for $t \in [0,2\pi)$.
    \end{rmk}


    \begin{rmk}{Line}{}
        Let $A$ and $B$ be points in $\R^n$. The parametrization of the line segment connecting $A$ to $B$ is \begin{equation}
            (\vec{f}(t) = \overline{OA} + t(\overline{OB} - \overline{OA}), [0,1])
        \end{equation}
    \end{rmk}


    \begin{defn}{}{}
        The tangent vector to a curve $\mathscr{C}$ with parametrization $(\vec{f}, I)$ exists at a point $t = t_0$ if $\vec{f}$ is $\vec{f}'(t)$ exists.
    \end{defn}


    \begin{defn}{}{}
        Let $\mathscr{C}$ be a parametric curve with parametrization $(\vec{f}(t),[a,b])$. If $\vec{f}(t)$ is differentiable at $t = t_0$, then $\vec{f}(t_0)$ is called a \Emph{tangent vector} to $\mathscr{C}$ at $P_0 = tip(\vec{f}(t_0))$, provided $\vec{f}'(t_0) \neq \vec{0}$. If $\vec{f}(t)$ is differentiable at every $t \in (a,b)$ and $\vec{f}'(t) \neq \vec{0}$, we say that the parametric curve $\mathscr{C}$ is a \Emph{smooth parametric curve}.
    \end{defn}
    

    \begin{defn}{}{}
        Let $\mathscr{C}$ be a parametric curve and let $(\vec{f}(t), I)$ be a parametrization. If $P_0 = tip(\vec{f}(t_0))$, for $t_0 \in I$, such that $\vec{f}'(t_0) \neq \vec{0}$. Then \begin{equation}
            \vec{T}(t_0) = \frac{1}{|\vec{f}'(t_0)|}\vec{f}'(t_0)
        \end{equation}
        is called the \Emph{unit tangent vector} associated with the parametrization $(\vec{f}(t), I)$. $\vec{T}(t_0)$ always forces the direction in which $\vec{f}(t)$ traces $\mathscr{C}$. The vector \begin{equation}
            \vec{N}(t_0) = \frac{1}{|\vec{T}'(t_0)|}\vec{T}'(t_0)
        \end{equation}
        is perpendicular to $\vec{T}(t_0)$ and is called the \Emph{unit principal normal} to $\mathscr{C}$ at $P_0$.
    \end{defn}


    
    \begin{defn}{}{}
        A curve $\mathscr{C}$ is called \Emph{piecewise smooth} if it consists of a finite number of smooth parametric curves $\mathscr{C}_1,...,\mathscr{C}_k$, where the endpoint of $\mathscr{C}_i$ is the starting point of $\mathscr{C}_{i+1}$ for $i = 1,2,...,k-1$.
    \end{defn}


    \begin{defn}{}{}
        Let $\mathscr{C}$ be a bounded continuous curve specified by a parametrization $\vec{f}:[a,b]\rightarrow \R^n$. We consider partitions of $[a,b]$ into $n$-subintervals by \begin{equation}
            a= t_0 < t_1 < ... < t_n = b
        \end{equation}
        So the points $\vec{f}(t_i)$ subdivide $\mathscr{C}$, and using the \Emph{chord length} $|\vec{f}(t_i) - \vec{f}(t_{i-1})|$ we define the sequence of lengths approximating $\mathscr{C}$ by \begin{equation}
            s_n = \sum\limits_{i=1}^n|\vec{f}(t_i) - \vec{f}(t_{i-1})|
        \end{equation}
        We say $\mathscr{C}$ is \Emph{rectifiable} if there exists $K \in \R$ such that $s_n \leq K$ for all $n \in \N$ and all choices of points. From the completeness axiom of $\R$ there exists a least such $K$. This $K$ we define as the \Emph{length} of $\mathscr{C}$ and we denote it by $s$. Let $\Delta t_i = t_i - t_{i-1}$ and $\Delta \vec{f}_i = \vec{f}(t_i) - \vec{f}(t_{i-1})$ so \begin{equation}
            s_n = \sum\limits_{i=1}^n\left|\frac{\Delta \vec{f}_i}{\Delta t_i}\right|\Delta t_i
        \end{equation}
        If $\vec{f}(t)$ has a continuous derivative, then \begin{equation}
            s = \lim\limits_{\underset{sup\Delta t_i\rightarrow 0}{n\rightarrow \infty}} s_n = \int\limits_a^b\left|\frac{d\vec{f}}{dt}\right|dt
        \end{equation}
        is the \Emph{arclength}.
    \end{defn}


    \begin{defn}{}{}
        Let $\vec{f}:[a,b] \rightarrow \R^n$ be a smooth parammetrization of a curve $\mathscr{C}$. Then the \Emph{arclength function} of $\vec{f}$ is a function $s:[a,b] \rightarrow \R$ where \begin{equation}
            s(t) := \int\limits_a^b\left|\frac{d\vec{f}}{dt}\right|dt
        \end{equation}
        and the \Emph{arclength element} for $\mathscr{C}$ is given by \begin{equation}
            ds := \left|\frac{d}{dt}\vec{f}\right|dt
        \end{equation}
    \end{defn}

    \begin{defn}{}{}
        If $\vec{f}:J\subseteq \R\rightarrow \R^n$ parametrizes a curve $\mathscr{C}$ with the parameter being the arclenght along the curve relative to some inital point, then we call this an \Emph{arclength} or \Emph{intrinsic parametrization}. Such a parametrization traces $\mathscr{C}$ at unit speed \begin{equation}
            \left|\frac{d\vec{f}(s)}{dt}\right| = 1
        \end{equation}
    \end{defn}


    \subsection{Functions of Several Variables}

    \begin{defn}{}{}
        A function $f:\mathscr{D}(f) \subseteq \R^n\rightarrow \R$, where $\mathscr{D}(f)$ is the \Emph{domain} of $f$, is called a \Emph{scalar field on $\mathscr{D}(f)$}. The image of $f$ is \begin{equation}
            \ran(f):=\{x \in \R:\exists\vec{v} \in \mathscr{D}(f), f(\vec{v}) = x\}
        \end{equation}
        The natural domain of $f$ is the largest subset of $\R^n$ such that $f$ is well-defined.
    \end{defn}

    
    \begin{defn}{}{}
        The \Emph{graph} of a function $f:A\rightarrow B$ is the set \begin{equation}
            \Gamma(f):=\{(x,f(x)):x \in A\}
        \end{equation}
        For \begin{equation}
            f:\prod\limits_{i=1}^nX_i\rightarrow B
        \end{equation}
        we have the graph \begin{equation}
            \Gamma(f):= \{((x_1,...,x_n),f(x_1,...,x_n)):(x_1,...,x_n) \in \prod\limits_{i=1}^nX_i\}
        \end{equation}
    \end{defn}

    \begin{rmk}{}{}
        The grap of a function $f:\R^n\rightarrow \R$ can be considered as a surface in $\R^{n+1}$ by a natural embedding.
    \end{rmk}


    \begin{defn}{}{}
        Given a function $f:\R^n\rightarrow \R$, a \Emph{k-level surface} of $f$ is a set \begin{equation}
            S_k := \{\vec{x} \in \R^n: f(\vec{x}) = k\}
        \end{equation}
        where $k$ is a fixed constant.
    \end{defn}


    \begin{defn}{}{}
        Let $\vec{x}_0 \in \R^n$ and $r > 0$ a real number. Then the \Emph{open ball of radius $r$} centered at $\vec{x}_0$ is defined as \begin{equation}
            B_r(\vec{x}_0) := \{\vec{x} \in \R^n: ||\vec{x} - \vec{x}_0|| < r\}
        \end{equation}
        the \Emph{closed ball} is defined by \begin{equation}
            \overline{B}_r(\vec{x}_0) := \{\vec{x} \in \R^n: ||\vec{x} - \vec{x}_0|| \leq r\}
        \end{equation}
    \end{defn}

    \begin{defn}{}{}
        A \Emph{neighborhood} of a point $\vec{x}_0 \in \R^n$ is any set $U \subseteq \R^n$ such that there exists $r > 0$ so that $B_r(\vec{x}_0) \subseteq U$.
    \end{defn}


    \begin{defn}{}{}
        Let $E \subseteq \R^n$, where we equip $\R^n$ with $\mathscr{T}_{st}$. We say $E$ is \Emph{open} if $E \in \mathscr{T}_{st}$. Equivalently, $E$ is \Emph{open} if for all $\vec{x} \in E$, $E$ is a \Emph{neighborhood} of $\vec{x}$. We sau $E$ is \Emph{closed} if its \Emph{complement} $E^C = \R^n\backslash E$ is \Emph{open}.
    \end{defn}

    \begin{defn}{}{}
        A point $\vec{x}_0 \in \R^n$ is called a \Emph{boundary point} of $E$ if for any $r > 0$, $B_r(\vec{x}_0) \cap E \neq \emptyset$ and $B_r(\vec{x}_0) \cap E^C \neq \emptyset$.
    \end{defn}

    \begin{defn}{}{}
        The set of all \Emph{boundary points} of a set $E \subseteq \R^n$ is called the \Emph{boundary} of $E$.
    \end{defn}

    \begin{defn}{}{}
        Let $E \subseteq \R^n$. We say that $E$ is \Emph{bounded} if there exists $R > 0$ such that $||\vec{x}|| \leq R$ for all $\vec{x} \in E$. $E$ is \Emph{unbounded} if for all $R > 0$ there exists $\vec{x}_0 \in E$ such that $||\vec{x}_0|| > R$.
    \end{defn}


    \begin{defn}{}{}
        Let $f:\R^n\rightarrow \R$ be a function. We say $\lim_{\vec{x}\rightarrow \vec{x}_0}f(\vec{x}) = L$, provided that \begin{enumerate}
            \item Every punctered neighborhood $B_r^*(\vec{x}_0)$ of $\vec{x}_0$ intersects $\mathscr{D}(f)$ 
                \begin{equation}
                        B_r^*(\vec{x}_0) \cap \mathscr{D}(f) \neq \emptyset
                \end{equation}
                that is, $\vec{x}_0$ is a limit point of $\mathscr{D}(f)$.
            \item For all $\varepsilon > 0$, there exists $\delta > 0$ such that $f(\vec{x}) \in B_{\varepsilon}(L)$ whenever $\vec{x} \in B_{\delta}^*(\vec{x}_0) \cap \mathscr{D}(f)$. That is \begin{equation}
                    f(\mathscr{D}(f)\cap B_{\delta}^*(\vec{x}_0)) \subseteq B_{\varepsilon}(L)
            \end{equation}
        \end{enumerate}
    \end{defn}

    
    \begin{rmk}{}{}
        As $\R^n$ and $\R$ are metric spaces, they are Hausdorff, so if the limit exists it is unique.
    \end{rmk}


    \begin{rmk}{Limit Properties}{}
        Let $f:\R^n\rightarrow \R$ and $g:\R^n\rightarrow \R$ and $\vec{x}_0\in \R^n$ such that $\lim_{\vec{x}\rightarrow \vec{x}_0}f(\vec{x}_0) = L$ and $\lim_{\vec{x}\rightarrow \vec{x}_0}g(\vec{x}) = M$. Then if $\vec{x}_0$ is not an isolated point of $\mathscr{D}(f) \cap \mathscr{D}(g)$, then \begin{enumerate}
            \item $\lim_{\vec{x}\rightarrow \vec{x}_0}(f(\vec{x}) \pm g(\vec{x})) = L \pm M$
            \item $\lim_{\vec{x}\rightarrow \vec{x}_0}f(\vec{x})g(\vec{x}) = LM$
            \item $\lim_{\vec{x}\rightarrow \vec{x}_0}\frac{f(\vec{x})}{g(\vec{x})} = \frac{L}{M}$ if $M \neq 0$
            \item If $F:\R\rightarrow X$ is continuous at $L$, then $\lim_{\vec{x}\rightarrow \vec{x}_0}F(f(\vec{x})) = F(L)$.
        \end{enumerate}
    \end{rmk}

    \begin{defn}{}{}
        We say a function $f:\R^n\rightarrow \R$ is \Emph{continuous at a point} $\vec{x}_0\in \R^n$ if \begin{equation}
            \lim_{\vec{x}\rightarrow \vec{x}_0} f(\vec{x}) = f(\vec{x}_0)
        \end{equation}
    \end{defn}


    

    \begin{rmk}{}{}
        If setting $x = x_0$ and $y = y_0$ in the expression for $f(x,y)$ does not evalutate to a real number, then we can try using polar coordinates: $x = x_0 + r\cos(\theta)$ and $y = y_0 + r\sin(\theta)$. Recall $r = \sqrt{(x-x_0)^2 + (y-y_0)^2}, 0 \leq \theta < 2\pi$. As a result, $(x,y) \rightarrow (x_0,y_0)$ is equivalent to $r\rightarrow 0$, so \begin{equation}
            \lim\limits_{(x,y) \rightarrow (x_0,y_0)}f(x,y) = \lim\limits_{r\rightarrow 0}f(x_0+r\cos(\theta), y_0+r\sin(\theta))
        \end{equation}
    \end{rmk}

    \begin{thm}{Squeeze Theorem}{}
        Let $f(x,y), g(x,y)$ and $h(x,y)$ be defined in a neighborhood $U$ of $(x_0,y_0)$, except maybe at $(x_0,y_0)$, and such that \begin{equation}
            g(x,y) \leq f(x,y) \leq h(x,y), \forall (x,y) \in U\backslash\{(x_0,y_0)\}
        \end{equation}
        If \begin{equation}
            \lim\limits_{(x,y)\rightarrow (x_0,y_0)}g(x) = L,\lim\limits_{(x,y)\rightarrow (x_0,y_0)}h(x) = L
        \end{equation}
        Then $\lim\limits_{(x,y)\rightarrow (x_0,y_0)} f(x) = L$.
    \end{thm}

    \begin{thm}{}{}
        If one can find two continuous parametric curves $\mathscr{C}_1$ and $\mathscr{C}_2$ that pass through the point $(x_0,y_0)$ such that \begin{equation}
            \lim\limits_{\underset{(x,y) \in \mathscr{C}_1}{(x,y)\rightarrow (x_0,y_0)}}f(x,y) = L_1, \lim\limits_{\underset{(x,y) \in \mathscr{C}_2}{(x,y)\rightarrow (x_0,y_0)}}f(x,y) = L_2,\;and\;L_1 \neq L_2
        \end{equation}
        then $\lim\limits_{(x,y)\rightarrow (x_0,y_0)}f(x,y)$ does not exist.
    \end{thm}

   
    \begin{defn}{}{}
        Let $f:\R^n\rightarrow \R$ be a function. \begin{enumerate}
            \item We say that $f$ is \Emph{continuous} at $\vec{x}_0 \in \R^n$ if \begin{enumerate}
                    \item There exists a neighborhood $U$ of $\vec{x}_0$ such that $f(U)$ is defined
                    \item $\lim_{\vec{x}\rightarrow \vec{x}_0}f(\vec{x}) = f(\vec{x}_0)$
            \end{enumerate}
            \item We say that $f$ is continuous on a region $D$ if it s continuous at every point $\vec{x}$ in the region.
        \end{enumerate}
    \end{defn}


    \begin{rmk}{Constructing Continuous Functions}{}
        Let $f,g:\R^n\rightarrow \R$ be continuous at $\vec{x}_0 \in \R^n$, and if $\lambda \in \R$, then \begin{enumerate}
            \item $f\pm g$, $f\cdot g$, and $\lambda f$ are continuous at $\vec{x}_0$
            \item $f/g$ is continuous at $\vec{x}_0$ provided $g(\vec{x}_0) \neq 0$.
        \end{enumerate}
        Suppose $u(t)$ is continuous at $t_0 = f(\vec{x}_0)$. Then $u(f(\vec{x}))$ is continuous at $\vec{x}_0$.
    \end{rmk}

    \begin{namthm}{Extreme Value Theorem}{}
        If $f:\R^n \rightarrow \R$ is continuous on a closed and bounded region $D \subseteq \R^n$, then there exist $\vec{x}_m,\vec{x}_M \in D$ such that \begin{equation}
            f(\vec{x}_m) \leq f(\vec{x}) \leq f(\vec{x}_M), \forall \vec{x} \in D
        \end{equation}
        $m = f(\vec{x}_m)$ is called the \Emph{absolute minimum} of $f$ on $D$, while $M = f(\vec{x}_M)$ is called the \Emph{absolute maximum} of $f$ on $D$.
    \end{namthm}
    

    \subsection{Partial Derivatives}

    \begin{defn}{}{}
        The \Emph{first partial derivatives} of a function $f:\R^n\rightarrow \R$ with respect to the variable $x_i$, $1 \leq i \leq n$, is the function \begin{equation}
            f_i(x_1,...,x_n) = \lim_{h\rightarrow 0}\frac{f(x_1,...,x_i+h,...,x_n) - f(x_1,...,x_i,...,x_n)}{h}
        \end{equation}
        provided the limit exists and $f$ is defined in a neighbrohood of $(x_1,...,x_n)$.
    \end{defn}

    \begin{nota*}{}{}
        We often write \begin{equation}
            \frac{\partial}{\partial x_i}f(x_1,...,x_n) = f_i(x_1,...,x_n) = D_if(x_1,...,x_n)
        \end{equation}
        and \begin{equation}
            \left(\frac{\partial}{\partial x_i}f(\vec{x})\right)\Big\rvert_{\vec{x}_0} = f_i(\vec{x}_0) = D_if(\vec{x}_0)
        \end{equation}
    \end{nota*}
    

    \begin{rmk}{}{}
        If a function $f:J\subseteq \R^n\rightarrow \R$ has first partial derivatives at $\vec{x}_0$ in a region $D \subseteq \R^n$, then this defines $n$ new functions \begin{equation}
            \frac{\partial f}{\partial x_i}\Big\rvert:\R\rightarrow \R
        \end{equation}
        where we differentiate $f$ with respect to $x_i$.
    \end{rmk}

    \begin{defn}{}{}
        Given a function $f:\R^n\rightarrow \R$, the \Emph{gradient} of $f$ is the vector function \begin{equation}
            \nabla f = grad(f):\R^n\rightarrow\R^n
        \end{equation}
        such that \begin{equation}
            \nabla f(x_1,...,x_n) = \left\langle \frac{\partial}{\partial x_1}f,...,\frac{\partial}{\partial x_n}f\right\rangle
        \end{equation}
        where $\nabla$ is the \Emph{del operator} \begin{equation}
            \nabla = \left[\frac{\partial}{\partial x_1},...,\frac{\partial}{\partial x_n}\right]^T
        \end{equation}
    \end{defn}


    \begin{defn}{}{}
        Let $f:\R^n\rightarrow \R$ be defined in a neighborhood of a point $\vec{x}_0$ such that its first partial derivatives exist at $\vec{x}_0$. Then by definition, the \Emph{linear approximation} of $f$ at $\vec{x}_0$ is the polynomial of degree $1$, $L(\vec{x})$, that matches $f$ at $\vec{x}_0$ and matches its partials at $\vec{x}_0$. In particular, we have that \begin{equation}
            L(\vec{x},\vec{x}_0) = f(\vec{x}_0) + \sum_{i=1}^n \frac{\partial}{\partial x_i}f(\vec{x}_0)(x_i - x_{i,0})
        \end{equation}
    \end{defn}


    \begin{defn}{}{}
        Let $f:D\subseteq \R^n\rightarrow \R$ be a function defined around $\vec{x}_0 \in \R^n$ with first partials also defined. Let $L(\vec{x})$ be its linear approximation at $\vec{x}_0$. We say that $f$ is \Emph{differentiable} at $\vec{x}_0$ if the limit \begin{equation}
            \lim_{\vec{x}\rightarrow \vec{x}_0} \frac{f(\vec{x}) - L(\vec{x})}{||\vec{x}-\vec{x}_0||} = 0
        \end{equation}
    \end{defn}


    \begin{defn}{}{}
        Let $f:D\subseteq \R^n\rightarrow \R^m$ be a multivariate function defined in a neighborhood of $\vec{x}_0 \in \R^n$ with first partial derivatives also defined. Then the \Emph{Jacobian matrix} of $f$ is defined to be \begin{equation}
            Df(\vec{x}) = \begin{bmatrix} \partial_1f_1(\vec{x}) & \partial_2f_1(\vec{x}) & \hdots & \partial_nf_1(\vec{x}) \\
                \partial_1f_2(\vec{x}) & \ddots & \ddots & \vdots \\
                \vdots & \ddots & \ddots & \vdots \\
                \partial_1f_m(\vec{x}) & \hdots & \hdots & \partial_nf_m(\vec{x})
            \end{bmatrix}
        \end{equation}
        Then we say that $f$ is differentiable at $\vec{x}_0$ if \begin{equation}
            \lim_{\vec{x}\rightarrow \vec{x}_0}\frac{||f(\vec{x}) - f(\vec{x}_0) - Df(\vec{x} - \vec{x}_0)||}{||\vec{x} - \vec{x}_0||} = 0
        \end{equation}
        If all the first partial derivatives of $f$ are continuous in a neighborhood of $\vec{x}_0$ then this holds.
    \end{defn}


    \begin{rmk}{Properties}{}
        For $f:D \subseteq \R^n\rightarrow \R^m$: \begin{enumerate}
            \item If $f$ is differentiable at $\vec{x}_0$, then $f$ is continuous at $\vec{x}_0$
            \item If $f$ and $g$ are differentiable at $\vec{x}_0$, then $f\pm g, kf, fg$ ($m = 1$) are differentiable at $\vec{x}_0$.
            \item If the partials of $f$ are continuous in a neighborhood of $\vec{x}_0$, then $f$ is differentiable at $\vec{x}_0$. The converse is not true in general.
        \end{enumerate}
    \end{rmk}

    \begin{defn}{}{}
        Consider $f:D\subseteq \R^n\rightarrow \R$, we have \begin{equation}
            \partial_if(\vec{x}_0) = \lim_{t\rightarrow 0}\frac{f(\vec{x}_0+t\hat{e}_i) - f(\vec{x}_0)}{t}
        \end{equation}
        so we can generalize this to define the \Emph{directional derivative} in the direction of $\hat{u}$: \begin{equation}
            \partial_{\hat{u}}f(\vec{x}_0) = \lim_{t\rightarrow 0}\frac{f(\vec{x}_0+t\hat{u}) - f(\vec{x}_0)}{t}
        \end{equation}
    \end{defn}

    \begin{thm}{}{}
        If $f:D\subseteq \R^n\rightarrow \R$ is differentiable at $\vec{x}_0$, then the directional derivative of $f$ at $\vec{x}_0$ exists in the direction of $\hat{u}$, and is equal to \begin{equation}
            \partial_{\hat{u}}f(\vec{x}_0) = \nabla f(\vec{x}_0) \cdot \hat{u}
        \end{equation}
    \end{thm}


    \begin{thm}{}{}
        Notice if $f:D\subseteq \R^n\rightarrow \R$ is differentiable at $\vec{x}_0$, and $\hat{u}$ is a unit vector, we have that \begin{equation}
            \partial_{\hat{u}}f(\vec{x}_0) = \nabla f(\vec{x}_0) \cdot \hat{u} = |\nabla f(\vec{x}_0)|\cos(\theta)
        \end{equation}
        with $\theta$ being the angle between $\nabla f(\vec{x}_0)$ and $\hat{u}$. As a result: \begin{enumerate}
            \item The largest value of $\partial_{\hat{u}}f(\vec{x}_0)$ is equal to $|\nabla f(\vec{x}_0)|$, and occurs when $\hat{u}$ is in the same direction as the gradient
            \item The smallest value of $\partial_{\hat{u}}f(\vec{x}_0)$ is equal to $-|\nabla f(\vec{x}_0)$ when $\hat{u}$ is in the same direction as $-\nabla f(\vec{x}_0)$
            \item When $\hat{u}$ is perpendicular to $\nabla f(\vec{x}_0)$, the directional derivative is zero.
        \end{enumerate}
    \end{thm}

    \begin{thm}{Chain Rule V1}{}
        Let $f:D_f\subseteq \R^n\rightarrow \R$ and $g:D_g\subseteq \R\rightarrow \R$ such that $g(t) = f(\vec{x}(t))$. Then \begin{equation}
            \frac{dg}{dt} = \nabla f(\vec{x}) \cdot \frac{d}{dt}\langle x_1,...,x_n\rangle = \sum_{i=1}^n \partial_i f\frac{dx_i}{dt}
        \end{equation}
    \end{thm}

    \begin{thm}{Chain Rule V2}{}
        Let $f:D_f\subseteq \R^n\rightarrow \R$ and $g:D_g\subseteq \R^m\rightarrow \R$ such that $g(\vec{x}) = f(y_1(\vec{x}),...,y_n(\vec{x}))$. Then \begin{equation}
        \partial_i g(\vec{x}) = \nabla f(y_1(\vec{x}),...,y_n(\vec{x})) \cdot \frac{\partial}{\partial x_i}\vec{y}
        \end{equation}
        for $\vec{y} = \langle y_1,...,y_n\rangle$.
    \end{thm}


    \begin{thm}{Clairout's Theorem}{}
        Suppose $f:\R^n\rightarrow \R$ has continous first and second partials on an open ball $B_r$. Then $f_{ij}(\vec{x}) = f_{ji}(\vec{x})$ for all $\vec{x} \in D$.
    \end{thm}


    \subsection{Implicit Differentiation}

    \begin{thm}{Implicit Function Theorem (Two variables)}{}
        Consider $F(x,y) = 0$. Let $(x_0,y_0) \in \R^2$ such that $F(x_0,y_0) = 0$, and suppose $F$'s first partials are continuous in a neighborhood of $(x_0,y_0)$. Then \begin{enumerate}
            \item If $F_y(x_0,y_0) \neq 0$, then $F(x,y) = 0$ uniquely defines $y$ as a continuously differentiable function of $x$ in a neighborhood of $x_0$, and we have that \begin{equation}
                    \frac{dy}{dx} = -\frac{\partial_x F(x,y)}{\partial_y F(x,y)}
                \end{equation}
            \item Similarly for $F_x(x_0,y_0) \neq 0$.
        \end{enumerate}
    \end{thm}


    \begin{thm}{Implicit Function Theorem (n variables)}{}
        Consider $F(\vec{x}) = 0 (\star)$, $\vec{x} = (x_1,...,x_n)$. Let $\vec{a}$ satisfy $F(\vec{a}) = 0$, and suppose $F(\vec{x})$ has continuous first partial derivatives at and in a neighborhood of $\vec{a}$. Let $\beta$ be one of the variables $\{x_1,...,x_n\}$, and let $\vec{\alpha}$ be the rest. If $\partial_{\beta}F(\vec{a}) \neq 0$, then the equation $(\star)$ uniquely defines the variable $\beta$ as a continuously differentiable function of $\vec{\alpha}$, and for $x_j \neq \beta$, we have \begin{equation}
            \partial_{x_j}\beta(\vec{\alpha}) = -\frac{\partial_{x_j}F(\vec{a})}{\partial_{\beta}F(\vec{a})}
        \end{equation}
    \end{thm}


    \begin{namthm}{Implicit Function Theorem (General)}{}
        Consider a system of $n$ equations in $n+m$ variables \begin{equation}
            \left\{\begin{array}{l} F_{(1)}(x_1,...,x_m,y_1,...,y_n) = 0 \\ \vdots \\ F_{(n)}(x_1,...,x_m,y_1,...,y_n) = 0 \end{array}\right.
        \end{equation}
        and a point $P_0$ which satisfies the system. Suppose each $F_{(i)}$ is differentiable near $P_0$, so they have continuous first partial derivatives. Finally, suppose \begin{equation}
            \frac{\partial(F_{(1)},...,F_{(n)})}{\partial(y_1,...,y_n)}\Big\rvert_{P_0} \neq 0
        \end{equation}
        Then the system defines $y_1,...,y_n$ uniquely as continuously differentiable functions of $x_1,...,x_m$ in some neighborhood of $P_0$. Moreover, \begin{equation}
            \partial_{x_j}y_i = -\frac{\frac{\partial(F_{(1)},...,F_{(n)})}{\partial(y_1,...,x_j,...,y_n)}}{\frac{\partial(F_{(1)},...,F_{(n)})}{\partial(y_1,...,y_i,...,y_n)}}
        \end{equation}
        This formula is a consequence of Cramer's Rule applied to the n linear equations in $n$ unknowns which is the system differentiated with respect to $x_j$.
    \end{namthm}



    \subsection{Differentials}


    \begin{defn}{}{}
        Let $f:D\subseteq \R^n\rightarrow \R$ be a function defined at and around a point $\vec{a}$. Given $\Delta \vec{x}$, of small magnitude, \begin{equation}
            \Delta f_{\vec{a}}(\Delta \vec{x}) = f(\vec{a}+\Delta \vec{x}) - f(\vec{a})
        \end{equation}
        represents the change in the value of the function associated with the change $\Delta \vec{x}$ in $\vec{x}$ at $\vec{a}$. Then, we approximate this change with the \Emph{differential} at $\vec{a}$ defined by \begin{equation}
            df_{\vec{a}}(\Delta \vec{x}) = \nabla f(\vec{a}) \cdot \Delta \vec{x}
        \end{equation}
        If $\Delta \vec{x}$ is sufficiently small, these changes are approximately equal.
    \end{defn}


    
    \subsection{Taylor Polynomials}
    
    \begin{namthm}{Taylor's Theorem (One Variable)}
        Let $f(x)$ be a function wit $n+1$ continuous derivatives in the open interval $(a,b)$. Let \begin{equation}
            T_n(x) := \sum\limits_{i=0}^n\frac{f^{(i)}(c)(x-c)^i}{i!}
        \end{equation}
        be the \Emph{degree n Taylor polynomial} of $f(x)$ centered at $x = c \in (a,b)$. Then, for any $x \in (a,b)$, there exists a number $\theta$ between $c$ and $x$ such that \begin{equation}
            f(x) = T_n(x) + \frac{f^{(n+1)}(\theta)}{(n+1)!}(x-c)^{n+1}
        \end{equation}
    \end{namthm}
    
    \begin{defn}{Two Variable Taylor Polynomial}{}
        Let $f(x,y)$ be a smooth function (continuous partial derivatives up to whatever degree needed) in a open set $D \subset \R^2$. The \Emph{degree $n$ Taylor polynomial} of $f(x,y)$ at a point $(a,b) \in D$, is the polynomial $T_n(x,y)$ of degree $n$ that equals $f(x,y)$ and its first $n$ partial derivatives at $(a,b)$. It can be written as \begin{equation}
            T_n(x,y) := \sum\limits_{i=0}^n\frac{\left[(x-a)\partial_x + (y-b)\partial_y\right]^{(i)}f(a,b)}{i!}
        \end{equation}
        where $\left[(x-a)\partial_x + (y-b)\partial_y\right]^{(i)}$ is to be expanded as an algebraic expression and the products of $\partial_x$ and $\partial_y$ correspond to composition of operators.
    \end{defn}
    
    \begin{namthm}{Taylor's Theorem (Two variables)}
        Let $f(x,y)$ be a function with continuous partial derivatives up to $(n+1)$ in some neighborhood $D$ of $(a,b) \in \R^2$. Let $T_n(x,y)$ be the degree $n$ Taylor polynomial of $f(x,y)$ at $(a,b)$. Then for any $(x,y) \in D$, there exists $(\alpha,\beta) \in D$ such that \begin{equation}
            f(x,y) = T_n(x,y) + \frac{\left[(x-a)\partial_x + (y-b)\partial_y\right]^{(n+1)}f(\alpha,\beta)}{(n+1)!}
        \end{equation}
        This is called the \Emph{Taylor formula/expansion of order $n$} of $f$ at $(a,b)$. \begin{equation}
            R_n(x,y) := \frac{\left[(x-a)\partial_x + (y-b)\partial_y\right]^{(n+1)}f(\alpha,\beta)}{(n+1)!} = f(x,y) - T_n(x,y)
        \end{equation}
        is called the \Emph{remainder} of the expansion.
    \end{namthm}
    
    \begin{rmk}
        If all partial derivatives of order $(n+1)$ are bounded by some constant $M > 0$, then \begin{equation}
            |f(x,y) - T_n(x,y)| \leq \frac{M}{(n+1)!}\left[\sum\limits_{j=0}^{n+1}\begin{pmatrix} n + 1 \\ j \end{pmatrix}|x-a|^{n+1-j}|y-b|^j\right]
        \end{equation}
    \end{rmk}
    
    
    \begin{namthm}{Taylor's Theorem (General)}
        Let $f:D\subset\R^n \rightarrow \R$ be a function with continuous partial derivatives of order up to $m+1$ in a neighborhood $D$ of $\vec{a} \in \R^n$. Then for all $\vec{x} \in D$ there exists $\vec{theta} \in D$ such that \begin{equation}
            f(\vec{x}) = T_m(\vec{x}) + R_m(\vec{x},\vec{\theta})
        \end{equation}
        where \begin{equation}
            T_m(\vec{x}) := \sum\limits_{k=0}^m\frac{\left[(\vec{x} - \vec{a})\cdot \nabla\right]^{(k)}f(\vec{a})}{k!}
        \end{equation}
        is the degree $m$ Taylor polynomial of $f$ at $\vec{a}$, and \begin{equation}
            R_m(\vec{x}, \vec{\theta}) := \frac{\left[(\vec{x} - \vec{a})\cdot \nabla\right]^{(m+1)}f(\vec{\theta})}{(m+1)!}
        \end{equation}
        is the remainder. If all partial derivatives of $f$ are continuous and there exists $r \in \R^{+}$ such that whenever $||\vec{x} - \vec{a}|| < r$ we have for all $t \in [0,1]$ \begin{equation}
            \lim_{m\rightarrow \infty} R_m(\vec{x}, \vec{a}+t(\vec{x} - \vec{a})) = 0
        \end{equation}
        Then we can represent $f(\vec{x})$ as the \Emph{Taylor series} \begin{equation}
            f(\vec{x}) = \sum\limits_{n = 0}^{\infty}\frac{\left[(\vec{x} - \vec{a})\cdot \nabla\right]^{(n)}f(\vec{a})}{n!}
        \end{equation}
    \end{namthm}
    
    \subsection{Local Extrema of Multivariate Functions}
    
    \begin{defn}{}{}
        Let $f:\R^n\rightarrow \R$ be a function of $n$ variables defined in a neighborhood of $\vec{x}_0 \in \R^n$: \begin{enumerate}
            \item We say $f$ has a \Emph{local maximum} at $\vec{x}_0$ if there exists a neighborhood $D$ of $\vec{x}_0$ for which $f$ is defined and \begin{equation}
                f(\vec{x}) \leq f(\vec{x}_0), \forall \vec{x} \in D
            \end{equation}
            \item We say $f$ has a \Emph{local minimum} at $\vec{x}_0$ if there exists a neighborhood $D$ of $\vec{x}_0$ for which $f$ is defined and \begin{equation}
                f(\vec{x}) \geq f(\vec{x}_0), \forall \vec{x} \in D
            \end{equation}
            \item[$\drsh$] If $f$ has a local maximum or minimum at $\vec{x}_0$, we say $f$ has a \Emph{local extremum} at $\vec{x}_0$.
        \end{enumerate}
    \end{defn}
    
    \begin{thm}{Fermat}{}
        If $f:\R^n\rightarrow \R$ has a local extremum at $\vec{x}_0$, then one of the following must hold:\begin{enumerate}
            \item $\nabla f(\vec{x}_0) = \vec{0}$ when all first partials of $f$ exist at $\vec{x}_0$
            \item At least one of the first partials of $f$ are not defined at $\vec{x}_0$
        \end{enumerate}
    \end{thm}
    
    \begin{defn}{}{}
        Suppose $f:\R^n\rightarrow \R$ is defined in a neighborhood of $\vec{x}_0$. If one of the conditions of Fermat's Theorem is satisfied by $\vec{x}_0$, we say $\vec{x}_0$ is a \Emph{critical point} of $f$.
    \end{defn}
    
    \begin{rmk}{}{}
        To determine if $f$ has a local max or min at a critical point $\vec{x}_0$, study the sign of \begin{equation}
            f(\vec{x}_0 + \vec{h}) - f(\vec{x}_0)
        \end{equation}
        for small $|\vec{h}|$. If it is always positive, $f$ has a local minimum, if it is always negative $f$ has a local maximum, and if it changes sign, $f$ does not have a local extremum and in this case we say $f$ has a \Emph{saddle point} at $\vec{x}_0$.
    \end{rmk}
    
    \begin{namthm}{Second Derivative Test}
        Suppose $f(x,y)$ has continuous second partial derivatives in a neighborhood of a critical point $(x_0,y_0)$. Define the \Emph{Hessian} matrix of $f$ at $(x_0,y_0)$ to be \begin{equation}
            H_f(x_0,y_0) := \begin{bmatrix} f_{xx}(x_0,y_0) & f_{xy}(x_0,y_0) \\ f_{yx}(x_0,y_0) & f_{yy}(x_0,y_0) \end{bmatrix}
        \end{equation}
        Let $\delta_1 = f_{xx}(x_0,y_0)$ and $\delta_2 = \det(H_f(x_0,y_0))$. Then \begin{enumerate}
            \item If $\delta_1 > 0$ and $\delta_2 > 0$, then $f$ has a local minimum at $(x_0,y_0)$
            \item If $\delta_1 < 0$ and $\delta_2 > 0$, then $f$ has a local maximum at $(x_0,y_0)$
            \item If $\delta_2 \neq 0$ but neither case 1 nor case 2 hold, then $f$ has a saddle point at $(x_0,y_0)$.
            \item If $\delta_2 = 0$ the test is inconclusive.
        \end{enumerate}
    \end{namthm}
    
    \begin{namthm}{Second Derivative Test (general)}
        Suppose $f:D \subset \R^n \rightarrow \R$ has continuous second partial derivatives in a neighborhood of a critical point $\vec{x}_0 \in D$. Define the \Emph{Hessian} matrix of $f$ at $\vec{x}_0$ to be \begin{equation}
            H_f(\vec{x}_0) := \begin{bmatrix} f_{11}(\vec{x}_0) & f_{12}(\vec{x}_0) & \hdots & f_{1n}(\vec{x}_0) \\ f_{21}(\vec{x}_0) & \ddots & \ddots & \vdots \\ \vdots & \ddots & \ddots & \vdots \\ f_{n1}(\vec{x}_0) & \hdots & \hdots & f_{nn}(\vec{x}_0) \end{bmatrix}
        \end{equation}
        Denote the the $k$th principal minor of $H_f(\vec{x}_0)$ by \begin{equation}
            \delta_k := \begin{vmatrix} f_{11}(\vec{x}_0) & f_{12}(\vec{x}_0) & \hdots & f_{1k}(\vec{x}_0) \\ f_{21}(\vec{x}_0) & \ddots & \ddots & \vdots \\ \vdots & \ddots & \ddots & \vdots \\ f_{k1}(\vec{x}_0) & \hdots & \hdots & f_{kk}(\vec{x}_0) \end{vmatrix}
        \end{equation}. Then \begin{enumerate}
            \item If for all $i \in \{1,2,...,n\}$, $\delta_i > 0$, then $f$ has a local minimum at $\vec{x}_0$
            \item If for all $i \in \{1,2,...,n\}$, $\delta_{2i-1} <  0$ and $\delta_{2i} > 0$, then $f$ has a local maximum at $\vec{x}_0$
            \item If $\delta_n = \det(H_f(\vec{x}_0) \neq 0$ but neither case 1 nor case 2 hold, then $f$ has a saddle point at $\vec{x}_0$.
            \item If $\delta_n = \det(H_f(\vec{x}_0) = 0$ the test is inconclusive.
        \end{enumerate}
    \end{namthm}
    
    \subsection{Vector Fields}
    
    \begin{defn}{}{}
        A \Emph{vector field} is a vector function $\vec{F}:D\subset \R^n \rightarrow \R^n$. In the case of three variables we write \begin{equation}
            \vec{F}(x,y,z) = \langle P(x,y,z), Q(x,y,z), R(x,y,z)\rangle
        \end{equation}
    \end{defn}
    
    \begin{rmk}{}{}
        A vector field $\vec{F}:D\subset \R^n \rightarrow \R^n$ is said to be of class $C^k$ for $k \in \Z^{+}$ in $D$ if the first $k$ partial derivatives of the component functions of $\vec{F}$ are continuous in $D$.
    \end{rmk}
    
    \begin{defn}{Conservative Fields}{}
        A vector field $\vec{F}:D\subset \R^n \rightarrow \R^n$ is called \Emph{conservative} in a region $E \subseteq D$ if there exists a scalar function $f:D_f \subset \R^n \rightarrow \R$ such that \begin{equation}
            \vec{F}(\vec{x}) = \nabla f(\vec{x}), \forall \vec{x} \in E
        \end{equation}
        where $f$ is called a \Emph{potential function} of the vector field $\vec{F}$.
    \end{defn}
    
    \begin{defn}{}{}
        Let $\vec{F}:D\subset \R^n \rightarrow \R^n$ be a differentiable vector field. The \Emph{divergence} of $\vec{F}$ is the scalar field \begin{equation}
            \nabla\cdot \vec{F} = \sum\limits_{i=1}^n\partial_iF_i
        \end{equation}
        where $F_i$ are the component function of $\vec{F}$.
    \end{defn}
    
    \begin{defn}{}{}
        Let $\vec{F}:D\subset \R^3 \rightarrow \R^3$ be a differentiable vector field. The \Emph{curl} of $\vec{F}$ is the vector field \begin{equation}
            \nabla\times \vec{F} = \begin{vmatrix} \hat{i} & \hat{j} & \hat{k} \\ \partial_x & \partial_y & \partial_z \\ P & Q & R \end{vmatrix}
        \end{equation}
        where $F_i$ are the component function of $\vec{F}$.
    \end{defn}
    
    \begin{prop}{Properties of Divergence}{}
        If $\vec{F}:D\subset \R^n \rightarrow \R^n$ and $\vec{G}:D\subset \R^n \rightarrow \R^n$ vector fields and $f:D\subset \R^n \rightarrow \R$ is a scalar field, and $C_1,C_2 \in \R$, then \begin{enumerate}
            \item (Linearity) $\nabla \cdot (C_1\vec{F} + C_2\vec{G}) = C_1\nabla \cdot \vec{F} + C_2\nabla \cdot \vec{F}$
            \item (Product rule) $\nabla \cdot (f\vec{F}) = \nabla f \cdot \vec{F} + f\nabla\cdot \vec{F}$
            \item (Laplacian) $\Delta f = \nabla \cdot \nabla f = \partial_{xx}^2f + \partial_{yy}^2f + \partial_{zz}^2f$
        \end{enumerate}
    \end{prop}
    
    \begin{prop}{Properties of Divergence}{}
        If $\vec{F}:D\subset \R^3 \rightarrow \R^3$ and $\vec{G}:D\subset \R^3 \rightarrow \R^3$ vector fields and $f:D\subset \R^3 \rightarrow \R$ is a scalar field, that are all defined and differentiable in $D$. Let $C_1,C_2 \in \R$, then \begin{enumerate}
            \item (Linearity) $\nabla \times (C_1\vec{F} + C_2\vec{G}) = C_1\nabla \times \vec{F} + C_2\nabla \times \vec{F}$
            \item (Product rule) $\nabla \times (f\vec{F}) = \nabla f \times \vec{F} + f(\nabla\times \vec{F})$
            \item (Conservative Property) $\nabla \times (\nabla f) = 0$
            \item $\nabla \times(\nabla \times \vec{F}) = \nabla(\nabla\cdot \vec{F}) - (\nabla \cdot \nabla)\vec{F}$ (provided $\vec{F}$ has continuous second partial derivatives, and where $(\nabla \cdot \nabla)\vec{F} = (\Delta P, \Delta Q, \Delta R)$)
        \end{enumerate}
    \end{prop}
    
    \begin{defn}{}{}
        Let $E \subseteq \R^n$ \begin{enumerate}
            \item We say that $E$ is \Emph{path connected} if for any two points $A$ and $B$ in $E$ if there exists a continuous function $f:[0,1] \rightarrow E$ such that $f(0) = A$ and $f(1) = B$.
            \item We say $E$ is \Emph{simply connected} if $E$ is path connected and any simple closed curve $\mathcal{C}$ that completely lies in $E$ can be continuously deformed into a single point without leaving $E$.
        \end{enumerate}
    \end{defn}
    
    
    \begin{thm}{}{}
        Let $\vec{F}$ be a class $C^1$ in an open region $E \subseteq \R^3$. If $\vec{F}$ is conservative in $E$, then $\nabla \times \vec{F} = \vec{0}$ at every point of $E$. Moreover, if $\nabla \times \vec{F} = \vec{0}$ at every point of $E$ and $E$ is simply connected, then $\vec{F}$ is conservative in $E$.
    \end{thm}
    
    
    \subsection{Line Integrals}
    
    \begin{defn}{}{}
        Let $\mathcal{C}$ be a bounded continuous parametric curve in $\R^n$. Recall that $\mathcal{C}$ is a \Emph{smooth curve} if it has a parameterization $\vec{r}:I\subset \R\rightarrow \R^n$ such that $\frac{d\vec{r}}{dt}$ is continuous and nonzero in $I$. We say $\mathcal{C}$ is a \Emph{smooth arc} if it is a smooth curve with finite parameter interval $I = [a,b]$.  
    \end{defn}
    
    \begin{defn}{}{}
        Given a smooth curve $\mathcal{C}$ with parameterization $\vec{r}:[a,b]\rightarrow \R^n$ we have \begin{equation}
            l_{\mathcal{C}} = \int\limits_{\mathcal{C}}ds = \int\limits_a^b\left|\frac{d\vec{r}}{dt}\right|dt
        \end{equation}
        In general, for a scalar field $f:\R^n \rightarrow \R$, we define the \Emph{line integral along $\mathcal{C}$} to be \begin{equation}
            \int\limits_{\mathcal{C}}f(\vec{x})ds = \int\limits_a^bf(\vec{r}(t))\left|\frac{d\vec{r}}{dt}\right|dt
        \end{equation}
        This definition is parameterization independent.
    \end{defn}
    
    \begin{defn}{}{}
        If $\vec{F}$ is a continuous vector field and $\mathcal{C}$ is an oriented smooth curve, then the \Emph{line integral of the tangential component of $\vec{F}$ along $\mathcal{C}$} is \begin{equation}
            \int_{\mathcal{C}}\vec{F}\cdot d\vec{r} = \int_{\mathcal{C}}\vec{F}\cdot \hat{T}ds
        \end{equation}
    \end{defn}
    
    \begin{defn}{}{}
        If $\mathcal{C}$ is a closed curve we also call this line integral the \Emph{circulation} of $\vec{F}$ around $\mathcal{C}$, and we denote it by \begin{equation}
            \oint_{\mathcal{C}}\vec{F} \cdot d\vec{r}
        \end{equation}
    \end{defn}
    
    \begin{rmk}{}{}
        A line integral over a piecewise smooth path is the sum of the line integrals over the individual smooth arcs \begin{equation}
            \int\limits_{\bigcup_{i=1}^n\mathcal{C}_i}ds = \sum\limits_{i=1}^n\int\limits_{\mathcal{C}_i}ds
        \end{equation}
    \end{rmk}
    
    \subsection{Line Integral Theorems}
    
    \begin{namthm}{Fundamental Theorem of Line Integrals}
        Let $\mathcal{C}$ be a piecewise smooth parametric curve with initial point $A$ and terminal point $B$. If $f:D \subseteq \R^n \rightarrow \R$ is a scalar function with continuous first partial derivatives in an open region containing $\mathcal{C}$, then \begin{equation}
            \int_{\mathcal{C}}\nabla f\cdot d\vec{r} = f(B) - f(A)
        \end{equation} 
    \end{namthm}
    
    \begin{cor}{}{}
        If $\mathcal{C}$ is a piecewise smooth closed curve contained in a region $D$ where the vector field $\vec{F}:D\subseteq \R^n\rightarrow \R^n$ is conservative, then \begin{equation}
            \oint_{\mathcal{C}}\vec{F}\cdot d\vec{r}
        \end{equation}
    \end{cor}
    
    \begin{defn}{}{}
        A vector field $\vec{F}$ is said to be \Emph{path-independent} in a region $\Omega$ if for every pair of points $A$ and $B$ in $\Omega$ and every pair of piecewise smooth curves $\mathcal{C}_1$ and $\mathcal{C}_2$ with initial point $A$ and terminal point $B$, we have \begin{equation}
            \int_{\mathcal{C}_1}\vec{F}\cdot d\vec{r} = \int_{\mathcal{C}_2}\vec{F}\cdot d\vec{r}
        \end{equation}
    \end{defn}
    
    
    \begin{thm}{}{}
        Let $D$ be an open connected domain in $\R^n$ and let $\vec{F}$ be a smooth vector field defined on $D$. Then the following properties are equivalent:\begin{enumerate}
            \item $\vec{F}$ is conservative in $D$ 
            \item $\oint_{\mathcal{C}} \vec{F}\cdot d\vec{r} = 0$ for every piecewise smooth closed curve $\mathcal{C} \subset D$ 
            \item $\vec{F}$ is path independent in $D$.
        \end{enumerate}
    \end{thm}
    
    \begin{namthm}{Green's Theorem}
        Let $R$ be a closed region in the $xy$-plane whose boundary $\partial R$ consists of a finite number of piecewise smooth simple closed curves that are positively oriented with respect to $R$. If $\vec{F}$ is a smooth vector field on $R$, then \begin{equation}
            \oint_{\partial R}\vec{F}\cdot d\vec{r} = \int\int_{R}(\nabla \times \vec{F})\cdot \hat{k}dA
        \end{equation}
        In particular $\hat{k}$ is the unit normal field specifying the orientation of $R$, and $\partial R$ is oriented such that its principal normal field $\vec{N}$ points away from the region and \begin{equation}
            \hat{N} = \hat{T} \times \hat{k}
        \end{equation}
    \end{namthm}
    
    \begin{rmk}{}{}
        You don't need anything past this point yet.
    \end{rmk}
    
    \begin{namthm}{Plane Divergence Theorem}
        Let $R$ be a closed region in the $xy$-plane whose boundary $\partial R$ consists of a finite number of piecewise smooth simple closed curves. Let $\vec{N}$ denote the unit outward (from $R$) normal field on $\mathcal{C}$. If $\vec{F}$ is a smooth vector field on $R$, then \begin{equation}
            \oint_{\partial R}\vec{F}\cdot \hat{N} ds = \int\int_{R}\nabla \cdot \vec{F}dA
        \end{equation}
    \end{namthm}
    
    \begin{namthm}{Stoke's Theorem}
        Let $\mathcal{S}$ be a piecewise smooth oriented surface in $3$-space having a unit normal field $\hat{N}$ and boundary $\mathcal{C}$ consisting of a finite number of piecewise smooth closed curves with orientation inherited from $\mathcal{S}$. If $\vec{F}$ is a smooth vector field defined on an open set containing $\mathcal{S}$, then \begin{equation}
            \oint_{\mathcal{C}}\vec{F}\cdot d\vec{r} = \int\int_{\mathcal{S}}(\nabla \times \vec{F})\cdot \hat{N}dS
        \end{equation}
    \end{namthm}
    
    \subsection{Surface Integrals}
    
    \begin{defn}{}{}
        A \Emph{parametric surface} in $3$-space is a continuous function $\vec{r}: R \subseteq \R^2 \rightarrow \R^3$ for some rectangle $R$ \begin{equation}
            R = \{(u,v) \in \R^2: a \leq u \leq b, c \leq v \leq d\}
        \end{equation}
        in the $uv$-plane having values in $3$-space: \begin{equation}
            \vec{r}(u,v) = \langle x(u,v), y(u,v), z(u,v)\rangle, (u,v) \in R
        \end{equation}
    \end{defn}
    
    \begin{rmk}{}{}
        If $\vec{r}$ is one-to-one the surface does not intersect itself. In this case $\vec{r}$ maps the boundary of $R$ onto a curve in $3$-space called the \Emph{boundary of the parametric surface}. A surface with no boundary is called a \Emph{closed surface}.
    \end{rmk}
    
    \begin{defn}{}{}
        If a finite number of parametric surfaces are joined pairwise along their boundaries one obtains a \Emph{composite surface}, or just a surface thinking geometrically.
    \end{defn}
    
    \begin{defn}{}{}
        A set $S \subseteq\R^3$ is a \Emph{smooth surface} if any point $P \in S$ has a neighborhood $N$ that is the domain of a smooth function $g:N \rightarrow \R$ satisfying \begin{enumerate}
            \item $N \cap S = \{Q \in N:g(Q) = 0\}$
            \item $\nabla g(Q) \neq \vec{0}$ if $Q \in N \cap S$
        \end{enumerate}
    \end{defn}
    
    \begin{rmk}{}{}
        This means the surface has a unique tangent plane at any non-boundary point $P$.
    \end{rmk}
    
    \begin{defn}{}{}
        If $\vec{r}:D\subseteq 
        \R^2 \rightarrow \R^3$ is a parameterization of a smooth surface $S$, the \Emph{normal vector} to $S$ at $\vec{r}(u,v)$ is \begin{equation}
            \vec{n} = \frac{\partial \vec{r}}{\partial u}\times \frac{\partial \vec{r}}{\partial v}
        \end{equation}
    \end{defn}
    
    \begin{defn}{}{}
        The \Emph{area element} at $\vec{r}(u,v)$ on $S$ is given by \begin{equation}
            dS = \left|\frac{\partial \vec{r}}{\partial u}\times \frac{\partial \vec{r}}{\partial v}\right|dudv
        \end{equation}
        Then if $f(\vec{r})$ is continuous on $\mathcal{S}$ and the domain of $\vec{r}$ is $D$ in the $uv$-plane \begin{equation}
            \int\int_{\mathcal{S}}fdS = \int\int_{D}f(\vec{r}(u,v)) \left|\frac{\partial \vec{r}}{\partial u}\times \frac{\partial \vec{r}}{\partial v}\right|dudv
        \end{equation}
    \end{defn}
    
    \begin{defn}{}{}
        A smooth surface $\mathcal{S}$ in $3$-space is said to be \Emph{orientable} if there exists a unit vector field $\hat{N}$ defined on $\mathcal{S}$ that varies \Emph{continuously} over $\mathcal{S}$, and is everywhere normal to $\mathcal{S}$. Any such vector field $\hat{N}$ induces an orientation on $\mathcal{S}$. The side of $\mathcal{S}$ out of which $\hat{N}$ points is the \Emph{positive side}, and the other side is the \Emph{negative side}. An \Emph{oriented surface} is a smooth surface with a particular choice of orienting unit normal vector field $\hat{N}$.
    \end{defn}
    
    \begin{rmk}{}{}
        An oriented surface $\mathcal{S}$ \Emph{induces an orientation} on any of its boundary curves $\mathcal{C}$; if we stand on the positive side of the surface $\mathcal{S}$ and walk around $\mathcal{C}$ in the direction of its orientation, then $\mathcal{S}$ will be on our left side.
    \end{rmk}
    
    
    \begin{defn}{}{}
        Given any continuous vector field $\vec{F}$, the \Emph{flux} of $\vec{F}$ across the orientable surface $\mathcal{S}$ is the surface integral of the normal component of $\vec{F}$ over $\mathcal{S}$ \begin{equation}
            \int\int_{\mathcal{S}}\vec{F}\cdot \hat{N}dS = \int\int_{\mathcal{S}}\vec{F}\cdot d\vec{S}
        \end{equation}
        and when the surface is closed we write \begin{equation}
            \oiint_{\mathcal{S}}\vec{F}\cdot \hat{N}dS = \oiint_{\mathcal{S}}\vec{F}\cdot d\vec{S}
        \end{equation}
    \end{defn}
    
    
    \begin{rmk}{}{}
        If $\vec{r}(u,v)$ parametrizes $\mathcal{S}$ with domain $D$, we have normal \begin{equation}
            \vec{n} = \frac{\partial \vec{r}}{\partial u}\times \frac{\partial \vec{r}}{\partial v}
        \end{equation}
        and $dS = |\vec{n}|dudv$. Hence \begin{equation}
            d\vec{S} = \hat{N}dS = \pm\frac{\vec{n}}{|\vec{n}|}|\vec{n}|dudv = \pm\vec{n}dudv
        \end{equation}
        where the sign reflects the orientation of the surface and parameterization.
    \end{rmk}
    
    \begin{namthm}{Divergence Theorem}
        Let $D$ be a three dimensional domain bounded by piecewise smooth closed surfaces. Suppose its boundary $\mathcal{S}$ is an oriented closed surface with unit normal field $\hat{N}$ pointing out of $D$. If $\vec{F}$ is a smooth vector field defined on $D$, then \begin{equation}
            \oiint_{\mathcal{S}}\vec{F}\cdot \hat{N}dS = \int\int\int_D\nabla\cdot \vec{F}dV
        \end{equation}
    \end{namthm}
    
\end{appendices}


\end{document}


%%%%%% END %%%%%%%%%%%%%u
