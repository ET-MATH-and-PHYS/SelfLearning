%%%%%%%%%%%%%%%%%%%%% chapter.tex %%%%%%%%%%%%%%%%%%%%%%%%%%%%%%%%%
%
% sample chapter
%
% Use this file as a template for your own input.
%
%%%%%%%%%%%%%%%%%%%%%%%% Springer-Verlag %%%%%%%%%%%%%%%%%%%%%%%%%%
%\motto{Use the template \emph{chapter.tex} to style the various elements of your chapter content.}
\chapter{Special Functions}
\label{SpecFuncs} % Always give a unique label
% use \chaptermark{}
% to alter or adjust the chapter heading in the running head

%%%%%%%%%%%%%%%%%%%% Section 3.2.1
\section{The Gamma Function}

\begin{definition}
    The gamma function $\Gamma(z)$ is a meromorphic function defined on the right half-plane by \begin{equation*}
        \Gamma(z) = \int_0^{\infty}t^{z-1}e^{-t}dt,\;\;\;\;\mathscr{R}e(z) > 0
    \end{equation*}
    The integral defining $\Gamma(z)$ is absolutely convergent, and \begin{equation*}
        |\Gamma(x+iy)| \leq \Gamma(x) = \int_0^{\infty}t^{x-1}e^{-t}dt,\;\;\;\;x>0
    \end{equation*}
\end{definition}

Integrating by parts we observe that \begin{align*}
    \Gamma(z+1) &= \int_0^{\infty}e^{-t}t^zdt = -\int_0^{\infty}t^2d(e^{-t}) = -[t^2e^{-t}]_0^{\infty} + z\int_0^{\infty}e^{-t}t^{z-1}dt \\
    &= 0 + z\Gamma(z)  = z\Gamma(z),\;\;\;\;\mathscr{R}e(z) > 0
\end{align*}
Thus, we have \begin{equation*}
    \boxed{\Gamma(z+1) = z\Gamma(z),\;\;\;\;\mathscr{R}e(z) > 0}
\end{equation*}
Noting that $\Gamma(1) = 1$, and with the equation $\Gamma(2) = 1$, we obtain by induction \begin{equation*}
    \boxed{\Gamma(n+1) = n!,\;\;\;\;n\geq 0}
\end{equation*}
We use the functional equation above to extend $\Gamma(z)$ to the left half-plane as follows: apply the function equation $m$ times to obtain $\Gamma(z+m)=(z+m-1)...(z+1)z\Gamma(z)$, which we rewrite as \begin{equation*}
    \boxed{\Gamma(z) = \frac{\Gamma(z+m)}{(z+m-1)...(z+1)z}}
\end{equation*}
where the right hand side is defined and meromorphic for $\mathscr{R}e(z) > -m$, with simple poles $0,-1,...,-m+1$. By the uniqueness principle, the meromorphic extension is unique and it satisfies the functional equation. Passing to the limit as $m\rightarrow \infty$ we obtain:

\begin{theorem}
    The function $\Gamma(z)$ extends to be meromorphic on the entire complex plane, where it satisfies the functional equation $\Gamma(z+1)=z\Gamma(z)$. Its poles are simple poles at $z = 0,-1,-2,...$.
\end{theorem}

Now, define \begin{equation*}
    \Gamma_n(z) = \int_0^nt^{z-1}\left(1-\frac{t}{n}\right)^ndt,\;\;\;\;\mathscr{R}e(z) > 0, n\geq 1
\end{equation*}
Since $(1-t/n)^n \leq e^{-t}$ and $(1-t/n)^n \rightarrow e^{-t}$ for $t \geq 0$, we have \begin{equation*}
    \lim\limits_{m\rightarrow \infty}\Gamma_n(z) = \Gamma(z)
\end{equation*}

(To be continued)



%%%%%%%%%%%%%%%%%%%% Section 3.2.2
\section{Laplace Transforms}

\begin{definition}
    Let $h(s),s\geq0,$ be a continuous or piecewise continuous function on the positive real axis. The \Emph{Laplace transform of $h(s)$} is the function \begin{equation*}
        \mathcal{L}[h](z) = \int_0^{\infty}e^{-sz}h(s)ds
    \end{equation*}
    provided that the integral converges.
\end{definition}


\begin{proposition}
    If there are constants $B$ and $C$ such that $|h(s)| \leq Ce^{Bs}$ for $0 \leq s < \infty$, then the Laplace transform integral converges absolutely and defines an analytic function in the half plane $\mathscr{R}e(z) > B$. The estimate \begin{equation*}
        |\mathcal{L}[h](x+iy)| \leq C\int_0^{\infty}e^{-zs}e^{Bs}ds = \frac{C}{x-B},\;\;\;\; x > B
    \end{equation*}
    shows that $\mathcal{L}[h](z)$ is bounded on the half plane $\mathscr{R}e(z) > B+\varepsilon$ for any $\varepsilon > 0$, and $\mathcal{L}[h](z)\rightarrow 0$ as $\mathscr{R}e(z) \rightarrow \infty$.
\end{proposition}



