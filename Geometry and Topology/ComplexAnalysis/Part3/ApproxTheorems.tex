%%%%%%%%%%%%%%%%%%%%% chapter.tex %%%%%%%%%%%%%%%%%%%%%%%%%%%%%%%%%
%
% sample chapter
%
% Use this file as a template for your own input.
%
%%%%%%%%%%%%%%%%%%%%%%%% Springer-Verlag %%%%%%%%%%%%%%%%%%%%%%%%%%
%\motto{Use the template \emph{chapter.tex} to style the various elements of your chapter content.}
\chapter{Approximation Theorems}
\label{Approx} % Always give a unique label
% use \chaptermark{}
% to alter or adjust the chapter heading in the running head

%%%%%%%%%%%%%%%%%%%% Section 3.1.2
\section{The Mittag-Leffler Theorem}

Recall that if $f(z)$ is a meromorphic function with pole at $z_0$, and the Laurent expansion of $f(z)$ at $z_0$ is given by $\sum_{k=-m}^{\infty}a_k(z-z_0)^k$, then the principal part $P(z)$ of $f(z)$ at $z_0$ is the sum of the terms with negative powers, $P(z) = \sum_{k=-m}^{k=-1}a_k(z-z_0)^k$. Thus $P(z)$ is a polynomial in $1/(z-z_0)$, and $f(z)-P(z)$ is analytic at $z_0$. The Mittag-Leffler theorem asserts that we can prescribe the poles and principal parts of a meromorphic function.

\begin{theorem}[Mittag-Leffler Theorem]
    Let $D$ be a domain in the complex plane. Let $\{z_k\}$ be a sequence of distinct points in $D$ with no accumulation point in $D$, and let $P_k(z)$ be a polynomial in $1/(z-z_k)$. Then there is a meromorphic function $f(z)$ on $D$ whose poles are the points $z_k$, and such that $f(z) - P_k(z)$ is analytic at $z_k$.
\end{theorem}


\begin{example}
    Consider $P_k(z) = \frac{1}{z-k}$ for $k \in \N$. Then suppose we take \begin{equation*}
        f_{naive}(z) = \sum_{k=1}^{\infty}\frac{1}{z-k}
    \end{equation*}
    and consider the domain $D = \C$. But, then $f_{naive}(0) = -\sum_{k=1}^{\infty}\frac{1}{k} = -\infty$, which is to say it is not analytic at $0$. Instead, let's take \begin{equation*}
        f(z) = \sum_{k=1}^{\infty}\left(\frac{1}{z-k}+\frac{1}{k}\right)
    \end{equation*}
    and then $f(0) = 0$, and we have analyticity on bounded sets. In particular, it converges uniformly on bounded sets by comparison the series $\sum 1/k^2$. Indeed, if $|z| \leq R$ and $k > 2R$, then $|z-k| > k/2$, so the $k$th summand is bounded by $2/k^2$.
\end{example}

Let $w_1$ and $w_2$ be two complex numbers that do not lie on the same line through the origin. Then we can construct the \Emph{Weierstrass P-Function $\mathcal{P}(z)$} associated with $w_1$ and $w_2$. This is a doubly periodic meromorphic function on the complex plane.


\begin{example}
    We wish to show the following partial fractions decomposition identity \begin{equation*}
        \frac{\pi^2}{\sin^2(\pi z)} = \sum_{k=-\infty}^{\infty}\frac{1}{(z-k)^2}
    \end{equation*}
    (note this is meromorphic on $\C$, but not on $\C^*$ since it has an essential singularity at $\infty$). Observe $\pi^2\int\csc^2(\pi z)dz = -\pi\cot(\pi z)+C$, so \begin{equation*}
        \pi\cot(\pi z) = \frac{1}{z}+2\pi\sum_{k=1}^{\infty}\frac{1}{z^2-k^2}
    \end{equation*}
\end{example}



%%%%%%%%%%%%%%%%%%%% Section 3.1.3
\section{Infinite Products}

\begin{definition}
    An \Emph{infinite product} is an expression of the form $\prod_{j=1}^{\infty}p_j$, where the $p_j$'s are complex numbers. We say that the infinite product \Emph{converges} if $p_j\rightarrow 1$ and $\sum\text{Log}p_j$ converges, where we sum only over terms for which $p_j\neq 0$. If the infinite product converges, we define its value to be $0$ if one of the $p_j$'s is $0$; otherwise, we define it to be \begin{equation*}
        \prod_{j=1}^{\infty}p_j = \exp\left(\sum_{j=1}^{\infty}\text{Log}p_j\right)
    \end{equation*}
\end{definition}

If $\prod p_j$ converges, then at most finitely many of the $p_j$'s can be $0$. This is because $p_j\rightarrow 1$. Second, if $\prod p_j$ converges, then \begin{equation*}
    \prod_{j=1}^{\infty}p_j = \lim\limits_{m\rightarrow \infty}\prod_{j=1}^mp_j
\end{equation*}
This is because $\exp(\text{Log}p_j) = p_j$.

\begin{example}
    If $p_{42} = 0$ and $p_j = 1$ for all $j \neq 42$, then $\prod_{j=1}^{\infty}p_j = 0$.
\end{example}


\begin{example}
    If $p_1 = 1-\frac{1}{j^2}$ for all $j \geq 2$, then you can directly calculate the partial products. By induction we can show \begin{equation*}
        \prod_{j=2}^n(1-1/j^2) = \frac{1}{2}\left(1+\frac{1}{n}\right)
    \end{equation*}
    Then it follows that $\prod_{j=2}^{\infty}\left(1-\frac{1}{j^2}\right) = \frac{1}{2}$. 
\end{example}

\begin{example}
    The product $\prod_{j=2}^{\infty}(1-1/j)$ is divergent despite the fact $1-\frac{1}{j}\rightarrow 1$, since $\sum_{j=2}^{\infty}\text{Log}(1-1/j)$ diverges to $-\infty$. Formally, the product diverges to $0$ since $\exp(-\infty) = 0$.
\end{example}

\begin{example}
    Consider the infinite product formed by $p_k = 1+\frac{(-1)^{k+1}}{k}$, $k \geq 1$, so \begin{align*}
        \prod_{j=1}^{\infty}\left(1+\frac{(-1)^{j+1}}{j}\right) &= (1+1/1)(1-1/2)(1+1/3)(1-1/4)... \\
        &= \lim\limits_{k\rightarrow \infty}\left[2\cdot \frac{1}{2}\cdot\frac{4}{3}\cdot \frac{3}{4}\cdot ...\cdot \left(1+\frac{(-1)^{k+1}}{k}\right)\right]
    \end{align*}
    Evidently the product of evenly many terms is $1$, so we see the product tends to $1$ as $k\rightarrow \infty$.
\end{example}

\begin{theorem}
    If $t_j \geq 0$, then $\prod(1+t_j)$ converges if and only if $\sum t_j$ converges.
\end{theorem}

\begin{definition}
    The infinite product $\prod(1+a_j)$ is said to \Emph{converge absolutely} if $a_j\rightarrow 0$ and $\sum\text{Log}(1+a_j)$ converges absolutely, where we sum over the terms for which $a_j \neq -1$. If $\prod(1+a_j)$ converges absolutely, then $\sum\text{Log}(1+a_j)$ converges and $\prod(1+a_j)$ converges.
\end{definition}

\begin{theorem}
    The infinite product $\prod(1+a_j)$ converges absolutely if and only if $\sum a_j$ converges absolutely. This occurs if and only if $\prod(1+|a_j|)$ converges.
\end{theorem}


\begin{theorem}
    Suppose that $g_k(x) = 1+h_k(x)$, $k \geq 1$, are functions on a set $E$. Suppose that there are constants $M_k > 0$ such that $\sum M_k < \infty$, and $|h_k(x)| \leq M_k$ for $x \in E$. Then $\prod_{k=1}^mg_k(x)$ converges to $\prod_k^{\infty}g_k(x)$ uniformly on $E$ as $m\rightarrow \infty$.
\end{theorem}
\begin{proof}
    (to be completed)
\end{proof}

If $G(z) = g_1(z)...g_m(z)$ is a finite product of analytic functions, then by taking logarithms and differentiating we obtain\begin{equation*}
    \frac{G'(z)}{G(z)} = \frac{g_1'(z)}{g_1(z)} + ... + \frac{g'_m(z)}{g_m(z)}
\end{equation*}
This procedure is called \Emph{logarithmic differentiation}.

\begin{theorem}
    Let $g_k(z), k \geq 1$, be analytic functions on a domain $D$ such that $\prod_{k=1}^mg_k(z)$ converges normally on $D$ to $G(z) = \prod_{k=1}^{\infty}g_k(z)$. Then \begin{equation*}
        \frac{G'(z)}{G(z)} = \sum_{k=1}^{\infty}\frac{g_k'(z)}{g_k(z)},\;\;\;\;z \in D
    \end{equation*}
    where the sum converges normally on $D$.
\end{theorem}
\begin{proof}
    (To be completed)
\end{proof}



%%%%%%%%%%%%%%%%%%%% Section 3.1.4
\section{The Weierstrass Product Theorem}

This is a companion to the Mittag-Leffler theorem, which asserts that we can prescribe the poles and principal parts of a meromorphic function, while the Weierstrass product theorem asserts that we can prescribe the zeros and poles of a meromorphic function together with their orders.

\begin{theorem}[Weierstrass Product Theorem]
    Let $D$ be a domain in the complex plane. Let $\{z_k\}$ be a sequence of distinct points of $D$ with no accumulation point in $D$, and let $\{n_k\}$ be a sequence of integers (positive or negative). Then there is a meromorphic function $f(z)$ on $D$ whose only zeros and poles are at the points $z_k$ such that the order of $f(z)$ at $z_k$ is $n_k$.
\end{theorem}
\begin{proof}
    (To be completed)
\end{proof}

\begin{example}
    Consider $\prod_{k=1}^{\infty}\left(1-\frac{z^2}{k^2}\right)$. Write $g_k(z) = 1-\frac{z^2}{k^2}$ and $h_k(z) = \frac{-z^2}{k^2}$. Suppose $|z| \leq R$ then $\left|\frac{z^2}{k^2}\right| \leq \frac{R^2}{k^2}$. Observe $\sum_{k=1}^{\infty}\frac{R^2}{k^2} = \frac{\pi R^2}{6}$ hence we find the product is uniformly convergent for $|z| < R$. But as $R$ is arbitrary we find the product $\prod_{k=1}^{m}\left(1-\frac{z^2}{k^2}\right)$ converges uniformly to $\prod_{k=1}^{\infty}\left(1-\frac{z^2}{k^2}\right)$ on $\C$.
\end{example}





