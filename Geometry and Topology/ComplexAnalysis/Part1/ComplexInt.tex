%%%%%%%%%%%%%%%%%%%%% chapter.tex %%%%%%%%%%%%%%%%%%%%%%%%%%%%%%%%%
%
% sample chapter
%
% Use this file as a template for your own input.
%
%%%%%%%%%%%%%%%%%%%%%%%% Springer-Verlag %%%%%%%%%%%%%%%%%%%%%%%%%%
%\motto{Use the template \emph{chapter.tex} to style the various elements of your chapter content.}
\chapter{Complex Integration}
\label{CompInt} % Always give a unique label
% use \chaptermark{}
% to alter or adjust the chapter heading in the running head

%%%%%%%%%%%%%%%%%%%% Section 1.4.1
\section{Complex Line Integrals}

Let $h(z)$ be a complex valued function on a curve $\gamma$, then defining $dz = dx+idy$ we define \begin{equation*}
    \int_{\gamma}h(z)dz = \int_{\gamma}h(z)dx + i\int_{\gamma}h(z)dy
\end{equation*}
Suppose $\gamma$ is parameterized by $t \mapsto z(t) = x(t)+iy(t)$, $a \leq t \leq b$. The Riemann sum approximating $\int_{\gamma}h(z)(dx+idy)$ corresponding to the subdivision $a = t_0 < t_1 < ... < t_n = b$ is given by \begin{equation*}
    \sum_{i=1}^nh(z_i)\delta x_i+i\sum_{i=1}^nh(z_i)\delta y_i
\end{equation*}
where $z(t_j) = z_j = x_j + iy_j$. Then we obtain \begin{equation*}
    \int_{\gamma}h(z)dz = \lim\limits_{n\rightarrow \infty}\sum_{i=1}^nh(z_i)\delta z_i
\end{equation*}


Recall that if $\gamma$ is a smooth path in $\C$, then for a analytic complex valued function $f:D\rightarrow \C$, \begin{equation*}
    \frac{d}{dt}\left[f(\gamma(t))\right] = \frac{df}{dz}(\gamma(t))\frac{d\gamma}{dt}(t)
\end{equation*}
Then we have $\frac{d}{dt}(e^{\lambda t}) = \lambda e^{\lambda t}$ and \begin{equation*}
    \int e^{\lambda t}dt = \frac{1}{\lambda}e^{\lambda t}+C
\end{equation*}
Thus, if $\lambda = \alpha+i\beta$, $e^{\lambda t} = e^{\alpha t}\cos(\beta t)+ie^{\alpha t}\sin(\beta t)$, so this says \begin{align*}
    \int e^{\alpha t}\cos(\beta t)dt + i\int e^{\alpha t}\sin(\beta t)dt &= \frac{1}{\alpha+i\beta}(e^{\alpha t}\cos(\beta t)+ie^{\alpha t}\sin(\beta t)) + C \\
    &= \frac{\alpha-i\beta}{\alpha^2+\beta^2}(e^{\alpha t}\cos(\beta t)+ie^{\alpha t}\sin(\beta t)) + C \\
    &= \frac{e^{\alpha t}}{\alpha^2+\beta^2}[\alpha\cos(\beta t) + \beta\sin(\beta t) + i(\alpha\sin(\beta t) - \beta\cos(\beta t))] + C
\end{align*}
equating the real and complex terms we have \begin{equation*}
    \int e^{\alpha t}\cos(\beta t)dt = \frac{e^{\alpha t}}{\alpha^2+\beta^2}[\alpha\cos(\beta t) + \beta\sin(\beta t)] + C
\end{equation*}
and \begin{equation*}
    \int e^{\alpha t}\sin(\beta t)dt = \frac{e^{\alpha t}}{\alpha^2+\beta^2}[\alpha\sin(\beta t) - \beta\cos(\beta t)] + C
\end{equation*}

\begin{definition}
    We can rewrite our line integral as follows for $f = u+iv$: \begin{equation*}
        \int_{\gamma}f(z)dz = \int_{\gamma}(u+iv)(dx+idy) = \int_{\gamma}(udx-vdy)+i\int_{\gamma}(vdx+udy)
    \end{equation*}
\end{definition}

for $\gamma$ a parameterization we have \begin{equation*}
    \int_{\gamma}f(z)dz = \int_{\gamma}f(z)\frac{d\gamma}{dt}dt
\end{equation*}

\begin{example}
    Consider the ccw-oriented unit circle, and the following integral taken along it: \begin{equation*}
        \int_{C}\frac{dz}{z} = \int_{0}^{2\pi}\frac{ie^{it}dt}{e^{it}} = \int_{0}^{2\pi}idt = i2\pi
    \end{equation*}
    using the parameterization $z = e^{it}$, so $dz = ie^{it}dt$.
\end{example}

\begin{example}
    Consider the circle centered at $z_0$ with radius $R$, so we have ccw parameterization $z(t) = z_0 + Re^{it}$, $0 \leq t \leq 2\pi$: \begin{align*}
        \int_C(z-z_0)^ndz &= \int_{0}^{2\pi}(Re^{it})^n(iRe^{it})dt = \int_{0}^{2\pi}iR^{n+1}e^{i(n+1)t}dt \\
        &= \frac{iR^{n+1}}{i(n+1)}e^{i(n+1)t}\Bigg\rvert_0^{2\pi} \\
        &= 0
    \end{align*}
    for $n \in \Z$ and $n \neq -1$ (so that it is single valued on the whole complex plane). If $n = -1$ then the integral is $2\pi i$. 
\end{example}


\begin{remark}
    If $\omega$ is a closed differential form an a region between and containing two simple closed curves, $\gamma_1$ and $\gamma_2$, then one curve can be deformed into the other such that \begin{equation*}
        \int_{\gamma_1}\omega = \int_{\gamma_2}\omega
    \end{equation*}
\end{remark}
This implies that in our previous example, for $n = -1$, for any simple closed curve encircling the origin we obtain $2\pi i$ by integrating the function along it.

\begin{definition}
    For a smooth curve $\gamma$ and a complex valued function $f$ we define \begin{equation*}
        \int_{\gamma}f(z)|dz| = \int_{t_0}^{t_1}f(\gamma(t))\left|\frac{d\gamma}{dt}\right|dt
    \end{equation*}
    where we often write $|dz| = ds$, which gives the arclength integrals \begin{equation*}
        \int_{\gamma}f(z)|dz| = \int_{\gamma}uds + i\int_{\gamma}vds
    \end{equation*}
    for $f = u+iv$.
\end{definition}



\begin{theorem}[ML Theorem]
    Suppose $\gamma$ is a piecewise smooth curve. If $h(z)$ is a continuous function on $\gamma$, then \begin{equation*}
        \left|\int_{\gamma}h(z)dz\right|\leq \int_{\gamma}|h(z)||dz|
    \end{equation*}
    Further, if $\gamma$ has length $\int_{\gamma}|dz| = L$, and $|h(z)| \leq M$ on $\gamma$, then \begin{equation*}
        \left|\int_{\gamma}h(z)dz\right| \leq ML
    \end{equation*}
\end{theorem}


\begin{example}
    Consider $f(z) = \frac{1}{z}$ on $|z| = 1$. Then $|f(z)| = \frac{1}{|z|} = 1$, which we can use for $M$. Moreover, the length is $L = 2\pi$. Thus, we have by the ML theorem that \begin{equation*}
        \left|\int_C \frac{dz}{z}\right| \leq 2\pi
    \end{equation*}
\end{example}


\begin{example}
    Consider \begin{equation*}
        \int_{|z| = R}\frac{dz}{z^2-6}
    \end{equation*}
    Observe $|z^2 - 6| \geq |z^2| - |6| = R^2-6$, so assuming $R > 3$, $\frac{1}{z^2-6} \leq \frac{1}{R^2-6} = M$, and $L = 2\pi R$. Thus, by the ML theorem \begin{equation*}
        \left|\oint_{|z| = R}\frac{dz}{z^2-6}\right| \leq \frac{2\pi R}{R^2-6}
    \end{equation*}
\end{example}



%%%%%%%%%%%%%%%%%%%% Section 1.4.2
\section{Fundamental Theorem of Calculus for Analytic Functions}

\begin{definition}
    Let $f(z)$ be a continuous function on a domain $D$. A function $F(z)$ on $D$ is a \Emph{(complex) primitive} for $f(z)$ if $F(z)$ is analytic and $F'(z) = f(z)$ for all $z \in D$.
\end{definition}

First, observe that if $F'(z) = f(z)$ for all $z \in D$, then \begin{equation*}
    \int_{\gamma}f(z)dz = \int_{\gamma}F'(z)dz = \int_{\gamma}dF = F(\gamma(t_1)) - F(\gamma(t_0))
\end{equation*}
as $dF = F'(z)dx+iF'(z)dy = F'(z)dz$, and of course $dF$ is an exact form so it is path independent. 

\begin{theorem}
    If $f(z)$ is continuous on a domain $D$ with primitive $F(z)$, then \begin{equation*}
        \int_A^Bf(z)dz = F(A) - F(B)
    \end{equation*}
    where the integral can be taken over any path in $D$ from $A$ to $B$.
\end{theorem}

\begin{remark}
    $f(z) = \frac{1}{z}$ cannot have a primitive on $\C^{\times}$. Indeed, if it did it would be path independent, and hence its integral would be zero over all closed loops.
\end{remark}

\begin{theorem}
    Let $D$ be a star shaped domain, and let $f(z)$ be holomorphic on $D$. Then $f(z)$ has a primitive on $D$, and the primitive is unique up to a constant. A primitive for $f(z)$ is given explicitly by \begin{equation*}
        F(z) = \int_{z_0}^{z}f(w)dw
    \end{equation*}
    for $z \in D$, where $z_0$ is any fixed point in $D$ for which the integral can be taken in $D$.
\end{theorem}


\begin{example}
    Consider $\int \frac{dz}{z}$ for parameterization $z = e^{it}$, $-\pi + \varepsilon \leq t \leq \pi - \varepsilon$, where $\varepsilon > 0$. Then \begin{equation*}
        \int\frac{dz}{z} = \text{Log}(z)\vert_{e^{i(\varepsilon-\pi)}}^{e^{i(\pi-\varepsilon)}} = i(\pi-\varepsilon) - i(\varepsilon - \pi) = 2i(\pi-\varepsilon)
    \end{equation*}
\end{example}


%%%%%%%%%%%%%%%%%%%% Section 1.4.3
\section{Cauchy's Theorem and Integral Formula}


Let $f(z) = u+iv$ be a smooth complex valued function, and we express $f(z)dz = (u+iv)(dx+idy) = (u+iv)dx+(-v+iu)dy$. The condition that $f(z)dz$ is a closed differential is \begin{equation*}
    \frac{\partial}{\partial y}(u+iv) = \frac{\partial}{\partial x}(-v+iu)
\end{equation*}
so equating real and imaginary components we obtain $u_y = -v_x$ and $v_y = u_x$, in other words the Cauchy Riemann equations. 

\begin{theorem}
    A continuously differentiable function $f(z)$ on $D$ is analytic if and only if the differential $f(z)dz$ is closed.
\end{theorem}

From Green's theorem we then obtain the following:

\begin{theorem}[Cauchy's Theorem]
    Let $D$ be a bounded domain with piecewise smooth boundary. If $f(z)$ is holomorphic and continuously differentiable on $D$ that extends continuously to $\partial D$, then \begin{equation*}
        \int_{\partial D}f(z)dz = 0
    \end{equation*}
\end{theorem}

Extending continuously to the boundary of $D$ means we can extend to a slightly bigger domain which contains the boundary and in which the hypotheses hold.


\begin{theorem}[Cauchy's Integral Formula]
    Let $D$ be a bounded domain with piecewise smooth boundary. If $f(z)$ is holomorphic on $D$, and $f(z)$ extends smoothly to the boundary $\partial D$, then for each $z \in D$ \begin{equation*}
        f(z) = \frac{1}{2\pi i}\int_{\partial D}\frac{f(w)}{w-z}dw,
    \end{equation*}
\end{theorem}
\begin{proof}
    Assume the conditions of the theorem. Let $z \in D$, and since $D$ is open there exists $\varepsilon > 0$ such that $\{w \in \C: |z-w| < \varepsilon\} \subseteq D$. Then, define $D_{\varepsilon} = D\backslash \{w \in \C:|z-w| \leq \varepsilon\}$. The boundary $\partial D_{\varepsilon}$ is the union $\partial D\cup\{w \in \C:|w-z|=\varepsilon\}$, with the circle being clockwise oriented. Since $f(z)/(w-z)$ is analytic for $z \in D_{\varepsilon}$, Cauchy's theorem yields: \begin{equation*}
        \int_{\partial D_{\varepsilon}}\frac{f(w)}{w-z}dw = 0
    \end{equation*}
    Reversing the orientation of the circle to counter clockwise produces a sign change, which gives \begin{equation*}
        0 = \int_{\partial D}\frac{f(w)}{w-z}dw - \int_{|w-z| = \varepsilon, ccw}\frac{f(w)}{w-z}dw
    \end{equation*}
    so \begin{equation*}
        \int_{|w-z|=\varepsilon,ccw} \frac{f(w)}{w-z}dw = \int_{\partial D}\frac{f(w)}{w-z}dw
    \end{equation*}
    Then we have parameterization $w = z+\varepsilon e^{it}$ for the circle, $0 \leq t \leq 2\pi$. Then observe \begin{equation*}
        \int_{0}^{2\pi}\frac{f(z+\varepsilon e^{it})}{\varepsilon e^{it}}\varepsilon ie^{it}dt = 2\pi i\int_0^{2\pi}f(z+\varepsilon e^{it})\frac{dt}{2\pi}
    \end{equation*}
    But, then by the mean value property of analytic functions we have that the integral on the right hand side coincides with $2\pi if(z)$, so \begin{equation*}
        f(z) = \frac{1}{2\pi i}\int_{\partial D}\frac{f(w)}{w-z}
    \end{equation*}
\end{proof}

If we differentiate under the integral sign and use \begin{equation*}
    \frac{d^m}{dz^m}\frac{1}{w-z} = \frac{m!}{(w-z)^{m+1}}
\end{equation*}
we obtain integral formulae for $f^{(m)}(z)$ of $f(z)$.

\begin{theorem}[Cauchy Integral Formula (general)]
    Let $D$ be a bounded domain with piecewise smooth boundary. If $f(z)$ is holomorphic on $D$, and $f(z)$ extends smoothly to the boundary $\partial D$, then $f(z)$ has complex derivatives of all orders on $D$, which are given for each $z \in D$ by \begin{equation*}
        f^{(m)}(z) = \frac{m!}{2\pi i}\int_{\partial D}\frac{f(w)}{(w-z)^{m+1}}dw,
    \end{equation*}
    for $m \geq 0$.
\end{theorem}

\begin{corollary}
    If $f(z)$ is analytic on a domain $D$, then $f(z)$ is infinitely differentiable, and the successive complex derivatives $f'(z),f''(z),...,$ are all analytic on $D$.
\end{corollary}


\begin{example}
    Consider \begin{align*}
        \oint_{|z| = 2}\frac{\sin(2i)}{(z-i)^6}dz &= \frac{2\pi i}{5!}\frac{d^5}{dz^5}\Bigg\rvert_{z=i}\sin(2z) \\
        &= \frac{2^6\pi i}{5!}\cos(2i) \\
        &= \frac{8\pi i\cos(2i)}{15} = \frac{8\pi i\cosh(2)}{15}
    \end{align*}
\end{example}


We usually use the Cauchy integral formula in the form \begin{equation*}
    \int_{\partial D}\frac{f(w)}{(w-z_0)^{m+1}}dw = \frac{2\pi if^{(m)}(z_0)}{m!}
\end{equation*}

\begin{example}
    Consider a domain without $z_0$, so by Cauchy's Theorem \begin{align*}
        \int_{\partial D}\frac{\sin(z^2)dz}{(z-z_0)^2} = 0 
    \end{align*}
    If $z_0 \in D$, then \begin{equation*}
        \int_{\partial D}\frac{\sin(z^2)dz}{(z-z_0)^2} = 2\pi i\frac{d}{dz}\Bigg\rvert_{z=z_0}\sin(z^2) = 2\pi i2z_0\cos(z_0^2) = 4\pi iz_0\cos(z_0^2)
    \end{equation*}
\end{example}


\begin{example}
    Consider the circle $C= \{|z+i| = 1/2\}$, and we find \begin{align*}
        \int_C\frac{dz}{z^4+1} &= \int_C\frac{dz}{(z-e^{i\pi/4})(z-e^{3i\pi/4})(z-e^{5i\pi/4})(z-e^{7i\pi/4})} \\
        &= 0
    \end{align*}
    since none of the roots are contained in the inside of $C$, so the integrand is holomorphic on the disk contained in the circle. If we consider $C= \{|z+i| = 1\}$, we have two singularities inside the circle. 
\end{example}



%%%%%%%%%%%%%%%%%%%% Section 1.4.4
\section{Liouville's Theorem}


Suppose that $f(z)$ is holomorphic on some domain containing the disk $\{z \in \C:|z-z_0| \leq \rho\}$. Then by Cauchy's integral formula \begin{equation*}
    f^{(m)}(z_0) = \frac{m!}{2\pi i}\int_{|z-z_0|=\rho}\frac{f(z)}{(z-z_0)^{m+1}}dz
\end{equation*}
We parametrize the boundary circle by $z = z_0 + \rho e^{i\theta}$, $dz = i\rho e^{i\theta}d\theta$. Then \begin{equation*}
    \frac{1}{2\pi i}\frac{f(z)}{(z-z_0)^{m+1}}dz = \frac{f(z_0+\rho e^{i\theta}}{\rho^me^{im\theta}}\frac{d\theta}{2\pi}
\end{equation*}
and we obtain \begin{equation*}
    f^{(m)}(z_0) = \frac{m!}{\rho^m}\int_0^{2\pi}f(z_0+\rho e^{i\theta})e^{-im\theta}\frac{d\theta}{2\pi}
\end{equation*}
This gives the estimate \begin{equation*}
    \left|f^{(m)}(z_0)\right| \leq \frac{m!}{\rho^m}\int_{0}^{2\pi}|f(z_0+\rho e^{i\theta})|\frac{d\theta}{2\pi}
\end{equation*}
which then leads to: 

\begin{theorem}[Cauchy Estimates]
    Suppose $f(z)$ is analytic for $|z-z_0| \leq \rho$. If $|f(z)| \leq M$ for $|z - z_0| = \rho$, then \begin{equation*}
        \left|f^{(m)}(z_0)\right| \leq \frac{m!}{\rho^m}M, m \geq 0 
    \end{equation*}
\end{theorem}

This comes in part by the ML Theorem and the Maximum Theorem.

\begin{theorem}[Liouville's Theorem]
    Let $f(z)$ be an analytic function on the complex plane. If $f(z)$ is bounded, then $f(z)$ is constant.
\end{theorem}
\begin{proof}
    Assume $f(z)$ and $f'(z)$ are continuous on $\C$ and $|f(z)| \leq M$ for all $z \in \C$. Let us consider the disk of radius $R$ centered at $z_0$. From Cauchy's Estimate with $m = 1$ we obtain \begin{equation*}
        |f'(z_0)| \leq \frac{M}{R}
    \end{equation*}
    Observe, as $R$ goes to infinity $|f'(z_0)|$ goes to $0$. so $f'(z_0) = 0$. But $z_0$ was an arbitrary point in $\C$, hence $f'(z) = 0$ for all $z \in \C$ and as $\C$ is a connected domain we find $f(z) = c$ for some $c \in \C$, and for all $z \in \C$.
\end{proof}

\begin{definition}
    We define an \Emph{entire function} to be a function that is analytic on the entire complex plane.
\end{definition}


\begin{example}
    Suppose we want to calculate \begin{equation*}
        \oint_C\frac{\cos(z)}{z^2+4}dz
    \end{equation*}
    Note that the integrand is singular when $z= 2e^{i \pi/2}e^{i\pi k/2}$, so $z = \pm 2i$ gives the singularities. Let us rewrite \begin{equation*}
        \frac{1}{z^2+4} = \frac{a}{z+2i}+\frac{b}{z-2i} 
    \end{equation*}
    Then we have $1 = a(z-2i)+b(z+2i)$, setting $z = 2i$ we get $b = \frac{1}{4i}$ and $a = \frac{-1}{4i}$. Then, we can write the integral as \begin{equation*}
        \oint_C\frac{-\cos(z)}{4i(z+2i)}dz + \oint_C\frac{\cos(z)}{4i(z-2i)}dz
    \end{equation*}
    From Cauchy's theorem and Cauchy's Integral Formula, we have four possibilities. If neither singularity is in $C$ the whole integral is $0$, if $-2i$ is in $C$ but $2i$ is not we have $$2\pi i(-\cos(-2i))/4i = -\frac{\pi\cos(2i)}{2} = -\frac{\pi\cosh(2)}{2}$$ if $2i$ is in $C$ but $-2i$ is not we have \begin{equation*}
        2\pi i(\cos(2i))/4i = \frac{\pi\cosh(2)}{2}
    \end{equation*}
    and finally if both singularities are in $C$ the integral is $0$, since the sum of the two component integrals are $0$.
\end{example}


%%%%%%%%%%%%%%%%%%%% Section 1.4.5
\section{Morera's Theorem}


Recall that we observed that $f(z)$ is analytic on $D$ if and only if $f(z)dz$.

\begin{theorem}[Morera's Theorem] \label{namthm:morera}
    Let $f(z)$ be a continuous function on a domain $D$. If \begin{equation*}
        \int_{\partial R}f(z)dz = 0
    \end{equation*}
    for every closed rectangle $R$ contained in $D$ with sides parallel to the coordinate axes, then $f(z)$ is analytic on $D$.
\end{theorem}
\begin{proof}
    Without loss of generality suppose $D$ is a disk with center $z_0$. Define \begin{equation*}
        F(z) = \int_{z_0}^zf(\xi)d\xi, z \in D
    \end{equation*}
    where the path of integration runs along a horizontal line and then a vertical line. We could as well define $F(z)$ using the path starting from $z_0$ along a vertical line followed by a horizontal line. By hypothesis these two paths yield the same integral. Now we differentiate $F(z)$ by hand. We have \begin{equation*}
        F(z+\delta z) - F(z) = \int_z^{z+\delta z}f(w)dw
    \end{equation*}
    where the path of integration is the path from $z$ to $z+\delta z$ following a horizontal line then a vertical line. Since we are fixing $z$, the value of $f(z)$ is constant for the integration and we obtain \begin{align*}
        F(z+\delta z)-F(z)&= f(z)\int_z^{z+\delta z}dw+\int_z^{z+\delta z}(f(w)-f(z))dw \\
        &= f(z)\delta z + \int_z^{z+\delta z}(f(w)-f(z))dw
    \end{align*}
    Now, the length of the contour from $z$ to $z+\delta z$ is at most $|2\delta z|$. If we divide by $\delta z$ and use the ML-estimate on the last integral, we obtain \begin{equation*}
        \left|\frac{F(z+\delta z)-F(z)}{\delta z} - f(z) \right| \leq 2M_{\varepsilon}
    \end{equation*}
    where $|\delta z| < \varepsilon$ and $M_{\varepsilon}$ is the maximum of $|f(w)-f(z)|$ over all of $w$ satisfying $|w-z| \leq \varepsilon$. Since $f(z)$ is continuous at $z$, $M_{\varepsilon}$ to $0$ as $\varepsilon$ goes to $0$. Consequently, $F(z)$ is complex differentiable with complex derivative $F'(z) = f(z)$. Since $f(z)$ is continuous, $F(z)$ is analytic, and since $f(z)$ is the derivative of an analytic function, $f(z)$ is analytic.
\end{proof}



%%%%%%%%%%%%%%%%%%%% Section 1.4.6
\section{Goursat's Theorem}

We defined $f(z)$ to be analytic on a domain $D$ if the complex derivative $f'(z)$ exists at each point of $D$ and further, $f'(z)$ is a continuous function of $z$. Goursat's theorem asserts that the continuity requirement is redundant:

\begin{theorem}[Goursat's Theorem]
    If $f(z)$ is a complex-valued function on a domain $D$ such that \begin{equation*}
        f'(z_0) = \lim\limits_{z\rightarrow z_0}\frac{f(z)-f(z_0)}{z-z_0}
    \end{equation*}
    exists at each point $z_0 \in D$, then $z\mapsto f'(z)$ is continuous on $D$.
\end{theorem}
\begin{proof}
    Let $R$ be a closed rectangle in $D$. We subdivide $R$ into four identical subrectangles. Since the integral f $f(z)$ around $\partial R$ is the sum of the integrals of $f(z)$ around the four subrectangles, there is at least one of the subrectangles, call it $R_1$, for which \begin{equation*}
        \left|\int_{\partial R_1}f(z)dz\right| \geq \frac{1}{4}\left|\int_{\partial R}f(z)dz\right|
    \end{equation*}
    Now subdivide $R_1$ into four equal subrectangles and repeat the process. This yields a nested sequence of integrals \begin{equation*}
        \left|\int_{\partial R_n}f(z)dz\right| \geq \frac{1}{4}\left|\int_{\partial R_{n-1}}f(z)dz\right| \geq ... \geq \frac{1}{4^n}\left|\int_{\partial R}f(z)dz\right|
    \end{equation*}
    Since the $R_n$'s are decreasing and have diameters tending to $0$, the $R_n$'s converge to some point $z_0 \in D$. Since $f(z)$ is differentiable at $z_0$, we have an estimate of the form \begin{equation*}
        \left|\frac{f(z)-f(z_0)}{z-z_0} - f'(z_0)\right| \leq \varepsilon_n, z \in R_n
    \end{equation*}
    where $\varepsilon_n\rightarrow 0$ as $n\rightarrow \infty$. Let $L$ be the length of $\partial R$. Then the length of $\partial R_n$ is $L/2^n$. For $z$ belonging to $R_n$ we have the estimate \begin{equation*}
        |f(z) - f(z_0) - f'(z_0)(z-z_0)| \leq \varepsilon_n|z-z_0| \leq 2\varepsilon_nL/2^n
    \end{equation*}
    By the ML estimate and Cauchy's theorem we have \begin{align*}
        \left|\int_{\partial R_n}f(z)dz\right| &= \left|\int_{\partial R_n}[f(z)-f(z_0)-f'(z_0)(z-z_0)]dz\right| \\
        &\leq (2\varepsilon_nL/2^n)(L/2^n) = 2L^2\varepsilon_n/4^n
    \end{align*}
    where we subtracted the zero $\int_{\partial R_n}(f(z_0)+f'(z_0)(z-z_0))dz =0$, using Cauchy's theorem. Hence, \begin{equation*}
        \left|\int_{\partial R}f(z)dz\right| \leq 4^n\left|\int_{\partial R_n}f(z)dz\right| \leq 2L^2\varepsilon_n
    \end{equation*}
    Since $\varepsilon_n\rightarrow 0$ as $n\rightarrow \infty$, we must have \begin{equation*}
        \int_{\partial R}f(z)dz = 0
    \end{equation*}
    By Morera's Theorem (\ref{namthm:Morera}), $f(z)$ is analytic.
\end{proof}



%%%%%%%%%%%%%%%%%%%% Section 1.4.7
\section{Pompeiu's Formulas}


Using the fact that for $z = x+iy$, $x = (z+\overline{z})/2$ and $y = -i(z-\overline{z})/2$, we define two particular complex differential operators as follows: \begin{align*}
    \frac{\partial}{\partial z} &= \frac{\partial x}{\partial z}\frac{\partial}{\partial x}+\frac{\partial y}{\partial z}\frac{\partial}{\partial y} = \frac{1}{2}\left[\frac{\partial}{\partial x}-i\frac{\partial}{\partial y}\right] \\
    \frac{\partial}{\partial \overline{z}} &= \frac{\partial x}{\partial \overline{z}}\frac{\partial}{\partial x}+\frac{\partial y}{\partial \overline{z}}\frac{\partial}{\partial y} = \frac{1}{2}\left[\frac{\partial}{\partial x}+i\frac{\partial}{\partial y}\right]
\end{align*}
We may think of $\frac{\partial f}{\partial z}$ as an average of the derivatives of $f$ in the $x$ and $iy$ directions: \begin{equation*}
    \frac{\partial f}{\partial z} = \frac{1}{2}\left[\frac{\partial f}{\partial x}+\frac{\partial f}{\partial(iy)}\right]
\end{equation*}
Recall that from our work with Cauchy-Riemann equations we have the formulas $f'(z) = \frac{\partial f}{\partial x}$ and $f'(z) = -i\frac{\partial f}{\partial y} = \frac{\partial f}{\partial(iy)}$ provided that $f$ is analytic, so \begin{equation*}
    f'(z) = \frac{\partial f}{\partial z}
\end{equation*}
Writing $f = u+iv$ we observe that \begin{equation*}
    \frac{\partial f}{\partial \overline{z}} = \frac{1}{2}\left[u_x -v_y\right]+\frac{i}{2}\left[u_y+v_x\right]
\end{equation*}
which when equated to zero is equivalent to the Cauchy Riemann equations \begin{equation*}
    \frac{\partial f}{\partial \overline{z}} = 0
\end{equation*}
Some intuition for this is that holomorphic functions are those of $z$-alone. For example, $f(z,\overline{z}) = z^3$, holomorphic, but $f(z,\overline{z}) = z\overline{z}$ is not holomorphic. 

