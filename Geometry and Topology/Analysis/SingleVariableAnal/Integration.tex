%%%%%%%%%% Integration %%%%%%%%%%
\chapter{Integration}\label{Int}
% use \chaptermark{}
% to alter or adjust the chapter heading in the running head

\section{Introduction to Definite Integrals}

\begin{definition}\index{Partition}
    Let $a < b$. A \Emph{partition} of the interval $I= [a,b]$ is a finite collection of points in $[a,b]$, one of which is $a$, and one of which is $b$. Equivalently, a partition of $I$ into $N$ subintervals consists pf endpoints $x_0 = a<x_1 < ... < x_N = b$ with subinterval $k$ being $J_k = [x_k,x_{k+1}]$ for $0 \leq k \leq N-1$.
\end{definition}
The points in a partition can be numbered $t_0,...,t_n$ so that \begin{equation}
    a = t_0 < t_1 < ... < t_{n-1} < t_n = b
\end{equation}
we shall always assume that such a numbering has been assigned.

\begin{definition}\index{Upper sum}\index{Lower sum}
    Suppose $f:I\rightarrow \R$ is bounded on $I=[a,b]$ so there exists $M \in \R^+$ such that $||f||_I \leq M$. Let $P = \{t_0,...,t_N\} \in \prod([a,b])$ be a partition of $[a,b]$. The \Emph{lower Riemann sum} of $f$ for $P$, denoted $L(f,P)$, is defined as \begin{equation}
        L(f,P) := \sum\limits_{k=0}^{N-1}\inf_{J_k}(f)\ell(J_k)
    \end{equation}
    The \Emph{upper Riemann sum} of $f$ for $P$, denoted $U(f,P)$, is defined as \begin{equation}
        U(f,P) = \sum\limits_{i=0}^{N-1}\sup_{J_k}(f)\ell(J_k)
    \end{equation}
    where $\ell(J_k) = t_{k+1} - t_k > 0$. Note $$-M \leq \sup_{J_k}(f) \leq M$$ and $$-M\leq \inf_{J_k}(f) \leq M$$
\end{definition}


\begin{remark}
    If $P$ is any partition, then \begin{equation}
        L(f,P) \leq U(f,P)
    \end{equation}
    because \begin{align*}
        L(f,P) &= \sum\limits_{k=0}^{N-1}\inf_{J_k}(f)\ell(J_k) \\
        U(f,P) &= \sum\limits_{i=0}^{N-1}\sup_{J_k}(f)\ell(J_k)
    \end{align*}
    and for each $i$ we have $\inf_{J_k}(f)\ell(J_k) \leq \sup_{J_k}(f)\ell(J_k)$.
\end{remark}

\begin{definition}\index{Refinement}
    Given two partitions $P$ adn $Q$ of $I$, we say $P$ is a \Emph{refinement} of $Q$ if every endpoint in $Q$ belongs to $P$, and we write $$P \succ Q$$
\end{definition}

\begin{lemma}
    If $P$ is a partition of $[a,b]$ which contains $Q$, that is $P \succ Q$, then \begin{align*}
        L(f,Q) &\leq L(f,P) \text{ and } U(f,Q) &\geq U(f,P)
    \end{align*}
\end{lemma}
\begin{proof}
    Consider first the special case in which $Q$ contains just one more point than $P$;\begin{align*}
        P &=\{t_0,...,t_n\} \\
        Q &= \{t_0,...,t_{k-1},u,t_k,...,t_n\}
    \end{align*}
    where $$a= t_0 < t_1 < ... < t_{k-1} < u < t_k < ... < t_n = b$$
    Let \begin{align*}
        m' &= \inf\{f(x):t_{k-1}\leq x \leq u\} \\
        m'' &= \inf\{f(x):u \leq x \leq t_k\}
    \end{align*}
    Then \begin{align*}
        L(f,P) &= \sum\limits_{i=1}^nm_i(t_i - t_{i-1}) \\
        L(f,Q) &= \sum\limits_{i=1}^{k-1}m_i(t_i - t_{i-1}) + m'(u-t_{k-1}) + m''(t_k-u) + \sum\limits_{i=k+1}^nm_i(t_i - t_{i-1})
    \end{align*}
    To prove that $L(f,P) \leq L(f,Q)$ it therefore suffices to show that \begin{equation*}
        m_k(t_k-t_{k-1}) \leq m'(u-t_{k-1}) + m''(t_k-u)
    \end{equation*}
    Now, the set $\{f(x):t_{k-1}\leq x \leq t_k\}$ contains all the numbers in $\{f(x):t_{k-1}\leq x \leq u\}$ and possibly some smaller        ones, so the greatest lower bound of the first set is less than or equal to the greatest lower bound of the second; thus                    \begin{equation*}
        m_k \leq m'
    \end{equation*}
    Similarly, \begin{equation*}
        m_k \leq m''
    \end{equation*}
    Therefore, \begin{equation*}
        m_k(t_k-t_{k-1}) = m_k(t_k-u)+m_k(u-t_{k-1}) \leq m''(t_k-u)+m'(u-t_{k-1})
    \end{equation*}
    This proves, in this special case that $L(f,P) \leq L(f,Q)$. Now, let \begin{align*}
        M' &= \sup\{f(x):t_{k-1} \leq x \leq u\} \\
        M'' &= \sup\{f(x):u \leq x \leq t_k\}
    \end{align*}
    Then \begin{align*}
        U(f,P) &= \sum\limits_{i=1}^nM_i(t_i - t_{i-1}) \\
        U(f,Q) &= \sum\limits_{i=1}^{k-1}M_i(t_i - t_{i-1}) + M'(u-t_{k-1}) + M''(t_k-u) + \sum\limits_{i=k+1}^nM_i(t_i - t_{i-1})
    \end{align*}
    Hence, to prove that $U(f,Q) \leq U(f,P)$ it suffices to show that \begin{equation*}
        M'(u-t_{k-1}) + M''(t_k - u) \leq M_k(t_k-t_{k-1})
    \end{equation*}
    As before, the set $\{f(x):t_{k-1}\leq x \leq t_k\}$ contains all the numbers in $\{f(x):t_{k-1}\leq x \leq u\}$ and possibly some          larger ones, so the smallest upper bound of the first set is greater than or equal to the smallest upper bound of the second; thus          \begin{equation*}
        M_k \geq M'
    \end{equation*}
    Similarly, \begin{equation*}
        M_k \geq M''
    \end{equation*}
    Therefore, \begin{equation*}
        M_k(t_k-t_{k-1}) = M_k(t_k - u) + M_k(u-t_{k-1}) \geq M''(t_k-u) + M'(u-t_{k-1})
    \end{equation*}
    This proves, in this special case that $U(f,P) \geq U(f,Q)$.


    The general case can now be deduced quite easily. The partition $Q$ can be obtained from $P$ by adding one point at a time; in otherwords, there is a sequence of partition \begin{equation*}
        P = P_1\subsetneq P_2 \subsetneq P_3 \subsetneq ... \subsetneq P_{\alpha} = Q
    \end{equation*}
    such that $P_{j+1} = P_j\cup\{u_{j+1}\}$ for some $u_{j+1} \in [a,b]-P_j$. Then \begin{equation*}
        L(f,P) = L(f,P_1) \leq L(f,P_2) \leq ... \leq L(f,P_{\alpha}) = L(f,Q)
    \end{equation*}
    and \begin{equation*}
        U(f,P) = U(f,P_1) \geq U(f,P_2) \geq ... \geq U(f,P_{\alpha}) = U(f,Q)
    \end{equation*}
    completing the proof.
\end{proof}



\begin{theorem}
    Let $P_1$ and $P_2$ be partitions of $[a,b]$, and let $f$ be a function which is bounded on $[a,b]$. Then \begin{equation}
        L(f,P_1) \leq U(f,P_2)
    \end{equation}
\end{theorem}
\begin{proof}
    There is a partition $P$ which contains both $P_1$ and $P_2$ (let $P = P_1 \cup P_2$). According to the lemma \begin{equation*}
        L(f,P_1) \leq L(f,P) \leq U(f,P) \leq U(f,P_1)
    \end{equation*}
\end{proof}

\begin{definition}
    Let $\prod(I)$ denote the set of all partitions of $I$. We define the \Emph{upper} and \Emph{lower Riemann integrals} by $$U_{I}(f) = \inf_{P\in \prod(I)}U(f,P) \geq L_I(f) = \sup_{P\in \prod(I)}L(f,P)$$
\end{definition}



\begin{definition}[Definite Integral]\index{Integral}
    A function $f$ which is bounded on $[a,b]$ is said to be \Emph{Riemann integrable} on $[a,b]$ if and only if $$L_I(f) = U_I(f)$$ In this case, this common number is called the \Emph{integral} of $f$ on $[a,b]$ and is denoted by \begin{equation}                            
        \int_If = \int_a^bf(x)dx = L_I(f) = U_I(f)
    \end{equation}
    The integral $\int_If$ is also called the \Emph{area} of $R(f,a,b)$ when $f(x) \geq 0$ for all $x \in [a,b]$.
\end{definition}


\begin{theorem}
    If $f$ is bounded on $[a,b]$, then $f$ is integrable on $[a,b]$ if and only if for every $\epsilon > 0$ there is a partition $P$ of $[a, b]$ such that $$U(f,P) - L(f,P) < \epsilon$$
\end{theorem}
\begin{proof}
    Suppose first that for every $\epsilon > 0$ there is such a partition $P$. Since \begin{align*}
        U_I(f) &\leq U(f,P) \\
        L_I(f) &\geq L(f,P)
    \end{align*}
    it follows that \begin{equation*}
        U_I(f) - L_I(f) \leq U(f,P) - L(f,P) < \epsilon
    \end{equation*}
    Since this is true for all $\epsilon > 0$, it follows that \begin{equation*}
        U_I(f) = L_I(f)
    \end{equation*}
    so by definition, then, $f$ is integrable. Conversely, if $f$ is integrable then \begin{equation*}
        U_I(f) = L_I(f)
    \end{equation*}
    Let $M$ denote the value of this. Then for each $\epsilon > 0$ there exist partitions $P'$ and $P''$ such that $|U(f,P') - M| <\epsilon/2$ and $|L(f,P'') - M| < \epsilon/2$. Then as $U(f,P') \geq L(f,P'')$ from the previous theorem, we have that \begin{equation*}
        U(f,P') - L(f,P'') = |U(f,P') - L(f,P'')| \leq |U(f,P') - M| + |M - L(f,P'')| < \epsilon
    \end{equation*}
    Let $P = P' \cup P''$ be a common refinement. Then, according to the lemma $U(f,P) \leq U(f,P')$ and $L(f,P) \geq L(f,P'')$ so \begin{equation*}
		U(f,P) - L(f,P) \leq U(f,P'') - L(f,P') <\epsilon
	\end{equation*}
\end{proof}

\begin{example}
    Define $f(x) = \left\{\begin{array}{cc} 1 & x \in \Q\cap[0,1] \\ 0 & x \notin \Q\cap [0,1]  \end{array}\right.$ Since both $S = \Q\cap [0,1]$ and $[0,1]\backslash S$ are dense in $[0,1]$, for any subinterval $J_k$ we have $\sup_{J_k}(f) = 1$ and $\inf_{J_k}(f) = 0$ so $L_I(f) = 0 < 1 = U_I(f)$. Thus $f$ is not Riemann integrable, but it is Lebesque integrable since $\Q$ is a set of measure $0$.
\end{example}


\begin{theorem}
    If $f \in \mathcal{C}(I)$ is continuous on $I=[a,b]$, then $f \in \mathcal{R}(I)$ is integrable on $I=[a,b]$.
\end{theorem}
\begin{proof}
    Notice, first, that $f$ is bounded on $[a,b]$, because it is continuous on $[a,b]$. To prove that $f$ is integrable on $[a,b]$, we want to use our previous theorem, and show that for every $\epsilon > 0$ there is a partition $P$ of $[a,b]$ such that \begin{equation*}
        U(f,P) - L(f,P) < \epsilon
    \end{equation*}
    Now we know, by our result on uniform continuity, that $f$ is uniformly continuous on $[a,b]$. So there is some $\delta > 0$ such that for all $x,y \in [a,b]$, if $|x-y| < \delta$, then $|f(x) - f(y)| < \epsilon/[2(b-a)]$. We choose a partition $P = \{t_0,...,t_N\}$ such that each $|t_{k+1}-t_{k}| < \delta$. Then for each subinterval $J_k$ we have \begin{equation*}
        |f(x) - f(y)| < \frac{\epsilon}{2(b-a)}
    \end{equation*}
    for all $x,y \in J_k$. Then, $f(x) < \frac{\epsilon}{2(b-a)}+f(y)$ for all $x\in J_k$, so $\sup_{J_k}(f) \leq \frac{\epsilon}{2(b-a)} + f(y)$. Further, $f(y) \geq \sup_{J_k}(f) - \frac{\varepsilon}{2(b-a)}$ for all $y \in J_k$, so $\inf_{J_k}(f) \geq \sup_{J_k}(f) - \frac{\varepsilon}{2(b-a)}$, so $\sup_{J_k}(f) - \inf_{J_k}(f) \leq \frac{\varepsilon}{2(b-a)}$. Since this is true for all $k$, we have that \begin{align*}
        U(f,P) - L(f,P) &= \sum_{k=0}^{N-1}(\sup_{J_k}(f) - \inf_{J_k}(f))\ell(J_k) \\
        &< \frac{\epsilon}{b-a}\sum_{k=0}^{N-1}\ell(J_k) \\
        &= \frac{\epsilon}{b-a}\ell(I) \\
        &= \epsilon
    \end{align*}
    Thus, by our previous theorem $f$ is integrable.
\end{proof}


\begin{proposition}\label{prop:4.2.1}
    $\mathcal{R}(I)$ is a $\R$-vector space. If $f,g \in \mathcal{R}(I), a \in \R$, then $af+g \in \mathcal{R}(I)$, and $$\int_I(af+g) = a\int_If+\int_Ig$$
\end{proposition}
\begin{proof}
    First we do additivity. Let $J_k \subseteq I$. Then $\sup_{J_k}(f+g) \leq \sup_{J_k}(f) + \sup_{J_k}(g)$ and $\inf_{J_k}(f+g) \geq \inf_{J_k}(f) + \inf_{J_k}(f)$. So it follows that $U(f+g,P) \leq U(f,P) + U(g,P)$ for any $P \in \prod(I)$ and $L(f+g,P) \geq L(f,P) + L(g,P)$. Then we have $$L_I(f) + L_I(g) \leq L_I(f+g) \leq U_I(f+g) \leq U_I(f) + U_I(g)$$ where by assumption $L_I(f) = U_I(f)$ and $L_I(g) = U_I(g)$, so $U_I(f+g) = L_I(f+g)$, and $f+g \in \mathcal{R}(I)$. Further, $\int_I(f+g) = U_I(f+g) = U_I(f) + U_I(g) = \int_If + \int_Ig$. 

    Next, let $a \in \R$. If $a = 0$, then $af = 0$ so $U(af,P) = 0 = L(af,P)$ for all partitions $P$ of $[a,b]$, and so $U_I(af) = 0 = L_I(af)$. Thus, $af \in \mathcal{R}(I)$, and $\int_Iaf = 0 =a\int_If$. Next, if $a > 0$, then $U(af,P) = aU(f,p)$ and $L(af,P) = aL(f,P)$. Then $U_I(af) = aU_I(f) = aL_I(f) = L_I(af)$, and we have our desired result. On the other hand, if $a < 0$, $U(af,P) = aL(f,P)$, and $L(af,P) = aU(f,P)$. Further, $U_I(af) = aL_I(f) = aU_I(f) = L_I(af)$, so we again obtain our desired result.
\end{proof}



\begin{theorem}
    Let $a < c < b$. If $f$ is integrable on $[a,b]$, then $f$ is integrable on $[a,c]$ and one $[c,b]$. Conversely, if $f$ is integrable on $[a,c]$ and on $[c,b]$, then $f$ is integrable on $[a.b]$. Finally, if $f$ is integrable on $[a,b]$, then \begin{equation}
        \int_a^bf = \int_a^cf + \int_c^bf
    \end{equation}
\end{theorem}
\begin{proof}
    (1) Suppose $f$ is integrable on $[a,b]$. Then $f$ is bounded on $[a,b]$, so it is bounded on $[a,c]$ and $[c,b]$. Indeed, $f$ being bounded implies that there exists $M \in \R$ such that for all $x \in [a,b]$ $|f(x)| \leq M$. Thus, as this applies for all $x \in [a,b]$ and $[a,c],[c,b] \subset [a,b]$, we have that it holds for all $x \in [a,c]$ and all $x \in [c,b]$. Now fix $\epsilon > 0$. Then there exists a partition $P$ of $[a,b]$ such that $$U(f,P) - L(f,P) < \epsilon$$
    Without loss of generality suppose $c=t_j$ for some $t_j \in P = \{t_0,t_1,...,t_n\}$. Then we have partitions $P' = \{t_0,...,t_j\}$ and $P'' = \{t_j,...,t_n\}$ for $[a,c]$ and $[c,b]$ respectively. Moreover, \begin{align*}
        U(f,P) &= U(f,P') + U(f,P'') \\
        L(f,P) &= L(f,P') + L(f,P'')
    \end{align*}
    Hence, we have that \begin{equation*}
        [U(f,P') - L(f,P')] + [U(f,P'') - L(f,P'')] = U(f,P) - L(f,P) < \epsilon
    \end{equation*}
    But $U(f,P') \geq L(f,P')$ and $U(f,P'') \geq L(f,P'')$, so \begin{align*}
        U(f,P') - L(f,P') &\leq U(f,P) - L(f,P) < \epsilon \\
        U(f,P'') - L(f,P'') &\leq U(f,P) - L(f,P) < \epsilon
    \end{align*}
    Therefore, $f$ is integrable on $[a,c]$ and $[c,b]$


    (2) Suppose $f$ is integrable on $[a,c]$ and $[c,b]$. Thus, there exists $M_1, M_2 \in \R$ such that for all $x \in [a,c]$ $|f(x)| \leq M_1$ and for all $x \in [c,b]$ $|f(x)| \leq M_2$. Let $M = \max(M_1,M_2)$. Then for all $x \in [a,b]$ we have $|f(x)| \leq M$, so $f$ is bounded on $[a,b]$. Let $\epsilon > 0$. Then there exist partitions $P_1,P_2$ of $[a,c]$ and $[c,b]$ respectively such that \begin{align*}
        U(f,P_1) - L(f,P_1) &< \epsilon/2 \\
        U(f,P_2) - L(f,P_2) &< \epsilon/2
    \end{align*}
    Let $P = P_1 \cup P_2$, where $P_1 \cap P_2 = \{c\}$. Then we have that \begin{equation*}
        U(f,P) - L(f,P) = [U(f,P_1) - L(f,P_1)] + [U(f,P_2) - L(f,P_2)] < \epsilon/2 + \epsilon/2 = \epsilon
    \end{equation*}
    Therefore, by definition $f$ is integrable on $[a,b]$.
    

    (3) Suppose $f$ is integrable on $[a,b]$, so by the previous results $f$ is integrable on $[a,c]$ and $[c,b]$. Let $\int_a^bf = R$, $\int_a^cf = R_1$, and $\int_c^bf = R_2$. Let $P$ be a partition of $[a,b]$, and without loss of generality suppose $c \in P = \{t_0,...,t_j = c,...,t_n\}$. Then let $P_1 = \{t_0,...,t_j\}$ and $P_2 = \{t_j,...,t_n\}$ be partitions of $[a,c]$ and $[c,b]$. It then follows that \begin{align*}
        L(f,P_1) \leq &R_1 \leq U(f,P_1) \\
        L(f,P_2) \leq &R_2 \leq U(f,P_2)
    \end{align*}
    Hence, we have that \begin{align*}
        L(f,P) &= L(f,P_1) + L(f,P_2) \leq R_1 + R_2 \\
        U(f,P) &= U(f,P_1) + U(f,P_2) \geq R_1 + R_2
    \end{align*}
    Thus $L(f,P) \leq R_1 + R_2\leq U(f,P)$. Note that this holds for all partitions $P$, as if $P'$ is a partition, then considering the partition $P'_{c} = P' \cup \{c\}$ we have that $$L(f,P') \leq L(f,P'_{c}) \leq R_1 + R_2 \leq U(f,P'_{c}) \leq U(f,P')$$
    Therefore, this holds for all partitions of $[a,b]$, but $R$ is the unique number which does this so we must have that $R = R_1 + R_2$. Thus \begin{equation*}
        \int_a^bf = \int_a^cf + \int_c^bf
    \end{equation*}
\end{proof}


\begin{definition}
    Using the previous theorem, we define \begin{equation}
        \int_a^af := 0 \;\;\;and\;\;\;\int_a^bf := -\int_b^af,\;for\;a>b
    \end{equation}
\end{definition}


\begin{theorem}\label{thm:boundint}
    Suppose $f$ is integrable on $[a,b]$ and that \begin{equation}
        m \leq f(x) \leq M \forall x \in [a,b]
    \end{equation}
    Then \begin{equation}
        m(b-a) \leq \int_a^bf \leq M(b-a)
    \end{equation}
\end{theorem}
\begin{proof}
    It is clear that $M \geq \sup\{f(x):x \in [a,b]\}$ and $m \leq \inf\{f(x):x\in [a,b]\}$, so \begin{equation*}
        m(b-a) \leq L(f,P)\;and\;M(b-a) \geq U(f,P)
    \end{equation*}
    for every partition $P$. Since $\int_a^bf = \sup\{L(f,P)\} = \inf\{U(f,P)\}$ we have that \begin{equation*}
        m(b-a) \leq \sup\{L(f,P)\} = \int_a^bf = \inf\{U(f,P)\} \leq M(b-a)
    \end{equation*}
\end{proof}

If $f$ is integrable on $[a,b]$, we can define a new function $F$ on $[a,b]$ by \begin{equation}
    F(x) = \int_a^xf = \int_a^xf(t)dt
\end{equation}



\begin{theorem}
    If $f$ is integrable on $[a,b]$ and $F$ is defined on $[a,b]$ by \begin{equation*}
        F(x) = \int_a^xf
    \end{equation*}
    then $F$ is continuous on $[a,b]$.
\end{theorem}
\begin{proof}
    Suppose $c \in [a,b]$. Since $f$ is integrable on $[a,b]$ it is, by definition, bounded on $[a,b]$; let $M$ be a number such that \begin{equation*}
        |f(x)| \leq M,\forall x \in [a,b]
    \end{equation*}
    If $h > 0$ for $c+h \in [a,b]$, then \begin{equation*}
        F(c+h) - F(c) = \int_a^{c+h}f - \int_a^cf = \int_c^{c+h}f
    \end{equation*}
    Since $-M \leq f(x) \leq M$ for all $x \in [a,b]$ it follows from the Theorem \ref{thm:boundint} that \begin{equation*}
        -M\cdot h \leq \int_c^{c+h}f \leq M\cdot h
    \end{equation*}
    In other words \begin{equation*}
        -M\cdot h \leq F(c+h) - F(c) \leq M\cdot h
    \end{equation*}
    If $h < 0$, with $c+h \in [a,b]$ we find the inequality \begin{equation*}
        -M\cdot h \geq F(c+h) - F(c) \geq M\cdot h
    \end{equation*}
    In either case we have that \begin{equation*}
        |F(c+h) - F(c)| \leq M\cdot |h|
    \end{equation*}
    Therefore, if $\epsilon > 0$, we have \begin{equation*}
        |F(c+h) - F(c)| < \epsilon
    \end{equation*}
    provided that $|h| < \epsilon/M$. This proves that \begin{equation*}
        \lim\limits_{h\rightarrow 0}F(c+h) = F(c)
    \end{equation*}
    so $F$ is continuous at $c$.
\end{proof}

\section{Reimann Sums}

\begin{definition}\index{Riemann sum}
    Suppose $P = \{t_0,...,t_N\}$ is a partition of $[a,b]$, and for each $k$ choose $\xi_k \in J_k$. Then we clearly have that \begin{equation}
        L(f,P) \leq \sum\limits_{k=0}^{N-1}f(\xi_k)\ell(J_k) \leq U(f,P)
    \end{equation}
    Any sum of the form \begin{equation}
        \sum\limits_{k=0}^{N-1}f(\xi_k)\ell(J_k)
    \end{equation}
    is called a \Emph{Reimann sum} of $f$ for $P$.
\end{definition}


Before proving an important result related to Riemann sums and definite integrals, we handle the following proposition:

\begin{proposition}
    Suppose $f:[a,b]\rightarrow \R$ is bounded by $M \in \R^+$. If $P$ and $Q$ are partitions of $[a,b]$ such that for some $k > 1$ $$\text{maxsize}(P) \leq \frac{\text{minsize}(Q)}{k}$$ then $$U(f,P) \leq U(f,Q) + \frac{2M\ell([a,b])}{k}$$ and $$L(f,P) \geq L(f,Q) - \frac{2M\ell([a,b])}{k}$$
\end{proposition}
\begin{proof}
    Let $P_1 = P\cup Q$ be the common refinement of $P$ and $Q$. The intervals in $P$, denoted $\tilde{J}_i$, can be separated into two classes, \begin{enumerate}
        \item[(i)] $\tilde{J}_i \subseteq J_k$ for some $k$, where $J_k$ are the intervals in $Q$
        \item[(ii)] $\tilde{J}_i\cancel{\subseteq} J_k$ for all $k$
    \end{enumerate}
    Note that in the case of (i), $\tilde{J}_i$ also belongs to $P_1$. In the second case, $\tilde{J}_i$ is split into two intervals in $P_1$. Then $$|U(f,P) - U(f,P_1)| \leq \sum_{\text{intervals in (ii)}}|\sup_{\tilde{J}_i}(f)\ell(\tilde{J}_i) - (\sup_{\tilde{J}_i^+}(f)\ell(\tilde{J}_i^+) + \sup_{\tilde{J}_i^-}(f)\ell(\tilde{J}_i^-))|$$ where the intervals in class (i) cancel since they belong to both partitions. If $\tilde{J}_i$ is in class (ii), then $\tilde{J}_i = \tilde{J}_i^+\cup\tilde{J}_i^-$ in $P_1$. Then proceeding with the inequality \begin{align*}
        |U(f,P) - U(f,P_1)| &\leq M\sum_{\in (ii)}\ell(\tilde{J}_i) + \ell(\tilde{J}_i^+) + \ell(\tilde{J}_i^-) \\
        &= 2M\sum_{\in (ii)}\ell(\tilde{J}_i)
    \end{align*}
    If $\tilde{J}_i$ is in class (ii), then there exists a unique endpoint $x_l$ in $Q$ such that $x_l \in \tilde{J}_i^{\circ}$ (the interior of the interval) so $$\ell(\tilde{J}_i) \leq \frac{\ell(J_l)}{k}$$ Now \begin{align*}
        |U(f,P) - U(f,P_1)| \leq 2M\sum_{\in (ii)}\ell(\tilde{J}_i) \leq 2M\sum_{\in (ii)}\frac{\ell(J_l)}{k} \leq \frac{2M\ell(I)}{k}
    \end{align*}
    As $P_1$ refines $Q$, $U(f,P_1) \leq U(f,Q)$, so $$U(f,P_1) \leq U(f,P) \leq U(f,P_1) + \frac{2M\ell(I)}{k} \leq U(f,Q) + \frac{2M\ell(I)}{k}$$ Similarly, $L(f,P_1)$ refines $Q$ so $L(f,P_1) \geq L(f,Q)$, and by the same chain as above $|L(f,P) - L(f,P_1)| \leq \frac{2M\ell(I)}{k}$, so $$L(f,P_1) \geq L(f,P) \geq L(f,P_1) - \frac{2M\ell(I)}{k} \geq L(f,Q) - \frac{2M\ell(I)}{k}$$ as desired.
\end{proof}

Now we move on to our fundamental result:

\begin{theorem}[Darboux's Theorem]\index{Darboux's Theorem}
    Let $f:[a,b]\rightarrow \R$ be bounded. Let $P_{\nu} \in \prod([a,b])$ be a sequence of partitions of $I=[a,b]$ such that $$\lim\limits_{\nu\rightarrow \infty}\text{maxsize}(P_{\nu}) = 0$$ Then $$U_I(f) = \lim\limits_{\nu\rightarrow \infty}U(f,P_{\nu})$$ and $$L_I(f) = \lim\limits_{\nu\rightarrow \infty}L(f,P_{\nu})$$ In particular, $f \in \R([a,b])$ if and only if $$\int_If = \lim\limits_{\nu\rightarrow \infty}\sum_{k=0}^{N_{\nu}-1}f(\xi_{\nu,k})\ell(J_{\nu,k})$$ where $P_{\nu} = (a=x_{\nu,0} < ... < x_{\nu,N_{\nu}}=b)$, $J_{\nu,k} = [x_{\nu,k},x_{\nu,k+1}]$, and where $\xi_{\nu,k}$ is an arbitrary point of $J_{\nu,k}$.
\end{theorem}
\begin{proof}
    Let $P_{\nu} \in \prod([a,b])$ be a sequence of partition such that $\lim\limits_{\nu\rightarrow \infty}\text{maxsize}P_{\nu} = 0$. Note $U_I(f) = \inf_{P\in\prod([a,b])}U(f,P)$, which exists and is well defined if $f$ is bounded, and similarly for $L_I(f) = \sup_{P \in \prod([a,b])}L(f,P)$. Given $\varepsilon > 0$, there exists $Q \in \prod(I)$ such that $U_I(f) + \varepsilon \geq U(f,Q) \geq U_I(f)$ and taking a refinement if needed, $L_I(f) - \varepsilon \leq L(f,Q) \leq L_I(f)$. Let $\nu,\mu \geq N$ such that $$\text{maxsize}P_{\nu} \leq \varepsilon\text{minsize}Q,\;\forall\nu\geq N$$ In particular, we can choose $\varepsilon = \frac{1}{k}$, $k \in \N$, $k > 1$. By the previous proposition \begin{align*}
        U(f,P_{\nu}) &\leq U(f,Q) + \varepsilon 2M\ell(I) \\
        L(f,P_{\nu}) &\geq L(f,Q) - \varepsilon 2M\ell(I)
    \end{align*}
    Then $U(f,P_{\nu}) \leq U_I(f) + \varepsilon(2M\ell(I)+1)$ and $L(f,P_{\nu}) \geq L_I(f) - \varepsilon(2M\ell(I)+1)$. Then $$|U(f,P_{\nu}) - U_I(f)| \leq \varepsilon(2M\ell(I)+1)$$ and $$|L_I(f) - L(f,P_{\nu})| \leq \varepsilon(2M\ell(I)+1)$$ for all $\nu \geq N$. As we can make $\varepsilon$ as small as we wish, $$\lim\limits_{\nu\rightarrow \infty}U(f,P_{\nu}) = U_I(f)\;\text{ and }\;\lim\limits_{\nu\rightarrow \infty}L(f,P_{\nu}) = L_I(f)$$ Then $f \in \mathcal{R}(I)$ if and only if $U_I(f) = L_I(f)$, which occurs if and only if $\lim\limits_{\nu\rightarrow \infty}U(f,P_{\nu}) = \lim\limits_{\nu\rightarrow \infty}L(f,P_{\nu})$. We observe that these last limits are equal if and only if $\lim\limits_{\nu\rightarrow \infty}\sum_{k=0}^{N_{\nu}-1}f(\xi_{\nu,k})\ell(J_{\nu,k})$ exists whenever $\text{maxsize}P_{\nu}\rightarrow 0$ for any choices of $\xi_{\nu,k} \in J_{\nu,k}$, as \begin{equation*}
        L(f,P_{\nu}) \leq \sum_{k=0}^{N_{\nu}-1}f(\xi_{\nu,k})\ell(J_{\nu,k}) \leq U(f,P_{\nu})
    \end{equation*}
\end{proof}



\section{The Fundamental Theorem of Calculus}

\begin{theorem}[The Fundamental Theorem of Calculus Part 1]\label{thm:FTC1}\index{FTC I}
    Let $f \in \mathcal{R}([a,b])$ be integrable on $[a,b]$, and define $F$ on $[a,b]$ by \begin{equation}
        F(x) = \int_a^xf = \int_a^xf(t)dt
    \end{equation}
    If $f$ is continuous at $c$ in $[a,b]$, then $F$ is differentiable at $c$, and \begin{equation}
        F'(c) = f(c)
    \end{equation}
    (if $c = a$ or $b$, then $F'(c)$ is understood to mean the right or left hand derivative of $F$).
\end{theorem}
\begin{proof}
    First, consider $c \in (a,b)$. By definition,\begin{equation*}
        F'(c) = \lim\limits_{h\rightarrow 0}\frac{F(c+h)-F(c)}{h}
    \end{equation*}
    Suppose first that $h > 0$. Then \begin{equation*}
        F(c+h) - F(c) = \int_c^{c+h}f
    \end{equation*}
    Define $m_h$ and $M_h$ as follows:\begin{align*}
        m_h &= \inf\{f(x):c\leq x \leq c+h\} \\
        M_h &= \sup\{f(x):c\leq x \leq c+h\}
    \end{align*}

    It follows from Theorem \ref{thm:boundint} that \begin{equation*}
        m_h\cdot h\leq \int_c^{c+h}f \leq M_h\cdot h
    \end{equation*}
    Therefore, \begin{equation*}
        m_h\leq \frac{F(c+h) - F(c)}{h} \leq M_h
    \end{equation*}
    If $h < 0$, let \begin{align*}
        m_h &= \inf\{f(x):c+h\leq x \leq c\} \\
        M_h &= \sup\{f(x):c+h\leq x \leq c\}
    \end{align*}

    It follows from Theorem \ref{thm:boundint} that \begin{equation*}
        m_h\cdot (-h)\leq \int_{c+h}^{c}f \leq M_h\cdot (-h)
    \end{equation*}
    Since \begin{equation*}
        F(c+h) - F(c) = \int_c^{c+h}f = -\int_{c+h}^cf
    \end{equation*}
    this yields \begin{equation*}
        m_h\cdot h\geq F(c+h) - F(c) \geq M_h\cdot h
    \end{equation*}
    Since $h < 0$, we have that \begin{equation*}
        m_h \leq \frac{F(c+h) - F(c)}{h} \leq M_h
    \end{equation*}
    This inequality is true for any integrable function, continuous or not. Since $f$ is continuous at $c$, however, \begin{equation*}
        \lim\limits_{h\rightarrow 0}m_h = \lim\limits_{h\rightarrow 0}M_h = f(c)
    \end{equation*}
    and this proves that \begin{equation*}
        F'(c) = \lim\limits_{h\rightarrow 0}\frac{F(c+h) - F(c)}{h} = f(c)
    \end{equation*}


    Now, if $c = a$ we need only look at when $h > 0$, and in this case we still have  \begin{equation*}
        m_h \leq \frac{F(a+h) - F(a)}{h} \leq M_h
    \end{equation*}
    and from our previous limits, \begin{equation*}
        \lim\limits_{h\rightarrow 0^+}m_h = \lim\limits_{h\rightarrow 0^+}m_h = f(a)
    \end{equation*}
    thus we have that $$F'(a) = \lim\limits_{h\rightarrow 0^+}\frac{F(a+h)-F(a)}{h} = f(a)$$
    Similarly, if $c = b$ we need only look at $h < 0$, so we have that \begin{equation*}
        \lim\limits_{h\rightarrow 0^-}m_h = \lim\limits_{h\rightarrow 0^-}m_h = f(b)
    \end{equation*}
    and $$F'(b) = \lim\limits_{h\rightarrow 0^-}\frac{F(b+h)-F(b)}{h} = f(b)$$
    completing the proof.
\end{proof}

We may consider \begin{equation}
    F(x) = \int_a^xf
\end{equation}
when $x < a$. In this case we have that \begin{equation}
    F(x) = -\int_x^af = -\left(\int_b^af - \int_b^xf\right)
\end{equation}
so for $c \in [a,b]$, \begin{equation}
    F'(c) = -(-f(c)) = f(c)
\end{equation}
as before.

\begin{theorem}[Fundamental Theorem of Calculus Part 2]\label{thm:FTC2}\index{FTC II}
    Suppose $G$ is continuous in $[a,b]$ and differentiable in $(a,b)$, with $G' \in \mathcal{R}([a,b])$. Then $$\int_a^bG'(t)dt = G(b)-G(a)$$
\end{theorem}
\begin{proof}
    Let $G$ be as described. Let $P_n$ be a partition with endpoints $a + \frac{b-a}{n}k$ for $0 \leq k \leq n$. By Darboux's theorem, for arbitrary $\xi_{n,k} \in J_{n,k} = \left[a+\frac{b-a}{n}k,a+\frac{b-a}{n}(k+1)\right]$, \begin{equation*}
        \int_a^bG'(t)dt = \lim\limits_{n\rightarrow \infty}\sum_{k=0}^{n-1}G'(\xi_{n,k})\ell(J_{n,k}) = \lim\limits_{n\rightarrow \infty}\sum_{k=0}^{n-1}G'(\xi_{n,k})\frac{b-a}{n}
    \end{equation*}
    Now observe we have the telescopic sum $$G(b) - G(a) = G(x_n) - G(x_0) = \sum_{k=0}^{n-1}G(x_{k+1})-G(x_k)$$ As $G$ is continuous on $[a,b]$ and differentiable on $(a,b)$, we have by the mean value theorem that there exists $\xi^*_{n,k} \in (x_k,x_{k+1})$ such that $$G(x_{k+1})-G(x_k) = G'(\xi^*_{n,k})(x_{k+1}-x_k)$$ Then let the arbitrary $\xi_{n,k}$ be the $\xi^*_{n,k}$, so \begin{align*}
        \int_a^bG'(t)dt &= \lim\limits_{n\rightarrow \infty}\sum_{k=0}^{n-1}G'(\xi^*_{n,k})\frac{b-a}{n} \\
        &= \lim\limits_{n\rightarrow \infty}\sum_{k=0}^{n-1}G(x_{k+1})-G(x_k) \\
        &= \lim\limits_{n\rightarrow \infty}(G(b) - G(a)) = G(b) - G(a)
    \end{align*}
    as desired.
\end{proof}


It is important to note that this is merely a useful result for certain functions $f$, \Emph{not} a definition.




If $f$ is any bounded function on $I=[a,b]$, then \begin{equation}
    L_I(f)\;\;and\;\;U_I(f)
\end{equation}
will both exist. These numbers are called the \Emph{lower integral} of $f$ on $[a,b]$ and the \Emph{upper integral} of $f$ on $[a,b]$, respectively, and will sometimes be denoted by \begin{equation}
    \mathbf{L}\int_a^bf\;\;and\;\;\mathbf{U}\int_a^bf
\end{equation}
If $a < c < b$, then \begin{equation}
    \mathbf{L}\int_a^bf = \mathbf{L}\int_a^cf + \mathbf{L}\int_c^bf\;\;and\;\;\mathbf{U}\int_a^bf = \mathbf{U}\int_a^cf + \mathbf{U}\int_c^bf
\end{equation}
and if $m \leq f(x) \leq M$ for all $x \in [a,b]$, then \begin{equation}
    m(b-a)\leq \mathbf{L}\int_a^bf \leq \mathbf{U}\int_a^bf\leq M(b-a)
\end{equation}
$f$ is integrable precisely when \begin{equation}
    \mathbf{L}\int_a^bf = \mathbf{U}\int_a^bf
\end{equation}


We shall now demonstrate an alternate proof for the following theorem stated previously.



\begin{theorem}
    If $f$ is continuous on $[a,b]$, then $f$ is integrable on $[a,b]$.
\end{theorem}
\begin{proof}
    Define function $L$ and $U$ on $[a,b]$ by \begin{equation*}
        L(x) = \mathbf{L}\int_a^xf\;\;and\;\;U(x) = \mathbf{U}\int_a^xf
    \end{equation*}
    Let $x \in (a,b)$. If $h > 0$ and \begin{align*}
        m_h &= \inf\{f(t):x\leq t \leq x+h\} \\
        M_h &= \sup\{f(t): x\leq t\leq x+h\} 
    \end{align*}
    then \begin{equation*}
        m_h\cdot h \leq \mathbf{L}\int_x^{x+h}f \leq \mathbf{U}\int_x^{x+h}f\leq M_h\cdot h
    \end{equation*}
    so \begin{equation*}
        m_h\cdot h \leq L(x+h) - L(x) \leq U(x+h) - U(x) \leq M_h\cdot h
    \end{equation*}
    or \begin{equation*}
        m_h\leq \frac{L(x+h)-L(x)}{h} \leq \frac{U(x+h)-U(x)}{h} \leq M_h
    \end{equation*}
    If $h < 0$ and \begin{align*}
        m_h &= \inf\{f(t):x+h\leq t \leq x\} \\
        M_h &= \sup\{f(t): x+h\leq t\leq x\} 
    \end{align*}
    one obtains the same inequality, precisely as in the proof of \ref{thm:FTC1}.

    Since $f$ is continuous at $x$, we have \begin{equation*}
        \lim\limits_{h\rightarrow 0}m_h = \lim\limits_{h\rightarrow 0}M_h = f(x)
    \end{equation*}
    and this proves that \begin{equation*}
        L'(x) = U'(x) = f(x),\forall x\in(a,b)
    \end{equation*}
    THis means that there is a number $c$ such that \begin{equation*}
        U(x) = L(x) + c,\forall x \in [a,b]
    \end{equation*}
    Since $U(a) = L(a) = 0$, the number $c$ must be equal to $0$, so \begin{equation*}
        U(x) = L(x) \forall x \in [a,b]
    \end{equation*}
    In particular, \begin{equation*}
        \mathbf{U}\int_a^bf = U(b) = L(b) = \mathbf{L}\int_a^bf
    \end{equation*}
    so $f$ is integrable on $[a,b]$.
\end{proof}


\section{Content of Sets}

Using properties of sets, we can investigate the collection of Riemann integrable functions more carefully.

\begin{definition}\index{Content}
    Given $S \subseteq I$ we define the \Emph{characteristic function for $S$} by \begin{equation*}
        \chi_S(x) = \left\{\begin{array}{cc} 1 & x \in S \\ 0 & x \notin S \end{array}\right.
    \end{equation*}
    and the \Emph{upper content of $S$} and the \Emph{lower content of $S$} by \begin{equation*}
        \text{cont}^+(S) := U_I(\chi_S)\;\text{ and }\;\text{cont}^-(S) = L_I(\chi_S)
    \end{equation*}
    If $\text{cont}^+(S) = \text{cont}^-(S)$ we say that $S$ has \Emph{content} $$m(S) = \text{cont}^+(S) = \text{cont}^-(S)$$ and say $S$ is \Emph{contented}.
\end{definition}

We observe by Darboux's theorem $\text{cont}^+(S) = U_I(\chi_S) = \lim\limits_{\nu\rightarrow \infty}U(\chi_S,P_{\nu})$ if $\text{maxsize}P_{\nu}\rightarrow 0$. Since $\sup_{J}\chi_S = 1$ if $S \cap J \neq \emptyset$, $\sup_{J}\chi_S = 0$ if $S\cap J = \emptyset$, $\inf_J(\chi_S) = 1$ if $S \supseteq J$, and $\inf_J(\chi_S) = 0$ if $S \cancel{\supseteq} J$, we can formulate the upper and lower contents by $$\text{cont}^+(S) = \inf\left\{\sum_{k=1}^N\ell(J_k):S \subseteq \bigcup_{k=1}^NJ_k\right\}$$ and $$\text{cont}^-(S) = \sup\left\{\sum_{k=1}^N\ell(J_k):S \supseteq \sqcup_{k=1}^NJ_k\right\}$$ (Note we need disjoint sets for the lower content)

We now define the Lebesque measure, which extends the upper content to what is known as an outer measure.


\begin{definition}\index{Lebesque measure}
    We define the \Emph{Lebesque measure} by $$m^*(S) = \inf\left\{\sum_{k\geq 1}\ell(J_k): S\subseteq \bigcup_{k\geq 1}J_k\right\}$$ which is an \Emph{outer measure}, and we can define the associated \Emph{inner measure} by $$m_*(S) = \sup\left\{\sum_{k\geq 1}\ell(J_k):S\supseteq \sqcup_{k\geq 1}J_k\right\}$$
\end{definition}

\begin{example}
    We observe that $\text{cont}^+(\Q\cap [0,1]) = 1$ and $\text{cont}^-(\Q\cap [0,1]) = 0$, but $m^*(\Q\cap [0,1]) = 0$. Indeed, as $\Q\cap [0,1] = \{q_1,q_2,...\}$ is countable, if $\varepsilon > 0$ we can choose $J_k = \left[q_k - \frac{\varepsilon}{2^{k+2}}, q_k + \frac{\varepsilon}{2^{k+2}}\right]\cap[0,1]$. Then $\Q\cap[0,1] \subseteq \bigcup_{k\geq 1}J_k$ and $$\sum_{k\geq 1}\ell(J_k) \leq \sum_{k\geq 1}\frac{\varepsilon}{2^{k+1}} \leq \frac{\varepsilon}{2} < \varepsilon$$ 
\end{example}

In general, if $S$ is countable then $m^*(S) = 0$. We can now investigate a necessary and sufficient condition for Riemann integrability: 

\begin{proposition}\label{prop:4.2.12}
    If $f:I=[a,b]\rightarrow \R$ is bounded, and $S$ is the set of points of discontinuity of $f$ in $[a,b]$, then $m^*(S) 0$ implies $f \in \mathcal{R}(I)$, where $$m^*(S) = \inf\left\{\sum_{k\geq 1}\ell(J_k):S\subseteq \bigcup_{k\geq 1}J_k\right\}$$
\end{proposition}
\begin{proof}
    As $f$ is bounded, there exists $M > 0$ such that $||f||_{\infty} \leq M$. Let $\varepsilon > 0$. As $m^*(S) = 0$, there exists $J_k$, $k\geq 1$, such that $$S \subseteq \bigcup_{k\geq 1}J_k\;\text{ and }\;\sum_{k=1}^{\infty}\ell(J_k) < \varepsilon$$ If $x \in I\backslash S$, then there exists an open interval $K_x$ containing $x$ such that $$0 \leq \sup_{K_x}(f) - \inf_{K_x}(f) < \varepsilon$$ as $f$ is continuous at $x$. Then $I \subseteq \left(\bigcup_{k\geq 1}J_k\right) \cup\left(\bigcup_{x \in I\backslash S}K_x\right)$. But $I$ is compact so there exists a finite covering $$I \subseteq \left(\bigcup_{k=1}^NJ_k\right) \cup\left(\bigcup_{i=1}^MK_i\right)$$ Let $P \in \prod(I)$ be the partition conformed by all the endpoints of intervals $J_l, 1 \leq l \leq N$, and $K_i, 1 \leq i \leq M$. Write $P = \{L_1,...,L_{\mu}\}$. We have two cases: \begin{enumerate}
        \item $L_j \subseteq K_i$ for some $i$ or 
        \item $L_j \subseteq J_l$ for some $l$
    \end{enumerate}
    Let $A = \{1\leq j\leq \mu:\exists i;L_j \subseteq K_i\}$ and $B = \{1\leq j \leq \mu:\exists l;L_j\subseteq J_l\}$. Note $\sum_{j\in B}\ell(L_j) \leq \sum_{k=1}^N\ell(J_k)<\varepsilon$ and \begin{align*}
        0 \leq U(f,P) - L(f,P) &= \sum_{j \in A}\left(\sup_{L_j}(f) - \inf_{L_j}(f)\right)\ell(L_j) + \sum_{j\in B}\left(\sup_{L_j}(f) - \inf_{L_j}(f)\right)\ell(L_j) \\
        &< \varepsilon\sum_{j\in A}\ell(L_j) + 2M\sum_{j\in B}\ell(L_j) \\
        &\leq \varepsilon \ell(I) + 2M\varepsilon \\
        &= \varepsilon(\ell(I)+2M)
    \end{align*}
    which goes to $0$ as we can make $\varepsilon$ arbitrarily small. Thus $f \in \mathcal{R}([a,b])$ as desired.
\end{proof}


%
\section*{Appendix A: Trigonometric Functions}
\addcontentsline{toc}{section}{Appendix A: Trigonometric Functions}
%

\begin{definition}
    We define the mathematical constant $\pi$ as the area of the unit circle, or in this case, twice the area of a semi-circle:\begin{equation}
        \pi:=2\cdot \int_{-1}^1\sqrt{1-x^2}dx
    \end{equation}
\end{definition}


\begin{definition}
    If $-1 \leq x \leq 1$, then the area of the sector bounded between the upper unit circle from $[x,1]$ and the $x$-axis and radial arm is \begin{equation}
        A(x) := \frac{x\sqrt{1-x^2}}{2} + \int_x^1\sqrt{1-t^2}dt
    \end{equation}
\end{definition}

\begin{remark}
    For $-1 < x < 1$, $A$ is differentiable at $x$ and \begin{align*}
        A'(x) &= \frac{1}{2}\left[\sqrt{1-x^2} +x\cdot\frac{-2x}{2\sqrt{1-x^2}}\right] -\sqrt{1-x^2} \\
        &= \frac{1}{2}\frac{1-x^2-x^2}{\sqrt{1-x^2}} - \frac{1-x^2}{\sqrt{1-x^2}} \\
        &= \frac{1}{2}\frac{-1}{\sqrt{1-x^2}} \\
        &= \frac{-1}{2\sqrt{1-x^2}}
    \end{align*}
    Note that on $[-1,1]$, the function $A$ decreases from $A(-1) = \frac{\pi}{2}$ to $A(1) = 0$.
\end{remark}

\begin{definition}
    If $0 \leq x \leq \pi$, then $\cos x$ is the unique number in $[-1,1]$ such that \begin{equation}
        A(\cos x) = \frac{x}{2}
    \end{equation}
    and \begin{equation}
        \sin x := \sqrt{1-(\cos x)^2}
    \end{equation}
    Note that such a $\cos x$ exists as $A$ is continuous on $[-1,1]$, and $A(-1) = \frac{\pi}{2}$ while $A(1) = 0$. Hence, by \ref{thmname:intval} there exists $y \in [-1,1]$ such that $A(y) = \frac{x}{2}$ for all $x \in [0,\pi]$.
\end{definition}


\begin{theorem}
    If $0 < x < \pi$, then \begin{align*}
        \cos'(x) &= -\sin x \\
        \sin'(x) &= \cos x
    \end{align*}
\end{theorem}
\begin{proof}
    If $B = 2A$, then the definition $A(\cos x) = x/2$ can be written \begin{equation*}
        B(\cos x) = x
    \end{equation*}
    in other words, $\cos$ is just the inverse of $B$. We have already computed taht \begin{equation*}
        A'(x) = -\frac{1}{2\sqrt{1-x^2}}
    \end{equation*}
    from which we conclude \begin{equation*}
        B'(x) = -\frac{1}{\sqrt{1-x^2}}
    \end{equation*}
    Consequently we have that \begin{align*}
        \cos'(x) &= (B^{-1})'(x) \\
        &= \frac{1}{B'(B^{-1}(x))} \\
        &= \frac{1}{-\frac{1}{\sqrt{1-[B^{-1}(x)]^2}}} \\
        &= -\sqrt{1-(\cos x)^2} \\
        &= - \sin x
    \end{align*}
    Then, since $\sin x = \sqrt{1-(\cos x)^2}$ we also obtain \begin{align*}
        \sin'(x) &= \frac{1}{2}\cdot \frac{-2\cos x\cdot \cos'(x)}{\sqrt{1-(\cos x)^2}} \\
        &= \frac{-\cos x\cdot (-\sin x)}{\sin x}\\
        &= \cos x
    \end{align*}
\end{proof}


\begin{definition}
    Now, to define $\sin$ and $\cos$ on $\R$, we proceed as follows \begin{enumerate}
        \item If $\pi \leq x \leq 2\pi$, the \begin{align*}
                \sin x &= -\sin(2\pi - x) \\
                \cos x &= \cos(2\pi - x)
            \end{align*}
        \item If $x = 2\pi k+x'$ for some integer $k$ and some $x' \in [0,2\pi]$, then \begin{align*}
                \sin x &= \sin x' \\
                \cos x &= \cos x'
        \end{align*}
    \end{enumerate}
\end{definition}


\begin{lemma}
    Suppose $f$ has a second derivative everywhere and that \begin{align*}
        f''+f&= 0\\
        f(0) &= 0\\
        f'(0) &= 0\\
    \end{align*}
    Then $f = 0$
\end{lemma}
\begin{proof}
    Multiplying both sides of the first equation by $f'$ yields \begin{equation*}
        f'f'' + ff' = 0
    \end{equation*}
    Thus \begin{equation*}
        [(f')^2+f^2]' = 2(f'f'' + ff') = 0
    \end{equation*}
    so $(f')^2+f^2$ is a constant function. From $f(0) = 0$ and $f'(0) = 0$ it follows that the constant is $0$; thus \begin{equation*}
        [f'(x)]^2+[f(x)]^2=0\forall x
    \end{equation*}
    This implies that \begin{equation*}
        f(x) = 0 \forall x
    \end{equation*}
\end{proof}

\begin{theorem}
    If $f$ has a second derivative everywhere and \begin{align*}
        f'' + f &= 0 \\
        f(0) &= a \\
        f'(0) &= b 
    \end{align*}
    then \begin{equation*}
        f = b\cdot \sin + a \cdot \cos
    \end{equation*}
\end{theorem}
\begin{proof}
    Let \begin{equation*}
        g(x) = f(x) - b\sin x - a \cos x
    \end{equation*}
    Then \begin{align*}
        g'(x) &= f'(x) - b\cos x + a \sin x \\
        g''(x) &= f''(x) + b\sin x + a\cos x
    \end{align*}
    Consequently, \begin{align*}
        g'' + g &= 0 \\
        g(0) &= 0 \\
        g'(0) &= 0
    \end{align*}
    which shows by the previous lemma that \begin{equation*}
        0 = g(x) = f(x) - b\sin x - a\cos x, \forall x
    \end{equation*}
\end{proof}


\begin{theorem}
    If $x$ and $y$ are any two numbers, then \begin{align*}
        \sin(x+y) &= \sin x\cos y + \cos x \sin y \\
        \cos(x+y) &= \cos x\cos y - \sin x \sin y
    \end{align*}
\end{theorem}
\begin{proof}
    For any particular $y \in \R$, we can define a function $f$ by \begin{equation*}
        f(x) = \sin(x+y)
    \end{equation*}
    Then $f'(x) = \cos(x+y)$ and $f''(x) = -\sin(x+y)$. Consequently, \begin{align*}
        f'' + f &= 0 \\
        f(0) &= \sin y \\
        f'(0) &= \cos y
    \end{align*}
    It follows from the previous theorem that \begin{equation*}
        f = (\cos y)\cdot \sin + (\sin y) \cdot \cos
    \end{equation*}
    that is \begin{equation*}
        \sin(x+y) = \cos y\sin x+\sin y \cos x,\forall x
    \end{equation*}
    Since any number $y$ could have been chosen to begin with, this proves the first formula for $x$ and $y$.


    Similarly, for any $y \in \R$ define $f(x) = \cos(x+y)$, so $f'(x) = -\sin(x+y)$ and $f''(x) = -\cos(x+y)$. Then $f'' + f = 0$, $f(0) = \cos y$ and $f'(0) = -\sin y$. Then we have that \begin{equation*}
        \cos(x+y) = \cos y\cos x - \sin y \sin x
    \end{equation*}
    proving the second formula.
\end{proof}

\begin{remark}
    Since \begin{equation*}
        \arcsin'(x) = \frac{1}{\sqrt{1-x^2}}, -1 < x < 1
    \end{equation*}
    it follows from \ref{thm:FTC2} that \begin{equation*}
        \arcsin x = \arcsin x - \arcsin 0 = \int_0^x\frac{1}{\sqrt{1-t^2}}dt
    \end{equation*}
    Using this definition of $\arcsin$ we could define $\sin$ as $\arcsin^{-1}$, and the formula for the derivative of an inverse function would show that \begin{equation*}
        \sin'(x) = \sqrt{1-\sin^2 x}
    \end{equation*}
    which could be defined as $\cos x$.
\end{remark}

%
\section*{Appendix B: The Logarithm and Exponential Functions}
\addcontentsline{toc}{section}{Appendix B: The Logarithm and Exponential Functions}
%

    
\begin{definition}
    If $x > 0$, then define \begin{equation}
        \log x := \int_1^x\frac{1}{t}dt
    \end{equation}
\end{definition}

\begin{theorem}
    If $x.y > 0$, then \begin{equation}
        \log(xy) = \log x+ \log y
    \end{equation}
\end{theorem}
\begin{proof}
    Notice first that $\log'(x) = 1/x$, by \ref{thm:FTC1}. Now, choose a number $y > 0$ and let \begin{equation*}
        f(x) = \log(xy)
    \end{equation*}
    Then\begin{equation*}
        f'(x) = \log'(xy)\cdot y = \frac{1}{xy}\cdot y = \frac{1}{x}
    \end{equation*}
    Thus, $f' = \log'$. This means that there is a number $c$ such that $f(x) = \log(x) + c$ for all $x > 0$, that is, \begin{equation*}
        \log(xy) = \log x+c,\;\forall x > 0
    \end{equation*}
    The number $c$ can be evaluated by noting that $\log(1) = 0$, so $\log(1\cdot y) = c$. Thus \begin{equation*}
        \log(xy) = \log x + \log y
    \end{equation*}
    for all $x$. Since this is true for all $y > 0$, the theorem is proved.
\end{proof}

\begin{corollary}
    If $n$ is a natural number and $x > 0$, then \begin{equation}
        \log(x^n) = n\log x
    \end{equation}
\end{corollary}
\begin{proof}
    We proceed by induction on $n \in \N$. If $n = 1$ we simply have $\log(x^1) = 1\cdot \log x$, so the base case holds. Now suppose inductively that there exists $k \geq 1$ such that if $n = k$, \begin{equation*}
        \log(x^k) = k\log x
    \end{equation*}
    Then, observe that by the previous theorem \begin{align*}
        \log(x^{k+1}) &= \log(x^kx) \\
        &= \log(x^k) + \log x \\
        &= k\log x + \log x \tag{by the Induction Hypothesis} \\
        &= (k+1)\log x
    \end{align*}
    as desired. Thus by mathematical induction we conclude that for all $n \geq 1$, $\log(x^n) = n\log x$.
\end{proof}


\begin{corollary}
    If $x,y > 0$, then \begin{equation*}
        \log\left(\frac{x}{y}\right) = \log x - \log y
    \end{equation*}
\end{corollary}
\begin{proof}
    This result follows from the equation \begin{equation*}
        \log x = \log \left(\frac{x}{y}\cdot y\right) = \log\left(\frac{x}{y}\right) + \log y
    \end{equation*}
\end{proof}

\begin{definition}
    The \Emph{exponential function}, $\exp$, is defined as $\log^{-1}$.
\end{definition}

\begin{theorem}
    For all numbers $x$,\begin{equation*}
        \exp'(x) = \exp(x)
    \end{equation*}
\end{theorem}
\begin{proof}
    Observe that \begin{align*}
        \exp'(x) &= (\log^{-1})'(x) \\
        &= \frac{1}{\log'(\log^{-1}(x))} \\
        &= \frac{1}{\frac{1}{\log^{-1}(x)}} \\
        &= \log^{-1}(x) = \exp(x)
    \end{align*}
\end{proof}

\begin{theorem}
    If $x$ and $y$ are any two numbers, then \begin{equation*}
        \exp(x+y) = \exp(x)\cdot \exp(y)
    \end{equation*}
\end{theorem}
\begin{proof}
    Let $x' = \exp(x)$ and $y' = \exp(y)$, so that $x = \log x'$ and $y = \log y'$. Then \begin{equation*}
        x+y = \log x' + \log y' = \log(x'y')
    \end{equation*}
    This means that \begin{equation*}
        \exp(x+y) = x'y' = \exp(x) \cdot \exp(y)
    \end{equation*}
\end{proof}


\begin{definition}
    We define \begin{equation}
        e := \exp(1)
    \end{equation}
    and this is equivalent to the equation \begin{equation}
        1 = \log e = \int_1^e\frac{1}{t}dt
    \end{equation}
    Then, we note that $\exp(x) = [\exp(1)]^x = e^x$ for rational $x$, so we define for any $x \in \R$, \begin{equation}
        e^x =\exp(x)
    \end{equation}
\end{definition}


\begin{definition}
    If $a > 0$, then, for any real number $x$, \begin{equation}
        a^x := e^{x\log a}
    \end{equation}
    If $a = e$ this definition agrees with our previous one.
\end{definition}


\begin{theorem}
    If $a > 0$, then \begin{equation*}
        (1)\;\;(a^b)^c = a^{bc},\;\forall a,b\in\R
    \end{equation*}
    and \begin{equation*}
        (2)\;\;a^1=a\;and\;a^{x+y}=a^x\cdot a^y,\;\forall x,y\in\R
    \end{equation*}
\end{theorem}
\begin{proof}
    First, observe that \begin{align*}
        (a^b)^c &= e^{c\log a^b} \\
        &= e^{c\log e^{b\log a}} \\
        &= e^{cb\log a} \\
        &= a^{bc}
    \end{align*}
    Next, observe that \begin{equation*}
        a^1 = e^{1\log a} = e^{\log a} = a
    \end{equation*}
    and \begin{align*}
        a^{x+y} &= e^{(x+y)\log a} \\
        &= e^{x\log a + y\log a} \\
        &= e^{x\log a} \cdot e^{y\log a} \\
        &= a^x \cdot a^y
    \end{align*}
\end{proof}

\begin{theorem}
    If $f$ is differentiable and \begin{equation*}
        f'(x) = f(x),\;\forall x\in\R
    \end{equation*}
    then there is a number $c$ such that \begin{equation*}
        f(x) = ce^x,\;\forall x\in\R
    \end{equation*}
\end{theorem}
\begin{proof}
    Let $g(x) = f(x)/e^x$, which is possible as $e^x \neq 0$ for all $x$. Then \begin{equation*}
        g'(x) = \frac{e^xf'(x) - f(x)e^x}{(e^x)^2} = 0
    \end{equation*}
    THerefore, there is a number $c$ such that \begin{equation*}
        g(x) = \frac{f(x)}{e^x} = c,\;\forall x
    \end{equation*}
\end{proof}


\begin{theorem}
    For any natural number $n$, \begin{equation}
        \lim\limits_{x\rightarrow \infty}\frac{e^x}{x^n} = \infty
    \end{equation}
\end{theorem}
\begin{proof}
    Step 1. We claim that $e^x > x$ for all $x$, and consequently $\lim\limits_{x\rightarrow \infty}e^x = \infty$.

    For $x \leq 0$ this is immediate. Now, it suffices to show $x > \log x$ for all $x > 0$. If $x < 1$ this clearly holds since $\log x < 0$. If $x > 1$, then $x-1$ is an upper sum for $f(t) = \frac{1}{t}$ on $[1,x]$, so $\log x < x-1 < x$.


    Step 2. We claim $\lim\limits_{x\rightarrow \infty}\frac{e^x}{x} = \infty$. First, note that \begin{equation*}
        \frac{e^x}{x} = \frac{e^{x/2}\cdot e^{x/2}}{\frac{x}{2}\cdot 2} = \frac{1}{2}\left(\frac{e^{x/2}}{\frac{x}{2}}\right)\cdot e^{x/2}
    \end{equation*}
    By Step 1. the expression in parentheses is greater than $1$, and $\lim\limits_{x\rightarrow \infty}e^{x/2} = \infty$; this shows that $\lim\limits_{x\rightarrow \infty}e^x/x = \infty$.


    Step 3. To prove the main claim note that \begin{equation*}
        \frac{e^x}{x^n} = \frac{(e^{x/n})^x}{\left(\frac{x}{n}\right)^n\cdot n^n} = \frac{1}{n^n}\cdot \left(\frac{e^{x/n}}{\frac{x}{n}}\right)^n
    \end{equation*}
    The expression in parentheses becomes arbitrarily large, by Step 2., so the $n$th power certainly becomes arbitrarily large.
\end{proof}


