\documentclass[12pt, a4paper, oneside, openright, titlepage]{book}
\usepackage[utf8]{inputenc}
\raggedbottom
\usepackage{import}


%%%%%%%%%%%%%%%%% Book Formatting Comments:

%%%%%%%%%%%%%%%%%%%%%%%%%%%%%%%%%%%%% for Part

%%%%%%%%%%%%%%%%%%%%%% for chapter

%%%%%%%%%%%%%%%%%%%% for section




%%%%%% PACKAGES %%%%%%%
\usepackage{hyperref}
\hypersetup{
    colorlinks,
    citecolor=black,
    filecolor=black,
    linkcolor=black,
    urlcolor=black
}
\usepackage{amsmath} % Math display options
\usepackage{amssymb} % Math symbols
%\usepackage{amsfonts} % Math fonts
%\usepackage{amsthm}
\usepackage{mathtools} % General math tools
\usepackage{array} % Allows you to write arrays
\usepackage{empheq} % For boxing equations
% \usepackage{mathabx}
% \usepackage{mathrsfs}
\usepackage{nameref}
\usepackage{wrapfig}

\usepackage{soul}
\usepackage[normalem]{ulem}

\usepackage{txfonts}
\usepackage{cancel}
\usepackage[toc, page]{appendix}
\usepackage{titletoc,tocloft}
\setlength{\cftchapindent}{1em}
\setlength{\cftsecindent}{2em}
\setlength{\cftsubsecindent}{3em}
%\setlength{\cftsubsubsecindent}{4em}
\usepackage{titlesec}

%\titleformat{\section}
%  {\normalfont\fontsize{25}{15}\bfseries}{\thesection}%{1em}{}
%\titleformat{\section}
%  {\normalfont\fontsize{20}{15}\bfseries}%{\thesubsection}{1em}{}
%\setcounter{secnumdepth}{1}  
  
  

%\newcommand\numberthis{\refstepcounter{equation}\tag{\theequation}} % For equation labelling
\usepackage[framemethod=tikz]{mdframed}

\usepackage{tikz} % For drawing commutative diagrams
\usetikzlibrary{cd}
\usetikzlibrary{calc}
\tikzset{every picture/.style={line width=0.75pt}} %set default line width to 0.75p

\usepackage{datetime}
\usepackage[margin=1.5in]{geometry}
\setlength{\parskip}{1em}
\usepackage{makeidx}         % allows index generation
\usepackage{graphicx}       % standard LaTeX graphics tool
\usepackage{multicol}        % used for the two-column index
\usepackage[bottom]{footmisc}% places footnotes at page bottom

\usepackage{newtxtext}       % 
\usepackage{newtxmath}       % selects Times Roman as basic font
\usepackage{float}
\usepackage{fancyhdr}
\setlength{\headheight}{15pt} 
\pagestyle{fancy}
\lhead[\leftmark]{}
\rhead[]{\leftmark}

%\usepackage{enumitem}

\usepackage{url}
\allowdisplaybreaks

%%%%%% ENVIRONMENTS %%%
\definecolor{purp}{rgb}{0.29, 0, 0.51}
\definecolor{bloo}{rgb}{0, 0.13, 0.80}



%%\newtheoremstyle{note}% hnamei
%{3pt}% hSpace above
%{3pt}% hSpace belowi
%{}% hBody fonti
%{}% hIndent amounti
%{\itshape}% hTheorem head fonti
%{:}% hPunctuation after theorem headi
%{.5em}% hSpace after theorem headi
%{}% hTheorem head spec (can be left empty, meaning ‘normal’)i





% %%%%%%%%%%%%% THEOREM DEFINITIONS

\spnewtheorem{axiom}{Axiom}[chapter]{\bfseries}{\itshape}


\spnewtheorem{construction}{Construction}[chapter]{\bfseries}{\itshape}

\spnewtheorem{props}{Properties}[chapter]{\bfseries}{\itshape}


\renewcommand{\qedsymbol}{$\blacksquare$}


\numberwithin{equation}{section}

\newenvironment{qest}{
    \begin{center}
        \em
    }
    {
    \end{center}
    }

%%%%%% MACROS %%%%%%%%%
%% New Commands
\newcommand{\ip}[1]{\langle#1\rangle} %%% Inner product
\newcommand{\abs}[1]{\lvert#1\rvert} %%% Modulus
\newcommand\diag{\operatorname{diag}} %%% diag matrix
\newcommand\tr{\mbox{tr}\.} %%% trace
\newcommand\C{\mathbb C} %%% Complex numbers
\newcommand\R{\mathbb R} %%% Real numbers
\newcommand\Z{\mathbb Z} %%% Integers
\newcommand\Q{\mathbb Q} %%% Rationals
\newcommand\N{\mathbb N} %%% Naturals
\newcommand\F{\mathbb F} %%% An arbitrary field
\newcommand\ste{\operatorname{St}} %%% Steinberg Representation
\newcommand\GL{\mathbf{GL}} %%% General Linear group
\newcommand\SL{\mathbf{SL}} %%% Special linear group
\newcommand\gl{\mathfrak{gl}} %%% General linear algebra
\newcommand\G{\mathbf{G}} %%% connected reductive group
\newcommand\g{\mathfrak{g}} %%% Lie algebra of G
\newcommand\Hbf{\mathbf{H}} %%% Theta fixed points of G
\newcommand\X{\mathbf{X}} %%% Symmetric space X
\newcommand{\catname}[1]{\normalfont\textbf{#1}}
\newcommand{\Set}{\catname{Set}} %%% Category set
\newcommand{\Grp}{\catname{Grp}} %%% Category group
\newcommand{\Rmod}{\catname{R-Mod}} %%% Category r-modules
\newcommand{\Mon}{\catname{Mon}} %%% Category monoid
\newcommand{\Ring}{\catname{Ring}} %%% Category ring
\newcommand{\Topp}{\catname{Top}} %%% Category Topological spaces
\newcommand{\Vect}{\catname{Vect}_{k}} %%% category vector spaces'
\newcommand\Hom{\mathbf{Hom}} %%% Arrows

\newcommand{\map}[2]{\begin{array}{c} #1 \\ #2 \end{array}}

\newcommand{\Emph}[1]{\textbf{\ul{\emph{#1}}}}




%% Math operators
\DeclareMathOperator{\ran}{Im} %%% image
\DeclareMathOperator{\aut}{Aut} %%% Automorphisms
\DeclareMathOperator{\spn}{span} %%% span
\DeclareMathOperator{\ann}{Ann} %%% annihilator
\DeclareMathOperator{\rank}{rank} %%% Rank
\DeclareMathOperator{\ch}{char} %%% characteristic
\DeclareMathOperator{\ev}{\bf{ev}} %%% evaluation
\DeclareMathOperator{\sgn}{sign} %%% sign
\DeclareMathOperator{\id}{Id} %%% identity
\DeclareMathOperator{\supp}{Supp} %%% support
\DeclareMathOperator{\inn}{Inn} %%% Inner aut
\DeclareMathOperator{\en}{End} %%% Endomorphisms
\DeclareMathOperator{\sym}{Sym} %%% Group of symmetries


%% Diagram Environments
\iffalse
\begin{center}
    \begin{tikzpicture}[baseline= (a).base]
        \node[scale=1] (a) at (0,0){
          \begin{tikzcd}
           
          \end{tikzcd}
        };
    \end{tikzpicture}
\end{center}
\fi




\newdateformat{monthdayyeardate}{%
    \monthname[\THEMONTH]~\THEDAY, \THEYEAR}
%%%%%%%%%%%%%%%%%%%%%%%

%%% Specific Macros %%%


%%%%%% BEGIN %%%%%%%%%%


\begin{document}

%%%%%% TITLE PAGE %%%%%%

\begin{titlepage}
    \centering
    \scshape
    \vspace*{\baselineskip}
    \rule{\textwidth}{1.6pt}\vspace*{-\baselineskip}\vspace*{2pt}
    \rule{\textwidth}{0.4pt}
    
    \vspace{0.75\baselineskip}
    
    {\LARGE Complex Analysis: A Complete Guide}
    
    \vspace{0.75\baselineskip}
    
    \rule{\textwidth}{0.4pt}\vspace*{-\baselineskip}\vspace{3.2pt}
    \rule{\textwidth}{1.6pt}
    
    \vspace{2\baselineskip}
    Complex Analysis \\
    \vspace*{3\baselineskip}
    \monthdayyeardate\today \\
    \vspace*{5.0\baselineskip}
    
    {\scshape\Large Elijah Thompson, \\ Physics and Math Honors\\}
    
    \vspace{1.0\baselineskip}
    \textit{Solo Pursuit of Learning}
    \vfill
    \enlargethispage{1in}
    \begin{figure}[b!]
    \makebox[\textwidth]{\includegraphics[width=\paperwidth, height =10cm]{../../Crab.jpg}}
    \end{figure}
\end{titlepage}

%%%%%%%%%%%%%%%%%%%%%%%
\tableofcontents


%%%%%%%%%%%%%%%%%%%%%%%%%%%%%%%%%%%%% Part 1.
\part{Part 1}

%%%%%%%%%%%%%%%%%%%%%% Chapter 1.1
\chapter{The Complex Plain and Basic Functions}




%%%%%%%%%%%%%%%%%%%% Section 1.1.1
\section{Complex Numbers}

The complex numbers, $\C$, consist of all formul sums $z = x+iy$, for $x,y \in \R$, where $i^2 = -1$ is the root of $x^2 + 1 = 0$. Then, for multiplication we proceed by $z\cdot w = (a+ib)(x+iy) = (ax-by)+i(xb+ay)$.

Gauss concieved of $\C$ as $\R^2$ with a binary operation $*$, where $(a,b)*(x,y) = (ax-by,xb+ay)$. Then, we observe that $(1,0)*(a,b) = (a,b)$, so $(1,0)$ acts as $1$. Moreover, $(0,1)*(0,1) = (-1,0) = -(1,0)$. 


The matrix model of $\C$ is $$\C = \left\{\begin{bmatrix} a & -b // b & a \end{bmatrix}: a,b \in \R\right\}$$

In terms of extension fields, we can consider $\C$ to be $\R[x]/(x^2+1)$. 

\begin{defn}
    If $z = x+iy$, with $x,y \in \R$, then we define $\mathscr{R}e(z) = x$ and $\mathscr{I}m(z) = y$.
\end{defn}


\begin{defn}
    If $z = x+iy$, we define the \Emph{conjugate} of $z$ to be $\overline{z} x-iy$.
\end{defn}

\begin{props}
    Let $z,w \in \C$. \begin{itemize}
        \item $\overline{z+w} = \overline{z}+\overline{w}$
        \item $\overline{zw} = \overline{z}\cdot \overline{w}$
        \item $\overline{\overline{z}} = z$
        \item $\overline{z} = 0$ if and only if $z = 0$
        \item $\mathscr{R}e(z) = \frac{z+\overline{z}}{2}$
        \item $\mathscr{I}m(z) = \frac{z-\overline{z}}{2i} = \frac{i(\overline{z}-z)}{2}$
    \end{itemize}
\end{props}

\begin{prop}
    If $z \neq 0$, then $z^{-1} = \frac{\overline{z}}{z\overline{z}}$.
\end{prop}


\begin{defn}
    Let $z \in \C$. Then the \Emph{modulus}, $|\cdot |$ of $z=a+ib$ (the norm), is the length of $z$ as a vector: \begin{equation*}
        |z| = \sqrt{a^2+b^2} = \sqrt{z\overline{z}}
    \end{equation*}
\end{defn}


\subsection{Geometry of the Complex Numbers}

The complex numbers, $\C = \{a+ib:a,b \in \R\}$, can be considered as a plane of points, or we can consider the complex numbers as vectors in the plane eminating from $0$. 
\begin{center}
	\begin{tikzpicture}[x=0.75pt,y=0.75pt,yscale=-1,xscale=1]
%uncomment if require: \path (0,398); %set diagram left start at 0, and has height of 398

%Straight Lines [id:da39464285181286485] 
\draw [color={rgb, 255:red, 255; green, 0; blue, 0 }  ,draw opacity=1 ]   (200,310) -- (287.66,212.27) ;
\draw [shift={(289,210.78)}, rotate = 491.89] [color={rgb, 255:red, 255; green, 0; blue, 0 }  ,draw opacity=1 ][line width=0.75]    (10.93,-3.29) .. controls (6.95,-1.4) and (3.31,-0.3) .. (0,0) .. controls (3.31,0.3) and (6.95,1.4) .. (10.93,3.29)   ;
%Straight Lines [id:da13724064512600265] 
\draw  [dash pattern={on 0.84pt off 2.51pt}]  (289,210.78) -- (289.67,309.44) ;
%Shape: Arc [id:dp896874676115538] 
\draw  [draw opacity=0][fill={rgb, 255:red, 74; green, 144; blue, 226 }  ,fill opacity=1 ][dash pattern={on 0.84pt off 2.51pt}] (211.58,297.56) .. controls (212.75,299.62) and (213.81,301.84) .. (214.75,304.19) .. controls (215.48,306.03) and (216.1,307.86) .. (216.61,309.67) -- (199.73,310.18) -- cycle ; \draw  [dash pattern={on 0.84pt off 2.51pt}] (211.58,297.56) .. controls (212.75,299.62) and (213.81,301.84) .. (214.75,304.19) .. controls (215.48,306.03) and (216.1,307.86) .. (216.61,309.67) ;
%Shape: Axis 2D [id:dp8729881414013165] 
\draw  (182.33,310.18) -- (356.33,310.18)(199.73,166.78) -- (199.73,326.11) (349.33,305.18) -- (356.33,310.18) -- (349.33,315.18) (194.73,173.78) -- (199.73,166.78) -- (204.73,173.78)  ;
%Straight Lines [id:da5981962432151724] 
\draw [color={rgb, 255:red, 74; green, 144; blue, 226 }  ,draw opacity=1 ]   (199.73,310.18) -- (130.17,214.18) ;
\draw [shift={(129,212.56)}, rotate = 414.07] [color={rgb, 255:red, 74; green, 144; blue, 226 }  ,draw opacity=1 ][line width=0.75]    (10.93,-3.29) .. controls (6.95,-1.4) and (3.31,-0.3) .. (0,0) .. controls (3.31,0.3) and (6.95,1.4) .. (10.93,3.29)   ;
%Shape: Arc [id:dp8546612577752639] 
\draw  [draw opacity=0][fill={rgb, 255:red, 243; green, 19; blue, 19 }  ,fill opacity=1 ][dash pattern={on 0.84pt off 2.51pt}] (194.73,303.29) .. controls (196.25,302.4) and (198.06,301.89) .. (200,301.89) .. controls (205.33,301.89) and (209.65,305.75) .. (209.67,310.53) -- (200,310.56) -- cycle ; \draw  [dash pattern={on 0.84pt off 2.51pt}] (194.73,303.29) .. controls (196.25,302.4) and (198.06,301.89) .. (200,301.89) .. controls (205.33,301.89) and (209.65,305.75) .. (209.67,310.53) ;

% Text Node
\draw (354,302.4) node [anchor=north west][inner sep=0.75pt]    {$\mathscr{R} e$};
% Text Node
\draw (184.5,151.07) node [anchor=north west][inner sep=0.75pt]    {$\mathscr{I} m$};
% Text Node
\draw (290,199.51) node [anchor=north west][inner sep=0.75pt]  [font=\scriptsize]  {$z=a+ib$};
% Text Node
\draw (220,312.18) node [anchor=north west][inner sep=0.75pt]  [font=\scriptsize]  {$a=|z|\cos \theta $};
% Text Node
\draw (292.67,254.84) node [anchor=north west][inner sep=0.75pt]  [font=\scriptsize]  {$ib=i|z|\sin \theta $};
% Text Node
\draw (218,294.51) node [anchor=north west][inner sep=0.75pt]  [font=\scriptsize]  {$\theta $};
% Text Node
\draw (202.67,286.62) node [anchor=north west][inner sep=0.75pt]  [font=\scriptsize]  {$\beta $};
% Text Node
\draw (100,200.29) node [anchor=north west][inner sep=0.75pt]  [font=\scriptsize]  {$w=x+iy$};

\draw   (122.72, 166.5) circle [x radius= 5, y radius= 5]   ;
\draw   (123.55, 165.96) circle [x radius= 5, y radius= 5]   ;
\draw   (288.47, 51.73) circle [x radius= 5, y radius= 5]   ;
\draw   (372.18, 109) circle [x radius= 5, y radius= 5]   ;
\draw   (537.1, 23.04) circle [x radius= 5, y radius= 5]   ;
\draw   (249.38, 297.62) circle [x radius= 5, y radius= 5]   ;
\draw   (376.5, 243.59) circle [x radius= 5, y radius= 5]   ;
\draw   (414.3, 248.15) circle [x radius= 5, y radius= 5]   ;
\end{tikzpicture}	
\end{center}


Noting this geometric picture, we can write $z = |z|\cos\theta+i|z|\sin\theta = |z|(\cos\theta + i\sin\theta)$. Suppose we had another complex number $w = |w|(\cos\beta+i\sin\beta)$. Then we observe that \begin{align*}
    zw &= |z||w|(\cos\theta+i\sin\theta)(\cos\beta+i\sin\beta) \\
    &= |zw|(\cos\theta\cos\beta -\sin\theta\sin\beta + i(\sin\theta\cos\beta+\cos\theta\sin\beta)) \\
    &= |zw|(\cos(\theta+\beta)+i\sin(\theta+\beta))
\end{align*}
so complex multiplication aligns with angle addition in the plane.


\begin{defn}
    Define \Emph{Euler's Formula} \begin{equation*}
        e^{i\theta} = \cos\theta+i\sin\theta
    \end{equation*}
\end{defn}

From our above work we have that \begin{equation*}
    zw = (|z|e^{i\theta})(|w|e^{i\beta}) = |zw|e^{i(\theta+\beta)}
\end{equation*}
so \begin{equation*}
    e^{i\theta}e^{i\beta} = e^{i(\theta+\beta)}
\end{equation*}
The conjugate of $e^{i\theta}$ is \begin{equation*}
    \overline{e^{i\theta}} = \overline{\cos\theta+i\sin\theta} = \cos\theta-i\sin\theta=\cos(-\theta)+i\sin(-\theta) = e^{-i\theta}
\end{equation*}


\begin{prop}
    $e^{i\theta} = e^{i\beta}$ if and only if $\theta = \beta + 2\pi k$ for some $k \in \Z$.
\end{prop}

Then we have that $e^{i\theta} = e^{-i\theta}$ if and only if $\theta = \pi k$ for some $k \in \Z$.

\begin{defn}
    Let $z \in \C$. The \Emph{argument} of $z = |z|e^{i\theta}$ is $\arg(z) = \{\theta+2\pi k:k \in \Z\}$, and the \Emph{principal argument} of $z$, $\text{Arg}(z) = \theta_0$, where $z = |z|e^{i\theta_0}$ and $\theta_0 \in (-\pi,\pi]$.
\end{defn}


\begin{eg}
    Consider $z = -42-42i$. Then $|z| = 42\sqrt{2}$, and $\text{Arg}(z) = -\frac{3\pi}{4}$, so $z = 42\sqrt{2}e^{-i\frac{3\pi}{4}}$, and $\arg(z) = -\frac{3\pi}{4} + 2\pi\Z$.
\end{eg}

\begin{props}
    For $z,w \in \C$, $\arg(zw) = \arg(z)+\arg(w)$, but $\text{Arg}(zw) \neq \text{Arg}(z)+\text{Arg}(w)$.
\end{props}

\begin{thm}[DeMoivre's Theorem] \label{thm:DeMoivre}
    For all $n \in \Z$, \begin{equation*}
        (e^{i\theta})^n = e^{in\theta}
    \end{equation*}
\end{thm}
\begin{proof}
    We proceed by induction on $n \in \N$. If $n=1$ then $(e^{i\theta})^1 = e^{i1\theta}$, so the base case holds. Now, suppose that the claim holds for some $n \geq 1$. It follows that \begin{align*}
        (e^{i\theta})^{n+1} &= (e^{i\theta})^1(e^{i\theta})^n \\ 
        &= e^{i\theta}e^{in\theta} \tag{by I.H} \\
        &= e^{i(\theta+n\theta)} \\
        &= e^{i(n+1)\theta}
    \end{align*}
    completing the proof.
\end{proof}



\begin{defn}
    Suppose that $n \in \N$, $w,z \in \C$ such that $z^n = w$, then $z$ is said to be an \Emph{$n$th root of $w$}. Moreover, the set of all $n$th roots is dentoed $w^{1/n} \neq \sqrt[n]{w}$.
\end{defn}

Let $z^n = w$, where $z = \rho e^{i\theta}$. Then it follows that \begin{align*}
    (\rho e^{i\theta})^n &= w \\
    \rho^ne^{in\theta} &= |w|e^{i\text{Arg}(w)} \tag{by \ref{thm:DeMoivre}}
\end{align*}
This gives the two equations $\rho^n = |w|$ and $e^{in\theta} = e^{i\text{Arg}(w)}$, so $\rho = \sqrt[n]{|w|}$, and \begin{align*}
    n\theta &= \text{Arg}(w)+2\pi k \\
    \theta &= \frac{\text{Arg}(w)}{n}+\frac{2\pi k}{n}
\end{align*}
This gives the following result: 
\begin{cor}
    $$w^{1/n} = \left\{\sqrt[n]{|w|}e^{i\frac{\text{Arg}(w)}{n}}e^{i\frac{2\pi k}{n}}: k \in \Z\right\}$$
\end{cor}
\begin{defn}
    If $w = 1$, we have that \begin{equation*}
        1^{1/n} = \left\{e^{i\frac{2\pi k}{n}}:k\in \Z\right\}
    \end{equation*}
    These are the \Emph{$n$th roots of unity}.
\end{defn}

Then we observe that for any $w \in \C$, $$w^{1/n} = \sqrt[n]{|w|}e^{i\frac{\text{Arg}(w)}{n}}1^{1/n}$$


\begin{eg}
    Consider the fourth roots of $81i$, so $(81i)^{1/4}$. Then we have that \begin{align*}
        (81i)^{1/4} &= \sqrt[4]{81}e^{i\frac{\pi}{8}}1^{1/4} \\
        &= 3e^{i\frac{\pi}{8}}\{1,i,-1,-i\}
    \end{align*}
\end{eg}

\begin{eg}
    Let $w = \exp\left(\frac{2\pi i}{6}\right)$. Then \begin{align*}
        1^{1/6} &= \{z \in \C:z^6 = 1\} = \{w,w^2,w^3,w^4,w^5,w^6 = 1\}
    \end{align*}
    Note $w = e^{i\pi/3} = \cos(\pi/3)+i\sin(\pi/3) = \frac{1+i\sqrt{3}}{2}$. Now, if we consider the polynomial $f(z) = z^6 - 1$, we now know six roots for this polynomial. Then we can factor \begin{equation*}
        f(z) = \prod_{i=1}^6(z-w^i)
    \end{equation*}
    In short, to solve $z^n = \rho$, we take the $n$th roots of $\rho$, $\rho^{1/n}$. 
\end{eg}



\begin{eg}
    Consider $z^2+bz+c = 0$. Then completing the square we obtain $z \in \left\{\frac{-b+ (b^2-4c)^{1/2}}{2}\right\}$, where \begin{equation*}
        \left(\frac{b^2-4c}{2}\right)^{1/2} = \left\{\begin{array}{ccc} \sqrt{\left|\frac{b^2-4c}{2}\right|} & -\sqrt{\left|\frac{b^2-4c}{2}\right|} & if\;b^2-4c \geq 0 \\ \sqrt{\left|\frac{b^2-4c}{2}\right|}i & -\sqrt{\left|\frac{b^2-4c}{2}\right|}i & if\; b^2-4c < 0 \end{array}\right\}
    \end{equation*}
\end{eg}


%%%%%%%%%%%%%%%%%%%% Section 1.1.2
\section{Local Inverses and Branch-cut}

\begin{defn}    
    Let $z,w \in \C$, then the line segment from $z$ to $w$ is \begin{equation*}
        [z,w] = \{z+t(w-z):0\leq t \leq 1\}
    \end{equation*}
    where we also allow $z$ or $w$ to be plus or minus infinity.
\end{defn}

\begin{defn}
    The \Emph{negative slit plane}, $\C^-$, is defined by $\C^- = \C\backslash(-\infty,0]$, and the \Emph{positive slit plane}, $\C^+$, is defined by $\C^+ = \C\backslash[0,\infty)$. In general, we define \begin{equation*}
        C^{\alpha} = \C\backslash[0,e^{i\alpha}\infty)
    \end{equation*}
    to denote the exclusion of the ray along the $\alpha$th angle from the positive real axis.
\end{defn}

\begin{qst}
    What is a function?
\end{qst}

\begin{defn}
    If $f:S\rightarrow \C$ is a function, and $U \subseteq S$, then $f\vert_U:U\rightarrow \C$ is defined by $f\vert_U(z) = f(z)$ for all $z \in U$.
\end{defn}

We may have the case that $f$ is not injective (so it cannot be inverted). But, for a smart choice of $U$, we may have that $f\vert_U$ is one-to-one, and hence invertible. Such a restriction is known as a \Emph{local inverse} for $f$.

Rigourously, a \Emph{branch cut} is a curve in the complex plane such that it is possible to define a single analytic branch (sheets of a multivalued function) of a multivalued function on the plane minus that curve. That is, a branch is a way of making the multivalued function single valued, and in the context of determining inverses a branch is a choice of inverse.

\begin{eg}
    For $f(z) = z^n$, then for $U = \left\{z \in \C: -\frac{\pi}{n}<\text{Arg}(z) < \frac{\pi}{n}\right\}$, $f\vert_{U}$ is invertible, and $f\vert_{U}^{-1}$ is called the \Emph{principal branch}. $f\vert_U^{-1}$ is a choice of the $n$th root of $w \in \C^-$. 
\end{eg}


\begin{defn}
    The \Emph{$\alpha$-argument} for $\alpha \in \R$ is denoted $\text{Arg}_{\alpha}:\C^{\times}\rightarrow (\alpha,\alpha+2\pi)$. In particular, for each $z \in \C^{\times}$ we define $\text{Arg}_{\alpha} \in \arg(z)$ such that $z \in (\alpha,\alpha+2\pi)$
\end{defn}

We can give branch cuts for the $n$th root function which delete the ray at standard angle $\alpha$. These correspond to local inverse functions $f(z) = z^n$ restricted to $\{z \in \C^{\times}:\arg(z) = (\alpha/n,(\alpha+2\pi)/2)+2\pi\Z\}$.


\subsection{Square-Root Function}

If we have $z^2 = w$, this is equivalent to $(|z|e^{i\theta})^2 = |w|e^{i\beta}$, so $|z|^2=|w|$ and $e^{i2\theta} = e^{i\beta}$. Then $|z| = \sqrt{|w|}$, and $\theta = \frac{\beta}{2} + \pi k$ for $k \in \Z$. Then our solutions are \begin{equation*}
    z = \sqrt{|w|}e^{i(\beta/2+\pi k)} = \sqrt{|w|}e^{i\beta/2}e^{i\pi k} = \sqrt{|w|}e^{i\beta/2}\cos(\pi k)
\end{equation*}
Thus, in general \begin{equation*}
    z = \sqrt{|w|}e^{i\text{Arg}(w)/2}(-1)^k = \pm \sqrt{|w|}e^{i\text{Arg}(w)/2}
\end{equation*}
and \begin{equation*}
    w^{1/2} = \{\sqrt{|w|}e^{i\text{Arg}(w)/2}, -\sqrt{|w|}e^{i\text{Arg}(w)/2}\}
\end{equation*}
In general we have \begin{equation*}
    w^{1/n} = \{\sqrt[n]{w},\zeta\sqrt[n]{w},...,\zeta^{n-1}\sqrt[n]{w}\}
\end{equation*}
where $\sqrt[n]{w} = \sqrt[n]{|w|}\exp\left(\frac{i\text{Arg}(w)}{n}\right)$ is the principal root, and $\zeta = e^{\frac{2\pi i}{n}}$ is an $n$th root of unity. The principal root is the local inverse for the principal branch $U = 
\{z:-\pi/n < \text{Arg}(z) < \pi/n\}$. 


%%%%%%%%%%%%%%%%%%%% Section 1.1.3
\section{Complex Exponential}

\begin{defn}
    We define the complex exponential for $z \in \C$ to be \begin{equation*}
        e^z = e^{\mathscr{R}e(z)}e^{i\mathscr{I}m(z)} = e^{\mathscr{R}e(z)}(\cos(\mathscr{I}m(z))+i\sin(\mathscr{I}m(z)))
    \end{equation*}
\end{defn}

\begin{props}
    Let $z = x+iy, w = a+ib \in \C$. \begin{itemize}
        \item $e^ze^w = e^{z+w}$
        \item $|e^{x+iy}| = |e^x||e^{iy}| = e^x$, which is never zero so the complex exponential is never zero. that is,
        \item $e^z \neq 0$ for all $z \in \C$.
        \item $\arg(e^z) = \arg(e^xe^{iy}) = y+2\pi\Z$.
    \end{itemize}
\end{props}


\subsection{Failure to Inject}

If $e^{z_1} = e^{z_2}$, then $e^{x_1}e^{iy_1} = e^{x_2}e^{iy_2}$, so $x_1 = x_2$ and $y_1 \in y_2 + 2\pi\Z$. Thus, $e^z$ has a $2\pi i$-periodicity; $e^z = e^{z+2\pi ik}$ for $k \in \Z$. To make the complex exponential, we must restrict the domain to some horizontal strip of height at most $2\pi$ (with endpoints not included). In particular, if we take $U = \{x+iy: -\pi < y < \pi\}$ we obtain the branch $\C^-$, and branch cut $(-\infty,0]$. Then, suppose we write $e^z = w = |w|e^{i\text{Arg}(w)}$. Then a solution is $e^x = |w|$, and $y = \text{Arg}(w)$. We can then define \begin{equation*}
    \text{Log}(w) = \ln|w| +i\text{Arg}(w) = z = x+iy
\end{equation*}
for $w \in \C^-$, which is the branch cut to the multivalued log \begin{equation*}
    \log(z) = \ln|z|+i\arg(z)
\end{equation*}
taking the restriction $U$ in the range.



%%%%%%%%%%%%%%%%%%%% Section 1.1.4
\section{Sine, Cosine, Cosh, Sinh}

Recall $e^{i\theta} = \cos\theta +i\sin\theta$ and $e^{-i\theta} = \cos\theta-i\sin\theta$. Then we have that \begin{equation*}
    e^{i\theta}+e^{-i\theta}=2\cos\theta
\end{equation*}
and \begin{equation*}
    e^{i\theta}-e^{-i\theta} = 2i\sin\theta
\end{equation*}
Thus, we can obtain formulas for $\sin$ and $\cos$, $\theta \in \C$: \begin{equation*}
    \cos\theta = \frac{e^{i\theta}+e^{-i\theta}}{2}
\end{equation*}
and \begin{equation*}
    \sin\theta = \frac{e^{i\theta}-e^{-i\theta}}{2i}
\end{equation*}
Then we define:

\begin{defn}
    We define the complex sine and cosine, $z \in \C$, by \begin{equation}
        \boxed{\cos z = \frac{e^{iz}+e^{-iz}}{2}}
    \end{equation}
    and \begin{equation}
        \boxed{\sin z = \frac{e^{iz}-e^{-iz}}{2i}}
    \end{equation}
\end{defn}

Observe that \begin{equation*}
    e^x = \underbrace{\frac{1}{2}(e^x+e^{-x})}_{\cosh(x)} + \underbrace{\frac{1}{2}(e^x-e^{-x})}_{\sinh(x)}
\end{equation*}
\begin{defn}
    We define the complex hyperbolic sine and hyperbolic cosine, $z \in \C$, by \begin{equation}
        \boxed{\cosh z = \frac{e^{z}+e^{-z}}{2}}
    \end{equation}
    and \begin{equation}
        \boxed{\sinh z = \frac{e^{z}-e^{-z}}{2}}
    \end{equation}
\end{defn}

Then we have the identities \begin{equation*}
    \cosh z = \cos(iz), \sinh z = -i\sin(iz)
\end{equation*}
and \begin{equation*}
    \cos(z) = \cosh(iz), \sin z = -i\sinh(iz)
\end{equation*}

\subsection{Complex Cosine is Not Bounded}

Observe \begin{equation*}
    \cos(z) = \cos(x+iy) = \frac{e^{ix-y}+e^{-ix+y}}{2} = \frac{e^{ix}e^{-y}+e^{-ix}e^y}{2}
\end{equation*}
Now, using angle formulas we have \begin{align*}
    \cos(z) &= \cos(x+iy) \\
    &= \cos(x)\cos(iy)-\sin(x)\sin(iy) \\
    &= \cos(x)\cosh(y)-i\sin(x)\sinh(y)
\end{align*}
so \begin{equation*}
    |\cos z|^2 = \cos^2x\cosh^2y+\sin^2x\sinh^2y = \cos^2x+\sinh^2y
\end{equation*}
so as $\cosh$ and $\sinh$ are unbounded, so is complex $\cos$. 

\begin{claim}
    \begin{align*}
        \cos(z+w) &= \cos(z)\cos(w)-\sin(z)\sin(w) \\
        \sin(z+w) &= \sin(z)\cos(w)+\sin(w)\cos(z) 
    \end{align*}
    and \begin{align*}
        \cosh(z+w) &= \sinh z\sinh w + \cosh z\cosh w \\
        \sinh(z+w) &= \sinh z\cosh w + \cosh z\sinh w
    \end{align*}
\end{claim}


\begin{claim}
    $\cos^2z+\sin^2z = 1$
\end{claim}
\begin{proof}
    First, observe \begin{equation*}
        \cos^2z = \left[\frac{1}{2}(e^{iz}+e^{-iz})\right]^2 = \frac{1}{4}(e^{2iz}+2+e^{-2iz})
    \end{equation*}
    and \begin{equation*}
        \sin^2z = \left[\frac{1}{2i}(e^{iz}-e^{-iz})\right]^2 = \frac{-1}{4}(e^{2iz}-2+e^{-2iz})
    \end{equation*}
    Hence, indeed, $\cos^2z + \sin^2z = 1$.
\end{proof}


%%%%%%%%%%%%%%%%%%%% Section 1.1.5
\section{Power Functions}

\begin{defn}
    Let $\alpha \in \C$ be arbitrary. For $z \in \C^{\times}$ we define the power function $z^{\alpha}$ to be the multivalued function \begin{equation*}
        z^{\alpha} = e^{\alpha\log z}
    \end{equation*}
    Thus, the values of $z^{\alpha}$ are given by \begin{align*}
        z^{\alpha} &= e^{\alpha(\log|z| +i\arg(z))} \\
        &= e^{\alpha\text{Log}(z)}e^{2\pi i\alpha m}, m = 0, \pm 1,\pm 2,...
    \end{align*}
\end{defn}

Consequently, the various values of $z^{\alpha}$ are obtained by multiplying the principal value $e^{\alpha\text{Log}|z|}$ by the integer power of $e^{2\pi i\alpha}$. Consequently, if $\alpha$ is itself an integer $e^{2\pi i\alpha} = 1$, and the power function is single valued and equal to the principal value, $e^{\alpha\text{Log}|z|}$. If $\alpha = 1/n$, for $n \in \N$, then the factor is precisely the $n$th roots of unity, and $z^{1/n}$ are the $n$th roots of unity of $z$.

It is important to note that the usual algebraic rules do not apply to power functions when they are multivalued. 

To haze the power function move continuously with $z$ we make the branch cut $[0,\infty)$. Then we define a continuous branch on $\C^+$ to be \begin{equation*}
    w = r^{\alpha}e^{i\alpha \theta},\; \text{ for }\;z = re^{i\theta}, 0<\theta < 2\pi
\end{equation*}
At the top edge of the slit, $\theta = 0$, we have the usual power function $r^{\alpha} = e^{\alpha\text{Log}r}$. At the bottom of the slit, $\theta = 2\pi$, we have the function $r^{\alpha}e^{2\pi i\alpha}$. For a fixed $r$, as $\theta$ ranges the values of $w= r^{\alpha}e^{i\theta\alpha}$ move continuously. Thus, the values of this branch of $z^{\alpha}$ on the bottom edge are $e^{2\pi i\alpha}$ times the values at the top edge. This multiple, $e^{2\pi i\alpha}$, is called the \Emph{phase factor} of $z^{\alpha}$ at $z = 0$. 

For any other choice of branch, $w=r^{\alpha}e^{i\alpha(\theta+2\pi m)}$, the same phase factor is observed. 

\begin{lem}[Phase Change Lemma]\label{lem:phasechange}
    Let $g(z)$ be a single-valued function that is defined and continuous near $z_0$. For any continuously varying branch of $(z-z_0)^{\alpha}$ the function $f(z) = (z-z_0)^{\alpha}g(z)$ is multiplied by the phase factor $e^{2\pi i \alpha}$ when $z$ traverses a complete circle about $z_0$ in the positive direction.
\end{lem}




%%%%%%%%%%%%%%%%%%%%%% Chapter 1.2
\chapter{Analytic Functions}

%%%%%%%%%%%%%%%%%%%% Section 1.2.1
\section{Basic Analysis and Topology}


\begin{defn}
    A \Emph{sequence} of complex numbers is a function $f:\N\rightarrow \C$, where $f(i) = a_i$ for each $i \in \N$. We often write this sequence as $\{a_n\}_{n=1}^{\infty}$.
\end{defn}

\begin{defn}
    We say a sequence $\{a_n\}_{n=1}^{\infty}$ converges to a limit $L \in \C$, $a_n\rightarrow L$, as $n$ approaches infinity, $n\rightarrow \infty$, if for each $\varepsilon > 0$, there exists $N \in \N$ such that for $n \geq N$, $|a_n - L| < \varepsilon$. We also write \begin{equation*}
        \lim\limits_{n\rightarrow \infty}a_n = L
    \end{equation*}
\end{defn}

That is, if we pick a particular limiting value for the limit value, we can find a point past which the tail of the sequence is between limiting value and our limit.

\begin{defn}
    A sequence $\{a_n\}_{n=1}^{\infty} \subseteq \C$ is said to be \Emph{bounded} if there exists $M \in \R$ such that for all $n \in \N$, $|a_n| < M$ (where $|\cdot|$ is the modulus).
\end{defn}

That is, all points in the sequence exist in an $M$ radius disk of the origin in the complex plane. 

\begin{eg}
    Consider a sequence $a_n = (e^{ir})^n$, where $r \notin \Q$. Then $|a_n| = 1$, for all $n$. Then $\lim\limits_{n\rightarrow \infty}|a_n| = 1$, but the sequence itself does not cnverge in $\C$.
\end{eg}


\begin{thm}
    Suppose $s_n\rightarrow s$ and $t_n\rightarrow t$ for sequences in $\C$. Then \begin{itemize}
        \item $s_n+t_n\rightarrow s+l$
        \item $cs_n \rightarrow cs$, for all $c \in \C$
        \item $s_nt_n \rightarrow st$
        \item $s_n/t_n\rightarrow s/t$, provided $t \neq 0$.
    \end{itemize}
\end{thm}

We remark that the squeeze theorem only applies for real-sequences, as it relies on the ordering of the reals. Moreover, the theorem that ``A bounded monotone sequence of real numbers converges" also does not hold, again due to ordering.

\begin{defn}
    Let $(a_n)$ be a sequence. Let $n_1 < n_2 < n_3 < ...$ be an increasing sequence of natural numbers. Then $\{a_{n_k}\}_{k=1}^{\infty}$ is a \Emph{subsequence} of the sequence $(a_n)$.
\end{defn}

\begin{eg}
    Consider the sequence $(-1)^n$. Then we have that \begin{equation*}
        \lim\sup\limits_{n\rightarrow\infty}(-1)^n = 1,\;\;\text{ and }\;\;\lim\inf\limits_{n\rightarrow\infty}(-1)^n = -1
    \end{equation*}
\end{eg}


\begin{thm}
    Suppose $z_n = x_n+iy_n$ for real sequences $x_n$ and $y_n$. Let $z = x+iy \in \C$. Then $z_n\rightarrow z$ if and only if $x_n\rightarrow x$ and $y_n\rightarrow y$.
\end{thm}
\begin{proof}
    $\implies$. Assume that $z_n\rightarrow z$. Hence, for each $\varepsilon > 0$, there exists $N \in \N$ such that if $n \geq N$ then $|z_n - z| < \varepsilon$. Consider $|x_n - x|$. Then observe that $|x_n-x| = \sqrt{(x_n-x)^2}\leq \sqrt{(x_n-x)^2+(y_n-y)^2} = |z_n - z$. Then for any $n \geq N$ we have that $|x_n - x| \leq |z_n - z| < \varepsilon$, and similarly, $|y_n - y| \leq |z_n - z| < \varepsilon$. Hence, $x_n\rightarrow x$ and $y_n\rightarrow y$, as desired.

    $\impliedby$. Now, suppose that $x_n\rightarrow x$ and $y_n\rightarrow y$. Fix $\varepsilon > 0$. Then there exist $N_1,N_2 \in \N$ such that $n_1 \geq N_1$ and $n_2 \geq N_2$ imply $|x_{n_1} - x| < \varepsilon/2$ and $|y_{n_2}-y| < \varepsilon$. Let $N = \max(N_1,N_2)$. Then for all $n \geq N$, \begin{equation*}
        |z_n - z| \leq |x_n - x| + |i(y_n - y)| = |x_n - x| + |y_n - y| < \varepsilon/2+\varepsilon/2 = \varepsilon
    \end{equation*}
    Hence, we conclude that $z_n\rightarrow z$, as desired.
\end{proof}

\begin{defn}
    A sequence $\{z_n\}_{n=1}^{\infty}$ is said to be \Emph{Cauchy} if for every $\varepsilon > 0$, there exists $N \in \N$ such that for all $n,m \geq N$, \begin{equation*}
        |z_n - z_m| < \varepsilon
    \end{equation*}
\end{defn}

Thus, the tail of a Cauchy sequence gets arbitrarily close as we go arbitrarily far.

$\C$ is a complete metric space, so the Cauchy sequences are precisely the convergent sequences.


\begin{defn}
    An open disk of radius $\varepsilon > 0$ centered at $z_0 \in \C$ is defined to be \begin{equation*}
        D_{\varepsilon}(z_0) := \{z \in \C: |z-z_0| < \varepsilon\}
    \end{equation*}
\end{defn}

\begin{defn}
    Let $S \subseteq \C$ and $z_0 \in S$. Then $z_0$ is an \Emph{interior point} if there exists $\varepsilon > 0$ such that $D_{\varepsilon}(z_0) \subseteq S$. 

    We say that $S$ is an \Emph{open set} in $\C$ if each point in $S$ is an interior point.
\end{defn}

\begin{defn}
    We say that $S \subseteq \C$ is a \Emph{closed set} if and only if $\C\backslash S$ is open.
\end{defn}

\begin{eg}
    Consider the half-plane $S = \{z \in \C: \mathscr{I}m(z) \geq 1\}$. This is not a closed set as all points on the boundary ($\mathscr{I}m(z) = 1$) are not interior.
\end{eg}

\begin{defn}
    Let $S \subseteq \C$ and $z_0 \in \C$. Then $z_0$ is said to be a \Emph{limit/accumulation point} of $S$ if for every $\varepsilon > 0$, $D_{\varepsilon}^*(z_0)\cap S \neq \emptyset$. 
\end{defn}
That is, neighborhoods of limit points always intersect the set in a point different from the limit point.

\begin{defn}
    Let $f:S\rightarrow \C$ be a complex function, $S \subseteq \C$. Suppose $z_0 \in \C$ is a limit point of $S$. Then we say that $f(z) \rightarrow L$, for $L \in \C$, if and only if for every $\varepsilon > 0$, there exists $\delta > 0$ such that if $0 < |z-z_0| < \delta$, then $|f(z) - L| < \varepsilon$. 


    Equivalently, the functional limit converges to $L$ if and only if for every $\varepsilon > 0$, there exists $\delta > 0$ such that \begin{equation*}
        f(D_{\delta}^*(z_0)\cap S) \subseteq D_{\varepsilon}(L)
    \end{equation*}
\end{defn}


\begin{thm}
    Let $f:S\rightarrow \C$ and $g:U\rightarrow \C$ be complex functions such that $\lim\limits_{z\rightarrow z_0}f(z) = L$ and $\lim\limits_{z\rightarrow z_0}g(z) = M$ for some $L,M \in \C$. Let $c \in \C$. Then\begin{itemize}
        \item $\lim\limits_{z\rightarrow z_0}(f(z)+g(z)) = L+M$
        \item $\lim\limits_{z\rightarrow z_0}cf(z) = cL$
        \item $\lim\limits_{z\rightarrow z_0}f(z)g(z) = LM$
        \item $\lim\limits_{z\rightarrow z_0}f(z)/g(z) = L/M$, provided $M \neq 0$.
    \end{itemize}
\end{thm}

\begin{defn}
    We say that $f:S\rightarrow \C$ is continuous at $z_0 \in S$ if and only if $\lim\limits_{z\rightarrow z_0}f(z) = f(z_0)$.
\end{defn}

\begin{thm}
    Let $f:S\rightarrow \C$. Then $\lim\limits_{z\rightarrow z_0}f(z) = L$ if and only if for all sequences $z_n$ such that $z_n\rightarrow z_0$, $\lim\limits_{n\rightarrow\infty}f(z_n) = L$.
\end{thm}


\begin{defn}
    A subset $S \subseteq \C$ is connected if and only if there exists no $A,B\subseteq \C$ non-empty such that $S = A\cup B$ and $\overline{A}\cap B = \emptyset$ and $A\cap \overline{B} = \emptyset$.
\end{defn}

Noting that connected and path-connected subsets are equivalent in $\C$, we formulate (path) connectedness in another way as follows:

\begin{defn}
    A \Emph{polygonal chain} $P$ is a curve composed of a finite number of connected line segments. That is, there exist $z_0,z_1,...,z_n \in \C$ such that \begin{equation*}
        P = [z_0,z_1] \cup [z_1,z_2] \cup ... \cup [z_{n-1},z_n]
    \end{equation*}
\end{defn}

\begin{defn}
    A subset $U \subseteq \C$ is (path) \Emph{connected} if and only if for each $p,q \in U$, there exists a polygonal chain $\gamma$ which begins at $p$ and terminates at $q$, and $\gamma \subseteq U$.
\end{defn}
    


\begin{defn}
    A subset $D \subseteq \C$ is called a \Emph{domain} if $D$ is both open and connected.
\end{defn}


\begin{defn}
    A \Emph{region} is a domain paired with some or all of its topological boundary.
\end{defn}

\begin{defn}
    Let $U \subseteq \C$. Then $U^o$ denotes the \Emph{interior} of $U$, and is the set of all interior points in $U$.
\end{defn}

\begin{defn}
    A set $U \subseteq \C$ is \Emph{compact} if and only if (Heine-Borel) $U$ is closed and bounded.
\end{defn}


\begin{defn}
    A set $U \subseteq \C$ is said to be \Emph{star-shaped} at a point $z_0 \in \C$ if $z_0 \in U$, and for any $z \in U$ we have $[z_0,z] \subseteq U$.
\end{defn}

Star-shaped implies simply connected: every closed curve in the region can be continuously deformed to a point.

\begin{eg}
    The slit complex plane $\C^- = \C\backslash(-\infty,0]$ is star shaped, with any point along $(0,\infty)$ serving as a star center. In particular, consider the star center $1$. Then, let $z \in \C^-$ and consider $[1,z] = \{1+t(z-1):0\leq t\leq 1\}$. Then, for each $a+ib \in [1,z]$ we have $a+ib = 1+t(x+iy-1) = 1+(x-1)t+iyt$ for some $t \in [0,1]$. For $z \in (0,\infty)$ we have that $y = 0$, and $x-1 > -1$, so $(x-1)t > -1t > -1$, so $a+ib = a > 0$. On the other hand, if $z \in \C^-$ with $y \neq 0$, then $a+ib \notin (-\infty,0]$ for all $t > 0$, and by construction $a+ib = 1$ for $t = 0$, which is again not in $(-\infty,0]$. Thus, $[1,z] \cap (-\infty,0] = \emptyset$ for all $z \in \C^-$, as claimed.
\end{eg}


\begin{thm}\label{thm:constant}
    If $h:D\subseteq \R^2\rightarrow\R^2$ is continuously differentiable on $D$ and $\nabla h= \vec{0}$ on $D$, then $h$ is constant on $D$.
\end{thm}
\begin{proof}
    Let $\gamma:[t_0,t_1]\rightarrow D$ be a polygonal chain in $D$, with $\gamma(t_0) = p$ and $\gamma(t_1) = q$. Then \begin{equation*}
        \frac{d}{dt}(h(\gamma(t)) = \nabla h(\gamma(t)) \cdot \frac{d\gamma}{dt} = 0
    \end{equation*}
    since the gradient is zero in $D$. Then by properties of single variable differentiable functions, $h(\gamma(t))$ is constant so $h(p) = h(\gamma(t_0)) = h(\gamma(t_1)) = h(q)$. But this is true for all $p,q \in D$, so $h$ is constant.
\end{proof}



%%%%%%%%%%%%%%%%%%%% Section 1.2.2
\section{Analytic Functions}

\begin{defn}
    Let $f:S\rightarrow \C$. If the limit \begin{equation*}
        \lim\limits_{z\rightarrow z_0}\left(\frac{f(z) - f(z_0)}{z-z_0}\right)
    \end{equation*}
    exists, we say $f$ is \Emph{complex differentiable} at $z_0$, and we denote \begin{equation*}
        f'(z_0) = \lim\limits_{z\rightarrow z_0}\left(\frac{f(z) - f(z_0)}{z-z_0}\right)
    \end{equation*}
    Futhermore, the mapping $z\mapsto f'(z)$ is the complex derivative of $f$.
\end{defn}

\begin{eg}
    Consider $f(z) = 2z^2 - 1$. Then \begin{equation*}
        \lim\limits_{z\rightarrow z_0}\left(\frac{f(z) - f(z_0)}{z-z_0}\right) = \lim\limits_{z\rightarrow z_0}\left(\frac{2z^2 - 1 - (2z_0^2 - 1)}{z-z_0}\right) = 2\lim\limits_{z\rightarrow z_0}\left(\frac{(z-z_0)(z+z_0)}{z-z_0}\right) = 2\cdot 2z_0 = 4z_0
    \end{equation*}
\end{eg}

\begin{namthm}[Caratheodory Theorem]\label{namthm:car}
    Let $f:D\subseteq \C\rightarrow \C$ be a function $a \in D$, a limit point. Then $f$ is \Emph{complex differentiable} at $a$ if and only if there exists a function $\phi:D\rightarrow \C$ such that \begin{enumerate}
        \item $\phi$ is continuous at $a$
        \item $f(z)-f(a) = \phi(z)(z-a)$ for all $z \in D$.
    \end{enumerate}
\end{namthm}
\begin{proof}
    $\implies.$ Suppose $f:D\subseteq \C\rightarrow \C$ is complex differentiable at $a$. Construct $\phi(z) = \frac{f(z) - f(a)}{z-a}$ for $z \neq 0$ anf $\phi(a) = f'(z)$. Then, observe that \begin{equation*}
        \lim\limits_{z\rightarrow a}\phi(z) = \lim\limits_{z\rightarrow a}\left(\frac{f(z)-f(a)}{z-a}\right) = f'(a)
    \end{equation*}
    so $\phi$ is continuous at $a$. Next, for $z \neq a$ we have by definition that $f(z) - f(a) = \phi(z)(z-a)$. At $z = 0$ we just have $0 = 0$, so this also holds. 
    
    $\impliedby.$ By 2. we have that $\phi(z) = \frac{f(z) - f(a)}{z-a}$ for $z \neq a$, so by continuity $\phi(a) = f'(a)$, so $f'$ is complex differentiable at $a$.
\end{proof}

\begin{thm}
    If $f:S\rightarrow \C$ is complex differentiable at $a \in \C$, then that implies that $f$ is continuous at $z = a$. 
\end{thm}
\begin{proof}
    By Theorem \ref{namthm:car}, there exists $\phi:S\rightarrow \C$, where $a \in S$, and $f(z) = f(a)+\phi(z)(z-a)$. Then, \begin{equation*}
        \lim\limits_{z\rightarrow a}f(z) = \lim\limits_{z\rightarrow a}(f(a)+\phi(z)(z-a)) = f(a) + \phi(a)(a-a) = f(a)
    \end{equation*}
    which implies that $f$ is continuous at $a$, as desired.
\end{proof}

Assume that $f,g$ are complex differentiable at $z = a$. Then there exist $\phi_f,\phi_g$ such that $f(z) = f(a) + (z-a)\phi_f(z)$ and $g(z) = g(a)+(z-a)\phi_g(z)$ for some domain of $a$. Then, observe that \begin{equation*}
    f(z)g(z) = f(a)g(a) + (z-a)[\phi_f(z)g(a)+\phi_g(z)f(a) + (z-a)\phi_f(z)\phi_g(z)]
\end{equation*}
We claim $\phi_{fg}(z) = \phi_f(z)g(a)+\phi_g(z)f(a) + (z-a)\phi_f(z)\phi_g(z)$. By continuity of the component functions $\phi_{fg}(z)$ is continuous at $a$. Hence, $f(z)g(z)$ is differentiable and \begin{equation*}
    (fg)'(a) = f'(a)g(a)+f(a)g'(a)
\end{equation*}


\begin{thm}
    Suppose that $g$ is complex differentiable at $a$, and that $f$ is complex differentiable at $g(a)$.
\end{thm}
\begin{proof}
    Suppose $g'(a)$ and $f'(g(a))$ exist. Also, let $g(z) = g(a)+(z-a)\phi_g(z)$, and $f(w) = f(g(a)) + (w-g(a))\psi_f(w)$. Simply compose, set $w = g(z)$. Then \begin{align*}
        f(g(z)) &= f(g(a)) + (g(z) - g(a))\psi_f(g(z)) \\
        &= f(g(a)) + (z-a)\phi_g(z)\psi_f(g(z))
    \end{align*}
    Then by continuity of $\phi_g$ and $\psi_f$ we have that $(f\circ g)'(a) = f'(g(a))g'(a)$, completing the proof. Moreover, $\phi_g(z) \psi_f(g(z)) = \phi_{f\circ g}(z)$.
\end{proof}

\begin{thm}
    Suppose that $f$ and $g$ are complex differentiable at $a$, and $c \in \C$. Then \begin{itemize}
        \item $(f+g)'(a) = f'(a)+g'(a)$
        \item $(cf)'(a) = cf'(a)$
        \item $(fg)'(a) = f'(a)g(a) + f(a)g'(a)$
        \item $(f/g)'(a) = (f'(a)g(a) - f(a)g'(a))/(g(a))^2$ provided $g(a) \neq 0$.
    \end{itemize}
\end{thm}


%%%%%%%%%%%%%%%%%%%% Section 1.2.3
\section{The Cauchy Riemann Equations}

\begin{thm}
    If $f(z) = u(z) + iv(z)$ is a complex valued function of a complex variable, $u$ and $v$ real valued functions of a complex variable, and \begin{enumerate}
        \item $u$ and $v$ are continuously differentiable on an open set containing $z_0$
        \item $u_x(z_0) = v_y(z_0)$ and $u_y(z_0) = -v_x(z_0)$
    \end{enumerate}
    Then $f'(z_0) = u_x(z_0)+iv_x(z_0)$ is complex differentiable.
\end{thm}

\begin{defn}
    $g:\R^2\rightarrow \R$ is continuously differentiable on $U \subseteq \R^2$ if $g_x$ and $g_y$ exist and are continuous on $U$.
\end{defn}

\begin{eg}
    Suppose $f(z) = z^2$. Then $f(x+iy) = x^2-y^2 + i2xy$, and let $u(z) = x^2-y^2$ and $v(z) = 2xy$. Then $u_x(z) = 2x, u_y(z) = -2y, v_x = 2y, v_y = 2x$, so the partials are continuously differentiable. Moreover, $u_x = v_y$ and $u_y = -v_x$, so the Cauchy Riemann equations hold and $f'(z) = 2x+i2y = 2z$.
\end{eg}

\begin{eg}
    Suppose $f(z) = e^z$. Then $f(x+iy) = e^{x+iy} = e^xe^{iy} = e^x\cos y + ie^x\sin y$. Let $u(z) = e^x\cos y$ and $v(z) = e^x\sin y$. Then $u_x(z) = e^x\cos y, u_y(z) = -e^x\sin y, v_x(z) = e^x\sin y, v_y(z) = e^x\cos y$. Thus, the partial derivatives are continuous on $\C$, and $u_x = v_y$ and $u_y = -v_y$, so $f'(z) = u_x(z)+iv_x(z) = u(z)+iv(z) = f(z)$.
\end{eg}

\begin{eg}
    Consider $f(z) = \overline{z} = x-iy$. Then $u = x$ and $v = -y$. Now, observe $u_x = 1 \neq -1 = v_y$, so $f(z)$ is nowhere complex differentiable.
\end{eg}


\begin{eg}
    Consider $f(z) = \cos(z) = \cos(x+iy) = \cos(x)\cos(iy)-\sin(x)\sin(iy) = \cos(x)\cosh(y) - i\sin(x)\sinh(y)$. Then $u = \cos(x)\cosh(y)$ and $v = -\sin(x)\sinh(y)$. Now, $u_x = -\sin(x)\cosh(y), u_y = \cos(x)\sinh(y), v_x = -\cos(x)\sinh(y), v_y = -\sin(x)\cosh(y)$. Thus, the partials are continuous and $u_x = v_y$ and $u_y = -v_x$, so the Cauchy-Riemann equations hold and $f'(z) = -\sin(x)\cosh(y)-i\cos(x)\sinh(y) = -\sin(x)\cos(iy)-\cos(x)\sin(iy) = -\sin(x+iy)$.
\end{eg}

\begin{defn}
    A function $f:D\rightarrow \C$ is \Emph{holomorphic} on the domain $D$ if and only if $f$ is complex differentiable at each point in $D$.
\end{defn}


\begin{defn}
    A function $f:D\rightarrow \C$ is \Emph{holomorphic} at a point $z_0 \in D$ if there exists some open disk $D_r(z_0)$ such that $f\vert_{D_r(z_0)}$ is holomorphic.
\end{defn}

\subsection{Real Differentiability}

If we identify $\C=\R^2$, then $f:\C\rightarrow \C$ naturally is associated with $f:\R^2\rightarrow \R^2$.

\begin{defn}
    If $F:U\subseteq \R^2\rightarrow\R^2$ has a linear transformation $L:\R^2\rightarrow \R^2$ such that \begin{equation*}
        \lim\limits_{h\rightarrow \vec{0}}\left(\frac{F(p+h)-F(p)-L(h)}{||h||}\right) = 0
    \end{equation*}
    then $F$ is \Emph{real-differentiable} at $p \in \R^2$, and we denote $dF_p = L$. Moreover, the standard matrix for the differential, where $F= F(u,v)$, is \begin{equation*}
        [dF_p] = J_F(p) = \begin{bmatrix} \frac{\partial u}{\partial x} & \frac{\partial u}{\partial y} \\ \frac{\partial v}{\partial x} & \frac{\partial v}{\partial y}\end{bmatrix} = \left[\begin{array}{c|c} \frac{\partial F}{\partial x} & \frac{\partial F}{\partial y}\end{array}\right]
    \end{equation*}
\end{defn}

\begin{eg}
    Let $F(x,y) = (x,-y)$, then  $J_F(p) = \begin{bmatrix} 1 & 0 \\ & -1\end{bmatrix}$
\end{eg}

\begin{defn}
    $F(u,v):\R^2\rightarrow \R^2$ is continuously differentiable at $z_0 = (x_0,y_0)$ if $u_x,u_y,v_x,v_y$ are all continuous near $z_0$.
\end{defn}

\begin{defn}
    If $F$ is continuously differentiable at $z_0 \in \R^2$, then $F$ is differentiable at $z_0$. 
\end{defn}
\begin{proof}
    Assume $F= (u,v)$ is continuously differentiable with $dF = (du,dv)$ and we can focus on $u$. We wish to show that \begin{equation*}
        \lim\limits_{h\rightarrow \vec{0}}\left(\frac{u(p+h)-u(p)-L(h)}{||h||}\right) = 0
    \end{equation*}
    Let $L(h) = \nabla u(p)\cdot h = h_1u_x(p)+h_2u_y(p)$. Observe that \begin{align*}
        u(p+h)-u(p) &= u(p+h_1e_1+h_2e_2)-u(p) \\
        *= u(p+h_1e_1+h_2e_2) - u(p+h_1e_1)+u(p+h_1e_1)-u(p) \\
        &= u(p_1+h_1,p_2+h_2) - u(p_1+h_1,p_2) + u(p_1+h_1,p_2) - u(p_1,p_2) \\
        &= h_2u_y(p_1+h_1,c_2) + h_1u_x(c_1,p_2)
    \end{align*}
    where the last equality is by the Mean Value Theorem. Then it follows that \begin{equation*}
        \lim\limits_{h\rightarrow \vec{0}}\left(\frac{u(p+h)-u(p)-L(h)}{||h||}\right) = \lim\limits_{h\rightarrow \vec{0}}\left(\frac{h_2u_y(p_1+h_1,c_2) + h_1u_x(c_1,p_2)-h_1u_x(p) - h_2u_y(p)}{||h||}\right) 
    \end{equation*}
    where as $h\rightarrow \vec{0}$, $c_1$ and $c_2$ go to $p_1$ and $p_2$, respective, so by continuity of the partial derivatives this limit goes to zero. The same holds for $v$, so $F$ is differentiable.
\end{proof}


In our convention we write $F(u,v) = u+iv$, so we have $e_1 = 1$ and $e_2 = i$. Now, note that $F(x,y) = (x,-y) = x-iy = \overline{z}$, so $F$ would be nowhere complex differentiable.


        

\subsection{Cauchy Riemann Equations Sketch}

Suppose $f$ is complex differentiable at $z_0$, so by Theorem \ref{namthm:car} we have $f(z_0+h) - f(z_0) = \phi(z_0+h)h$. Then, I claim for $f:\R^2\rightarrow \R^2$, $df_{z_0}(h) = f'(z_0)h$, shifting between $\C$ and $\R^2$. For $h \neq 0$ we have \begin{align*}
    \frac{f(z_0+h)-f(z_0) -f'(z_0)h}{|h|} &= \frac{\phi(z_0+h)h-f'(z_0)h}{|h|} \\
    &= \frac{h}{|h|}(\phi(z_0+h)-f'(z_0)) \\
    &\leq \frac{|h|}{|h|}(\phi(z_0+h)-f'(z_0)) = 0
\end{align*}
so $f$ is real differentiable at $z_0$ with $df_{z_0}(h) = f'(z_0)h$. Let $f'(z_0) = a+ib$. Then $$f'(z_0)h = ah_1-bh_2 + i(ah_2+bh_1) = \begin{bmatrix} a & -b \\ b & a \end{bmatrix}\begin{bmatrix} h_1 & h_2 \end{bmatrix}$$ So in particular $$df_{z_0} = \begin{bmatrix} u_x & u_y \\ v_x & v_y \end{bmatrix} = \begin{bmatrix} a & -b \\ b & a \end{bmatrix}$$ so we attain the Cauchy Riemann equations $u_x = v_y$ and $u_y = -v_x$. Reversing this argument we attain that if a function is continuously real-differentiable and satisfies the Cauchy Riemann equations, then the function is complex differentiable.



\begin{thm}
    If $f = (u,v) = u+iv$ is real differentiable on a domain $D$ and $u_x = v_y, u_y = -v_x$ at $z_0 \in D$, then $f'(z_0) = u_x(z_0) + iv_x(z_0)$.
\end{thm}
\begin{proof}
    Let $u_x(z_0) = a$ and $v_x(z_0) = b$. Then $$df_{z_0} = \begin{bmatrix} u_x(z_0) & u_y(z_0) \\ v_x(z_0) & v_y(z_0) \end{bmatrix} = \begin{bmatrix} a & -b \\ b & a \end{bmatrix}$$
    We have \begin{align*}
        df_{z_0}h &= \begin{bmatrix} a & -b \\ b & a \end{bmatrix}\begin{bmatrix} h_1 \\ h_2\end{bmatrix} \\
            &= \begin{bmatrix} ah_1-bh_2 \\ bh_1+ah_2 \end{bmatrix} \\
                &= (ah_1-bh_2) +i(bh_1+ah_2) = (a+ib)(h_1+ih_2) = (a+ib)h
    \end{align*}
    We claim that $f'(z_0) = a+ib$. Observe \begin{equation*}
        \lim\limits_{h\rightarrow 0} \left(\frac{f(z_0+h) - f(z_0) - (a+ib)h}{h}\right) = 0
    \end{equation*}
    Moreover, since $\lim\limits_{h\rightarrow 0}\frac{(a+ib)h}{h} = a+ib$, so by algebraic properties of limits we have that \begin{equation*}
        \lim\limits_{h\rightarrow 0} \left(\frac{f(z_0+h) - f(z_0)}{h}\right) = a+ib = u_x(z_0)+iv_x(z_0)
    \end{equation*}
\end{proof}

Then, we can write \begin{equation*}
    \boxed{\frac{df}{dz} = f'(z) = \frac{\partial f}{\partial x} = -i\frac{\partial f}{\partial y}}
\end{equation*}
for a holomorphic function on some domain.



%%%%%%%%%%%%%%%%%%%% Section 1.2.4
\section{Analytic Functions (cont.)}

\begin{defn}
    If $f:\C\rightarrow \C$ and $f$ is holomorphic on $\C$, then $f$ is \Emph{entire}. We say $f \in \mathcal{O}(\C)$. If $f$ is holomorphic on $D$, we write $f \in \mathcal{O}(D)$.
\end{defn}


\begin{thm}
    If $f \in \mathcal{O}(D)$ and $f = \mathscr{R}e(f)$, then $f$ is constant.
\end{thm}
\begin{proof}
    First, this implies $f(z) = u(z)$. Then $\nabla u = \langle u_x,u_y\rangle$ and $\nabla v = \langle v_x, v_y\rangle$. But $v = 0$, so $\nabla v = \vec{0}$, and $u_x = v_y, u_y = -v_x$ by the Cauchy Riemann equations so $\nabla u = \vec{0}$. Thus, we have that $u$ and $v$ are constant, so $f$ is constant by Theorem \ref{thm:constant}.
\end{proof}

If $f = (u,v)$ is continuously differentiable at $(a,b)$, then for $(h_1,h_2)$ near $(a,b)$ we have that \begin{equation*}
    f(a+h_1,b+h_2) \approx f(a,b) + J_f(a,b)\begin{bmatrix} h_1 \\ h_2 \end{bmatrix}
\end{equation*}
Assume the cauchy riemman equations hold. Then \begin{equation*}
    J_f(a,b) = \begin{bmatrix} a & -b \\ b & a \end{bmatrix} = \sqrt{a^2+b^2}\begin{bmatrix} a/\sqrt{a^2+b^2} & -b/\sqrt{a^2+b^2} \\ b/\sqrt{a^2+b^2} & a/\sqrt{a^2+b^2} \end{bmatrix}
\end{equation*}
where the matrix on the right, $R$, is orthogonal ($RR^T = I$), and the term in front is a dilation. Hence, at an infinitesimal level changes in holomorphic functions correspond to rotations and dilations. Also, observe that $\det(J_f(z_0)) = a^2+b^2 = |f'(z_0)|^2$.

\begin{thm}
    $f'(z_0) \neq 0$ if and only if $\det(J_f(z_0)) \neq 0$, for $f$ complex differentiable at $z_0$. 
\end{thm}

\begin{thm}[Complex Inverse Function Theorem]\label{thm:invfunc}
    Of $f(z)$ is analytic on a domain $D$, $z_0 \in D$, and $f'(z_0) \neq 0$. Then there exists $\varepsilon > 0$ such that $D_{\varepsilon}(z_0) \subseteq D$ such that $f\vert_{D_{\varepsilon}(z_0)}$ is injective, the image $V = f(D_{\varepsilon}(z_0))$ is open and the inverse function $f^{-1}:V\rightarrow U$ is analytic and satisfies \begin{equation*}
        (f^{-1})'(f(z)) = 1/f'(z)\;\;\;\text{ for } z \in D_{\varepsilon}(z_0)
    \end{equation*}
\end{thm}

\begin{eg}
    Let $\text{Log}(z) = w$, so $e^w = z$. Then $e^w\frac{dw}{dz} = 1$, so $\frac{d}{dz}(\text{Log}(z)) = 1/z$ in $\C^-$. In general, this can be extended to arbitrary branches and slit complex plains such that the function is single valued and continuous.
\end{eg}

\begin{eg}
    Consider the power function on a particular branch: \begin{align*}
        \frac{d}{dz}(z^c) &= \frac{d}{dz}(e^{c\text{Log}(z)}) = e^{c\text{Log}(z)}\frac{d}{dz}({c\text{Log}(z)}) \\
        &= e^{c\text{Log}(z)}\frac{c}{z} = ce^{c\text{Log}(z)}e^{-\text{Log}(z)} \\
        &= ce^{(c-1)\text{Log}(z)} = cz^{c-1}
    \end{align*}
    on the domain of $\C$ for which the particular branch of the log is holomorphic.
\end{eg}




%%%%%%%%%%%%%%%%%%%% Section 1.2.5
\section{Harmonic Functions}

\begin{defn}
    If $u:V\subseteq \R^2\rightarrow \R$ and $\nabla\cdot \nabla u = 0$, then $u$ is said to be \Emph{harmonic} on $V$. \begin{itemize}
        \item For $n = 2$: $u_{xx} + u_{yy} = 0$.
    \end{itemize}
\end{defn}

If $f'(z) = u_x + iv_x = v_y - iu_y$ over some domain $D$, it's then true that $z \mapsto f'(z)$ is continuous. In contrast, for $f(x,y) = (x|x|,0)$, $J_f = \begin{bmatrix} 2|x|  & 0 \\ 0 & 0 \end{bmatrix}$, but $\frac{\partial^2 f}{\partial x^2}$ does not exist at $(0,0)$, and hence is not continuous there.

\begin{thm}
    If $f= u+iv$ is \Emph{holomorphic} on a domain $D$, then $u$ and $v$ are \Emph{harmonic}.
\end{thm}
\begin{proof}
    Asumme $f'(z)$ exists for $z \in D$, then $z \mapsto f'(z)$ is continuous. Then (as we will prove later) $u_{xx},v_{xx},u_{xy},v_{xy}$ exist and are continuous. By the Cauchy Riemann equations we have that $\partial_xu = \partial_yv$ and $\partial_yu = -\partial_xv$. Differentiating with respect to $x$ and with respect to $y$ gives $\partial_x^2u = \partial_x\partial_yv$ and $\partial_y^2u = -\partial_y\partial_xv$. Then since the second partials are continuous we have by Clairout's Theorem that \begin{equation*}
        \partial_x^2u+\partial_y^2u = \partial_x\partial_yv - \partial_y\partial_xv = 0
    \end{equation*}
    Hence $u$ is harmonic. A similar argument shows $v$ is harmonic as well. Alternatively, to argue that $v$ is harmonic, we note that if $f=u+iv$ is holomorphic, then $-if = -iu+v$ is also holomorphic, so as $v$ is the real part of this function so by the previous argument $v$ is harmonic.
\end{proof}


\begin{defn}
    If $u$ is harmonic and $f = u+iv$ is holomorphic, then $v$ is a \Emph{harmonic conjugate}.
\end{defn}



\begin{xca}
    Given a function $u(x,y) = $ something, is $u$ harmonic and if so, find its \Emph{harmonic conjugate}.
\end{xca}


\begin{eg}
    Consider $u(x,y) = e^{2x}\cos(2y)$. Then $u_{xx} = 4u$, and $u_{yy} = -4u$, so $u_{xx}+u_{yy} = 0$ and $u$ is harmonic on its domain, $\C$. We want to solve $u_x = v_y$ and $u_y = -v_x$. Observe that $v_y = 2e^{2x}\cos(2y)$, so $v = e^{2x}\sin(2y)+C(x)$ for some function of $x$, $C(x)$. Similarly, $v_x = 2e^{2x}\sin(2y)$, so $v = e^{2x}\sin(2y) + C(y)$ for some function of $y$. Thus, $v = e^{2x}\sin(2y)$ is \emph{a} complex conjugate of $u = e^{2x}\cos(2y)$. (In general $v = e^{2x}\sin(2y) + C$ is a family of complex conjugates of $u$). Then, observe $u+iv = e^{2x}e^{i2y} = e^{2z}$.
\end{eg}

If we look at $u(x,y) = k_1$ (level curve 1) and $v(x,y) = k_2$ (level curve 2). Assume $f = u+iv$ has $f \in \mathcal{O}(D)$. Let's consider $\nabla u \cdot \nabla v = u_xv_x+u_yv_y = -u_xu_y+u_yu_x = 0$, so the gradients of the level curves are orthogonal where they meet. 


%%%%%%%%%%%%%%%%%%%% Section 1.2.6
\section{Conformal Functions}




\begin{defn}
    A smooth complex-valued function $g(z)$ is \Emph{conformal at $z_0$} if whenever $\gamma_0, \gamma_1$ are curves terminating at $z_0$ with nonzero tangents, then the curves $g \circ \gamma_0$ and $g \circ \gamma_1$ haave nonzero tangents at $g(z_0)$ and the angle between $g\circ \gamma_0$ and $g\circ \gamma_1$ at $g(z_0)$ is the same as the angle between $\gamma_0$ and $\gamma_1$ at $z_0$. 
\end{defn}

\begin{thm}
    If $f(z)$ is holomorphic at $z_0$ and $f'(z_0) \neq 0$ then $f(z)$ is conformal at $z = z_0$.
\end{thm}


Consequently, if $f(z)$ is holomorphic in a domain $D$, then $f$ takes orthogonal hashes to orthogonal hashes in $D$.

\begin{prop}
    If $f:D\subseteq \C\rightarrow \C$ and $f'(z_0)$ exists for $z_0 \in D$. Let $\gamma = \langle x,y\rangle:I\subseteq \R\rightarrow D$ is a smooth curve with $\gamma(t_0) = z_0$. Then \begin{equation*}
        \frac{d}{dt}\left[f(\gamma(t))\right] = f'(\gamma(t))\frac{d\gamma}{dt}
    \end{equation*}
    where we have complex multiplication on the right.
\end{prop}
\begin{proof}
    First, observe \begin{align*}
        \frac{d}{dt}f(\gamma(t)) &= \frac{d}{dt}\left[u(\gamma(t)) + iv(\gamma(t))\right] \\
        &= \frac{d}{dt}u(\gamma(t)) + i\frac{d}{dt}v(\gamma(t)) \tag{by linearity} \\
        &= \nabla u(\gamma(t))\cdot \frac{d}{dt}\gamma(t)+i\nabla v(\gamma(t))\cdot \frac{d}{dt}\gamma(t) \\
        &= u_xx'(t)+u_yy'(t)+i(v_xx'(t)+v_yy'(t)) \tag{for $\gamma(t) = \langle x(t),y(t)\rangle$}  \\
        &= u_xx'(t)+u_yy'(t)+i(-u_yx'(t)+u_xy'(t)) \\
        &= u_x(x'(t)+iy'(t))-iu_y(x'(t) + iy'(t)) \\
        &= u_x\gamma'(t)+iv_x\gamma'(t) \\
        &= f'(\gamma(t))\frac{d\gamma(t)}{dt}
    \end{align*}
\end{proof}
Note that this is not guaranteed for non-holomorphic functions real differentiable functions. 

By the inverse function theorem we also observe that the inverse map exists, is holomorphic, and hence also conformal. This shows an alternative result that when level curves intersect, they intersect at right angles.




%%%%%%%%%%%%%%%%%%%% Section 1.2.7
\section{Fractional Linear Transformations (FLT's)}

\begin{defn}
    A function $f:\C\rightarrow \C$ is an elementary transformation if it is of one of the following forms: \begin{itemize}
        \item $f_1(z) = z+z_0$, for some fixed $z_0 \in \C$: translation
        \item $f_2(z) = cz$ for some fixed $c \in \C$: dilation ($c\neq 0$)
        \item $f_3(z) = 1/z$, for $z \neq 0$: inversion
    \end{itemize}
\end{defn}

For a circle, $c(t) = z_0+Re^{it}$ we have $f_1(c(t)) = z_0+z_0'+Re^{it}$ (shifts the center) and $f_2(c(t)) = cz_0+cRe^{it}$ (dilates and shifts the circle). Moreover, the composition of $f_1$ and $f_2$ is something called an \Emph{affine transformation} (i.e. $f_1\circ f_2(z) = cz+d$ for $c,d \in \C$).

\begin{defn}
    A \Emph{fractional linear transformation} is a function of the form \begin{equation*}
        f(z) = \frac{az+b}{cz+d}
    \end{equation*}
    where $a,b,c,d$ are complex constants satisfying $ad-bc \neq 0$. These are also called \Emph{M$\ddot{o}$bius Transformations}. Since \begin{equation*}
        f'(z) = \frac{ad-bc}{(cz+d)^2}
    \end{equation*}
    the condition $ad-bc \neq 0$ simply guarantees that $f$ is not constant.
\end{defn}


\subsection{Extended Complex Plane}

The extended complex plane corresponds to the one-point compactification of the plane. It can be viewed through the Riemann sphere and the stereographic projection, where we draw a line out of the north pole, and where the line intersects the plane and the sphere are identified. Then, the north pole is identified with $\infty$, giving $\C^* = \C\cup\{\infty\}$.

\begin{thm}
    Given any three distinct points $z_0,z_1,z_2$ in the extended complex plane, and given any three distinct values $w_0,w_1,w_2$ in the extended complex plane, there is a unique fractional linear transformation $w = w(z)$ such that $w(z_0) = w_0,w(z_1) = w_1,$ and $w(z_2) = w_2$.
\end{thm}

\begin{defn}
    The cross ratio for the fractional linear transformation specified in the above theorem is \begin{equation*}
        \frac{(w_1 - w)(w_3 - w_2)}{(w_1 - w_2)(w_3 - w)} = \frac{(z_1-z)(z_3-z_2)}{(z_1-z_2)(z_3-z)}
    \end{equation*}
\end{defn}


\begin{defn}
    Let $D$ be a domain, we say $f \in \mathcal{O}(D)$ is \Emph{biholomorphic} mapping of $D$ onto $D'$ if $f(D)  = D'$ and $f^{-1}:D'\rightarrow D$ is holomorphic.
\end{defn}

\begin{defn}
    An $\varepsilon > 0$ neighborhood of infinity is defined to be \begin{equation*}
        D_{\varepsilon}(\infty) = \{z \in \C:\frac{1}{|z|} < \varepsilon\}
    \end{equation*}
\end{defn}
A neighborhood of infinity is an exterior annulus.

\begin{defn}
    For $f:D\rightarrow \C$, we say that $\lim\limits_{z\rightarrow \infty}f(z) = L$ if and only if for evey $\varepsilon > 0$, there exists $\delta > 0$ such that if $\frac{1}{\delta} < |z|$ implies $|f(z) - L| < \varepsilon$.
\end{defn}

We claim that this is equivalent to writing $\lim\limits_{z\rightarrow 0}f\left(\frac{1}{z}\right) = L$. 

\begin{defn}
    For $f:D\rightarrow \C$, we write $\lim\limits_{z\rightarrow z_0}f(z) = \infty$ if and only if for every $\varepsilon > 0$, there exists $\delta > 0$ such that if $0 < |z-z_0| < \delta$, then $|f(z)| > \frac{1}{\varepsilon}$.
\end{defn}

\begin{eg}
    \leavevmode
    \begin{itemize}
        \item $\lim\limits_{z\rightarrow \infty}(1/z) = 0$
        \item $\lim\limits_{z\rightarrow \infty}(z) = \infty$
        \item $\lim\limits_{z\rightarrow 0}(1/z) = \infty$
    \end{itemize}
\end{eg}

\begin{eg}
    Observe that \begin{equation*}
        \lim\limits_{z\rightarrow \infty}\frac{z^2+z}{3z^2+2} = \frac{1}{3}
    \end{equation*}
\end{eg}

\begin{eg}
    $w = \sin(z) = f(z)$. First, write $w=\sin(z) = \frac{1}{2i}(e^{iz}-e^{-iz})$ as $2iw = e^{iz}-e^{-iz}$. Then, we can write $e^{2iz}-1 = 2iwe^{iz}$, so letting $\eta = e^{iz}$, $\eta^2 - 2iw\eta -1 = 0$. Completing the square we have that $1 -w^2= (\eta-iw)^2$, so $\eta-iw = \pm \sqrt{1-w^2}$, and $e^{iz}=\eta = iw\pm\sqrt{1-w^2}$. Then, we have that \begin{equation*}
        iz \in \log(iw\pm\sqrt{1-w^2})
    \end{equation*}
    If we take the branch $z = -i\text{Log}(iw\pm \sqrt{1-w^2})$
\end{eg}





%%%%%%%%%%%%%%%%%%%%%% Chapter 1.3
\chapter{Line Integrals and Harmonic Functions}


%%%%%%%%%%%%%%%%%%%% Section 1.3.1
\section{Line Integrals}


\begin{defn}
    For a complex function $P:\C\rightarrow \C$ and a path $\gamma:[a,b]\rightarrow \C$ in $\C$, the line integral of $Pdx$ along $\gamma$ is defined by \begin{equation*}
        \int_{\gamma}Pdx = \int_{a}^{b}P(\gamma(t))\frac{dx}{dt}dt
    \end{equation*}
    and the line integral of $Pdy$ along $\gamma$ is defined by \begin{equation*}
        \int_{\gamma}Pdy = \int_{a}^{b}P(\gamma(t))\frac{dy}{dt}dt
    \end{equation*}
\end{defn}


\begin{prop}
    Suppose $P:\C\rightarrow \C$ and $Q:\C\rightarrow \C$ are complex valued functions of $\C$. Then for a path $\gamma$ defined on its way, \begin{equation*}
        \int_{\gamma}(cP+Q)dx = c\int_{\gamma}Pdx+\int_{\gamma}Qdx
    \end{equation*}
    for all $c \in \C$ (and similarly for $dy$).
\end{prop}

\begin{defn}
    For $P:\C\rightarrow \C$ and $Q:\C\rightarrow \C$ complex valued functions of $\C$, if $\gamma$ is a path in $\C$ defined in their shared domain, we define \begin{equation*}
        \int_{\gamma}(Pdx+Qdy) = \int_{\gamma}Pdx+\int_{\gamma}Qdy
    \end{equation*}
\end{defn}

Writing $P = P_1 + iP_2$ and $Q = Q_1+iQ_2$, we can write \begin{align*}
    \int_{\gamma}(Pdx+Qdy) &= \int_{\gamma}(P_1+iP_2)dx + \int_{\gamma}(Q_1+iQ_2)dy \\
    &= \int_{\gamma}(P_1dx+Q_1dy)+i\int_{\gamma}(P_2dx+Q_2dy)
\end{align*}

For $\gamma = \gamma_1\cup...\cup\gamma_n$, a piecewise smooth path, we have that \begin{equation*}
    \int_{\gamma}(Pdx+Qdy) = \sum_{i=1}^n\int_{\gamma_i}(Pdx+Qdy)
\end{equation*}


For double integrals, we also have \begin{equation*}
    \int\int_S(F+iG)dA = \int\int_SFdA + i\int\int_SGdA
\end{equation*}
for $F$ and $G$ real valued functions, and \begin{equation*}
    \frac{\partial}{\partial x}(F+iG) = \frac{\partial F}{\partial x}+i\frac{\partial G}{\partial x}
\end{equation*}

\begin{thm}[Green's Theorem]
    Let $D$ be a bounded domain in the plane whose boundary $\partial D$ consists of a finite number of disjoint piecewise smooth closed curves. Let $P$ and $Q$ be continuously differentiable functions on $D\cup \partial D$. Then \begin{equation*}
        \int_{\partial D}(Pdx+Qdy) = \int\int_D\left(\frac{\partial Q}{\partial x} - \frac{\partial P}{\partial y}\right)dxdy
    \end{equation*}
\end{thm}

\begin{defn}
    If $h$ is a complex valued function with continuous $h_x$ and $h_y$, then $dh = \frac{\partial h}{\partial x}dx + \frac{\partial h}{\partial y}dy$ is a \Emph{differential form}. Then we say a differential form $Pdx+Qdy = w$ is \Emph{exact} on $U \subseteq \C$ if there exists $h$ such that $dh = w$ on $U$.
\end{defn}

\begin{namthm}[Fundamental Theorem of Calculus Part I (Complex]
    If $\gamma$ is a piecewise smooth curve from $A$ to $B$, and if $h(x,y)$ is continuously differentiable on $\gamma$, then \begin{equation*}
        \int_{\gamma}dh = h(B) - h(A)
    \end{equation*}
\end{namthm}


\begin{defn}
    Let $P$ and $Q$ be continuous complex valued functions on a domain $D$. We say that the line integral $\int Pdx+Qdy$ is \Emph{independent of path} in $D$ if for any two points $A$ and $B$ in $D$, the integrals $\int_{\gamma}Pdx+Qdy$ are the same for any path $\gamma$ in $D$ from $A$ to $B$.
\end{defn}

\begin{lem}
    Let $P$ and $Q$ be continuous complex valued functions on a domain $D$. Then $\int Pdx+Qdy$ is independent of path in $D$ if and only if $Pdx+Qdy$ is exact, that is, there exists a continuously differentiable function $h(x,y)$ such that $dh = Pdx+Qdy$ on $D$. Moreover, the function $h$ is unique up to adding a constant.
\end{lem}



\begin{defn}
    Let $P$ and $Q$ be complex valued functions on some domain $D$. Then $Pdx+Qdy$ is a \Emph{closed form} on a domain $D$ if and only if $\partial_yP = \partial_xQ$ on $D$.
\end{defn}

\begin{prop}
    If $\omega$ is exact on $D \subseteq \C$, then $\omega$ is closed.
\end{prop}
\begin{proof}
    Let $\omega$ be exact on $D$. Thus, there exists $h:D\rightarrow \C$ such that $\omega = \frac{\partial h}{\partial x}dx+\frac{\partial h}{\partial y}dy$. Moreover, $P = \frac{\partial h}{\partial x}$ and $Q = \frac{\partial h}{\partial y}$. As will will show later, since $h$ is complex differentiable on the domain $D$, it is complex smooth on $D$ and hence has continuous partial derivatives of all orders. Thus, by Clairout's Theorem $$\partial_yP = \partial_y \partial_xh = \partial_x\partial_yh = \partial_xQ$$
    so indeed $\omega$ is closed.
\end{proof}


\begin{eg}
    Let $\omega = \frac{-ydx + xdy}{x^2+y^2}$ on $\C^{\times}$. Then observe that $$\partial_y\frac{-y}{x^2+y^2} = \frac{-x^2-y^2 + 2y^2}{(x^2+y^2)^2} = \frac{y^2-x^2}{(x^2+y^2)^2}$$ and $$\partial_x\frac{x}{x^2+y^2} = \frac{x^2+y^2 - 2x^2}{(x^2+y^2)^2} = \frac{y^2-x^2}{(x^2+y^2)^2}$$
    so indeed $\omega$ is a closed form. Recall if $\omega$ is exact, than the integral around a loop in $\C$ is $0$. Observe \begin{align*}
        \oint_{|z|=1}\omega &= \int_{0}^{2\pi}\left(\frac{-\sin(t)}{1^2}\cdot (-\sin(t))+\frac{\cos(t)}{1^2}\cdot \cos(t)\right)dt \\
        &= \int_{0}^{2\pi}(\sin^2(t)+\cos^2(t))dt \\
        &= 2\pi \neq 0
    \end{align*}
    so $\omega$ cannot be exact on all of $\C^{\times}$.
\end{eg}


\begin{thm}
    Let $P$ and $Q$ be continuously differentiable complex valued functions on a domain $D$. Suppose \begin{itemize}
        \item $D$ is a star-shaped domain, and 
        \item the differential $Pdx+Qdy$ is closed
    \end{itemize}
    Then $Pdx+Qdy$ is exact on $D$.
\end{thm}
\begin{proof}
    Suppose that $A$ is a star center of $D$. For all $B \in D$, we define \begin{equation*}
        h(B) = \int_{[A,B]}Pdx+Pdy
    \end{equation*}
    where $[A,B]$ is a line-segment and is in $D$ since $D$ is star-shaped with respect to $A$. Fix $B = (x_0, y_0)$, and let $C = (x,y_0)$ lie on the horizontal line through $B$ and close enough to $B$ so that the triangle with vertices $A, B, C$ lies within $D$. We apply Green's Theorem to the triangle to obtain \begin{equation*}
        \left(\int_{[A,B]} + \int_{[B,C]}+\int_{[C,A]}\right)(Pdx+Qdy) = 0
    \end{equation*}
    since the form is closed on $D$. Thus, \begin{equation*}
        \int_{[A,C]}(Pdx+Qdy)-\int_{[A,B]}(Pdx+Qdy) = \int_{[B,C]}(Pdx+Qdy)
    \end{equation*}
    or equivalently \begin{equation*}
        h(x,y_0) - h(x_0,y_0) = \int_{x_0}^xP(t,y_0)dt
    \end{equation*}
    From the fundamental theorem of calculus we obtain \begin{equation*}
        \frac{\partial h}{\partial x}(x_0,y_0) = P(x_0,y_0)
    \end{equation*}
    Similarly, we can obtain \begin{equation*}
        \frac{\partial h}{\partial y}(x_0,y_0) = Q(x_0,y_0)
    \end{equation*}
    (traversing along vertical lines). Consequently, $dh = Pdx+Qdy$, and $Pdx+Qdy$ is exact.
\end{proof}


\begin{thm}
    Let $D$ be a domain, and let $\gamma_0(t)$ and $\gamma_1(t)$, $a \leq t \leq b$, be two paths in $D$ from $A$ to $B$. Suppose that $\gamma_0$ can be continuously deformed to $\gamma_1(t)$, in the sense that for $0 \leq s \leq 1$ there are paths $\gamma_s(t)$, $a \leq t \leq b$, from $A$ to $B$ such that $\gamma_s(t)$ depends continuously on $s$ and $t$ for $0 \leq s \leq t$, and $a \leq t \leq b$. Then \begin{equation*}
        \int_{\gamma_0}Pdx+Qdy = \int_{\gamma_1}Pdx+Qdy
    \end{equation*}
    for any closed differential $Pdx+Qdy$ on $D$.
\end{thm}

The idea of the proof can be attained from Green's Theorem applied to $D$ using the closedness of the differential. Another idea of this is that if we can continuosly stretch two curves from one into the other over a region for which the form is closed, the integral over the original curve and the deformed curve are equal.


%%%%%%%%%%%%%%%%%%%% Section 1.3.2
\section{Harmonic Conjugates}

\begin{rec}
    If $f= u+iv$ then $f$ is \Emph{harmonic} on $D$ if and only if $u$ and $v$ are harmonic on $D$: $u_{xx}+u_{yy} = 0$ and $v_{xx}+v_{yy}=0$.
\end{rec}


\begin{lem}
    If $u$ is harmonic, then $-\partial_yudx + \partial_xudy$ is closed.
\end{lem}
\begin{proof}
    Assume $u$ is harmonic. Then $u_{xx}+u_{yy} = 0$. For the form to be closed we need $-\partial_y\partial_yu = \partial_x\partial_xu$, but this is precisely the condition for $u$ being harmonic, with $-u_{yy} = u_{xx}$.
\end{proof}

Then, if $D$ is star-shaped and $u$ is harmonic, then $-\partial_yudx+\partial_xudy = dv = \partial_xvdx+\partial_yvdy$ is exact. Hence, $-u_y = v_x$ and $u_x = v_y$, so the Cauchy Riemann equations hold and consequently since $u$ and $v$ are continuously real differentiable on $D$ by assumption, $f = u+iv$ is complex differentiable on $D$, so $f = u+iv \in \mathcal{O}(D)$. From our previous argument for closed implies exact on a star shaped domain, explicitly we have \begin{equation*}
    v(B) = \int_A^B(-\partial_yudx+\partial_xudy)
\end{equation*}
where $A$ is fixed and the integral is path independent in $D$.

\begin{eg}
    Consider $u = \ln|z|$ for $D = \C^-$, star-shaped, and we express $u$ in the form \begin{equation*}
        u(x,y) = \frac{1}{2}\ln(x^2+y^2)
    \end{equation*}
    and we compute \begin{equation*}
        du = \frac{x}{x^2+y^2}dx + \frac{y}{x^2+y^2}dy
    \end{equation*}
    our equation in our previous discussion becomes \begin{equation*}
        dv = \frac{-y}{x^2+y^2}dx + \frac{x}{x^2+y^2}dy
    \end{equation*}
    Then we have \begin{equation*}
        v = \int_1^z\frac{-y}{x^2+y^2}dx + \frac{x}{x^2+y^2}dy, z \in \C^-
    \end{equation*}
    and this is in fact the principal branch of the argument function $\text{Arg}(z) = v(z)$ on $\C^-$, normalized to vanish at $z = 1$. This gives the holomorphic function $f = u+iv = \ln|z|+i\text{Arg}(z)=\text{Log}(z)$.
\end{eg}


%%%%%%%%%%%%%%%%%%%% Section 1.3.3
\section{The Mean Value Property}

\begin{defn}
    Let $h:D\rightarrow \R$ be a continuous real valued function on a domain $D$, $z_0 \in D$ such that the disk $\{z \in \C\vert |z-z_0| < \rho\} = D_{\rho}(z_0)\subseteq D$, then the \Emph{average value} of $h(z)$ on the the circle $\{z \in \C:|z-z_0| < r\} = D_r(z_0)$ to be \begin{equation*}
        A(r) = \int_0^{2\pi}h(z_0+re^{i\theta})\frac{d\theta}{2\pi}, \;\;\; 0 < r < \rho 
    \end{equation*}
\end{defn}
Note the relation $ds = rd\theta$, for $ds$ infinitesimal arclength, so this formula indeed describes an integral over the circle divided by its one dimensional volume, i.e., circumference.

Since $h(z)$ is continuous, the average value $A(r)$ varies continuously with the radius $r$. Moreover, when $r$ is small, $A(r)$ tends to $h(z_0)$.

\begin{thm}
    If $u(z)$ is a harmonic function on a domain $D$, and if the disk $\{z \in \C:|z-z_0| < \rho\} \subseteq D$, then \begin{equation*}
        u(z_0) = \int_0^{2\pi}u(z_0+re^{i\theta})\frac{d\theta}{2\pi}, \;\;\;\; 0 < r < \rho
    \end{equation*}
\end{thm}
\begin{proof}
    Note harmonic implies $-u_ydx+u_xdy$ is closed, which implies $\oint_{|z-z_0| = r}(-u_ydx+u_xdy) = 0$. Parametrizing the circle as $x(\theta) = x_0+r\cos(\theta)$ and $y(\theta) = y_0+r\sin(\theta)$, and we obtain \begin{equation*}
        0 = r\int_{0}^{2\pi}\left[u_x\cos(\theta)+u_y\sin(\theta)\right]d\theta = r\int_0^{2\pi}\frac{\partial u}{\partial r}(z_0+re^{i\theta})d\theta
    \end{equation*}
    Since $u(z)$ is smooth, we can interchange the order of integration and differentiation. We obtain after dividing by $2\pi r$ that \begin{equation*}
        0 = \frac{\partial}{\partial r}\int_0^{2\pi}u(z_0+re^{i\theta})\frac{d\theta}{2\pi} = \frac{dA(r)}{dr}
    \end{equation*}
    Thus, $A(r)$ is constant for $0 < r < \rho$, since we are in a connected open set. Since $u(z)$ is continuous at $z_0$, the average value tends to $u(z_0)$, and $A(r) = u(z_0)$.
\end{proof}


In other words, the average value of a harmonic function on the boundary circle of any disk contained in $D$ is its value at the center of the disk.


\begin{defn}
    We say that a function $h(z)$ on a domain $D$ has the \Emph{mean value property} if for each $z_0 \in D$, $h(z_0)$ is the average of its values over any small circle centered at $z_0$. That is, for all $z_0 \in D$, there exists $\varepsilon > 0$ such that \begin{equation*}
        h(z_0) = \int_0^{2\pi}u(z_0+re^{i\theta})\frac{d\theta}{2\pi},\;\;\;\; 0 < r < \varepsilon
    \end{equation*}
\end{defn}


Thus, from our theorem we have that harmonic functions have the mean value property.



%%%%%%%%%%%%%%%%%%%% Section 1.3.4
\section{The Maximum Principle}


\begin{namthm}[Strict Maximum Principle (Real)]
    Let $u(z)$ be a real-valued harmonic function on a domain $D$ such that $u(z) \leq M$ for all $z \in D$. If $u(z_0) = M$ for some $z_0 \in D$, then $u(z) = M$ for all $z \in D$.
\end{namthm}

The proof follows from the observation that using the mean value property of harmonic functions, the set $\{u(z) = M\}$ is open, and by continuity of $u$, $\{u(z) < m\}$ is open, so as $D$ is a domain it is in particular connected and hence one of these sets must be empty, since $D$ is the disjoint union of these open sets.


\begin{namthm}[Strict Maximum Principle (Complex)]
    Let $h$ be a bounded complex-valued harmonic function on a domain $D$. If $|h(z)| \leq M$ for all $z \in D$, and $|h(z_0)| = M$ for some $z_0 \in D$, then $h(z)$ is constant on $D$.
\end{namthm}

The following version of the maximum principle asserts that a complex valued harmonic function on a bounded domain attains its maximum modulus on the boundary:

\begin{namthm}[Maximum Principle]
    Let $h(z)$ be a complex-valued harmonic function on a bounded domain $D$ such that $h(z)$ extends continuously to the boundary $\partial D$ of $D$. If $|h(z)|\leq M$ for all $z \in \partial D$, then $|h(z)|\leq M$ for all $z \in D$.
\end{namthm}

The proof of this principle follows from the fact that compact sets remain compact under continuous transformations, along with the fact that $\C$ satisfies the Heine-Borel theorem, stating that the complex sets are precisely the closed and bounded sets. In this case the compact set is the union $D\cup \partial D$. If the harmonic function attains its maximum modulus at some point of $D$, then it is constant. Thus in all cases it attains its maximum modulus on the boundary of $D$.




%%%%%%%%%%%%%%%%%%%% Section 1.3.5
\section{Applications to Physics}

Recall, for a vector field $V = \langle P,Q\rangle$ in $\R^2 = \C$, we have a few different methods of integration: \begin{equation*}
    circulation\;of\;V = \int_{\gamma}(V\cdot T)ds = \int_{\gamma}Pdx+Qdy
\end{equation*}
\begin{equation*}
    flux\;of\;V\;through\;\gamma = \int_{\gamma}(V\cdot n)ds = \int_{\gamma}Pdy-Qdx
\end{equation*}
where $T = \frac{d\gamma}{dt}/\left|\frac{d\gamma}{dt}\right|$, $ds = \left|\frac{d\gamma}{dt}\right|dt$, and $n = -\langle -\frac{dy}{dt},\frac{dx}{dt}\rangle/\left|\frac{d\gamma}{dt}\right|$.

\begin{eg}
    Let $V = x+iy = \langle x,y\rangle$. Parametrize the unit circl $x = \cos\theta, y = \sin\theta$, $0 \leq \theta \leq 2\pi$. Then 
    \begin{equation*}
        \int_{\gamma}(V\cdot n)ds = \int_0^{2\pi}Pdy-Qdx = \int_0^{2\pi}(\cos^2\theta-(-\sin^2\theta))d\theta = 2\pi
    \end{equation*}
\end{eg}

We consider fluid flow in a 2D domain $D$ in the plane. We associate with the particle at the point $z$ its velocity vector $V(z) = P+iQ$. The direction of $V(z)$ is the direction the particle is moving and the magnitude $|V(z)|$ is its speed. We make the following assumptions: 
\begin{itemize}
    \item The flow is independent of time, so $V(z)$ does not change with time.
    \item There are no sources or sinks in $D$; no fluid is created or destroyed - i.e. flux is zero for small loops in $D$
    \item The flow is incompressible; that is, the density of the fluid is the same at each point in $D$
    \item The flow is irrotational; that is, there is no circulation of fluid around small circles in $D$.
\end{itemize}
That is, for any $\varepsilon > 0$ and $z_0 \in D$ such that $D_{\varepsilon}(z_0) \subseteq D$, we have \begin{equation*}
    \int_{\partial D_{\varepsilon}(z_0)}(V\cdot n)ds = \int_{\partial D_{\varepsilon}(z_0)}(Pdy-Qdx)= 0
\end{equation*}
and \begin{equation*}
    \int_{\partial D_{\varepsilon}(z_0)}(V\cdot T)ds = \int_{\partial D_{\varepsilon}(z_0)}(Pdx+Qdy) = 0
\end{equation*}
Thus, these conditions imply that the differential forms $Pdy-Qdx$ and $Pdx+Qdy$ are path independent, and hence exact on $D$. So we can write $Pdx+Qdy = d\phi$, which is real, $\phi:D\rightarrow \R$, since $P$ and $Q$ are real valued, and is called the potential. Then $\phi_x = P$ and $\phi_y = Q$. Let $Pdy-Qdx = d\varphi$. Then $\varphi_x = -Q = -\phi_y$ and $\varphi_y = P = \phi_x$, so $\phi$ and $\varphi$ satisfy the Cauchy-Riemann equations. Thus, $\phi+i\varphi$ is complex differentiable on $D$. Thus, from a pre-cursory result $\phi$ and $\varphi$ are harmonic, that is $\phi_{xx}+\phi_{yy} = 0$ and $\varphi_{xx}+\varphi_{yy} = 0$.

\begin{eg}
    Consider the constant flow $V = C$ on the upper half-plane, for $C > 0$. Then $\phi_x = C,\phi_y = 0$, so $\phi = Cx$ is a potential. The harmonic conjugate is given by $\varphi_x = -0 = 0,\varphi_y = C$, so $\varphi= Cy$ is a stream function for the flow. Note the level curves of $\phi$ and $\varphi$ given orthogonal trajectories. Then our complex potential is $f(z) = c(x+iy)=cz$. 
\end{eg}
The level curves of the \Emph{stream function}, $\varphi$, are streamlines along the flow. On the other hand, the potential function $\phi$, has the velocity field going perpendicular to it.

\begin{eg}
    Consider the velocity field $V(z) = \frac{x+iy}{x^2+y^2}$. Then, we have the potential $\phi(x+iy) = \ln\sqrt{x^2+y^2} = \ln|z|$, so the stream function is $\varphi(z) = \text{Arg}(z)$, so $f(z) = \phi(z)+i\varphi(z) = \ln|z|+i\text{Arg}(z) = \text{Log}(z)$ is the principal logarithm on $\C^-$. 
\end{eg}


If we have an electric field $E = \langle P,Q\rangle$, we can express it in terms of an electric potential as $E = -\nabla \alpha$ for $\alpha$, the voltage function. 

\subsection{The Laplace Equation}

We consider the laplace equation $u_{xx}+u_{yy} = 0$. If we want to solve this on the upper half plane with certain boundary conditions $u(z) = -1$ for $x < 0, y = 0$, and $u(z) = 0$ for $x > 0, y = 0$. Then, looking at $\text{Arg}(z)$ we know that it is harmonic on this domain since $\text{Log}(z) = \ln|z| + i\text{Arg}(z)$ is holomorphic on this branch, and $\text{Arg}(z)$ goes to $0$ on the positive $x$ axis and $\pi$ on the negative $x$ axis. Then $u(z) = -\frac{1}{\pi}\text{Arg}(z)$ is a solution to the laplace equations. 

Recall that if $f=u+iv \in \mathcal{O}(D)$, then $\nabla^2u = 0$ and $\nabla^2v = 0$ satisfy the laplace equations. If $f$ and $g$ are both holomorphic, with $f$ holomorphic on a domain containing the image of $g$, then $h(z) = f(g(z))$ is holomorphic. We can think of $w = f(z) = u+iv$ as a function from the $z$ plane to the $w$ plane. Then, if the derivative of $f$ is non-zero we have by the inverse function theorem a local holomorphic inverse $z = f^{-1}(w) = x+iy$ from the $w$ plane to the $z$ plane. Moreover, $u,v$ are harmonic functions of $x,y$, and $x,y$ are harmonic functions of $u,v$ (as an abuse of language). Then, if we have some set of boundary conditions on the domain $D$, i.e. conditions on $\partial D$, in the $z$ plane, we obtain boundary conditions on $\partial f(D) = f(\partial D)$ in the $w$ plane.

\begin{eg}
    Suppose we wish to solve Laplace's equation $\nabla^2\phi$ in the anulus with inner radius $1$ and outer radius $4$, with inner boundary condition $\phi = 3$ and outer boundary condition $\phi = 7$. Then it turns out the general solution is of the form $\phi(z) = A\ln|z| +B$, which is real valued (this can be interpreted as a voltage function). Then, on the inner radius $3 = \phi = A\ln|1| + B = B$. On the outer radius we have $7 = \phi = A\ln|4| + 3$, so $A = 2/\ln(2)$. Thus, our solution to Laplace's equation is $\phi(z) = \frac{2}{\ln(2)}\ln|z| + 3$.
\end{eg}

\begin{eg}
    We can extend the previous example to an anulus centered at a point $z_0$, which gives a general solution $\phi(z) = A\ln|z-z_0|+B$.
\end{eg}


\begin{eg}
    Consider the segment of the complex plane between $i5$ and $i8$, with boundary conditions $\phi(x+i5) = 3$ and $\phi(x+i8) = 10$. We consider a solution $\phi(x,y) = Ay+B$. Then, $3 = \phi(x+i5) = 5A+B$ and $10 = \phi(x+i8) = 8A+B$, so $3A = 7$ and $A = 7/3$. Moreover, $B = 3 - 35/3 = -26/3$, so our solution is $\phi(x,y) = \frac{7}{3}y - \frac{26}{3}$.
\end{eg}

General principal in mathematical physics: the presence of a singularity in the field indicates the presence of charge. Topologically speaking, the presence of a singularity in a vector field indicates a non-trivial topology.


\begin{eg}
    Consider a unit circle where we want to solve $\nabla^2\phi = 0$ with the boundary conditions $\phi = 1$ on the upper half-sphere, $\phi = -1$ on the lower half-sphere, and undefined on the two connecting points. We map the disk to the vertical half-plane, mapping $i$ to $i$, $-1$ to $0$, and $-i$ to $-i$ (and $1$ to $\infty$). So $f(z) = \frac{z+1}{1-z} = w$ and $f^{-1}(w) = \frac{w-1}{w+1}$. On the $w$ plane, $\theta = \pi/2$ when $\psi = \phi(f^{-1})$ is $1$, and $\theta = -\pi/2$ when $\psi = \phi(f^{-1})$ is $-1$. Then our solution is $\psi(w) = \frac{2}{\pi}\text{Arg}(w)$. Then $\phi(z) = \psi(f(z))$, so \begin{equation*}
        \phi(z) = \psi(f(z)) = \frac{2}{\pi}\text{Arg}(f(z)) = \frac{2}{\pi}\text{Arg}\left(\frac{x+iy+1}{1-x-iy}\right)
    \end{equation*}
    Then simplifying we have \begin{equation*}
        \phi(z) = \frac{2}{\pi} \text{Arg}\left(\frac{1-x^2-y^2+2iy}{(1-x)^2+y^2}\right) = \frac{2}{\pi}\text{Arg}(1-x^2-y^2+2iy) = \frac{2}{\pi}\tan^{-1}\left(\frac{2y}{1-x^2-y^2}\right)
    \end{equation*}
\end{eg}




%%%%%%%%%%%%%%%%%%%%%% Chapter 1.4
\chapter{Complex Integration and Analyticity}

%%%%%%%%%%%%%%%%%%%% Section 1.4.1
\section{Complex Line Integrals}

Let $h(z)$ be a complex valued function on a curve $\gamma$, then defining $dz = dx+idy$ we define \begin{equation*}
    \int_{\gamma}h(z)dz = \int_{\gamma}h(z)dx + i\int_{\gamma}h(z)dy
\end{equation*}
Suppose $\gamma$ is parameterized by $t \mapsto z(t) = x(t)+iy(t)$, $a \leq t \leq b$. The Riemann sum approximating $\int_{\gamma}h(z)(dx+idy)$ corresponding to the subdivision $a = t_0 < t_1 < ... < t_n = b$ is given by \begin{equation*}
    \sum_{i=1}^nh(z_i)\delta x_i+i\sum_{i=1}^nh(z_i)\delta y_i
\end{equation*}
where $z(t_j) = z_j = x_j + iy_j$. Then we obtain \begin{equation*}
    \int_{\gamma}h(z)dz = \lim\limits_{n\rightarrow \infty}\sum_{i=1}^nh(z_i)\delta z_i
\end{equation*}


Recall that if $\gamma$ is a smooth path in $\C$, then for a analytic complex valued function $f:D\rightarrow \C$, \begin{equation*}
    \frac{d}{dt}\left[f(\gamma(t))\right] = \frac{df}{dz}(\gamma(t))\frac{d\gamma}{dt}(t)
\end{equation*}
Then we have $\frac{d}{dt}(e^{\lambda t}) = \lambda e^{\lambda t}$ and \begin{equation*}
    \int e^{\lambda t}dt = \frac{1}{\lambda}e^{\lambda t}+C
\end{equation*}
Thus, if $\lambda = \alpha+i\beta$, $e^{\lambda t} = e^{\alpha t}\cos(\beta t)+ie^{\alpha t}\sin(\beta t)$, so this says \begin{align*}
    \int e^{\alpha t}\cos(\beta t)dt + i\int e^{\alpha t}\sin(\beta t)dt &= \frac{1}{\alpha+i\beta}(e^{\alpha t}\cos(\beta t)+ie^{\alpha t}\sin(\beta t)) + C \\
    &= \frac{\alpha-i\beta}{\alpha^2+\beta^2}(e^{\alpha t}\cos(\beta t)+ie^{\alpha t}\sin(\beta t)) + C \\
    &= \frac{e^{\alpha t}}{\alpha^2+\beta^2}[\alpha\cos(\beta t) + \beta\sin(\beta t) + i(\alpha\sin(\beta t) - \beta\cos(\beta t))] + C
\end{align*}
equating the real and complex terms we have \begin{equation*}
    \int e^{\alpha t}\cos(\beta t)dt = \frac{e^{\alpha t}}{\alpha^2+\beta^2}[\alpha\cos(\beta t) + \beta\sin(\beta t)] + C
\end{equation*}
and \begin{equation*}
    \int e^{\alpha t}\sin(\beta t)dt = \frac{e^{\alpha t}}{\alpha^2+\beta^2}[\alpha\sin(\beta t) - \beta\cos(\beta t)] + C
\end{equation*}

\begin{defn}
    We can rewrite our line integral as follows for $f = u+iv$: \begin{equation*}
        \int_{\gamma}f(z)dz = \int_{\gamma}(u+iv)(dx+idy) = \int_{\gamma}(udx-vdy)+i\int_{\gamma}(vdx+udy)
    \end{equation*}
\end{defn}

for $\gamma$ a parameterization we have \begin{equation*}
    \int_{\gamma}f(z)dz = \int_{\gamma}f(z)\frac{d\gamma}{dt}dt
\end{equation*}

\begin{eg}
    Consider the ccw-oriented unit circle, and the following integral taken along it: \begin{equation*}
        \int_{C}\frac{dz}{z} = \int_{0}^{2\pi}\frac{ie^{it}dt}{e^{it}} = \int_{0}^{2\pi}idt = i2\pi
    \end{equation*}
    using the parameterization $z = e^{it}$, so $dz = ie^{it}dt$.
\end{eg}

\begin{eg}
    Consider the circle centered at $z_0$ with radius $R$, so we have ccw parameterization $z(t) = z_0 + Re^{it}$, $0 \leq t \leq 2\pi$: \begin{align*}
        \int_C(z-z_0)^ndz &= \int_{0}^{2\pi}(Re^{it})^n(iRe^{it})dt = \int_{0}^{2\pi}iR^{n+1}e^{i(n+1)t}dt \\
        &= \frac{iR^{n+1}}{i(n+1)}e^{i(n+1)t}\Bigg\rvert_0^{2\pi} \\
        &= 0
    \end{align*}
    for $n \in \Z$ and $n \neq -1$ (so that it is single valued on the whole complex plane). If $n = -1$ then the integral is $2\pi i$. 
\end{eg}


\begin{rec}
    If $\omega$ is a closed differential form an a region between and containing two simple closed curves, $\gamma_1$ and $\gamma_2$, then one curve can be deformed into the other such that \begin{equation*}
        \int_{\gamma_1}\omega = \int_{\gamma_2}\omega
    \end{equation*}
\end{rec}
This implies that in our previous example, for $n = -1$, for any simple closed curve encircling the origin we obtain $2\pi i$ by integrating the function along it.

\begin{defn}
    For a smooth curve $\gamma$ and a complex valued function $f$ we define \begin{equation*}
        \int_{\gamma}f(z)|dz| = \int_{t_0}^{t_1}f(\gamma(t))\left|\frac{d\gamma}{dt}\right|dt
    \end{equation*}
    where we often write $|dz| = ds$, which gives the arclength integrals \begin{equation*}
        \int_{\gamma}f(z)|dz| = \int_{\gamma}uds + i\int_{\gamma}vds
    \end{equation*}
    for $f = u+iv$.
\end{defn}



\begin{namthm}[ML Theorem]
    Suppose $\gamma$ is a piecewise smooth curve. If $h(z)$ is a continuous function on $\gamma$, then \begin{equation*}
        \left|\int_{\gamma}h(z)dz\right|\leq \int_{\gamma}|h(z)||dz|
    \end{equation*}
    Further, if $\gamma$ has length $\int_{\gamma}|dz| = L$, and $|h(z)| \leq M$ on $\gamma$, then \begin{equation*}
        \left|\int_{\gamma}h(z)dz\right| \leq ML
    \end{equation*}
\end{namthm}


\begin{eg}
    Consider $f(z) = \frac{1}{z}$ on $|z| = 1$. Then $|f(z)| = \frac{1}{|z|} = 1$, which we can use for $M$. Moreover, the length is $L = 2\pi$. Thus, we have by the ML theorem that \begin{equation*}
        \left|\int_C \frac{dz}{z}\right| \leq 2\pi
    \end{equation*}
\end{eg}


\begin{eg}
    Consider \begin{equation*}
        \int_{|z| = R}\frac{dz}{z^2-6}
    \end{equation*}
    Observe $|z^2 - 6| \geq |z^2| - |6| = R^2-6$, so assuming $R > 3$, $\frac{1}{z^2-6} \leq \frac{1}{R^2-6} = M$, and $L = 2\pi R$. Thus, by the ML theorem \begin{equation*}
        \left|\oint_{|z| = R}\frac{dz}{z^2-6}\right| \leq \frac{2\pi R}{R^2-6}
    \end{equation*}
\end{eg}



%%%%%%%%%%%%%%%%%%%% Section 1.4.2
\section{Fundamental Theorem of Calculus for Analytic Functions}

\begin{defn}
    Let $f(z)$ be a continuous function on a domain $D$. A function $F(z)$ on $D$ is a \Emph{(complex) primitive} for $f(z)$ if $F(z)$ is analytic and $F'(z) = f(z)$ for all $z \in D$.
\end{defn}

First, observe that if $F'(z) = f(z)$ for all $z \in D$, then \begin{equation*}
    \int_{\gamma}f(z)dz = \int_{\gamma}F'(z)dz = \int_{\gamma}dF = F(\gamma(t_1)) - F(\gamma(t_0))
\end{equation*}
as $dF = F'(z)dx+iF'(z)dy = F'(z)dz$, and of course $dF$ is an exact form so it is path independent. 

\begin{thm}
    If $f(z)$ is continuous on a domain $D$ with primitive $F(z)$, then \begin{equation*}
        \int_A^Bf(z)dz = F(A) - F(B)
    \end{equation*}
    where the integral can be taken over any path in $D$ from $A$ to $B$.
\end{thm}

\begin{rmk}
    $f(z) = \frac{1}{z}$ cannot have a primitive on $\C^{\times}$. Indeed, if it did it would be path independent, and hence its integral would be zero over all closed loops.
\end{rmk}

\begin{thm}
    Let $D$ be a star shaped domain, and let $f(z)$ be holomorphic on $D$. Then $f(z)$ has a primitive on $D$, and the primitive is unique up to a constant. A primitive for $f(z)$ is given explicitly by \begin{equation*}
        F(z) = \int_{z_0}^{z}f(w)dw
    \end{equation*}
    for $z \in D$, where $z_0$ is any fixed point in $D$ for which the integral can be taken in $D$.
\end{thm}


\begin{eg}
    Consider $\int \frac{dz}{z}$ for parameterization $z = e^{it}$, $-\pi + \varepsilon \leq t \leq \pi - \varepsilon$, where $\varepsilon > 0$. Then \begin{equation*}
        \int\frac{dz}{z} = \text{Log}(z)\vert_{e^{i(\varepsilon-\pi)}}^{e^{i(\pi-\varepsilon)}} = i(\pi-\varepsilon) - i(\varepsilon - \pi) = 2i(\pi-\varepsilon)
    \end{equation*}
\end{eg}


%%%%%%%%%%%%%%%%%%%% Section 1.4.3
\section{Cauchy's Theorem and Integral Formula}


Let $f(z) = u+iv$ be a smooth complex valued function, and we express $f(z)dz = (u+iv)(dx+idy) = (u+iv)dx+(-v+iu)dy$. The condition that $f(z)dz$ is a closed differential is \begin{equation*}
    \frac{\partial}{\partial y}(u+iv) = \frac{\partial}{\partial x}(-v+iu)
\end{equation*}
so equating real and imaginary components we obtain $u_y = -v_x$ and $v_y = u_x$, in other words the Cauchy Riemann equations. 

\begin{thm}
    A continuously differentiable function $f(z)$ on $D$ is analytic if and only if the differential $f(z)dz$ is closed.
\end{thm}

From Green's theorem we then obtain the following:

\begin{namthm}[Cauchy's Theorem]
    Let $D$ be a bounded domain with piecewise smooth boundary. If $f(z)$ is holomorphic and continuously differentiable on $D$ that extends continuously to $\partial D$, then \begin{equation*}
        \int_{\partial D}f(z)dz = 0
    \end{equation*}
\end{namthm}

Extending continuously to the boundary of $D$ means we can extend to a slightly bigger domain which contains the boundary and in which the hypotheses hold.


\begin{namthm}[Cauchy's Integral Formula]
    Let $D$ be a bounded domain with piecewise smooth boundary. If $f(z)$ is holomorphic on $D$, and $f(z)$ extends smoothly to the boundary $\partial D$, then for each $z \in D$ \begin{equation*}
        f(z) = \frac{1}{2\pi i}\int_{\partial D}\frac{f(w)}{w-z}dw,
    \end{equation*}
\end{namthm}
\begin{proof}
    Assume the conditions of the theorem. Let $z \in D$, and since $D$ is open there exists $\varepsilon > 0$ such that $\{w \in \C: |z-w| < \varepsilon\} \subseteq D$. Then, define $D_{\varepsilon} = D\backslash \{w \in \C:|z-w| \leq \varepsilon\}$. The boundary $\partial D_{\varepsilon}$ is the union $\partial D\cup\{w \in \C:|w-z|=\varepsilon\}$, with the circle being clockwise oriented. Since $f(z)/(w-z)$ is analytic for $z \in D_{\varepsilon}$, Cauchy's theorem yields: \begin{equation*}
        \int_{\partial D_{\varepsilon}}\frac{f(w)}{w-z}dw = 0
    \end{equation*}
    Reversing the orientation of the circle to counter clockwise produces a sign change, which gives \begin{equation*}
        0 = \int_{\partial D}\frac{f(w)}{w-z}dw - \int_{|w-z| = \varepsilon, ccw}\frac{f(w)}{w-z}dw
    \end{equation*}
    so \begin{equation*}
        \int_{|w-z|=\varepsilon,ccw} \frac{f(w)}{w-z}dw = \int_{\partial D}\frac{f(w)}{w-z}dw
    \end{equation*}
    Then we have parameterization $w = z+\varepsilon e^{it}$ for the circle, $0 \leq t \leq 2\pi$. Then observe \begin{equation*}
        \int_{0}^{2\pi}\frac{f(z+\varepsilon e^{it})}{\varepsilon e^{it}}\varepsilon ie^{it}dt = 2\pi i\int_0^{2\pi}f(z+\varepsilon e^{it})\frac{dt}{2\pi}
    \end{equation*}
    But, then by the mean value property of analytic functions we have that the integral on the right hand side coincides with $2\pi if(z)$, so \begin{equation*}
        f(z) = \frac{1}{2\pi i}\int_{\partial D}\frac{f(w)}{w-z}
    \end{equation*}
\end{proof}

If we differentiate under the integral sign and use \begin{equation*}
    \frac{d^m}{dz^m}\frac{1}{w-z} = \frac{m!}{(w-z)^{m+1}}
\end{equation*}
we obtain integral formulae for $f^{(m)}(z)$ of $f(z)$.

\begin{namthm}[Cauchy Integral Formula (general)]
    Let $D$ be a bounded domain with piecewise smooth boundary. If $f(z)$ is holomorphic on $D$, and $f(z)$ extends smoothly to the boundary $\partial D$, then $f(z)$ has complex derivatives of all orders on $D$, which are given for each $z \in D$ by \begin{equation*}
        f^{(m)}(z) = \frac{m!}{2\pi i}\int_{\partial D}\frac{f(w)}{(w-z)^{m+1}}dw,
    \end{equation*}
    for $m \geq 0$.
\end{namthm}

\begin{cor}
    If $f(z)$ is analytic on a domain $D$, then $f(z)$ is infinitely differentiable, and the successive complex derivatives $f'(z),f''(z),...,$ are all analytic on $D$.
\end{cor}


\begin{eg}
    Consider \begin{align*}
        \oint_{|z| = 2}\frac{\sin(2i)}{(z-i)^6}dz &= \frac{2\pi i}{5!}\frac{d^5}{dz^5}\Bigg\rvert_{z=i}\sin(2z) \\
        &= \frac{2^6\pi i}{5!}\cos(2i) \\
        &= \frac{8\pi i\cos(2i)}{15} = \frac{8\pi i\cosh(2)}{15}
    \end{align*}
\end{eg}


We usually use the Cauchy integral formula in the form \begin{equation*}
    \int_{\partial D}\frac{f(w)}{(w-z_0)^{m+1}}dw = \frac{2\pi if^{(m)}(z_0)}{m!}
\end{equation*}

\begin{eg}
    Consider a domain without $z_0$, so by Cauchy's Theorem \begin{align*}
        \int_{\partial D}\frac{\sin(z^2)dz}{(z-z_0)^2} = 0 
    \end{align*}
    If $z_0 \in D$, then \begin{equation*}
        \int_{\partial D}\frac{\sin(z^2)dz}{(z-z_0)^2} = 2\pi i\frac{d}{dz}\Bigg\rvert_{z=z_0}\sin(z^2) = 2\pi i2z_0\cos(z_0^2) = 4\pi iz_0\cos(z_0^2)
    \end{equation*}
\end{eg}


\begin{eg}
    Consider the circle $C= \{|z+i| = 1/2\}$, and we find \begin{align*}
        \int_C\frac{dz}{z^4+1} &= \int_C\frac{dz}{(z-e^{i\pi/4})(z-e^{3i\pi/4})(z-e^{5i\pi/4})(z-e^{7i\pi/4})} \\
        &= 0
    \end{align*}
    since none of the roots are contained in the inside of $C$, so the integrand is holomorphic on the disk contained in the circle. If we consider $C= \{|z+i| = 1\}$, we have two singularities inside the circle. 
\end{eg}



%%%%%%%%%%%%%%%%%%%% Section 1.4.4
\section{Liouville's Theorem}


Suppose that $f(z)$ is holomorphic on some domain containing the disk $\{z \in \C:|z-z_0| \leq \rho\}$. Then by Cauchy's integral formula \begin{equation*}
    f^{(m)}(z_0) = \frac{m!}{2\pi i}\int_{|z-z_0|=\rho}\frac{f(z)}{(z-z_0)^{m+1}}dz
\end{equation*}
We parametrize the boundary circle by $z = z_0 + \rho e^{i\theta}$, $dz = i\rho e^{i\theta}d\theta$. Then \begin{equation*}
    \frac{1}{2\pi i}\frac{f(z)}{(z-z_0)^{m+1}}dz = \frac{f(z_0+\rho e^{i\theta}}{\rho^me^{im\theta}}\frac{d\theta}{2\pi}
\end{equation*}
and we obtain \begin{equation*}
    f^{(m)}(z_0) = \frac{m!}{\rho^m}\int_0^{2\pi}f(z_0+\rho e^{i\theta})e^{-im\theta}\frac{d\theta}{2\pi}
\end{equation*}
This gives the estimate \begin{equation*}
    \left|f^{(m)}(z_0)\right| \leq \frac{m!}{\rho^m}\int_{0}^{2\pi}|f(z_0+\rho e^{i\theta})|\frac{d\theta}{2\pi}
\end{equation*}
which then leads to: 

\begin{thm}[Cauchy Estimates]
    Suppose $f(z)$ is analytic for $|z-z_0| \leq \rho$. If $|f(z)| \leq M$ for $|z - z_0| = \rho$, then \begin{equation*}
        \left|f^{(m)}(z_0)\right| \leq \frac{m!}{\rho^m}M, m \geq 0 
    \end{equation*}
\end{thm}

This comes in part by the ML Theorem and the Maximum Theorem.

\begin{namthm}[Liouville's Theorem]
    Let $f(z)$ be an analytic function on the complex plane. If $f(z)$ is bounded, then $f(z)$ is constant.
\end{namthm}
\begin{proof}
    Assume $f(z)$ and $f'(z)$ are continuous on $\C$ and $|f(z)| \leq M$ for all $z \in \C$. Let us consider the disk of radius $R$ centered at $z_0$. From Cauchy's Estimate with $m = 1$ we obtain \begin{equation*}
        |f'(z_0)| \leq \frac{M}{R}
    \end{equation*}
    Observe, as $R$ goes to infinity $|f'(z_0)|$ goes to $0$. so $f'(z_0) = 0$. But $z_0$ was an arbitrary point in $\C$, hence $f'(z) = 0$ for all $z \in \C$ and as $\C$ is a connected domain we find $f(z) = c$ for some $c \in \C$, and for all $z \in \C$.
\end{proof}

\begin{defn}
    We define an \Emph{entire function} to be a function that is analytic on the entire complex plane.
\end{defn}


\begin{eg}
    Suppose we want to calculate \begin{equation*}
        \oint_C\frac{\cos(z)}{z^2+4}dz
    \end{equation*}
    Note that the integrand is singular when $z= 2e^{i \pi/2}e^{i\pi k/2}$, so $z = \pm 2i$ gives the singularities. Let us rewrite \begin{equation*}
        \frac{1}{z^2+4} = \frac{a}{z+2i}+\frac{b}{z-2i} 
    \end{equation*}
    Then we have $1 = a(z-2i)+b(z+2i)$, setting $z = 2i$ we get $b = \frac{1}{4i}$ and $a = \frac{-1}{4i}$. Then, we can write the integral as \begin{equation*}
        \oint_C\frac{-\cos(z)}{4i(z+2i)}dz + \oint_C\frac{\cos(z)}{4i(z-2i)}dz
    \end{equation*}
    From Cauchy's theorem and Cauchy's Integral Formula, we have four possibilities. If neither singularity is in $C$ the whole integral is $0$, if $-2i$ is in $C$ but $2i$ is not we have $$2\pi i(-\cos(-2i))/4i = -\frac{\pi\cos(2i)}{2} = -\frac{\pi\cosh(2)}{2}$$ if $2i$ is in $C$ but $-2i$ is not we have \begin{equation*}
        2\pi i(\cos(2i))/4i = \frac{\pi\cosh(2)}{2}
    \end{equation*}
    and finally if both singularities are in $C$ the integral is $0$, since the sum of the two component integrals are $0$.
\end{eg}


%%%%%%%%%%%%%%%%%%%% Section 1.4.5
\section{Morera's Theorem}


Recall that we observed that $f(z)$ is analytic on $D$ if and only if $f(z)dz$.

\begin{namthm}[Morera's Theorem] \label{namthm:morera}
    Let $f(z)$ be a continuous function on a domain $D$. If \begin{equation*}
        \int_{\partial R}f(z)dz = 0
    \end{equation*}
    for every closed rectangle $R$ contained in $D$ with sides parallel to the coordinate axes, then $f(z)$ is analytic on $D$.
\end{namthm}
\begin{proof}
    Without loss of generality suppose $D$ is a disk with center $z_0$. Define \begin{equation*}
        F(z) = \int_{z_0}^zf(\xi)d\xi, z \in D
    \end{equation*}
    where the path of integration runs along a horizontal line and then a vertical line. We could as well define $F(z)$ using the path starting from $z_0$ along a vertical line followed by a horizontal line. By hypothesis these two paths yield the same integral. Now we differentiate $F(z)$ by hand. We have \begin{equation*}
        F(z+\delta z) - F(z) = \int_z^{z+\delta z}f(w)dw
    \end{equation*}
    where the path of integration is the path from $z$ to $z+\delta z$ following a horizontal line then a vertical line. Since we are fixing $z$, the value of $f(z)$ is constant for the integration and we obtain \begin{align*}
        F(z+\delta z)-F(z)&= f(z)\int_z^{z+\delta z}dw+\int_z^{z+\delta z}(f(w)-f(z))dw \\
        &= f(z)\delta z + \int_z^{z+\delta z}(f(w)-f(z))dw
    \end{align*}
    Now, the length of the contour from $z$ to $z+\delta z$ is at most $|2\delta z|$. If we divide by $\delta z$ and use the ML-estimate on the last integral, we obtain \begin{equation*}
        \left|\frac{F(z+\delta z)-F(z)}{\delta z} - f(z) \right| \leq 2M_{\varepsilon}
    \end{equation*}
    where $|\delta z| < \varepsilon$ and $M_{\varepsilon}$ is the maximum of $|f(w)-f(z)|$ over all of $w$ satisfying $|w-z| \leq \varepsilon$. Since $f(z)$ is continuous at $z$, $M_{\varepsilon}$ to $0$ as $\varepsilon$ goes to $0$. Consequently, $F(z)$ is complex differentiable with complex derivative $F'(z) = f(z)$. Since $f(z)$ is continuous, $F(z)$ is analytic, and since $f(z)$ is the derivative of an analytic function, $f(z)$ is analytic.
\end{proof}



%%%%%%%%%%%%%%%%%%%% Section 1.4.6
\section{Goursat's Theorem}

We defined $f(z)$ to be analytic on a domain $D$ if the complex derivative $f'(z)$ exists at each point of $D$ and further, $f'(z)$ is a continuous function of $z$. Goursat's theorem asserts that the continuity requirement is redundant:

\begin{namthm}[Goursat's Theorem]
    If $f(z)$ is a complex-valued function on a domain $D$ such that \begin{equation*}
        f'(z_0) = \lim\limits_{z\rightarrow z_0}\frac{f(z)-f(z_0)}{z-z_0}
    \end{equation*}
    exists at each point $z_0 \in D$, then $z\mapsto f'(z)$ is continuous on $D$.
\end{namthm}
\begin{proof}
    Let $R$ be a closed rectangle in $D$. We subdivide $R$ into four identical subrectangles. Since the integral f $f(z)$ around $\partial R$ is the sum of the integrals of $f(z)$ around the four subrectangles, there is at least one of the subrectangles, call it $R_1$, for which \begin{equation*}
        \left|\int_{\partial R_1}f(z)dz\right| \geq \frac{1}{4}\left|\int_{\partial R}f(z)dz\right|
    \end{equation*}
    Now subdivide $R_1$ into four equal subrectangles and repeat the process. This yields a nested sequence of integrals \begin{equation*}
        \left|\int_{\partial R_n}f(z)dz\right| \geq \frac{1}{4}\left|\int_{\partial R_{n-1}}f(z)dz\right| \geq ... \geq \frac{1}{4^n}\left|\int_{\partial R}f(z)dz\right|
    \end{equation*}
    Since the $R_n$'s are decreasing and have diameters tending to $0$, the $R_n$'s converge to some point $z_0 \in D$. Since $f(z)$ is differentiable at $z_0$, we have an estimate of the form \begin{equation*}
        \left|\frac{f(z)-f(z_0)}{z-z_0} - f'(z_0)\right| \leq \varepsilon_n, z \in R_n
    \end{equation*}
    where $\varepsilon_n\rightarrow 0$ as $n\rightarrow \infty$. Let $L$ be the length of $\partial R$. Then the length of $\partial R_n$ is $L/2^n$. For $z$ belonging to $R_n$ we have the estimate \begin{equation*}
        |f(z) - f(z_0) - f'(z_0)(z-z_0)| \leq \varepsilon_n|z-z_0| \leq 2\varepsilon_nL/2^n
    \end{equation*}
    By the ML estimate and Cauchy's theorem we have \begin{align*}
        \left|\int_{\partial R_n}f(z)dz\right| &= \left|\int_{\partial R_n}[f(z)-f(z_0)-f'(z_0)(z-z_0)]dz\right| \\
        &\leq (2\varepsilon_nL/2^n)(L/2^n) = 2L^2\varepsilon_n/4^n
    \end{align*}
    where we subtracted the zero $\int_{\partial R_n}(f(z_0)+f'(z_0)(z-z_0))dz =0$, using Cauchy's theorem. Hence, \begin{equation*}
        \left|\int_{\partial R}f(z)dz\right| \leq 4^n\left|\int_{\partial R_n}f(z)dz\right| \leq 2L^2\varepsilon_n
    \end{equation*}
    Since $\varepsilon_n\rightarrow 0$ as $n\rightarrow \infty$, we must have \begin{equation*}
        \int_{\partial R}f(z)dz = 0
    \end{equation*}
    By Morera's Theorem (\ref{namthm:Morera}), $f(z)$ is analytic.
\end{proof}



%%%%%%%%%%%%%%%%%%%% Section 1.4.7
\section{Pompeiu's Formulas}


Using the fact that for $z = x+iy$, $x = (z+\overline{z})/2$ and $y = -i(z-\overline{z})/2$, we define two particular complex differential operators as follows: \begin{align*}
    \frac{\partial}{\partial z} &= \frac{\partial x}{\partial z}\frac{\partial}{\partial x}+\frac{\partial y}{\partial z}\frac{\partial}{\partial y} = \frac{1}{2}\left[\frac{\partial}{\partial x}-i\frac{\partial}{\partial y}\right] \\
    \frac{\partial}{\partial \overline{z}} &= \frac{\partial x}{\partial \overline{z}}\frac{\partial}{\partial x}+\frac{\partial y}{\partial \overline{z}}\frac{\partial}{\partial y} = \frac{1}{2}\left[\frac{\partial}{\partial x}+i\frac{\partial}{\partial y}\right]
\end{align*}
We may think of $\frac{\partial f}{\partial z}$ as an average of the derivatives of $f$ in the $x$ and $iy$ directions: \begin{equation*}
    \frac{\partial f}{\partial z} = \frac{1}{2}\left[\frac{\partial f}{\partial x}+\frac{\partial f}{\partial(iy)}\right]
\end{equation*}
Recall that from our work with Cauchy-Riemann equations we have the formulas $f'(z) = \frac{\partial f}{\partial x}$ and $f'(z) = -i\frac{\partial f}{\partial y} = \frac{\partial f}{\partial(iy)}$ provided that $f$ is analytic, so \begin{equation*}
    f'(z) = \frac{\partial f}{\partial z}
\end{equation*}
Writing $f = u+iv$ we observe that \begin{equation*}
    \frac{\partial f}{\partial \overline{z}} = \frac{1}{2}\left[u_x -v_y\right]+\frac{i}{2}\left[u_y+v_x\right]
\end{equation*}
which when equated to zero is equivalent to the Cauchy Riemann equations \begin{equation*}
    \frac{\partial f}{\partial \overline{z}} = 0
\end{equation*}
Some intuition for this is that holomorphic functions are those of $z$-alone. For example, $f(z,\overline{z}) = z^3$, holomorphic, but $f(z,\overline{z}) = z\overline{z}$ is not holomorphic. 




%%%%%%%%%%%%%%%%%%%%%% Chapter 1.5
\chapter{Power Series}


%%%%%%%%%%%%%%%%%%%% Section 1.5.1
\section{Infinite Series}


\begin{defn}
    A sequence in $\C$ is a function $f:\N\rightarrow \C$, which can be thought of as an ordered list of complex numbers.
\end{defn}

\begin{defn}
    Given a sequence $\{z_n\}_{n=1}^{\infty}\subseteq \C$, we define the series of complex numbers associated to this sequence to be \begin{equation*}
        \sum_{n=1}^{\infty}z_n = z_1+z_2+z_3+...
    \end{equation*}
    We say that this series converges, or that the underlying sequence is summable, if the sequence of partial sums $s_n = \sum_{i=1}^nz_i$ converges to some value in $\C$, and we write \begin{equation*}
        \lim\limits_{n\rightarrow \infty}\sum_{i=1}^nz_i = \sum_{n=1}^{\infty}z_n
    \end{equation*}
\end{defn}

\begin{prop}
    Extending the Algebraic Laws for Complex sequences we have that if $c \in \C$, $\sum a_n = A$ and $\sum b_n = B$ for sequences $(a_i),(b_i)$ in $\C$, then \begin{itemize}
        \item $\sum_{n=1}^{\infty}(a_n+b_n) = \sum_{n=1}^{\infty}a_n+\sum_{n=1}^{\infty}b_n$
        \item $\sum_{n=1}^{\infty}(ca_n) = a\sum_{n=1}^{\infty}$
    \end{itemize}
\end{prop}


\begin{thm}
    Let $(x_k),(y_k) \subseteq \R$ be real sequences, then the sequence $(x_k+iy_k)$ is summable if and only if $(x_k)$ and $(y_k)$ are summable. Moreover, in the convergent case \begin{equation*}
        \sum_{n=1}^{\infty}(x_n+iy_n) = \sum_{n=1}^{\infty}x_n+i\sum_{n=1}^{\infty}y_n
    \end{equation*}
\end{thm}


\begin{thm}[Comparison Test]
    If $(a_n),(b_n) \subseteq \R$ are real sequences, and $0 \leq a_n \leq b_n$ for all $n$, then if $\sum_{n=1}^{\infty}b_n$ converges $\sum_{n=1}^{\infty}a_n$ converges and $\sum_{n=1}^{\infty}a_n \leq \sum_{n=1}^{\infty}b_n$.
\end{thm}


\begin{thm}
    If $\sum_{n=1}^{\infty}a_n$ converges, then $a_n$ goes to $0$ as $n$ goes to infinity.
\end{thm}


\begin{prop}
    If $z \in \C$ and $|z| < 1$, then $\sum_{n=0}^{\infty}z^n = \frac{1}{1-z}$. If $z \in \C$ and $|z| \geq 1$ then $\sum_{n=0}^{\infty}z^n$ diverges.
\end{prop}
\begin{proof}
    Let $S_n$ be the $n$th partial sum, so $S_n = 1+z+...+z^{n-1}$. Then $zS_n = z+z^2+...+z^n$, so $S_n-zS_n = 1-z^n$. If $z \neq 1$, then we have $S_n = \frac{1-z^n}{1-z}$. Then \begin{equation*}
        \lim\limits_{n\rightarrow \infty}S_n = \lim\limits_{n\rightarrow \infty}\frac{1-z^n}{1-z} = \frac{1}{1-z}\lim\limits_{n\rightarrow \infty}(1-z^n)
    \end{equation*}
    This limit converges only if $z^n$ converges as $n$ goes to infinity. Note $|z^n-c| \geq |z|^n-|c|$, and for $|z| > 1$ the right hand side is unbounded and goes to infinity as $n$ goes to infinity, so $z^n$ does not converge to any value $c \in \C$ for $|z| > 1$. If $|z| < 1$, $|z|^n\rightarrow 0$, so the series converges and \begin{equation*}
        \sum_{n=0}^{\infty}z^n = \frac{1}{1-z}
    \end{equation*}
\end{proof}

\begin{defn}
    A complex series $\sum_{n=0}^{\infty}a_n$ is said to \Emph{converge absolutely} if $\sum_{n=0}^{\infty}|a_n|$ converges.
\end{defn}

\begin{thm}
    If $\sum_{n=0}^{\infty}a_n$ converges absolutely, then $\sum_{n=0}^{\infty}a_n$ converges and \begin{equation*}
        \left|\sum_{n=0}^{\infty}a_n\right| \leq \sum_{n=0}^{\infty}|a_n|
    \end{equation*}
\end{thm}


\begin{eg}
    Consider $|z| < 1$. Then \begin{equation*}
        \left|\frac{1}{1-z}\right| = \left|\sum_{n=0}^{\infty}z^n\right| \leq \sum_{n=0}^{\infty}|z|^n = \frac{1}{1-|z|}
    \end{equation*}
\end{eg}

Consequently, geometric series are absolutely convergent, given $|z| < 1$, and the bound on its convergence is \begin{equation*}
    \left|\sum_{n=0}^{\infty}z^n\right| \leq \frac{1}{1-|z|}
\end{equation*}



%%%%%%%%%%%%%%%%%%%% Section 1.5.2
\section{Sequences and Series of Functions}

\begin{defn}
    Let $\{f_n\}$ be a sequence of complex valued functions on $E \subseteq \C$. Then the sequence is said to \Emph{pointwise converge} to $f:E\rightarrow \C$ if for all $x \in E$, for all $\varepsilon > 0$, there exists $N \in \N$ such that whenever $n \geq N$, then $|f_n(x) - f(x)| < \varepsilon$.
\end{defn}

\begin{eg}
    Consider the sequence of functions $f_n(x) = x^n, 0 < x \leq 1$. It converges pointwise to the function \begin{equation*}
        f(x) = \left\{\begin{array}{lc} 0 & \text{if } 0 < x < 1 \\ 1 & \text{if } x = 1\end{array}\right.
    \end{equation*}
\end{eg}

In limit notation this example shows that in general for a pointwise convergence of a sequence of functions, $$\lim\limits_{x\rightarrow x_0}\lim\limits_{n\rightarrow \infty}f_n(x) \neq \lim\limits_{n\rightarrow \infty}\lim\limits_{x\rightarrow x_0}f_n(x)$$

\begin{eg}
    Consider the sequence of tent functions \begin{equation*}
        f_n(x) = \left\{\begin{array}{lc} n^2x & \text{if } 0 \leq x \leq 1/n \\ 2n-n^2x & \text{if } 1/n \leq x \leq 2/n \\ 0 & \text{if } 2/n \leq x \leq 1 \end{array}\right.
    \end{equation*}
    The height of the $n$th tent is $n$, and the width is $2/n$, so that the area under the integral is always $1$. Hence, $\int_0^1g_n(x)dx = 1$ for all $n$. But, the sequence converges pointwise to $0$ on the unit interval, and $\int_0^10dx = 0$.
\end{eg}

Again in limit notation this example shows that in general for a pointwise convergence of a sequence of function, $$\lim\limits_{n\rightarrow \infty}\int_a^bf_m(x)dx \neq \int_a^b\lim\limits_{n\rightarrow \infty}f_n(x)dx$$

\begin{defn}
    Let $\{f_n\}$ be a sequence of complex valued functions on $E \subseteq \C$. We say that $\{f_n\}$ \Emph{converges uniformly} on $E$ to $f$ on $E$ if for all $\varepsilon > 0$, there exists $N \in \N$ such that if $n \geq N$, then $|f_n(x) - f(x)| < \varepsilon$ for all $x \in E$.
\end{defn}

\begin{defn}
    Let $f:E\rightarrow \C$ be a complex valued function on $E$. Then we define the \Emph{uniform norm} of $f$ to be the supremum \begin{equation*}
        ||f||_{\infty} = \sup\{|f(x)|:x \in E\} = \sup\limits_{x \in E}|f(x)|
    \end{equation*}
\end{defn}

Using this concept, we can define uniform convergence in an equivalent manner. A sequence of functions $\{f_n\}$ converges uniformly to $f$ on $E \subseteq \C$ if and only if for all $\varepsilon > 0$, there exists $N \in \N$ such that if $n \geq N$, $||f_n - f||_{\infty} < \varepsilon$. In other words, the sequence converges uniformly if and only if $$\lim\limits_{n\rightarrow \infty}||f_n-f||_{\infty} = 0$$

\begin{thm}
    Let $\{f_n\}$ be a sequence of complex-valued functions defined on a subset $E$ of the complex plane. If each $f_n$ is continuous on $E$, and if $\{f_n\}$ converges uniformly to $f$ on $E$, then $f$ is continuous on $E$.
\end{thm}
\begin{proof}
    Suppose that $\{f_n\}$ is a sequence of continuous. complex-valued functions defined on a subset $E$ of the complex plane which converge uniformly to $f$ on $E$. Let $x \in E$, and fix $\varepsilon > 0$. First, observe that \begin{equation*}
        |f(x)-f(y)| \leq |f(x)-f_n(x)|+|f_n(x)-f_n(y)|+|f_n(y)-f(y)|
    \end{equation*}
    for all $y \in E$. First, there exists $N \in \N$ such that if $n \geq N$, $||f_n - f||_{\infty} < \varepsilon/3$. Moreover, since $f_N$ is continuous, there exists $\delta > 0$ such that if $|z - x| < \delta$, $|f_N(z) - f_N(x)| < \varepsilon/3$. It follows that for $z \in E$ such that $|z - x| < \delta$, \begin{align*}
        |f(x) - f(z)| &\leq |f(x) - f_N(x)| + |f_N(x)-f_N(z)| + |f_N(z) - f(z)| \\
        &< \varepsilon/3 + \varepsilon/3 + \varepsilon/3 = \varepsilon
    \end{align*}
    Hence, $f$ is continuous on $E$, as claimed.
\end{proof}


\begin{thm}
    Let $\gamma$ be a piecewise smooth curve in the complex plane. If $\{f_n\}$ is a sequence of continuous complex-valued functions on $\gamma$, and if $\{f_n\}$ converges uniformly to $f$ on $\gamma$, then $\int_{\gamma}f_n(z)dz$ converges to $\int_{\gamma}f(z)dz$.
\end{thm}   
\begin{proof}
    By our previous result $f$ is continuous on $\gamma$, and hence the integral exists. Let $L = \int_{\gamma}|dz|$ be the length of the curve. Then, if $\varepsilon > 0$ is fixed, there exists $N \in \N$ such that if $n \geq N$, $||f_n - f||_{\infty} <\varepsilon/L$. Then it follows by the ML theorem that \begin{equation*}
        \left|\int_{\gamma}f_n(z)dz -\int_{\gamma}f(z)dz\right| = \left|\int_{\gamma}(f_n(z)-f(z))dz\right| \leq ||f_n-f||_{\infty}L < \varepsilon L/L = \varepsilon
    \end{equation*}
    Hence, the integrals of the terms converges to the integral of the limit.
\end{proof}



\begin{eg}
    Consider $f_n(x) = x^n$. Observe that $||f_n - f||_{\infty} = ||f_n||_{\infty} = 1$, so we see that the uniform norm of the difference function does not converge to $0$ as $n$ goes to infinity, and hence the sequence does not converge uniformly to $f$ on $[0,1]$.
\end{eg}

\begin{eg}
    Suppose $f_n$ is the tent sequence. Then $||f_n - f||_{\infty} = ||f_n||_{\infty} = n$, which does not converge to $0$ as $n$ goes to infinity, and in fact blows up to infinity itself. Thus, the sequence does not converge uniformly on $[0,1]$.
\end{eg}



\begin{namthm}[Weierstrass M-Test]
    Suppose $M_k \geq 0$ and the sequence $\{M_k\}$ is summable. If $\{f_k\}$ is a sequence of complex valued functions on a set $E \subseteq \C$ such that $||f_k||_{\infty} \leq M_k$, then $\{f_k\}$ converges uniformly on $E$.
\end{namthm}
\begin{proof}
    For each fixed $x \in E$ we have that the series $\{f_k(x)\}$ is Cauchy as it is bounded termwise by a convergent series. Then, by the completeness of $\C$, there exists $f(x) \in \C$ such that $\sum_{k=0}^{\infty}f_k(x) = f(x)$. Thus, as this holds for all $x \in E$, the series converges pointwise to $f$. Since the series of $M_k$ converges, for $\varepsilon > 0$ fixed there exists $N \in \N$ such that the tail past $N$ is less than $\varepsilon$. Then for all $m > n \geq N \in \N$, we have that \begin{equation*}
        \left|\left|\sum_{i=n+1}^mf_i\right|\right|_{\infty} \leq \sum_{i=n+1}^mM_k < \varepsilon
    \end{equation*}
    so $\sum_{n=0}^{\infty}f_n$ is uniformly cauchy, so converges uniformly to $f$.
\end{proof}


\begin{thm}
    If $\{f_n\}$ is a sequence of analytic functions on a domain $D$ that converges uniformly to $f$ on $D$, then $f(z)$ is analytic on $D$.
\end{thm}
\begin{proof}
    From our previous result we know that $f$ is continuous on $D$. Let $R \subseteq D$ be a closed rectangle with edges oriented with respect to the axes. By Cauchy's Theorem we have that $\int_{\partial R}f_n(z)dz = 0$ for all $n \in \N\cup \{0\}$. Then, by our second result for uniform sequences, \begin{equation*}
        \int_{\partial R}f(z)dz = \lim\limits_{n\rightarrow \infty}\int_{\partial R}f_n(z)dz = \lim\limits_{n\rightarrow \infty}0 = 0
    \end{equation*}
    Thus, by Morera's Theorem $f(z)$ is analytic on $D$, as required.
\end{proof}


\begin{thm}
    Let $z_0 \in \C$ and suppose that $\{f_n\}$ is analytic for $|z-z_0| \leq R$, and that the sequence converges uniformly to $f(z)$ for $|z-z_0| \leq R$. Then for each $r < R$ and each $m\geq 1$, the sequence of $m$th derivatives $\left\{f^{(m)}_n(z)\right\}$, which exists by a corollary to the Cauchy Integral Formula, converges uniformly to $f^{(m)}(z)$ for $|z-z_0| \leq r$.
\end{thm}
\begin{proof}
    Since the $f_n$ converge uniformly we can pick $\varepsilon_n$ such that $|f_n(z) - f(z)| < \varepsilon_n$ for $|z - z_0| < R$, where $\varepsilon_n\rightarrow 0$ is a decreasing sequence. Fix $s$ such that $r < s < R$. Applying the Cauchy Integral Formula to the $m$th derviative of $f_n(z)-f(z)$ on $|z-z_0|\leq s$: \begin{equation*}
        f_n^{(m)}(z) - f^{(m)}(z) = \frac{m!}{2\pi i}\oint_{|z-z_0|=s}\frac{f_n(w)- f(w)}{(w-z)^{m+1}}dw
    \end{equation*}
    for $|z-z_0| \leq r$. Consider if $|w-z_0| = s$ and $|z-z_0| \leq r$, then \begin{equation*}
        |w-z| = |w-z_0+z_0-z| \geq |w-z_0|-|z-z_0| = s-|z-z_0| \geq s-r 
    \end{equation*}
    and so \begin{equation*}
        \left|\frac{f_n(w)-f(w)}{(w-z)^{m+1}}\right|\leq \frac{\varepsilon_n}{(s-r)^{m+1}}
    \end{equation*}
    From the ML-estimate it follows that \begin{equation*}
        \left|f_n^{(m)}(z) - f^{(m)}(z) \right|= \left|\frac{m!}{2\pi i}\oint_{|z-z_0|=s}\frac{f_n(w)- f(w)}{(w-z)^{m+1}}dw\right| \leq \frac{m!}{2\pi}\frac{\varepsilon_n}{(s-r)^{m+1}}2\pi s = \rho_n
    \end{equation*}
    so $\rho_n \frac{m!\varepsilon_ns}{(s-r)^{m+1}}$. But except for $\varepsilon_n$ everythin is a constant, so $\rho_n\rightarrow 0$ as $n\rightarrow \infty$, and we obtain uniform convergence of the $m$th derivatives.
\end{proof}

\begin{defn}
    A sequence $\{f_n\}$ of holomorphic functions on a domain $D$ \Emph{converges normally} to an analytic function $f$ on $D$ if it converges uniformly to $f$ on each closed disk contained in $D$.
\end{defn}


\begin{thm}
    Suppose that $\{f_n\}$ is a sequence of holomorphic functions on a domain $D$ that converges normally on $D$ to the holomorphic function $f$. Then for each $m \geq 1$, the sequence of $m$th derivatives, $\{f_n^{(m)}\}$ converges normally to $f^{(m)}$ on $D$.
\end{thm}





%%%%%%%%%%%%%%%%%%%% Section 1.5.3
\section{Power Series}

\begin{defn}
    A \Emph{power series} centered at $z_0 \in \C$ is a series of the form $\sum_{n=0}^{\infty}a_n(z-z_0)^n$. Setting $w = z-z_0$ we can always reduce to the case where $z_0 = 0$.
\end{defn}


\begin{namthm}[Abel's Convergence Lemma]
    Suppose for the power series $\sum_{n=0}^{\infty}a_nz^n$ there are positive real numbers $s$ and $M$ such that $|a_n|s^n \leq M$ for all $n$. Then this power series is normally convergent in $\{z \in \C\vert |z|<s\}$. 
\end{namthm}
\begin{proof}
    Consider $r \in \R$ with $0 < r < s$. Then observe that for all $z \in \{z\in\C\vert |z| \leq r\}$, \begin{equation*}
        |a_nz^n| \leq |a_n|r^n = |a_n|s^n\left(\frac{r}{s}\right)^n \leq M\left(\frac{r}{s}\right)^n
    \end{equation*}
    But, $r/s < 1$, so $\sum_{n=0}^{\infty}M\left(\frac{r}{s}\right)^n$ is a geomotric series with ratio of magnitude less than $1$, and hence converges. Thus, setting $M_n = M\frac{r^n}{s^n}$, we have by the Weierstrass M-test that $\sum_{n=0}^{\infty}a_nz^n$ is normally convergent on $\{z \in \C\vert|z| <s\}$, in the sense that the series converges absolutely uniformly on every closed disk contained in $\{z \in \C\vert|z| < s\}$.
\end{proof}

\begin{cor}
    If the series $\sum_{n=0}^{\infty}a_nz^n$ converges at $z_0 \neq 0$, then it converges normally in the open disk $|z| < |z_0|$.
\end{cor}
\begin{proof}
    As the sum $\sum_{n=0}^{\infty}a_nz_0^n$ converges, $a_nz_0^n$ goes to zero as $n$ goes to infinity, and similarly $|a_n||z_0|^n$ goes to zero as $n$ goes to infinity. But, $|a_n||z_0|^n$ is a sequence of positive terms which converges, so there exists $M \in \R$ such that $|a_n||z_0|^n \leq M$ for all $n \in \N$. But then by Abel's Convergence Lemma, the power series is normally convergent in $\{z \in \C\vert |z| < |z_0|\}$.
\end{proof}

\begin{defn}
    A power series $\sum_{n=0}^{\infty}a_n(z-z_0)^n$ has \Emph{radius of covergence} $R$ is the series converges for $|z-z_0| < R$ but diverges for $|z-z_0| > R$. In the case the series converges everywhere we say $R = \infty$ and in the case the series only converges at $z = z_0$ we say $R = 0$.
\end{defn}

\begin{thm}
    Let $\sum_{n=0}^{\infty}a_n(z-z_0)^n$ be a power series. Then there is $R$, $0 \leq R \leq \infty$ such that $\sum_{n=0}^{\infty}a_n(z-z_0)^n$ converges normally on $\{z \in \C\vert |z-z_0| < R\}$, and $\sum_{n=0}^{\infty}a_n(z-z_0)^n$ does not converge if $|z-z_0| > R$.
\end{thm}
\begin{proof}
    Let us define \begin{equation*}
        R = \sup\{t \in [0,\infty):|a_n|t^n\text{ is a bounded sequence}\}
    \end{equation*}
    If $R = 0$, then the series only converges if $z = z_0$, since boundedness is necessary for convergence. Suppose $R > 0$, and let $0 < r < R$. By construction of $R$ the sequence $|a_n|r^n$ is bounded, and by Abel's convergence lemma $\sum_{n=0}^{\infty}a_n(z-z_0)^n$ converges normally in $\{z \in \C:|z-z_0| < r\}$. However, $\{z\in \C:|z-z_0| < R\}$ is found by the union of open $r$ disks, and thus we find normal convergence on the open $R$ disk centered at $z_0$. 
\end{proof}


\begin{eg}
    The series $\sum_{n=0}^{\infty}z^n$ is the geometric series. We have shown it converges if and only if $|z| < 1$, which shows $R = 1$.
\end{eg}

\begin{eg}
    The series $\sum_{n=1}^{\infty}\frac{z^n}{n^4}$ has Weierstrass M, $M_k = 1/k^4$ for $|z| < 1$. Recall, by the previous test, with $p = 4 > 1$, the series $\sum_{n=1}^{\infty}\frac{1}{n^4}$ converges. Thus, the given series in $z$ is normally convergent on $|z| <1$.
\end{eg}

\begin{eg}
    The series $\sum_{n=0}^{\infty}\frac{(-1)^n}{4^n}(z-i)^{2n}$, which is geometric. Then this converges when $\left|\frac{-1(z-i)^2}{4}\right| < 1$, which is to say $|z-i| < 2$. So we have radius of convergence $R = 2$ with center $z_0 = i$.
\end{eg}

\begin{eg}
    The series $\sum_{n=0}^{\infty}n^nz^n$ has $R = 0$ since it diverges for all $z \neq 0$ by the divergence test.
\end{eg}
    

\begin{eg}
    The series $\sum_{n=1}^{\infty}n^{-n}z^n$ has $R = \infty$. This can be shown by a theorem to follow.
\end{eg}

\begin{thm}
    Let $\sum_{n=0}^{\infty}a_n(z-z_0)^n$ be a power series with radius of convergence $R > 0$. Then the function \begin{equation*}
        f(z) = \sum_{n=0}^{\infty}a_n(z-z_0)^n
    \end{equation*}
    for $|z-z_0| < R$ is holomorphic. The derivatives of $f(z)$ are obtained by term-by-term differentiation \begin{equation*}
        f^{(m)}(z) = \sum_{n=m}^{\infty}\frac{n!}{(n-m)!}a_n(z-z_0)^{n-m}
    \end{equation*}
    The coefficients of are then given by \begin{equation*}
        a_n = \frac{f^{(m)}(z_0)}{m!}
    \end{equation*}
    $m \geq 0$.
\end{thm}
\begin{proof}
    By the previous results the series of normally convergent on $\{z \in \C:|z-z_0| < R\}$. Then, note that each term in the series is holomorphic, so as the series converges normally, and hence uniformly, on the disk, $f(z)$ is holomorphic on the disk as well. Furthermore, $f^{(m)}$ is holomorphic on the disk, and the series $f_n^{(m)} = a_n\frac{n!}{(n-m)!}(z-z_0)^{n-m}$ for $n \geq m$ converges uniformly to $f^{(m)}$, extending our result for sequences from before to the sequence of partial sums using the linearity of the derivative. 
\end{proof}

\begin{eg}
    Consider the series \begin{equation*}
        \sum_{n=0}^{\infty}z^{3k+4} = z^4\sum_{n=0}^{\infty}z^{3k} = z^4\frac{1}{1-z^3} = \frac{z^4}{1-z^3}
    \end{equation*}
    if $|z| < 1$. Then, the series represents $f(z) = \frac{z^4}{1-z^3}$ on the open disk $|z| < 1$ (and the series is normally convergent on this disk).
\end{eg}


\begin{eg}
    Consider the series \begin{align*}
        \sum_{n=0}^{\infty}(z^{2n} + (z-1)^{2n}) &= \sum_{n=0}^{\infty}z^{2n} + \sum_{n=0}^{\infty}(z-1)^{2n} \\
        &= \frac{1}{1-z^2} + \frac{1}{1-(1-z)^2}
    \end{align*}
    provided that $|z| < 1$ and $|z-1| < 1$.
\end{eg}

\begin{eg}
    Consider $f(z) = \text{Log}(1-z) = \ln|1-z| + i\text{Arg}(1-z)$ on the slit plane $\C\backslash[1,\infty)$. Then $f'(z) = \frac{1}{1-z}$. We have $\frac{1}{1-z} = \sum_{n=0}^{\infty}z^n$ for $|z| < 1$. Integrating we have \begin{equation*}
        f(z) = C+\sum_{n=0}^{\infty}\frac{z^{n+1}}{n+1}
    \end{equation*}
    Taking $z = 0$, we see that $C = \text{Log}(1-0) = 0$, so \begin{equation*}
        \text{Log}(1-z) = \sum_{n=0}^{\infty}\frac{z^{n+1}}{n+1}
    \end{equation*}
    for $|z| < 1$.
\end{eg}


\begin{thm}[Ratio Test]
    If $|a_n/a_{n+1}|$ has a limit as $n\rightarrow \infty$, either finite or $+\infty$, then the limit is the radius of convergence $R$ of $\sum_{n=0}^{\infty}a_nz^n$ \begin{equation*}
        R = \lim\limits_{n\rightarrow \infty}\left|\frac{a_n}{a_{n+1}}\right|
    \end{equation*}
\end{thm}

\begin{thm}[Root Test]
    If $\sqrt[n]{|a_n|}$ has a limit as $n\rightarrow \infty$, either finite or $+\infty$, then the radius of convergence of $\sum_{n=0}^{\infty}a_nz^n$ is given by: \begin{equation*}
        R = \frac{1}{\lim\limits_{n\rightarrow \infty}\sqrt[n]{|a_n|}}
    \end{equation*}
\end{thm}


\begin{thm}
    Suppose that $f(z)$ is holomorphic for $|z - z_0| < \rho$. Then $f(z)$ is represented by the power series \begin{equation*}
        f(z) = \sum_{n=0}^{\infty}a_n(z-z_0)^n,\;\;\;\;\;\;\; |z-z_0| < \rho
    \end{equation*}
    where \begin{equation*}
        a_n = \frac{f^{(n)}(z_0)}{n!},\;\;\;\;n \geq 0
    \end{equation*}
    and where the power series has radius of convergence $R \geq \rho$. For any fixed $r$, $0 < r < \rho$, we have \begin{equation*}
        a_n = \frac{1}{2\pi i}\oint_{|w-z_0| = r}\frac{f(w)}{(w-z_0)^{n+1}}dw,\;\;\;\;\;n\geq 0
    \end{equation*}
    Further, if $|f(z)| \leq M$ for $|z-z_0| = r$, then \begin{equation*}
        |a_n| \leq \frac{M}{r^n},\;\;\;\;\; n \geq 0
    \end{equation*}
\end{thm}
\begin{proof}
    (To be completed)
\end{proof}




%%%%%%%%%%%%%%%%%%%%%% Chapter 1.6
\chapter{Laurent Series and Isolated Singularities}



%%%%%%%%%%%%%%%%%%%%%% Chapter 1.7
\chapter{The Residue Calculus}



%%%%%%%%%%%%%%%%%%%%%%%%%%%%%%%%%%%%% Part 2.
\part{Part 2}

%%%%%%%%%%%%%%%%%%%%%% Chapter 2.1
\chapter{The Logarithmic Integral}




%%%%%%%%%%%%%%%%%%%%%% Chapter 2.2
\chapter{Conformal Mapping}





%%%%%%%%%%%%%%%%%%%%%%%%%%%%%%%%%%%%% Part 3.
\part{Part 3}

%%%%%%%%%%%%%%%%%%%%%% Chapter 3.1
\chapter{Approximation Theorems}




%%%%%%%%%%%%%%%%%%%%%% Chapter 3.2
\chapter{Special Functions}








%%%%%%%%%%%%%%%%%%%%%% - Appendices
\begin{appendices}


\end{appendices}


\end{document}


%%%%%% END %%%%%%%%%%%%%
