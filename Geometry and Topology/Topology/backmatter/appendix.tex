%%%%%%%%%%%%%%%%%%%%% appendix.tex %%%%%%%%%%%%%%%%%%%%%%%%%%%%%%%%%
%
% sample appendix
%
% Use this file as a template for your own input.
%
%%%%%%%%%%%%%%%%%%%%%%%% Springer-Verlag %%%%%%%%%%%%%%%%%%%%%%%%%%

\appendix
\motto{All's well that ends well}
\chapter{Appendix A: Set Theory}
\section{Relations}
    
    \begin{definition}[Relation]
        A \Emph{relation} on a set $A$ is a subset $C$ of the cartesian product $A \times A$. If $C$ is a relation on $A$ we write $xCy$ to denote $(x,y) \in C$.
    \end{definition}
    

    \begin{definition}[Equivalence Relation]
        An \Emph{equivalence relation} on a set $A$ is a relation $C$ on $A$ having the following three properties: \begin{enumerate}
            \item (\Emph{Reflexivity}) $xCx$ for all $x \in A$
            \item (\Emph{Symmetry}) For $x,y \in A$, if $xCy$, then $yCx$
            \item (\Emph{Transitivity}) For $x,y,z \in A$, if $xCy$ and $yCz$, then $xCz$
        \end{enumerate}
    \end{definition}


    \begin{definition}[Equivalence Class]
        Given an equivalence relation $\sim$ on a set $A$ and an element $x$ of $A$, we define the \Emph{equivalence class} determined by $x$ to be \begin{equation}
            [x]_{\sim} := \{y \in A\vert y\sim x\}
        \end{equation}
    \end{definition}


    \begin{definition}[Partition]
        A \Emph{partition} of a set $A$ is a collection of disjoint nonempty subsets of $A$ whose union is all of $A$.
    \end{definition}



    \begin{proposition}
        Given an equivalence relation $\sim$ on a set $A$, the distinct equivalence classes of $\sim$ form a partition of $A$.
    \end{proposition}
    \begin{proof}
        (Left to the reader)
    \end{proof}

    \subsection{Order Relations}

    \begin{definition}{Order Relation}
        A relation $C$ on a set $A$ is called an \Emph{order relation} (or \Emph{simple order}, or a \Emph{linear order}) if it has the following properties: \begin{enumerate}
            \item (\Emph{Comparability}) For every $x,y \in A$ for which $x \neq y$, either $xCy$ or $yCx$
            \item (\Emph{Nonreflexivity}) For no $x \in A$ does the relation $xCx$ hold
            \item (\Emph{Transitivity}) If $xCy$ and $yCz$, then $xCz$
        \end{enumerate}
    \end{definition}

    \begin{definition}[Strict Partial Order]
        A relation $C$ on a set $A$ is called a \Emph{strict partial order relation} if it satisfies properties $2.$ and $3.$ of an order relation.
    \end{definition}

    
    \begin{definition}[Open Interval]
        If $X$ is a set and $<$ is an order relation on $X$, and if $a < b$, we use the notation $(a,b)$ to denoted the set \begin{equation*}
            \{x \in X\vert a < x < b\}
        \end{equation*}
        it is called an \Emph{open interval} in $X$. If this set is empty, we call $a$ the \Emph{immediate predecessor} of $b$ and we call $b$ the \Emph{immediate successor} of $a$.
    \end{definition}

    \begin{definition}[Order Isomorphism]
        Suppose that $A$ and $B$ are two sets with order relations $<_A$ and $<_B$ respectively. We say that $A$ and $B$ have the same \Emph{order type} if there is a bijective correspondence between them that preserves order; that is, if there exists a bijective function $f:A\rightarrow B$ such that \begin{equation*}
            a_1 <_A a_2 \implies f(a_1) <_B f(a_2)
        \end{equation*}
    \end{definition}

    \begin{definition}
        Suppose that $A$ and $B$ are two sets with order relations $<_A$ and $<_B$ respectively. Define an order relation $<$ on $A \times B$ by defining \begin{equation*}
            a_1\times b_1 < a_2\times b_2
        \end{equation*}
        if $a_1 <_A a_2$, or if $a_1 = a_2$ and $b_1 <_B b_2$. It is called the \Emph{dictionary order relation} on $A\times B$.
    \end{definition}

    
    For the next definitions, suppose that $A$ is a set ordered by the relation $<$. Let $A_0$ be a subset of $A$.


    \begin{definition}[Minima and Maxima]
        We say that the element $b \in A_0$ is the \Emph{maximum or largest element} of $A_0$ if for all $x \in A_0$, $x \leq b$. Similarly, we say that $a \in A_0$ is the \Emph{minimum or smallest element} of $A_0$ if for all $x \in A_0$, $a \leq x$. We use $\leq$ to denote $<$ or $=$.
    \end{definition}


    \begin{definition}[Supremum and Upper Bounds]
        We say that the subset $A_0$ of $A$ is \Emph{bounded above} if there is an element $b$ of $A$ such that $x \leq b$ for every $x \in A_0$; the element $b$ is called an \Emph{upper bound} for $A_0$. If the set of all upper bounds for $A_0$ has a smallest element, that element is called the \Emph{least upper bound}, or \Emph{supremum}, of $A_0$, and it is denoted by $\sup A_0$. The supremum may or may not belong to $A_0$.
    \end{definition}


    \begin{definition}[Infimum and Lower Bounds]
        We say that $A_0$ is \Emph{bounded below} if there is an element $a$ of $A$ such that $a \leq x$ for all $x \in A_0$; the element $a$ is called a \Emph{lower bound} for $A_0$. If the set of all lower bounds for $A_0$ has a largest element, that element is called the \Emph{greatest lower bound}, or \Emph{infimum}, of $A_0$, and it is denoted by $\inf A_0$. Again, the infimum may or may not belong to $A_0$.
    \end{definition}


    \begin{definition}[Least Upper Bound and Greatest Lower Bound Properties]
        An ordered set $A$ is said to have the \Emph{least upper bound property} if every nonempty subset $A_0$ of $A$ that is bounded above has a least upper bound. Analogously, the set $A$ is said to have the \Emph{greatest lower bound property} if every nonempty subset $A_0$ of $A$ that is bounded below has a greatest lower bound. 

        These two properties are in fact equivalent.
    \end{definition}


    \section{Properties of the Integers and Reals}


    \begin{definition}[Binary Operation]
        A \Emph{binary operation} on a set $A$ is a function $f:A\times A\rightarrow A$.
    \end{definition}


    \begin{construction}[The Reals]
        We assume there exists a set $\R$, called the set of \Emph{real numbers}, two binary operations $+$ and $\cdot$ on $\R$, called addition and multiplication operations, respectively, and an order relation $<$ on $\R$, such that the following properties hold: \begin{enumerate}
            \item (Associativity) For all $x,y,z \in \R$:\begin{align}
                    (x+y)+z &= x+(y+z) \\
                    (x\cdot y)\cdot z &= x\cdot (y\cdot z) 
                \end{align}
            \item (Commutivity) For all $x,y \in \R$:\begin{align}
                x+y &= y+x \\
                x\cdot y &= y \cdot x
                \end{align}
            \item (Additive Identity) There exists a unique element of $\R$ called \Emph{zero}, denoted by $0$, such that $x+0 = x$ for all $x \in \R$.
            \item (Multiplicative Identity) There exists a unique element of $\R$ called \Emph{one}, different from $0$ and denoted by $1$, such that $x\cdot 1 = x$ for all $x \in \R$.
            \item (Additive Inverses) For each $x \in\R$, $\exists y \in \R$ such that $x+y = 0$
            \item (Multiplicative Inverses) For each $x \in \R$, with $x \neq 0$, $\exists y \in \R$ such that $x\cdot y = 1$.
            \item (Distributivity) For all $x,y,z \in \R$, \begin{equation}
                    x\cdot (y+z) = (x\cdot y) + (x\cdot z)
                \end{equation}
            \item (Order Closure) If $x > y$, then $x+z > y+z$
            \item If $x > y$ and $z > 0$, then $x\cdot z > y\cdot z$
            \item (Least Upper Bound Property) The order relation $<$ has the least upper bound property.
            \item If $x < y$, there exists an element $z$ such that $x < z$ and $z < y$.
        \end{enumerate}
        Under axioms 1-6 $\R$ is known as a \Emph{field}. With the addition of axioms 7 and 8, $\R$ becomes an \Emph{ordered field}. Moreover, any set with an order relation satisfying 9 and 10 is called a \Emph{linear continuum}.
    \end{construction}

    \begin{definition}
        A subset $A$ of the real numbers is said to be \Emph{inductive} if it contains the number $1$, and if for every $x \in A$, the number $x+1$ is also in $A$. Let $\mathscr{A}$ be the collection of all inductive subsets of $\R$. Then the set $\Z_{+}$ of \Emph{positive integers} is defined by the equation \begin{equation}
            \Z_+ := \bigcap\limits_{A\in\mathscr{A}}A
        \end{equation}
        Then, $\Z_+$ is inductive, and if $A$ is an inudctive set of positive integers, then $A = \Z_+$.
    \end{definition}

    \begin{definition}[Integers]
        We define the set $\Z$ of \Emph{integers} to be the set consisting of the positive integers $\Z_+$, the number $0$, and the negatives of the elements of $\Z_+$. The set $\Q$ of quotients of integers is called the set of \Emph{rational numbers}.
    \end{definition}

    \begin{theorem}[Well-ordering Property]
        Every nonempty subset of $\Z_+$ has a smallest element.
    \end{theorem}
    \begin{proof}
        We first prove that for each $n \in \Z_+$, the following statement holds: Every nonempty subset of $\{1,...,n\}$ has a smallest element.

        Let $A$ be the set of all positive integers $n$ for which this statement holds. Then $A$ contains $1$, since if $n = 1$, the only nonempty subset of $\{1,..,n\}$ is the set $\{1\}$ itself. Then, supposing $A$ contains some $n \geq 1$, we show that it contains $n+1$. So let $C$ be a nonempty subset of $\{1,...,n+1\}$. If consists of the single element $n+1$, then that element is the smallest element of $C$. Otherwise, consider the set $C\cap\{1,...,n\}$, which is nonempty. Because $n \in A$, this set has a smallest element, which will automatically be the smallest element of $C$ also. Thus $A$ is inductive, so we conclude that $A = \Z_+$; hence the statement is true for all $n \in \Z_+$.

        To prove the theorem suppsoe $D$ is a nonempty subset of $\Z_+$. Choose an element $n$ of $D$. Then the set $A = D\cap\{1,...,n\}$ is nonempty, so that $A$ has a smallest element $k$. The element $k$ is automatically the smallest element of $D$ as well.
    \end{proof}

    \begin{theorem}[Strong Induction Principle]
        Let $A$ be a set of positive integers. Suppose that for each positive integer $n$, the statement $S_n \subseteq A$ implies the statement $n \in A$, where $S_n = \{1,...,n-1\}$. Then $A = \Z_+$.
    \end{theorem}
    \begin{proof}
        If $A$ does not equal all of $\Z_+$, let $n$ be the smallest positive integer that is not in $A$. Then every positive integer less than $n$ is in $A$ so that $S_n \subseteq A$. Our hypothesis implies that $n \in A$, contrary to assumption, so we must have that $A = \Z_+$.
    \end{proof}


    \section{Cartesian Products}

    \begin{definition}
        Let $\mathscr{A}$ be a nonempty collection of sets. An \Emph{indexing function} for $\mathscr{A}$ is a surjective function $f$ from some set $J$, called the \Emph{index set}, to $\mathscr{A}$. The collection $\mathscr{A}$, together with the indexing function $f$, is called an \Emph{indexed family of sets}. Given $\alpha \in J$, we shall denote the set $f(\alpha)$ by the symbol $A_{\alpha}$. And, we shall denote the indexed family itself by the symbol \begin{equation*}
            \{A_{\alpha}\}_{\alpha \in J}
        \end{equation*}
        which is read ``the family of all $A_{\alpha}$, as $\alpha$ ranges over $J$."
    \end{definition}

    \begin{definition}
        Suppose $f:J\rightarrow \mathscr{A}$ is an indexing function for $\mathscr{A}$, and let $f(\alpha) = A_{\alpha}$. Then we define \begin{equation*}
            \bigcup\limits_{\alpha \in J}A_{\alpha}:=\{x\;\vert\;\text{for at least one $\alpha \in J, x \in A_{\alpha}$}\} = \bigcup\limits_{A\in \mathscr{A}}A,
        \end{equation*}
        and \begin{equation*}
            \bigcap\limits_{\alpha \in J}A_{\alpha} :=\{x\;\vert\;\text{for every $\alpha \in J, x \in A_{\alpha}$}\} = \bigcap\limits_{A\in\mathscr{A}}A
        \end{equation*}
    \end{definition}


    \begin{definition}
        Let $m$ be a positive integer. Given a set $X$, we define an \Emph{$m$-tuple} of elements of $X$ to be a function \begin{equation*}
            \mathbb{x}:\{1,...,m\}\rightarrow X
        \end{equation*}
        If $\mathbb{x}$ is an $m$-tuple, we often denote the value of $\mathbb{x}$ at $i$ by the symbol $x_i$, and call it the \Emph{$i$th coordinate of $\mathbb{x}$}.

        Now, let $\{A_1,...,A_m\}$ be a family of sets indexed with the set $\{1,...,m\}$. Let $X = A_1\cup ... \cup A_m$. We define the \Emph{cartesian product} of this indexed family, denoted by \begin{equation*}
            \prod\limits_{i=1}^mA_i = A_1\times ... \times A_m
        \end{equation*}
        to be the set of all $m$-tuples, $(x_1,...,x_m)$, of elements of $X$ such that $x_i \in A_i$ for each $i$.
    \end{definition}

    \begin{definition}
        Given a set $X$, we define an \Emph{$\omega$-tuple} of elements of $X$ to be a function \begin{equation*}
            \mathbb{x}:\Z_+\rightarrow X;
        \end{equation*}
        we also call such a function a \Emph{sequence}, or an \Emph{infinite sequence}, of elements of $X$. If $\mathbb{x}$ is an $\omega$-tuple, we often denote the value of $\mathbb{x}$ at $I$ by $x_i$, and call it the $i$th coordinate of $\mathbb{x}$, as before. We denote $\mathbb{x}$ itself by $(x_1,x_2,...)$ or $(x_n)_{n\in\Z_+}$. Let $\{A_1,A_2,...\}$ be a family of sets indexed by $\Z_+$; let $X$ be the union of the sets in this family. The \Emph{cartesian product} of this indexed family of sets, denoted by \begin{equation*}
            \prod\limits_{i\in\Z_+}A_i = A_1\times A_2 \times ...
        \end{equation*}
        is defined to be the set of all $\omega$-tuples of elements of $X$ such that $x_i \in A_i$ for each $i$.
    \end{definition}

    \section{Finite Sets}

    \begin{definition}
        A set $A$ is said to be \Emph{finite} if there is a bijective correspondence of $A$ with some section of the positive integers. That is, $A$ is finite if it is empty or if there is a bijection $f:A\rightarrow \{1,...,n\}$ for some $n \in \Z_+$. In the former case we say that $A$ has \Emph{cardinality $0$}, and in the latter case we say that $A$ has \Emph{cardinality $n$}.
    \end{definition}

    \begin{lemma}
        Let $n$ be a positive integer. Let $A$ be a set; let $a_0 \in A$. Then there exists a bijective correspondence $f$ of the set $A$ with the set $\{1,...,n+1\}$ if and only if there exists a bijective correspondence $g$ of the set $A-\{a_0\}$ with the set $\{1,...,n\}$.
    \end{lemma}
    \begin{proof}
        First, suppose there is a bijection correspondence $g:A-\{a_0\}\rightarrow \{1,...,n\}$. We then define $f:A\rightarrow \{1,...,n+1\}$ by setting $f(x) = g(x)$ for $x \in A-\{a_0\}$, and $f(a_0) = n+1$. Then, $f$ is surjective for every $m \in \{1,...,n+1\}$, if $m = n+1$ then $f(a_0) = m$, while if $m \neq n+1$, $m \in \{1,...,n\}$ so there exists $x \in A-\{a_0\}$ such that $f(x) = g(x) = m$ since $g$ is a bijection. Moreover, if $f(a_1) = f(a_2)$ for $a_1,a_2 \in A$, if one of $a_1$ or $a_2$ is $a_0$, then the other must also be $a_0$ since for all $x \in A-\{a_0\}$, $f(x) \in \{1,...,n\}$. Additionally, if both $a_1 \neq a_0 \neq a_2$, then $a_1,a_2 \in A-\{a_0\}$, so $g(a_1) = f(a_1) = f(a_2) = g(a_2)$, so $a_1 = a_2$ as $g$ is injective. Thus $f$ is bijective. 

        Conversely, assume there is a bijective correspondence $f:A\rightarrow \{1,...,n+1\}$. If $f$ maps $a_0$ to the number $n+1$, we can simply take the restriction $f\rvert_{A-\{a_0\}}$. Otherwise, let $f(a_0) = m$, and let $a_1$ be a point of $A$ such that $f(a_1) = n+1$. Then $a_1 \neq a_0$. Define a new function $h:A\rightarrow \{1,...,n+1\}$ by setting $h(a_0) = n+1$, $h(a_1) = m$, and $h(x) = f(x)$ for $x \in A-\{a_0,a_1\}$. It follows that $h$ is bijective, and from the previous case, the restriction $h\rvert_{A-\{a_0\}}$ is the desired bijection.
    \end{proof}


    \begin{theorem}
        Let $A$ be a set; suppose that there exists a bijection $f:A\rightarrow \{1,...,n\}$ for some $n \in \Z_+$. Let $B$ be a proper subset of $A$. Then there exists no bijection $g:B\rightarrow \{1,...,n\}$; but (provided $B\neq \emptyset$) there does exist a bijection $h:B\rightarrow \{1,...,m\}$ for some $m < n$.
    \end{theorem}
    \begin{proof}
        The case in which $B = \emptyset$ is immediate, for there cannot exist a bijection of the empty set $B$ with the nonempty set $\{1,...,n\}$.


        We shall proceed by induction. Let $C$ be the subset of $\Z_+$ consisting of those integers $n$ for which the theorem holds. We shall show that $C$ is inductive. From this we conclude that $C = \Z_+$, so the theorem is true for all positive integers $n$.

        First we show the theorem is true for $n = 1$. In this case $A$ consists of a single element $\{a\}$, and its only proper subset $B$ is the empty set.

        Now assume that the theorem is true for $n$; we prove it true for $n+1$. Suppose that $f:A\rightarrow \{1,...,n+1\}$ is a bijection, and $B$ is a nonempty proper subset of $A$. Choose an element $a_0 \in B$ and an element $a_1 \in A-B$. We apply the preceding lemma to conclude there is a bijection \begin{equation*}
            g:A-\{a_0\}\rightarrow \{1,...,n\}
        \end{equation*}
        Now $B - \{a_0\}$ is a proper subset of $A - \{a_0\}$, for $a_1$ belongs to $A-\{a_0\}$ and not to $B-\{a_0\}$. Because we have assumed the theorem holds for the integer $n$, we conclude there exists no bijection $h:B-\{a_0\}\rightarrow \{1,...,n\}$, and either $B-\{a_0\} = \emptyset$ or there exists a bijection \begin{equation*}
            k:B-\{a_0\}\rightarrow \{1,...,p\}
        \end{equation*}
        for some $p<n$. The preceding lemma combined with the lack of a bijection $H$ implies that there is no bijection of $B$ with $\{1,...,n+1\}$. This is the first half of what we wanted to prove. To prove the second half note that if $B - \{a_0\} = \emptyset$, there is a bijection of $B$ with the set $\{1\}$; while if $B-\{a_0\} \neq \emptyset$, we can apply the preceding lemma, along with the induction hypothesis to conclude that there is a bijection of $B$ with $\{1,...,p+1\}$. In either case, there is a bijection of $B$ with $\{1,...,m\}$ for some $m < n+1$, as desired. The induction principle now shows that the theorem is true for all $n \in \Z_+$.
    \end{proof}

    \begin{corollary}
        If $A$ is finite, there is no bijection of $A$ with a proper subset of itself.
    \end{corollary}
    \begin{proof}
        Assume that $B$ is a proper subset of $A$ and that $f:A\rightarrow B$ is a bijection. By assumption there is a bijection $g:A\rightarrow \{1,...,n\}$ for some $m$. The composite $g \circ f^{-1}$ is then a bijection of $B$ with $\{1,...,n\}$. This contradicts the preceding theorem.
    \end{proof}


    \begin{corollary}
        $\Z_+$ is not finite.
    \end{corollary}
    \begin{proof}
        The function $f:\Z_+\rightarrow \Z_+ -\{1\}$ defined by $f(n) = n+1$ is a bijection of $\Z_+$ with a proper subset of itself, so it cannot be finite.
    \end{proof}


    \begin{corollary}
        The cardinality of a finite set $A$ is uniquely determined by $A$.
    \end{corollary}
    \begin{proof}
        For the sake of contradiction, let $m < n$. Suppose there are bijections $f:A\rightarrow \{1,...,n\}$ and $g:A\rightarrow \{1,...,m\}$. Then the composite $g\circ f^{-1}:\{1,...,n\}\rightarrow \{1,...,m\}$ is a bijection of the finite set with a proper subset of itself, contradiction the previous corollary.
    \end{proof}


    \begin{corollary}
        If $B$ is a subset of the finite set $A$, then $B$ is finite. If $B$ is a proper subset of $A$, then the cardinality of $B$ is less than the cardinality of $B$.
    \end{corollary}
    

    \begin{corollary}
        Let $B$ be a nonempty set. Then the following are equivalent:\begin{enumerate}
            \item $B$ is finite.
            \item There is a surjective function from a section of the positive integers into $B$.
            \item There is an injective function from $B$ into a section of the positive integers.
        \end{enumerate}
    \end{corollary}
    \begin{proof}
        ($(1)\implies (2)$) Since $B$ is nonempty, there is, for some $n$, a bijection function $f:\{1,...,n\}\rightarrow B$ by definition, which is inherently surjective.

        ($(2)\implies (3)$) If $f:\{1,...,n\}\rightarrow B$ is surjective, define $g:B\rightarrow \{1,...,n\}$ by the equation \begin{equation*}
            g(b) = \text{smallest element of } f^{-1}(\{b\})
        \end{equation*}
        Because $f$ is surjective, the set $f^{-1}(\{b\})$ is nonempty; then the well-ordering property of $\Z_+$ tells us that $g(b)$ is uniquely defined. The map $g$ is injective, for if $b \neq b'$, then the sets $f^{-1}(\{b\})$ and $f^{-1}(\{b'\})$ must be disjoint, so their smallest elements must be different.


        ($(3)\implies (1)$) If $g:B\rightarrow \{1,...,n\}$ is injective, then changing the range of $g$ gives a bijection of $B$ with a subset of $\{1,...,n\}$. It follows from the preceding corollary that $B$ is finite.
    \end{proof}


    \begin{corollary}
        Finite unions and finite cartesian products of finite sets are finite.
    \end{corollary}
    \begin{proof}
        We first show that if $A$ and $B$ are finite, so is $A\cup B$. If $A$ or $B$ is empty the result is immediate. Otherwise, there are bijections $f:\{1,...,m\}\rightarrow A$ and $g:\{1,...,n\}\rightarrow B$ for some choice of $m$ and $n$. Define a function $h:\{1,...,m+n\}\rightarrow A\cup B$ by setting $h(i) = f(i)$ for $i \in \{1,...,m\}$ and $h(i) = g(i-m)$ for $i \in \{m+1,...,m+n\}$. Evidently, for all $a \in A$ there exists $i \in \{1,...,m\}$ such that $h(i) = f(i) = a$ since $f$ is a bijection. Moreover, for all $b \in B$ there exists $i \in \{1,...,n\}$ such that $h(i+m) = g(i) = b$, since $g$ is a bijection. Hence $h$ is surjective so by the previous corollary $A\cup B$ is finite.

        Now we show by induction that finiteness of the sets $A_1,...,A_n$ implies finiteness of their union. This result is immediate for $n=1$. Assuming it true for $n-1$ we note that $A_1\cup...\cup A_n$ is the union of two finite sets $A_1\cup ... \cup A_{n-1}$ and $A_n$ by the induction hypothesis, so the result of the preceding paragraph applies.

        Now, we show that the cartesian product of two finite sets $A$ and $B$ is finite. Given $a \in A$, the set $\{a\}\times B$ is finite, being in bijective correspondence with $B$. The set $A \times B$ is the union of these sets; since there are only finitely many of them, $A\times B$ is a finite union of finite sets and thus finite.

        Now we argue by induction that the finiteness of the sets $A_1,...,A_n$ implies the finiteness of $A_1\times ...\times A_n$. For $n = 1$ this result is immediate. Assuming it true for $n-1$, consider $A_1\times ... \times A_n$. Note that this is in bijective correspondence with the set $(A_1\times ... \times A_{n-1})\times A_n$, which is the cross product of two finite sets by the induction hypothesis. Hence by the previous argument $(A_1\times ... \times A_{n-1})\times A_n$ is finite. In particular, since it is in bijective correspondence with $A_1\times ... \times A_n$, this cartesian product is also finite.
    \end{proof}



    \section{Countable and Uncountable Sets}

    
    \begin{definition}
        A set $A$ is said to be \Emph{infinite} if it is not finite. It is said to be \Emph{countably infinite} if there is a bijective correspondence \begin{equation*}
            f:A\rightarrow \Z_+
        \end{equation*}
    \end{definition}


    \begin{definition}
        A set is said to be \Emph{countable} if it is either finite or countably infinite. A set that is not countable is said to be \Emph{uncountable}.
    \end{definition}

    \begin{theorem}
        Let $B$ be a nonempty set. Then the following are equivalent:\begin{enumerate}
            \item $B$ is countable.
            \item There is a surjective function $f:\Z_+\rightarrow B$.
            \item There is an injective function $g:B\rightarrow \Z_+$.
        \end{enumerate}
    \end{theorem}
    \begin{proof}
        ($(1)\implies(2)$) Suppose that $B$ is countable. If $B$ is countably infinite there is a bijection $f:\Z_+\rightarrow B$ by definition, and we are done. If $B$ is finite, there is a bijection $h:\{1,...,n\}\rightarrow B$ for some $n \geq 1$. We can extend $h$ to a surjection $f:\Z_+\rightarrow B$ by defining \begin{equation*}
            f(i) = \left\{\begin{array}{ll} h(i) & \text{for } 1\leq i \leq n \\
            h(1) & \text{for } i > n \end{array}\right.
        \end{equation*}

        ($(2)\implies (3)$) Let $f:\Z_+\rightarrow B$ be a surjection. Define $g:B\rightarrow \Z_+$ by the equation \begin{equation*}
            g(b) = \text{smallest element of $f^{-1}(\{b\})$}
        \end{equation*}
        Because $f$ is surjective, $f^{-1}(\{b\})$ is nonempty; thus $g$ is well defined by the well-ordering of $\Z_+$. The map $g$ is injective, for if $b \neq b'$, the sets $f^{-1}(\{b\})$ and $f^{-1}(\{b'\})$ are disjoint, so their elements are different.

        ($(3)\implies (1)$) Let $g:B\rightarrow \Z_+$ be an injection; we wish to show $B$ is countable. By changing the range of $g$, we can obtain a bijection of $B$ with a subset of $\Z_+$. Thus to prove our result it suffices to show that every subset of $\Z_+$ is countable. So let $C$ be a subset of $\Z_+$.

        If $C$ is finite, it is countable by definition. For $C$ an infinite subset of $\Z_+$ we provide a proof in the lemma to follow.
    \end{proof}


    \begin{lemma}
        If $C$ is an infinite subset of $\Z_+$, then $C$ is countably infinite.
    \end{lemma}
    \begin{proof}
        We define a bijection $h:\Z_+\rightarrow C$. We proceed by induction. Define $h(1)$ to be the smallest element of $C$; it exists because every nonempty subset $C$ of $\Z_+$ has a smallest element by the well-ordering of $\Z_+$. Then assuming that $h(1),...,h(n-1)$ are defined, define \begin{equation*}
            h(n) = \text{smallest element of } [C-h(\{1,...,n-1\})].
        \end{equation*}
        The set $C-h(\{1,...,n-1\})$ is not empty; for if it were empty, then $h:\{1,...,n-1\}\rightarrow C$ would be surjective, so that $C$ would be finite. Thus $h(n)$ is well defined. By induction we have defined $h(n)$ for all $n \in \Z_+$.


        To show that $h$ is injective consider $m \neq n$. Without loss of generality take $m < n$, and note that $h(m) \in \{h(\{1,...,n-1\})$, whereas $h(n)$, by definition, is not. Hence $h(n) \neq h(m)$.

        To show that $h$ is surjective, let $c$ be any element of $C$; we show that $c$ lies in the image set of $h$. First note that $h(\Z_+)$ cannot be contained in the finite set $\{1,...,c\}$ because $h(\Z_+)$ is infinite since $h$ is injective. Therefore, there is an $n \in \Z_+$, such that $h(n) > c$. Let $m$ be the smallest element of $\Z_+$ such that $h(m) \geq c$. Then for all $i < m$, we must have that $h(i) < c$. Thus, $c$ does not belong to the set $h(\{1,...,m-1\})$. Since $h(m)$ is defined as the smallest element of the set $C-h(\{1,...,m-1\})$, we must have that $h(m) \leq c$. Putting the two inequalities together, we have $h(m) = c$, as desired.
    \end{proof}


    \begin{definition}[Principle of Recursive Definition]
        Let $A$ be a set. Given a formula theat defines $h(1)$ as a unique element of $A$, and for $i > 1$ defines $h(i)$ uniquely as an element of $A$ in terms of the values of $h$ for positive integers less than $i$, this formula deterimes a unique function $h:\Z_+\rightarrow A$.
    \end{definition}


    \begin{corollary}
        A subset of a countable set is countable.
    \end{corollary}
    \begin{proof}
        Suppose $A \subseteq B$, where $B$ is countable. There is an injection of $f$ of $B$ into $\Z_+$; the restriction of $f$ to $A$ is an injection of $A$ into $\Z_+$.
    \end{proof}


    \begin{corollary}
        The set $\Z_+\times \Z_+$ is countably infinite.
    \end{corollary}
    \begin{proof}
        It suffices to construct an injective map $f:\Z_+\times \Z_+\rightarrow \Z_+$. We define $f$ by the equation $$f(n,m) = 2^n3^m$$ It is clear that $f$ is injective. For suppose that $2^n3^m = 2^p3^q$. If $n < p$, then $3^m = 2^{p-n}3^q$, contradicting the fact that $3^m$ is odd for all $m$. Therefore, $n = p$. As a result, $3^m = 3^q$. Then if $m < q$, it follows that $1 = 3^{q-m}$, another contradiction. Hence $m = q$.
    \end{proof}


    \begin{theorem}
        A countable union of countable sets is countable.
    \end{theorem}
    \begin{proof}
        Let $\{A_n\}_{n\in J}$ be an indexed family of countable sets, where the index set $J$ is either $\{1,...,N\}$ or $\Z_+$. Assume that each set $A_n$ is nonempty, for convenience.

        Because $A_n$ is countable, we can choose, for each $n$, a surjective function $f_n:\Z_+\rightarrow A_n$. Similarly, we can choose a surjective function $g:\Z_+\rightarrow J$. Now define \begin{equation*}
            h:\Z_+\times \Z_+\rightarrow \bigcup\limits_{n\in J}A_n
        \end{equation*}
        by the equation $h(k,m) = f_{g(k)}(m)$. Then indeed $h$ is surjective. Since $\Z_+\times\Z_+$ is in bijective correspondence with $\Z_+$, the countability of the union follows.
    \end{proof}

    \begin{theorem}
        A finite product of countable sets is countable.
    \end{theorem}
    \begin{proof}
        First let us show that the product of two countable sets is $A$ and $B$ is countable. The result is trivial if $A$ or $B$ is empty. Otherwise, choose surjective functions $f:\Z_+\rightarrow A$ and $g:\Z_+\rightarrow B$. Then the function $h:\Z_+\times \Z_+\rightarrow A\times B$ defined by $h(n,m) = (f(n),g(m))$ is surjective, so that $A\times B$ is countable.


        In general, we proceed by induction. Assuming that $A_1\times ... \times A_{n-1}$ is countable if each $A_i$ is countable, we prove the same thing for the product $A_1\times ... \times A_n$. FIrst, note that there is a bijective correspondence \begin{equation*}
            g:A_1\times ... \times A_n\rightarrow (A_1\times ... \times A_{n-1})\times A_n
        \end{equation*}
        defined by the equation \begin{equation*}
            g(x_1,...,x_n) = ((x_1,...,x_{n-1}),x_n)
        \end{equation*}
        Becuase the set $A_1\times ... A_{n-1}$ is countable by the induction hypothesis and $A_n$ is countable by assumption, the product of these two sets is countable, as proved in the preceding paragraph. We conclude that $A_1\times ... \times A_n$ is countable as well.
    \end{proof}

    It is not true that the countable product of countable sets is countable. Indeed, this fails almost immediately.

    \begin{theorem}
        Let $X$ denote the two element set $\{0,1\}$. Then the set $X^{\omega}$ is uncountable.
    \end{theorem}
    \begin{proof}
        We show that given any function $$g:\Z_+\rightarrow X^{\omega}$$ $g$ is not surjective. For this, let us denote $g(n)$ by \begin{equation*}
            g(n_ = (x_{n1},x_{n2},...,x_{nm},...)
        \end{equation*}
        where each $x_{ij}$ is either $0$ or $1$. Then we define an element $\mathbf{y} = (y_1,y_2,...,y_n,...)$ of $X^{\omega}$ by \begin{equation*}
            y_n = \left\{\begin{array}{ll} 0 & \text{if } x_{nn} = 1,\\ 1 & \text{if } x_{nn} = 0\end{array}\right.
        \end{equation*}
        Now $\mathbf{y}$ is an element of $X^{\omega}$, and $\mathbf{y}$ does not lie in the image of $g$; given $n$, the point $g(n)$ and the point $\mathbf{y}$ differ in at least one coordinate, namely, the $n$th. Thus, $g$ is not surjective.
    \end{proof}


    \begin{theorem}
        Let $A$ be a set. There is no injective map $f:\mathscr{P}(A)\rightarrow A$, and there is no surjective map $g:A\rightarrow \mathscr{P}(A)$. 
    \end{theorem}
    \begin{proof}
        In general, if $B$ is a nonempty set, the existence of an injective map $f:B\rightarrow C$ implies the existence of a surjective map $g:C\rightarrow B$; one defines $g(c) = f^{-1}(\{c\})$ for each $c \in \ran(f)$, and define $g$ arbitrarily on the rest of $C$.

        Therefore, it suffices to show that given a map $g:A\rightarrow \mathscr{P}(A)$, the map $g$ is not surjective. For each $a \in A$, the image $g(a)$ of $a$ is a subset o f$A$, which may or may not contain the point $a$ itself. Let $B$ be the subset of $A$ consisting of all those points $a$ such taht $g(a)$ does not contain $a$; \begin{equation*}
            B=\{a\;\vert\;a\in A-g(a)\}
        \end{equation*}
        Now, $B$ may be empty, or it may be all of $A$, but that does not matter. We assert that $B$ is a subset of $A$ that does not lie in the image of $g$. For suppose that $B = g(a_0)$ for some $a_0 \in A$. Does $a_0$ belong to $B$ or not? By definition of $B$, \begin{equation*}
            a_0 \in B \iff a_0 \in A-g(a_0) \iff a_0 \in A-B
        \end{equation*}
        In either case, we have a contradiction.
    \end{proof}
    

    \section{Principle of Recursive Definition}

    \begin{remark}
        Given the infinite subset $C$ of $\Z_+$, let us consider the following recursion formula for a function $h:\Z_+\rightarrow C$:\begin{equation*}
            \begin{array}{cl} h(1) &= \text{ smallest element of $C$}, \\
            h(i) &= \text{ smallest element of } [C-h(\{1,...,i-1\})]\;\;\text{ for } i > 1.
            \end{array} \tag{$(\star)$}
        \end{equation*}
        We shall prove that a unqiue such function $h$ exists in steps.
    \end{remark}

    \begin{lemma}{}{recexist}
        Given $n \in \Z_+$, there exists a function \begin{equation*}
            f:\{1,...,n\}\rightarrow C
        \end{equation*}
        that satisfies $(\star)$ for all $i$ in its domain.
    \end{lemma}
    \begin{proof}
        Let $A$ be the set of all $n$ for which the lemma holds. We show that $A$ is inductive. It then follows that $A = \Z_+$.


        The lemma is true for $n=1$, since the function $f:\{1\}\rightarrow C$ defined by the equation $f(1) = $ smallest element of $C$ satsfies $(\star)$ by the well ordering of $\Z_+$.

        Suppose the lemma to be true for $n-1$, we prove it true for $N$. By hypothesis there is a function $f':\{1,...,n-1\}\rightarrow C$ satisfying $(\star)$ for all $i$ in its domain. Define $f:\{1,...,n\}\rightarrow C$ by the equations \begin{align*}
            f(i) &= f'(i)\;\;\text{ for } i\in\{1,...,n-1\} \\
            f(n) &= \text{ smallest element of } [C-f'(\{1,...,n-1\})]
        \end{align*}
        Since $C$ is infinite, $f'$ is not surjective; hence the set $C-f'(\{1,...,n-1\})$ is nonempty, and $f(n)$ is well defined. Note that this definition does not define in terms of itself but in terms of the given function $f'$, so it is acceptable. Then by construction $f$ satisfies $(\star)$ for all $i$ in its domain since $f'$ does, and for $i = n$, $f(n) = $ smallest element of $[C-f'(\{1,...,n-1\})]$ and $f'(\{1,...,n-1\}) = f(\{1,...,n-1\})$.
    \end{proof}

    \begin{lemma}{}{recunique}
        Suppose that $f:\{1,...,n\}\rightarrow C$ and $g:\{1,...,m\}\rightarrow C$ both satisfy $(\star)$ for all $i$ in their respective domains. Then $f(i) = g(i)$ for all $i$ in both domains.
    \end{lemma}
    \begin{proof}
        Suppose towards a contradiction that the conclusion does not hold. Let $I$ be the smallest integer for which $f(i) \neq g(i)$. The integer $i$ is not $1$, because \begin{equation*}
            f(1) = \text{ smallest element of $C$ } = g(1),
        \end{equation*}
        by $(\star)$. Now, for all $j < i$, we have $f(j) = g(j)$. Because $f$ and $g$ satisfy $(\star)$, \begin{align*}
            f(i) &= \text{ smallest element of } [C-f(\{1,...,i-1\})] \\
            g(i) &= \text{ smallest element of } [C-g(\{1,...,i-1\})]
        \end{align*}
        Since $f(\{1,...,i-1\}) = g(\{1,...,i-1\})$, we have $f(i) = g(i)$, contrary to the choice of $i$.
    \end{proof}


    \begin{theorem}
        There exists a unique function $h:\Z_+\rightarrow C$ satisfying $(\star)$ for all $i \in \Z_+$.
    \end{theorem}
    \begin{proof}
        By Lemma \ref{lem:recexist}, there exists for each $n$ a function that maps $\{1,...,n\}$ into $C$ and satisfies $(\star)$ for all $i$ in its domain. Given $n$, Lemma \ref{lem:recunique} shows that this function is unique. Let $f_n:\{1,...,n\}\rightarrow C$ denote this unique function.


        We define a function $h:\Z_+\rightarrow C$ by defining its rule to be the union $U$ of the rules of the functions $f_n$. The rule for $f_n$ is a subset of $\{1,...,n\}\times C$; therefore, $U$ is a subset of $\Z_+\times C$. We must show that $U$ is a rule for a function $h:\Z_+\rightarrow C$.

        The integer $i$ lies in the domain of $f_n$ if and only if $n \geq i$. Therefore, the set of elements of $U$ of which $i$ is the first coordinate is precisely the set of all pairs of the form $(i,f_n(i))$ for $n \geq i$. Now Lemma \ref{lem:recunique} says that $f_n(i) = f_m(i)$ if $n,m \geq i$. Therefore, all these elements of $U$ are equal; that is, there is only one element of $U$ that has $i$ as its first coordinate.

        To show that $h$ satisfies $(\star)$ we note that \begin{align*}
            &h(i) = f_n(i)\;\;\text{ for } i \leq n \\
            &f_n\text{ satisfies $(\star)$ for all $i$ in its domain}
        \end{align*}
        The proof of uniqueness mirrors that of Lemma \ref{lem:recunique}.
    \end{proof}

    \begin{theorem}[Principle of Recursive Definition]
        Let $A$ be a set; let $a_0$ be an element of $A$. Suppose $\rho$ is a function that assigns, to each function $f$ mapping a nonempty section of the positive integers into $A$, an element of $A$. Then there exists a unique function \begin{equation*}
            h:\Z_+\rightarrow A
        \end{equation*}
        such that \begin{equation*}
            \begin{array}{cl} h(1) &= a_0, \\
                h(i) &= \rho(h\rvert_{\{1,...,i-1\}})\;\;\text{ for } i > 1.
            \end{array} \tag{$(\star)$}
        \end{equation*}
        The formula $(\star)$ is called a \Emph{recursion formula} for $h$. 
    \end{theorem}
    \begin{proof}
        (Left to the reader)
    \end{proof}


    \section{Infinite Sets and the Axiom of Choice}


    \begin{theorem}\label{thm:infequivs}
        Let $A$ be a set. The following statements about $A$ are equivalent:\begin{enumerate}
            \item There exists an injective function $f:\Z_+\rightarrow A$
            \item There exists a bijection of $A$ with a proper subset of itself
            \item $A$ is infinite
        \end{enumerate}
    \end{theorem}
    \begin{proof}
        $1.\implies 2.$ Assume there is an injective function $f:\Z_+\rightarrow A$. Let the image set $f(\Z_+)$ be denoted by $B$; and let $f(n)$ be denoted by $a_n$. Because $f$ is injective, $a_n\neq a_m$ if $n \neq m$. Define \begin{equation*}
            g:A\rightarrow A-\{a_1\}
        \end{equation*}
        by the equations \begin{align*}
            g(a_n) = a_{n+1}\;\;\text{ for }a_n \in B, \\
            g(x) = x\;\;\text{ for } x \in A-B
        \end{align*}
        By construction $g$ is surjective, and by the injectivity of $f$, $g$ is also injective and hence a bijection.

        $2.\implies 3.$ This result is the contrapositive of a previous result on finite sets.


        $3.\implies 1.$ Assume that $A$ is infinite and construct ``by induction" an injective function $f:\Z_+\rightarrow A$. First, since the set $A$ is nonempty, we choose a point $a_1 \in A$, and define $f(1) = a_1$. Then, assuming that we have defined $f(1),...,f(n-1)$, we wish to define $f(n)$. The set $A-f(\{1,...,n-1\})$ is not empty, for if it were empty the map $f:\{1,...,n-1\}\rightarrow A$ would be a surjection and $A$ would be finite. Hence, we can choose an element of the set $A-f(\{1,...,n-1\})$ and define $f(n)$ to be this element. ``Using the induction principle," we have defined $f$ for all $n \in \Z_+$.

        It follows by construction that $f$ is injective, for if $m < n$, then $f(m) \in f(\{1,...,n-1\})$, while by definition $f(n) \notin f(\{1,...,n-1\})$. Therefore, $f(n) \neq f(m)$.
    \end{proof}

    \begin{remark}
        What we have just done is not really a proof. Indeed, on the basis of the properties of set theory we have discussed thus far, it is not possible to prove this theorem. Precisely, before we have described certain allowable methods for specifying sets:\begin{enumerate}
            \item Defining a set by listing its elements, or by taking a given set $A$ and specifying a subset $B$ of it by giving a property that the elements of $B$ are to satisfy.
            \item Taking unions or intersections of the elements of a given collection of sets, or taking the difference of two sets.
            \item Taking the set of all subsets of a given set.
            \item Taking cartesian products of sets.
        \end{enumerate}
    \end{remark}

    \begin{axiom}[Axiom of Choice]
        Given a collection $\mathscr{A}$ of disjoint nonempty sets, there exists a set $C$ consisting of exactly one element from each element of $\mathscr{A}$. That is, a set $C$ such that $C$ is contained in the union of the elements of $\mathscr{A}$, and for each $A \in \mathscr{A}$, the set $C\cap A$ is a singleton.
    \end{axiom}

    \begin{lemma}[Existence of a Choice Function]
        Given a collection $\mathscr{B}$ of nonempty sets (not necessarily disjoint), there exists a function \begin{equation}
            c:\mathscr{B}\rightarrow \bigcup\limits_{B\in\mathscr{B}}B
        \end{equation}
        such that $c(B) \in B$ for each $B \in \mathscr{B}$. The function $c$ is called a \Emph{choice function} for the collection $\mathscr{B}$.
    \end{lemma}
    \begin{proof}
        Given an element $B$ of $\mathscr{B}$, we define a set $B'$ by $B' :=\{(B,x)\vert x \in B\}$. That is, $B'$ is the collection of all ordered pairs where the first coordinate of the ordered pair is the set $B$ and the second coordinate is an element of $B$. The set $B'$ is a subset of the cartesian product \begin{equation*}
            \mathscr{B}\times \bigcup\limits_{B\in \mathscr{B}}B
        \end{equation*}
        Because $B$ contains at least one element $x$, the set $B'$ contains at least the element $(B,x)$, so it is nonempty.

        Now, we claim that if $B_1$ and $b_2$ are two different sets in $\mathscr{B}$, then the corresponding sets $B'_1$ and $B'_2$ are disjoint. Indeed, for the typical element $(B_1,x_1) \in B_1'$ and $(B_2,x_2) \in B_2'$, $(B_1,x_1) \neq (B_2,x_2)$ as $B_1 \neq B_2$. Now let us form the collection \begin{equation*}
            \mathscr{C} :=\{B'\vert B \in \mathscr{B}\};
        \end{equation*}
        it is a collection of disjoint nonempty subsets of \begin{equation*}
            \mathscr{B}\times \bigcup\limits_{B\in \mathscr{B}}B
        \end{equation*}
        By the choice axiom, there exists a set $c$ consisting of exactly one element from each element of $\mathscr{C}$. Then $c$ contains exactly one element from each set $B'$, so for each $B \in \mathscr{B}$, the set $c$ contains exactly one ordered pair $(B,x)$ whose first coordiante is $B$. Thus $c$ is indeed the rule for a function from the collection $\mathscr{B}$ to the set $\bigcup_{B\in\mathscr{B}}B$. Finally, if $(B,x) \in c$, then $x$ belongs to $B$, so tha t$c(B) \in B$, as desired.
    \end{proof}




    \begin{proof}[Proof of Theorem \ref{thm:infequivs}]
        Given the infinite set $A$, we wish to construct an injective function $f:\Z_+\rightarrow A$. Let us form the collection $\mathscr{B}$ of all nonempty subsets of $A$. The lemma just proved asserts the existence of a choice function for $\mathscr{B}$; that is, that is a function \begin{equation*}
            c:\mathscr{B}\rightarrow \bigcup\limits_{B\in\mathscr{B}}B = A
        \end{equation*}
        such that $c(B) \in B$ for each $B \in \mathscr{B}$. Let us now define a function $f:\Z_+\rightarrow A$ by the recursion formula \begin{align*}
            f(1) &= c(A) \\
            f(i) &= c(A-f(\{1,...,i-1\}))\;\;\text{ for } i >1
        \end{align*}
        Because $A$ is infinite, the set $A-f(\{1,...,i-1\})$ is nonempty; therefore, the right side of this equation is well defined. Since this formula defines $f(i)$ uniquely in terms of $f\rvert_{\{1,...,i-1\}}$, the principle of recursive definition applies. We conclude that there exists a unieq function $f:\Z_+\rightarrow A$ satisfying $(\star)$ for all $i \in \Z_+$. Injectivity of $f$ follows as before.
    \end{proof}



    \section{Well-Ordered Sets}

    
    \begin{definition}
        A set $A$ with an order relation $<$ is said to be \Emph{well-ordered} if every nonempty subset of $A$ has a smallest element.
    \end{definition}

    \begin{example}
        $\Z_+$ is the prototypical example of a well-ordered set. Moreover, $\Z_+\times \Z_+$ with the dictionary order is a well-ordered set.
    \end{example}

    \begin{remark}
        Here are two ways of constructing well-ordered sets:\begin{enumerate}
            \item If $A$ is a well-ordered set, then any subset of $A$ is well-ordered in the restricted order relation.
            \item If $A$ and $B$ are well-ordered sets, then $A\times B$ is well-ordered in the dictionary order.
        \end{enumerate}
        It follows that $\Z_+\times(\Z_+\times \Z_+)$ is well ordered, and in general for $n \in \Z_+$, $(\Z_+)^n$ is well-ordered.
    \end{remark}

    \begin{theorem}
        Every nonempty finite ordered set has the order type of a section $\{1,...,n\}$ of $\Z_+$, so it is well-ordered.
    \end{theorem}
    \begin{proof}
        First, we show that every finite ordered set $A$ has a largest element. If $A$ has one element this is trivial. Supposing it true for sets having $n-1$ elements, let $A$ have $n$ elements and let $a_0 \in A$. Then $A-\{a_0\}$ has a largest element $a_1$, and $\max(\{a_0,a_1\})$ is the largest element of $A$.


        Second, we show there is an order-preserving bijection of $A$ with $\{1,...,n\}$ for some $n$. If $A$ has one element this fact is immediate. Suppose that it is true for sets having $n-1$ elements. Let $b$ be the largest element of $A$. By hypothesis there is an order preserving bijection \begin{equation*}
            f':A-\{b\}\rightarrow \{1,...,n-1\}
        \end{equation*}
        Define an order preserving bijection $f:A\rightarrow \{1,...,n\}$ by settign $f(x) = f'(x)$ for all $x \neq b$, and $f(b) = n$.
    \end{proof}

    \begin{theorem}[Well-ordering Theorem]
        If $A$ is a set, there exists an order relation on $A$ that is a well-ordering.
    \end{theorem}


    \begin{corollary}
        There exists an uncountable well-ordered set.
    \end{corollary}


    \begin{definition}
        Let $X$ be a well-ordered set. Given $\alpha \in X$, let $S_{\alpha}$ denote the set \begin{equation*}
            S_{\alpha} := \{x \in X\vert x < \alpha\}
        \end{equation*}
        It is called the \Emph{section} of $X$ by $\alpha$.
    \end{definition}


    \begin{lemma}
        There exists a well-ordered set $A$ having a largest element $\Omega$, such that the section $S_{\Omega}$ of $A$ by $\Omega$ is uncountable, but every other section of $A$ is countable.
    \end{lemma}
    \begin{proof}
        Let $B$ be an uncountable well-ordered set. Let $C$ be the well-ordered set $\{1,2\}\times B$ in the dictionary order; then some section of $C$ is uncountable. Indeed, the section of $C$ by any element of the form $2\times b$ is uncountable. Let $\Omega$ be the smallest element of $C$ for which the section of $C$ by $\Omega$ is uncountable. Then let $A$ consist of this section along with the element $\Omega$.
    \end{proof}

    \begin{remark}
        $S_{\Omega}$ is an uncountable well-ordered set every section of which is countable. Its order type is in fact uniquely determined by this condition, and we call it a \Emph{minimal uncountable well-ordered set}. Furthermore, we denote the well-ordered set $A = S_{\Omega}\cup\{\Omega\}$ by $\overline{S}_{\Omega}$.
    \end{remark}

    \begin{theorem}
        If $A$ is a countable subset of $S_{\Omega}$, then $A$ has an upper bound in $S_{\Omega}$.
    \end{theorem}
    \begin{proof}
        Let $A$ be a countable subset of $S_{\Omega}$. For each $a \in A$, the section $S_a$ is countable. Therefore, the union $B = \bigcup_{a\in A}S_a$ is also countable. Since $S_{\Omega}$ is uncountable, the set $B$ is notall of $S_{\Omega}$; let $x$ be a point of $S_{\Omega}$ that is not in $B$. Then $x$ is an upper bound for $A$. For if $x < a$ for some $a \in A$, then $x$ belongs to $S_a$ and hence to $B$, contrary to choice.
    \end{proof}


    \section{The Maximum Principle}

    \begin{definition}
        Given a set $A$, a relation $\prec$ on $A$ is called a \Emph{strict partial order} on $A$ if it has the following two properties: \begin{enumerate}
            \item (Nonreflexivity) The relation $a \prec a$ never holds,
            \item (Transitivity) If $a \prec b$ and $b \prec c$, then $a \prec c$.
        \end{enumerate}
    \end{definition}

    \begin{theorem}[The Maximum Principle]
        Let $A$ be a set; let $\prec$ be a strict partial order on $A$. Then there exists a maximal simply ordered subset $B$ of $A$.
    \end{theorem}


    \begin{example}
        If $\mathscr{A}$ is any collection of sets, the relation $\subset$ (proper subset) is a strict partial order on $\mathscr{A}$. 
    \end{example}

    \begin{definition}
        Let $A$ be a set and let $\prec$ be a strict partial order on $A$. If $B$ is a subset of $A$, an \Emph{upper bound} of $B$ is an element $c \in A$ such that for every $b \in B$, either $b = c$ or $b \prec c$. A \Emph{maximal element} of $A$ is an element $m$ of $A$ such that for no element of $a$ of $A$ does the relation $m \prec a$ hold.
    \end{definition}

    \begin{lemma}[Zorn's Lemma]
        Let $A$ be a set that is strictly partially ordered. If every simply ordered subset of $A$ has an upper bound in $A$, then $A$ has a maximal element.
    \end{lemma}

    \begin{remark}
        A \Emph{partial order} on a set $A$ is a relation $\preceq$ such that $a \preceq b$ if either $a = b$ or $a \prec b$ where $\prec$ is a strict partial order on $A$.
    \end{remark}



