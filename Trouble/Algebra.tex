\documentclass[12pt, a4paper, twoside, openright, titlepage]{book}
\usepackage[utf8]{inputenc}
\raggedbottom
%%%%%%%%%%%%%%%%% Book Formatting Comments:

%%%%%%%%%%%%%%%%%%%%%%%%%%%%%%%%%%%%% for Part

%%%%%%%%%%%%%%%%%%%%%% for chapter

%%%%%%%%%%%%%%%%%%%% for section




%%%%%% PACKAGES %%%%%%%
\usepackage{hyperref}
\hypersetup{
    colorlinks,
    citecolor=black,
    filecolor=black,
    linkcolor=black,
    urlcolor=black
}
\usepackage{amsmath} % Math display options
\usepackage{amssymb} % Math symbols
%\usepackage{amsfonts} % Math fonts
%\usepackage{amsthm}
\usepackage{mathtools} % General math tools
\usepackage{array} % Allows you to write arrays
\usepackage{empheq} % For boxing equations
% \usepackage{mathabx}
% \usepackage{mathrsfs}
\usepackage{nameref}
\usepackage{wrapfig}

\usepackage{soul}
\usepackage[normalem]{ulem}

\usepackage{txfonts}
\usepackage{cancel}
\usepackage[toc, page]{appendix}
\usepackage{titletoc,tocloft}
\setlength{\cftchapindent}{1em}
\setlength{\cftsecindent}{2em}
\setlength{\cftsubsecindent}{3em}
%\setlength{\cftsubsubsecindent}{4em}
\usepackage{titlesec}

%\titleformat{\section}
%  {\normalfont\fontsize{25}{15}\bfseries}{\thesection}%{1em}{}
%\titleformat{\section}
%  {\normalfont\fontsize{20}{15}\bfseries}%{\thesubsection}{1em}{}
%\setcounter{secnumdepth}{1}  
  
  

%\newcommand\numberthis{\refstepcounter{equation}\tag{\theequation}} % For equation labelling
\usepackage[framemethod=tikz]{mdframed}

\usepackage{tikz} % For drawing commutative diagrams
\usetikzlibrary{cd}
\usetikzlibrary{calc}
\tikzset{every picture/.style={line width=0.75pt}} %set default line width to 0.75p

\usepackage{datetime}
\usepackage[margin=1.5in]{geometry}
\setlength{\parskip}{1em}
\usepackage{makeidx}         % allows index generation
\usepackage{graphicx}       % standard LaTeX graphics tool
\usepackage{multicol}        % used for the two-column index
\usepackage[bottom]{footmisc}% places footnotes at page bottom

\usepackage{newtxtext}       % 
\usepackage{newtxmath}       % selects Times Roman as basic font
\usepackage{float}
\usepackage{fancyhdr}
\setlength{\headheight}{15pt} 
\pagestyle{fancy}
\lhead[\leftmark]{}
\rhead[]{\leftmark}

%\usepackage{enumitem}

\usepackage{url}
\allowdisplaybreaks

%%%%%% ENVIRONMENTS %%%
\definecolor{purp}{rgb}{0.29, 0, 0.51}
\definecolor{bloo}{rgb}{0, 0.13, 0.80}



%%\newtheoremstyle{note}% hnamei
%{3pt}% hSpace above
%{3pt}% hSpace belowi
%{}% hBody fonti
%{}% hIndent amounti
%{\itshape}% hTheorem head fonti
%{:}% hPunctuation after theorem headi
%{.5em}% hSpace after theorem headi
%{}% hTheorem head spec (can be left empty, meaning ‘normal’)i





% %%%%%%%%%%%%% THEOREM DEFINITIONS

\spnewtheorem{axiom}{Axiom}[chapter]{\bfseries}{\itshape}


\spnewtheorem{construction}{Construction}[chapter]{\bfseries}{\itshape}

\spnewtheorem{props}{Properties}[chapter]{\bfseries}{\itshape}


\renewcommand{\qedsymbol}{$\blacksquare$}


\numberwithin{equation}{section}

\newenvironment{qest}{
    \begin{center}
        \em
    }
    {
    \end{center}
    }

%%%%%% MACROS %%%%%%%%%
%% New Commands
\newcommand{\ip}[1]{\langle#1\rangle} %%% Inner product
\newcommand{\abs}[1]{\lvert#1\rvert} %%% Modulus
\newcommand\diag{\operatorname{diag}} %%% diag matrix
\newcommand\tr{\mbox{tr}\.} %%% trace
\newcommand\C{\mathbb C} %%% Complex numbers
\newcommand\R{\mathbb R} %%% Real numbers
\newcommand\Z{\mathbb Z} %%% Integers
\newcommand\Q{\mathbb Q} %%% Rationals
\newcommand\N{\mathbb N} %%% Naturals
\newcommand\F{\mathbb F} %%% An arbitrary field
\newcommand\ste{\operatorname{St}} %%% Steinberg Representation
\newcommand\GL{\mathbf{GL}} %%% General Linear group
\newcommand\SL{\mathbf{SL}} %%% Special linear group
\newcommand\gl{\mathfrak{gl}} %%% General linear algebra
\newcommand\G{\mathbf{G}} %%% connected reductive group
\newcommand\g{\mathfrak{g}} %%% Lie algebra of G
\newcommand\Hbf{\mathbf{H}} %%% Theta fixed points of G
\newcommand\X{\mathbf{X}} %%% Symmetric space X
\newcommand{\catname}[1]{\normalfont\textbf{#1}}
\newcommand{\Set}{\catname{Set}} %%% Category set
\newcommand{\Grp}{\catname{Grp}} %%% Category group
\newcommand{\Rmod}{\catname{R-Mod}} %%% Category r-modules
\newcommand{\Mon}{\catname{Mon}} %%% Category monoid
\newcommand{\Ring}{\catname{Ring}} %%% Category ring
\newcommand{\Topp}{\catname{Top}} %%% Category Topological spaces
\newcommand{\Vect}{\catname{Vect}_{k}} %%% category vector spaces'
\newcommand\Hom{\mathbf{Hom}} %%% Arrows

\newcommand{\map}[2]{\begin{array}{c} #1 \\ #2 \end{array}}

\newcommand{\Emph}[1]{\textbf{\ul{\emph{#1}}}}




%% Math operators
\DeclareMathOperator{\ran}{Im} %%% image
\DeclareMathOperator{\aut}{Aut} %%% Automorphisms
\DeclareMathOperator{\spn}{span} %%% span
\DeclareMathOperator{\ann}{Ann} %%% annihilator
\DeclareMathOperator{\rank}{rank} %%% Rank
\DeclareMathOperator{\ch}{char} %%% characteristic
\DeclareMathOperator{\ev}{\bf{ev}} %%% evaluation
\DeclareMathOperator{\sgn}{sign} %%% sign
\DeclareMathOperator{\id}{Id} %%% identity
\DeclareMathOperator{\supp}{Supp} %%% support
\DeclareMathOperator{\inn}{Inn} %%% Inner aut
\DeclareMathOperator{\en}{End} %%% Endomorphisms
\DeclareMathOperator{\sym}{Sym} %%% Group of symmetries


%% Diagram Environments
\iffalse
\begin{center}
    \begin{tikzpicture}[baseline= (a).base]
        \node[scale=1] (a) at (0,0){
          \begin{tikzcd}
           
          \end{tikzcd}
        };
    \end{tikzpicture}
\end{center}
\fi




\newdateformat{monthdayyeardate}{%
    \monthname[\THEMONTH]~\THEDAY, \THEYEAR}
%%%%%%%%%%%%%%%%%%%%%%%


%%%%%% BEGIN %%%%%%%%%%


\begin{document}

%%%%%% TITLE PAGE %%%%%

\begin{titlepage}
    \centering
    \scshape
    \vspace*{\baselineskip}
    \rule{\textwidth}{1.6pt}\vspace*{-\baselineskip}\vspace*{2pt}
    \rule{\textwidth}{0.4pt}
    
    \vspace{0.75\baselineskip}
    
    {\LARGE Group, Ring, and Field Theory: A Complete Guide}
    
    \vspace{0.75\baselineskip}
    
    \rule{\textwidth}{0.4pt}\vspace*{-\baselineskip}\vspace{3.2pt}
    \rule{\textwidth}{1.6pt}
    
    \vspace{2\baselineskip}
    Abstract Algebra \\
    \vspace*{3\baselineskip}
    \monthdayyeardate\today \\
    \vspace*{5.0\baselineskip}
    
    {\scshape\Large Elijah Thompson, \\ Physics and Math Honors\\}
    
    \vspace{1.0\baselineskip}
    \textit{Solo Pursuit of Learning}
\end{titlepage}

%%%%%%%%%%%%%%%%%%%%%%%
\tableofcontents

%%%%%%%%%%%%%%%%%%%%%%%%%%%%%%%%%%%%% Part 1
\part{Group Theory}

%%%%%%%%%%%%%%%%%%%%%% - P1.Chapter 1
\chapter{\textsection Basic Definitions and Examples: Groups}

\section{Initial Definitions}

\begin{defn}{}{}
    A \Emph{binary operation} on a non-empty set $S$ is a map $$\beta:S\times S\rightarrow S$$ where $S\times S := \{(a,b): a,b \in S\}$, and $(a,b)\mapsto a*b = \beta(a,b)$.
\end{defn}

\begin{eg}{}{}
    \leavevmode
    \begin{enumerate}
        \item $\map{S\times S\rightarrow S}{(a,b)\mapsto a}$ known as projection to the first factor
        \item $\map{\Z\times \Z\rightarrow \Z}{(a,b)\mapsto 0 = a*b}$ known as the zero map
        \item $\map{\Z\times \Z\rightarrow \Z}{(a,b)\mapsto a+b}$ addition
    \end{enumerate}
    \begin{rmk}{}{}
        There can be many different binary operations on a given set.
    \end{rmk}
\end{eg}

\begin{defn}{Group}{}
    A pair $(G, \star)$ where $G$ is a set and $\star$ is a binary operation on $G$ is called a \Emph{group} if: \begin{enumerate}
        \item[G1.] (\Emph{Associativity}) For all $a,b,c \in G$, $$(a\star b)\star c = a\star (b\star c)$$
        \item[G2.] (\Emph{Identity Element}) There exists $e \in G$ such that for all $a \in G$ $$a\star e = e \star a = a$$
        \item[G3.] (\Emph{Inverses}) For all $a \in G$ there exists $b \in G$ such that $$a \star b = b \star a = e$$ In this case we write $b = a^{-1}$
    \end{enumerate}
\end{defn}


\begin{rmk}{}{}
    The operation $\star$ can be denoted in many ways: $\cdot$, $+$, juxtaposition. The identity element $e$ is sometimes denoted $1_G$, $1$, $e_G$, and $0$.
\end{rmk}

\begin{eg}{}{}
    \leavevmode
    \begin{enumerate}
        \item $(\Z,+)$ is a group with $e = 0$ and the inverse of $a$ is denoted $-a$
        \item ($\R_{>0},\cdot$) where $\cdot$ is multiplication. $\cdot$ is a binary operation on $\R_{>0}$ because for all $a,b \in \R_{>0}$ we have $a,b >0$ so $a\cdot b > 0$ and $a \cdot b \in \R_{>0}$. Moreover, ($\R_{>0},\cdot$) is a group with identity $1$.
        \item (Non-example) $(\Z,-)$, $(a,b)\mapsto a-b$. This is not a group. Indeed, although $-$ is a binary operation on $\Z$, tere is no identity element and it's not associative.
        \item (Non-example) $(\Z\backslash \{0\},\cdot)$ is associative and has identity $1$, but does not have inverses for all $a \in \Z$.
    \end{enumerate}
\end{eg}

\begin{defn}{}{}
    A group $(G,\star)$ is called \Emph{abelian} if $a \star b = b \star a$ for all $a, b \in G$.
    \begin{rmk}{}{}
        Abelian groups are also known as \Emph{commutative groups}.
    \end{rmk}
\end{defn}


\begin{eg}{}{}
    $\GL_n(\R)$ is the general linear group of dimension $n \geq 1$, defined by \begin{equation}
        \GL_n(\R) := \left(\left\{ A \in M_{n\times n}(\R): \det(A) \neq 0\right\},\underbrace{\circ}_{matrix product}\right)
    \end{equation}
    \begin{xca*}{}{}
        $\GL_n(\R)$ is a group with identity $I_n = (\delta_{ij})$, but it is non-abelian for $n \geq 2$.
    \end{xca*}
\end{eg}

\begin{xca*}{}{}
    If $(G,\star)$ is a group, then $(G,\star')$ is also group with $$a \star' b := b\star a, \forall a,b \in G$$
    \begin{proof*}{}{}
        Note that for all $a,b \in G$, $a \star' b = b \star a \in G$ since $\star$ is a binary operation on $G$ by assumption, so $\star'$ is also a binary operation on $G$. (cont.)
    \end{proof*}
\end{xca*}


\section{The Group of Symmetries}

\begin{defn}{Symmetric Group}{}
    Let $X$ be a non-empty set. Then, define \begin{equation}
        S_X:=\{\sigma:X\rightarrow X:\sigma\;\text{is a bijection}\}
    \end{equation}
    Such a $\sigma$ is called a \Emph{permutation} of $X$. It follows that $(S_X, \circ)$ is a group where \begin{equation}
        \map{\circ:S_X\times S_X\rightarrow S_X}{(\sigma,\tau)\mapsto \sigma \circ \tau}
    \end{equation}
    is function composition. The group is also commonly denoted as $Sym(X)$.
    \begin{proof*}{}{}
        Let $X$ be a non-empty set, and define $S_X$ as above. (cont.)
    \end{proof*}
\end{defn}

\begin{defn}{}{}
    If $X = \{1,2,...,n\}$ for $n \in \N$, then $S_X = S_{\{1,2,...,n\}}$ is denoted $S_n$ and is called the \Emph{symmetric group of degree n} or \Emph{symmetric group on n letters}.
\end{defn}

\begin{eg}{}{}
    Take $n = 3$: $\sigma \in S_3$ can be represented as a $2\times n$ matrix by $$\begin{pmatrix} 1 & 2 & 3 \\ \sigma(1) & \sigma(2) & \sigma(3) \end{pmatrix}$$ where $\sigma(1)\;\sigma(2)\;\sigma(3)$ is a permutation of $1\;2\;3$.
    \begin{eg}{}{}
        $\sigma = \begin{pmatrix} 1 & 2 & 3 \\ 1 & 3 & 2 \end{pmatrix}$, $\gamma = \begin{pmatrix} 1 & 2 & 3 \\ 2 & 3 & 1 \end{pmatrix}$, then $$\sigma \circ \gamma = \begin{pmatrix} 1 & 2 & 3 \\ 3 & 2 & 1 \end{pmatrix}$$
        Observe $\sigma^{-1} = \begin{pmatrix} 1 & 2 & 3 \\ 1 & 3 & 2 \end{pmatrix} = \sigma$ and $\gamma^{-1} = \begin{pmatrix} 1 & 2 & 3 \\ 3 & 1 & 2 \end{pmatrix}$. 
    \end{eg}
    The identity permutation is denoted $\id = \begin{pmatrix} 1 & 2 & 3 \\ 1 & 2 & 3 \end{pmatrix}$
    \begin{note*}{}{}
        We can also have written $$\begin{pmatrix} 2 & 1 & 3 \\ \sigma(2) & \sigma(1) & \sigma(3) \end{pmatrix}$$
        instead
    \end{note*}
\end{eg}

\begin{eg}{}{}
    $n=2: S_2 = \{\id, \tau\}$, where $$\id = \begin{pmatrix} 1 & 2  \\ 1 & 2  \end{pmatrix}, \tau = \begin{pmatrix} 1 & 2 \\ 2 & 1 \end{pmatrix}, \tau^2 = \id$$
\end{eg}

\subsection{Notation: Cycles}

The cycle notation is more compact then the matrix-type notation, although this does come with some ambiguity:

\begin{eg}{}{}
    \leavevmode
    \begin{enumerate}
        \item $(1\;2) \in S_3$ means $1 \mapsto 2$, $2\mapsto 1$, and $3\mapsto 3$ (a \Emph{transposition}).
        \begin{enumerate}[label =\(\drsh\)]
            \item Note this is the same as $(2\;1)$ which is where ambiguity can arise.
        \end{enumerate}
        \item $(1\;2\;3) \in S_3$, means $1 \mapsto 2$, $2\mapsto 3$, $3\mapsto 1$
        \begin{enumerate}[label = $\drsh$]
            \item Visual - 
            \begin{tikzpicture}[->,scale=.7]
               \node (1) at (90:1cm)  {$1$};
               \node (2) at (-30:1cm) {$2$};
               \node (3) at (210:1cm) {$3$};
            
               \draw (70:1cm)  arc (70:-10:1cm);
               \draw (-50:1cm) arc (-50:-130:1cm);
               \draw (190:1cm) arc (190:110:1cm);
            \end{tikzpicture}
            (Note $(1\;2\;3) = (2\;3\;1) = (3\;1\;2)$)
        \end{enumerate}
    \end{enumerate}
\end{eg}


\begin{defn}{}{}
    Let $k_1,k_2,..., k_r \in \{1,2,...,n\}$ be distinct $(r \leq n)$. Then the permutation in $S_n$ which sends $k_1\mapsto k_2\mapsto k_3\mapsto ... \mapsto k_r \mapsto k_1$ and fixes all other numbers is denoted \begin{equation}
        (k_1\;k_2\;...\;k_r)
    \end{equation}
    which is a \Emph{cycle of length $r$} or an \Emph{$r$-cycle}.
\end{defn}

\begin{rmk}{}{}
    The only $1$-cycle is the identity permutation in $S_n$
    \begin{enumerate}
        \item[$\drsh$] $(1) = (2) = ... = (n)$
    \end{enumerate}
\end{rmk}

\begin{defn}{}{}
    Cycles of length $\geq 2$ in $S_n$ are called \Emph{disjoint} if they \emph{move} disjoint sets of numbers
    \begin{enumerate}
        \item[$\drsh$] \begin{eg}{}{}
            $(1\;2), (3\;4)$ are disjoint in $S_4$, but $(1\;2),(3\;1)$ are \underline{not}.
        \end{eg} 
    \end{enumerate}
\end{defn}


\begin{rmk}{}{}
    Every permutation in $S_n$($\neq \id$), can be written as a product of disjoint cycles of length $\geq 2$. Moreover, this factorization is unique up to the order of the factors.
    \begin{proof*}{}{}
        (Left to the reader)
    \end{proof*}
\end{rmk}

\begin{eg}{}{}
    In $S_5$ $(1\;4\;5)\circ(2\;3) = (2\;3) \circ(1\;4\;5)$ as they are disjoint
    \begin{enumerate}
        \item[$\drsh$] (so we can't hope for full unicity of the factorization) 
    \end{enumerate}
\end{eg}


\section{General Properties}

\begin{defn}{}{}
    For a group $(G, \star)$, the number of elements (cardinal) of $G$ is denoted $|G|$ and called the \Emph{order} of the group (can be infinite).
\end{defn}

\begin{eg}{}{}
    \begin{enumerate}
        \item $(\Z,+)$ has infinite order
        \item $(S_n,\circ)$ has order $n!$ (permutations)
    \end{enumerate}
\end{eg}

\begin{prop}{}{}
    Let $(G,\star)$ be a group. Then we have the following properties:
    \begin{enumerate}
        \item The identity element is unique.
        \begin{proof*}{}{}
            If $e,i \in G$ are identity elements, then $e = e\star i = i$ as $e \star g = e$ and $g \star i = g$ for all $g \in G$ by assumption.
        \end{proof*}
        \item The inverse of an element is unique.
        \begin{lem}{Cancellation Lemma}{}
            If $a \star b = a \star c$ or $b \star = c \star a$, then $b = c$ for all $a,b,c \in G$. 
        \end{lem}
        \begin{proof*}{}{}
            Let $a^{-1}$ be an inverse of $a$. Then $$b = e\star b = (a^{-1} \star a) \star b = a^{-1} \star a \star c = e \star c = c$$ and similarly for $b \star a = c \star a$.
        \end{proof*}
        \begin{rmk}{}{}
            $a \star b = c \star a$ does \underline{not} tell us anything in general.
        \end{rmk}
        \begin{proof*}{}{}
            Take $b,c,$ inverses of $a$. Then $b \star a = e = c \star a$, so by the cancellation lemma $b = c$.
        \end{proof*}
        \begin{enumerate}
            \item[$\drsh$] We will denote \underline{the} inverse of $a$ by $a^{-1}$ for all $a \in G$. 
        \end{enumerate}
        \begin{cor}{}{}
            For all $a \in G$, if $b \star a = e$ (or $a \star b = e$) the identity element, then we have that $b = a^{-1}$, so $b$ is the inverse of $a$.
        \end{cor}
    \end{enumerate}
\end{prop}


\begin{defn}{}{}
    Let $(G, \star)$ be a group. Let $n \in \Z$, $g \in G$, then \begin{equation}
        g^n = \left\{\begin{array}{ll} e, & \text{if } n = 0 \\ \underbrace{g\star g \star ... \star g}_{\text{n-fold times}}, & \text{if } n > 0 \\ \underbrace{g^{-1}\star g^{-1} \star ... \star g^{-1}}_{\text{-n-fold times}}, & \text{if } n < 0\end{array}\right.
    \end{equation}
\end{defn}

\begin{prop}{}{}
    For all $g \in G$ and all $n,m \in \Z$, \begin{enumerate}
        \item $(g^n)^m = g^{nm}$
        \item $g^n\star g^m = g^{n+m}$
    \end{enumerate}
    \begin{proof*}{}{}
        (Left to the reader)
    \end{proof*}
\end{prop}

\begin{eg}{}{}
    Let $G = \Z/2\Z$, $g = [1]$, and the operation be $+$. Then $g^{-1} = -1\cdot g = [-1] = [1]$, $g^0 = 0\cdot g = [0]$, $g^1 = 1\cdot g = [1]$, $g^2 = 2\cdot g = [1]+[1] = [2] = [0]$, and $g^3 = 3\cdot g = [3] = [1]$, etc.
\end{eg}

\begin{rmk}{}{}
    Due to associativity, the placement of parenthesis is unambiguous and unnecessary:
    $$\drsh ((a\star (b\star c)) \star d) = ((a \star b) \star (c \star d))$$
    Thus, $g_1\star g_2 \star ... \star g_n$ is well-defined for all $n \in \N$. 
    \begin{proof*}{}{}
        (Left to the reader)
    \end{proof*}
\end{rmk}

\begin{note*}{}{}
    However, because we don't necessarily have commutivity, it is \underline{not} true in general that $(g\star h)^n = g^n \star h^n$. Indeed, $(g\star h )^2 = g\star h \star g \star h$, not necessarily $g^2\star h^2$.
\end{note*}

\begin{rmk}{Inverse of a product}{}
    Let $a,b \in G$, a group. Then \begin{enumerate}
        \item $(a \star b)^{-1} = b^{-1}\star a^{-1}$
        \item More generally \begin{equation}
            (a_1\star a_2 \star ... \star a_n)^{-1} = a_n^{-1}\star ... \star a_2^{-1} \star a_1^{-1}
        \end{equation}
        for $a_i \in G$, $1 \leq i \leq n$
    \end{enumerate}
    \begin{proof*}{}{}
        (Left to the reader)
    \end{proof*}
\end{rmk}

\begin{eg}{}{}
    In $(\Z,+)$, $g^2$ for $g = 3$ is $g^2 = 2\cdot g = 2\cdot 3 = 6$. In general $g^n = ng$ for additive groups.
\end{eg}




%%%%%%%%%%%%%%%%%%%%%% - P1.Chapter 2
\chapter{\textsection Group Homomorphisms}

\section{Basic Definitions and Examples: Group Homomorphisms}

\begin{defn}{A}{}
    A \Emph{group homomorphism} from a group $A$ to a group $B$ is a map of sets $\phi:A\rightarrow B$ that satisfies the condition: \begin{equation}
        \forall a_1,a_2\in A,\;\;\;\phi(a_1\cdot a_2) = \phi(a_1)\cdot \phi(a_2)
    \end{equation}
    We can rewrite the condition on $\phi$ in terms of the requirement that the following diagram commute: \begin{center}
            \begin{tikzpicture}[baseline = (a).base]
            \node[scale = 1] (a) at (0,0){
                \begin{tikzcd}
                    A\times A  \ar[d, "\phi\times\phi", swap] \ar[r, "mult_A"] & A \ar[d,"\phi"] \\
                    B\times B \ar[r, "mult_B"] & B
                \end{tikzcd}
            };
            \end{tikzpicture}
        \end{center}
        We denote the set of all group homomorphisms from $A$ to $B$ by $\Hom_{\Grp}(A,B)$.
\end{defn}


\begin{defn}{B}{}
    Let $(G,\star)$ and $(M,\circ)$ be groups. A map $G\xrightarrow{f} M$ is a \Emph{homomorphism of groups} if \begin{equation}
        f(x\star y) = f(x)\circ f(y),\forall x,y \in G
    \end{equation} 
\end{defn}


\begin{eg}{}{}
    \leavevmode
    \begin{enumerate}
        \item $H \leq G$, then $H \xhookrightarrow{\iota} G$ the inclusion map is a \Emph{monomorphism} (an injective homomorphism)
        \item $\map{(\R,+)\rightarrow \R^{\times}}{x\mapsto 2^x}$ is a monomorphism.
        \item $\map{S_3 \hookrightarrow S_4}{\sigma \mapsto \sigma'}$, where $\sigma'(i) = \sigma(i)$ for $i \in \{1,2,3\}$ and $\sigma'(4) = 4$. In particular $$\begin{pmatrix} 1 & 2 & 3 \\ \sigma(1) & \sigma(2) & \sigma(3)  \end{pmatrix}\mapsto \begin{pmatrix} 1 & 2 & 3 & 4 \\ \sigma'(1) & \sigma'(2) & \sigma'(3) & 4  \end{pmatrix}$$
        is a monomorphism.
        \item $\map{\det:\GL_n(\R) \rightarrow \R^{\times}}{A \mapsto \det(A)}$ is an \Emph{epimorphism} (surjective homomorphism)
        \item $(G,\star)$ a group, then for $g \in G$, $$\map{\Z\xrightarrow{\phi_g}G}{x\mapsto g^n}$$ is a group homomorphism. It needs not be injective or surjective.
        \item $\map{|\cdot|:\C^{\times}\rightarrow \R^{\times}}{z\mapsto |Z|}$ is an epimorphism.
    \end{enumerate}
\end{eg}


\begin{props}{}{}
    Let $f:(G,\star)\rightarrow (H,\circ)$ be a group homomorphism. Then \begin{enumerate}
        \item $f(e_G) = e_H$
        \item for all $g \in G$, $f(g^{-1}) = (f(g))^{-1}$
        \item The composition of two homomorphisms $G \xrightarrow{f} H \xrightarrow{\phi}K$ is a homomorphism $G\xrightarrow{\phi \circ f}K$.
    \end{enumerate}
\end{props}
\begin{proof*}{}{}
    (Left to the reader)
\end{proof*}

\begin{rmk}{}{}
    \leavevmode
    \begin{enumerate}
        \item If $H_1 \leq G\xrightarrow{f} K$ where $f$ is a homomorphism, then the canonical map \begin{equation}
            f\rvert_{H_1}H_1\rightarrow K
        \end{equation}
        is a homomorphism called the \Emph{restriction of $f$ to $H_1$}. Moreover, $f\rvert_{H_1} = f \circ \iota_{H_1}$, where $\iota_{H_1}:H_1\hookrightarrow G$ is the inclusion homomorphism.
        \item Let $f:G\rightarrow K$ be a homomorphism. If the image of $f$, $\ran(f)$, is contained in a subgroup $H_2 \leq K$, then the associated map \begin{equation}
            f':G\rightarrow H_2
        \end{equation}
        is a homomorphism of groups.
    \end{enumerate}
\end{rmk}

\begin{eg}{}{}
    For $\det:\GL_n(\R)\rightarrow \R^{\times}$, we have $\SL_n(\R) \leq \GL_n(\R)$, so $\SL_n(\R)\xrightarrow{\det}\R^{\times}$ is a homomorphism, and \begin{equation}
        \ran\left(\det\rvert_{\SL_n(\R)}\right) = \{1\} \leq \R^{\times}
    \end{equation}
    so \begin{equation}
        \det:\SL_2(\R)\rightarrow \{1\}
    \end{equation}
    is a homomorphism.
\end{eg}


\begin{prop}{}{}
    Let $G\xrightarrow{f} K$ be a group homomorphism. Then \begin{enumerate}
        \item Let $H_1 \leq G$, then the image $f(H_1) \leq K$ is a subgroup of $K$
        \item Let $H_2 \leq K$, then $f^{-1}(H_2) \leq G$ is a subgroup of $G$, called the inverse image of $H_2$, or the pre-image of $H_2$ by $f$.
    \end{enumerate}
\end{prop}
\begin{proof*}{}{}
    (Left to the reader)
\end{proof*}


\begin{rmk}{}{}
        Note that the image of a cyclic subgroup $\langle g \rangle \leq G$ under a group homomorphism $f: G \rightarrow K$ is the cyclic subgroup \begin{equation}
                \langle f(g) \rangle \leq K
        \end{equation}
\end{rmk}
\begin{proof*}{}{}
        (Left to the reader)
\end{proof*}

\begin{cor}{}{}
        Let $G \xrightarrow{f} K$ be a group homomorphism. \begin{enumerate}
                \item The \Emph{image} $f(G) = \ran(f)$ is a subgroup of $K$
                \item The \Emph{kernel} $\ker(f) := f^{-1}(\{e_K\})$ is a subgroup of $G$
        \end{enumerate}
\end{cor}

\begin{prop}{}{}
        Let $G\xrightarrow{f} K$ be a group homomorphism. Then $f$ is injective (i.e. a \Emph{monomorphism}) if and only if $\ker(f) = \{e_G\}$.
\end{prop}
\begin{proof*}{}{}
        (Left to the reader)
\end{proof*}


\begin{eg}{}{}
        \leavevmode
        \begin{enumerate}
                \item $f$ a homomorphism is an isomorphism if and only if $\ker(f) = \{e_G\}$ and $\ran(f) = K$ (for $G\xrightarrow{f} K$)
                \item $H \leq G$, $H \xhookrightarrow{\iota} G$ the inclusion map is a homomorphism with $\ran(\iota) = H$ and $\ker(\iota) = \{e_G\}$
                \item The determinant map $\GL_n(\R)\xrightarrow{\det}\R^{\times}$ is a group homomorphism. Moroever, we obtain the subgroup \begin{equation}
                                \SL_n(\R) := \{A \in \GL_n(\R):\det(A) = 1\} = \ker(\det)
                        \end{equation}
                \item The map $\map{\Z\xrightarrow{\phi_g} G}{n \mapsto g^n}$ for some fixed $g \in G$ is a group homomorphism with image $\ran \phi_g = \langle g \rangle$, and $\ker(\phi_g) = \{n \in \Z:g^n = e_g\} = S_g$. Thus, $\phi_g$ is surjective if and only if $\langle g\rangle = G$ and $\phi_g$ is injective if and only if $o(g) = +\infty$.
                \item The modulus map $\C^{\times} \xrightarrow{|\cdot|}\R^{\times}$ is a group homomorphism with $\ker(|\cdot|) = \{z = \C^{\times}:|z| = 1\} = S^1$ the circle group, and $\ran(|\cdot|) = \R_{>0}$.
                \item The map $\map{\R\xrightarrow{\alpha}\C^{\times}}{\theta\mapsto \exp(2\pi i \theta)}$ is a group homomorphism with $\ker(\alpha) = \Z$ and $\ran(\alpha) = S^1$.
        \end{enumerate}
\end{eg}




\subsection{Group Isomorphisms}

\begin{eg}{motivating examples}{}
    \leavevmode
    \begin{enumerate}
        \item $g_1 := s_{\{1,2,3\}}$ and $g_2 := s_{\{a,b,c\}}$ are essentially the ``same" group, but their elements are not the same. indeed ``everything we do" in $g_1$ using the group operation we can do in $g_2$ by renaming $1$ as $a$, $2$ as $b$, and $3$ and $c$: order ($|g_1| = |g_2|$), subgroups, orders of elements, ``equations," etc.
        \item $\Z/2\Z = \{[0],[1]\}$, $g = \{+1,-1\} \leq (\R\backslash\{0\},\cdot) = \R^{\times}$ are the ``same" group. let us consider their \emph{cayley tables}:
        \begin{equation*}
            \begin{array}{c|cc}
                (\Z/2\Z,+) & [0] & {[1]}  \\ \hline
                {[0]} & [0] & {[1]} \\
                {[1]} & [1] & {[0]} \\
            \end{array}
            \hspace{40pt}
            \begin{array}{c|cc}
                (g, \cdot) & +1 & -1  \\ \hline
                +1 & +1 & -1 \\
                -1 & -1 & +1\\
            \end{array}
        \end{equation*}
        $[0]$ plays the role of $+1$ and $[1]$ plays the role of $-1$.
    \end{enumerate} 
\end{eg}



\begin{prop}{}{}
    $\Z/n\Z$ is isomorphic to the group of $n$th roots of unity: \begin{equation}
        \{z\in\C:z^n = 1\}
    \end{equation}
\end{prop}

\begin{defn}{}{}
    Let $(G,\star)$ and $(M,\circ)$ be groups. A bijective map $G\xrightarrow{f} M$ which is a group homomorphism is called an \Emph{isomorphism} of the groups $G$ and $M$. The groups $(G,\star)$ and $(M,\circ)$ are said to be \Emph{isomorphic}, denoted $(G,\star) \cong (M,\circ)$.
\end{defn}

\begin{eg}{}{}
    \leavevmode
    \begin{enumerate}
        \item $\map{\Z\xrightarrow{f}2\Z}{n\mapsto 2n}$ is a group isomorphism.
        \item If $G$ is a cyclic group, $G = \langle g \rangle$, and $|G| = n <+\infty$, then \begin{equation}
            \map{\Z/n\Z\xrightarrow{\phi}G}{{[m]}\mapsto g^m}
        \end{equation}
        is a group isomorphism. Recall that $g^m = g^{m'}$ if and only if $m\equiv m' \mod n$ ($o(g) = n$), so $\phi$ is well-defined and injective. By construction $\phi$ is also surjective. Therefore, $\Z/n\Z \cong G$.
        \item If $G = \langle g \rangle$ and $o(g) = +\infty$, then $G \cong \Z$. Indeed, the map \begin{equation}
            \map{\Z\xrightarrow{\phi} G}{m\mapsto g^m}
        \end{equation}
        is a group isomorphism.
        \item The map \begin{equation}
            \map{(\R,+)\xrightarrow{\exp}(\R_{>0},\cdot)}{x\mapsto e^x}
        \end{equation}
        is a group isomorphism, so $(\R,+)\cong (\R_{>0},\cdot)$. Another isomorphism is \begin{equation}
            \map{(\R,+)\xrightarrow{\phi}(\R_{>0},\cdot)}{x\mapsto 2^x}
        \end{equation}
        \item For all $h \in G$, where $(G,\star)$ is an arbitrary group, \begin{equation*}
            \map{(G,\star) \xrightarrow{\alpha_h}(G,\star)}{a \mapsto h^{-1}\star a \star h}
        \end{equation*}
        is a group isomorphism with inverse $\alpha_{h^{-1}}$.
        \item $\map{\Z\xrightarrow{\beta}\Z}{n\mapsto -n}$ is a group isomorphism.
    \end{enumerate}
\end{eg}

\begin{defn}{}{}
    An \Emph{isomorphism} $(G,\star)\rightarrow(G,\star)$ is called an \Emph{automorphism} of $(G,\star)$.
\end{defn}

\begin{prop}{}{}
    The set of automorphisms of a group $G$ is a subgroup of the symmetric group $S_G$.\begin{enumerate}
        \item The identity map $\map{\id:G\rightarrow G}{g\mapsto g}$ is an isomorphism (hence an automorphism) which acts as an identity for the group
        \item If $G\xrightarrow{\phi}H$ is an isomorphism, then the inverse $H\xrightarrow{\phi^{-1}}G$ is an isomorphism
        \item If $G\xrightarrow{\phi}H$ and $H\xrightarrow{\psi}K$ are isomorphisms, then the composition $\psi \circ \phi$ is also an isomorphism
    \end{enumerate}
\end{prop}
\begin{proof*}{}{}
    (Left to the reader)
\end{proof*}

\begin{cor}{}{}
    The set $\aut(G)$ of automorphisms of $G$ is a group for the composition of maps.
\end{cor}

\begin{eg}{Non-example}{}
    $\Q\cancel{\cong} \Q^{\times}$ because $-1 \in \Q^{\times}$ has order $2$, but $\Q$ does not have any element of order $2$. Indeed, if $x \in \Q$ such that $x+x=2x = 0$, then $x = 0$ so $o(x) = 1$. But, if $\Q^{\times}\xrightarrow{\phi}\Q$ is an isomorphism, then $o(\phi(-1)) = 2$, which is not possible.
\end{eg}

\begin{thm}{Dihedral Group Isomorphisms}{}
    Let $x,y \in G$ such that $G = \langle x,y \rangle (:= \langle \{x,y\}\rangle)$ with the relations $x^n = y^2 = e$, $yx = x^{-1}y$, and $n \geq 1$. Then $|G| \leq 2n$. If $|G| = 2n$, then the relation characterizes the group up to isomorphism.
\end{thm}
\begin{proof*}{}{}
    If $n = 1$, $G \cong \Z/2\Z \cong D_1$. If $n = 2$, $G \cong \Z/2\Z \times \Z/2\Z \cong D_2$, mapping $x \mapsto ([1],[0])$ and $y \mapsto ([0],[1])$. Now, for all $n \geq 1$, let $Y_n := \{e,x,x^2,...,x^{n-1},y,xy,...,x^{n-1}y\}$. We know that $G = Y_n$ as sets from our study of the relations on the dihedral group. But $|Y_n| = 2n$ as a set, so $|G| \leq 2n$ as a group. If $|G| = 2n$, then all elements described in $Y_n$ are distinct in $G$, and the Caley table of the group is fixed by the relations. Thus, the group $G$ is uniquely determined up to isomorphism by the relations and $|G| = 2n$. If $n \geq 3$ we have seen that $\langle x = \phi_{2\pi/n},y = \psi_0\rangle = D_n$, $x,y$ satisfy the relations and $|D_n| = 2n$, so $G \rightarrow D_n$ is an isomorphism (taking $x$ in $G$ to $x$ in $D_n$ and $y$ in $G$ to $y$ in $D_n$).
\end{proof*}

\begin{cor}{}{}
    $S_3 \cong D_3$ for $x = (1\;2\;3)$, $y = (1\;2)$.
    \begin{proof*}{}{}
        (Left to the reader)
    \end{proof*}
\end{cor}



\section{Automorphisms}

\begin{defn}{}{}
    Let $G$ be a group. An isomorphism from $G$ onto itself is called an \Emph{automorphism} of $G$. The group of all automorphisms on $G$ is denoted by $\aut(G)$.
\end{defn}


\begin{prop}{}{}
    Let $H$ be a normal subgroup of the group $G$. Then $G$ acts by conjugation on $H$ as automorphisms of $H$. More specifically, the action of $G$ on $H$ by conjugation is defined for each $g \in G$ by \begin{equation*}
        h\mapsto ghg^{-1}\;\;\;\text{ for each }\;h\in H
    \end{equation*}
    For each $g \in G$, conjugation by $g$ is an automorphism of $H$. The permutation representation afforded by this action is a homomorphisms of $G$ into $\aut(H)$ with kernel $C_G(H) = \{g \in G:\forall h\in H, ghg^{-1} =h\}$ (The centralizer of $H$ with respect to $G$). In particular, $G/C_G(H)$ is isomorphic to a subgroup of $\aut(H)$.
\end{prop}

\begin{rmk}{}{}
    This proposition implies that a group acts by conjugation on a normal subgroup as \emph{structure preserving permutations}, i.e. automorphisms.
\end{rmk}

\begin{cor}{}{}
    If $K$ is any subgroup of the group $G$ and $g \in G$, then $K \cong gKg^{-1}$. Conjugate elements and conjugate subgroups have the same order (as the induced map is an automorphism).
\end{cor}


\begin{cor}{}{}
    For any subgroup $H$ of a group $G$, the quotient group $N_G(H)/C_G(H)$ is isomorphic to a subgroup of $\aut(H)$. In particular, $G/Z(G)$ is isomorphic to a subgroup of $\aut(G)$.
\end{cor}
\begin{proof*}{}{}
    Since $H$ is a normal subgroup of $N_G(H)$, our previous proposition implies that $N_G(H)$ acts by conjugation on $H$. Moreover, $C_G(H) \subseteq N_G(H)$, so the kernel of the permutation representation of $N_G(H)$ in $\aut(H)$ afforded by this action is $C_G(H)$. Hence by the first isomorphism theorem $N_G(H)/C_G(H)$ is isomorphic to a subgroup of $\aut(H)$.

    The second case follows from taking $H = G$, so $N_G(G) = G$ and $C_G(G) = Z(G)$.
\end{proof*}


\begin{defn}{}{}
    Let $G$ be a group and let $g \in G$. Conjugation by $g$ is called an \Emph{inner automorphism} of $G$ and the subgroup of $\aut(G)$ consisting of all inner automorphisms is denoted by $\inn(G)$.
\end{defn}


\begin{defn}{}{}
    A subgroup $H$ of a group $G$ is called \Emph{characteristic in $G$} if and only if every automorphism of $G$ maps $H$ onto itself, i.e., $\sigma(H) = H$ for all $\sigma \in \aut(G)$.
\end{defn}


\begin{prop}
    Let $H$ be a subgroup of a group $G$: \begin{enumerate}
        \item If $H$ is characteristic in $G$ then $H \vartriangleleft G$,
        \item If $H$ is the unique subgroup of $G$ of a given order, then $H$ is characteristic in $G$,
        \item If $K$ is a characteristic subgroup of $H$ and $H \vartriangleleft G$, then $K\vartriangleleft G$.
    \end{enumerate}
\end{prop}
\begin{proof*}{}{}
    (To be completed)
\end{proof*}


\begin{cor}{}{}
    If $C$ is a cyclic group of order $n$, then every subgroup of $C$ is characteristic in $C$.
\end{cor}


\begin{prop}{}{}
    The automorphism group of the cyclic group of order $n$ is isomorphic to $(\Z/n\Z)^{\times}$, an abelian group of order $\varphi(n)$ (where $\varphi$ is the Euler-totient function).
\end{prop}
\begin{proof*}{}{}
    Let $x$ be a generator of the cyclic group $\Z/n\Z$. If $\psi \in \aut(\Z/n\Z)$, then $\psi(x) = x^a$ for some $a \in \Z$, and the integer $a$ uniquely determines $\psi$. Denote this automorphism by $\psi_a$. As usual, since $|x| = n$, the integer $a$ is only defined modulo $n$. Since $\psi_a$ is an automorphism, $x$ and $x^a$ must have the same order, hence $\gcd(a,n) = 1$. Furthermore, for $a$ relatively prime to $n$, the map $x\mapsto x^a$ is an automorphism of $\Z/n\Z$. Hence we have a surjective map \begin{equation*}
        \map{\Psi:\aut(\Z/n\Z)\rightarrow (\Z/n\Z)^{\times}}{\psi_a\mapsto a\; (\mod n)}
    \end{equation*}
    The map $\Psi$ is a homomorphism because \begin{equation*}
        \psi_a\circ\psi_b(x) = \psi_a(x^b) = x^{ab} = \psi_{ab}(x)
    \end{equation*}
    for all $\psi_a,\psi_b \in \aut(\Z/n\Z)$, so that \begin{equation*}
        \Psi(\psi_a\circ\psi_b) = \Psi(\psi_{ab}) = ab\;(\mod n) = \Psi(\psi_a)\Psi(\psi_b)
    \end{equation*}
    Finally, $\Psi$ is injective by construction of the $\psi_a$, and hence is an isomorphism.
\end{proof*}


\begin{eg}{}{}
    \leavevmode
    \begin{enumerate}
        \item If $p$ is an odd prime and $n \in \Z^+$, then the automorphism group of the cyclic group of order $p$ is cyclic of order $p-1$. More generally, the automorphism group of the cyclic group of order $p^n$ is cyclic of order $p^{n-1}(p-1)$.
        \item Let $p$ be a prime and let $V$ be an abelian group (written additively) with the property that $pv =0$ for all $v \in V$. If $|V| = p^n$, then $V$ is an $n$-dimensional vector space over the field $\F_p = \Z/p\Z$. The automorphisms of $V$ are precisely the non-singular linear transformations from $V$ to itself, that is \begin{equation*}
                \aut(V) \cong \GL(V) \cong \GL_n(\F_p)
        \end{equation*}
    \end{enumerate}
\end{eg}






%%%%%%%%%%%%%%%%%%%%%% - P1.Chapter 3
\chapter{\textsection Subgroups}

\section{Basic Definitions and Examples: Subgroups}

\begin{defn}{}{}
    A subset $H$ of a group $(G,\star)$ (i.e. $H \subseteq G$) is a \Emph{subgroup} if it satisfies the following properties:
    \begin{enumerate}
        \item[S1] (\Emph{identity}) $e \in H$, where $e$ is the identity in $G$ (so $H \neq \emptyset$)
        \item[S2] (\Emph{closure}) If $h_1,h_2 \in H$, then $h_1\star h_2 \in H$.
        \item[S3] (\Emph{inverses}) Of $h \in H$, then $h^{-1} \in H$.
    \end{enumerate}
    Thus, the group $(H,\star\vert_{H})$ makes sense. We denote this by $H \leq G$, to say $H$ is a subgroup of $G$ ($\star$ is understood from context)
\end{defn}

\begin{eg}{}{}
    \leavevmode
    \begin{enumerate}
        \item $\{e\} \leq G$ and $G \leq G$, for any group $G$. These subgroups are called the \Emph{trivial subgroups of $G$}.
        \item $\Z\leq \R$ with addition
        \item $l\Z := \{ln:n \in \Z\}\subset \Z$, where $l$ is fixed in $\Z$. $l\Z \leq \Z$. Indeed \begin{enumerate}
            \item[S1] $0 = l\cdot 0 \in l\Z$
            \item[S2] $ln+lm = l(n+m) \in l\Z$
            \item[S3] $ln + l(-n) = l(n+(-n)) = l\cdot 0 = 0$, so $-(ln) \in l\Z$.
        \end{enumerate}
        \item $SL_n(\R) := \{A \in Mat_n(\R):\det(A) = 1\} \leq \GL_2(\R)$ is a subgroup called the \Emph{special linear group} of degree $n$.
        \begin{proof*}{}{}
            (Left to the reader)
        \end{proof*}
        \item Let $S^1 := \{z \in \C:|z| = 1\}$, then $S^1 \leq \C\backslash\{0\} = \C^{\times}$, where the operation is multiplication. $S^1$ is called the \Emph{circle group}
    \end{enumerate}
\end{eg}


\begin{prop}{}{}
    If $H_1 \leq G$, $H_2 \leq G$ are subgroups of $G$, then $H = H_1 \cap H_2 \leq G$ is a subgroup.
    \begin{proof*}{}{}
        (Left to the reader)
    \end{proof*}
\end{prop}

\begin{cor}{}{}
    Let $\{H_{\alpha}\}_{\alpha \in J}$ be a collection of subgroups of a group $G$, where $J$ is an indexing set (possible infinite). Then \begin{equation}
        \bigcap_{\alpha \in J}H_{\alpha} \leq G
    \end{equation}
    \begin{proof*}{}{}
        (Left to the reader)
    \end{proof*}
\end{cor}

\subsection{Center}

\begin{defn}{}{}
    Let $(G,\star)$ be a group and let $g \in G$. The \Emph{centralizer} of $g \in G$ is \begin{equation}
        Z(g) := \{x \in G: \underbrace{x\star g = g \star x}_{x\text{ and }g\text{ commute}}\}
    \end{equation}
\end{defn}

\begin{claim}{}{}
    For all $g \in G$, for a group $(G,\star)$, $Z(g) \leq G$ is a subgroup of $G$.
    \begin{proof*}{}{}
        (Left to the reader)
    \end{proof*}
\end{claim}

\begin{eg}{}{}
    \leavevmode
    \begin{enumerate}
        \item $Z(2) \leq (\Z,+)$, and actually $Z(2) = \Z$ as $\Z$ is abelian.
        \item $Z(I_2 + E_{2,2}) \leq \GL_2(\R)$, and in particular $$Z(I_2+E_{2,2}) = \left\{\begin{pmatrix} a & 0 \\ 0 & d \end{pmatrix} \in \GL_2(\R):ad \neq 0, ad\in \R\right\}$$
    \end{enumerate}
\end{eg}

\begin{defn}{}{}
    The \Emph{center} $Z(G)$ of the group $G$ is \begin{equation}
        Z(G) := \{x \in G:x\star g = g \star x \forall g \in G\}
    \end{equation}
\end{defn}

\begin{prop}{}{}
    $Z(G) = \bigcap_{g \in G} Z(g)$, so also $Z(G) \leq G$.
\end{prop}

\begin{xca*}{}{}
    For $n \geq 2$, $Z(\GL_n(\R)) \{aI_n: a \in \R^{\times}\}$
\end{xca*}

\begin{rmk}{}{}
    For any group $G$, $Z(G)$ is an \Emph{abelian} group.
\end{rmk}


\section{Cyclic Subgroups}


\begin{defn}{}{}
    Let $S \subset G$ a subset of a group $G$. Then the \Emph{subgroup of $G$ generated by $S$} is \begin{equation}
        \langle S\rangle := \bigcap\{H \in \mathcal{P}(G): S \subset H \leq G\}
    \end{equation}
    (The intersection of all the subgroups of $G$ containing $S$) which is the smallest subgroup of $G$ containing $S$.
    \begin{note*}{}{}
        $S \subset G \leq G$, so $S \subset \langle S \rangle$ and $\langle S \rangle$ is well-defined. $\langle S \rangle$ is the smallest subgroup of $G$ containing $S$ with respect to set inclusion.
    \end{note*}
\end{defn}

\begin{eg}{}{}
    $\langle \emptyset\rangle = \{e\}$ and $\langle G \rangle = G$ (trivial subgroups)
\end{eg}

\begin{prop}{}{}
    Let $g \in G$, then \begin{equation}
        \langle \{g\}\rangle = \{g^n:n \in \Z\}
    \end{equation}
    and we write $\langle g \rangle := \langle \{g\}\rangle$.
    \begin{proof*}{}{}
        (Left to the reader)
    \end{proof*}
\end{prop}

\begin{eg}{}{}
    \leavevmode
    \begin{enumerate}
        \item $\langle e \rangle = \{e\}$
        \item For $[2] \in \Z/3\Z$, the $\langle [2] \rangle = \{[2], [1], [0]\} = \Z/3\Z$.
        \item For $l \in \Z$, $\langle l \rangle = \{nl:n\in\Z\}\leq \Z$. We write $l\Z := \langle l \rangle$.
    \end{enumerate}
\end{eg}

\begin{defn}{}{}
    A group $K$ such that there exists $g \in K$ with $\langle g \rangle = K$ is called a \Emph{cyclic group}.
\end{defn}
\begin{enumerate}
    \item[$\drsh$] i.e$\rangle$ A group generated by single element is called \Emph{cyclic}.
\end{enumerate}


\begin{defn}{}{}
    Then, for all $g \in G$, $\langle g \rangle \leq G$ is cyclic. The order of $|\langle g \rangle|$ of the group $\langle g \rangle$ is called the \Emph{order of $g$}, and denoted $o(g)$ (could be infinite!).
\end{defn}


\begin{eg}{}{}
    \leavevmode
    \begin{enumerate}
        \item For $1 \in \Z$, $o(1) = \infty$ given $\langle 1 \rangle = \Z$. Thus, $\Z$ is cyclic of infinite order.
        \begin{enumerate}
            \item[$\drsh$] \underline{Note:} The \underline{only} other generator of $\Z$ is $-1$.
            \begin{enumerate}
                \item[$\drsh$] For $2 \in \Z$ does not generate $\Z$ because $\langle 2 \rangle = 2\Z \subsetneq \Z$ 
            \end{enumerate}
        \end{enumerate}
        \item For $n \in \Z_{>0}$, $[1]_n \in \Z/n\Z$, $o([1]_n) = n$ given $$\langle [1]_n\rangle = \{k[1]_n:k \in \Z\} = \Z/n\Z$$
        Thus $\Z/n|Z$ is a cyclic group of order $n$.
        \item For any group $G$, $o(e) = 1$ where $e$ is the identity of $G$.
        \item $\Q^{\times} = (\Q\backslash\{0\},\cdot)$ is not cyclic.
        \begin{proof*}{}{}
            Assume $\Q^{\times} = \langle \frac{a}{b}\rangle$, then take $p$ relatively prime to $a$ and to $p$. Then $p \notin \left\{\left(\frac{a}{b}\right)^n: n \in \Z\right\}$, which contradicts the assumption and $\Q^{\times}$ is not cyclic.
        \end{proof*}
    \end{enumerate}
\end{eg}


\begin{rmk}{}{}
    A cyclic group is abelian.
    \begin{proof*}{}{}
        (Left to the reader)
    \end{proof*}
\end{rmk}

\begin{cor}{}{}
    If a group is non-abelian it cannot be cyclic.
    \begin{proof*}{}{}
        Contrapositive of the previous statement.
    \end{proof*}
\end{cor}
\begin{enumerate}
    \item[$\drsh$] $\GL_n(\R)$, $n \geq 2$, and $S_m$, $m \geq 3$, are not cyclic as they are non-abelian. 
\end{enumerate}


$\Z$ is cyclic and properties of $\Z$ will have consequences for all cyclic groups (through isomorphisms).

\begin{thm}{}{}
    Every subgroup $H$ of $(\Z,+)$ is of the form $n\Z = \langle n \rangle$ for some $n \in \Z$.
    \begin{proof*}{}{}
        (Left to the reader)
    \end{proof*}
\end{thm}

\begin{cor}{GCD}{}
    Let $a,b \in \Z$ and define \begin{equation}
        a\Z + b\Z := \{an + bm: n,m \in \Z\} \subset \Z
    \end{equation}
    $a\Z+b\Z$ is a subgroup of $(\Z,+)$ generated by $\{a,b\}$. Then, by the previous theorem $a\Z + b\Z = d\Z$ for some $d \in \Z$, and if $(a,b) \neq (0,0)$, we have that $d = \gcd(a,b)$. We choose $d > 0$ to maintain uniqueness.
    \begin{proof*}{}{}
        (Left to the reader)
    \end{proof*}
\end{cor}

\begin{cor}{}{}
    Suppose $a\Z + b\Z = d\Z$, $(a,b) \neq (0,0)$, so $d \neq 0$. Then \begin{enumerate}
        \item $d\;\vert\;a$ and $d\;\vert\;b$.
        \item If $e \;\vert\;a$ and $e\;\vert\;b$, then $e \;\vert\;d$.
        \item There exist $x,y \in \Z$ such that $d = ax + by$.
    \end{enumerate}
\end{cor}

\begin{prop}{}{}
    Take the cyclic subgroup $\langle g \rangle \leq G$ of $G$. Then define \begin{equation}
        S_g := \{k \in \Z: g^k = e\}\subset \Z
    \end{equation}
    It follows that \begin{enumerate}
        \item $S_g \leq \Z$ is a subgroup for all $g \in G$.
        \item For $r,s \in \Z$, $g^r = g^s$ if and only if $r-s \in S_g$.
        \item If $S_g \neq \{0\}$, then $S_g = n\Z$ for some $n >0$, and \begin{equation}
            \langle g \rangle = \{e,g,g^2,...,g^{n-1}\}
        \end{equation}
        with $o(g) =n$.
        \item $S_g = \{0\}$ if and only if $o(g) = \infty$, in which case $g^m = g^n$ if an only if $m = n$.
        \item The order of $g^l$ is $\frac{n}{\gcd(l,n)}$ if $o(g) = n$.
    \end{enumerate}
\end{prop}
\begin{proof*}{}{}
    (Left to the reader)
\end{proof*}

\begin{rmk}{}{}
    If $o(g) = \infty$, $g^r = g^s$ if and only if $r = s$. If $o(g) = n < \infty$, $g^r = g^s$ if and only if $r \equiv s \mod n$.
    \begin{enumerate}
        \item[$\drsh$] $o(g)$ is the smallest integer $n > 0$ such that $g^n = e$. (If $\cancel{\exists}n > 0$ such that $g^n = e$, then $o(g) = \infty$) 
    \end{enumerate}
\end{rmk}

\begin{cor}{}{}
    For $g \in G$, $g^l$ is a generator of $\langle g \rangle$ if and only if $\gcd(l,n) = 1$, where $n = o(g)$.
\end{cor}
\begin{enumerate}
    \item[$\drsh$] If $o(g) = \infty$ then $\langle g^l \rangle = \langle g \rangle$ if and only if $l \in \{1,-1\}$
\end{enumerate}
\begin{proof*}{}{}
    (Left to the reader)
\end{proof*}

\begin{thm}{}{}
    Every subgroup of a cyclic group is itself cyclic.
    \begin{proof*}{}{}
        (Left to the reader)
    \end{proof*}
\end{thm}

\begin{eg}{}{}
    \leavevmode
    \begin{enumerate}
        \item For $2 \in (
        \Z,+)$, $o(2) = \infty$, so $\langle 2 \rangle = 2\Z$, and $|2\Z| = \infty$.
        \item For $[2] \in \Z/3\Z$, $o([2]) = 3$, so $\langle [2]\rangle = \Z/3\Z$. Indeed, $\langle [1]\rangle = \Z/3\Z$, so $$o([2]) = o(2[1]) = \frac{o([1])}{\gcd(o([1]),2)} = \frac{3}{\gcd(3,2)} = 3$$
        \item For $[2] \in \Z/4\Z$, $o([2]) = 2$, so $\langle [2] \rangle < \Z/4\Z$ is a proper subgroup.
    \end{enumerate}
\end{eg}


\begin{cor}{}{}
        Let $G = \langle g \rangle$ be cyclic. Then every subgroup of $G$ is cyclic.
\end{cor}
\begin{proof*}{}{}
        Consider the epimorphism \begin{equation}
                \map{\Z\xrightarrow{\phi_g}G}{n\mapsto g^n}
        \end{equation}
        Let $H \subseteq G$. Then $\phi^{-1}_g(H) = H' \subseteq \Z$ because of the properties of the inverse images of subgroups under group homomorphisms. Thus since $\phi_g$ is cyclic, $\phi_g(H') = H$, so $H$ is the image of a subgroup $H'$ of $\Z$. But, all subgroups of $\Z$ are cyclic. Thus $\phi_g(H')$ is cyclic since it is the image of a cyclic subgroup under a group homomorphism.
\end{proof*}


\begin{thm}{}{}
        If $G$ is a cyclic group of order $n < +\infty$, then the order of every subgroup $H$ of $G$ divides $n$. Moreover, for every divisor $q$ of $n$, there exists a unique subgroup of $G$ of order $q$.
\end{thm}
\begin{proof*}{}{}
        Let $G = \langle g \rangle$, $H \subseteq G$, and $o(g) = n$. By the previous corollary $H = \langle g^l \rangle$ for some $l \geq 0$. It follows that $|H| = \frac{n}{\gcd(n,l)}$. Thus, $$\frac{n}{\gcd(n,l)} = |H|\;\vert\;|G| = n$$ Suppose $|H| = |H'| (=\langle g^{l'}\rangle)$, so $\gcd(l,n) = \gcd(l',n)$. Note that $o(g^{\gcd(l,n)}) = \frac{n}{\gcd(n,\gcd(l,n))} = \frac{n}{\gcd(n,l)} = o(g^l)$ and since $\gcd(l,n)\;\vert\;l$ $g^l \in \langle g^{\gcd(l,n)}\rangle$. Hence, we have that \begin{equation}
                H = \langle g^l \rangle = \langle g^{\gcd(l,n)} \rangle = \langle g^{\gcd(l',n)} \rangle = \langle g^{l'} \rangle = H'
        \end{equation}
        Thus we have uniqueness, and that the order of any subgroup must divide that of the group. For existence, suppose $n = qr$ for some integer $r$. Then $H = \langle g^r\rangle$ is a subgroup of order $$|H| = o(g^r) = \frac{n}{\gcd(n,r)} = \frac{qr}{r} = q$$ satisfying existence.
\end{proof*}



\section{Dihedral Groups}

\begin{claim}{}{}
    If $S$ is a non-empty subset of $G$, then \begin{equation}
        \langle S\rangle = \{s_1^{k_1}\star s_2^{k_2} \star ... \star s_m^{k_m}: m\geq 1, s_1,s_2,...,s_m \in S,k_1,k_2,...,k_m\in\Z\}
    \end{equation}
    Or, equivalently \begin{equation}
        \langle S\rangle = \{r_1^{\alpha_1}\star r_2^{\alpha_2} \star ... \star r_n^{\alpha_n}: n\geq 1, r_1,r_2,...,r_n \in S,\alpha_1,\alpha_2,...,\alpha_n\in\{1,-1\}\}
    \end{equation}
\end{claim}

\begin{rmk}{}{}
    The $s_i$'s (and $r_i$'s) in these two descriptions need not be distinct. This is a generalization of the description of $\langle g \rangle$. This is by no means a unique way to write elements of $\langle S\rangle$.
\end{rmk}

\begin{eg}{}{}
    For $S = \{a,b\}$, \begin{enumerate}
        \item If $a \star b = b \star a$, for all $m \geq 1$, \begin{equation}
            s_1^{k_1}\star s_2^{k_2} \star ... \star s_m^{k_m} = a^k \star b^l
        \end{equation}
        \item In additive notation we get $a^k \star b^l = ka + lb$, so $\langle \{a,b\}\rangle = \Z a + \Z b$. (In general this need not be cyclic)
    \end{enumerate}
\end{eg}

\begin{rmk}{}{}
    If we don't assume $a \star b = b \star a$, then we cannot simplify a general element to the form $a^k \star b^l$ in $\langle S \rangle$.
\end{rmk}


\begin{defn}{}{}
    A polygon $X$ is \Emph{regular} if it is \Emph{equiangular} (all angles are equal in measure) and \Emph{equilateral} (all sides have the same length).
\end{defn}
\begin{enumerate}
    \item[$\drsh$] \begin{note*}{}{}
        The vertices of such a figure (if it is convex) can always be drawn on a circle called the \Emph{circumcircle}.
    \end{note*} 
\end{enumerate}

\begin{eg}{}{}
    \leavevmode
    \begin{enumerate}
        \item $3$-sides $=$ equilateral triangle: \begin{center}
            \begin{tikzpicture}[x=0.75pt,y=0.75pt,yscale=-1,xscale=1]
            %uncomment if require: \path (0,300); %set diagram left start at 0, and has height of 300
            
            %Shape: Circle [id:dp6369909468365647] 
            \draw   (66.67,160.5) .. controls (66.67,111.44) and (106.44,71.67) .. (155.5,71.67) .. controls (204.56,71.67) and (244.33,111.44) .. (244.33,160.5) .. controls (244.33,209.56) and (204.56,249.33) .. (155.5,249.33) .. controls (106.44,249.33) and (66.67,209.56) .. (66.67,160.5) -- cycle ;
            %Shape: Triangle [id:dp2358003807310185] 
            \draw   (155.83,71) -- (235.5,200) -- (75.67,200) -- cycle ;
            
            
            
            \draw [fill={rgb, 255:red, 3; green, 3; blue, 3 }  ,fill opacity=1 ]  (156.25, 71.67) circle [x radius= 2, y radius= 2]   ;
            \draw [fill={rgb, 255:red, 3; green, 3; blue, 3 }  ,fill opacity=1 ]  (235.27, 199.63) circle [x radius= 2, y radius= 2]   ;
            \draw [fill={rgb, 255:red, 3; green, 3; blue, 3 }  ,fill opacity=1 ]  (235.09, 200) circle [x radius= 2, y radius= 2]   ;
            \draw [fill={rgb, 255:red, 3; green, 3; blue, 3 }  ,fill opacity=1 ]  (75.91, 200) circle [x radius= 2, y radius= 2]   ;
            \draw [fill={rgb, 255:red, 3; green, 3; blue, 3 }  ,fill opacity=1 ]  (155.42, 71.67) circle [x radius= 2, y radius= 2]   ;
            \draw [fill={rgb, 255:red, 3; green, 3; blue, 3 }  ,fill opacity=1 ]  (75.8, 199.78) circle [x radius= 2, y radius= 2]   ;
            \end{tikzpicture}
        \end{center}
        \item $4$-sides $=$ square 
        \begin{center}
            \begin{tikzpicture}[x=0.75pt,y=0.75pt,yscale=-1,xscale=1]
            %uncomment if require: \path (0,300); %set diagram left start at 0, and has height of 300
            
            %Shape: Circle [id:dp6369909468365647] 
            \draw   (66.67,160.5) .. controls (66.67,111.44) and (106.44,71.67) .. (155.5,71.67) .. controls (204.56,71.67) and (244.33,111.44) .. (244.33,160.5) .. controls (244.33,209.56) and (204.56,249.33) .. (155.5,249.33) .. controls (106.44,249.33) and (66.67,209.56) .. (66.67,160.5) -- cycle ;
            %Shape: Square [id:dp9411752355047514] 
            \draw   (92.4,97.4) -- (218.6,97.4) -- (218.6,223.6) -- (92.4,223.6) -- cycle ;
            
            
            
            \draw [fill={rgb, 255:red, 3; green, 3; blue, 3 }  ,fill opacity=1 ]  (92.97, 97.4) circle [x radius= 2, y radius= 2]   ;
            \draw [fill={rgb, 255:red, 3; green, 3; blue, 3 }  ,fill opacity=1 ]  (218.03, 97.4) circle [x radius= 2, y radius= 2]   ;
            \draw [fill={rgb, 255:red, 3; green, 3; blue, 3 }  ,fill opacity=1 ]  (218.6, 97.97) circle [x radius= 2, y radius= 2]   ;
            \draw [fill={rgb, 255:red, 3; green, 3; blue, 3 }  ,fill opacity=1 ]  (218.6, 223.03) circle [x radius= 2, y radius= 2]   ;
            \draw [fill={rgb, 255:red, 3; green, 3; blue, 3 }  ,fill opacity=1 ]  (218.03, 223.6) circle [x radius= 2, y radius= 2]   ;
            \draw [fill={rgb, 255:red, 3; green, 3; blue, 3 }  ,fill opacity=1 ]  (92.97, 223.6) circle [x radius= 2, y radius= 2]   ;
            \draw [fill={rgb, 255:red, 3; green, 3; blue, 3 }  ,fill opacity=1 ]  (92.4, 97.97) circle [x radius= 2, y radius= 2]   ;
            \draw [fill={rgb, 255:red, 3; green, 3; blue, 3 }  ,fill opacity=1 ]  (92.4, 223.03) circle [x radius= 2, y radius= 2]   ;
            \end{tikzpicture}
        \end{center}
        \item $5$-sides $=$ pentagon
        \begin{center}
            \begin{tikzpicture}[x=0.75pt,y=0.75pt,yscale=-1,xscale=1]
            %uncomment if require: \path (0,300); %set diagram left start at 0, and has height of 300
            
            %Shape: Circle [id:dp6369909468365647] 
            \draw   (66.67,160.5) .. controls (66.67,111.44) and (106.44,71.67) .. (155.5,71.67) .. controls (204.56,71.67) and (244.33,111.44) .. (244.33,160.5) .. controls (244.33,209.56) and (204.56,249.33) .. (155.5,249.33) .. controls (106.44,249.33) and (66.67,209.56) .. (66.67,160.5) -- cycle ;
            %Shape: Regular Polygon [id:dp6960584174528508] 
            \draw   (239.03,131.12) -- (209.26,230.86) -- (105.19,233.37) -- (70.65,135.17) -- (153.37,71.98) -- cycle ;
            
            
            
            
            \end{tikzpicture}
        \end{center}
        \item $n$-sides $=$ regular convex $n$-gon
        \begin{center}
            \begin{tikzpicture}[x=0.75pt,y=0.75pt,yscale=-1,xscale=1]
            %uncomment if require: \path (0,300); %set diagram left start at 0, and has height of 300
            
            %Shape: Circle [id:dp6369909468365647] 
            \draw   (66.67,160.5) .. controls (66.67,111.44) and (106.44,71.67) .. (155.5,71.67) .. controls (204.56,71.67) and (244.33,111.44) .. (244.33,160.5) .. controls (244.33,209.56) and (204.56,249.33) .. (155.5,249.33) .. controls (106.44,249.33) and (66.67,209.56) .. (66.67,160.5) -- cycle ;
            %Shape: Regular Polygon [id:dp6960584174528508] 
            \draw   (239.03,131.12) -- (240.35,185.83) -- (209.26,230.86) -- (157.63,249.02) -- (105.19,233.37) -- (71.97,189.88) -- (70.65,135.17) -- (101.74,90.14) -- (153.37,71.98) -- (205.81,87.63) -- cycle ;
            
            
            
            
            \end{tikzpicture}
        \end{center}
    \end{enumerate}
\end{eg}


\begin{defn}{}{}
    Let $n \geq 3$. The \Emph{dihedral group} $D_n$ is the \Emph{symmetry group} of the regular convex $n$-gon. Consider the following maps $\R^2 \rightarrow \R^2$:\begin{enumerate}
        \item For all $\alpha \in \R$, $\phi_{\alpha}$ is the rotation about the origin of $|R^2$ of angle $\alpha$ radians.
        \item For all $\alpha \in \R$, $\psi_{\alpha}$ is the reflection with respect to a line $l_{\alpha}$ going through the origin and forming an angle $\alpha$ radians with the $x$-axis:
        \begin{center}
            \begin{tikzpicture}[x=0.75pt,y=0.75pt,yscale=-1,xscale=1]
            %uncomment if require: \path (0,300); %set diagram left start at 0, and has height of 300
            
            \draw   (-1,170.8) -- (277.83,170.8)(138.42,89.2) -- (138.42,252.4) ;
            %Straight Lines [id:da1735016350905001] 
            \draw    (35.2,212.4) -- (245.35,125.56) ;
            \draw [shift={(247.2,124.8)}, rotate = 517.55] [color={rgb, 255:red, 0; green, 0; blue, 0 }  ][line width=0.75]    (10.93,-3.29) .. controls (6.95,-1.4) and (3.31,-0.3) .. (0,0) .. controls (3.31,0.3) and (6.95,1.4) .. (10.93,3.29)   ;
            %Shape: Arc [id:dp5422827964833219] 
            \draw  [draw opacity=0] (161.34,160.12) .. controls (163.99,162.87) and (165.6,166.46) .. (165.6,170.4) .. controls (165.6,170.48) and (165.6,170.55) .. (165.6,170.63) -- (148.4,170.4) -- cycle ; \draw   (161.34,160.12) .. controls (163.99,162.87) and (165.6,166.46) .. (165.6,170.4) .. controls (165.6,170.48) and (165.6,170.55) .. (165.6,170.63) ;
            \draw   (273.83,166.67) -- (278.17,170.58) -- (273.83,174.5) ;
            \draw   (134.42,93.1) -- (138.33,88.75) -- (142.26,93.07) ;
            
            % Text Node
            \draw (237.83,113.73) node [anchor=north west][inner sep=0.75pt]  [font=\tiny]  {$l_{\alpha }$};
            % Text Node
            \draw (169.5,160.4) node [anchor=north west][inner sep=0.75pt]  [font=\tiny]  {$\alpha $};
            
            
            \end{tikzpicture}
        \end{center}
    \end{enumerate}
\end{defn}

\begin{claim}{}{}
    For all $\alpha,\beta \in \R$, \begin{enumerate}
        \item $\phi_{\alpha} \circ \phi_{\beta} = \phi_{\alpha + \beta}$ (rotation) 
        \item $\psi_{\alpha} \circ \psi_{\beta} = \phi_{2(\alpha - \beta)}$ (rotation) 
        \item $\phi_{\alpha} \circ \psi_{\beta} = \psi_{\beta + \frac{1}{2}\alpha}$ (reflection) 
        \item $\psi_{\beta} \circ \phi_{\alpha} = \psi_{\beta - \frac{1}{2}\alpha}$ (reflection) 
    \end{enumerate}
    \begin{enumerate}
        \item[$\drsh$] (Where $\circ$ is the composition of maps $\R^2\rightarrow \R^2$). 
    \end{enumerate}
\end{claim}

\begin{note*}{}{}
    This is \underline{not} commutative.
\end{note*}


\begin{cor}{}{}
    The composition of maps induces a binary operation on the set of rigid motions and reflections about the origin, $\R^2\rightarrow \R^2$.
\end{cor}

\begin{prop}{}{}
    This set is a group with \begin{equation}
        \phi_{\alpha}^{-1} = \phi_{-\alpha}, \psi^{-1}_{\beta} = \psi_{\beta}, \;\text{and}\;\phi_0 = \id
    \end{equation}
    for all $\alpha,\beta \in \R$. Moreover, since $\circ$ is associative, it is indeed a group.
\end{prop}

\begin{defn}{}{}
    This group is called the \Emph{orthogonal group}, $O_2(\R)$.
\end{defn}

\begin{prop}{}{}
    Note that $O_2(\R) \leq \GL_2(\R)$ since the maps are linear. Indeed: \begin{equation}
        \phi_{\alpha} = \begin{bmatrix} \cos\alpha & -\sin\alpha \\ \sin\alpha & \cos\alpha \end{bmatrix}, \psi_{\alpha} = \begin{bmatrix} \cos2\alpha & \sin2\alpha \\ \sin2\alpha & -\cos2\alpha \end{bmatrix}, 
    \end{equation}
    in the standard basis of $\R^2$.
\end{prop}

\begin{defn}{}{}
    Let $N \geq 3$, the $n$-th \Emph{dihedral group $D_n$} is the subgroup of $O_2(\R)$ preserving a regular $n$-gon $X_n$ with a circumcircle centered at the origin of $\R^2$:\begin{equation}
        D_n := \{f \in O_2(\R): f(X_n) = X_n\}
    \end{equation}
\end{defn}

\begin{eg}{}{}
    For $n =6$ we have \begin{center}
        \begin{tikzpicture}[x=0.75pt,y=0.75pt,yscale=-1,xscale=1]
        %uncomment if require: \path (0,300); %set diagram left start at 0, and has height of 300
        
        %Shape: Circle [id:dp6369909468365647] 
        \draw   (66.67,160.5) .. controls (66.67,111.44) and (106.44,71.67) .. (155.5,71.67) .. controls (204.56,71.67) and (244.33,111.44) .. (244.33,160.5) .. controls (244.33,209.56) and (204.56,249.33) .. (155.5,249.33) .. controls (106.44,249.33) and (66.67,209.56) .. (66.67,160.5) -- cycle ;
        %Shape: Regular Polygon [id:dp6960584174528508] 
        \draw   (243.76,160.65) -- (199.42,237.3) -- (110.87,237.22) -- (66.67,160.5) -- (111.01,83.85) -- (199.55,83.93) -- cycle ;
        \draw   (49,161.13) -- (261.5,161.13)(155.25,65) -- (155.25,257.25) ;
        %Straight Lines [id:da00457972531911599] 
        \draw    (57,102.75) -- (259.5,222.5) ;
        %Straight Lines [id:da41124966975132526] 
        \draw    (49.5,227.25) -- (251,101.75) ;
        %Straight Lines [id:da32514486377192964] 
        \draw    (95.5,56.75) -- (219.5,271.25) ;
        %Straight Lines [id:da39830526708721803] 
        \draw    (211.5,65.75) -- (93,267.25) ;
        %Shape: Arc [id:dp7870194689907395] 
        \draw  [draw opacity=0] (183.31,146.13) .. controls (186.15,150.43) and (187.91,155.5) .. (188.2,160.96) -- (158.25,162.63) -- cycle ; \draw   (183.31,146.13) .. controls (186.15,150.43) and (187.91,155.5) .. (188.2,160.96) ;
        \draw   (183.75,151.63) -- (182.82,146.15) -- (188.37,145.84) ;
        
        % Text Node
        \draw (191.5,148.4) node [anchor=north west][inner sep=0.75pt]  [font=\tiny]  {$\alpha $};
        
        \draw [fill={rgb, 255:red, 3; green, 3; blue, 3 }  ,fill opacity=1 ]  (200.75, 84.04) circle [x radius= 2, y radius= 2]   ;
        \draw [fill={rgb, 255:red, 3; green, 3; blue, 3 }  ,fill opacity=1 ]  (110.67, 237.21) circle [x radius= 2, y radius= 2]   ;
        \draw [fill={rgb, 255:red, 3; green, 3; blue, 3 }  ,fill opacity=1 ]  (110.77, 237.04) circle [x radius= 2, y radius= 2]   ;
        \draw [fill={rgb, 255:red, 3; green, 3; blue, 3 }  ,fill opacity=1 ]  (200.17, 85.01) circle [x radius= 2, y radius= 2]   ;
        \draw [fill={rgb, 255:red, 3; green, 3; blue, 3 }  ,fill opacity=1 ]  (155.41, 161.13) circle [x radius= 2, y radius= 2]   ;
        \draw [fill={rgb, 255:red, 3; green, 3; blue, 3 }  ,fill opacity=1 ]  (155.25, 161.4) circle [x radius= 2, y radius= 2]   ;
        \draw [fill={rgb, 255:red, 3; green, 3; blue, 3 }  ,fill opacity=1 ]  (155.49, 160.99) circle [x radius= 2, y radius= 2]   ;
        \draw [fill={rgb, 255:red, 3; green, 3; blue, 3 }  ,fill opacity=1 ]  (155.26, 161.38) circle [x radius= 2, y radius= 2]   ;
        \draw [fill={rgb, 255:red, 3; green, 3; blue, 3 }  ,fill opacity=1 ]  (155.63, 160.76) circle [x radius= 2, y radius= 2]   ;
        \draw [fill={rgb, 255:red, 3; green, 3; blue, 3 }  ,fill opacity=1 ]  (111.02, 83.59) circle [x radius= 2, y radius= 2]   ;
        \draw [fill={rgb, 255:red, 3; green, 3; blue, 3 }  ,fill opacity=1 ]  (199.95, 237.43) circle [x radius= 2, y radius= 2]   ;
        \draw [fill={rgb, 255:red, 3; green, 3; blue, 3 }  ,fill opacity=1 ]  (199.65, 236.91) circle [x radius= 2, y radius= 2]   ;
        \draw [fill={rgb, 255:red, 3; green, 3; blue, 3 }  ,fill opacity=1 ]  (111.17, 83.85) circle [x radius= 2, y radius= 2]   ;
        \draw [fill={rgb, 255:red, 3; green, 3; blue, 3 }  ,fill opacity=1 ]  (155.84, 161.13) circle [x radius= 2, y radius= 2]   ;
        \draw [fill={rgb, 255:red, 3; green, 3; blue, 3 }  ,fill opacity=1 ]  (155.25, 160.11) circle [x radius= 2, y radius= 2]   ;
        \draw [fill={rgb, 255:red, 3; green, 3; blue, 3 }  ,fill opacity=1 ]  (155.9, 161.24) circle [x radius= 2, y radius= 2]   ;
        \draw [fill={rgb, 255:red, 3; green, 3; blue, 3 }  ,fill opacity=1 ]  (155.79, 161.05) circle [x radius= 2, y radius= 2]   ;
        \end{tikzpicture}
    \end{center}
    We have $6$-rotations $$\{\phi_{2\pi/6},\phi_{4\pi/6},\phi_{6\pi/6},\phi_{8\pi/6},\phi_{10\pi/6},\phi_{12\pi/6}\}$$ and $6$-reflections $$\{\psi_{\pi/6},\psi_{2\pi/6},\psi_{3\pi/6},\psi_{4\pi/6},\psi_{5\pi/6},\psi_{6\pi/6}\}$$
\end{eg}

\begin{cor}{}{}
    The order $|D_n|$ is $2n$ ($n$-rotations and $n$-reflections). In particular \begin{equation}
        D_n = \left\langle \left\{\phi_{2\pi/n},\psi_0\right\}\right\rangle
    \end{equation}
\end{cor}

\begin{rmk}{}{}
    $D_n \leq O_2(\R) \leq \GL_2(\R)$
\end{rmk}

\begin{defn}{Algebraic Description}{}
    $D_n$ is the group of order $2n$ generated by $2$ elements $x,y$ satisfying the \Emph{relations} \begin{equation}
        x^n = e, y^2 = e, yx = x^{-1}y
    \end{equation}
    (they imply $xyx = y$, and $x^kyx^k = y$ for all $k \in \Z$)
\end{defn}

\begin{rmk}{}{}
    The elements of $D_n$ are \begin{equation}
        D_n = \{e,x,x^2,...,x^{n-1},y,yx,yx^2,...,yx^{n-1}\}
    \end{equation}
    (This is closed under the multiplication using the relations above). In fact, $(x^ky)^{-1} = x^ky$ because $x^kyx^k = y$ implies $x^ky = yx^{-k}$.
\end{rmk}

\begin{rmk}{}{}
    Let $Y \subset X$. Then \begin{equation}
        \{f\in S_X:f(Y) = Y\} \leq S_X
    \end{equation}
    where $S_X$ is the group of symmetries of the set $X$.
\end{rmk}


\begin{rmk}{}{}
    For the orthogonal group $O_2(\R)$ we have the subgroup \begin{equation}
        D_n = \{f\in O_2(\R) \leq S_{\R^2}:f(X_n) = X_n\} \leq O_2(\R)
    \end{equation}
    which is the group of symmetries of the regular convex $n$-gon, $X_n$, denoted by $D_n$.
\end{rmk}

\section{Lattice Subgroups of a Group}

\begin{cons}{}{}
    Given a finite group $G$, we plot subgroups of $G$ with $\{e\}$ at the bottom and $G$ at the top. We draw paths upward between subgroups using the rule that an upward line connects a subgroup $A$ to a subgroup $B$ if and only if $A \leq B$, and there are no subgroups properly between $A$ and $B$.
\end{cons}

\begin{rmk}{}{}
    If $G \cong H$, then $G$ and $H$ have the same lattice structure. That is, group isomorphism induces a one-to-one correspondence between subgroups preserving containment.
\end{rmk}

\begin{eg}{}{}
    \leavevmode
    \begin{enumerate}
        \item For $G = \Z/n\Z$ the lattice of subgroups is the lattice of divisors of $\Z$. For instance: 
        \begin{figure}[H]
            \centering
            \begin{tikzcd}
                & \Z/12\Z & & \\
                \ip{3} \arrow[ur, dash] & & \ip{2} \arrow[ul, dash] & \\
                & & & \ip{4} \arrow[ul, dash] \\
                & \ip{6} \arrow[uul, dash] \arrow[uur, dash] & & \\
                & & \ip{12} = \{0\} \arrow[ul, dash] \arrow[uur, dash] &
            \end{tikzcd}
            \label{fig:Z12Lattice}
        \end{figure}
        and for a prime p:
        \begin{figure}[H]
            \centering
            \begin{tikzcd}
                \Z/p^n\Z = \ip{1} \\
                \ip{p} \arrow[u, dash] \\
                \ip{p^2} \arrow[u, dash] \\
                \vdots \arrow[u, dash] \\
                \ip{p^{n-1}} \arrow[u, dash] \\
                \ip{p^n} = \{0\} \arrow[u, dash] \\
            \end{tikzcd}
            \label{fig:ZpLattice}
        \end{figure}
        \item The Klien $4-$group, $V_4 = \langle a,b,c: a^2=b^2=c^2=1\rangle$:
        \begin{figure}[H]
            \centering
            \begin{tikzcd}
                &V_4& \\
                \ip{a} \arrow[ur, dash] & \ip{b} \arrow[u, dash] & \ip{c} \arrow[ul, dash] \\
                &1 \arrow[ul, dash] \arrow[u, dash] \arrow[ur, dash]& \\
            \end{tikzcd}
            \label{fig:V4Lattice}
        \end{figure}
        \item The symmetric group on 3-letters, $S_3$:
        \begin{figure}[H]
            \centering
            \begin{tikzcd}
                & & S_3 & \\
                & & & \ip{(1\;2\;3)} \arrow[ul, dash] \\
                \ip{(1\;2)} \arrow[uurr, dash]& \ip{(2\;3)} \arrow[uur, dash] & \ip{(1\;3)} \arrow[uu, dash] &  \\
                & & (1) \arrow[ull, dash] \arrow[ul, dash] \arrow[u, dash] \arrow[uur, dash] &
            \end{tikzcd}
            \label{fig:S3Lattice}
        \end{figure}
    \end{enumerate}
\end{eg}


%%%%%%%%%%%%%%%%%%%%%% - P1.Chapter 4
\chapter{\textsection Free Groups}

\section{Basic Definitions and Examples: Free Groups}

\begin{rmk}{}{}
        The idea of a free group, $F(S)$, generated by a set $S$ is that there are no relations satisfied by the elements of $S$. ($S$ is ``free of relations")
\end{rmk}

\begin{defn}{Universal Property of Free Groups}{}
        Given any set map $\varphi$ from the set $S$ to the set underlying the group $G$, there is a unique group homomorphism \begin{equation}
                \Phi:F(S) \rightarrow G
        \end{equation}
        such that $\Phi\circ \iota = \varphi$. That is to say, the following diagram commutes: 
        \begin{center}
            \begin{tikzpicture}[baseline = (a).base]
            \node[scale = 1] (a) at (0,0){
                \begin{tikzcd}
                        S \ar[dr, "\forall\varphi", swap] \ar[r, "\iota"] & F(S) \ar[d, dashed, "\exists!\Phi"] \\
                        & \forall G
                \end{tikzcd}
            };
            \end{tikzpicture}
        \end{center}
\end{defn}

\begin{cons}{}{}
        Let $S$ be a set and let $S^{-1}$ be any set disjoint from $S$ such that there is a bijection from $S$ to $S^{-1}$. For each $s \in S$ denote its corresponding element in $S^{-1}$ by $s^{-1}$, and for each $t \in S^{-1}$ denote its corresponding element in $S$ by $t^{-1}$ (so $(s^{-1})^{-1} \in S$). 

        Take a singleton set not contained in $S \cup S^{-1}$ and call it $\{1\}$. Let $1^{-1} = 1$, and for any $x \in S\cup S^{-1}\cup\{1\}$, let $x^1 = x$.

        A \Emph{word} on $S$ is defined by a sequence \begin{equation}
                (s_1,s_2,s_3,...)
        \end{equation}
        where $s_i \in S\cup S^{-1} \cup \{1\}$ for all $i$, and there exists $N \in \N$ such that if $i \geq N$, then $s_i = 1$.

        The word $(s_1,s_2,s_3,...)$ is said to be \Emph{reduced} if \begin{enumerate}
                \item $s_{i+1} \neq s_i^{-1}$ for all $i$ with $s_i \neq 1$
                \item if $s_k = 1$ for some $k$, then $s_i = 1$ for all $i \geq k$
        \end{enumerate}
        The reduced word $(1,1,1,...)$ is called the \Emph{empty word} and is denoted by $1$. We write the reduced word $(s_1^{\varepsilon_1},...,s_n^{\varepsilon_n},1,1,...)$ with $\varepsilon_i = \pm 1$ as $s_1^{\varepsilon_1}...s_n^{\varepsilon_n}$. Note by definition reduced words \begin{equation}
                r_1^{\delta_1}...r_m^{\delta_m}\;and\;s_1^{\varepsilon_1}...s_n^{\varepsilon_n}
        \end{equation}
        are equal if and only if $n = m$ and $\delta_i = \varepsilon_i$ for all $1\leq i \leq n$.

        Let $F(S)$ be the set of reduced words on $S$ and embed $S$ into $F(S)$ by \begin{equation}
                s \mapsto (s,1,1,...)
        \end{equation}
        Note if $S = \emptyset$, $F(S) = \{1\}$.
        
        \Emph{Operation:} Let $r_1^{\delta_1}...r_m^{\delta_m}$ and $s_1^{\varepsilon_1}...s_n^{\varepsilon_n}$ be reduced words, and assume without loss of generality that $m \leq n$. Let $k$ be the smallest integer in the range $1 \leq k \leq m+1$ such that \begin{equation}
                s_k^{\varepsilon_k} \neq r_{m-k+1}^{-\delta_{m-k+1}}
        \end{equation}
        Then the product of these reduced words is defined as \begin{equation}
                (r_1^{\delta_1}...r_m^{\delta_m})(s_1^{\varepsilon_1}...s_n^{\varepsilon_n}) := \left\{\begin{array}{ll}
                        r_1^{\delta_1}...r_{m-k+1}^{\delta_{m-k+1}}s_k^{\varepsilon_k}...s_n^{\varepsilon_n}, & \text{if $k \leq m$} \\
                        s_{m+1}^{\varepsilon_{m+1}}...s_n^{\varepsilon_n}, & \text{if $k = m+1\leq n$} \\
                        1, & \text{if $k = m+1$ and $m = n$}
                \end{array}\right.
        \end{equation}
\end{cons}


\begin{thm}{}{}
        $F(S)$ is a group under the above binary operation.
\end{thm}
\begin{proof*}{}{}
        By construction we note that $1$ is an identity element of the binary operation, and that the inverse of a reduced word $s_1^{\varepsilon_1}...s_n^{\varepsilon_n}$ is $s_n^{-\varepsilon_n}...s_1^{-\varepsilon_1}$. For each $s \in S\cup S^{-1} \cup\{1\}$ define a map $\sigma_s:F(S)\rightarrow F(S)$ by \begin{equation}
                \sigma_s(s_1^{\varepsilon_1}...s_n^{\varepsilon_n}) = \left\{\begin{array}{ll} 
                        s\cdot s_1^{\varepsilon_1}...s_n^{\varepsilon_n} & \text{if $s_1^{\varepsilon} \neq s^{-1}$} \\
                        s_2^{\varepsilon_2}...s_n^{\varepsilon_n} & \text{if $s_1^{\varepsilon_1} = s^{-1}$}
                \end{array}\right.
        \end{equation}
        Since $\sigma_{s^{-1}}\circ \sigma_s$ is the identity map on $F(S)$, $\sigma_s$ is a permutation of $F(S)$.Let $A(F)$ be the subgroup of the symmetric group on $F(S)$ generated by $\{\sigma_s:s\in S\}$. We observe that the map \begin{equation}
                s_1^{\varepsilon_1}...s_n^{\varepsilon_n} \mapsto \sigma_{s_1}^{\varepsilon_1}\circ ... \circ \sigma_{s_n}^{\varepsilon_n}
        \end{equation}
        is a set bijection between $F(S)$ and $A(F)$ which respects their binary operation. Since $A(F)$ is a group, and hence its operation is associative, so is $F(S)$.
\end{proof*}

\begin{namthm*}{Universal Property of Free Groups}{}
        Let $G$ be a group, $S$ a set, and $S\xrightarrow{\varphi}G$ a set map. Then there exists a unique group homomorphism $\Phi:F(S)\rightarrow G$ such that the diagram commutes 
        \begin{center}
            \begin{tikzpicture}[baseline = (a).base]
            \node[scale = 1] (a) at (0,0){
                \begin{tikzcd}
                        S \ar[dr, "\varphi", swap] \ar[r, "\iota"] & F(S) \ar[d, dashed, "\exists!\Phi"] \\
                        & G
                \end{tikzcd}
            };
            \end{tikzpicture}
        \end{center}
\end{namthm*}
\begin{proof*}{}{}
        Choose $\Phi:s_1^{\varepsilon_1}...s_n^{\varepsilon_n}\mapsto \varphi(s_1)^{\varepsilon_1}...\varphi(s_n)^{\varepsilon_n}$
\end{proof*}


\begin{cor}{}{}
        The free group $F(S)$ is unique up to an isomorphism which is an identity on the set $S$.
\end{cor}
\begin{proof*}{}{}
        Suppose $F(S)$ and $F'(S)$ are two free groups generated by $S$. Since $S$ is contained in both $F(S)$ and $F'(S)$ we have natural injections \begin{equation}
                S\xhookrightarrow{\iota}F(S),S\xhookrightarrow{\iota'}F'(S)
        \end{equation}
        By the universal property of Free groups there exist unique group homomorphisms $\Phi:F(S)\rightarrow F'(S)$ and $\Phi':F'(S)\rightarrow F(S)$ such that $\Phi \circ \iota = \iota'$ and $\Phi'\circ \iota' = \iota$, which are both the identity on $S$. Then, $\Phi'\circ \Phi$ is a map which makes the diagram \begin{center}
            \begin{tikzpicture}[baseline = (a).base]
            \node[scale = 1] (a) at (0,0){
                \begin{tikzcd}
                        S \ar[dr, "\iota", swap] \ar[r, "\iota"] & F(S) \ar[d, dashed, "?"] \\
                        & F(S)
                \end{tikzcd}
            };
            \end{tikzpicture}
        \end{center}
        commute. But, $\id_{F(S)}$ also makes this commute, so by uniqueness $\Phi'\circ \Phi = \id_{F(S)}$. Similarly, $\Phi\circ \Phi' = \id_{F'(S)}$, so $\Phi$ and $\Phi'$ are inverses, and hence bijections. Thus, $\Phi$ and $\Phi'$ are isomorphisms which are the identity on $S$, so $F(S)\cong F'(S)$ as claimed.
\end{proof*}

\begin{defn}{Free Group}{}
        The group $F(S)$ is called the \Emph{free group} on the set $S$. A group $F$ is a free group if there is some set $S$ such that $F \cong F(S)$. In this case we call $S$ a set of \Emph{free generators} or a \Emph{free basis} of $F$. The cardinality of $S$ is called the \Emph{rank} of the free group $F$.
\end{defn}

\begin{thm}{Schreier}{}
        Subgroups of a free group are themselves free.
\end{thm}





        



\section{Presentations}


\begin{rmk}{}{}
        For a group $G$, $G$ is a homomorphic image of a free group. Take $S = G$ and $\varphi$ as the identity map from $G$ to $G$. Then by the universal property of free groups there is a surjective group homomorphism from $F(G)$ onto $G$.
        \begin{enumerate}
                \item[$\drsh$] In general, if $S \subseteq G$ such that $G = \langle S \rangle$, then there exists a unique group epimorphism $\varphi:F(S)\twoheadrightarrow G$ which is the identity on $S$.
        \end{enumerate}
\end{rmk}

\begin{defn}{}{}
        Let $S$ be a subset of a group $G$ such that $G = \langle S \rangle$. \begin{enumerate}
                \item A \Emph{presentation} for $G$ is a pair $(S,R)$ where $R$ is a set of words in $F(S)$ such that the \Emph{normal closure} of $\langle R \rangle$ in $F(S)$ (the smallest normal subgroup containing $\langle R\rangle$) equals the kernel of the homomorphism $\pi:F(S)\rightarrow G$, where $\pi$ extends the identity map from $S$ to $S$. The elements of $S$ are called \Emph{generators} and those of $R$ are called \Emph{relations} of $G$.
                \item We say $G$ is \Emph{finitely generated} if there is a presentation $(S,R)$ such that $S$ is a finite set, and we say $G$ is \Emph{finitely presented} if there is a presentation $(S,R)$ were both $S$ and $R$ are finite sets.
        \end{enumerate}
\end{defn}




%%%%%%%%%%%%%%%%%%%%%% - P1.Chapter 5
\chapter{\textsection Quotient Groups}

\section{Cosets}

\begin{defn}{Left Cosets}{}
        Let $H \leq G$ be a subgroup, $g \in G$. The \Emph{left coset} of $H$ containing $g$, or generated by $g$, is \begin{equation}
                gH := \{g\star h:h\in H\}\subseteq G
        \end{equation}
\end{defn}
\begin{enumerate}
        \item[$\drsh$] Notation depends on the operation: $g+H, g\star H, gH, etc.$
\end{enumerate}

\begin{note*}{}{}
        For $H \leq G$ and all $g \in G$, $g = g\star e_G \in gH$ since $e_G \in H$. If $g \in H$, $gH = H$, as $h = g \star (g^{-1} \star h)$ for all $h \in H$. Hence we can have $gH = g'H$ for $g \neq g'$. Note $gH$ is only a subgroup if $g \in H$ so $gH = H$.
\end{note*}

\begin{eg}{}{}
        Take $G = \Z/4\Z$, $H = \langle [2]\rangle \leq G$. So $H = \{[0], [2]\}$. Then we have the cosets \begin{table}[H]
                \centering
                \begin{tabular}{cc}
                        $g$ & $gH = g+H$ \\
                        $[0]$ & $H$ \\
                        $[1]$ & $\{ [1],[3] \} = [1] +H$\\
                        $[2]$ & $H$ \\
                        $[3]$ & $\{[3], [1]\} = [3]+H = [1] +H$
                \end{tabular}
        \end{table}
        we have $H = [2]+H$ and $[1]+H = [3]+H$, so we get 2 distinct left cosets and they form a partition of $G$.
\end{eg}

\begin{lem}{}{}
        The left cosets of $H \leq G$ form a partition of $G$.
\end{lem}
\begin{proof*}{}{}
        First, as $g \in gH$ for all $g \in G$ we have that $$\bigcup\limits_{g\in G}gH = G$$
        Then, suppose $gH\cap g'H \neq \emptyset$, and let $gh = g'h' \in gH\cap g'H$. Then $g = g'h'h^{-1} \in g'H$ and $g' = gh{h'}^{-1} \in gH$. Then, for all $gh'' \in gH$ and $g'\overline{h} \in g'H$, $gh'' = g'(h'h^{-1}h'') \in g'H$ and $g'\overline{h} = g(h{h'}^{-1}\overline{h}) \in gH$, so $g'H \supseteq gH$ and $g'H \subseteq gH$. Thus $g'H = gH$, so by proof by contrapositive we have that all distinct cosets are disjoint, completing the proof.
\end{proof*}

\begin{rmk}{}{}
        The left cosets of $H$ are the equivalence classes for the equivalence relation \begin{equation}
                a \equiv \iff a^{-1}b \in H
        \end{equation}
\end{rmk}

\begin{defn}{Index}{}
        The number of left cosets of $H$ in $G$ is called the \Emph{index} of $H$ in $G$, denoted by \begin{equation}
                |G:H|\;or\;[G:H]
        \end{equation}
\end{defn}


\begin{lem}{}{}
        For all $g \in G$, $G$ a group, and all $H \leq G$, $|gH| = |H|$.
\end{lem}
\begin{proof*}{}{}
        Define the map $\phi:H\rightarrow gH$ by $h \mapsto gh$. Then $\phi$ is a bijection with inverse $\varphi:gH\rightarrow H$ defined by $gh \mapsto g^{-1}gh$. Indeed, we have $\varphi \circ \phi(h) = \varphi(gh) = g^{-1}gh = h$. Thus, by definition of cardinality of sets $|H| = |gH|$.
\end{proof*}



\begin{defn}{}{}
        Let $G$ be a group with $H \leq G$. $g \in G$ is a \Emph{representative} of a left coset $xH$ of $H$ in $G$ if and only if $g \in xH$ ($\iff gH = xH$). A \Emph{complete set of representatives} of the left cosets of $H$ in $G$ is a set $S \subseteq G$ such that $S$ contains one, and only one, representative of each left coset of $H$ in $G$.
\end{defn}

\begin{eg}{}{}
        \leavevmode
        \begin{enumerate}
                \item For $n \geq 1$, and $H = n\Z \leq \Z =G$, a number $m \in \Z$ is a representative of $a + n\Z$ if and only if $n\;\vert\;m-a$. A complete set of representatives would be $\{0,1,2,...,n-1\}$ of the left cosets of $n\Z$ in $\Z$.
                \item Consider $H = \{g\in\C^{\times}:|g| = 1\}\leq \C^{\times}$.
                        \begin{center}
							\begin{tikzpicture}[x=0.75pt,y=0.75pt,yscale=-1,xscale=1]
								%uncomment if require: \path (0,310); %set diagram left start at 0, and has height of 310

								%Straight Lines [id:da5661096376446424] 
								\draw    (90,160.2) -- (290.4,160.2) ;
								%Straight Lines [id:da14833488359267633] 
								\draw    (189.2,79.8) -- (189.6,240.6) ;
								%Shape: Circle [id:dp40139760849871564] 
								\draw   (164.4,160.2) .. controls (164.4,146.39) and (175.59,135.2) .. (189.4,135.2) .. controls (203.21,135.2) and (214.4,146.39) .. (214.4,160.2) .. controls (214.4,174.01) and (203.21,185.2) .. (189.4,185.2) .. controls (175.59,185.2) and (164.4,174.01) .. (164.4,160.2) -- cycle ;
								%Shape: Circle [id:dp7821381105807637] 
								\draw   (144.3,160.2) .. controls (144.3,135.29) and (164.49,115.1) .. (189.4,115.1) .. controls (214.31,115.1) and (234.5,135.29) .. (234.5,160.2) .. controls (234.5,185.11) and (214.31,205.3) .. (189.4,205.3) .. controls (164.49,205.3) and (144.3,185.11) .. (144.3,160.2) -- cycle ;
								%Shape: Circle [id:dp9798232031691783] 
								\draw   (125.5,160.2) .. controls (125.5,124.91) and (154.11,96.3) .. (189.4,96.3) .. controls (224.69,96.3) and (253.3,124.91) .. (253.3,160.2) .. controls (253.3,195.49) and (224.69,224.1) .. (189.4,224.1) .. controls (154.11,224.1) and (125.5,195.49) .. (125.5,160.2) -- cycle ;

								% Text Node
								\draw (173.8,140.2) node [anchor=north west][inner sep=0.75pt]  [font=\tiny]  {$\frac{1}{2} H$};
								% Text Node
								\draw (152.6,138.2) node [anchor=north west][inner sep=0.75pt]  [font=\tiny]  {$H$};
								% Text Node
								\draw (135.8,127.8) node [anchor=north west][inner sep=0.75pt]  [font=\tiny]  {$\frac{3}{2} H$};


							\end{tikzpicture}
						\end{center}
				A complete set of representatives is $\R_{>0}$. Indeed, the map \begin{equation}
					\map{\R_{>0}\xrightarrow{f}\{\text{left cosets of $H$ in $\C^{\times}$}\}}{r\mapsto rH}
				\end{equation}
				Indeed, for all $x \in \C^{\times}$ $x = re^{i\theta}$ so $xH = re^{i\theta}H = rH$, as $e^{i\theta} \in H$, so $f$ is surjective. If $r,r' \in rH$ then $r' = rz$ with $|z| = 1$, so $|r'| = |rz| = |r|$. But, for $r,r' \in \R_{>0}$ $|r'| = |r|$ implies $r' = r$. Thus, $f$ is a bijection.
			\item Consider $H = \R\leq \C$. Then $\R$ identified with the y axis is a complete set of representative under the map \begin{equation}
				\map{\R\mapsto\{\text{left cosets of $H$}\}}{r\mapsto ir+R}
			\end{equation}
			\item Consider $\R^{\times} \leq \C^{\times}$. Then $[0,\pi[$ is a complete set of representatives, with the map \begin{equation}
				\map{{[0,\pi[}\rightarrow\{\text{left cosets of $\R^{\times}$}\}}{\theta \mapsto e^{i\theta}\R^{\times}}
			\end{equation}
		\end{enumerate}
\end{eg}

\begin{defn}{}{}
        For a subgroup $H \leq G$, a \Emph{right coset} of $H$ in $G$ is a subset of $G$ of the form \begin{equation}
                Hg := \{hg:h\in H\}
        \end{equation}
        We say that $Hg$ is the right coset generated by $g$ or containing $g$.
\end{defn}

\begin{thm}{}{}
        The right cosets of $H$ in $G$ form a partition of $G$.
\end{thm}
\begin{proof*}{}{}
        (Left to the reader)
\end{proof*}


\begin{rmk}{}{}
        The right cosets of $H$ in $G$ are the equivalence classes of the equivalence relation \begin{equation}
                a\equiv b \iff ba^{-1} \in H
        \end{equation}
\end{rmk}

\begin{prop}{}{}
        There is a bijection \begin{equation}
                \map{\{\text{left cosets of $H$ in $G$}\}\rightarrow \{\text{right cosets of $H$ in $G$}\}}{aH \mapsto Ha^{-1}}
        \end{equation}
\end{prop}
\begin{proof*}{}{}
        (Left to the reader)
\end{proof*}

\begin{cor}{}{}
        The number of right cosets of $H$ in $G$ is also $|G:H|$, the index.
\end{cor}


\begin{rmk}{}{}
        When $G$ is abelian, every right coset $Hg$ is a left coset $gH$.
\end{rmk}

\begin{eg}{Non-example}{}
        For $H = \langle y \rangle \leq D_3 = \langle x,y\rangle$, the left and right cosets give two different partitions of $D_3$. Indeed, $H = yH$, $xH = xyH$, and $x^2H = x^2yH$ are the left cosets, while $H = yH$, $Hx = Hx^2y$, $Hx^2 = Hxy$ are the right cosets.
\end{eg}


\section{Lagrange's Theorem and Applications}

\begin{namthm*}{Lagrange's Theorem}{}
        If $H$ is a subgroup of a finite group $G$, then $|H|$ divides $|G|$.
\end{namthm*}
\begin{proof*}{}{}
        Let $H\leq G$ for $G$ finite. Then since the cosets of $H$ partition $G$ we have that \begin{equation}
                |G| = \sum\limits_{cosets}|gH| = \sum\limits_{cosets}|H| = |G:H||H|
        \end{equation}
        so by definition $|H|$ divides $|G|$.
\end{proof*}


\begin{eg}{}{}
        \leavevmode
        \begin{enumerate}
                \item Let $g \in G$ (a finite group). Then $o(g)\;\vert\;|G|$ and thus $g^{|G|} = e_G$.
                        \begin{proof*}{}{}
                                (Left to the reader)
                        \end{proof*}
                \item If $|G| = p$ is prime, then $G$ has only the trivial subgroups, $H = \{e_G\}$ and $H = G$, since $|H|\;\vert\;|G|$ implies $|H| \in \{1,p\}$. Actually:
                        \begin{cor}{}{}
                                If $|G| = p$ a prime, then $G \cong \Z/p\Z$.
                        \end{cor}
                \item If $|G| = p^2$ for a prime $p$ then either $G \cong \Z/p^2\Z$ or $g^p = e$ for all $g \in G$.
                        \begin{proof*}{}{}
                                (Left to the reader)
                        \end{proof*}
        \end{enumerate}
\end{eg}


\begin{rmk}{}{}
        A class $[m] \in \Z/n\Z$ has a multiplicative inverse if and only if $\gcd(m,n) = 1$ (they are \Emph{relatively prime}).
\end{rmk}
\begin{proof*}{}{}
        If $\gcd(m,n) = 1$ then there exist $a,b \in \Z$ such that $am + bn = 1$, so $[a][m] = [1]$ modulo $n$. On the other hand, if $[a][m] = [1]$ for some $m \in \Z$ then there exists $b \in \Z$ such that $1 = am+bn$. Hence, $m\Z + n\Z = \Z$, so $\gcd(m,n) = 1$.
\end{proof*}

\begin{defn}{}{}
        Fix an integer $n > 1$. Let $(\Z/n\Z)^{\times}$ be the set of these classes with multiplicative inverses from $\Z/n\Z$ with multiplication. Then, it is a group with identity $[1]$.
\end{defn}

\begin{defn}{Euler Totient Function}{}
        Let $\varphi(n) :=|(\Z/n\Z)^{\times}|$ (the \Emph{Euler totient function}), or equivalently \begin{equation}
                \varphi(n) := |\{m \in \{0,1,...,n-1\}:\gcd(m,n) = 1\}|
        \end{equation}
        This is also known as the \Emph{Euler phi function}.
\end{defn}

\begin{eg}{}{}
        Take $p$ a prime. Then \begin{enumerate}
                \item $\varphi(p) = p-1$
                \item $\varphi(p^k) = p^k - p^{k-1}$ for all $k \geq 1$
        \end{enumerate}
\end{eg}


\begin{namthm*}{Euler's Theorem}{}
        If $a$ and $n \geq 2$ are relatively prime integers, then \begin{equation}
                a^{\varphi(n)} \equiv 1 \mod n
        \end{equation}
\end{namthm*}
\begin{proof*}{}{}
        (Left to the reader)
\end{proof*}

\begin{namthm*}{Fermat's Theorem}{}
        If $p$ is a prime then $a^p = a \mod p$ for all $a \in \Z$.
\end{namthm*}
\begin{proof*}{}{}
        (Left to the reader)
\end{proof*}

\subsection{Classification of Groups of Order 2p for p a prime}

\begin{thm}{}{}
        Let $G$ be a group. If $|G| = 2p$, then either $G$ is cyclic $(\cong \Z/2p\Z)$ or $G$ is isomorphic to the dihedral group $D_p$ of order $2p$.
\end{thm}
\begin{proof*}{}{}
        This proof extends over this subsection and will be completed after stating a few lemmas
\end{proof*}


\begin{lem}{}{}
        If $G$ is a group in which $g^2 = e_G$ for all $g \in G$, then $G$ is abelian.
\end{lem}
\begin{proof*}{}{}
        Let $x,y \in G$. Then $(xy)^2 = e$ since $xy \in G$, so $xy = (xy)^{-1} = y^{-1}x^{-1} = yx$. Thus $G$ is abelian as claimed.
\end{proof*}


\begin{prooflab}{Proof of Theorem for p = 2}{}
        First, suppose $|G| = 2\cdot 2 = 2^2 = p^2$. Then, by application of Lagrange's Theorem we have $G$ is cyclic or $g^p = g^2 = e_G$ for all $g \in G$. If $G$ is cyclic $G \cong \Z/4\Z$ and we're done. If $G$ is not cyclic, $G = \{e_G,g_2,g_3,g_4\}$. Set $x = g_2$ and $y = g_3$. We have $|G| = 2n$, $x^n = e_G = y^2$ and $yx= xy = x^{-1}y$ since $G$ is abelian by the last Lemma. Thus, we have that $G \cong D_2$, the dihedral group of order $4$.
\end{prooflab}

\begin{prooflab}{Proof of Theorem for p > 2}{}
        If $G$ is cyclic we're done, so suppose $G$ is not cyclic. We must show that $G \cong D_p$. By Lagrange's Theorem $o(g) \in \{1,2,p\}$ for all $g \in G$ (since $G$ is assumed to not be cyclic). 

        \begin{claim}{}{}
                $G$ has an element of order $p$.
        \end{claim}
        \begin{proof*}{}{}
                If $g^2 = e_G$ for all $g \in G$ then $G$ is abelian by the Lemma. Take three distinct elements $\{e_G, a, b\}$ in $G$, so $\{e_G,a,b,ab\} \leq G$, which is isomorphic to $D_2$. But $|D_2| = 4$ and $4 \;\cancel{\vert}\;2p$ as $p > 2$ is odd. Thus, this is not possible by Lagrange's Theorem. Hence, there must exist $x \in G$ such that $o(x) = p$.
        \end{proof*}
        Set $H = \langle x \rangle$, so $|H| = o(x) = p$.
        
        \begin{claim}{}{}
                If $g \in G$ with $g \notin H$, then $o(g) = 2$.
        \end{claim}
        \begin{proof*}{}{}
                Note $g \neq e_G$ because $e_G \in H$, so $o(g) \neq 1$. We have that $G = H\coprod gH$ because $|G| = 2p = |H| + |gH|$ and $gH \cap H =\emptyset$ since $g \notin H$ by $g \in gH$. Next, note $g^2 \in gH$ if and only if $g \in H$, so $g^2 \notin gH$, whcih implies $g^2 \in H$. If $o(g) = p$ then $g = g^{p+1} = (g^2)^{\frac{p+1}{2}} \in H$ since $p+1$ is even. But $g \notin H$, so this is a contradiction. Hence, we must have tha $o(g) = 2$.
        \end{proof*}

        Now, let $y \in G$ such that $y \notin H$, so $o(y) = 2$. Then we have $\langle x,y\rangle \geq H$ and $\langle x, y \rangle \subseteq yH$. But $G = H\coprod gH$, so $\langle x,y \rangle = G$. Thus $|\langle x,y \rangle| = 2p$. Moreover, $o(x) = p$ and $o(y) = 2$. We want $yx = x^{-1}y$. Note $yx \in yH$ by definition of the left coset, so $(yx)^2 = e_G$. Hence, $yx = (yx)^{-1} = x^{-1}y$ as $y = y^{-1}$. Therefore, $G$ satisfies the criterion for the dihedral group of order $2p$, so $G \cong D_p$.
\end{prooflab}



\section{The Alternating Group}

\begin{defn}{}{}
        Let $n \geq 1$, $A$ be an $n\times n$ matrix, and let $\sigma \in S_n$. Define an action of $S_n$ on $\GL_n(\R)$ by letting $\sigma(A)$ be the $n \times n$ matrix with the $i$-th row being the $\sigma^{-1}(i)$-th row of $A$. That is \begin{equation}
                \sigma(A)_{\sigma(i)j} = A_{ij} \;or\;\sigma(A)_{ij} = A_{\sigma^{-1}(i)j}
        \end{equation}
        so $\sigma$ sends the $i$-th row of $A$ to the $\sigma(i)$-th row.
\end{defn}

\begin{claim}{}{}
        The map defined by \begin{equation}
                \map{S_n\xrightarrow{f}\GL_n(\R)}{\sigma \mapsto \sigma(\id_n)}
        \end{equation}
        is a well defined group homomorphism.
\end{claim}
\begin{proof*}{}{}
        First, let $\sigma, \eta \in S_n$. I claim $\sigma(A) = \sigma(\id_n)A$ for all $A \in \GL_n(\R)$. Indeed, observe that \begin{align*}
                \left(\sigma(\id_n)A\right)_{ik} &= \sum\limits_{j=1}^n\sigma(\id_n)_{ij}A_{jk} \\
                &= \sum\limits_{j=1}^n\id_{\sigma^{-1}(i)j}A_{jk} \\
                &= \sum\limits_{j=1}^n\delta_{\sigma^{-1}(i)j}A_{jk} \\
                &= A_{\sigma^{-1}(i)k} \\
                &= \sigma(A)_{ik}
        \end{align*}
        where $\delta_{ij} = \left\{\begin{array}{ll} 1, & i = j \\ 0, & i \neq j \\ \end{array}\right.$ is the \Emph{kronecker delta}. Hence we have that \begin{equation}
                f(\sigma)f(\eta) = \sigma(\id_n)\eta(\id_n) = \sigma(\eta(\id_n)) = (\sigma \circ \eta)(\id_n) = f(\sigma \circ \eta)
        \end{equation}
        Thus $f$ is multiplicative. Next we want to show $\det(f(\sigma)) \neq 0$, so $f(\sigma) \in \GL_n(\R)$ for all $\sigma \in S_n$. For $\sigma$ a 2-cycle, (i.e. a transposition) we have $\sigma(\id_n) = E$ an elementary matrix for the elementary operation of type I (exchanging two rows), so $\det(\sigma(\id_n)) = 1$ or $-1$. But, every permutation $\sigma \in S_n$ can be written as a product of transpositions, say $\sigma = \beta_1 \circ ...\circ \beta_r$. Hence, multiplicativity says $f(\sigma) = f(\beta_1)...f(\beta_r)$ where $\det(f(\beta_i)) \in \{1,-1\}$ for all $i$, so $\det(f(\sigma)) \in \{1,-1\}$. Hence we have that $f(\sigma) \in \GL_n(\R)$, completing the proof.
\end{proof*}


\begin{cor}{}{}
        Due to this result we have a homomorphism
        \begin{center}
            \begin{tikzpicture}[baseline = (a).base]
            \node[scale = 1] (a) at (0,0){
                \begin{tikzcd}
                        S_n \ar[dr, "f", swap] \ar[rr, "\det\circ f"] & & \{-1,1\}\leq \C^{\times} \\
                        &\GL_n(\R) \ar[ur, "\det", swap] &
                \end{tikzcd}
            };
            \end{tikzpicture}
        \end{center}
\end{cor}


\begin{defn}{Parity of Permutations}{}
        The permutation $\sigma \in S_n$ is called \Emph{even} if $\sgn(\sigma) = 1$ and \Emph{odd} if $\sgn(\sigma) = -1$ where we define \begin{equation}
                \sgn := \det \circ f
        \end{equation}
\end{defn}

\begin{rmk}{}{}
        From the proof above we see that this definition is compatible with the definition of odd or even in terms of the transposition decomposition of a permutation.
\end{rmk}
\begin{proof*}{}{}
        (Left to the reader)
\end{proof*}


\begin{defn}{}{}
        The subgroup of $S_n$ of even permutations, $\ker(\sgn)$, is called the \Emph{alternating group of degree $n$}, denoted $A_n$.
\end{defn}


\begin{prop}{}{}
        For all $n \geq 2$ we have $|A_n| = \frac{n!}{2} = \frac{|S_n|}{2}$.
\end{prop}
\begin{proof*}{}{}
        Let $Odd_n$ be the subset of all odd permutations. Then $|S_n| = |A_n| + |Odd_n|$. Moreover, the maps \begin{equation}
                \map{A_n\rightarrow Odd_n}{\sigma \mapsto (1\;2)\circ \sigma}
        \end{equation}
        and \begin{equation}
                \map{Odd_n\rightarrow A_n}{\gamma \mapsto (1\;2)\circ \gamma}
        \end{equation}
        are mutual inverses, and hence bijections of sets. Thus $|A_n| = |Odd_n| = \frac{n!}{2}$.
\end{proof*}

\begin{eg}{}{}
        $A_2 = \{(1)\}$ and $A_3 = \{(1),(1\;2\;3),(1\;3\;2)\} \cong \Z/3\Z \cong \langle x\rangle \leq D_3$, where $D_3 \cong S_3$ from before.
\end{eg}




\section{The Quotient Group Definition and Construction}

\begin{cons}{}{}
        We want to define a group structure $\star$ on the left cosets of $H \leq G$ in $G$ such that the map \begin{equation}
                \map{\pi:G\rightarrow \{\text{left cosets of $H$ in $G$}\}}{g\mapsto gH}
        \end{equation}
        is a group homomorphism. That is we want $\pi(gg') = \pi(g) \star \pi(g')$, so we must define the operation $\star$ by $aH\star bH := abH$. For $\star$ to be well defined we need $\pi(g) = \pi(a)$ and $\pi(g') = \pi(b)$ imply $\pi(gg') = \pi(ab)$. As $\pi(g) = \pi(a)$ and $\pi(g') = \pi(b)$ we have that $gh = a$ and $g'h' = b$ for some $h,h' \in H$. Then, we want $ghg'h' \in gg'H$ with occurs if and only if $hg'h' \in g'H$, which is to say $hg' \in g'H$. In other words, we must have that $h \in g'H{g'}^{-1}$ for all $h \in H$. In particular, $H \subseteq g'H{h'}^{-1}$. But, then ${g'}^{-1}Hg' \subseteq H$, and as this must hold for all $({g'}^{-1})^{-1}H{g'}^{-1} = g'H{g'}^{-1} \subseteq H$, so $H = g'H{g'}^{-1}$. In other words, $H = g^{-1}Hg$ for all $g \in G$ if our operation is to be well defined. Moreover, if $\star$ is well-defined then we obtain a group structure on the left cosets of $H$ in $G$ under $\star$. This follows from associativity in $G$ and the fact that $\pi$ is defined to be a group homomorphism.
\end{cons}

\begin{defn}{Normal Subgroups}{}
        A subgroup $H \leq G$ is a \Emph{normal subgroup}, denoted $H \vartriangleleft G$, if and only if for all $g \in G$ and all $h \in H$, $g^{-1}hg \in H$.
\end{defn}

\begin{note*}{}{}
        This is equivalent to $g^{-1}Hg = H$ for all $g \in G$.
\end{note*}

\begin{eg}{}{}
        \leavevmode
        \begin{enumerate}
                \item $\{e_G\}\vartriangleleft G$ and $G \vartriangleleft G$.
                \item If $G$ is abelian, $H \leq G$ $\implies$ $H \vartriangleleft G$.
                \item Every subgroup of $Z(G)$, the center of $G$, is normal in $G$. Indeed, for all $h \in Z(G)$ and all $g \in G$, $g^{-1}hg = g^{-1}gh = h \in Z(G)$.
        \end{enumerate}
\end{eg}

\begin{rmk}{}{}
        The following are equivalent:
        \begin{enumerate}
                \item $H \vartriangleleft G$
                \item For all $g \in G$, $gH = Hg$
                \item Every left coset is a right coset, and vice-versa
        \end{enumerate}
\end{rmk}
\begin{proof*}{}{}
        (Left to the reader)
\end{proof*}

\begin{prop}{}{}
        Let $G\xrightarrow{f}K$ be a group homomorphism, and $H_1 \vartriangleleft K$. Then $f^{-1}(H_1) \vartriangleleft G$.
\end{prop}
\begin{proof*}{}{}
        (Left to the reader)
\end{proof*}

\begin{cor}{}{}
        For all group homomorphisms $f:G\rightarrow K$, $\ker(f) = f^{-1}(\{e_K\})$ is a normal subgroup of $G$.
\end{cor}

\begin{eg}{}{}
        Then $A_n\vartriangleleft S_n$ as $A_n = \ker(\sgn)$, $\SL_n(\R) \vartriangleleft \GL_n(\R)$ as $\SL_n(\R) = \ker(\det)$, and for $n = 3$ $S_3 \cong D_3$ so $A_3 \cong \langle x \rangle \vartriangleleft D_3$.
\end{eg}

\begin{eg}{Non-example}{}
        $H := \langle y \rangle \leq D_3$ is \underline{not} a normal subgroup of $D_3$. Indeed, we have shown previously that the left cosets partition $D_3$ differently when compared to the right cosets. Alternatively, $x^{-1}yx = yx^2 \notin H$.
\end{eg}

\begin{rmk}{}{}
        The image of a normal subgroup under a group homomorphism is a subgroup, but \underline{not necessarily} a normal subgroup.
\end{rmk}

\begin{eg}{}{}
        $\langle y \rangle \xhookrightarrow{\iota}D_3$ is a group homomorphism and $\langle y \rangle \vartriangleleft \langle y \rangle$ but $$\iota(\langle y \rangle) = \langle y \rangle \cancel{\vartriangleleft} D_3$$ is not normal.
\end{eg}


\begin{nota*}{}{}
        For subsets $A,B\subseteq G$, we define the subset product \begin{equation}
                AB := \{ab:a\in A, b \in B\}
        \end{equation}
\end{nota*}

\begin{lem}{}{}
        If $N \vartriangleleft G$, then for all $a,b \in G$, $(aN)(bN) = \{anbn':n,n' \in N\}$ is the left coset $abN$ of $N$ in $G$.
\end{lem}
\begin{proof*}{}{}
        (Left to the reader)
\end{proof*}


\begin{defn}{Quotient Group}{}
        Let $N \vartriangleleft G$ be a normal subgroup. Then the quotient group $G/N$ of $G$ by $N$ is the set of all left cosets of $N$ in $G$ with the binary operation \begin{equation}
                (gN)\star(g'N) = gg'N
        \end{equation}
        Note that this is indeed a well-defined group structure as our previous argument shows, and its structure makes the canonical projection \begin{equation}
                \map{\pi:G\rightarrow G/N}{g\mapsto gN}
        \end{equation}
        a surjective group homomorphism.
\end{defn}

\begin{cor}{}{}
        If $G$ is abelian, $G/N$ is abelian and if $G$ is cyclic then $G/N$ is cyclic.
\end{cor}
\begin{proof*}{}{}
        (Left to the reader)
\end{proof*}


\begin{eg}{}{}
        \leavevmode
        \begin{enumerate}
                \item For all $n \geq 1$, $n\Z \vartriangleleft \Z$ and $\Z/n\Z$ is as previously defined.
                \item $G/G \cong \{e_G\}$
                \item $G/\{e_G\} \cong G$
                \item $\GL_n(\R)/\SL_n(\R) \cong \R^{\times}$
                        \begin{claim}{}{}
                                The set $\left\{\begin{bmatrix} r & \mathbf{0} \\ \mathbf{0} & I_{n-1} \end{bmatrix} : r \in \R^{\times}\right\}$ is a complete set of representatives of left cosets of $\SL_n(\R)$ in $\GL_n(\R)$.
                        \end{claim}
                        \begin{proof*}{}{}
                                (Left to the reader)
                        \end{proof*}
                        Consider the map \begin{equation}
                                \map{\R^{\times}\xrightarrow{j} \GL_n(\R)/\SL_n(\R)}{r\mapsto j(r) = \begin{bmatrix} r & \mathbf{0} \\ \mathbf{0} & I_{n-1} \end{bmatrix}\SL_n(\R)}
                        \end{equation}
                        We claim it is an isomorphism.
                        \begin{proof*}{}{}
                                $j$ is a bijection by the first claim. Let $r,r' \in \R^{\times}$. Then \begin{equation}
                                        j(r)j(r') = \begin{bmatrix} r & \mathbf{0} \\ \mathbf{0} & I_{n-1} \end{bmatrix}\SL_n(\R)\begin{bmatrix} r' & \mathbf{0} \\ \mathbf{0} & I_{n-1} \end{bmatrix}\SL_n(\R) = \begin{bmatrix} rr' & \mathbf{0} \\ \mathbf{0} & I_{n-1} \end{bmatrix}\SL_n(\R) = j(rr')
                                \end{equation}
                                so $j$ is a homomorphism. Hence, $\R^{\times} \cong \GL_n(\R)/\SL_n(\R)$.
                        \end{proof*}
                \item For $n \geq 2$, $S_n/A_n \cong \{-1,1\}$ with the isomorphism \begin{equation}
                                \map{\{-1,1\}\xrightarrow{\varphi} S_n/A_n}{\map{1\mapsto A_n}{-1\mapsto (1\;2)A_n}}
                \end{equation}
        \end{enumerate}
\end{eg}

\begin{rmk}{}{}
        Consider $H\leq G$ for $G$ not necessarily a finite group, then the number of left cosets of $H$ in $G$ is the index $|G:H|$ by definition. The index can be finite even if $H$ and $G$ are infinite, but it can also be infinite. Next, if $N \vartriangleleft G$, then the order of the group $G/N$ is $|G:N|$. Moreover, if $|G| <+\infty$, $|G/N| = \frac{|G|}{|N|}$ since $|G| = |N||G:N|$ by Lagrange's Theorem.
\end{rmk}

\begin{rmk}{}{}
        Given $N \vartriangleleft G$, one can study $G$ by studying the two groups $N$ and $G/N$.
\end{rmk}

\begin{thm}{}{}
        Let $K \leq Z(G)$ (so $K \vartriangleleft G$) such that $G/K$ is cyclic. Then $G$ is abelian.
\end{thm}
\begin{proof*}{}{}
        Let $G/K = \langle gK \rangle$. Then for all $a,b \in G$ there exist $m,n \geq 0$ such that $a = g^mk$ and $b = g^nk'$ for some $k,k' \in K$. Then $$ab = g^mkg^nk' = g^{m+n}kk' = g^{n+m}k'k = g^nk'g^mk = ba$$ so $G$ is abelian.
\end{proof*}

\begin{rmk}{}{}
        However, one can have $G/N$ and $N$ cyclic but $G$ is not cyclic. Similarly, we can have $G/N$ and $N$ abelian but $G$ is not abelian.
\end{rmk}

\begin{eg}{}{}
        For $A_3 \vartriangleleft S_3$, $S_3/A_3 \cong \Z/2\Z$, where $A_3$ and $\Z/2\Z$ are syclic, but $S_3$ is not even abelian.
\end{eg}

\begin{defn}{Simple}{}
        A group $G$ is called \Emph{simple} if its only normal subgroups are $\{e_G\}$ and $G$.
\end{defn}

\begin{eg}{}{}
        \leavevmode
        \begin{enumerate}
                \item $\Z/p\Z$ is simple for all primes $p$ (no proper subgroups at all by Lagrange's Theorem).
                \item $\{e\}$ is simple
                \item $A_n$ is simple for $n \geq 5$.
        \end{enumerate}
\end{eg}

\section{Isomorphism Theorems and Correspondence}

\begin{namthm*}{First Isomorphism Theorem of Groups}{}
        Let $f:G\rightarrow G'$ be a group homomorphism and let $N = \ker(f) \vartriangleleft G$. Then there exists a unique group homomorphism $\overline{f}:G/N\rightarrow G'$ such that the following diagram commutes: 
        \begin{center}
            \begin{tikzpicture}[baseline = (a).base]
            \node[scale = 1] (a) at (0,0){
                \begin{tikzcd}
                    G \ar[dr, twoheadrightarrow, "\pi", swap] \ar[rr, "f"] & & G' \\
                        &G/\ker(f) \ar[ur, dashed, "\exists!\overline{f}", swap] &
                \end{tikzcd}
            };
            \end{tikzpicture}
        \end{center}
        In particular, for all $a \in G$ and for all $b \in aN$, $f(b) = f(a)$, and the map \begin{equation}
                \map{G/N\xrightarrow{\overline{f}} G'}{aN\mapsto f(a)}
        \end{equation}
        is well defined and satisfies $f = \overline{f} \circ \pi$. Note that we call $\pi:G\rightarrow G/N$ the \Emph{canonical map} or \Emph{factor map}. Finally, $\overline{f}$ is a group monomorphism.
\end{namthm*}
\begin{proof*}{}{}
        First we shall show $\overline{f}$ as defined is a well-defined group monomorphism with $f = \overline{f} \circ \pi$. Let $b \in aN$ so $b = an$ for some $n \in \ker(f)$. Then $$f(b) = f(an) = f(a)f(n) = f(a)e_{G'} = f(a)$$ so $\overline{f}$ is well defined. Let $bN,b'N \in G/N$. Then $$\overline{f}(bN\star b'N) = \overline{f}(bb'N) = f(bb') = f(b)f(b') = \overline{f}(bN)\overline{f}(b'N)$$ so $\overline{f}$ is multiplicative. Finally, take $bN \in \ker(\overline{f})$, so $\overline{f}(bN) = f(b) = e_{G'}$. Hence, $b \in \ker(f) = N$, so $bN = N = e_{G/N}$. Thus, $\ker(\overline{f}) = \{e_{G/N}\}$, so $\overline{f}$ is a monomorphism. By construction we have that $f = \overline{f}\circ \pi$. To prove uniqueness suppose $g$ is another such group homomorphism such that $f = g\circ \pi$. Then for all $aN \in G/N$ we have $$g(aN) = (g\circ \pi)(a) = f(a) = \overline{f}(a)$$
        so $g = \overline{f}$. Hence, uniqueness is satisfied and the proof is complete.
\end{proof*}

\begin{eg}{}{}
        \leavevmode
        \begin{enumerate}
                \item The map $\GL_n(\R)\xrightarrow{\det}\R^{\times}$ gives the isomorphism \begin{equation}
                                \GL_n(\R)/\SL_n(\R)\xrightarrow{\underset{\sim}{\det}}\R^{\times}
                \end{equation}
                        Indeed, $\SL_n(\R) = \ker(\det)$ and $\det$ is a surjective group homomorphism.
                \item For $n\geq 2$, $S_n\xrightarrow{\sgn}\{-1,1\}$ induces the isomorphism \begin{equation}
                                S_n/A_n \xrightarrow{\underset{\sim}{\sgn}}\R^{\times}
                \end{equation}
                \item The map $\C^{\times} \xrightarrow{|\cdot|}\R^{\times}$ is a homomorphism of image $\R_{>0}$ and $\ker(|\cdot |) = \{z \in \C^{\times}:|z| = 1\} = S^1$ the circle group. Thus, we have the isomorphism \begin{equation}
                                \C^{\times}/S^1 \xrightarrow{\underset{\sim}{|\cdot|}}\R_{>0}
                \end{equation}
        \end{enumerate}
\end{eg}


\begin{cor}{}{}
        \leavevmode
        \begin{enumerate}
                \item The \Emph{corestriction} of $\overline{f}$ to $\ran(f)$ is an isomorphism \begin{equation}
                                \overline{f}:G/N\xrightarrow{\sim}f(G) = \ran(f)
                \end{equation}
                \item For $G$, $G'$, finite groups, \begin{equation}
                                |G| = |\ker(f)||G/N| = |\ker(f)||\ran(f)|
                \end{equation}
                        so $|\ker(f)|\;\vert\;|G|$ while $|\ran(f)|\;\vert\;|G|$ and $|\ran(f)|\;\vert\;|G'|$.
        \end{enumerate}
\end{cor}
\begin{proof*}{}{}
        (Left to the reader)
\end{proof*}


\begin{prop}{}{}
        Let $\varphi:G\rightarrow G'$ be an epimorphism of groups. If $N \vartriangleleft G$, then $\varphi(N) \vartriangleleft G'$.
\end{prop}
\begin{proof*}{}{}
        (Left to the reader)
\end{proof*}

\begin{namthm*}{Correspondence Theorem}{}
        Let $\varphi:G\rightarrow G'$ be an epimorphism of groups. Then the preimage by $\varphi$ induces the bijection \begin{equation}
                        \map{\{H\leq G: \ker(f) \leq H\} \leftrightarrow \{H' \leq G'\}}{\map{H \mapsto \varphi(H)}{\varphi^{-1}(H') \mapsfrom H'}}
        \end{equation}
        which preserves normality of subgroups. If $H' \leq G'$ and $H \leq G$ are finite groups and correspond to each other, then $|H| = |\ker(\varphi)||H'|$.
\end{namthm*}
\begin{proof*}{}{}
        We know that $\varphi^{-1}(H')$ and $\varphi(H)$ are subgroups (respectively, normal subgroups as $\varphi$ is surjective) if $H'$ and $H$ are. Moreover, $\phi^{-1}(H') \supseteq \ker(\varphi)$ since $e_{G'} \in H'$. Now, as $\varphi$ is surjcetive, $\varphi(\varphi^{-1}(H')) = H'$, and let us show $\varphi^{-1}(\varphi(H)) = H$ if $H\geq \ker(\varphi)$. Note $$\varphi^{-1}(\varphi(H)) \supseteq H$$ by definition. Then let $g \in G$ such that $\varphi(g) \in \varphi(H)$. But then $\varphi(g) = \varphi(h)$ for some $h \in H$, so $\varphi(gh^{-1}) = \varphi(g)\varphi(h)^{-1} = e_{G'}$, which implies $gh^{-1} \in \ker(\varphi)$. Since $\ker(\varphi) \subseteq H$, $gh^{-1} \in H$ so $g = (gh^{-1})h \in H$. Hence $$\varphi^{-1}(\varphi(H)) \subseteq H$$ so both inclusions hold and $\varphi^{-1}(\varphi(H)) = H$. Thus the correspondence is indeed a bijection. Finally, we have $\varphi\rvert_{H}:H\rightarrow H' = \varphi(H)$, where $\varphi\rvert_H = \varphi \circ \iota_H$ is a surjective homomorphism. Then, by the corollarly of the First Isomorphism Theorem we have that $$|H| = |\ker(\varphi\rvert_H)||\ran(\varphi\rvert_H)| = |\ker(\varphi)||H'|$$ completing the proof.
\end{proof*}


\begin{rmk}{}{}
        The correspondance preserves \Emph{containment} and \Emph{intersections}. That is \begin{equation}
                H'\subseteq K' \iff \varphi^{-1}(H') \subseteq \varphi^{-1}(K')
        \end{equation}
        and \begin{equation}
                \varphi^{-1}(H_1'\cap H_2') = \varphi^{-1}(H'_1) \cap \varphi^{-1}(H'_2)
        \end{equation}
        Note that we can apply the correspondance theorem to the canonical epimorphism $\pi:G\rightarrow G/N$ for $N \vartriangleleft G$, to get a correspondance between subgroups of $G/N$ and subgroups of $G$ containing $N$.
\end{rmk}


\begin{eg}{}{}
        \leavevmode
        \begin{enumerate}
                \item $\Z/2\Z \cong \{-1,1\}$ has only two subgroups, $\Z/2\Z$ and $\{1\}$. So, if $A_n \leq H \leq S_n$, then $H = A_n$ or $H = S_n$ because $\sgn:S_n\rightarrow \{-1,1\}$ is an epimorphism and $\ker(\sgn) = S_n$.
                \item Similarly, for $p\Z\leq \Z$ for $p$ a prime number, if $p\Z \leq H \leq \Z$, then $H = p\Z$ or $H = \Z$ because \begin{equation}
                                \map{\Z\rightarrow \Z/p\Z}{n\mapsto {[n]}}
                \end{equation}
                        is a surjective group homomorphism, and $\Z/p\Z$ has no proper subgroups.
                \item For $G$ finite, let $N$ be a proper normal subgroup which is maximal with respect to normality. That is, if $N \leq N' \vartriangleleft G$ then $N = N'$ or $N' = G$. By the Correspondance Theorem it follows that $G/N$ is a simple group. An example is $\Z/p\Z$ for a prime $p$.
        \end{enumerate}
\end{eg}




%%%%%%%%%%%%%%%%%%%%%% - P1.Chapter 6
\chapter{\textsection Group Actions}

\section{Basic Definitions and Examples: Group Actions}

\begin{rmk}{Motivation}{}
        We want to view elements of a group $G$ as symmetries of a set $X$. In particular, for every $g \in G$ we want to associate $X \xrightarrow{\alpha_g} X$ a bijection, with $\alpha_{gh} = \alpha_g \circ \alpha_h$ and $\alpha_{e_G} = \id_X$.
\end{rmk}

\begin{eg}{}{}
        \leavevmode
        \begin{enumerate}
                \item For $X_n$, the regular convex $n$-gon, and $G = D_n$, for all $g \in D_n$ we get a bijection $X_n \xrightarrow{g} X_n$.
                \item For $X = \{1,2,...,n\}$, and $G = S_n$, for all $\sigma \in S_n$ we get a permutation $X \xrightarrow{\sigma} X$.
                \item For $X = \R^n$, and $G = \GL_n(\R)$, for all $A \in \GL_n(\R)$ we get a bijection \begin{equation}
                                \map{\R^n\xrightarrow{L_A}\R^n}{\vec{v} \mapsto A\vec{v}}
                \end{equation}
                \item For $X = G$ a group, we have the action by \Emph{left multiplication}, where for all $g \in G$ we get the bijection \begin{equation}
                                \map{G\xrightarrow{\ell_g}G}{x\mapsto gx}
                \end{equation}
                \item For $X = G$ a group, and $H \leq G$, for all $h \in H$ we have the bijection \begin{equation}
                                \map{G \xrightarrow{\ell_h} G}{g \mapsto hg}
                \end{equation}
                \item For $X = G$ a group, we have the action by \Emph{conjugation}, where for all $g \in G$, we have the bijection \begin{equation}
                                \map{G\xrightarrow{\beta_g}G}{x\mapsto gxg^{-1}}
                \end{equation}
        \end{enumerate}
\end{eg}

\begin{rmk}{}{}
        There are two equivalent ways to formalize the notion of group actions.
\end{rmk}

\begin{defn}{Group Action}{}
        A \Emph{group action} of a group $G$ on a set $X$ is a \begin{enumerate}
                \item group homomorphism \begin{equation}
                                \map{\alpha:G\rightarrow S_X}{g\mapsto \alpha_g}
                \end{equation}
                \item map $a:G\times X \rightarrow X$ such that \begin{enumerate}
                                \item $a(e_G,x) = x$ for all $x \in X$
                                \item $a(gh,x) = a(g,a(h,x))$ for all $g,h \in G$ and all $x \in X$.
                \end{enumerate}
        \end{enumerate}
\end{defn}

\begin{defn}{}{}
        Let $G$ be a group acting on a set $X$. The data of 1. (or equivalently 2.) in the previous definitions is called an \Emph{action of $G$ on $X$} and $X$ is called a \Emph{$G$-set}.
\end{defn}

\begin{claim}{}{}
        Definitions 1. and 2. of a group action are equivalent. That is, for any group $G$ and any nonempty set $A$ there is a bijection between the actions of $G$ on $A$ and the group homomorphisms of $G$ into $S_A$.
\end{claim}
\begin{proof*}{}{}
        (Left to the reader)
\end{proof*}


\begin{defn}{}{}
    Let $G$ be a group acting on a nonempty set $A$. Then the homomorphism $\alpha:G\rightarrow S_A$ associated with the action of $G$ on $A$ is called a \Emph{permutation representation} associated to the given action. We say a given action of $G$ on $A$ \emph{affords} or \emph{induces} the associated permutation representation of $G$.
\end{defn}


\begin{defn}{}{}
        The kernel of the group homomorphism associated with a group action is $\ker(\alpha) = \{g \in G:\alpha(g) = \id_X\}$, or equivalently for definition 2. the set $\{g \in G: \forall x \in X, a(g,x) = x\}$. If $\ker(\alpha) = \{e_G\}$, then the action of $G$ on $X$ is said to be \Emph{faithful}.
\end{defn}

\begin{rmk}{}{}
    Two group elements in $G$ induce the same permutation of the set $A$ if and only if the exist in the same coset of the kernel of the action (i.e., if and only if they are in the same fiber of the permutation representation $\alpha$).


    Moreover, the inhereted action of the quotient space $G/\ker(\alpha)$ on $A$ is faithful.
\end{rmk}


\begin{eg}{}{}
        \leavevmode
        \begin{enumerate}
                \item For the left action of $G$ on $G$, $\ker = \{g \in G: \ell_g = \id_G\}$. We want $\ell_g(h) = h$ for all $h \in G$, where $\ell_g(h) = gh$, so $g = e_G$. Thus, $\ker = \{e_G\}$, so the action is faithful.
                \item For the conjugation action, with the associated group homomorphism \begin{equation}
                                \map{\beta:G\rightarrow S_G}{g\mapsto\map{\beta_g:G\rightarrow G}{x\mapsto gxg^{-1}}}
                        \end{equation}
                        We claim that $\ker(\beta) = Z(G)$, the center of $G$.
        \end{enumerate}
\end{eg}


\begin{defn}{}{}
    The permutation representation afforded by the left multiplication action on the elements of the group $G$ is called the \Emph{left regular representation} of $G$.
\end{defn}

\begin{namthm*}{Cayley's Theorem}{}
        Every group $G$ is isomorphic to a subgroup of its group of symmetries $\sym(G)$.
\end{namthm*}
\begin{proof*}{}{}
        (Left to the reader)
\end{proof*}


\begin{nota*}{}{} 
        If $G\times Y\xrightarrow{a}Y$ is a group action, we denote $a(g,y)$ by $g.y$ (or even $gy$ if there is no confusion). Note that we have $(gh).y = g.(h.y)$ and $e_G.y = y$ for all $g,h \in G$ and all $y \in Y$. Moreover, to say that a group $G$ acts on a set $Y$ we write $G \circlearrowright Y$.
\end{nota*}


\begin{defn}{}{}
        Let $G\times Y \xrightarrow{a} Y$ be an action and let $y \in Y$. \begin{enumerate}
                \item The \Emph{orbit} of $y$ under the action by $G$ is the set $\mathcal{O}_y = \{g.y:g\in G\}\subseteq Y$ (also denoted $G.y$)
                \item The \Emph{stabilizer} of $y$ under the action by $G$ is the set $G_y = \{g\in G:g.y = y\}\subseteq G$.
                \item $y \in Y$ is called a \Emph{fixed point} of the action if $G_y = G$, so for all $g \in G$, $g.y = y$.
        \end{enumerate}
\end{defn}

\begin{defn}{}{}
    Let $G$ be a group and $A$ a nonempty set. The action of $G$ on $A$ is said to be \Emph{transitive} if there is only one orbit, i.e., given any two elements $a,b \in A$, there exists $g \in G$ such that $b = g.a$.
\end{defn}



\begin{prop}{}{}
        For $G \circlearrowright Y$ and all $y \in Y$, $G_y \leq G$.
\end{prop}
\begin{proof*}{}{}
        (Left to the reader)
\end{proof*}


\begin{eg}{}{}
        \leavevmode
        \begin{enumerate}
                \item For the left multiplication action, $G \circlearrowright G$, for all $g \in G$ we have orbits $G.g = \mathcal{O}_g = G$, and stabilizers $G_g = \{e_G\}$.
                \item For the conjugation action, $G \circlearrowright G$, for all $g \in G$ we have orbits \begin{equation}
                                G.g = \mathcal{O}_g = \{a \in G:\exists h \in H, hgh^{-1} = a\}
                \end{equation}
                        These sets are called the \Emph{conjugacy classes} of $G$. The stabilizers of the action are \begin{equation}
                                G_g = \{a \in G: aga^{-1} = g\} = Z(g)
                        \end{equation}
                        the \Emph{centralizer} of $g$ in $G$.
                \item The left multiplication action of a subgroup $H \leq G$ on $G$, $H \circlearrowright G$, for all $g \in G$ the orbit is $H.g = Hg$ the right coset of $H$. Moreover, the stabilizers still are $H_g = \{e_G\}$.
        \end{enumerate}
\end{eg}

\begin{lem}{}{}
        Let $a:G\times Y \rightarrow Y$ be a group action. Then \begin{enumerate}
                \item The orbits $G.y = \mathcal{O}_y$ of the action form a partition of $Y$.
                \item For all $y \in Y$, the order of the orbit $|G.y| = |\mathcal{O}_y|$, is the index $|G:G_y|$ of the stabilizer $G_y$ of $y$ in $G$. (\Emph{Orbit Stabilizer Theorem})
        \end{enumerate}
\end{lem}
\begin{proof*}{}{}
        [1.] First, note that $y = e_G.y \in G.y$ for all $y \in Y$, so \begin{equation}
                Y = \bigcup_{y \in Y}G.y
        \end{equation}
        Next, let $y,y' \in Y$ and suppose $g.y = g'.y' \in G.y \cap G.y'$, for some $g,g' \in G$. Then we have that $y = g^{-1}.(g'.y') = (g^{-1}g').y' \in G.y'$, so for all $h.y \in G.y$, $h.y = (hg^{-1}g').y' \in G.y'$ so $G.y \subseteq G.y'$. Similarly we have that $G.y \supseteq G.y'$, so $G.y = G.y'$. Hence, the orbits indeed partition $Y$.


        [2.] Define a map \begin{equation}
                \map{G/G_y\xrightarrow{f} G.y}{aG_y\mapsto a.y}
        \end{equation}
        where $G/G_y$ denotes the set of left cosets of $G_y$ (not necessarily a group). First, to show the map is well defined suppose $aG_y = bG_y$. Then we have that $a = bg$ for some $g \in G_y$. It follows that $$a.y = (bg).y = b.(g.y) = b.y$$ since $g \in G_y$, so $f(aG_y) = f(bG_y)$ and the map is well defined. Now suppose $aG_y, cG_y \in G/G_y$ such that $a.y = c.y$. Then $(c^{-1}a).y = c^{-1}.(c.y) = e_G.y = y$, which implies $c^{-1}a \in G_y$. This implies by coset equality that $aG_y = cG_y$, so $f$ is an injection. Finally, if $g.y \in G.y$, we have that $f(gG_y) = g.y$, so $f$ is a surjection. Therefore $f$ is a bijection and we conclude that \begin{equation}
                |G:G_y| = |G/G_y| = |G.y|
        \end{equation}
        as claimed.
\end{proof*}


\subsection{Application to Cycle Decompositions}

Using the tools we have developed with group actions, we can provide an alternate proof to the fact that any permutation $\sigma \in S_n$ can be decomposed into disjoint cycles.

\begin{proof*}{}{}
    Let $A = \{1,2,...,n\}$, let $\sigma \in S_n$, and let $G =\langle \sigma \rangle$. Then consider the action of $\langle \sigma \rangle$ on $A$. Let $\mathcal{O}$ be one of the orbits of this action, and let $x \in \mathcal{O}$. Note that there exists a bijection between the elements of $\mathcal{O}$ and the left cosets of the stabilizer $G_x$ in $G$, given explicitly by \begin{equation*}
        \sigma^ix\mapsto \sigma^iG_x
    \end{equation*}
    Since $G$ is cyclic it is abelian, so $G_x \trianglelefteq G$ and $G/G_x$ is cyclic of order $d$, where $d$ is the order of $\sigma$ in $G/G_x$, in particular it is the smallest positive integer for which $\sigma^d \in G_x$. Also, $d = |G:G_x| = |\mathcal{O}|$. Thus, the distinct cosets of $G_x$ in $G$ are \begin{equation*}
        1G_x,\sigma G_x,\sigma^2G_x,...,\sigma^{d-1}G_x
    \end{equation*}
    This shows that the distinct elements of $\mathcal{O}$ are $x,\sigma(x),...,\sigma^{d-1}(x)$ by our bijection. Orderin the elements of $\mathcal{O}$ in this manner shows that $\sigma$ cycles the elements of $\mathcal{O}$, that is, on an orbit of size $d$, $\sigma$ acts as a $d$-cycle. This proves the existence of a cycle decomposition for each $\sigma \in S_n$.


    The orbits of $\langle \sigma \rangle$ are uniquely determined by $\sigma$. The only choice is in the order the orbits are listed in, which depends on our initial representative from $\mathcal{O}$. It follows that the cycle decomposition above is unique up to a rearrangement of the cycles and up to a cyclic permutation of the integers within each cycle.
\end{proof*}


\section{Counting and Combinatorial Formulas}


\begin{namthm*}{Counting Formula}{}
        Suppose $G \circlearrowright Y$. Then for all $y \in Y$ we have the \Emph{counting formula} \begin{equation}
                |G| = |G_y||G.y|
        \end{equation}
\end{namthm*}
\begin{proof*}{}{}
        (Left to the reader)
\end{proof*}


\begin{namthm*}{Orbit Decomposition Theorem}{}
        Let $Y$ be a finite set with $G \circlearrowright Y$. Let $Y_f \subseteq Y$ denote the set of fixed points of $Y$ under the action. Let $G.y_1, ..., G.y_n$ be the distinct non-singular orbits of $Y$ for some integer $n \geq 0$. Then \begin{equation}
                |Y| = |Y_f| + \sum_{i=1}^n|G:G_y|
        \end{equation}
\end{namthm*}
\begin{proof*}{}{}
        (Left to the reader)
\end{proof*}

\begin{cor}{Class Equation}{}
        For a group $G$ and the conjugation action $G \circlearrowright G$, we have the \Emph{class equation} \begin{equation}
                |G| = \sum\limits_{\text{Conjugacy Classes $C$}}|C|
        \end{equation}
\end{cor}

\begin{rmk}{}{}
        By the counting formula we have that $|C|\;\vert\;|G|$ for all conjugacy classes $C$.
\end{rmk}


\begin{prop}{}{}
        The set of fixed elements of the conjugacy action of $G$ on $G$, $G \circlearrowright G$, is the center of $G$, $Z(G)$.
\end{prop}
\begin{proof*}{}{}
        (Left to the reader)
\end{proof*}


\begin{prop}{}{}
        If $H \vartriangleleft G$, then $H$ is a union of conjugacy classes. Indeed, for all $h \in H$ and all $g \in G$, $g.h=ghg^{-1} \in H$, so $G.h \subseteq H$.
\end{prop}
\begin{proof*}{}{}
        (Left to the reader)
\end{proof*}


\begin{eg}{}{}
        \leavevmode
        \begin{enumerate}
                \item For an abelian group we get the class equation \begin{equation}
                                |G| = 1+1+...+1
                \end{equation}
                \item For $D_3 (\cong S_3)$, $|D_3| = 6$, and $D_3 = \langle x,y\rangle$. First, note that $|G.x| = [D_3:G_x] = [D_3:Z(x)] = 6/3 = 2$, where $\langle x \rangle \subseteq Z(x)$ while $y \notin Z(x)$ as $yxy = x^2$, so by Lagrange's Theorem $|Z(x)| = 3$ and $Z(x) = \langle x \rangle$. Similarly, $Z(y) = \langle y \rangle$ as $xyx^2 = x^2y \neq y$, so $x,x^2 \notin Z(y)$, so $|G.y| = [D_3:Z(y)] = 6/2 = 3$. Thus, we have that \begin{equation}
                                |D_3| = |D_3.e| + |D_3.x| + |D_3.y| = 1 + 2 + e
                \end{equation}
        \item The class equation of $A_5$ is \begin{equation}
                        |A_5| = 1+ 20 + 12 + 12 + 15
        \end{equation}
                        First note that $|A_5| = \frac{5!}{2} = 60$, and $A_5$ is composed all even permutations in $S_5$. First, $1 = |\{(1)\} = A_5.(1)|$. Then, for the three cycles in $A_5$, there are $\frac{5*4*3}{3} = 20$ three cycles. In $S_5$ all three cycles are in the same conjugacy class. It follows that for any three cycle $\sigma \in A_5$, $|S_5.\sigma| = 20 = |S_5:{S_5}_{\sigma}|$, which implies $|{S_5}_{\sigma}| = 6$. Note $\langle \sigma \rangle \subseteq {S_5}_{\sigma}$, but also for the $1 \leq i,j \leq 5$ not moved by $\sigma$, $(i\;j), \sigma(i\;j), \sigma^{-1}(i\;j) \in {S_5}_{\sigma}$. But, $(i\;j), \sigma(i\;j), \sigma^{-1}(i\;j) \notin A_5$ as they are odd, but $\langle \sigma \rangle \subseteq A_5$ so we have that $|{A_5}_{\sigma}| = 3$. Thus, $|A_5.\sigma| = [A_5:{A_5}_{\sigma}] = 60/3 = 20$, which is the entire conjugacy class. Next, for pairs of disjoint transpositions we have $\frac{5*4*3*2}{2*2*2} = 15$, which are all in the same conjugacy class in $S_5$. Thus, for a transposition pair $\tau$ we have $|S_5.\tau| = 15 = [S_5:{S_5}_{\tau}]$, so $|{S_5}_{\tau}| = 120/15 = 8$. Individual transpositions of the pair are in the centralizer, but they are not in $A_5$ as they are odd permutations, so $|{A_5}_{\tau}| \leq 6$. Moreover, the two four cycles composed of the pair of transpositions adjoined are in the centralizer while not being in $A_5$, so $|{A_5}_{\tau}| \leq 4$. But, note that if $|{A_5}_{\tau}| \leq 3$ then $|A_5.\tau| \geq 20$, but there are only $15$ elements in the conjugacy class of pairs of transpositions in $S_5$. Thus, we have that $|{A_5}_{\tau}| = 4$ so $|A_5.\tau| = |A_5:{A_5}_{\tau}| = 60/4 = 15$. Finally, we have $5!/5 =24$ five cycles in $A_5$. Let $\alpha$ be one such five cycle. Then $|S_5.\alpha| = 24 = [S_5:{S_5}_{\alpha}]$, which implies that $|{S_5}_{\alpha}| = 5$. But $\langle \alpha \rangle \subseteq {S_5}_{\alpha}$ and $\langle \alpha \rangle \subseteq A_5$ with $|\langle \alpha\rangle| = 5$, so ${S_5}_{\alpha} = {A_5}_{\alpha}$. Thus, we have that $|A_5.\alpha| = [A_5:{A_5}_{\alpha}] = 60/5 = 12$. Therefore, the conjugacy class is split in two, and we have arrived at our class equation.
        \end{enumerate}
\end{eg}

\begin{cor}{}{} 
        $A_5$ is a simple group.
\end{cor}
\begin{proof*}{}{}
        Let $N \vartriangleleft A_5$. We must have that $|N|\;\vert\;|A_5| = 60$ and $|N| = \sum|C|$ for conjugacy classes $C$ of $A_5$, and $\{(1)\}$ is one of them. But by the class equation the only possibilities are $|N| = 1$ and $|N| = 60$. Thus, $N \vartriangleleft A_5$ implies either $N = \{(1)\}$ or $N = A_5$, so $A_5$ is a simple group by definition.
\end{proof*}

\begin{prop}{}{}
        Let $|G| = p^n$ for a prime $p$ (such a group $G$ is called a \Emph{p-group}). The center of $G$ is not the trivial subgroup $\{e_G\}$.
\end{prop}
\begin{proof*}{}{}
        The class equation of $G$ is \begin{equation}
                p^n = |G| = 1+\sum_i|C_i|
        \end{equation}
        Note $g \in Z(G)$ if and only if $G.g = \{g\}$. So, if $Z(g) = \{e_G\}$ then $|C_i| > 1$ for all $i$. Thus, $p\;\vert\;|C_i|$ for all $i$ since $1 < |C_i|\;\vert\;|G| = p^n$. Hence, \begin{equation}
                p\;\vert\;\left(|G|-\sum_i|C_i|\right) = |C_1| = 1
        \end{equation}
        which is a contradiction as this implies $p = 1$, and $1$ is not prime. So, $|Z(G)| > 1$ as claimed.
\end{proof*}


\begin{namthm*}{Cauchy's Theorem}{}
        Let $G$ be a finite group. If $p\;\vert\;|G|$ for $p$ a prime, then $G$ has an element of order $p$.
\end{namthm*}
\begin{proof*}{}{}
        We want to define an action on the set $$G^p = \prod\limits_{i=1}^pG$$ by a cyclic group of order $p$ ($\cong \Z/p\Z$). Let $H = \langle \sigma \rangle$ for $o(\sigma) = p$, and we define the action by \begin{equation}
                \sigma.(g_1,g_2,...,g_p):=(g_2,g_3,...,g_p,g_1)
        \end{equation}
        This gives a well-defined group action of $H$ on $G^p$. Moreover, $x \in G^p$ is a fixed point of the action if and only if \begin{equation}
                (g_1,g_2,...,g_p) = (g_2,g_3,...,g_p,g_1)
        \end{equation}
        so $g_1=g_2=...=g_p$. We are interested in a subset $Y \subseteq G^p$ defined by \begin{equation}
                Y:= \{(g_1,g_2,...,g_p)\in G^p: g_1g_2...g_p = e\}
        \end{equation}
        We see that for all $y \in Y$, $H.y \subseteq Y$. Indeed, if $g_1...g_p = e$, then $e = g_1^{-1}g_1 = g_2...g_pg_1$, which is associated to $\sigma.(g_1,...,g_p)$. Hence, if $y \in Y$, then $\sigma.y \in Y$. Thus, we have an action of $H$ on $Y$ taken by corestriction. Next, $|Y| = |G|^{p-1}$. Indeed, choose $g_1,g_2,...,g_{p-1}$ freely, which constitutes $|G|^{p-1}$ choices, then choose $G_p = (g_1g_2...g_{p-1})^{-1}$. Then we have that $(g_1,...,g_p) \in Y$. Note that $y \in Y$ is fixed by $H$, for $y = (g_1,...,g_p)$, if and only if $g=g_1=...=g_p$, and $g_1...g_p = g^p = e$. Then, $o(g) \in \{1,p\}$. Note that $(e,...,e) \in Y$ is a fixed point. We want to show that $|Y_f| > 1$. Applying the Orbit Decomposition Theorem to the action of $H$ on $Y$ we have that \begin{equation}
                |G|^{p-1} = |Y| = |Y_f| + \sum_{i=1}^n|H:H_{y_i}|
        \end{equation}
        where each $|H:H_{y_i}|$ divides $|H| = p$ and is greater than $1$ as they are not fixed points, so in particular $|H:H_{y_i}| = p$ for each $i$. Thus, \begin{equation}
                |Y_f| = |G|^{p-1} - \sum_{i=1}^n|H:H_{y_i}|
        \end{equation}
        which $p$ divides as $p\;\vert\;|G|$ by our initial assumption. Then, $p\;\vert\;|Y_f|$ so in particular $|Y_f| > 1$. Hence, there exists $(g,g,...,g) \in Y_f$ such that $g \neq e$ and $g^p = e$ as desired.
\end{proof*}


\section{Conjugacy Actions and Actions on Subgroups}

In this section we redefine and make precise certain concepts previously mentioned off hand in examples in relation to conjugation and subgroup actions.


\begin{thm}{}{}
    Let $G$ be a group, let $H$ be a subgroup of $G$, and let $G$ act by left multiplication on the set $A$ of left cosets of $H$ in $G$. Let $\pi_H$ be the associated permutation representation afforded by this action. Then \begin{enumerate}
        \item $G$ acts transitively on $A$
        \item the stabilizer in $G$ of the point $1H \in A$ is the subgroup $H$
        \item the kernel of the action (i.e., the kernel of $\pi_H$) is $\bigcap_{x\in G}xHx^{-1}$, and $\ker \pi_H$ is the largest normal subgroup of $G$ contained in $H$.
    \end{enumerate}
\end{thm}
\begin{proof*}{}{}
    (Left to the reader)
\end{proof*}

\begin{defn}{}{}
    Let $G$ be a group and define $Sub(G)$ as the set of all subgroups of $G$. Then $G$ acts on the set $Sub(G)$ by \begin{equation*}
        a:G\times Sub(G)\rightarrow Sub(G);\;(g,H) \mapsto g\cdot H:= gHg^{-1}
    \end{equation*}
    where \begin{equation*}
        gHg^{-1} := \{ghg^{-1}:h \in H\}
    \end{equation*}
    Observe that the orbits of the action partition the subgroups of $G$ into conjugacy classes.
\end{defn}


\begin{defn}{}{}
    Let $g \in G$, where $G$ is a group. The stabilizer of $g$ under the conjugation action of $G$ on itself is equal to \begin{equation*}
        Z(g) = \{h \in G: hgh^{-1} = g\}
    \end{equation*}
    and is called the \Emph{centralizer of $g$}. It follows that the center of $G$ is equal to \begin{equation*}
        Z(G) = \bigcap\limits_{g\in G}Z(g)
    \end{equation*}
\end{defn}


\begin{defn}{}{}
    Let $g \in G$. The equivalence class of $g$ with respect to the equivalence relation coming from the conjugation action of $G$ on itself is called the \Emph{conjugacy class of $g$ in $G$}, sometimes denoted $C(g)$; thus \begin{equation*}
        C(g) := \{g' \in G: \exists h \in G; hgh^{-1} = g' \}
    \end{equation*}
\end{defn}


\begin{defn}{}{}
    Let $H\leq G$ be a subgroup of $G$. The stabilizer of $H$ under the action of conjugation on $Sub(G)$ is \begin{equation*}
        N_G(H) = \{g \in G:gHg^{-1} = H\}
    \end{equation*}
    and it is called the \Emph{normalizer of $H$ in $G$}.
\end{defn}


\begin{defn}{}{}
    Let $G$ be a group and $S \subseteq G$ a subset of $G$. Let $g \in G$ and define $gSg^{-1} :=\{gsg^{-1}:s \in S\}$. Then $G$ acts on its power set $\mathcal{P}(G)$ of all subsets of itself by defining $g\cdot S = gSg^{-1}$ for any $g \in G$ and $S \in \mathcal{P}(G)$. 
\end{defn}

\begin{defn}{}{}
    Two subsets $S$ and $T$ of $G$ are said to be \Emph{conjugate} in $G$ if there is some $g \in G$ such that $T = gSg^{-1}$.
\end{defn}

\begin{prop}{}{}
    The number of conjugates of a subset $S$ in $G$ is the index of the normalizer of $S$, $|G:N_G(S)|$. Moreover, the number of conjugates of an element $s$ of $G$ is the index of the centralizer of $s$, $|G:Z(s)|$.
\end{prop}

\begin{namthm}{Class Equation (Alternate Form)}{}
    Let $G$ be a finite group and let $g_1,g_2,...,g_r$ be representatives of the distinct conjugacy classes of $G$ not contained in the center $Z(G)$ of $G$. Then \begin{equation*}
        |G| = |Z(G)| + \sum\limits_{i=1}^r|G:Z(g_i)|
    \end{equation*}
\end{namthm}
\begin{proof*}{}{}
    Note that for $x \in G$, the conjugacy class of $x$ is the singleton $\{x\}$ if and only if $x \in Z(G)$, since then $gxg^{-1} = x$ for all $g \in G$. Let $Z(G) = \{1,z_2,...,z_m\}$, let $\mathcal{K}_1,...,\mathcal{K}_r$ be the conjugacy classes of $G$ not contained in the center, and let $g_i$ be a representative of $\mathcal{K}_i$ for each $i$. Then the full set of conjugacy classes of $G$ is given by \begin{equation*}
        \{1\},\{z_2\},...,\{z_r\},\mathcal{K}_1,...,\mathcal{K}_r
    \end{equation*}
    Since these partition $G$ we have \begin{align*} 
        |G| &= \sum\limits_{i=1}^m1 + \sum\limits_{i=1}^r|\mathcal{K}_i| \\
        &= |Z(G)| + \sum\limits_{i=1}^r|G:Z(g_i)|
    \end{align*}
    This proves the class equation.
\end{proof*}

\begin{thm}{}{}
    If $p$ is a prime and $P$ a group of prime power order $p^{\alpha}$ for some $\alpha \geq 1$, then $P$ has a nontrivial center: $Z(P) \neq \{1\}$.
\end{thm}
\begin{proof*}{}{}
    By the class equation \begin{equation*}
        |P| = |Z(P)| + \sum\limits_{i=1}^r|P:Z(g_i)|
    \end{equation*}
    By definition $Z(g_i) \neq P$ for $i \in \{1,2,...,r\}$, so $p$ divides $|P:Z(g_i)|$. Since $p$ also divides $|P|$ it follows that $p$ must divide $|Z(P)|$, hence the center must be nontrivial.
\end{proof*}


\begin{cor}{}{}
    If $|P| = p^2$ for some prime $p$, then $P$ is abelian. More precisely, $P$ is isomorphic to either $\Z/p^2\Z$ or $\Z/p\Z\times \Z/p\Z$.
\end{cor}
\begin{proof*}{}{}
    Since $|Z(P)| \neq 1$ by the previous theorem, it follows that $|Z(P)| \in \{p,p^2\}$. Thus $|P/Z(P)| \in \{1,2\}$, so $P/Z(P)$ is cyclic. Let $P/Z(P) = \langle gZ(P)\rangle$ for some $g \in P$. Let $x,y \in P$, then since $P/Z(P)$ is cyclic there exist $n,m \in \Z$ such that $xZ(P) = g^nZ(P)$ and $yZ(P) = g^mZ(P)$. That is, there exist $z,z' \in Z(P)$ such that $x = g^nz$ and $y = g^mz'$. It follows that \begin{equation*}
        xy = g^nzg^mz' = g^{n+m}zz' = g^mg^nz'z = g^mz'g^nz = yx
    \end{equation*}
    However, $x,y$ were arbitrary elements of $P$ so $P$ must be abelian. Hence $Z(P) = P$. If $P$ has an element of order $p^2$, then $P$ is cyclic and $P \cong \Z/p^2\Z$. Assume therefore that every nonidentity element of $P$ has order $p$. Let $x$ be a non-identity element of $P$ and let $y \in P\backslash\langle x\rangle$. Since $|\langle x,y \rangle| > |\langle x\rangle|=p$, we must have that $P = \langle x,y\rangle$. Both $x$ and $y$ have order $p$ so $\langle x\rangle\times \langle y\rangle \cong \Z/p\Z\times \Z/p\Z$. It now follows directly that the map $(x^a,y^b)\mapsto x^ay^b$ is an isomorphism from $\langle x\rangle\times \langle y\rangle$ onto $P$. This completes the proof.
\end{proof*}

\subsection{Conjugation in Special groups}

\begin{rmk}{}{}
    Note that in the matrix group $\GL_n(\F)$, conjugation is equivalent to a change of basis: $A \mapsto PAP^{-1}$. An analogous situation arises in $S_n$.
\end{rmk}

\begin{prop}{}{}
    Let $\sigma$ and $\tau$ be elements of the symmetric group $S_n$, and suppose $\sigma$ has cycle decomposition \begin{equation*}
        (a_1\;a_2\;...\;a_{k_1})(b_1\;b_2\;...\;b_{k_2})...
    \end{equation*}
    Then $\tau\sigma\tau^{-1}$ has cycle decompossition: \begin{equation*}
        (\tau(a_1)\;\tau(a_2)\;...\;\tau(a_{k_1}))(\tau(b_1)\;\tau(b_2)\;...\;\tau(b_{k_2}))...
    \end{equation*}
\end{prop}
\begin{proof*}{}{}
    (Left to the reader)
\end{proof*}


\begin{defn}{}{}
    \leavevmode
    \begin{enumerate}
        \item If $\sigma \in S_n$ is the product of disjoint cycles of lengths $n_1,n_2,..., n_r$ with $n_1\leq n_2\leq ...\leq n_r$ (including its $1$-cycles) then the integers $n_1,n_2,...,n_r$ are called the \Emph{cycle type} of $\sigma$.
        \item If $n \in \Z^+$, a partition of $n$ is any nondecreasing sequence of positive integers whose sum is $n$.
    \end{enumerate}
\end{defn}


\begin{prop}{}{}
    Two elements of $S_n$ are conjugate in $S_n$ if and only if they have the same cycle type. The number of conjugacy classes in $S_n$ equals the number of partitions of $S_n$.
\end{prop}
\begin{proof*}{}{}
    (Left to the reader - hint: D\& F p.126)
\end{proof*}


\subsection{Right Group Actions}

\begin{defn}{}{}
    Let $G$ be a group and define the \Emph{right group action} of $G$ on a nonempty set $A$ as a map from $A\times G$ to $A$, denoted by $a\cdot g$ for $a \in A$ and $g\in G$, that satisfies the axioms: \begin{enumerate}
        \item $(a\cdot g_1)\cdot g_2 = a\cdot(g_1g_2)$ for all $a \in A$, and $g_1,g_2 \in G$, and 
        \item $a\cdot 1 = a$ for all $a \in A$.
    \end{enumerate}
\end{defn}

\begin{rmk}{}{}
    Conjugation as a write group action is denoted by $a \cdot g = g^{-1}ag$, and it is sometimes notated with $a\cdot g = a^g$.
\end{rmk}







%%%%%%%%%%%%%%%%%%%%%% - P1.Chapter 7
\chapter{\textsection Product Groups}


\section{Basic Definitions and Examples: Product Groups}

\begin{rec}{}{}
        For $G$, $G'$ group, recall that $G \times G'$ with the binary operation $(a,a') \star (b,b') := (a\star_G b, a' \star_{G'} b')$ is a group with identity $(e_G, e_{G'})$ and inverse $(a,a')^{-1} = (a^{-1},{a'}^{-1})$.
\end{rec}


\begin{defn}{Direct Product}{}
        The group $G \times G'$ is called the \Emph{direct product} of $G$ and $G'$.
\end{defn}

\begin{rmk}{}{}
        We have four homomorphisms which characterize the direct product of $G$ and $G'$:
        \begin{center}
            \begin{tikzpicture}[baseline = (a).base]
            \node[scale = 1] (a) at (0,0){
                \begin{tikzcd}
                    G \ar[dr, hookrightarrow, "\iota_G", swap] & & G \\
                        &G \times G' \ar[ur, twoheadrightarrow, "p_G", swap] \ar[dr, twoheadrightarrow, "p_{G'}"] & \\
                        G' \ar[ur, hookrightarrow, "\iota_{G'}"] & & G'
                \end{tikzcd}
            };
            \end{tikzpicture}
        \end{center}
        defined by \begin{align}
                \iota_G(g) = (g,e_{G'}),&\;\iota_{G'}(g') = (e_G,g') \\
                p_G(g,g') = g,&\;p_{G'}(g,g') = g'
        \end{align}
        for all $g \in G$ and $g' \in G'$. Moreover, this map satisfy the following properties: \begin{enumerate}
                \item $\iota_G$ and $\iota_{G'}$ are \Emph{monomorphisms} of homomorphic image $G \times \{e_{G'}\} \leq G\times G'$ and $\{e_G\}\times G' \leq G\times G'$, respectively.
                \item $p_G$ and $p_{G'}$ are \Emph{epimorphisms} called \Emph{projections}, with kernels $\ker(p_G) = \{e_G\}\times G'$ and $\ker(p_{G'}) = G\times \{e_{G'}\}$. Hence, we have that normal subgroups \begin{equation}
                                \{e_G\}\times G' \vartriangleleft G \times G' \vartriangleright G\times \{e_{G'}\}
                \end{equation}
        \end{enumerate}
\end{rmk}

\begin{namthm*}{Universal Property of Product Groups}{}
        Let $G$ and $G'$ be groups. Then the direct product of $G$ and $G'$ is defined uniquely up to isomorphism by the triple \begin{equation}
                (H,\rho_G:H\rightarrow G, \rho_{G'}:H\rightarrow G')
        \end{equation}
        that satisfies the universal property \begin{center}
            \begin{tikzpicture}[baseline = (a).base]
            \node[scale = 1] (a) at (0,0){
                \begin{tikzcd}
                        & \forall K \ar[ddl, bend right, "\forall \sigma_G", swap] \ar[ddr, bend left, "\forall \sigma_{G'}"] \ar[d, dashed, "\exists! \varphi"] & \\
                        & H \ar[dl, twoheadrightarrow, "\rho_G"] \ar[dr, twoheadrightarrow, "\rho_{G'}", swap] & \\
                        G & & G' 
                \end{tikzcd}
            };
            \end{tikzpicture}
        \end{center}
\end{namthm*}
\begin{proof*}{}{}
        (Left to the reader)
\end{proof*}


\begin{rmk}{}{}
        Given a group $K$, it is desireable to decompose $K$ as a product $K \cong H \times H'$ for $H,H' \leq K$ proper subgroups. Indeed, $H$ and $H'$ are simpler groups, and it is easy to relate properties of $K$ to properties of $H$ and $H'$.
\end{rmk}


\begin{note*}{}{}
        A group cannot necessarily be written in this way for non-trivial $H$ and $H'$.
\end{note*}


\begin{eg}{}{}
        Observe that $\Z/6\Z \cong \Z/2\Z \times \Z/3\Z$ as $([1]_2,[1]_3)$ is an element of order $6$ and $|\Z/2\Z\times \Z/3\Z| = 6$.
\end{eg}

\begin{prop}{Cyclic Group Decomposition}{}
        Let $m,n \in \Z$, for $m,n \geq 1$. Then we have that \begin{equation}
                \Z/mn\Z \cong \Z/m\Z\times \Z/n\Z
        \end{equation}
        if and only if $m$ and $n$ are relatively prime.
\end{prop}
\begin{proof*}{}{}
        (Left to the reader)
\end{proof*}

\begin{eg}{Non-example}{}
        Observe that $\Z/4\Z \cancel{\cong} \Z/2\Z \times \Z/2\Z$. Indeed, every element of $\Z/2\Z \times \Z/2\Z$ has order $1$ or $2$, whereas $\Z/4\Z$ is generated by an element of order $4$.
\end{eg}


\begin{prop}{}{}
        Let $H,H' \leq K$ and let \begin{equation}
                \map{f:H\times H' \rightarrow K}{(h,h') \mapsto hh'}
        \end{equation}
        be the multiplication map (not a homomorphism in general). Then the image of $f$ is \begin{equation}
                HH' = \{hh':h\in H, h' \in H'\}
        \end{equation}
        We then have that \begin{enumerate}
                \item \begin{enumerate}
                                \item $f$ is injective if and only if $H \cap H' = \{e_K\}$
                                \item $f$ is surjective if and only if $K = HH'$
                        \end{enumerate}
                \item $f$ is a group homomorphism from the direct product group $H \times H'$ to $K$ if and only if $hh' = h'h$ for all $h \in H$ and all $h' \in H'$
                \item $f$ is a group isomorphism if and only if $H \cap H' = \{e_K\}$, $HH' = K$, and $H,H' \vartriangleleft K$.
        \end{enumerate}
\end{prop}
\begin{proof*}{}{}
        [1. a)] First, let $x \in H \cap H'$ for $x \neq e_K$. Then we have that $f(x,e_K) = x = f(e_K, x)$, where $(x,e_K) \neq (e_K, x)$ by assumption, so $f$ is not injective. Conversely, suppose $H \cap H' = \{e_K\}$ and $(a,b) \in \ker(f)$ so $f(a,b) = e_K$. Then we have that $H \ni a = b^{-1} \in H'$ since it is a subgroup, so $a, b^{-1} \in H \cap H'$. In particular, $a, b^{-1} = e_K$, so $b = e_K$ as well. Thus, $\ker(f) = \{(e_K,e_K)\}$, which implies $f$ is injective.

        [1. b)] Note that $f$ is surjective if and only if $K = HH'$ by definition of $f$. 

        [2.] Let $h_1,h_2 \in H$, $h_1',h_2' \in H'$. Then $f$ is a homomorphism if and only if $$h_1h_1'h_2h_2' = f(h_1,h_1')f(h_2,h_2') = f(h_1h_2,h_1'h_2') = h_1h_2h_1'h_2'$$
        which holds if and only if $h_1'h_2 = h_2h_1'$. But, this is true for all $h_2 \in H$ and all $h_1' \in H'$, so the if and only if statement is true.

        [3.] Note $f$ is injective if and only if $H \cap H' = \{e_K\}$ by 1.a), and $f$ is surjective if and only if $HH' = K$, by 1.b). First, suppose $f$ is an isomorphism. Note that from the four fundamental makes of the group direct product we know that $H \times \{e_K\},\{e_K\}\times H' \vartriangleleft H \times H'$. Thus, since $f$ is assumed to be surjective and $f(H\times \{e_K\}) = H$, we have that $H \vartriangleleft K$, and similarly $H' \vartriangleleft K$. Conversely, suppose $H, H' \vartriangleleft K$. Then, let $h \in H$ and $h' \in H'$ and consider the commutator $[h,h'] = hh'h^{-1}{h'}^{-1}$. Since $H$ and $H'$ are normal we have that $H \ni h(h'h^{-1}{h'}^{-1}) = (hh'h^{-1}){h'}^{-1} \in H$. Hence, $[h,h'] \in H \cap H'$, but $H \cap H' = \{e_K\}$ so $[h,h'] = e_K$. Therefore, we have that $hh' = h'h$, so by 2. $f$ is a homomorphism, and since $f$ is shown to be injective and surjective, it is an isomorphism.
\end{proof*}

\begin{prop}{}{}
        Let $H,H' \leq K$. If $H$ (or $H'$) is a normal subgroup of $K$, then $HH'$ is a subgroup of $K$.
\end{prop}
\begin{proof*}{}{}
        (Left to the reader)
\end{proof*}

\begin{rmk}{}{}
        Note that the multiplication map can be bijective without being a homomorphism. For example, if we take $H = \langle x \rangle, H' = \langle y \rangle \in D_3$, and $H \cap H' = \{1\}$, $D_3 = HH'$, but $D_3 \cancel{\cong}\langle x \rangle \times \langle y \rangle$ because $\langle y \rangle $ is not a normal subgroup.
\end{rmk}

\begin{cor}{}{}
        Let $G$ be a finite group with $H,H' \leq G$. \begin{enumerate}
                \item If $H \cap H' = \{e_G\}$, then $|H||H'| = |HH'|$
                \item If $H \cap H' = \{e_G\}$, $H,H' \vartriangleleft G$, and $|G| = |H||H'|$, then \begin{equation}
                                G \cong H \times H'
                \end{equation}
        \end{enumerate}
\end{cor}
\begin{proof*}{}{}
        [1.] Suppose $H \cap H' = \{e_G\}$. Then by 1.a) of the previous proposition the multiplication map $f:H\times H' \rightarrow G$ is injective. Moreover, its image is precisely $HH' \subseteq G$. Thus, the corestriction $f:H\times H' \rightarrow HH'$ is a bijection. THerefore \begin{equation}
                |H||H'| = |H\times H'| = |HH'|
        \end{equation}

        [2.] Suppose $H \cap H' = \{e_G\}$, $H,H' \vartriangleleft G$, and $|G| = |H||H'|$. Since $H \cap H' = \{e_G\}$ we have by 1. that $|H||H'| = |HH'|$, so $|G| = |HH'|$ which implies $G = HH'$ as $HH' \subseteq G$. Thus, by 3. of the previous proposition $G \cong H \times H'$.
\end{proof*}


\begin{rmk}{Application}{}
        Suppose $G$ is abelian and $|G| = p^2$ for a prime $p$. Then either $G$ is cyclic or \begin{equation}
                G \cong \Z/p\Z\times \Z/p\Z
        \end{equation}
\end{rmk}
\begin{proof*}{}{}
        Assume $G$ is not cyclic. Then for all $g \in G$ with $g \neq e_G$ we have $o(g) = p$ by Lagrange's Theorem. Take $g,g' \in G$ such that $o(g) = o(g') = p$ and $g' \notin \langle g\rangle$, which is possible since there are $p$ elements not in $\langle g \rangle$ of order $p$. Let $H = \langle g \rangle$ and $H' = \langle g' \rangle$. Since $G$ is abelian $H$ and $H'$ are normal subgroups. Moreover, $H \cap H' = \{e_G\}$. Indeed, $H\cap H'$ is a subgroup of $H$ and $H'$, so $|H\cap H'| \in \{1,p\}$. But, if $|H\cap H'| = p$ then $H = H \cap H' = H'$, which implies that $g' \in H$, contradicting our initial assumption. Thus $|H \cap H'| = 1$ so $H\cap H' = \{e_G\}$. Finally, $|G| = p^2 = |H||H'|$. Thus, by 2. of the previous corollary we conclude that \begin{equation}
                G \cong H \times H' \cong \Z/p\Z \times \Z/p\Z
        \end{equation}
\end{proof*}



%%%%%%%%%%%%%%%%%%%%%%%%%%%%%%%%%%%%% Part 2
\part{Ring Theory}


%%%%%%%%%%%%%%%%%%%%%% - P2.Chapter 1
\chapter{\textsection Basic Definitions and Examples: Rings}

\section{Initial Definitions and Examples}



\begin{defn}{}{}
    A set $R$ with two binary operations $+$ (\Emph{addition}) and $\cdot$ (\Emph{multiplication}), is called a unital \Emph{ring} if the following are satisfied:
    \begin{enumerate}
        \item $(R,+)$ is an abelian group with identity $0$
        \item $(R,\cdot)$ is a monoid with identity $1$
        \item Distributivity: for all $a,b,c \in R$ $$(a + b)\cdot c = a\cdot c + b\cdot c$$
        and $$a\cdot (b+c) = a\cdot b + a\cdot c$$
    \end{enumerate}
\end{defn}

\begin{rmk}{}{}
    \leavevmode
    \begin{enumerate}
        \item The multiplicative identity $1$ is unique for a given ring $R$ and multiplication $\cdot$.
        \item We often denote $a \cdot b$ by $ab$
        \item For all $r \in R$, $0\cdot r = r \cdot 0 = 0$. Indeed, $r\cdot 0 = r\cdot (0+0) = r\cdot 0+r\cdot 0$, so $r \cdot 0 = 0$. $0\cdot r = 0$ is similar.
        \item For all $r \in R$, $(-1)\cdot r = -r$. Indeed, $r + (-1)\cdot r = 1\cdot r + (-1)\cdot r = (1+(-1))\cdot r = 0\cdot r = 0$, so $(-1)\cdot r = -r$.
        \item Powers in the group $(R,+)$ are denoted \begin{equation}
            na = \left\{\begin{array}{ll} 0, & \text{if } n = 0 \\ \underbrace{a+a + ... +a}_{\text{n-fold times}}, & \text{if } n > 0 \\ \underbrace{(-a)+ (-a) + ... + (-a)}_{\text{-n-fold times}}, & \text{if } n < 0\end{array}\right.
        \end{equation}
        Moreover, for all $m,n \in \Z$, and for all $a \in R$:
        \begin{enumerate}
            \item $m(na) = (mn)a$
            \item $(m+n)a = ma+na$
        \end{enumerate}
        and for all $n'\cdot a = na$ for all $n \in \Z$ for $n' \in R$ defined by: \begin{equation}
            n = \left\{\begin{array}{ll} 0, & \text{if } n = 0 \\ \underbrace{1_R+1_R + ... +1_R}_{\text{n-fold times}}, & \text{if } n > 0 \\ \underbrace{(-1_R)+ (-1_R) + ... + (-1_R)}_{\text{-n-fold times}}, & \text{if } n < 0\end{array}\right.
        \end{equation}
    \end{enumerate}
\end{rmk}

\begin{qest}
    How small can a ring be?
\end{qest}
\begin{ans*}{}{}
    The smallest ring is $R = \{0\}$, the zero ring, so $1 = 0$.
\end{ans*}

\begin{cons}{Endomorphism Rings}{}
    The best way to obtain rings (which are called \Emph{endomorphism rings}) is to start with an abelian group $(A,+,0)$. Let $R = \en(A) := \{f:A\rightarrow A: f \in \Hom_{\Grp}(A,A)\}$. We define addition on $R$ to be \begin{equation}
        (f+g)(x) := f(x)+g(x), \forall f,g \in R, \forall x \in A
    \end{equation}
    Since the addition in the group $A$ is commutative so is the addition on $R$. Moreover, we have zero element \begin{equation}
        0_R(x) = 0_A,\forall x \in A
    \end{equation}
    so $0_R  + f = f$. Additive inverses are defined such that $(-f)(a) = -(f(a))$. We define the multiplication law as \begin{equation}
        (f\times g)(a) = f(g(a))
    \end{equation}
    Then this operation is naturally associative, and the multiplicative identity is \begin{equation}
        1_R(a) = a, \forall a \in A
    \end{equation}
    Note that from these definitions, we see that multiplication is not necessarily commutative, and does not necessarily have an inverse, as f has an inverse $\iff$ it is an isomorphism of groups. That is, the group of units of $R$ is $R^{\times} = \aut(A)$ equipped with the multiplication operation of function composition.
\end{cons}

\begin{eg}{Constructing Rings from Endomorphisms on Cyclic Groups}{}
        I claim that the ring $(\Z,+,\cdot,1,0) = \en(\Z,+,0)$. Suppose we have a group homomorphism $f:\Z\rightarrow \Z$. Then $f(1) = n \in \Z$ determines everything, since $f(k) = f(1+1+...+1) = f(1)+...+f(1) = kf(1)$. We take $f$, and associate to it the integer $f(1)$, which then gives a multiplication on $\Z$. For example: Suppose $f(1) = n$, and $g(1) = m$, then $f\times g(k) = f(g(k)) = f(k\cdot m) = n(k\cdot m)$, which gives multiplication on $\Z$. Now, suppose $f$ is associated to a negative integer, so $f(1) = n < 0$, then it switches the halfs of the real line. Then, $f\times f(1) = f(f(1)) = f(n) = f(-1-1-...-1) = -f(1)-f(1)-...-f(1) = -n-n-...-n > 0$. Likewise, $\Z/n\Z = End(\Z/n\Z,+,0)$, where we identify $f$ by $f(1)$. This works to give a ring structure on cyclic groups.
\end{eg}

\begin{eg}{Constructing Rings from Endomorphisms of other Abelian group}{}
        Take $A = (\Z/p\Z)^2 = \{(a_1,a_2): a_i \in \Z/p\Z$. Then $End(A) = M_2(\Z/p\Z)$. If we have a matrix \begin{equation}
                B = \begin{bmatrix} \alpha & \beta \\ \gamma & \delta \end{bmatrix}
        \end{equation}
        matrix multiplication gives us the multiplication in our ring. This is an example of a non-commutative ring. In general, if $A = (\Z/p\Z)^n$, then $End(A) = M_n(\Z/p\Z)$.
\end{eg}


\begin{defn}{}{}
    The order of $1$ in $(R,+)$ is called the \Emph{characteristic} of the ring $R$, and denoted $\ch R$, if $o(1_R) < +\infty$. If $o(1) = +\infty$ then $\ch R := 0$. In general, for a non-unital ring $R'$, $\ch R'$ is the smallest positive integer $n$ such that $n\cdot r = 0_{R'}$ for all $r \in R'$. If no such $n$ exists then $\ch R' := 0$. 
\end{defn}


\begin{eg}{}{}
    \leavevmode
    \begin{enumerate}
        \item $\Z, \Q, \R, \C$ with the usual addition and multiplication are all rings of characteristic $0$.
        \item For all $n > 0$, $\Z/n\Z$ is a ring with addition and multiplication modulo $n$ of characteristic $n$.
        \item Let $M_n(\R)$ be the set of $n\times n$ real matrices. Then $M_n(\R)$ is a ring for the usual addition and multiplication of matrices, $0 = $ the zero matrix and $1 = I_n$. Indeed, $M_n(R)$ is a ring for any arbitrary ring $R$.
        \item Let $X$ be a set and $R$ a ring. Then, the set $F(X,R)$ of all functions $f:X\rightarrow R$ is a ring for pointwise addition and multiplication: \begin{equation}
            \begin{array}{l} (f_1+f_2)(x) := f_1(x)+f_2(x) \\ (f_1\cdot f_2)(x) := f_1(x)\cdot_R f_2(x) \end{array} \forall x \in X, \forall f_1,f_2 \in F(X,R)
        \end{equation}
        Moreover, $0(x) = 0$ for all $x \in X$ and $1(x) = 1_R$ for all $x \in X$.
        \item Let $X = \R$ and $R = \R$. A function $f \in F(\R,\R)$ is a polynomial if it can be written in the form \begin{equation}
            f(x) = a_nx^n+a_{n-1}x^{n-1}+...+a_1x+a_0
        \end{equation}
        for some $a_i \in \R$, $0 \leq i \leq n$ and $n \in \Z$, $n \geq 0$.
        \begin{enumerate}
            \item[$\drsh$] Polynomial functions $\mathcal{P}(\R,\R)$ form a ring for the addition and multiplication in $F(\R,\R)$ - This is a \Emph{subring}
            \item[$\drsh$] (eg: $f(x) = 2x^2+1$, $g(x) = 6x$, $f\cdot g(x) = 12x^3 + 6x$ is a polynomial)
        \end{enumerate}
        \item Let $G$ be an abelian group and let $\en(G)$ be the set of endomorphisms on $G$. Then, $\en(G)$ is a ring with addition defined pointwise and multiplication defined by function composition:
        \begin{equation}
            \begin{array}{l} (f_1+f_2)(g) := f_1(g)+f_2(g) \\ (f_1\circ f_2)(g) := f_1(f_2(g)) \end{array} \forall g \in G, \forall f_1,f_2 \in \en(G)
        \end{equation}
        Moreover, $0(g) = e_G$ and $1(g) = \id(g) = g$ for all $g \in G$.
    \end{enumerate}
\end{eg}

\begin{note*}{}{}
    The multiplication need not be commutative. For instance multiplication is not necessarily commutative in $M_n(\R)$ for $n \geq 2$.
\end{note*}


\begin{defn}{}{}
    If the multiplication of a ring $R$ is commutative ($ab = ba \forall a,b\in R$) then $R$ is said to be a \Emph{commutative ring}
\end{defn}


\begin{defn}{}{}
    Let $R$ be a ring, the set of invertible elements for the multiplication is called the \Emph{group of units} of $R$, and is denoted $R^{\times}$.
\end{defn}

\begin{rec}{}{}
    For all $a \in R$, $a$ is invertible for $\cdot$ if there exists $b \in R$ such that $ab = ba = 1_R$.
\end{rec}

\begin{xca*}{}{}
    $(R^{\times}, \cdot)$ is a group.
    \begin{proof*}{}{}
        (Left to the reader)
    \end{proof*}
\end{xca*}

\begin{eg}{}{}
    \leavevmode
    \begin{enumerate}
        \item We've seen $\R^{\times} = \R\backslash\{0\}$, $\C^{x} = \C\backslash\{0\}$, and $\Q^{\times} = \Q\backslash\{0\}$. But, in general, this is not enough. For instance, $(\Z/n\Z)^{\times} \neq \Z/n\Z\backslash\{[0]\}$ for all $n \geq 1$.
        \item $\Z^{\times} = \{1,-1\}\cong  \Z/2\Z$
        \item $M_n(\R)^{\times} = \GL_n(\R)$
        \item $F(X,\R)^{\times} = \{f\in F(X,\R):\forall x \in X, f(x) \neq 0\}$
        \item In general, $F(X,R)^{\times} = \{f \in F(X,R):\forall x \in X, f(x) \in R^{\times}\}$
    \end{enumerate}
\end{eg}

\begin{defn}{}{}
    If $R$ is a ring such that $R^{\times} = R\backslash\{0_R\}$ (that is every nonzero element is invertible for $\cdot$), then $R$ is called a \Emph{division ring}, or \Emph{skew-field}
    \begin{enumerate}
        \item[$\drsh$] If $R$ is a \Emph{commutative division ring} then $R$ is called a \Emph{field}. 
    \end{enumerate}
\end{defn}

\begin{eg}{}{}
    \leavevmode
    \begin{enumerate}
        \item $\Q, \R, \C$ are fields ($\Z$ is \underline{not} a field)
        \item $\Z/p\Z$ for $P$ a prime is a field, denoted $\F_p$, called the finite field of \Emph{order p}
        \item $\Z/n\Z$ is not a field if $n$ is not a prime
        \item Division rings which are note commutative rings are rare. An example of them are the \Emph{Quaternions}.
    \end{enumerate}
\end{eg}

\subsection{Integral Domains}

\begin{defn}{}{}
    For a ring $R$, $a \in R$ is called a \Emph{zero divisor} if there exists $b \in R$, $b \neq 0$, such that $ab = 0$ or $ba = 0$.
\end{defn}

\begin{rmk}{}{}
    \leavevmode
    \begin{enumerate}
        \item If $1 \neq 0$, then $0$ is a zero divisor.
        \item $\begin{pmatrix} 1 & 0 \\ 0 & 0\end{pmatrix}$ is a non-zero zero divisor of $M_2(\R)$.
        \item If $r$ is a divisor of $n$ and $r \neq 1$, then $[r]_n \in \Z/n\Z$ is a zero divisor. Indeed, $n = rq$ for $q \in \Z$. If $r = n$, $q = 1$, and $[r]_n[1]_n = [0]_n$, where $[1]_n \neq [0]_n$. Otherwise, if $1 < r < n$, $1 < q < n$, so $[q]_n \neq [0]$. But, $[r]_n[q]_n = [0]_n$ so $[r]_n$ is a zero divisor.
    \end{enumerate}
\end{rmk}

\begin{eg}{}{}
    \leavevmode
    \begin{enumerate}
        \item The set of zero divisors in a division ring is $K = \{0\}$. Indeed, if $a \in K$, $ab = 0$ for some $b \neq 0$, then $a = abb^{-1} = 0b^{-1} = 0$.
        \item The set of zero divisors of $\Z$ is $\{0\}$, even though $\Z$ is not a division ring.
    \end{enumerate}
\end{eg}

\begin{prop}{}{}
    Let $R$ be a ring. The following are equivalent:
    \begin{enumerate}
        \item $0\in R$ is the only zero divisor.
        \item For all $a,b \in R$, $ab = 0$ implies $a = 0$ or $b = 0$
        \item For all $a,b,c \in R$, $ab = ac$ and $a \neq 0$ implies $b = c$.
        \item For all $a,b,c \in R$, $ba = ca$ and $a \neq 0$ implies $b = c$
        \begin{enumerate}
            \item[$\drsh$] ((3) and (4) are called \Emph{cancellation laws})
        \end{enumerate}
    \end{enumerate}
    \begin{proof*}{}{}
        (Left to the reader)
    \end{proof*}
\end{prop}

\begin{defn}{}{}
    If the equivalent conditions of the proposition are satisfied for a ring $R \neq \{0\}$, then $R$ is called a \Emph{domain}. If $R$ is also a commutative ring then $R$ is said to be an \Emph{integral domain}.
\end{defn}

\begin{rmk}{}{}
    Every division ring is a domain and every field is an integral domain, but the converse is not true.
    \begin{enumerate}
        \item[$\drsh$] $\Z$ is an integral domain, but not a field. $\mathcal{P}(\R,\R)$ is an integral domain. 
    \end{enumerate}
\end{rmk}

\subsection{Subrings}

\begin{defn}{}{}
    A subset $S$ of a ring $R$ that is closed under addition, subtraction, multiplication, and contains $1$ is called a \Emph{subring} of $R$.
    \begin{enumerate}
        \item[$\drsh$] $\forall a,b \in S, \{a+b,a-b,ab,1\} \subseteq S$.
    \end{enumerate}
\end{defn}

\begin{rmk}{}{}
    In other words, $S$ is a subring of $R$ if and only if $(S,+)$ is a subgroup of $(R,+)$ and $(S,\cdot)$ is a monoid with identity $1 \in R$.
\end{rmk}

\begin{note*}{}{}
    \leavevmode
    \begin{enumerate}
        \item There is no standard notation for ``$S$ is a subring of $R$"
        \item The definition directly implies that the intersection of an arbitrary number of subrings is again a subring:
        \begin{proof*}{}{}
            (Left to the reader)
        \end{proof*}
    \end{enumerate}
\end{note*}

\begin{defn}{}{}
    The subring generated by a subset $X \subseteq R$ is the intersection of all subrings of $R$ containing $X$.
    \begin{enumerate}
        \item[$\drsh$] ($R$ is a subring of $R$, so there is always at least one subring of $R$ containing $X$ and the definition is well-defined)
    \end{enumerate}
\end{defn}

\begin{eg}{}{}
    \leavevmode
    \begin{enumerate}
        \item $\Z\subseteq \Q\subseteq \R\subseteq \C$ are all subrings
        \item $M_n(\Z) \subseteq M_n(\Q) \subseteq M_n(\R) \subseteq M_n(\C)$ are all subrings
        \item $\mathcal{P}(\R,\R) \subseteq F(\R,\R)$ is a subring
        \item The subring of $\C$ generated by $i$ is \begin{equation}
            \{a+bi:a,b \in \Z\}=:\Z[i] \subseteq \C
        \end{equation}
        and is called the \Emph{Gaussian integers}.
        \begin{xca*}{}{}
            $\Z[i]$ is an integral domain.
            \begin{proof*}{}{}
                    (Left to the reader)
            \end{proof*}
        \end{xca*}
        \item The subring of $\C$ generated by $\frac{1}{2}$ $$\{\frac{a}{2^n}:a \in \Z, n \geq 0\} \subseteq \Q \subseteq \C$$
        it is an integral domain as well
        \item The set of all upper triangular matrices, $T_n(\R)$, is a subring of $M_n(\R)$. In general, $T_n(R)$ is a subring of $M_n(R)$ for an arbitrary ring $R$.
        \item The \Emph{center} of a ring $R$ is \begin{equation}
            Z(R):= \{r \in R:ra =ar\forall a \in R\}
        \end{equation}
        It is a subring of $R$
        \begin{enumerate}
            \item[$\drsh$] If $b \in Z(R)$, $b$ is called a \Emph{central element} of $R$.
            \begin{eg}{}{}
                \begin{enumerate}
                    \item $Z(\R) = \R$, and similarly $Z(R) = R$ for any commutative ring $R$
                    \item $Z(M_n(\R)) = \R I_n$
                \end{enumerate}
            \end{eg}
        \end{enumerate}
        \item A subring of a field which is itself a field is called a \Emph{subfield}.
        \begin{enumerate}
            \item[$\drsh$] \begin{eg}{}{}
                $\Q\subseteq \R \subseteq \C$ are all subfields
            \end{eg} 
        \end{enumerate}
    \end{enumerate}
\end{eg}


\section{Ring Homomorphisms}

\begin{defn}{}{}
    Let $R,S$ be rings. A map $$R\xrightarrow{f}S$$
    is called a \Emph{ring homomorphism} if the following conditions are satisfied for all $r,r' \in R$:
    \begin{enumerate}
        \item $f(r+r') = f(r)+f(r')$
        \item $f(rr') = f(r)f(r')$
        \item $f(1_R) = 1_S$
    \end{enumerate}
    A bijective ring homomorphism $A\xrightarrow{\phi}B$ is called a \Emph{ring isomorphism}, and $A, B$ are said to be \Emph{isomorphic rings}.
    \begin{enumerate}
        \item[$\drsh$] (In the case that $f$ is surjective, $f(1_R) = 1_S$ follows from the multiplicative condition) 
    \end{enumerate}
\end{defn}

\begin{rmk}{}{}
    The image of a ring homomorphism is a subring of the codomain.
\end{rmk}

\begin{eg}{}{}
    \leavevmode
    \begin{enumerate}
        \item The identity $\id:R \rightarrow R$ is a ring isomorphism
        \item $\map{\Z\rightarrow R}{n\mapsto n\cdot 1_R}$ is a ring homomorphism
        \begin{enumerate}
            \item[$\drsh$] \begin{eg}{}{}
                $\map{\Z\rightarrow \Z/n\Z}{r \mapsto [r]_n}$
            \end{eg} 
        \end{enumerate}
        \item The inclusion $S \subseteq R$ of a subring is a ring homomorphism.
        \item Let $a \in X$. Then \begin{equation}
            \map{\ev_a:F(X,R)\rightarrow R}{f\mapsto f(a)}
        \end{equation}
        is a ring homomorphism called the \Emph{evaluation at $a$}
        \begin{enumerate}
            \item[$\drsh$] Indeed, $\ev_a(f+g) = (f+g)(a) = f(a)+g(a) = \ev_a(f) + \ev_a(g)$, $\ev_a(f\cdot g) = (f\cdot g)(a) = f(a)\cdot g(a) = \ev_a(f)\cdot \ev_a(g)$, and $\ev_a(1) = 1(a) = 1$.
        \end{enumerate}
        \item If $|R| = p$, a prime number, then $R$ is isomorphic to the field $\F_p = \Z/p\Z$.
        \begin{proof*}{}{}
            Suppose $R$ is a ring of order $p$. Then, note that by definition $\ch R = o(1_R)$ in $(R,+)$. Then, by Lagrange's Theorem $o(1_R)\;\vert\;p$. Hence, $o(1_R) \in \{1,p\}$. Note that if $o(1_R) = 1$ then $1_R = 0_R$ so $R$ is the zero ring and $|R| = 1$. But, as $1$ is not a prime integer this is impossible. Thus $o(1_R) = p$. Now, define the map $$\map{\Z/p\Z\rightarrow R}{ {[n]}_p \mapsto n\cdot 1_R}$$. First, if $[n]_p = [m]_p$ then $n - m\;\vert\;p$. Hence, $(n-m)\cdot 1_R = 0_R$ since $\ch R = p$. Thus, by distributivity $n\cdot 1_R -m\cdot 1_R = 0_R$. By addition of $m\cdot 1_R$ on both sides we find $n\cdot 1_R = m\cdot 1_R$. Thus, the map is well-defined. Moreover, $\phi([1]_p) = 1\cdot 1_R = 1_R$, and for all $[n]_p,[m]_p \in \F_p$, we have $$\phi([n+m]_p) = (n+m)\cdot1_R = n\cdot1_R + m\cdot1_R = \phi([n]_p) + \phi([m]_p)$$ and $$\phi([nm]_p) = (nm)\cdot1_R = n\cdot(m\cdot1_R) = (n\cdot 1_R)\cdot(m\cdot1_R) = \phi([n]_p)\cdot \phi([m]_p)$$
            Hence, we find that $\phi$ is a ring homomorphism. Finally, if $[k]_p \in \ker(\phi)$, then $k\;\vert\;p$, which implies $[k]_p = [0]_p$ so $\ker(\phi) = \{[0]_p\}$, and since both sets are finite (and of the same order) we conclude that $\phi$ is a bijection. Therefore, $\phi$ is a ring isomorphism so $R \cong \F_p$, as claimed.
        \end{proof*}
        \item $\map{\C\xrightarrow{\theta}M_2(\R)}{a+bi\mapsto \begin{bmatrix} a & -b \\ b & a\end{bmatrix}}$ is an injective ring homomorphism.
        \begin{enumerate}
            \item[$\drsh$] Hence, $\C \xrightarrow{\sim} \left\{\begin{bmatrix} a & -b \\ b & a\end{bmatrix}:a,b \in \R\right\}\subseteq M_2(\R)$ 
        \end{enumerate}
    \end{enumerate}
\end{eg}

\begin{rmk}{}{}
    The property of being a field or an integral domain is preserved under ring isomorphism.
\end{rmk}


\begin{rmk}{Canonical Map}{}
        If we have a commutative ring $R$, there is a natural ring homomorphism $f:\Z\rightarrow R$ which is completely characterized by $f(1) = 1_R$, so for $n \geq 1$, $f(n) = f(\underbrace{1+1+...+1}_{\text{n-times}}) = \underbrace{1_R+...+1_R}_{\text{n-times}}$ and $f(-n) = -f(n)$. This is the canonical ring homomorphism associated to any commutative ring. Moreover, we know that the kernel of $f$ is an ideal of $\Z$, so it is of the form $n\Z$ for some $n \geq 0$. If $R = \{0\}$, then $\ker(f) =\Z$, and if $R = \Z,\Q,\R,\C$, then $\ker(f) = 0\Z = \{0\}$. Moreover, if $R= \Z/n\Z$, then $\ker(f) = n\Z$.
\end{rmk}

\begin{rmk}{Think about this}{}
        If $R$ is a field, and $h$ is the natural homomorphism given above, then $\ker h = \{0_R\}$, or $\ker h = p\Z$, where $p$ is prime.
\end{rmk}


\begin{prop}{}{}
        If $R$ is a field, then $\ker(f) = \{0\}$ or $\ker(f) = p\Z$ for $p$ a prime.
\end{prop}
\begin{proof*}{}{}
        For the sake of contradiction suppose $\ker(f) = n\Z$ for $n = ab$ composite, so $1 < a,b < n$. Then $f(n) = 0$ in $R$. But, $f(n) = f(a)f(b) = a_Rb_R = 0_R$, so since $R$ is a field, $a_R$ is zero or $b_R$ is zero. However, this contradicts the fact that the $\ker(f)$ is a multiple of $n$, and $a,b \notin n\Z$.
\end{proof*}





\section{Domains and Fields of Fractions}

\begin{prop}{}{}
    The characteristic of a domain is zero ($o(1) = \infty$) or a prime number $p$.
    \begin{proof*}{}{}
        Suppose that $R$ is a domain. If $R$ has a zero characteristic we are done, so suppose $\ch R = n$ where $n \in \Z$ and $n \geq 1$. If $n = 1$ then $R = \{0\}$, which contradicts the fact that $R$ is a domain. Thus, $n > 1$. We argue by contradiction and suppose $n$ is not prime. Then there exist $r,s \in \Z$ with $1 < r,s < n$ such that $n = rs$. It follows that $(r\cdot 1_R)(s\cdot 1_R) = rs\cdot 1_R = n\cdot 1_R = 0_R$ by definition of the characteristic of a ring. However, since $R$ is a domain it follows that $r\cdot 1_R = 0_R$ or $s \cdot 1_R = 0_R$. However, $r,s < n$, so either case would contradict the minimality of $n$ in $o(1) = n$. Thus, $n$ being composite leads to a contradiction so we conclude that $n$ must be prime, as claimed.
    \end{proof*}
\end{prop}

\begin{rmk}{}{}
    Every subring of a field is an integral domain.
    \begin{enumerate}
        \item[$\drsh$] ($R\subseteq F_{field}$, then for $a,b \in R$, if $ab = 0$ in $F$ and $b \neq 0$, then $0 = abb^{-1} = a \in F$. Thus $a = 0 \in R$. Hence, $R$ is an integral domain)
    \end{enumerate}
\end{rmk}

\begin{rmk}{}{}
    Actually, every subring of a division ring, or skew-field, is a domain.
\end{rmk}

\begin{thm}{}{}
    Every integral domain is a subring of a field.
\end{thm}

\begin{cons}{}{}
    Denote $R\backslash\{0\} = R^*$. We start with an integral domain $R$, and consider pairs $(a,b) \in R\times R^*$. We define a relation $\sim$ on $R\times R^*$ by $(a,b) \sim (a',b')$ if and only if $ab' = a'b$.
    \begin{claim}{}{}
        $\sim$ is an equivalence relation on $R\times R^*$.
    \end{claim}
    \begin{proof*}{}{}
        Let $(a,b),(a',b'), (a'',b'') \in R\times R*$. First, $(a,b)\sim (a,b)$ since by reflexitivity of ``$=$" $ab = ab$, so $\sim$ is reflexive. Then, suppose $(a,b)\sim (a',b')$, so $ab' = a'b$. By the symmetry of ``$=$" we have $a'b = ab'$, so $(a',b') \sim (a,b)$. Hence $\sim$ is symmetric. Now, suppose $(a',b') \sim (a'',b'')$, so $a'b'' = a''b'$. Then observe that \begin{align*}
            ab''a' &= aa''b'\\
            &= ab'a'' \tag{commutivity}\\
            &= a'ba'' \\
            &= a''ba' \tag{commutivity} \\
        \end{align*}
        Then, we have that $(ab'' - a''b)a' = 0_R$ by distributivity. Note if $a' = 0_R$, then $a''b' = a'b'' = 0_R$, so $a'' = 0_R$ since $R$ is an integral domain, and similarly $ab' = a'b = 0_R$ so $a=0$ and $ab'' = 0_R = a''b$. Now, suppose $a' \neq 0$. Then as $R$ is an integral domain $ab'' = a''b$, so $(a,b) \sim (a'',b'')$ and the relation is transitive, as desired. Therefore, $\sim$ is an equivalence relation of $R\times R^*$.
    \end{proof*}
    \begin{enumerate}
        \item[$\drsh$] We define addition and multiplication on the set \begin{equation}
            Frac(R) := \{[(a,b)]_{\sim}:(a,b) \in R\times R^*\}
        \end{equation}
        of equivalence classes by \begin{equation}
            \begin{array}{l}
                [(a,b)] + [(c,d)] := [(ad+cb,bd)] \\
                {[(a,b)]}\cdot [(c,d)] := [(ac,bd)] 
            \end{array}
        \end{equation}
    \end{enumerate}
    \begin{enumerate}
        \item[$\drsh$] Let us see that these operations are well-defined. Consider $[(ad+cb,bd)]$ and $[(a'd'+c'b',b'd')]$ for $[(a,b)] = [(a',b')]$ and $[(c,d)] = [(c',d')]$. Then, observe that \begin{align*}
            (ad+cb)(b'd') - (a'd'+c'b')(bd) &= adb'd'+cbb'd' - a'd'bd - c'b'bd \\
            &= a'bdd' + c'dbb' - a'bdd' - c'dbb' \\
            &= 0_R
        \end{align*}
        so $[(ad+cb,bd)]=[(a'd'+c'b',b'd')]$ and the addition is well defined. Similarly, \begin{align*}
            (ac)(b'd')-(a'c')(bd) &= acb'd'-a'c'bd \\
            &= a'bcd' - a'bcd' \\
            &= 0_R
        \end{align*}
        so $[(ac,bd)] = [(a'c',b'd')]$, and multiplication is also well-defined. Furthermore, we observe that $0_{Frac(R)} = [(0_R,b)]$ since $0_R\cdot b = 0_R = 0_R \cdot b'$ for all $b,b' \in R^*$. Additionally, $-[(a,b)] = [(-a,b)]$ and $1_{Frac(R)} = [(1,1)]$. Note that it is a tedious but rudimentary check to see that $Frac(R)$ as defined is a commutative ring.
        \begin{claim}{}{}
            $Frac(R)$ is a field.
            \begin{proof*}{}{}
                    Since $Frac(R)$ is a commutative ring, all we must show is that all non-zero elements are invertible. Indeed, $[(a,b)] \neq 0_{Frac(R)}$ if and only if $a \neq 0$. Hence $a \in R^*$ so $[(b,a)] \in Frac(R)$ is well-defined. Then, $(ab,ba) \sim (1,1)$ since $ab = ba$  by commutivity, so $[(a,b)]^{-1} = [(b,a)]$, and in particular $[(a,b)]$ is invertible. Thus, $Frac(R)$ is a field, as claimed.
            \end{proof*}
        \end{claim}
    \end{enumerate}
\end{cons}

\begin{namthm*}{Universal Property of Field of Fractions}{}
    Let $R$ be an integral domain.
    \begin{enumerate}
        \item $Frac(R)$ is a field containing $R$ as a subring by the inclusion \begin{equation}
            \map{R\xhookrightarrow{i} Frac(R)}{r\mapsto [(r,1)]}
        \end{equation}
        \item If $R \xhookrightarrow{j} \F$ is an injective ring homomorphism of rings with $\F$ a field, then there exists a unique injective homomorphism of rings $Frac(R) \xrightarrow{f}\F$ with $f \circ i = j$:
        \begin{center}
            \begin{tikzpicture}[baseline = (a).base]
            \node[scale = 1] (a) at (0,0){
                \begin{tikzcd}
                    R \ar[dr, hook, "i", swap] \ar[rr, hook, "j"] & & \F \\
                    &Frac(R) \ar[ur, dashed, "\exists!f", swap] &
                \end{tikzcd}
            };
            \end{tikzpicture}
        \end{center}
    \end{enumerate}
\end{namthm*}
\begin{proof*}{}{}
    \begin{enumerate}
        \item Define $i$ as above. Then, observe that $i(1) = [(1,1)] = 1_{Frac(R)}$, $i(a+b) = [(a+b,1)] = [(a,1)] + [(b,1)] = i(a) + i(b)$, and $i(ab) = [(ab,1)] = [(a,1)][(b,1)] = i(a)i(b)$. Moreover, $i(a) = 0_{Frac(R)}$ if and only if $a = 0_R$, so $i$ is an injective ring homomorphism, as desired.
        \item Suppose $j:R\hookrightarrow \F$ is an injective ring homomorphism and define $f:Frac(R) \rightarrow \F$ by $f([(a,b)]) := j(a)j(b)^{-1}$. Note that since $b \neq 0_R$ and $j$ is injective, $j(b) \neq 0_{\F}$ so $j(b) \in \F^{\times}$. Now, suppose $[(a,b)] = [(a',b')]$, so $ab' = a'b$. Then $j(a)j(b') = j(a')j(b)$ y multiplicativity. It follows that $j(a)j(b)^{-1} = j(a')j(b')^{-1}$, so the map $f$ is well-defined. Moreover, $f([(1,1)]) = j(1)j(1)^{-1} = 1_{\F}$, \begin{align*}
            f([(aa',bb')]) &= j(aa')j(bb')^{-1} \\
            &= (j(a)j(b)^{-1})(j(a')j(b')^{-1}) \\
            &= f([(a,b)])f([(a',b')])
        \end{align*}
        and \begin{align*}
            f([(ab' + a'b,bb')]) &= j(ab'+a'b)j(bb')^{-1} \\
            &= (j(a)j(b')j(b)^{-1}j(b')^{-1}+j(a')j(b)j(b)^{-1}j(b')^{-1}) \\
            &= j(a)j(b)^{-1} + j(a')j(b')^{-1} \\
            &= f([(a,b)])+f([(a',b')])
        \end{align*}
        Hence, $f$ is a ring homomorphism. Moreover, for all $a \in R$, $f\circ i(a) = f([(a,1)]) = j(a)j(1)^{-1} = j(a)$, so $f\circ i = j$ as desired. Injectivity shall be shown by the Lemma to follow. Now, suppose $[(a,b)] \in Frac(R)$. Then, observe that \begin{align*}
            g([(a,b)]) &= g([(a,1)])g([(1,b)] \tag{by multiplicativity} \\
            &= (g\circ i(a))g([(b,1)])^{-1} \tag{by multiplicativity and $b \neq 0$} \\
            &= j(a)(g\circ i(b))^{-1} \\
            &= j(a)j(b)^{-1} \\
            &= f([(a,b)]) \tag{by definition}
        \end{align*}
        Thus we have that $f = g$, so the map is unique.
    \end{enumerate}
\end{proof*}

\begin{lem}{}{}
    Let $K\xrightarrow{f} R$ be a ring homomorphism with $K$ a field and $R \neq \{0\}$. Then $f$ is injective.
    \begin{proof*}{}{}
        Take $a \neq 0, a \in K$. Then $1_R = f(1_K) = f(aa^{-1}) = f(a)f(a^{-1})$. If $f(a) = 0$ then $1 = 0$, but by assumption $R \neq \{0\}$ so $f(a) \neq 0$. Then $a \notin \ker(f)$, and in particular $\ker(f) = \{0_K\}$. Thus, $f$ is injective.
    \end{proof*}
\end{lem}

\begin{eg}{}{}
    \leavevmode
    \begin{enumerate}
        \item $Frac(\Z) \cong \Q$
        \item $Frac(\F) \cong \F$ for any field $\F$.
        \begin{proof*}{}{}
            Indeed, consider the identity $Id_{\F}:\F\rightarrow \F$, which is an injective ring homomorphism. Then, by the universal property of $\F$'s field of fractions there exists a unique injective ring homomorphism $j:Frac(\F) \rightarrow \F$ such that $j \circ i = Id_{\F}$. Then, let $f \in \F$, and observe that $j(i(f)) = Id_{\F}(f) = f$, so $j$ is also a surjection. Thus, $j$ is an isomorphism of rings, so $Frac(\F) \cong \F$ as claimed.
        \end{proof*}
        \item $Frac(\Z[1/2]) \cong \Q$
        \item $Frac(\Z[i]) \cong \Q[i]$. Indeed, by the universal property we have an injection $Frac(\Z[i])\hookrightarrow \Q[i]$, and $\Q \hookrightarrow Frac(\Z[i])$, which implies $\Q[i] \hookrightarrow Frac(\Z[i])$ since $i \in Frac(\Z[i])$.
        \item For $R = \mathcal{P}(\R,\R)$, \begin{equation}
            Frac(R) = \{f/g: f,g \in \mathcal{P}(\R,\R), g \neq 0\}
        \end{equation}
        is called the field of \Emph{rational polynomials}.
    \end{enumerate}
\end{eg}


\section{Special Definitions and Facts}

\begin{defn}{}{}
    A ring $R$ is said to be a \Emph{local ring} if the set of non-units in $R$ is an ideal.
\end{defn}

\begin{prop}{}{}
    If $R$ is a local ring with ideal of non-units $J(R)$, then $R/J(R)$ is a division ring.
    \begin{proof*}{}{}
        Let $R$ be a local ring with ideal of non-units $J(R)$. Then, let $a+J(R) \in R/J(R)$ such that $a+J(R) \neq 0_{R/J(R)}$, so $a \notin J(R)$. It follows that $a$ is a unit of $R$ so there exists $b \in R$ such that either $ab = 1_R$ or $ba = 1_R$. Without loss of generality suppose $ab = 1_R$. Then it follows that $(a+J(R))(b+J(R)) = ab+J(R) = 1_R + J(R) = 1_{R/J(R)}$ in $R/J(R)$. Thus, every non-zero element in $R/J(R)$ has an inverse so $R/J(R)$ is a division ring.
    \end{proof*}
\end{prop}

\begin{prop}{}{}
    If $R$ is a local ring with ideal of non-units $J(R)$, and $A \subseteq J(R)$ is an ideal of $R$, then $R/A$ is local and $J(R/A) = \{r+A:r \in J(R)\}$.
    \begin{proof*}{}{}
        Let $R$ be a local ring with ideal of non-units $J(R)$, and let $A \subseteq J(R)$ be an ideal of $R$. Then, by the Correspondence Theorem for quotient rings we have that $J(R)/A = \{r + A: r \in J(R)\}$ is an ideal of $R/A$. I claim that $J(R)/A = J(R/A)$ in the proposition. Let $r+A \in R/A$ be a non-unit. For the sake of contradiction suppose that $r+A \notin J(R)/A$. Then $r \notin J(R)$, so $r$ must be a unit of $r$. Then there exists $r' \in R$ such that $rr' = 1_R$ or $r'r = 1_R$. Without loss of generality suppose $rr' = 1_R$. Then $(r+A)(r'+A) = rr'+A = 1_R+A$, so $(r+A)$ is a unit of $R/A$, which contradicts the assumption that $r+A$ is a non-unit. Thus, $r+A \in J(R)/A$, so $J(R/A) \subseteq J(R)/A$. Next, let $r + A \in J(R)/A$, so $r \in J(R)$. Again towards a contradiction suppose $r + A \notin J(R/A)$. Then there exists $r'+A \in R/A$ such that $(r+A)(r'+A) = 1_R+A$ or $(r'+A)(r+A) = 1_R +A$. Without loss of generality suppose $(r+A)(r'+A) = 1_R +A$, so $rr' + A = 1_R + A$. Note that since $J(R)$ is an ideal, $rr' \in J(R)$ so $rr' + A \in J(R)/A$. Then we have that $rr' - 1_R \in A \subseteq J(R)$, so $1_R = rr' + (-(rr' - 1_R)) \in J(R)$. However, $1_R$ is a unit in $R$, and $J(R)$ is the set of non-units which is a contraction. Thus, we conclude that $r+A \in J(R/A)$, so $J(R)/A \subseteq J(R/A)$. Hence, $J(R/A) = J(R)/A$ is an ideal, so $R/A$ is a local ring as claimed. 
    \end{proof*}
\end{prop}


\section{The Gaussian Integers}

\begin{eg}{}{}
        Consider $R = \Z[i]$. What if we want $2+i = 0$? Let $I = (2+i)$ and take $\overline{R} = R/I$. We wish to identify $\overline{R}$. First, let's identify the intersection $I \cap \Z$. Note that $0 = (2+i)(2-i) = 4+1 = 5$, so $5 \in I\cap \Z$. In particular, $5\Z \subset I\cap \Z$, where $5\Z$ is a maximal subgroup of $\Z$, so in fact it is a maximal ideal. Therefore, either $I\cap \Z = \Z$ or $I\cap \Z = 5\Z$. Secondly, observe that if $(2+i)(a+bi) \in \Z$, then $(2a-b)+(2b+a)i \in \Z$. In particular, $2b+a = 0$, so $a = -2b$. It follows that $(2+i)(a+bi) = 2(-2b)-b = -4b - b = 5(-b) \in 5\Z$. Therefore, $I\cap \Z = 5\Z$. Then, if we take the canonical homomorphism $\Z\rightarrow R/I = \overline{R}$, it has kernel $5\Z$, and image $\cong \Z/5\Z$. In fact, $\overline{R} \cong \Z/5\Z$ under this map, or in other words, the map is surjective. Note that since $2+i \equiv 0 \mod I$, $i \equiv -2 \mod I$, and $a+bi \equiv a-2b \mod I$ in $R/I$, but $a-2b \in \Z$. Thus, the map is surjective, so the image of the integers, $\Z/5\Z$, must be isomorphic to $R/I$.
\end{eg}

\begin{thm}{}{}
        More generally, if $p$ is a prime number with $p\equiv 1 \mod 4$, there is an ideal $I \subset \Z[i]=R$ with $R/I \cong \Z/p\Z$.
\end{thm}
\begin{proof*}{}{}
        First, note that for the canonical homomorphism $f:R\rightarrow R/I$, if $R/I \cong \Z/p\Z$, then $f(i)$ must have order $4$ multiplicatively since $i^4 = 1$ and $i^2 = -1$, so $f(i)^2 \cong -1 \mod p$. If $f(i)$ has order $4$ in $(\Z/p\Z)^*$, then $p \equiv 1 \mod 4$. Then, recall by Wilson's Theorem that $(p-1)! \equiv -1 \mod p$. Now, consider the element $\left(\frac{p-1}{2}\right)!$, and complete it $1*2*...*\frac{p-1}{2}*\frac{p+3}{2}*...*(p-2)*(p-1) \cong -1 \mod p$. But, the terms in the first half are minus the terms in the second, so the product of the first half is equal to that of the second half times the number of minus signs. Note that the number of minus signs is $(-1)^{\frac{p-1}{2}}$, and since $p \equiv 1 \mod 4$, $\frac{p-1}{2}$ is even. Thus, the product of the first half is equal to the product of the second half. Hence, the square of $a\equiv\left(\frac{p-1}{2}\right)!$ is $-1$, so it is our element of order $4$. Then, let $I$ be the ideal generated by $p$ and $i-a$, so $I = (p,i-a)$. First, note that $I\cap \Z \supset p\Z$. Moreover, $(i-a)(b+ci) = (-ab-c)+(-ac+b)i$, where $-ac +b = 0$, so $-ab-c = -a^2c-c = -c(a^2+1)$. But, $a^2 \cong -1 \mod p$, so $a^2 + 1 \cong 0 \mod p$. Thus, $-c(a^2+1) \in p\Z$. Hence, $\Z\rightarrow R/I$ is surjective, as $i\cong a \in R/I$, with kernel $p\Z$, so $R/I\cong \Z/p\Z$.
\end{proof*}


\begin{thm}{Gauss's Theorem}{}
        For $R = \Z[i]$, every ideal $I \in R$ is principal.
\end{thm}

\begin{cor}{}{}
        Since every $I \subset \Z[i]$ is principal, so is $(p,i-a)$, which implies $(p,i-a) = (a+bi)$ for some $a+bi \in \Z[i]$, and from above, $R/(a+bi) \cong \Z/p\Z$. This implies that $a^2+b^2 = p$.
\end{cor}


\begin{thm}{Fermat's Theorem}{}
        For any prime number $p$ such that $p \equiv 1 \mod 4$, $p = a^2 + b^2$ for some $a,b \in \Z$.
\end{thm}


\begin{rmk}{}{}
        Gauss showed that if you can write all primes $p\equiv 1 \mod 4$ as $p = a^2 +b^2$, then every ideal $(a+bi) \subset \Z[i]$ must be principal.
\end{rmk}

\begin{rmk}{}{}
        The first step to prove Gauss's theorem is to show that for all prime $p\equiv 1 \mod 4$, there is an ideal $(a+bi)$ so that $R/(a+bi) \cong \Z/p\Z$, then the second step is to show that all ideals are principal, and then the third step is to show that if you have a quotient $R/(a+bi) \cong \Z/p\Z$, $a^2+b^2 = p$.
\end{rmk}


\begin{rmk}{}{}
        More generally, the order of the finite ring $R/(a+bi)$ is $a^2+b^2 = (a+bi)(a-bi)$ providing that $(a+bi) \neq (0)$.
\end{rmk}



%%%%%%%%%%%%%%%%%%%%%% - P2.Chapter 2
\chapter{\textsection Ideals and Quotient Rings}

\section{Basic Definitions and Examples: Ideals}


\begin{rmk}{}{}
    Let $f:R\rightarrow R'$ be a ring homomorphism, then $\ker(f)$ is a subgroup of $(R,+)$. Moreover, for all $r_1,r_2 \in R$ and all $k \in \ker(f)$, $r_1kr_2 \in \ker(f)$. Indeed, $f(k) = 0_{R'}$ so $f(r_1kr_2) = f(r_1)f(k)f(r_2) = f(r_1)0_{R'}f(r_2) = 0_{R'}$.
\end{rmk}

\begin{defn}{}{}
    Let $R$ be a ring. A subgroup $I$ of $(R,+)$ is called an \Emph{ideal of $R$} if for all $r_1,r_2 \in R$, and all $i \in I$, \begin{equation}
        r_1ir_2 \in I
    \end{equation}
     \begin{enumerate}
         \item[$\drsh$] (This is actually the definition of a two-sided ideal, and the definitions of left and right sided ideals can be derived by relaxing the condition in this definition) 
     \end{enumerate}
\end{defn}

\begin{eg}{}{}
    \leavevmode
    \begin{enumerate}
        \item For an arbitrary ring $R$, $\{0\}$ and $R$ are ideals of $R$.
        \item If $R$ is a commutative ring, then \begin{equation}
            aR := \{ar \in R: r \in R\}
        \end{equation}
        is an ideal of $R$ for all $a \in R$. $I$ is called the \Emph{principal ideal generated by a}, and is denoted $(a)$.
        \item $\ker(\ev_2:\R[x]\rightarrow \R)$ is an ideal of $\R[x]$. Indeed, $\ker(\ev_2) = (x-2)$.
        \item For a ring $R$, the ideal generated by a subset $X \subseteq R$ is given by \begin{equation}
            (X) := \left\{\sum\limits_{i=1}^na_ix_ib_i: n \geq 1, a_i,b_i \in R, x_i \in X\right\}
        \end{equation}
        \item Every ideal of $\Z$ is principal. Indeed, every ideal is a subgroup of $(\Z,+)$, and every subgroup of $(\Z,+)$ is cyclic so every ideal is principal. Moreover, every subgroup is an ideal. Indeed, we know that $n\Z$ is an ideal for all $n \in \Z$, but $n\Z$ is precisely the form for subgroups of $\Z$, so all subgroups are principal ideals of $\Z$ and all ideals of $\Z$ are principal.
        \item The ideal $(2,X)$ of $\Z[X]$ is not a principal ideal.
        \begin{proof*}{}{}
            For the sake of contradiction suppose $(2,X) = (p)$ for some $p \in \Z[X]$. Then $2 = pf$ for some $f \in \Z[X]$. Then $\deg(p) + \deg(f) \leq 0$, so $\deg(p) = 0$. Then $p = n \in \Z$ such that $n\;\vert\;2$. Hence, $p = 2$ or $p = 1$. But, $1 \notin (2,X)$ since $2 \;\cancel{\vert}\;1$ and $X\;\cancel{\vert}\;1$ in $\Z$. Thus, $p = 2$. However, $2\;\cancel{\vert}\;X$ in $\Z[X]$, which contradicts the assumption that $(2,X) = (p)$. Thus, $(2,X)$ can not be principal in $\Z[X].$
        \end{proof*}
    \end{enumerate}
\end{eg}


\begin{defn}{}{}
    Let $I \subseteq R$ be an ideal of a ring $R$. The quotient group $R/I$, for the additive group $(R,+)$, is a ring for the multiplication \begin{equation}
        (a+I)(b+I) = ab+I \tag{$\star$}\label{eq:quot_ring}
    \end{equation}
\end{defn}

\begin{thm}{}{}
    The addition on $R/I$ and the multiplication given by (\ref{eq:quot_ring}) makes $R/I$ into a ring such that the canonical quotient map \begin{equation}
        \map{\pi: R \twoheadrightarrow R/I}{r\mapsto r+I}
    \end{equation}
    is a surjective ring homomorphism of kernel $I$. We call $R/I$ the \Emph{quotient ring} of $R$ by \\$I$.
    \begin{proof*}{}{}
        (Left to the reader)
    \end{proof*}
\end{thm}

\begin{thm}{}{}
        Any ideal $I$ is the kernel of a natural ring homomorphism $R \rightarrow R/I$, where $R/I$ is the quotient ring, taking $a \mapsto a+I$. 
\end{thm}

\begin{rmk}{}{}
    If $R$ is a commutative ring then so is $R/I$ for all ideals $I$ of $R$.
\end{rmk}

\begin{eg}{}{}
        Consider $R = \Z/4\Z$. Then some ideals of $R$ are, $I = (0),(1),(2) = \{0,2\}$. In general, for $R = \Z/n\Z$ we have the ideal $(d)$ for all $d \;\vert\;n$. In particular, if $R = \Z/p^k\Z$, then the distinct ideals are $$(1)\supset(p)\supset(p^2)\supset...\supset(p^k)=(0)$$
\end{eg}


\begin{rmk}{}{}
        In general the set of ideals form a lattice for the ring.
\end{rmk}


\begin{eg}{}{}
    \leavevmode
    \begin{enumerate}
        \item For any ring $R$, $R/(0) \cong R$ and $R/R \cong \{0\}$.
        \item $\Z/n\Z$, where $n\Z = (n)$, is the ring of integers modulo n.
        \begin{enumerate}
            \item[$\drsh$] By the correspondence theorem, we have that ideals of $\Z$ containing $n\Z$ correspond to ideals of $\Z/n\Z$, which is to say, all ideals of the form $m\Z$ for $m \;\vert\;n$.
        \end{enumerate}
        \item Let $R$ be a commutative ring. Then $\ev_a:R[x]\rightarrow R$ for $a \in R$ is a surjective ring homomorphism, so by the First Isomorphism Theorem for rings and the fact that $\ker(\ev_a) = (x-a)$, we have that \begin{equation}
            \overline{\ev_a}: R[x]/(x-a) \xrightarrow{\sim} R
        \end{equation}
        is a ring isomorphism. Note by the correspondence theorem, ideals of $R[x]$ containing $(x-a)$ correspond to ideals of $R$. If $R = \F$ a field, then the only ideals of $\F$ are $\{0\}$ and $\F$, which implies $(x-a)$ is \Emph{maximal}. Thus, the only ideals of $\F[x]$ containing $(x-a)$ are $(x-a)$ and $\F[x]$.
        \item For $X$ a set and $z \in X$, $$\ev_z:F(X,R) \rightarrow R$$
        is a surjective ring homomorphism of kernel $$\ker(\ev_z) = \{f:X\rightarrow R\vert f(z) = 0_R\}$$
    \end{enumerate}
\end{eg}

\begin{defn}{}{}
    Let $I$ be an ideal of $R$, $a,b \in R$. Then $a + I \in R/I$ is called the \Emph{(congruence) class} of $a$ modulo $I$ (sometimes denoted $a \mod I$). If $a + I = b + I$, so $b - a \in I$, we say that $a$ and $b$ are \Emph{congruent modulo $I$}, written \begin{equation}
        a \equiv b \mod I, a = b \mod I, a = b (I)
    \end{equation}
\end{defn}

\subsection{Simple Ideals}

\begin{defn}{}{}
    A ring $R$ is \Emph{simple} if $R \neq \{0\}$ and if the only ideals of $R$ are $\{0\}$ and $R$. That is, $R$ has exactly two ideals.
\end{defn}

\begin{eg}{}{}
    \leavevmode
    \begin{enumerate}
        \item Division rings are simple. If $a \neq 0$ in an ideal $I$ of a division ring $R$, then $1 = a^{-1}a \in I$, so $I = (1) = R$. In particular, fields are simple rings.
        \item $M_n(R)$ for $R$ an arbitrary ring is simple if and only if $R$ is simple.
        \begin{proof*}{}{}
            Suppose $R$ is simple and let $\mathcal{A}$ be an ideal of $M_n(R)$. Then by the following lemma $\mathcal{A} = M_n(A)$ for some ideal $A$ of $R$. Thus, $A = \{0\}$ or $A = R$, so $\mathcal{A} = M_n(0) = \{0\}$ or $\mathcal{A} = M_n(R)$. Consequently, $M_n(R)$ is simple. Now, suppose $M_n(R)$ is simple and let $A$ be an ideal of $R$. Then $M_n(A)$ is an ideal of $M_n(R)$. Hence, $M_n(A) = \{0\}$ or $M_n(A) = M_n(R)$. Thus, $A = \{0\}$ or $A = R$, so $R$ is a simple ring as claimed.
        \end{proof*}
    \end{enumerate}
\end{eg}

\begin{lem}{}{}
    For a ring $R$, every ideal of $M_n(R)$ has the form $M_n(A)$ for an ideal $A$ of $R$.
\end{lem}

\begin{prop}{}{}
    A commutative ring $R$ is simple if and only if it is a field.
    \begin{proof*}{}{}
        If $R$ is a field then it is simple by the previous example. Now, suppose $R$ is a commutative simple ring, and let $0 \neq r \in R$. Consider the ideal $(r) = rR$. Since $R$ is simple, $(r) = R$ since $(r) \neq \{0\}$. Hence, $1 \in (r)$, so there exists $r' \in R$ such that $rr' = 1$. Hence, $r$ is a unit of $R$, so in particular $R$ is a field.
    \end{proof*}
\end{prop}

\subsection{Maximal and Prime Ideals}

\begin{defn}{}{}
    An ideal $I \subseteq R$ in a ring $R$ is called \Emph{maximal} if \begin{enumerate}
        \item $I \neq R$
        \item For all ideals $J \subseteq R$, if $I \subseteq J$, then either $I = J$ or $J = R$.
    \end{enumerate}
\end{defn}

\begin{prop}{}{}
    An ideal $I \subseteq R$ is maximal if and only if the quotient ring $R/I$ is simple.
    \begin{proof*}{}{}
        Let $R$ be a ring with ideal $I$. 
        
        $\implies$ Suppose $I$ is maximal. Then by the correspondence theorem the only ideals of $R/I$ are $\{0_{R/I}\}$ and $R/I$ corresponding to $I$ and $R$ respectively. Moreover, since $I \neq R$ $R/I \neq \{0\}$. Thus, we have that $R/I$ is simple.
        
        $\impliedby$ Suppose $R/I$ is a simple ring and let $I \subseteq J \subseteq R$ be an ideal containing $I$. Then by the correspondence theorem we have that $I/I \subseteq J/I \subseteq R/I$ is an ideal of $R/I$. However, as $R/I$ is simple $J/I = I/I$ or $J/I = R/I$. By the bijectivity of the correspondence, $J = I$ or $J = R$. Hence, $I$ is a maximal ideal in $R$ as claimed.
    \end{proof*}
\end{prop}

\begin{cor}{}{}
    If $R$ is commutative, then an ideal $I \subseteq R$ is maximal if and only if $R/I$ is a field.
\end{cor}

\begin{eg}{}{}
    \leavevmode
    \begin{enumerate}
        \item If $F$ is a field, then $\{0\}$ is the only maximal ideal of $F$.
        \item If $p$ is a prime number, then $p\Z \subseteq \Z$ is a maximal ideal. (indeed $\Z/p\Z$ is a field)
        \begin{enumerate}
            \item[$\drsh$] Actually, for $n \geq 0$, $n\Z \subseteq \Z$ is a maximal ideal if and only if $n$ is prime
        \end{enumerate}
        \item In $F[x]$ for $F$ a field, the ideal $(x-a)$ is maximal for all $a \in F$, because $F[x]/(x-a) \cong F$.
    \end{enumerate}
\end{eg}

\begin{defn}{}{}
    Let $R$ be a commutative ring. Then an ideal $I$ of $R$ is \Emph{prime} if \begin{enumerate}
        \item $I \subsetneq R$
        \item For all $r_1,r_2 \in R$, if $r_1r_2 \in I$ then either $r_1 \in I$ or $r_2 \in I$.
    \end{enumerate}
\end{defn}

\begin{eg}{}{}
    \begin{enumerate}
        \item A commutative ring $R$ is an integral domain if and only if $\{0\}$ is a prime ideal.
        \begin{enumerate}
            \item[$\drsh$] Indeed, $R$ is a commutative integral domain $\iff$ $R \neq \{0\}$ and whenever $ab = 0$, either $a = 0$ or $b = 0$ $\iff$ $R \neq \{0\}$ and whenever $ab \in \{0\}$, $a \in \{0\}$ or $b \in \{0\}$ $\iff$ $\{0\}$ is a prime ideal of $R$. 
        \end{enumerate}
        \item $p\Z \subset \Z$ is a prime ideal for $p$ a prime number. Indeed, $p\Z \neq \Z$ and $p \;\vert\;ab$ implies $p \;\vert \;a$ or $p\;\vert\;b$.
    \end{enumerate}
\end{eg}

\begin{prop}{}{}
    Let $R$ be a commutative ring with ideal $I \subseteq R$. Then $I$ is prime if and only if $R/I$ is an integral domain.
    \begin{proof*}{}{}
        Indeed, $I$ is prime $\iff$ $I \subsetneq R$ and $ab \in I$ implies $a \in I$ or $b \in I$ $\iff$ $R/I \neq \{0\}$ and $ab+I = I$ implies $a + I = I$ or $b + I = I$ $\iff$ $R/I$ is an integral domain.
    \end{proof*}
\end{prop}

\begin{cor}{}{}
    Every maximal ideal of a commutative ring $R$ is a prime ideal.
\end{cor}

\begin{eg}{}{}
    \leavevmode
    \begin{enumerate}
        \item $(x) \subset \Z[x]$ is a prime ideal. Indeed, $\Z[x]/(x) \cong \Z$ is an integral domain (not a field, so $(x)$ is not maximal)
        \item $(p) \subset \Z[x]$ for a prime number $p$ is a prime ideal which is not maximal. Consider \begin{equation}
            \map{\Z[x] \xrightarrow{f} {\Z/p\Z[x]}}{\sum_ia_ix^i\mapsto \sum_i[a_i]_px^i}
        \end{equation}
        Then $f$ is a surjective ring homomorphism and $\ker(f) = (p)$. So, by the first isomorphism theorem $f$ induces a ring isomorphism \begin{equation}
            \overline{f}: \Z[x]/(p) \xrightarrow{\sim} \Z/p\Z[x]
        \end{equation}
        Since $\Z/p\Z[x]$ is an integral domain $(p)$ is a prime ideal, but $\Z/p\Z[x]$ is not a field, so $(p)$ is not maximal.
    \end{enumerate}
\end{eg}


\section{Ideal Arithmetic and the Chinese Remainder Theorem}

\begin{defn}{}{}
    Let $R$ be a ring , and $I,J$ ideals of $R$. We define their sum as \begin{equation}
        I + J := \{i+j:i\in I, j \in J\}
    \end{equation}
    Then $I + J$ is an ideal of $R$. Next, define their product \begin{equation}
        IJ := \langle ij: i \in I, j \in J\rangle = \left\{\sum\limits_{k=1}^ni_kj_k: n \geq 1, i_k \in I, j_k \in J, \forall1 \leq k \leq n\right\}
    \end{equation}
    Then $IJ$ is an ideal of $R$ as well. Recall $I \cap J$ is also an ideal of $R$
\end{defn}

\begin{eg}{}{}
    For $R = \Z$ and $m,n > 0$, we have \begin{align}
        m\Z + n\Z &= \gcd(m,n) \\
        m\Z\cdot n\Z &= mn\Z \\
        m\Z \cap n\Z &= \text{lcm}(m,n)\Z
    \end{align}
    So, if $\gcd(m,n) = 1$, then $$m\Z\cdot n\Z = m\Z \cap n\Z$$
    because $mn = \text{lcm}(m,n)\gcd(m,n)$.
\end{eg}

\begin{rmk}{}{}
    Let $I,J, K$ be ideals of a ring $R$. Then \begin{enumerate}
        \item $(I+J)K = IK + JK$
        \item $K(I+J) = KI + KJ$
        \item $IR = I = RI$
    \end{enumerate}
    \begin{proof*}{}{}
        Let $I,J,K$ be ideals of a ring $R$
        
        \textbf{1.} Let $(i+j)k \in (I+J)K$. Then by distributivity $(i+j)k = ik+jk \in IK+JK$. Similarly, for all $ik+jk \in IK+JK$, $ik+jk = (i+j)k \in (I+J)K$. Thus, we have that $(I+J)K = IK+JK$.
        
        \textbf{2.} This statement is identical to 1., replacing right distributivity with left distributivity.
        
        \textbf{3.} Note that for all $i \in I$ and all $r \in R$, $ir,ri \in I$ since $I$ is an ideal, and $i = 1\cdot i \in RI$, $i = i\cdot 1 \in IR$ using the fact that $R$ is unital. Hence, $RI = I  = IR$, completing the proof.
    \end{proof*}
    \label{idealProps}
\end{rmk}

\begin{namthm*}{Chinese Remainder Theorem (CRT)}{}
    Let $R$ be a ring, and $I,J$ ideals of $R$. Assume that $I + J = R$ ($I$ and $J$ are said to be \Emph{relatively prime}). Then the ring homomorphism \begin{equation}
        \map{R\xrightarrow{\alpha} R/I \times R/J}{r \mapsto (r+I, r+J)}
    \end{equation}
    is surjective of kernel $I \cap J$. Consequently, by the first isomorphism theorem for rings we have an isomorphism \begin{equation}
        \overline{\alpha}: R/I\cap J \xrightarrow{\sim} R/I \times R/J
    \end{equation}
    Moreover, if $R$ is commutative, then $I+J = R$ implies that $I \cap J = IJ$.
\end{namthm*}
\begin{proof*}{}{}
    Suppose $R$ is a ring with relatively prime ideals $I, J$. Define a map $$\map{R\xrightarrow{\alpha} R/I \times R/J}{r \mapsto (r+I, r+J)}$$
    Let $r,r' \in R$. Then, observe that \begin{align*}
        \alpha(r+r') &= (r+r' + I, r+r' + J) & \alpha(rr') &= (rr' + I, rr' + J) \\
        &= (r+I+r'+I,r+J + r'+J) & &= ((r+I)(r'+I),(r+J)(r'+J)) \\
        &= (r+I,r+J)+(r'+I,r'+J) & &= (r+I,r+J)(r'+I,r'+J) \\
        &= \alpha(r) + \alpha(r') & &= \alpha(r)\alpha(r')
    \end{align*}
    and $$\alpha(1_R) = (1_R + I, 1_R+J) = (1_{R/I},1_{R/J})$$
    so $\alpha$ is a ring homomorphism. To show $\alpha$ is surjective, let $a,b \in R$. We want to find $r \in R$, $i \in I$, and $j \in J$ such that $r+i = a$ and $r+j = b$. But, we then have that $r = a-i = b-j$, so $a-b = i-j$. Note that $I+J = R$ since they are relatively prime, so there exist $i' \in I$ and $j' \in J$ such that $a-b = i' + j'$. Set $i = i'$ and $j = -j'$. Then, observe that $$\alpha(r) = (r+I,r+J) = (r+i + I, r+j + J) = (a+I, b+J)$$
    Therefore $\alpha$ is a surjective ring homomorphism. Observe that $I\cap J \subseteq \ker(\alpha)$ since for all $k \in I \cap J$, $\alpha(k) = (k+I,k+J) = (I,J)$. Then, let $t \in \ker(\alpha)$. Observe that then $(I,J) = \alpha(t) = (t+I,t+J)$, so by definition $t \in I$ and $t \in J$. Thus, $t \in I \cap J$, so $\ker(\alpha) \subseteq I \cap J$. Consequently $\ker(\alpha) = I \cap J$. By the first isomorphism theorem for rings we have our desired result. 
    
    
    Now, suppose $R$ is commutative. Then observe that \begin{align*}
        I \cap J &= (I\cap J)R \\
        &= (I\cap J)(I+J) \\
        &= (I\cap J)I + (I\cap J)J \\
        &\subseteq JI + IJ \\
        &= IJ + IJ \\
        &= IJ
    \end{align*}
    Moreover, $IJ \subseteq I \cap J$ since $IJ$ is generated by $ij$ for $i \in I$ and $j \in J$. However, since $I$ and $J$ are ideals $ij \in I$ and $ij \in J$, so in particular $ij \in I \cap J$. Thus, we conclude that $IJ = I\cap J$.
\end{proof*}

\begin{lem}{}{}
    If $I_1, I_2,..., I_n$ are pairwise relatively prime ideals of $R$, with $n \geq 2$, then for all $i \in \{1,2,...,n\}$ $\bigcap_{j\neq i} I_j$ and $I_i$ are relatively prime. That is, $\bigcap_{j\neq i} I_j + I_i = R$.
    \begin{proof*}{}{}
        Let $I_1, I_2,..., I_n$ be as in the statement, for $n \geq 2$. If $n = 2$ we are done by assumption. Then, if $n = 3$ there exist $i_1 \in I_{j_1}$, $i_2 \in I_{j_2}$, $i_3,i_3' \in I_i$ such that $i_1+i_3 = 1$, $i_2+i_3' = 1$, where $i \in \{1,2,3\}$. Then $1 = (i_1+i_3)(i_2+i_3') = i_1i_2 + i_3i_2 + i_1i_3' + i_3i_3'$, where $i_1i_2 \in I_{j_1} \cap I_{j_2}$, and $i_3i_2,i_1i_3',i_3i_3' \in I_i$ since they are ideals. Thus, $1 \in I_{j_1}\cap I_{j_2} + I$, so $I_{j_1} \cap I_{j_2} + I_i = R$, since it is an ideal. Hence, the base cases hold. Now, suppose there exists $k \geq 3$ such that if $n = k$, $I_{j_1}\cap I_{j_2} \cap ... \cap I_{j_{k-1}} + I_i = R$ for $i \in \{1,2,...,k\}$, and $\{j_1,j_2,...,j_{k-1}\} = \{1,2,...,k\}\backslash\{i\}$. Then, choose $i \in \{1,2,...,k+1\}$ and $\{j_1,j_2,...,j_{k}\} = \{1,2,...,k+1\}\backslash\{i\}$. Let $I = I_{j_1}\cap I_{j_2} \cap ... \cap I_{j_{k-1}}$. Then we have by the induction hypothesis and assumption that $I + I_i = R$, $I + I_{j_k} = R$ and $I_i + I_{j_k} = R$. Then, by our argument in the base case for $n = 3$ we have that $I \cap I_{j_k} + I_i = R$. In particular, $I_{j_1}\cap I_{j_2} \cap ... \cap I_{j_{k-1}} \cap I_{j_k} + I_i = R$, as desired. Therefore, by mathematical induction we conclude that for all $n \geq 2$ and all $i \in \{1,2,...,n\}$, $I_{j_1}\cap I_{j_2} \cap ... \cap I_{j_{n-1}} + I_i = R$, completing the proof.
    \end{proof*}
    \label{genCRTLem}
\end{lem}


\begin{cor}{}{}
    Let $R$ be a ring with ideals $I_1,I_2,...,I_n$ of $R$, for $n \geq 1$. Suppose that $I_i + I_j = R$ for all $i \neq j$. Then \begin{equation}
        \map{R\xrightarrow{\alpha} R/I_1 \times R/I_2 \times ... \times R/I_n = \bigotimes\limits_{i=1}^nR/I_i}{r \mapsto (r+I_1, r+I_2,...,r+I_n)}
    \end{equation}
    is a surjective ring homomorphism of kernel $\bigcap\limits_{i=1}^n I_i =  I_1\cap I_2 \cap ... \cap I_n$. Thus, we have an isomorphism \begin{equation}
        \overline{\alpha}: R/\bigcap\limits_{i=1}^n I_i \xrightarrow{\sim} \bigotimes\limits_{i=1}^nR/I_i
    \end{equation}
    If $R$ is commutative, we have $\bigcap\limits_{i=1}^n I_i = \prod\limits_{i=1}^n I_i$.
\end{cor}
\begin{proof*}{}{}
    Let $R$ be a ring with pairwise relatively prime ideals $I_1,I_2,...,I_n$ of $R$, for $n \geq 1$. Define a map $$\map{R\xrightarrow{\alpha} R/I_1 \times R/I_2 \times ... \times R/I_n = \bigotimes\limits_{i=1}^nR/I_i}{r \mapsto (r+I_1, r+I_2,...,r+I_n)}$$
    From the proof of the Chinese Remainder Theorem $\alpha$ is a ring homomorphism. Moreover, $\ker(\alpha) = \bigcap\limits_{i=1}^nI_i$. To show surjectivity let $a_1,a_2,...,a_n \in R$. Then, for each $i \in \{1,2,...,n\}$ choose $c_i \in I_{i_1}\cap ... \cap I_{i_{n-1}}$ and $i' \in I_i$ such that $c_i + i' = 1_R$. This choice is possible by the result of Lemma \ref{genCRTLem}. We then define $a = a_1c_1 + ... + a_nc_n$. Note that for each $i \in \{1,2,..,n\}$, $c_i \equiv 0 \mod I_j$ if $j \neq i$, and $c_i \equiv 1 \mod I_i$ since $c_i + i' = 1_R$. It follows that $a \equiv 0_R + ... + a_i + ... + 0_R \equiv a_i \mod I_i$. Thus, we have that $$\alpha(a) = (a+I_1,a+I_2,...,a+I_n) = (a_1+I_1,a_2+I_2,...,a_n+I_n)$$ 
    Therefore, $\alpha$ is a surjective ring homomorphism with kernel $\ker(\alpha) = \bigcap\limits_{i=1}^nI_i$, so by the first isomorphism theorem for rings $$\overline{\alpha}: R/\bigcap\limits_{i=1}^n I_i \xrightarrow{\sim} \bigotimes\limits_{i=1}^nR/I_i$$
    Then, by the existence of this isomorphism $b \cong a_i \mod I_i$ for each $i \in \{1,2,...,n\}$ if and only if $b \cong a \mod \bigcap\limits_{i=1}^n I_i$. 
    
    
    Finally, suppose $R$ is a commutative ring. First, for each basic element of the form $i_1i_2...i_n \in \prod\limits_{i=1}^n I_i$ we have $i_1i_2...i_n \in I_j$ for each $j \in \{1,2,...,n\}$ since they are ideals. Hence, $\prod\limits_{i=1}^n I_i \subseteq \bigcap\limits_{i=1}^n I_i$. Next, write $I_1\cap I_2 \cap ... \cap I_{n-1} = I$. Then by our Lemma \ref{genCRTLem} $I + I_n = R$. It follows that \begin{align*}
        I \cap I_n &= (I \cap I_n)R \tag{Lemma \ref{idealProps}}\\
        &= (I\cap I_n)(I+I_n) \\
        &= (I\cap I_n)I + (I\cap I_n)I_n \tag{Lemma \ref{idealProps}} \\
        &\subseteq I_nI + II_n  \\
        &= II_n + II_n \tag{Commutativity of $R$} \\
        &= II_n \tag{since $0_R \in I_j, \forall j$} 
    \end{align*}
    Thus, we have that $\bigcap\limits_{i=1}^n I_i = I \cap I_n \subseteq II_n = \prod\limits_{i=1}^n I_i$. Therefore, we conclude that if $R$ is commutative, $\prod\limits_{i=1}^n I_i = \bigcap\limits_{i=1}^n I_i$.
\end{proof*}

\begin{rmk}{Solving Congruences}{}
    Suppose we have a system of congruences \begin{equation}
        \left\{\begin{array}{l}
            x \equiv a_1 \mod m_1 \\
            x \equiv a_2 \mod m_2 \\
            \vdots \\
            x \equiv a_n \mod m_n 
            \end{array}\right.
    \end{equation}
    we can find the $c_i$ given in the above proof as follows. Write $M = m_1...m_n$ and \\$M_i = m_1...m_{i-1}m_{i+1}...m_n$. Then, note that $M_i = m_1...m_{i-1}m_{i+1}...m_n\;\vert\;c_i$ by definition of $c_i$, and that since $c_i \equiv 1 \mod m_i$, there exists $b_i \in \Z$ such that $c_i + b_im_i = 1$. Then, write $c_i = y_iM_i$, where we can solve for $y_i$ using the extended Euclidean Algorithm on $(M_i, m_i)$, or noting the inverse of $M_i$ modulo $m_i$, as they are relatively prime. 
\end{rmk}


\section{Adjunctions}

\begin{defn}{Ring Relations}{}
        Creating Relations in a ring $R$: Suppose we have an element $a \in R$. If we want a ring $\overline{R}$ which is an image of $R$, where $\overline{a} = 0$, then the largest such quotient is $\overline{R} = R/(a)$. If we want a ring where we have a number of relations $a_1=a_2=...=a_n=0$, we can take $(R/(a_1)/(a_2)/.../(a_n))=\overline{R}=R/(a_1,a_2,...,a_n)$. This is valid because the ideal $(a_1,...,a_n)$ contains $(a_i)$ for all $i$, and then this is successive applications of the Isomorphism Theorem.
\end{defn}

\begin{rmk}{}{}
        If $R$ is a ring and $a \in R$, if $a$ is a unit then $R/(a) = \{0\}$ since $(a) = R$. I.e$\rangle$ modding out by a unit mods out all elements of the ring.
\end{rmk}


\section{Isomorphism Theorems and Correspondence}

\begin{namthm*}{First Isomorphism Theorem (for rings)}{}
    Let $f:R\rightarrow R'$ be a ring homomorphism. Then by the First Isomorphism Theorem for groups there exists a unique group isomorphism for $(R,+)$ such that \begin{equation}
        \map{R/\ker(f) \xrightarrow{\overline{f}}f(R)}{r+\ker(f) \mapsto f(r)}
    \end{equation}
    and this is also a ring isomorphism. This theorem can be stated succinctly by the following diagram:
    \begin{center}
            \begin{tikzpicture}[baseline = (a).base]
            \node[scale = 1] (a) at (0,0){
                \begin{tikzcd}
                    R \ar[d, twoheadrightarrow, "\pi", swap] \ar[r, "\forall f"] & \forall R' \\
                    R/\ker(f) \ar[ur, dashed, "\exists!\overline{f}", swap] &
                \end{tikzcd}
            };
            \end{tikzpicture}
        \end{center}
\end{namthm*}
\begin{proof*}{}{}
    (Left to the reader)
\end{proof*}


\begin{namthm*}{Third Isomorphism Theorem}{}
    Let $R$ be a ring and suppose $A \subseteq B \subseteq R$ are ideals of $R$. Then $B/A$ is an ideal of $R/A$ and \begin{equation}
        (R/A)/(B/A) \cong R/B
    \end{equation}
    \begin{center}
            \begin{tikzpicture}[baseline = (a).base]
            \node[scale = 1] (a) at (0,0){
                \begin{tikzcd}
                    R \ar[r, twoheadrightarrow, "\pi_B"] \ar[d, twoheadrightarrow, "\pi_A", swap] & R/B \\
                    R/A \ar[ur, dashed, "\exists!\overline{\pi_B}"] \ar[r, twoheadrightarrow, "\pi_{B/A}", swap] & (R/A)/(B/A) \ar[u, dashed, "\exists!\overline{\overline{\pi_B}}", swap]
                \end{tikzcd}
            };
            \end{tikzpicture}
        \end{center}
\end{namthm*}
\begin{proof*}{}{}
    Suppose $R$ is a ring with ideals $A \subseteq B \subseteq R$. By the Correspondence Theorem we know that $B/A$ is an ideal of $R/A$. Then, define a map $$\map{\phi: R/A\rightarrow R/B}{a+A \mapsto a+ B}$$
    First, suppose that $a+A = b+A$. Then we have that $a-b \in A \subseteq B$, so $a - b \in B$. Hence, $a + B = b + B$ by definition of coset equality for quotient groups, so $\phi$ is well defined. Moreover, observe that for all $a+A,c+A \in R/A$, we have $$\phi(a+A+c+A) = \phi(a+c+A) = a+c+B = a+B+c+B = \phi(a+A)+\phi(c+A)$$
    $$\phi((a+A)(c+A)) = \phi(ac+A) = ac+B = (a+B)(c+B) = \phi(a+A)\phi(c+A)$$
    and $$\phi(1_{R/A})=\phi(1_R+A) = 1_R + B = 1_{R/B}$$
    so $\phi$ is a ring homomorphism. Moreover, by construction we have that $\phi$ is surjective. Now, note that $B/A \subseteq \ker(\phi)$. Then, let $k+A \in \ker(\phi)$, so $\phi(k+A) = k+B = B$. Hence, $k \in B$ so $k+A \in B/A$. Therefore, we conclude that $\ker(\phi) = B/A$, so by the First Isomorphism Theorem for rings, \begin{equation}
        (R/A)/(B/A) \cong R/B
    \end{equation}
\end{proof*}



\begin{namthm*}{Correspondence Theorem (for rings)}{}
    Let $\phi:R \rightarrow R'$ be a surjective ring homomorphism of kernel $K \subseteq R$. Then, the correspondence between subgroups of $(R',+')$ and subgroups of $(R,+)$ containing $K$ induces a bijection between ideals of $R'$ and ideals of $R$ containing $K$: 
    \begin{equation}
        \begin{array}{rcl}
            \left\{I:I\subseteq R'\;\text{an ideal}\right\} &\leftrightarrow& \left\{J:K \subseteq J \subseteq R\;\text{an ideal}\right\} \\
            I &\mapsto& \phi^{-1}(I) \\
            \phi(J) &\mapsfrom& J
        \end{array}
    \end{equation}
    are inverse bijections.
\end{namthm*}
\begin{proof*}{}{}
    (Left to the reader)
\end{proof*}

\begin{rmk}{Warning about image of ideals}{}
        If $f:R\rightarrow R'$ is a ring homomorphism that is not surjective, and $J \subset R$ is an ideal, then $f(J)$ is not necessarily an ideal as well. For example, $i:\Z\hookrightarrow \Q$ is a ring homomorphism and $\Z$ is an ideal in $\Z$, but $i(\Z)$ is not an ideal in $\Q$ since $\Q$ is a field with only trivial ideals.
\end{rmk}


%%%%%%%%%%%%%%%%%%%%%% - P2.Chapter 3
\chapter{\textsection Adjunction of Elements}




%%%%%%%%%%%%%%%%%%%%%% - P2.Chapter 4
\chapter{\textsection Unique Factorization Domains}

\section{Basic Definitions and Examples: UFDs}


\section{Unique Factorization in F[x]}

\begin{namthm*}{Unique Factorization Theorem (for F[x])}{}
    Take $F$ a field and $f \in F[x]$ of degree greater than or equal to $1$. Then \begin{enumerate}
        \item $f = aP_1P_2...P_m$ where $a \in F$ is the leading coefficient of $f$ and $P_i$ is monic irreducible in $F[x]$ for all $i$.
        \item The factorization in 1. is unique up to the order of the factors.
    \end{enumerate}
\end{namthm*}
\begin{proof*}{First Attempt}
    Let $F$ be a field and let $f \in F[x]$ of degree $n \geq 1$. It is sufficient to consider $f$ monic since we can replace $f$ with $a^{-1}f$, where $a$ is the leading coefficient of $f$. Then, we proceed by induction on the degree of $f$. If $\deg(f) = 1$ then $f$ is already a monic irreducible polynomial in $F[x]$, so we're done. Inductively suppose there exists $k \geq 1$ such that for all $j \leq k$, if $\deg(f) = j$ then $f = q_1q_1...q_m$ for monic irreducible polynomials $q_i \in F[x]$. Then, suppose $\deg(f) = k+1$. If $f$ is irreducible then we're done. On the other hand, if $f$ is not irreducible there exist $g,h \in F[x]$ such that $f = gh$ and $\deg(g),\deg(h) \geq 1$. Consequently, $\deg(f) = \deg(g)+\deg(h) > \deg(g),\deg(h)$, so $\deg(g),\deg(h) \leq k$. Thus, by the induction hypothesis $g = g_1g_2...g_m$ and $h = h_1h_2...h_l$ for monic irreducible polynomials $g_i,h_j \in F[x]$, $1\leq i \leq m$, $1 \leq j \leq l$. Hence, $f = g_1g_2...g_mh_1h_2...h_l$ is the product of monic irreducible polynomials in $F[x]$ as desired. Therefore, by mathematical induction we have that for all $f \in F[x]$, $\deg(f) \geq 1$, $f$ can be factored as the product of monic irreducible polynomials and its leading coefficient. Now, suppose the factorization is not necessarily unique. Let $f$ be a polynomial of lowest degree with two such factorizations $f = aP_1P_2...P_m$ and $f = bQ_1Q_2...Q_n$. Since the $P_i$ and $Q_j$ are monic we must have that $a = b \neq 0$, so multiplying by $a^{-1}$ on both sides we obtain $P_1P_2...P_m = Q_1Q_2...Q_n$. Note that $(P_i)$ is a prime ideal for each $P_i$, so in particular each $P_i$ is a prime element. Hence, $P_1$ divides $Q_1Q_2...Q_n$ which implies that $P_1$ divides $Q_j$ for some $1 \leq j \leq n$. Reorder the $Q_i$ if need be so that $P_1$ divides $Q_1$. Then, there exists $f \in F[x]$ such that $P_1f = Q_1$. But, $Q_1$ is irreducible so as $\deg(P_1) \geq 1$, we must have that $f \in F[x]^{\times}$, so $f = c \in F$ for some $c$. But, $P_1$ and $Q_1$ are monic, so $c = 1$. Thus $P_1 = Q_1$. Since $F[x]$ is an integral domain we can cancel elements, so $P_2...P_m = Q_2...Q_n$. But, this is a polynomial of strictly lower degree than $f$ with a distinct factorization, contradicting the minimality of $f$. Therefore, the factorization of $f$ must be unique up to reordering.
\end{proof*}


\begin{eg}{}{}
        Consider $f = 5(x^3-1) \in F[x]$ with $5 \neq 0 \in F$. We always have $f = 5(x-1)(x^2+x+1)$ (for any field) \begin{enumerate}
                \item If $F = \R$, this is the factorization from the UFT because $x^2+x+1$ is monic irreducible in $\R[x]$, as it has no roots over $\R$.
                \item if $F = \Z/2\Z$, $f = 5(x-1)(x^2+x+1) = (x+1)(x^2+x+1)$ is the UFT factorization as $x^2+x+1$ has no roots over $\Z/2\Z$
                \item If $F = \Z/3\Z$, $f = 5(x-1)(x^2+x+1) = 2(x-1)(x-1)^2 = 2(x+2)^3$ is the UFT factorization
                \item If $F = \C$ then we have $f = 5(x-1)(x^2+x+1) = 5(x-1)(x-u)(x-\overline{u})$ for $u = \frac{-1+\sqrt{3}i}{2}$.
        \end{enumerate}
\end{eg}



%%%%%%%%%%%%%%%%%%%%%% - P2.Chapter 5
\chapter{\textsection Principal Ideal Domains}

\section{Basic Definitions and Examples: PIDs}

\begin{defn}{}{}
    An integral domain of which every ideal is principal is called a \Emph{principal ideal domain} (\Emph{PID}).
\end{defn}

\begin{rmk}{}{}
    $\Z$ is a prototypical example of a PID. Additionally, for a field $\F$, $\F[X]$ is also a PID.
\end{rmk}

\begin{thm}{}{}
    Let $\F$ be a field. Then $\F[X]$ is a principal ideal domain. Moreover, every non-zero ideal $I$ of $\F[X]$ is generated by the monic polynomial of lowest degree in $I$.
\end{thm}
\begin{proof*}{}{}
    Note that since $\F$ is an integral domain so is $\F[X]$. Consider an ideal $I \subseteq \F[X]$. If $I = \{0\} = (0)$ we are done, so assume $I \neq \{0\}$ and that $0 \neq g \in I$ is of minimal degree. We can assume that $g$ is monic; if $a$ is the leading coefficient of $g$, then $a^{-1} \in \F$ so $a^{-1}g$ is monic, and we replace $g$ with $a^{-1}g \in I$ since $I$ is an ideal. We claim that $I = (g)$. Since $g \in I$ and $I$ is an ideal we have $g\F[X] = (g) \subseteq I$. Let $P \in I$. By the division algorithm there exist unique $q,r \in \F[X]$ such that $P = qg + r$ where $r = 0$ or $\deg(r) < \deg(g)$ since the leading coefficient of $g$ is a unit. Then $r = P - qg \in I$ since $qg \in I$ as it is an ideal. Then $r = 0$, since otherwise $\deg(r) < \deg(g)$, contradicting the minimality of $g$'s degree. Thus, $P = qg \in (g)$, so $I \subseteq (g)$. Therefore $I = (g)$ and $\F[X]$ is a PID as claimed.
\end{proof*}



%%%%%%%%%%%%%%%%%%%%%% - P2.Chapter 6
\chapter{\textsection Euclidean Domains}

\section{Basic Definitions and Examples: Euclidean Domains}

\begin{defn}{Norm}{}
    For an integral domain $R$, any function $N:R\rightarrow \Z^+\cup\{0\}$ with $N(0_R) = 0$ is called a \Emph{norm} on the integral domain $R$. If $N(a) > 0$ for $a \neq 0$ define $N$ to be a \Emph{positive norm}.
\end{defn}

\begin{defn}{Euclidean Domain}{}
    The integral domain $R$ is said to be a \Emph{Euclidean Domain} (or posses a \Emph{Division Algorithm}) if there is a norm $N$ on $R$ such that for any two elements $a,b \in R$, with $b \neq 0_R$, there exist elements $q,r \in R$ with $$a= qb+r$$ where $r = 0_R$ or $N(r) < N(b)$. The element $q$ is called the \Emph{quotient} and the element $r$ the \Emph{remainder} of the division.
\end{defn}

\begin{eg}{}{}
    \leavevmode
    \begin{enumerate}
        \item Fields are trivial examples of Euclidean Domains where any norm will satisfy the defining condition. (e.g., $N(a) = 0,\forall a$) This is because for all $a,b$, $b\neq 0$, we have $a = qb+0$, where $q = ab^{-1}$.
        \item The integers are a Euclidean Domain with norm $N(a) = |a|$, the usual absolute value.
        \item If $F$ is a field, the polynomial ring $F[x]$ is a Euclidean Domain with norm $N(p(x)) = \deg(p(x))$
        \item The Gaussian integers, $\Z[i]$, is a Euclidean domain with norm $N(a+bi) = a^2+b^2$
    \end{enumerate}
\end{eg}


\begin{prop}{}{}
    Every ideal in a Euclidean Domain is principal. In particular, if $I$ is any nonzero ideal in the Euclidean Domain $R$, then $I = (d)$ for $d$ any nonzero element of $I$ of minimal norm.
\end{prop}
\begin{proof*}{}{}
    If $I$ is the zero ideal there is nothing to prove. Otherwise, let $d$ be a nonzero element of $I$ of minimum norm. Such a $d$ exists since the set $\{N(a):a \in I\}$ is a nonempty subset of $\Z$, which is bounded below, and hence has a minimum element by the well-ordering of $\Z$. Clearly $(d) \subseteq I$ since $d \in I$. To show the reverse inclusion let $a$ be any element of $I$, and use the division algorithm to write $a = qd + r$ for $q,r \in R$, with $N(r) < N(d)$. Then, since $I$ is an ideal $-qd \in I$, so $r = a-qd \in I$. Thus, as $d$ is an element of minimal norm in $I$, so we must have that $r = 0$. Thus $a = qd \in (d)$, showing $I = (d)$.
\end{proof*}

\begin{rmk}{}{}
    This proposition can be used to prove that some integral domains $R$ are not Euclidean Domains if they have non-principal ideals.
\end{rmk}

\begin{defn}{}{}
    Let $R$ be a commutative ring and let $a,b \in R$ with $b \neq 0$. \begin{enumerate}
        \item $a$ is said to be a \Emph{multiple} of $b$ if there exists an element $x \in R$ with $a = bx$. In this case $b$ is said to \Emph{divide} $a$ or be a \Emph{divisor} of $a$, written $b\;\vert\;a$
        \item A \Emph{greatest common divisor} of $a$ and $b$ is a nonzero element $d \in R$ such that \begin{enumerate}
                \item $d\;\vert\;a$ and $d\;\vert\;b$, and 
                \item if $d'\;\vert\;a$ and $d'\;\vert\;b$ then $d'\;\vert\;d$.
        \end{enumerate}
    \end{enumerate}
    A greatest common divisor of $a$ and $b$ is denoted $\gcd(a,b)$.
\end{defn}


\begin{rmk}{}{}
    Translating into the language of ideals, if $I = (a,b) \subseteq R$, then $d$ is the greatest common divisor of $a$ and $b$ if \begin{enumerate}
        \item $I \subseteq (d)$, and
        \item if $I \subseteq (d')$, then $(d) \subseteq (d')$
    \end{enumerate}
    Hence, a greatest common divisor for $a$ and $b$ (if one exists) is a generator for the unique smallest principal ideal containing $a$ and $b$.
\end{rmk}


\begin{prop}{}{}
    If $a,b$ are nonzero elements in the commutative ring $R$ such that the ideal generated by $a$ and $b$ is a principal ideal $(d)$, then $d$ is the greatest common divisor of $a$ and $b$.
\end{prop}

\begin{defn}{}{}
    An integral domain in which every ideal $(a,b)$ generated by two elements is principal is called a \Emph{Bezout Domain}. Note that Bezout Domain's can have non-principal ideals.
\end{defn}

\begin{prop}{}{}
    Let $R$ be an integral domain. If two elements $d$ and $d'$ of $R$ generate the same principal ideal, i.e. $(d) = (d')$, then there exists $u \in R^{\times}$ such tha $d = ud'$. $d$ and $d'$ are called \Emph{associates} in this case.
\end{prop}

\begin{thm}{}{}
    Let $R$ be a Euclidean Domain and let $a,b \in R$ be nonzero elements of $R$. Let $d = r_n$ be the last nonzero remainder in the Euclidean Algorithm for $a$ and $b$. Then \begin{enumerate}
        \item $d = \gcd(a,b)$, and 
        \item $(d) = (a,b)$, so there exist $x,y \in R$ such that \begin{equation}
                d = ax + by
        \end{equation}
    \end{enumerate}
\end{thm}
\begin{proof*}{}{}
    (Left to the reader)
\end{proof*}

\begin{defn}{}{}
    Let $\widetilde{R} := R^{\times}\cup \{0\}$. An element $u \in R - \widetilde{R}$ is called a \Emph{universal side divisor} if for every $x \in R$ there is some $z \in \widetilde{R}$ such that $u$ divides $x-z \in R$.
\end{defn}


\begin{prop}{}{}
    Let $R$ be an integral domain that is not a field. If $R$ is a Euclidean Domain then there are universal side divisors in $R$.
\end{prop}
\begin{proof*}{}{}
    (Left to the reader)
\end{proof*}



%%%%%%%%%%%%%%%%%%%%%% - P2.Chapter 7
\chapter{\textsection Polynomial Rings}

\section{Basic Definitions and Examples: Polynomial Rings}

\begin{defn}{}{}
    Let $R$ be a ring and let $x$ be a formal symbol (not related to $R$). We want to define a ring $R[x]$ such that \begin{enumerate}
        \item $x \in Z(R[x])$
        \item $R \subseteq R[x]$ is a subring
        \item $R[x]$ is generated by $\{x\}\cup R$
    \end{enumerate}
    Then, for $P \in R[x]$ we have that $P = \sum\limits_{j=0}^nb_jx^j$ where $b_j \in R$ for all $j$, and $n \in \Z$, with $n \geq 0$ (note we define $x^0 = 1_R \in R$). We also require that \begin{equation}
        \sum\limits_{j=0}^nb_jx^j = \sum\limits_{i=0}^ma_ix^i, \iff b_j = a_j \forall j\geq 0
    \end{equation}
    where if $j > n$, $b_j = 0$, and if $j > m,$ $a_j = 0$. Next, we define addition and multiplication as 
    \begin{enumerate}
        \item[]\underline{\textbf{Addition}}: $$\sum\limits_{j=0}^nb_jx^j + \sum\limits_{i=0}^ma_ix^i := \sum\limits_{i=0}^{\max(m,n)}(a_i+b_i)x^i$$
        \item[] \underline{\textbf{Multiplication}}: $$\left[\sum\limits_{j=0}^nb_jx^j\right] \left[\sum\limits_{i=0}^ma_ix^i\right] := \sum\limits_{i=0}^{m+n}\left(\sum\limits_{k=0}^ia_{k}b_{i-k})\right)x^i$$
    \end{enumerate}
\end{defn}

\begin{rmk}{}{}
    Formally, the polynomial is determined by a sequence of coefficients $a_i$ \begin{equation}
                \mathbf{a} = (a_0,a_1,a_2,...)
        \end{equation}
    where $a_i \in R$ and only a finite number of $a_i$ are not zero. The sequence with $1$ in the ith position and zero everywhere else corresponds to the indeterminate monomial $x^i$, and the monomials form a basis of the space of polynomials.
\end{rmk}

\begin{claim}{}{}
    For all rings $R$, $(R[x], +, \cdot)$ is a ring with \begin{equation}
        0_{R[x]} = 0_R + 0_R\cdot x + ...
    \end{equation}
    and \begin{equation}
        1_{R[x]} = 1_R + 0_R\cdot x + ...
    \end{equation}
    Consequently $R$ can naturally be embedded in $R[x]$ by $\map{R \hookrightarrow R[x]}{r \mapsto r\cdot x^0}$. Also, $x \in Z(R[x])$ and $R[x]$ is generated by $R\cup \{x\}$.
    \begin{proof*}{}{}
        (Left to the reader)
    \end{proof*}
\end{claim}

\begin{prop}{}{}
    There is a unique commutative ring structure on the set of polynomials $R[x]$ having these properties:\begin{enumerate}
        \item Addition of polynomials is done coefficient wise for equal degree monomials (like vector addition)
        \item Multiplication of monomials is given by the rule above
        \item The ring $R$ is a subring of $R[x]$, when the elements of $R$ are identified with the constant polynomials
    \end{enumerate}
\end{prop}

\begin{rmk}{}{}
    \leavevmode
    \begin{enumerate}
        \item $(a+bx+cx^2)(\alpha+\beta x) = a\alpha + (a\beta + b\alpha)x + (b\beta + c\alpha)x^2 + c\beta x^3$
        \item $R = M_2(\R)$, $I_2.x^0 = 1 \in R[x]$, and $P = A_0 + A_1x + A_2x^2 + ... + A_nx^n$, $A_i \in M_2(\R)$.
        \item $R = \Z/2\Z = \F_2$. In $\F_2[x]$, $(x+1)^2 = x^2+2x+1 = x^2+1 \pmod{2}$.
    \end{enumerate}
\end{rmk}

\begin{defn}{Polynomials in Multiple Variables}{}
    Let $x_1,...,x_n$ be variables (indeterminates). A \Emph{monomial} is a formal product of these variables of the form \begin{equation}
            x_1^{i_1}x_2^{i_2}...x_n^{i_n}
    \end{equation}
    where the exponents $i_v$ are nonnegative numbers. The n-tuple $(i_1,...,i_n)$ of exponents determines the monomial. Such an n-tuple is called a \Emph{multi-index}, and vector notation $\mathbf{i} = (i_1,...,i_n)$ for multi-indices is convenient. Using it, we may write the monomial symbolically as \begin{equation}
            x^{\mathbf{i}}=x_1^{i_1}x_2^{i_2}...x_n^{i_n}
    \end{equation}
    The monomial $x^{\mathbf{0}}$ is denoted by $1$.


    A polynomial with coefficients in a ring $R$ is a finite linear combination of monomials with coefficients in $R$. Using the shorthand, any polynomial $f(x) = f(x_1,...,x_n)$ can be written uniquely in the form \begin{equation}
            f(x) = \sum\limits_{i} a_{i}x^{\mathbf{i}}
    \end{equation}
    And only finitely many of the coefficients $a_i \in R$ are different from zero.


    A polynomial which is the product of a nonzero element $r \in R$ with a monomial is also called a monomial \begin{equation}
            m=rx^{\mathbf{i}}
    \end{equation}


    Using multi-index notation, the addition and multiplication for polynomials in multiple variables is analogous to the case for one variable using the formulas defined above, and the above proposition also holds analogously for polynomials in multiple variables. 

    A ring of polynomials in several variables with coefficients in the ring $R$ is denoted by \begin{equation}
         R[x_1,...,x_n]\;or\;R[x],\;x=(x_1,...,x_n)
    \end{equation}
\end{defn}


\begin{rmk}{}{}
    For a general ring $R$, $R[x]$ is not isomorphic to $\mathcal{P}(R,R) \subseteq F(R,R)$ (polynomial functions over $R$).
\end{rmk}

\begin{eg}{}{}
    Consider $R = \F_p = \Z/p\Z$ for $p$ a prime. Consider now a polynomial function $f \in F(\Z/p\Z,\Z/p\Z)$ defined by $$f = \sum\limits_{i=0}^nb_i(\hat{x})^i$$ where $(\hat{x})^i$ is the product in $F(\Z/p\Z, \Z/p\Z)$. This gives us the subring $\mathcal{P}$
    \begin{claim}{}{}
        $\mathcal{P}(\F_p,\F_p) \cancel{\cong} \F_p[x]$
        \begin{proof*}{}{}
            First, observe that $\mathcal{P}(\F_p,\F_p)$ is actually finite because $|F(\F_p,\F_p)| = p^p < +\infty$ and $\mathcal{P}(\F_p,\F_p) \subseteq F(\F_p,\F_p)$. However, $|\F_p[x]|$ is infinite. Indeed, for all $i \neq j\geq 0$, $x^i \neq x^j$, so $\{1,x,x^2,...\} = \{x^n:n\geq 0\} \subseteq \F_p[x]$ is a subset of infinite order. Moreover, note that $(\hat{x})^p - \hat{x} = 0 \in F(\F_p,\F_p)$ by Fermat's theorem because $$((\hat{x})^p - \hat{x})(a) = a^p-a = a-a = 0 \pmod{p}$$
            But, $x^p - x \neq 0 \in \F_p[x]$.
        \end{proof*}
    \end{claim}
    Thus, it is not sufficient to consider polynomial functions.
\end{eg} 

\begin{claim}{}{}
    However, $\R[x] \cong \mathcal{P}(\R,\R)$
\end{claim}
\begin{proof*}{}{}
    Define a function $$\map{\Phi:\R[x] \rightarrow \mathcal{P}(\R,\R)}{p\mapsto \left(\map{\ev_p:\R\rightarrow \R}{r \mapsto \ev_r(p)}\right)}$$
    First, note that for all $p = \sum_ia_ix^i \in \R[x]$ and all $r \in \R$, $\ev_r(p) = \sum_ia_ir^i$, so we have that $\ev_p = \sum_ia_i(\hat{x})^i \in \mathcal{P}(\R,\R)$, so the function is well-defined. Recall that $\ev_r(p)$ is a ring homomorphism for all $r \in \R$. Now, let $p = \sum_ia_ix^i, q = \sum_ib_ix^i \in \R[x]$. Then, observe that \begin{align*}
        \Phi(p+q)(r) &= \ev_{p+q}(r)  & \Phi(p\cdot q)(r) &= \ev_{p\cdot q}(r)\\
        &= \ev_r(p+q) & &= \ev_r(p\cdot q) \\
        &= \ev_r(p) + \ev_r(q) & &= \ev_r(p) \cdot\ev_r(q) \\
        &= \ev_p(r) + \ev_q(r) &  &= \ev_p(r) \cdot \ev_q(r) \\
        &= \Phi(p)(r) + \Phi(q)(r) & &= \Phi(p)(r) \cdot \Phi(q)(r)\\
        &= (\Phi(p) + \Phi(q))(r) & &= (\Phi(p) \cdot \Phi(q))(r)
    \end{align*}
    and $$\Phi(1)(r) = \ev_{1}(r) = \ev_r(1) = 1$$
    for all $r \in \R$. Thus, we have that $\Phi$ is a homomorphism of rings. Then, let $p \in \ker(\Phi)$ and for the sake of contradiction suppose $\deg(p) = n$ for some $n \geq 0$. Then, we have that $p$ has at most $n$ roots, $\{r_1,r_2,...,r_n\}$. Let $r \in \R$ with $r \notin\{r_1,r_2,...,r_n\}$. Then, by assumption we have $$0 = \Phi(p)(r) = \ev_p(r) = \ev_r(p)$$
    But, $\ev_r(p)$ is the remainder for the division of $p$ by $x - r$, so this implies $x-r$ divides $p$. However, we would then have $r$ as a root of $p$, but by assumption $r$ is not one of the $n$ roots of $p$, so this is a contradiction. Therefore, we must have that $p = 0$, so $\ker(\Phi)$ is trivial. Now, let $f = \sum_ia_i(\hat{x})^i \in \mathcal{P}(\R,\R)$. Observe that for $q = \sum_ia_ix^i$, we have for all $t \in \R$ $$\Phi(q)(t) = \ev_q(t) = \ev_t(q) = \sum_ia_it^i = (\sum_ia_i(\hat{x})^i)(t)$$
    so $\Phi(q) = f$, and in particular $\Phi$ is a surjection. Therefore, we conclude that $\Phi$ is an isomorphism of rings, so \begin{equation}
        \R[x] \cong \mathcal{P}(\R,\R)
    \end{equation}
\end{proof*}


\begin{defn}{}{}
    Let $R$ be an arbitrary unital ring. $P \in R[x]$ is called a polynomial with coefficients in $R$. For $P = a_0+a_1x+...+a_nx^n$, $a_0$ is called the \Emph{constant coefficient}. Now, assume $P \neq 0$. Then $\max\{i\geq 0:a_i \neq0\}$ is the \Emph{degree of P}, denoted $\deg(P)$. $a_{\deg(P)}$ is called the \Emph{leading coefficient of P}.
\end{defn}

\begin{defn}{}{}
    Let $R$ be a ring and let $r \in Z(R)$. Then, the map \begin{equation}
        \map{\ev_r:R[x]\rightarrow R}{P = \sum_ia_ix^i \mapsto P(r) = \sum_ia_ir^i}
    \end{equation}
    is a surjective ring homomorphism, called the evaluation at $r$. Denote $\ev_r(P)$ by $P(r)$.
    \begin{proof*}{}{}
        (Left to the reader)
    \end{proof*}
\end{defn}

\section{Division Algorithm}

\begin{defn}{}{}
    A non-zero polynomial is called \Emph{monic} if its leading coefficient is $1$.
\end{defn}

\begin{eg}{}{}
    \leavevmode
    \begin{enumerate}
        \item $\deg(\underbrace{3x^2+x+1}_{\in\Z[x]}) = 2$, leading coefficient $= 3$.
        \item $x^2+7x_1 \in \Z[x]$ is monic of degree $2$
        \item $2$ has degree $0$.
    \end{enumerate}
\end{eg}

\begin{rmk}{}{}
    $P,Q \in R[x]$, then if $PQ \neq 0$, then $\deg(PQ) \leq \deg(P) + \deg(Q)$. If $R$ is a domain and $P \neq 0$, $Q \neq 0$, then $\deg(PQ) = \deg(P) + \deg(Q)$.
    \begin{proof*}{}{}
        (Left to the reader)
    \end{proof*}
\end{rmk}

\begin{eg}{}{}
    $\Z/6\Z[x]$, $(2x)(3x+1) = 6x^2+2x = 2x$ of degree $1 < \deg(P) + \deg(Q) = 1 + 1 = 2$.
\end{eg}

\begin{cor}{}{}
    \leavevmode
    \begin{enumerate}
        \item If $R$ is a domain, then $R[x]$ is a domain
        \item The units of $R[x]$ are the units of $R$.
    \end{enumerate}
    \begin{proof*}{}{}
        (Left to the reader)
    \end{proof*}
\end{cor}

\begin{thm}{Division Algorithm}{}
    Let $R$ be a ring. Let $P,Q \in R[x]$. Assume that $P \neq 0$ and that the leading coefficient of $P$ is a unit ($\in R$). Then there are unique polynomials $f$ and $g \in R[x]$ such that: \begin{enumerate}
        \item $Q = fP + g$
        \item $g = 0$ or $\deg(g) < \deg(P)$
    \end{enumerate}
    ($g$ is called the remainder of the division of $Q$ by $P$)
\end{thm}
\begin{proof*}{}{}
    Write $\deg(Q) = m$ and $\deg(P) = n$. If $Q = 0$ or $m < n$ then $Q = 0P + Q$ does it. Hence, suppose $m \geq n$, and proceed by induction on $m$. Write $P = ux^n + a_{n-1}x^{n-1} + ...$ and $Q = b_mx^m + b_{m-1}x^{m-1} + ...$, where $u \in R^{\times}$ by hypothesis. Consider the new polynomial \begin{align*}
        g_1 &= Q - b_mu^{-1}x^{m-n}P \\
        &= (b_{m-1} - b_mu^{-1}a_{n-1})x^{m-1} + ...
    \end{align*}
    where we use the fact that $x$ is central in $R[x]$. Hence, either $g_1 = 0$ or $\deg(g_1) < m$ so, by induction, polynomials $q_1$ and $r$ exist such that $g_1 = Pq_1 + r$ and either $r = 0$ or $\deg(r) < \deg(P)$. But then $$Q = g_1 + b_mu^{-1}x^{m-n}P = (q_1 + b_mu^{-1}x^{m-n})P + r$$
    Hence, the induction is satisfied, so $f$ and $g$ exist satisfying the claim. 
    
    To prove uniqueness, suppose that also $Q = f_1P + g_1$, where either $g_1 = 0$ or $\deg(g_1) < \deg(P)$. Then $(g-g_1) = (f_1-f)P$. If $(f_1-f) \neq 0$, then since the leading coefficient of $P$ is a unit $(f_1-f)P \neq 0$, and that $$\deg(g-g_1) = \deg[(f_1-f)P] = \deg(f_1-f)+\deg(P)$$ But, this implies $\deg(g-g_1) \geq \deg(P)$, but $\deg(g-g_1) \leq \max\{\deg(g),\deg(g_1)\} < \deg(P)$, a contradiction. Thus, we must have that $(f_1-f) = 0$, and whence $(g-g_1) = (f_1 - f)P = 0$, proving uniqueness.
\end{proof*}

\begin{cor}{}{}
    A non-zero polynomial of degree $n$ over any field has at most $n$ roots.
\end{cor}
\begin{proof*}
    (Left to the reader)
\end{proof*}


\begin{eg}{}{}
    \leavevmode
    \begin{enumerate}
        \item For $\Z[x]$, $P = -x+1$, $Q = x^2$, $Q = P(-x-1) + 1$, where $\deg(1) = 0 < 1 = \deg(P)$
        \item Conversely, $P = 0.Q + (-x+1)$, where $\deg(-x+1) = 1 < 2 = \deg(x^2)$.
    \end{enumerate}
\end{eg}

\begin{cor}{}{}
    Let $R$ be a commutative ring with $a \in R$, and $P \in R[x]$. \begin{enumerate}
        \item $\ev_a(P) = 0$ if and only if $P = (x-a)Q$ for some $Q \in R[x]$.
        \item The remainder of the division of $P$ by $x-a$ is $P(a)$.
    \end{enumerate}
    \begin{proof*}{}{}
        First, observe that if $P = (x-a)Q$ for some $Q \in R[x]$. Then $\ev_a(P) = 0\ev_a(Q) = 0$, satisfying the implication. On the other hand, since the leading coefficient of $x-a$ is a unit, we have by the division algorithm that there exists $f,g \in R[x]$ such that $P = f(x-a) + g$, where $g = 0$ or $\deg(g) < \deg(x-a) = 1$. Thus, $g = r$ for some $r \in R$. Then, observe that $\ev_a(P) = \ev_a(f(x-a) + r) = \ev_a(f)0+r = r$, since $\ev_a$ is a ring homomorphism. Note that this proves the second claim. Now, by assumption $\ev_P = 0$, $r = 0$. Thus, $P = (x-a)f$, completing the proof.
    \end{proof*}
\end{cor}


\begin{defn}{}{}
    Let $R$ be a commutative ring. We say that a polynomial $f \in R[X]$ \Emph{divides} a polynomial $g \in R[X]$ denoted $f\;\vert\;g$, and that $f$ is a \Emph{divisor} of $g$ if there exists $P \in R[X]$ such that $g = fP$.
\end{defn}

\begin{cor}{}{}
    Let $\F$ be a field. Let $f,g \in \F[X]$ not both zero. Then \begin{equation}
        (f,g) := \{fP+gQ: P,Q\in\F[X]\} = (d)
    \end{equation}
    where $d$ is the monic generator of minimal degree, which satisfies \begin{enumerate}
        \item $d$ divides $f$ and $g$
        \item If a polynomial $P$ divides $f$ and $g$, then $P$ divides $d$
        \item There exist $Q_1, Q_2 \in \F[X]$ such that $d = fQ_1 + gQ_2$ (one says that $d$ is a linear combination of $f$ and $g$)
    \end{enumerate}
    Thus, every ideal $I$ in the ring $R = F[x]$ is principal, $I = (f)$, generated by the monic polynomial $f$ in $I$ of least degree.
    \begin{prooflab*}{A}
        Let $I$ be an ideal. If $I \neq (0)$, take $f \in I$ of minimal degree, n. Scale $f$ by $c = a_n^{-1}$ to make $f$ monic. Note that since $c \in F[x]$, $c.f \in I$. Let $h$ be another polynomial in $I$, and write $h(x) = q(x)f(x) + r(x)$, with the degree of $r(x)$ less than $f(x)$. Note that $q(x) \in F[x]$, so $q(x)f(x) \in I$, and $h(x) - q(x)f(x) \in I$. Thus, $r(x) \in I$. But, $r(x)$ has a smaller degree than $f$, so $r(x) = 0$. Thus, $h(x) = q(x)f(x)$, so $f(x)$ divides $h(x)$, and $I = (f)$. 
    \end{prooflab*}
    \begin{prooflab*}{B}
        (Left to the reader)
    \end{prooflab*}
\end{cor}

\begin{rmk}{}{}
        Thus, the set of ideals is in a one-to-one correspondence with the set of monic polynomials, and the ideal associated to $f$, $I_f$, contains the ideal generated by the monic polynomial $g$, that is $I_f \supset I_g$, if and only if $f$ divides the polynomial $g$. Namely, $g(x)=f(x)q(x)$ for some $q(x) \in F[x]$. 
\end{rmk}

\begin{eg}{Non-principal Ideals}{}
        Take the ring $R = F[x,y] = \left\{\sum\limits_{i=1,j=1}^{n,m} a_{ij}x^iy^j:a_{ij} \in F\right\}$. Consider the map $h:R \rightarrow F$ by $f(x,y)\mapsto f(0,0)$. The kernel of $h$ is not generated by one element (so it's not principal). In fact, $\ker h = (x,y) = \{rx+sy:r,s \in R\}$.
\end{eg}




\begin{defn}{}{}
    Let $f,g \in \F[x]$ not both zero, then $d$ in the above corollary is called the \Emph{greatest common divisor} of $f$ and $g$, denoted $\gcd(f,g)$.
    \begin{enumerate}
        \item[$\drsh$] If $\gcd(f,g) = 1$, then $f$ and $g$ are said to be \Emph{relatively prime}. 
    \end{enumerate}
\end{defn}


\begin{qest}
    How do we compute the $\gcd$?
\end{qest}

\begin{eg}{}{}
    In $\Q[x]$
    \begin{enumerate}
        \item $(x^2+1,x) = (1)$ because $1 = (x^2+1) + (-x)(x) \in (x^2+1,x)$
        \item $\gcd(x^2-1,2x+2) = \gcd((x-1)(x+1),2(x+1)) = (x+1)$
    \end{enumerate}
\end{eg}


\begin{rmk}{}{}
    In general, there is the \Emph{euclidean algorithm} for polynomials over a field.
\end{rmk}

\begin{lem}{}{}
    Let $f,g \in \F[x]$, $\F$ a field, not both zero. If $g = Pf + Q$ for some $P,Q \in \F[x]$ then $\gcd(f,g) = \gcd(f,Q)$.
\end{lem}
\begin{proof*}{}{}
    Let $d = \gcd(f,g)$, $d' = \gcd(f,Q)$. Then I claim $(f,Q) = (f,g)$. Note $f \in (f,Q)$ and $g = Pf + Q \in (f,Q)$ so $(f,g) \subseteq (f,Q)$. Similarly, $f \in (f,g)$ and $Q = g - Pf \in (f,g)$, so $(f,Q) \subseteq (f,g)$. Thus, $(f,g) = (f,Q)$ so $(d) = (f,g) = (f,Q) = (d')$. Hence, $d = d'$ and the proof is complete.
\end{proof*}

\begin{namthm*}{Euclidean Algorithm}{}
    Let $f,g \in \F[x]$, not both zero. Assume $f \neq 0$, so the leading coefficient of $f$ is $a \neq 0$. Then $a$ is a unit in $\F$ since $\F$ is a field. By the division algorithm there exist unique $P, Q \in \F[x]$ such that \begin{equation}
        g = Pf + Q
    \end{equation}
    with $Q = 0$ or $\deg(Q) < \deg(f)$. By the Lemma $\gcd(f,g) = \gcd(f,Q)$. If $Q = 0$, $\gcd(f,g) = \gcd(f,0) = a^{-1}f$, where $a$ is the leading coefficient of $f$. Otherwise, if $Q \neq 0$ we apply the division algorithm again to obtain $P',Q' \in \F[x]$ satisfying $f = P'Q + Q'$ for $Q' = 0$ or $\deg(Q') < \deg(Q)$. We then have $\gcd(f,g) = \gcd(f,Q) = \gcd(Q, Q')$. This process must terminate eventually because the degree of the remainder strictly decreases at every step and must be non-negative (or the remainder is $0$).
\end{namthm*}

\begin{eg}{}{}
    The greatest common divisor of $f = x^3 + x + 2,$ $g = x^3 + x^2 + x + 1$ in $\Q[x]$: First $f = g\cdot 1 + (1-x^2)$. Then $g = (1-x^2)(-x-1) + 2x+2$, and finally $(-x^2+1) = (-x/2+1/2)(2x+2) + 0$. Thus, $\gcd(f,g) = \gcd(g,1-x^2) = \gcd(1-x^2,2x+2) = \gcd(2x+2,0) = x+1$, by making the remainder monic. Working backwards we have $$(x+1) = g/2 - (x+1)(x^2-1)/2 = g/2 - (x+1)(g-f)/2 = -gx/2+f(x+1)/2$$
\end{eg}


\subsection{Solutions to Polynomials}

\begin{prop}{}{}
    If $p \in \Z[x]$ and $\ev_c(p) = 0$ for some $c \in \Z$, then $\ev_c(p) = 0$ in $\Z/m\Z$ for all $m \in \Z^{+}$. The contrapositive is important so it shall be stated: If there exists $m \in \Z^{+}$ such that $\ev_c(p) \neq 0$ in $\Z/m\Z$ for a polynomial $p \in \Z[x]$, then $\ev_c(p) \neq 0$ in $\Z$, and in particular $c$ is not a root of $p$ in $\Z$.
    \begin{proof*}{}{}
        Let $p \in \Z[x]$ and suppose $\ev_c(p) = 0$. Then, let $m \in \Z^{+}$. Write $p = \sum_ia_ix^i$. Then observe that $\pi(p) = \sum_i[a_i]_mx^i = \left[\sum_ia_ix^i\right]$. Then, since $\ev_c$ is a ring homomorphism we have that $$\ev_c(\pi(p)) = \sum_i[a_i]_mc^i = \left[\sum_ia_ic^i\right] = \left[\ev_c(p)\right] = [0]$$ in $\Z/m\Z$, completing the proof.
    \end{proof*}
\end{prop}

\section{Substitution Principle}

\begin{namthm*}{Substitution Principle}{}
        Let $\phi:R\rightarrow R'$ be a ring homomorphism. \begin{enumerate}
                \item Given an element $\alpha \in R'$, there is a unique homomorphism $\Phi: R[x] \rightarrow R'$ which agrees with the map $\phi$ on constant polynomials, and which sends $x \mapsto \alpha$
                \item More generally, given $\alpha_1,...,\alpha_n \in R'$, there is a unique homomorphism $\Phi: R[x_1,...,x_n] \rightarrow R'$ from the polynomial ring in n variables to $R'$, which agrees with $\phi$ on constant polynomials and which sends $x_i\mapsto \alpha_i$, for $i=1,2,...,n$
        \end{enumerate}
\end{namthm*}
\begin{proof*}{}{}
        With vector notation for indices, the proof of (2) is identical to that of (1). Let us denote the image of an element $r \in R$ in $R'$ by $r'$. Using the fact that $\Phi$ is a homomorphism which restricts to $\phi$ on $R$, and sends $x_v \mapsto \alpha_v$, we find that it acts on a polynomial $f(x) = \sum r_ix^i$ by sending \begin{equation}
                \sum r_ix^i \mapsto \sum\phi(r_i)\alpha^i = \sum r_i'\alpha^i
        \end{equation}
        In other words, $\Phi$ acts on the coefficients of a polynomial as $\phi$, and it substitutes $\alpha$ for $x$. Since this formula describes $\Phi$ completely for us, we have proved the uniqueness of the substitution homomorphism. To prove its existence, we take this formula as the definition of $\Phi$, and we show that the map is a ring homomorphism $R[x] \rightarrow R'$. Since $\phi$ is a ring homomorphism, $\Psi$ sends $1$ to $1$, and by the above formula, it is compatible with addition of polynomials. Using the formula we also find that it is compatible with multiplication as \begin{align*}
                \Psi(fg) &= \Psi\left(\sum a_ib_jx^{i+j}\right) \\
                &= \sum \Psi(a_ib_jx^{i+j})\\
                &= \sum\limits_{i,j} a_i'b_j'\alpha^{i+j}\\
                &= \left(\sum\limits_ia_i'\alpha^i\right)\left(\sum\limits_jb_j'\alpha^j\right) \\
                &=\Psi(f)\Psi(g)
        \end{align*}
\end{proof*}


\begin{eg}{}{}
        We consider the case of a homomorphism $\Z \rightarrow \Z/p\Z$. This map extends to a homomorphism \begin{equation}
                \Z[x] \rightarrow \Z/p\Z[x], f(x) = a_nx^n+\hdots + a_0 \mapsto \overline{a_n}x^n+\hdots + \overline{a_0} = \overline{f}(x)
        \end{equation}
        where $\overline{a_i}$ denotes the \Emph{residue class} of $a_i$ modulo $p$. We call the polynomial $\overline{f}(x)$ the \Emph{residue of $f(x)$ modulo $p$}.
\end{eg}


\begin{cor}{}{}
        Let $x = (x_1,...,x_n)$ and $y = (y_1,...,y_m)$ denote set of variables. There is a unique isomorphism $R[x,y] \xrightarrow{\sim}R[x][y]$ which is the identity on $R$ and which sends the variables to themselves.
\end{cor}
\begin{proof*}{}{}
        Note that $R$ is a subring of $R[x]$, and that $R[x]$ is a subring of $R[x][y]$. So $R$ is also a subring of $R[x][y]$. Consider the inclusion map $\phi:R\hookrightarrow R[x][y]$. The Substitution Principle tells us that there is a unique homomorphism $\Phi:R[x,y] \rightarrow R[x][y]$ which extends the map and sends variables $x_{\mu},y_{\nu}$ wherever we wish. Thus, we can send the variables to themselves. The map $\Phi$ constructed is thus the desired isomorphism. Using the Substitution Principle once more, we note that $R[x]$ is a subring of $R[x,y]$, so we can extend the inclusion map $\psi:R[x] \rightarrow R[x,y]$ to a map $\Psi:R[x][y] \rightarrow R[x,y]$ by sending $y_j$ to itself. The composed homomorphism $\Psi\Phi: R[x,y] \rightarrow R[x,y]$ is the identity on $R$ and on $\{x_{\mu},y_{\nu}\}$. By uniqueness of the Substitution Principle, $\Psi\Phi$ is the identity map. Similarly, $\Phi\Psi$ is the identity on $R[x][y]$. Thus, $\Phi$ is a bijective homomorphism, so it is an isomorphism.
\end{proof*}


\begin{prop}{}{}
        Let $\mathcal{R}$ denote the ring of continuous real-valued functions on $\R^n$. The map $\phi:\R[x_1,...,x_n]\rightarrow \mathcal{R}$ sending a polynomial to its associated polynomial function is an injective homomorphism.
\end{prop}
\begin{proof*}{}{}
        The existence of this homomorphism follows from the Substitution Principle. To prove injectivity, it is enough to show that if the function associated to a polynomial $f(x)$ is the zero function, then $f(x)$ is the zero polynomial. Let the associated function be $\widetilde{f}(x)$. If $\widetilde{f}(x)$ is identically zero, then all its derivatives are zero too. On the other hand we can differentiate a formal polynomial by using the power rule and the linearity of the derivative. If some coefficients of $f(x)$ are nonzero, then the constant term of a suitable derivative will be nonzero too. Hence, that derivative will not vanish at the origin. Therefore, $\widetilde{f}(x)$ can't be the zero function.
\end{proof*}


\begin{prop}{}{}
        There is exactly one ring homomorphism \begin{equation}
                \phi:\Z\rightarrow R
        \end{equation}
        from the ring of integers to an arbitrary ring $R$. It is the map defined by $\phi(n) = 1_R+...+1_R$ n-times if $n > 0$, and $\phi(-n) = -\phi(n)$.
\end{prop}


\begin{rmk}{}{}
        This allows us to identify the images of the integers in an arbitrary ring $R$. We can hence interpret the symbol $3$ as $1+1+1$ in $R$.
\end{rmk}


\section{Roots and Factorization}

For this section let $R$ be a commutative ring and $P \in R[x]$.

\begin{defn}{}{}
    $r \in R$ is a \Emph{root} of $P$ if $\ev_r(P) = P(r) = 0$. Note that this happens if and only if $P = (x-r)Q$ for some $Q \in R[x]$.
\end{defn}

\begin{qest}
    What if $Q(r) = 0$ as well?
\end{qest}

\begin{defn}{}{}
    Let $\alpha \in R$ be a root of $P$. We say that $\alpha$ is a root of $P$ of \Emph{multiplicity} $n \geq 1$ if $P = (x-\alpha)^nQ'$ for some $Q' \in R[x]$ and $Q'(\alpha) \neq 0$.
\end{defn}

\begin{prop}{}{}
    Let $R$ be an integral domain, and let $P \neq 0$ be a polynomial of degree $n$. Then $P$ has at most $n$ roots counted with multiplicities.
\end{prop}
\begin{proof*}{}{}
    We argue by induction on the degree of $P$. If $n = 0$ then $P = c \neq 0$, so $P$ has no roots. Thus, the base case holds. Then, suppose that there exists $k \geq 0$ such that for all $j \leq k$, if $n = k$ $P$ has at most $k$ roots counting multiplicities. Then, consider $n = k+1$. If $P$ has no roots then we are done. Otherwise, let $\alpha$ be a root of $P$. Then $P = (x-\alpha)Q$ for some $Q \in R[x]$ such that $\deg(Q) = k+1-1 = k$. Thus, by our induction hypothesis $Q$ has at most $k$ roots counting with multiplicities. Thus, $P$ has at most $k + 1$ roots counting with multiplicities, as desired.
\end{proof*}

\begin{rmk}{}{}
    Note that if $R$ is not an integral domain this is not true.
    \begin{enumerate}
        \item[$\drsh$] If $R = \Z/6\Z$, then observe $$P = (x-2)(x-3) = x^2-5x+6 = x^2-5x = x(x-5)$$ 
        so $\{0,2,3,5\}$ are roots of $P$ and $P$ is of degree $2$.
    \end{enumerate}
    Moreover, even if $R$ is an integral domain, $P \in R$ with $\deg(P) > 1$ may have no roots. For example, $x^2+1 \in \R[x]$ has no roots in $\R$.
\end{rmk}

\begin{cor}{}{}
    If $R$ is an integral domain and $0 \neq P \in R[x]$ of degree $n$, then $P$ has at most $n$ distinct roots.
\end{cor}

\begin{defn}{Irreducible}{}
    Let $R$ be an integral domain.
    \begin{enumerate}
        \item An element $a \in R$ is \Emph{irreducible} if
        \begin{enumerate}
            \item $a \notin R^{\times}$ and $a \neq 0$
            \item If $a =bc$ for some $b,c \in R$, then $b \in R^{\times}$ or $c \in R^{\times}$
        \end{enumerate}
        \item If $P \in R[x]$ is irreducible, we say that $P$ is \Emph{irreducible over $R$}.
    \end{enumerate}
\end{defn}

\begin{eg}{}{}
    \leavevmode
    \begin{enumerate}
        \item $x^2+1$ is irreducible over $\R$
        \item $x^2+1 = (x+i)(x-i)$ is not irreducible over $\C$
        \item $2x+2$ is irreducible over $\Q$, but not over $\Z$ as $2 \notin \Z^{\times} = \{1,-1\}$
        \item A linear polynomial (i.e$\rangle$ of degree $1$) over a field is irreducible.
        \begin{proof*}{}{}
            Let $F$ be a field and let $P \in F[x]$ of degree $1$. Then $P \neq 0$ and $P \notin F[x]^{\times}$. Moreover, if $P = fg$ for some $f,g \in F[x]$ then $\deg(fg) = \deg(f)+\deg(g) = \deg(P) = 1$, so either $\deg(f) = 0$ or $\deg(g) = 0$. Thus, either $g \in F$ or $f \in F$ with $g,f \neq 0$, so in particular either $g \in F[x]^{\times}$ or $f \in F[x]^{\times}$. Therefore, $P$ is irreducible over $F$.
        \end{proof*}
    \end{enumerate}
\end{eg}

\begin{prop}{}{}
    Let $F$ be a field and $P \in F[x]$ with $\deg(P) \geq 2$.  
    \begin{enumerate}
        \item If $P$ is irreducible over $F$, then $P$ has no roots in $F$
        \item If $\deg(P) \in \{2,3\}$, then $P$ is irreducible over $F$ if and only if $P$ has no roots in $F$.
    \end{enumerate}
\end{prop}
\begin{proof*}{}{}
    If $P$ has a root $a \in F$, then $P = (x-a)Q$ for some $Q \in F[x]$. But, $\deg(P) \geq 2$ so $\deg(Q) \geq 1$, and hence $Q \notin F[x]^{\times}$. Thus, $P$ is not irreducible over $F$. On the other hand, suppose $\deg(P) \in \{2,3\}$. Suppose $P$ is not irreducible over $F$ so $P = p_1p_2$ for some $p_1, p_2 \in F[x]$ such that $\deg(p_1),\deg(p_2) \geq 1$. But, $\deg(p_1)+\deg(p_2) = \deg(P) \in \{2,3\}$ so either $\deg(p_1) = 1$ or $\deg(p_2) = 1$. Without loss of generality suppose $\deg(p_1) = 1$. Then $p_1 = ax+b$ for $a,b \in F$ and $a \neq 0$. Since $F$ is a field $a^{-1} \in F$ and $\ev_{a^{-1}(-b)}(p_1) = 0$. Hence, as $P = p_1p_2$ $P$ has a root in $F$, completing the proof.
\end{proof*}

\begin{rmk}{}{}
    \leavevmode
    \begin{enumerate}
        \item In general no roots $\cancel{\implies}$ irreducible
        \begin{enumerate}
            \item[$\drsh$] For example, $(x^2+1)^2 \in \R[x]$ is not irreducible but it has no roots in $\R$. 
        \end{enumerate}
        \item $x^2+x+1$ is irreducible over $\Z/2\Z$ (no roots)
        \item $x^2-2$ is irreducible over $\Q$ (no roots), but it has roots over $\R$.
    \end{enumerate}
\end{rmk}

\begin{defn}{Algebraically Closed}{}
    A field $F$ is \Emph{algebraically closed} if every non-constant polynomial $P \in F[x]$ has a root.
\end{defn}

\begin{namthm*}{Fundamental Theorem of Algebra}{}
    The field of complex numbers $\C$ is algebraically closed. So, $P = a(x-r_1)(x-r_2)...(x-r_n)$ for $P \in \C[x]$, $\deg(P) = n$, $a \in \C\backslash\{0\}$ the leading coefficient of $P$, and $\{r_1,r_2,..,r_n\}$ the roots of $P$ (not necessarily distinct).
\end{namthm*}

\subsection{\texorpdfstring{Polynomials over $\Q$ and $\Z$}{}}

\begin{namthm*}{Rational Roots Theorem}{}
    Let $P \in \Z[x]$, $P = a_0+a_1x+...+a_nx^n$, $a_n \neq 0 \neq a_0$. Then every root of $P$ in $\Q$ is of the form $\frac{c}{d}$ such that $c\;\vert\;a_0$ and $d\;\vert\;a_n$. In particular, if $P$ is monic, i.e $a_n = 1$, then every rational root of $P$ is in $\Z$.
\end{namthm*}
\begin{proof*}{}{}
    Suppose $\frac{c}{d}$ is a root of $P$ in $\Q$, and assume $\gcd(c,d) = 1$. Then, $$P\left(\frac{c}{d}\right) = a_0+a_1\frac{c}{d}+ ... + a_n\frac{c^n}{d^n} = 0$$
    Multiply by $d^n$ to obtain $$a_0d^n+a_1cd^{n-1}+...+a_{n-1}c^{n-1}d+a_nc^n = 0$$
    Then, since $a_1cd^{n-1},...,a_nc^n$ are divisible by $c$, we must have that $c$ divides $a_0d^n$. But, $\gcd(c,d) = 1$, so as can be shown by induction, $\gcd(c,d^n) = 1$. Hence, $c$ divides $a_0$. Indeed, $cx+d^ny = 1$ for some $x,y \in \Z$, so $a_0 = c(a_0x+ky)$, where $a_0d^n = ck$. Similarly, $a_nc^n$ is divisible by $d$ as $a_0d^n,...,a_{n-1}c^{n-1}d$ are divisible by $d$. Thus, again $\gcd(d,c^n) = 1$ so $d$ divides $a_n$, completing the proof.
\end{proof*}

\begin{eg}{}{}
    $P = x^3+2x^2+\frac{3}{5}x+2$ is irreducible over $\Q$. Indeed, since $\deg(P) = 3$, $P$ is irreducible over $\Q$ if it has no roots in $\Q$. To put $P$ in the form of Rational Roots Theorem we multiply by $5$:
    $$5P = 5x^3+10x^2+3x+10 \in \Z[x]$$
    By the Rational Roots Theorem, if $5P$ has a root $\frac{c}{d} \in \Q$ then $c\;\vert\;10$ and $d\;\vert\;5$. Thus, $d \in \{\pm 1,\pm 5\}$ and $c \in \{\pm1,\pm2,\pm5,\pm10\}$. In particular $$\frac{c}{d} = \{\pm1,\pm2,\pm5,\pm10,\pm\frac{1}{5},\pm\frac{2}{5}\}$$
    Upon direct computation none of these values are roots of $5P$, so in particular $5P$ has no roots in $\Q$. Thus, $P$ has no roots over $\Q$ and is hence irreducible over $\Q$.
\end{eg}

\begin{namthm*}{Gauss' Lemma}{}
    Let $f,g,h \in \Z[x]$ such that $f = gh$. If a prime $p$ divides every coefficient of $f$, then $p$ divides every coefficient of $g$ or $p$ divides every coefficient of $h$.
\end{namthm*}
\begin{proof*}{}{}
    Consider the surjective ring homomorphism $$\map{\Z[x]\xrightarrow{\alpha}{\Z/p\Z[x]}}{a_0+a_1x+...+a_nx^n\mapsto [a_0]_p+[a_1]_px+...+[a_n]_px^n}$$
    Then $[0]_p = \alpha(f) = \alpha(g)\alpha(h)$. But, since $\Z/p\Z$ is a field, it is an integral domain and we must have that $\alpha(g) = 0$ or $\alpha(h) = 0$. That is, $g \in (p)$ or $h \in (p)$, completing the proof.
\end{proof*}

\begin{defn}{}{}
    For all $f \in \Z[x]$, $\alpha(f) \in \Z/p\Z[x]$ is called \Emph{reduction modulo p of $f$}.
\end{defn}

\begin{cor}{}{}
    Let $f$ be a non-constant polynomial in $\Z[x]$. \begin{enumerate}
        \item If $f =gh$ with $g,h \in \Q[x]$, then there exist $g_0,h_0 \in \Z[x]$ so that $\deg(g_0) = \deg(g)$, $\deg(h_0) = \deg(h)$ and $f = g_0h_0$
        \item $f$ is irreducible over $\Q$ if and only if $f$ cannot be written $f = g_0h_0 \in \Z[x]$, where $g_0,h_0$ are non-constant.
    \end{enumerate}
\end{cor}
\begin{proof*}{}{}
    Let $f \in \Z[x]$, with $\deg(f) \geq 1$. 
    
    1) Suppose $f = gh$ with $g,h \in \Q[x]$. Let $a$ and $b$ be least common multiples of the denominators of the coefficients of $g$ and $h$, respectively. Then $g' = ag$ and $h' = bh$ are in $\Z[x]$. Moreover, we have the equation $abf = g'h'$ in $\Z[x]$. If $ab = 1$ then we're done, so suppose $ab > 1$ and let $p$ be a prime dividing $ab$. Then by Gauss' Lemma $p$ divides $g'$ or $h'$. Hence, $p$ can be cancelled to give \begin{equation}
        \frac{ab}{p}f = g_2h_2
    \end{equation}
    in $\Z[x]$. Repeat for all prime factors of $ab$ to obtain $f = g_0h_0$ in $\Z[x]$ with $\deg(g_0) = \deg(g)$ and $\deg(h_0) = \deg(h)$.
    
    2) If $f = g_0h_0 \in \Z[x]$ for $g_0,h_0$ non-constant, then $0 \neq g_0,h_0 \notin \Q[x]^{\times}$, so $f$ is not irreducible. Conversely, if $f$ is not irreducible over $\Q$ then $f = gh \in \Q[x]$ and the result follows from 1), completing the proof.
\end{proof*}

\begin{defn}{}{}
    Let $f \in \Z[x]$. A \Emph{proper factorization} of $f$ is $f = g_0h_0$ where $\deg(g_0) \geq 1$ and $\deg(h_0) \geq 1$.
\end{defn}


\begin{namthm*}{Modular Irreducibility Test}{}
    Let $0 \neq f \in \Z[x]$ such that there exists a prime number $p$ with: \begin{enumerate}
        \item $p$ does not divide the leading coefficient of $f$
        \item The reduction $\alpha(f)$ of $f$ modulo $p$ is irreducible over $\Z/p\Z$
    \end{enumerate}
    Then $f$ is irreducible over $\Q$.
\end{namthm*}
\begin{proof*}{}{}
    Suppose $f \in \Z[x]$ such that $f$ satisfies the conditions of the theorem. Then, for the sake of contradiction suppose $f$ is not irreducible over $\Q$. Then $f = gh$ for some non-constant $g,h \in \Q[x]$. By the corollary to Gauss' Lemma we have that $f = g_0h_0$ for $g_0,h_0 \in \Z[x]$ non-constant polynomials. Then, we have that $\alpha(f) = \alpha(g_0)\alpha(h_0)$. Note that if $a$ and $b$ are the leading coefficients of $g_0$ and $h_0$ respectively, then $ab$ is the leading coefficient of $f$, which by assumption is not divisible by $p$. Thus, $a$ and $b$ are not divisible by $p$, so $\deg(\alpha(g_0)) \geq 1$ and $\deg(\alpha(h_0)) \geq 1$. But, this implies that $\alpha(g_0),\alpha(h_0)$ are not units in $\Z/p\Z[x]$, so $\alpha(f) = \alpha(g_0)\alpha(h_0)$ is not irreducible over $\Z/p\Z$, contradicting the initial assumptions. Therefore, $f$ must be irreducible over $\Q$.
\end{proof*}

\begin{eg}{}{}
    \leavevmode
    \begin{enumerate}
        \item $f = x^3+4x^2+6x+2 \in \Z[x]$ is irreducible over $\Q$. Indeed, take $p = 3$, then $f\mod 3 = x^3+x^2+2 \in \Z/3\Z[x]$, which has no roots in $\Z/3\Z[x]$, so $f\mod 3$ is irreducible over $\Z/3\Z[x]$. Applying the Modular Irreducibility Test, $f$ is irreducible over $\Q$.
    \end{enumerate}
\end{eg}

\begin{rmk}{}{}
    Irreducible over $\Q$ $\cancel{\implies}$ irreducible over $\Z/p\Z$.
    \begin{enumerate}
        \item[$\drsh$] Eg: $x^2 - 2$ is irreducible over $\Q$, but has a root in $\Z/2\Z$. 
        \item[$\drsh$] For $p=2$, $x^4 + 1$ is irreducible over $\Q$, but not over $\Z/p\Z$.
    \end{enumerate}
\end{rmk}

\begin{namthm*}{Eisenstein's Criterion}{}
    Let $f = a_0+a_1x+...+a_nx^n \in \Z[x]$ with $n \geq 1$ and $a_n \neq 0$. Suppose there exists a prime $p$ such that \begin{enumerate}
        \item $p\;\vert\;a_i$ for all $0 \leq i < n$,
        \item $p\;\cancel{\vert}\;a_n$
        \item $p^2\;\cancel{\vert}\;a_0$
    \end{enumerate}
    Then $f$ is irreducible in $\Q[x]$
\end{namthm*}
\begin{proof*}{}{}
    Suppose $f \in \Z[x]$, $\deg(f) \geq 1$ and let $p \in \Z$ a prime satisfying the conditions of the theorem. If $f$ is not irreducible in $\Q[x]$, then there exists a proper factorization $f = gh$, $g,h \in \Z[x]$ by the corollary to Gauss' Lemma. Write $g = b_0+b_1x+...+b_kx^k$ and $h = c_0+c_1x+...+c_lx^l$, so we have $a_0 = b_0c_0$, $a_n = b_mc_l$, and $n = m+l$. Note, since $p$ does not divide $a_n$, $p$ does not divide $b_k$ nor $c_l$. Let $\alpha:\Z[x] \rightarrow \Z/p\Z[x]$ be the reduction modulo $p$. Then $\alpha(f) = [a_n]_px^n = \alpha(g)\alpha(h)$. Since $p$ does not divide the leading coefficients of $g$ nor $h$, this is a proper factorization. Note that since $p^2$ does not divide $a_0$, $p$ divides $b_0$ or $c_0$ but not both. Without loss of generality suppose $p$ divides $b_0$. Then, let $b_m$ be the first element of $b_0,b_1,...,b_k$ for which $p$ does not divide (this is possible as $p$ does not divide $b_k$). Then, note that $$a_m = b_mc_0 + b_{m-1}c_1 + ... + b_1c_{m-1} + b_0c_m$$ Then, since $m \leq k < n$, $p$ divides $a_m$. Moreover, by construction $p$ divides $b_{m-i}c_i$ for all $i \geq 1$. Thus, it follows that $p$ must divide $b_mc_0$, so $p$ divides $b_m$ or $p$ divides $c_0$. But, by assumption $p$ does not divide $b_m$ and $p$ does not divide $c_0$, leading to a contradiction. Therefore, $f$ must be irreducible over $\Q$.
\end{proof*}

\begin{note}{}{}
    The reduction modulo $p$ of $f$, $[a_n]_px^n$, is not irreducible over $\Z/p|Z$ if $n > 1$.
\end{note}

\begin{eg}{}{}
    \leavevmode
    \begin{enumerate}
        \item $x^{1000}+3x+6$ is irreducible over $\Q$. Apply Eisenstein's Criterion for $p = 3$, where $p\;\vert\;3,6$, $p^2\;\cancel{\vert}\;6$ and $p\;\cancel{\vert}\;1$. Thus, it is irreducible over $\Q$.
        \item If $p$ is a prime, the $p$th \Emph{cyclotomic polynomial} \begin{equation}
            \Phi_p = x^{p-1}+x^{p-2}+...+x+1
        \end{equation}
        is irreducible over $\Q$.
        \begin{proof*}{}{}
            Replacing $x$ by $x+1$, it suffices to show that $\Phi_p(x+1)$ is irreducible. Observe that $$(x-1)\Phi_p = (x-1)(x^{p-1}+x^{p-2}+...+x+1) = x^p - 1$$
            Replacing $x$ by $x+1$, $x\Phi_p(x+1) = (x+1)^p-1$, so by the binomial theorem $$\Phi_p(x+1) = x^{p-1} + \begin{pmatrix}p \\ 1 \end{pmatrix}x^{p-2} + ... + \begin{pmatrix}p \\ p-2 \end{pmatrix}x + p$$
            But, $p$ divides $\begin{pmatrix}p \\ k \end{pmatrix}$ for all $1 \leq k \leq p-1$, and $p^2\;\cancel{\vert}\;p$, so by Eisenstein's Criterion, $\Phi_p(x+1)$ is irreducible over $\Q$.
        \end{proof*}
    \end{enumerate}
\end{eg}


\subsection{Parallels between the Integers and Polynomials over a Field}


\begin{rmk}{Parallels}{}
        \leavevmode
        \begin{enumerate}
                \item \begin{enumerate}
                                \item[$\Z$] Integral domain that is not a field
                                \item[${F[x]}$] Same
                \end{enumerate}
                \item \begin{enumerate}
                                \item[$\Z$] Principal ideal domain ($I = n\Z$)
                                \item[${F[x]}$] Same ($I = (d)$ for $d$ monic)
                \end{enumerate}
                \item \begin{enumerate}
                                \item[$\Z$] For $n \in \Z$, $n \neq 0, \pm 1$, $\Z/n\Z$ is an integral domain if and only if $n = \pm p$, $p$ a prime if and only if $\Z/n\Z$ is a field 
                                \item[${F[x]}$] For $P \in F[x]$, $\deg(P) \geq 0$ (i.e. $P \neq 0$ and $P \notin F[x]^{\times}$) $F[x]/(P)$ is an integral domain if and only if $P$ is irreducible over $F$ (i.e. $P = aQ$ for $Q$ monic irreducible and $a \in F^{\times}$) if and only if $F[x]/(P)$ is a field
                \end{enumerate}
                \item \begin{enumerate}
                                \item[$\Z$] Unique Factorization Domain (in terms of unique prime numbers) 
                                \item[${F[x]}$] Same (in terms of unique monic irreducible polynomials)
                \end{enumerate}
                \item \begin{enumerate}
                                \item[$\Z$] Expression of $\gcd$ as largest common factor in the UFD factorization
                                \item[${F[x]}$] Same
                \end{enumerate}
        \end{enumerate}
\end{rmk}








\section{Polynomials over a Field}

\begin{thm}{}{}
    Let $F$ be a field, $P \in F[x]$, $\deg(P) > 0$. Then the following are equivalent:\begin{enumerate}
        \item $F[x]/(P)$ is an integral domain
        \item $P$ is irreducible in $F[x]$
        \item $F[x]/(P)$ is a field
    \end{enumerate}
\end{thm}
\begin{proof*}{}{}
    Suppose $F$ is a field and $P \in F[x]$, with $\deg(P) > 0$.
    
    [$1 \implies 2$] Suppose $F[x]/(P)$ is an integral domain. Then, $(P)$ is a prime ideal in $F[x]$, so in particular $P$ is a prime element of $F[x]$. Then, suppose $P = fg$ for some $f,g \in F[x]$. Since $P$ is a prime element $P$ divides $f$ or $g$. Without loss of generality suppose $P$ divides $f$ and write $f = Pq$ for some $q \in F[x]$. Then, we have that $P = Pqg$, so $P(1-qg) = 0$. But, $P \neq 0$ and $F[x]$ is an integral domain, so $1-qg = 0$. Hence, $1 = qg$, so $g \in F[x]^{\times}$. Therefore, $P$ is an irreducible element in $F[x]/(P)$.
    
    
    [$2 \implies 3$] Suppose $P$ is irreducible in $F[x]$. Let $I \subset F[x]$ be an ideal containing $(P)$. Then, since $F$ is a field, $F[x]$ is a PID so $I = (f)$ for some $f \in F[x]$. It follows that $(P) \subset (f)$, so $P \in (f)$. Thus, $P = fq$ for some $q \in F[x]$. Then either $f$ or $q$ is a unit. If $f$ is a unit then $(f) = F[x]$. On the other hand, if $q$ is a unit then $f = q^{-1}P \in (P)$, so $(f) = (P)$. Therefore, $(P)$ is a maximal ideal in $F[x]$, so $F[x]/(P)$ is a field, as claimed.
    
    
    [$3 \implies 1$] Suppose $F[x]/(P)$ is a field. Then, in particular it is an integral domain.
    
    Thus, all implications hold so the statements are equivalent.
\end{proof*}

\begin{note}{}{}
        The element $x+(P) \in \F[x]/(P) = K$ is a root of $P$ treated with coefficients in $K$. Indeed, we have an injective homomorphism \begin{equation}
                \map{F\xhookrightarrow{\iota} F[x]/(P)}{a \mapsto a+(P)}
        \end{equation}
        Then we can consider $P \in F[x] \hookrightarrow K[x]$. We claim that $P$ as a polynomial in $K[x]$ has a root $\alpha = x+(P) \in K$. In particular, denote elements of $K$ by $\overline{p} \in K$ where $p \in F[x]$. Then, observe that if $P = a_nx^n+...+a_1x+a_0$ in $F[x]$, then $P = \overline{a_n}x^n+...+\overline{a_1}x+\overline{a_0} \in K[x]$. It follows that \begin{align*}
                \ev_{\alpha}(P) &= P(\alpha) \\
                &= \overline{a_n}\alpha^n+...+\overline{a_1}\alpha+\overline{a_0}\\
                &= \overline{a_nx^n + ... a_1x+a_0} \\
                &= \overline{P} = \overline{0} \in K = \F[x]/(P)
        \end{align*}
\end{note}

\begin{eg}{}{}
        $\R[x]/(x^2+1)$ is a field such that \begin{equation}
                (x+(x^2+1))^2 = x^2+(x^2+1) = -1+(x^2+1)
        \end{equation}
        because $x^2+1 \in (x^2+1)$, so letting $\alpha = x+(x^2+1)$ we have that \begin{equation}
                \alpha^2+1 = 0 \in K = \R[x]/(x^2+1)
        \end{equation}
\end{eg}


\begin{cor}{}{}
    Let $P \in F[x]$ be irreducible over $F$, and let $f_1,...,f_n \in F[x]$. If $P\;\vert\;f_1f_2...f_n$, then there is $i$ such that $P\;\vert\;f_i$.
\end{cor}
\begin{proof*}{}{}
    By the previous theorem $F[x]/(P)$ is a field since $P \in F[x]$ is irreducible over $F$. Hence, $(P)$ is a maximal ideal so in particular $(P)$ is a prime ideal. Then if $fg \in (P)$, $f\in (P)$ or $g \in (P)$. We shall proceed by induction on $n$. For $n = 1$ and $n = 2$ the base case holds trivially by definition of $P$. Hence, suppose there exists $k \geq 2$ such that if $n = k$, $f_1f_2...f_k \in (P)$ implies $f_i \in (P)$ for some $i \in \{1,2,...,k\}$. Then, consider $n = k+1$, so $f_1f_2...f_kf_{k+1} \in (P)$. Since $(P)$ is a prime ideal either $f_1f_2...f_k \in (P)$ or $f_{k+1} \in (P)$. If $f_{k+1} \in (P)$ we're done. Hence, suppose $f_1f_2...f_k \in (P)$. But then, by the induction hypothesis there exists $i \in \{1,2,...,k\}$ such that $f_i \in (P)$. Therefore, by mathematical induction we conclude that if $f_1f_2...f_n \in (P)$, then there exists $i \in \{1,2,...,n\}$ such that $f_i \in (P)$ for all $n \geq 1$.
\end{proof*}

\begin{rmk}{}{}
    If $P \in F[x]$, $\deg(P) = n\geq 1$, (not necessarily irreducible) then \begin{equation}
        \map{\prod\limits_{i=1}^nF \xrightarrow{\phi} F[x]/(P)}{(a_0,a_1,...,a_{n-1}) \mapsto a_0+a_1x+...+a_{n-1}x^{n-1}+(P)}
    \end{equation}
    is a group isomorphism for $(F[x]/(P),+)$.
    \begin{proof*}{}{}
        By definition of addition in $F[x]/(P)$ we note that $\phi$ is a group homomorphism. First, let $(a_0,a_1,...,a_{n-1}) \in \ker(\phi)$. Then in particular $a_0+a_1x+...+a_{n-1}x^{n-1} \in (P)$ so there exists $g \in F[x]$ such that $$a_0+a_1x+...+a_{n-1}x^{n-1} = gP$$ But, since $F$ is a field we have that $\deg(gP) = \deg(g) + \deg(P) \geq n$ or $gP = 0$, and $a_0+a_1x+...+a_{n-1}x^{n-1} = 0$ or $\deg(a_0+a_1x+...+a_{n-1}x^{n-1}) = n-1$. Thus, we must have that $a_0+a_1x+...+a_{n-1}x^{n-1} = gP = 0$. Therefore, $a_0 = a_1 =...= a_{n-1} = 0$, so $\ker(\phi)$ is trivial. Hence, $\phi$ is injective. Write $P = b_0 + b_1x + ... + b_nx^n$. Then, for any $c_0+c_1x+...+c_kx^k + (P) \in F[x]/(P)$, for all $m \geq n$ we can replace $x^m$ by $x^{m-n}b_n^{-1}(-b_0-b_1x-...-b_{n-1}x^{n-1})$. Repeat this step until all powers of $x$ are less than or equal to $n-1$, so $c_0+c_1x+...+c_kx^k + (P) = c_0'+c_1'x+...+c_{n-1}'x^{n-1} + (P)$. Then, we have that $\phi(c_0',c_1',...,c_{n-1}') = c_0'+c_1'x+...+c_{n-1}'x^{n-1} + (P)$, so $\phi$ is indeed surjective. Hence, we have that $\phi$ is a group isomorphism.


        [Alternative Surjectivity] Let $Q \in F[x]$. If $\deg(Q) \leq n-1$ then $Q = b_0+b_1x+...+b_{n-1}x^{n-1} + (P) \in F[x]/(P)$ is the image of $(b_0,b_1,...,b_{n-1})$. If $\deg(Q) \geq n$, then $Q = aP+q$ by the division algorithm, with $q = 0$ or $\deg(q) < \deg(P) = n$. So, $Q+(P) = aP+q+(P) = q+(P)$ which is in the image by the first case.
    \end{proof*}
\end{rmk}


\begin{cor}{}{}
    If $F$ is a finite field of order $q$ and $P$ is an irreducible polynomial over $F$ of degree $n$, then $F[x]/(P)$ is a finite field of order $q^n$.
\end{cor}


\begin{eg}{}{}
    $\Z/2\Z[x]/(x^2+x+1)$ is a field of order $2^2 = 4$. Indeed, $x^2+x+1$ has no roots in $\Z/2\Z$, and is consequently irreducible.
\end{eg}


\subsection{Field Extensions}

\begin{rmk}{}{}
        Let $R\xrightarrow{f} S$ be a ring homomorphism for a commutative rings $R,S$ and let \begin{equation}
                P = a_0+a_1x+...+a_nx^n \in R[x]
        \end{equation}
        then for all $\alpha \in R$ \begin{equation}
                f(P(\alpha)) = P'(f(\alpha))
        \end{equation}
        where $P' = f(a_o) + f(a_1)x+....+f(a_n)x^n \in S[x]$. Indeed, \begin{align*}
                f(P(\alpha)) &= f(a_0+a_1\alpha+...+a_n\alpha^n) \\
                &= f(a_0) + f(a_1)f(\alpha)+...+f(a_n)f(\alpha)^n \\
                &= P'(f(\alpha))
        \end{align*}
\end{rmk}

\begin{defn}{}{}
        A ring homomorphism $F \xhookrightarrow{\iota} F'$ for $F, F'$ fields is called a \Emph{field extension} and $F'$ is called an \Emph{extension field} of $F$. Note that $\iota$ must be injective since $F$ is a field and it is assumed to be a ring homomorphism, so one often identifies $F$ with the isomorphic subfield $\iota(F)$ to $F$ in $F'$.
\end{defn}


\begin{thm}{Kronecker's Theorem}{}
        If $F$ is a field and $P \in F[x]$, $\deg(P) > 0$, then there is an extension field of $F$ in which $P$ has a root.
\end{thm}
\begin{proof*}{}{}
        By the Unique Factorization Theorem we have that $P = aP_1...P_n$ for a constant $a \neq 0$ in $F$ and monic irreducible polynomials $P_i$, for all $i$. Then we know that $F' = F[x]/(P_1)$ is a field since $P_1$ is irreducible. Moreover, $P$ has a root ($\alpha = x+(P_1) \in F'$) in $F'$, where we see $P$ as a polynomial with coefficients in $F'$ via the embedding \begin{equation}
                \map{F\xhookrightarrow{\iota}F'}{a\mapsto a+(P_a)}
        \end{equation}
        which is a field extension. Then, because $P_1(\alpha) = 0$ in $F'[x]$ and $P = P_1Q$ for $Q = aP_2...P_n$, we have that $P(\alpha) = 0$ in $F'[x]$. Hence, $\alpha$ is a root of $P$ in the extension field $F'$.
\end{proof*}


\section{GCD of Polynomials}

\begin{defn}{GCD}{}
        We have seen that for $f,g \in F[x]$, $F$ a field, if $f \neq 0$ or $g \neq 0$ and $d$ is the monic generator of \begin{equation}
                (f,g) = \{Pf+Qg\vert P,Q\in F[x]\}
        \end{equation}
        Then \begin{enumerate}
                \item $d$ is monic
                \item $d\;\vert\;f$ and $d\;\vert\;g$
                \item If $P\;\vert\;f$ and $P\;\vert\;g$, then $P\;\vert\;d$ ($\forall P \in F[x]$)
        \end{enumerate}
\end{defn}

\begin{rmk}{}{}
        If $d' \in F[x]$ satisfies 1.-2.-3. above, then $d'$ is the monic generator of $(f,g)$
\end{rmk}
\begin{proof*}{}{}
        Let $d = \gcd(f,g)$ be the monic generator of $(f,g)$. By 1. $(d') \supseteq (f,g) = (d)$ since $d'\;\vert\;f$ and $d'\;\vert\;g$. Thus, $d'\;\vert\;d$. By 2., since $d\;\vert\;f$ and $d\;\vert\;g$, $d\;\vert\;d'$. By the lemma below, $d = d'$
\end{proof*}

\begin{lem}{}{}
        For $F$ a field, $f,g \in F[x]$, and $f,g$ monic, if $f\;\vert\;g$ and $g\;\vert\;f$, then $f = g$.
\end{lem}
\begin{proof*}{}{}
        If $f\;\vert\;g$ and $g\;\vert\;f$ then $g = Qf$ and $f = Pg$ for some $P,Q \in F[x]$. Thus $f = Pg = PQf$, so $(1-PQ)f = 0$, where $f \neq 0$ and $F[x]$ is an integral domain, so $1 = PQ$. Hence, $P,Q \in F[x]^{\times} = F^{\times} = F\backslash\{0\}$. Then $g = af$ for $a \in F\backslash\{0\}$. Since $f$ is monic, the leading coefficient of $g$ is $a$. But, $g$ is monic as well, so $a = 1$ and $f = g$.
\end{proof*}


\begin{claim}{Greatest Common Factor}{}
        Let $f,g \in F[x]$, $\deg(f),\deg(g) \geq 1$. Let $f = aP_1...P_n$, $g = bQ_1...Q_m$ be their unique factorization into a constant times a product of monic irreducible polynomials. Let $h = P_{j_1}P_{j_2}...P_{j_l}$ be the greatest common factor (set $h = 1$ if they don't have a common monic irreducible factor). Then $\gcd(f,g) = h$, as $h$ satisfies 1.-2.-3. from the definition.
\end{claim}
\begin{proof*}{}{}
        By definition $h$ is monic and divides both $f$ and $g$. Now, let $P \in F[x]$ such that $P\;\vert\;f$ and $P\;\vert\;g$. Then there exists $Q,H \in F[x]$ such that $f = PQ$, $g = PH$. If $P$ is constant then $P\;\vert\;h$ automatically. If $\deg(P) \geq 1$ then by the unique factorization theorem $P = cU_1...U_t$ for some constant $c$ and irreducible monic polynomials $U_i$, for all $i \in \{1,...,t\}$. Similarly \textbf{To be continued}
\end{proof*}

\begin{eg}{}{}
        Let $F = \Q$, $f = 10(x-1)(x-2)^2(x-3)^2$, and $g = \frac{1}{11}(x-1)^3(x-2)^2(x-3)$. Then $$\gcd(f,g) = (x-1)(x-2)^2(x-3)$$
\end{eg}




%%%%%%%%%%%%%%%%%%%%%%%%%%%%%%%%%%%%% Part 3
\part{Field Theory}


%%%%%%%%%%%%%%%%%%%%%% - P3.Chapter 1
\chapter{\textsection Basic Definitions and Examples: Fields}



\begin{defn}{}{}
    A \Emph{field} is a commutative division ring.
\end{defn}


\begin{defn}{}{}
    The characteristic of a field $F$, denoted $\ch(F)$, is the smallest $p \in \N = \{1,2,...\}$ such that $p\cdot 1_F = 0_F$ if it exists, and $\ch(F) = 0$, otherwise.
\end{defn}

From here on out we will denote the multiplicative identity by $1$ and the additive identity by $0$ for any field.

\begin{rmk}{}{}
    If $F$ is a field and $\ch(F) = p$ for $p \neq 0$, then $p$ is a prime.
\end{rmk}

\begin{eg}{}{}
    \leavevmode
    \begin{enumerate}
        \item $\R, \Q, \C$, fields of characteristic zero.
        \item $\F_p = \Z/p\Z$ the finite field of $p$ elements.
        \item For $\F_p[x]$ polynomials, we have $\F_p(x)$ the field of rational functions with coefficients from $\F_p$. This is isomorphic to the quotient or fraction field of the integral domain $\F_p[x]$.
    \end{enumerate}
\end{eg}

\begin{rmk}{}{}
    For a field $F$, we have a unique ring homomorphism $\varphi:\Z\rightarrow F$ defined by $\varphi(n) := n\cdot 1 = \underbrace{1+1+...+1}_{\text{$n$ times}}$. Note that $\ker(\varphi) \subseteq \Z$ is an ideal, so $\ker(\varphi) = n\Z$ for $n \in \N\cup \{0\}$. Moreover, $\Z/\ker(\varphi) \cong \varphi(\Z)$, which is a subring of $F$ since $\varphi$ is a ring homomorphism. Moreover, $\varphi(\Z)$ is isomorphic to $\Z$ if $\ker(\varphi) = (0)$, and $\varphi(\Z)$ is isomorphic to $\Z/p\Z$ for $p$ a prime if $\ker(\varphi) = p\Z$. 

    Thus, each field $F$ has a subring which is isomorphic to either $\Z$ or $\Z/p\Z$. By the Field of Fractions technique we have that $F$ has a \Emph{subfield} isomorphic to either $\Q$ or $\F_p$.
\end{rmk}
\begin{proof*}{}{}
    (Left to the reader)
\end{proof*}


\begin{rec*}{}{}
    Reminder that the only ideals in a field $F$ are either $(0)$ or $F$.
\end{rec*}


\begin{defn}{}{}
    The \Emph{prime subfield} of a field $F$ is \Emph{generated} by $1_F$ and is isomorphic to either $\Q$ or $\F_p$ for some prime $p$.
\end{defn}




%%%%%%%%%%%%%%%%%%%%%% - P3.Chapter 2
\chapter{\textsection Field Extensions}

\section{Initial Definitions and Examples}

\begin{defn}{}{}
    If $K$ is a field containing the subfield $F$, then $K$ is said to be an \Emph{extension field} of $F$ denoted $K/F$.
        \begin{center}
            \begin{tikzpicture}[baseline = (a).base]
            \node[scale = 1] (a) at (0,0){
                \begin{tikzcd}
                    K \ar[d, dash] \\
                    F
                \end{tikzcd}
            };
            \end{tikzpicture}
        \end{center}
    We call $F$, the field being extended, the \Emph{base field}.
\end{defn}

\begin{rmk}{}{}
    Suppose $K/F$. Let $x,y \in K$ and $c \in F$ then $c_2 \in F \subseteq K$. Then we have \begin{enumerate}
        \item $c(x+y) = cx+cy$
        \item $c(c_2x) = (cc_2)x$
        \item $(c+c_2)x = cx+c_2x$
        \item $1\cdot x = x$
    \end{enumerate}
    along with the fact that $K$ is an abelian group over $+$ as it is a field, so $K$ is an $F$-vector space.
\end{rmk}


\begin{defn}{}{}
    The degree of $K$ over $F$ is $[K:F] = \dim_F(K)$. If $\dim_F(K)$ is finite then $K$ is said to be a \Emph{finite extension} of $F$. If $\dim_F(K)$ is infinite then $K$ is said to not be a finite extension of $F$ (or an \Emph{infinite extension}).
\end{defn}


\begin{thm}{}{}
    Let $F$ be a field and $p(x) \in F[x]$ an \Emph{irreducible} polynomial. Then there exists a field $K$ containing $F' \cong F$, in which $p(x)$ hasa root; that is $p(\alpha) = 0$ for some $\alpha \in K$, where $p$ is now envisioned as the corresponding polynomial in $F'$.
\end{thm}
\begin{proof*}{}{}
    Let $F$ be a field and $p(x) \in F[x]$ an irreducible polynomial. Then the ideal generated by $p(x)$, $(p(x))$, is maximal in $F[x]$ so $F[x]/(p(x)) = K$ is a field. Consider $p \in K[X]$ such that for $p(x) = a_nx^n+...+a_0$ in $F[x]$, we have $$p = (a_n+(p(x)))X^n+...+(a_0+(p(x)))$$  Moreover, for $\alpha = x + (p(x))$ in $K$, we have that \begin{align*}
        p(\alpha) &= (a_n+(p(x)))\alpha^n+...+(a_0+(p(x))) \\
        &= (a_nx^n+(p(x))) + ... + (a_0+(p(x))) \\
        &= (a_nx^n+...+a_0) + (p(x)) \\
        &= p(x) + (p(x)) \\
        &= 0 + (p(x))
    \end{align*}
    so $\alpha$ is a root of $p$ in $K[x]$. Finally, note that the map $\varphi:F\rightarrow K$ sending $\varphi(a) = a+(p(x))$ is a ring monomorphism since $F$ is a field, and thus restricting the codomain to $\varphi(F) = F'$, we have that $F \cong F'$, a subfield of $K$, as desired.
\end{proof*}


\begin{thm}{}{}
    Let $p(x) \in F[x]$ for $F$ a field, irreducible of degree $n$ over $F$, and $K = F[x]/p(x)$. Let $\theta = x\mod p(x)$. Then $1,\theta,\theta^2,...,\theta^{n-1}$ is a basis for $K$ over $F$. That is, \begin{equation}
        K = \spn(1,\theta,\theta^2,...,\theta^{n-1}) = \{a_0+a_1\theta+...+a_{n-1}\theta^{n-1}:a_0,...,a_{n-1} \in F\}
    \end{equation}
    so $\dim_F(K) = n = [K:F]$.
\end{thm}
\begin{proof*}{}{}
    (By the division algorithm on $F[x]$)
\end{proof*}


\begin{eg}{}{}
    Consider $\R[x]/(x^2+1) \cong \{a+b\theta:a,b \in \R\}$ by the previous Theorem. Moreover, in $\R[x]/(x^2+1)$, for $p = x^2+1$ $p(\theta) = 0$, so $\theta^2 = -1$.
\end{eg}


\begin{defn}{}{}
    If $K$ is an extension field of $F$ containing $\alpha, \beta,...$ Then the smallest field containing both $\alpha,\beta,...$ and $F$ is $F(\alpha,\beta,...)$. When we just adjoin $\alpha$, then $F(\alpha)$ is said to be a \Emph{simple extension} of $F$ with primitive element $\alpha$.
\end{defn}


\begin{thm}{}{}
    Let $F$ be a field and $p(x) \in F[x]$ and irreducble polynomial. Suppose $K$ is an extension field of $F$ containing the root $\alpha$ of $p(x)$. That is $p(\alpha) = 0$ in $K$. Let $F(\alpha)$ denote the subfield of $K$ generated by $F$ and $\alpha$. Then $F(\alpha) \cong F[x]/(p(x))$.
\end{thm}
\begin{proof*}{}{}
    Let $\varphi:F[x]\rightarrow F(\alpha) \subseteq K$, defined by $\varphi(f(x)) = f(\alpha)$. Indeed, if $f(x) = c_0+c_1x+...+c_nx^n$, then $\varphi(f(x)) = c_0+c_1\alpha+...+c_n\alpha^n \in F(\alpha)$. Moreover, $\varphi$ is ring homomorphism, since the evaluation map is a ring homomorphism. Then we have that $$\ker(\varphi) = \{g(x) \in F[x]:g(\alpha) = 0\}$$
    It follows that for all $h(x)p(x) \in (p(x))$, $\varphi(h(x)p(x)) = h(\alpha)p(\alpha) = 0$ in $F(\alpha)$, so $(p(x)) \subseteq \ker(\varphi)$. Since $(p(x)) \subseteq \ker(\varphi)$ we have by the Factor Theorem a ring homomorphism $$\overline{\varphi}:F[x]/(p(x)) \rightarrow F(\alpha)$$
    But, since $p(x)$ is irreducible in $F[x]$, $F[x]/(p(x))$ so $\overline{\varphi}$ must be injective. Then, we have that $F[x]/(p(x))$ is isomorphic to a subfield of $F(\alpha)$ containing $F$ and $\alpha$. But then by definition it contains $F(\alpha)$, so $\varphi$ must be surjective and hence an isomorphism. Thus $F[x]/(p(x)) \cong F(\alpha)$.
\end{proof*}


\begin{eg}{}{}
    Consider $\Q[x]/(x^2-2)$. Then by our previous theorem $\Q[x]/(x^2-2) \cong \Q(\sqrt{2})$, as $(\sqrt{2})^2 - 2 = 0$, so $p(x) = x^2-2$ has $\alpha = \sqrt{2}$ as a root. Note $\beta = -\sqrt{2}$ is also a root, so $\Q(\sqrt{2}) \cong \Q(-\sqrt{2})$.
\end{eg}

\begin{eg}{}{}
    Consider $\Q[x]/(x^3-2)$, so $\Q[x]/(x^3-2) \cong \Q(\sqrt[3]{2})$ has $p(x) = x^3-2$ has $p(\alpha) = 0$ for $\alpha = \sqrt[3]{2}$. Then, for $\omega = e^{2\pi i/3}$, we have roots $\omega\alpha$ and $\omega^2\alpha$ of $p(x)$. Then we have by our previous theorem $\Q(\sqrt[3]{2}) \cong \Q(\omega\sqrt[3]{2}) \cong \Q(\omega^2\sqrt[3]{2})$.
\end{eg}


\begin{rmk}{}{}
    Suppose $\phi:F\xrightarrow{\sim} F'$ is an isomorphism between fields $F$ and $F'$. Then we can extend $\phi$ to the isomorphism \begin{equation}
        \map{\phi':F[x]\xrightarrow{\sim}{F'[x]}}{\phi'(c_0+c_1x+...+c_nx^n) = \phi(c_0)+\phi(c_1)x+...+\phi(c_n)x^n}
    \end{equation}
    If $p(x)$ is irreducible over $F$, then $(p(x))$ is maximal in $F[x]$, so $\phi'((p(x))) = (p'(x))$ is maximal, which implies $p'(x)$ is irreducible in $F'[x]$. Then the theorem follows:
\end{rmk}


\begin{thm}{}{}
    Let $\alpha$ be a root of $p(x)$ and $\beta$ a root of $p'(x) = \phi'(p(x))$ in the extension fields $F(\alpha)$ and $F'(\beta)$. Then there is an isomorphism \begin{equation}
        \sigma:F(\alpha)\xrightarrow{\sim}F'(\beta)
    \end{equation}
    where $\sigma\rvert_F = \phi$ and $\alpha \mapsto \beta$. Then the diagram
        \begin{center}
            \begin{tikzpicture}[baseline = (a).base]
            \node[scale = 1] (a) at (0,0){
                \begin{tikzcd}
                    F(\alpha) \ar[r, "\sigma"] \ar[d,twoheadrightarrow]& F(\beta) \ar[d, twoheadrightarrow] \\
                    F \ar[r, "\phi"] & F'
                \end{tikzcd}
            };
            \end{tikzpicture}
        \end{center}
    commutes.
\end{thm}
\begin{proof*}{}{}
    (To the reader)
\end{proof*}



\section{Algebraic Extensions}

\begin{defn}{}{}
    $\alpha \in K$ over $F$ is \Emph{algebraic} over $F$ if $\alpha$ is a root of some nonzero polynomial $f(x) \in F[x]$. If $\alpha$ is not algebraic over $F$, $\alpha$ is said to be \Emph{transcendental} over $F$/ The extension $K/F$ is \Emph{algebraic} if and only if every element of $K$ is algebraic.
\end{defn}


\begin{prop}{}{}
    Let $\alpha$ be algebraic over $F$. Then there exists a unique monic irreducible polynomial $m_{\alpha,F}(x) \in F[x]$ which has $\alpha$ as a root. Furthermore, $m_{\alpha,F}(x)$ divides any $f(x) \in F[x]$ such taht $f(\alpha) = 0$.
\end{prop}


\begin{defn}{}{}
    $m_{\alpha,F}(x)$ in the previous proposition is the \Emph{minimal polynomial} of $\alpha$ over $F$.
\end{defn}


\begin{eg}{}{}
    $m_{\sqrt{2},\Q} = x^2-2$, but $m_{\sqrt{2},\R} = x-\sqrt{2}$.
\end{eg}

\begin{cor}{}{}
    $\alpha \in F$ if and only if $m_{\alpha,F} = x-\alpha$.
\end{cor}


\begin{cor}{}{}
    For a field extension $L/F$ and $\alpha$ is algebraic over both $F$ and $L$, then $m_{\alpha,L}(x)$ divides $m_{\alpha,F}(x)$ in $L[x]$.
\end{cor}

\begin{prop}{}{}
    Let $\alpha$ be algebraic over $F$, $F(\alpha)$ generated by $\alpha$ and $F$. Then $F(\alpha) \cong F[x]/(m_{\alpha,F}(x))$, and $[F(\alpha):F] = \deg(m_{\alpha,F}(x)) = \deg(\alpha)$.
\end{prop}


\begin{prop}{}{}
    The element $\alpha$ is algebraic over $F$ if and only if the simple extension $F(\alpha)/F$ is a finite extension.
\end{prop}
\begin{proof*}{}{}
    [$\implies$] By the previous proposition.

    [$\impliedby$] Suppose $F(\alpha)/F$ is finite, so $[F(\alpha),F] = n$ for some $n \in \N$. Thus $\{1,\alpha,\alpha^2,...,\alpha^n\}$ must be linearly dependent over $F$. Thus, there exist $c_i \in F$, $0 \leq i \leq n$, not all zero such that \begin{equation*}
        c_0 + c_1\alpha + ... + c_n\alpha^n = 0
    \end{equation*}
    Thus, we have that $f(x) = c_0+c_1x+...+c_nx^n$ is a non-zero polynomial in $F[x]$ which takes $\alpha$ as a root, so $\alpha$ is algebraic over $F$ by definition.
\end{proof*}

\begin{cor}{}{}
    If the extension $K/F$ is finite, then its algebraic.
\end{cor}
\begin{proof*}{}{}
    Let $\alpha \in K$ so $F(\alpha)$ is a subfield of $K$, and $[F(\alpha):F] \leq [K:F] = n$ for some $n \in \N$. Thus, by the previous proposition $\alpha$ is algebraic.
\end{proof*}




%%%%%%%%%%%%%%%%%%%%%% - P3.Chapter 3
\chapter{\textsection Galois Theory}


%%%%%%%%%%%%%%%%%%%%%%%%%%%%%%%%%%%%% Part 4
\part{Modules}

%%%%%%%%%%%%%%%%%%%%%% - P4.Chapter 1
\chapter{\textsection General Definitions and Examples}


\section{Basic Definitions and Examples: Modules}

Fix a (unital) ring $R$ for this section.

\begin{defn}{}{}
    An (left) $R$-module is an abelian group $M$ with an additional structure of a map \begin{equation}
        act_M:R\times M \rightarrow,\;\;\;(r,m)\mapsto r\cdot m
    \end{equation}
    such that the following properties hold:\begin{enumerate}
        \item For every $m \in M$, we have $1 \cdot m = m$.
        \item For every $r_1,r_2 \in R$ and $m \in M$, we have $$r_1\cdot(r_2\cdot m) = (r_1\cdot r_2)\cdot m$$
        \item For every $r_1,r_2 \in R$ and $m \in M$, we have $$(r_1+r_2)\cdot m = r_1\cdot m + r_2\cdot m$$
        For every $r \in R$ and $m_1,m_2 \in M$, we have $$r\cdot (m_1+m_2) = r\cdot m_1 + r\cdot m_2$$
    \end{enumerate}
    Note that the last condition is equivalent to saying that for any fixed $r \in R$, the map $$\map{M\rightarrow M}{m\mapsto r\cdot m}$$
    is a group endomorphism.
\end{defn}

\begin{defn}{(General)}{}
    Let $R$ be a ring (not necessarily commutative nor unital). A \Emph{left $R$-module} or a \Emph{left module over $R$} is a set $M$ together with \begin{enumerate}
        \item A binary operation $+$ on $M$ under which $M$ is an abelian group, and
        \item An action of $R$ on $M$ (that is, a map $R\times M \rightarrow M$) denoted by $r.m$, for all $r \in R$ and for all $m \in M$ which satisfies \begin{enumerate}
                \item $(r+s).m = r.m + s.m$, for all $r,s \in R$ and all $m \in M$
                \item $r.(m+n) = r.m + r.n$, for all $r \in R$ and all $m,n \in M$
                \item $r.(s.m) = (rs).m$, for all $r,s \in R$ and all $m \in M$

                    \noindent If the ring $R$ has a $1$ we impose the additional axiom:
                \item $1.m = m$ for all $m \in M$
        \end{enumerate}
    \end{enumerate}
\end{defn}

Note that when $R$ is a field $k$, our definition induces the definition of a $k$-vector space.

\begin{defn}{}{}
    Let $R$ be a ring and let $M$ be an $R$-module. An \Emph{$R$-submodule} of $M$ is a subgroup $N$ of $M$ which is closed under the action of ring elements, i.e., $r.n \in N$ for all $n \in N$ and all $r \in R$.
\end{defn}

We note that if $R$ is a field, then $R$-submodules correspond to subspaces. Moreover, a submodule of a module $M$ is precisely a subset of $M$ which is itself an $R$-module under the restricted action by ring elements.

Every $R$-module $M$ has the submodules $M$ and $\{0\}$, the second being the \Emph{trivial submodule}.

\begin{lem}{}{}
    For any $r \in R$ we have $r \cdot 0_M = 0_M$. For any $m \in M$ we have $0_R\cdot m = 0_M$ and $(-1)\cdot m = -m$.
\end{lem}


\begin{eg}{}{}
    \leavevmode
    \begin{enumerate}
        \item The $0$ module is an $R$-module.
        \item Take $M =R$, with the structure of an abelian group the same as that on $R$. We define $act_M := mult_R$. The module axioms follow from the ring axioms on $R$. Moreover, it follows that every field can be considered as a $1$-dimensional vector space over itself. Additionally, when $R$ is considered as a left module over itself in this fashion, its submodules are precisely its left ideals. If $R$ is not commutative it has a left and right module structure over itself, and these may be different. 
        \item Take $M = R^{1\times 2} := R\times R$. The abelian group structure is defined component wise, and so is the action of $R$.
        \item Generalizing the previous example we can take $M = R^{1\times n}$ for any positive integer $n$. In particular, for $n \in \Z^+$ we define \begin{equation*}
                R^n \cong R^{1\times n}  = \{[a_1,a_2,...,a_n]: a_i \in R, \forall i\}
        \end{equation*}
            The module $R^n$ is called the \Emph{free module of rank $n$ over $R$}. A clear submodule of $R^n$ is the $i$th component, in which arbitrary ring elements can exist in the $i$th component while zeros are in the $j$th component for all $j \neq i$.
        \item If we replace the ring in the previous example with a field $F$, we obtain the \Emph{affine $n$-space over $F$}, $F^n$.
    \end{enumerate}
\end{eg}

We note that if $M$ is an $R$-module and $S$ is a subring of $R$ with $1_S = 1_R$ (if identity exists), then $M$ is automatically an $S$-module. 

\begin{defn}{}{}
    If $M$ is an $R$-module, and $I$ is a two-sided ideal such that $a.m = 0$ for all $a \in I$ and all $m \in M$, then we say $M$ is \Emph{annihilated} by $I$. In this case we can make $M$ into an $(R/I)$-module by defining an action of the quotient ring $R/I$ on $M$ as follows: for each $m\in M$ and each coset $r+I \in R/I$, let \begin{equation*}
        (r+I).m := r.m
    \end{equation*}
    Since $a.m = 0$ for all $a \in I$ and $m \in M$, this is well-defined. In particular, if $I$ is a maximal ideal and $R$ is a commutative ring, then $M$ is a vector space over the field $R/I$.
\end{defn}


\begin{eg}{(Z-Modules)}{}
    Let $R=\Z$, let $A$ be an abelian group and write the operation of $A$ as $+$. We can make $A$ into a $\Z$-module as follows: for any $n \in \Z$ and $a \in A$ define \begin{equation*}
        n.a := \left\{\begin{array}{lc} \underbrace{a+a+...+a}_{n-times} & \text{if } n > 0 \\ 0 & \text{if } n = 0 \\ \underbrace{(-a)+(-a)+...+(-a)}_{-n-times} & \text{if } n < 0
        \end{array}\right.
    \end{equation*}
    where $0$ is the identity of the additive group $A$. This definition makes $A$ into a $\Z$-module, and by the module actions this is the only definition which makes $A$ into a (unital) $\Z$-module. Thus, every abelian group is a $\Z$-module. 


    Conversely, if $M$ is any $\Z$-module, a fortiori $M$ is an abelian group, so \begin{equation*}
        \Z-\text{modules are the same as abelian groups}
    \end{equation*}
    Furthermore, from the definition it is clear that \begin{equation*}
        \Z-\text{submodules are the same as subgroups}
    \end{equation*}
\end{eg}


\begin{eg}{(F{[x]}-modules)}{}
    Let $F$ be a field, let $x$ be an indeterminate and let $R$ be the polynomial ring $F[x]$. Let $V$ be a vector space over $F$ and let $T$ be a linear transformation from $V$ to $V$. We already know that $V$ is an $F$-module; the linear map $G$ will enable us to make $V$ into an $F[x]$-module.

    First, for the nonnegative integer $n$, define \begin{align*}
        T^0 &:= I, \\
        &\vdots \\
        T^n &:= \underbrace{T\circ T\circ ...\circ T}_{n-times}
    \end{align*}
    where $I$ is the identity map from $V$ to $V$ and $\circ$ denotes function composition. Also, for any two linear transformations $A, B$ from $V$ to $V$ and elements $\alpha,\beta \in F$, let $\alpha A+\beta B$ be defined by \begin{equation*}
        (\alpha A + \beta B)(v) := \alpha(A(v)) = \beta(B(v))
    \end{equation*}
    for all $v \in V$. Note that this is again a linear transformation from $V$ to $V$.


    Now let us define the action of any polynomial in $x$ on $V$. Let $p(x) \in F[x]$, $p(x) = a_nx^n+...a_1x+a_0$, where $a_0,...,a_n \in F$. For each $v \in V$ we define an action fo the ring element $p(x)$ on the module element $v$ by \begin{align*}
        p(x).v &= (a_nt^n+a_{n-1}T^{n-1}+...+a_1T+a_0)(v) \\
        &= a_nT^n(v) + a_{n-1}T^{n-1}(v) + ... + a_1T(v) + a_0v
    \end{align*}
    Put another way, $x$ acts on $V$ as the linear transformation $T$, and we extend this to an action of all of $F[x]$ on $V$, satisfying all the module axioms.

    Note that the action of $F[x]$ on $V$ is consistent with the original action of $F$ on the vector space $V$ when restricted to constant polynomials. This construction in fact describes all $F[x]$-modules.

    Moreover, there is a bijection between the collections of $F[x]$-modules and the collection of pairs $V,T$: \begin{equation*}
        \left\{V\text{ an }F[x]\text{-module}\right\} \leftrightarrow \left\{\begin{array}{c} V\text{ a vector space over } F \\ \text{and} \\ T:V\rightarrow V\text{ a linear transformation}\end{array}\right\}
    \end{equation*}
    given by \begin{equation*}
        \text{the element $x$ acts on $V$ as the linear transformation $T$}
    \end{equation*}

    Next, the $F[x]$-submodules $U$ of $V$ are precisely the $T$-stable (or invariant) subspaces of $V$ as seen with $V$ as a vector space over $F$. We obtain a similar bijection as before: \begin{equation*}
        \left\{W\text{ an }F[x]\text{-submodule}\right\} \leftrightarrow \left\{\begin{array}{c} W\text{ a vector subspace of } V \\ \text{and} \\ W\text{ is $T$-stable}\end{array}\right\}
    \end{equation*}
\end{eg}


\begin{prop}{(The Submodule Criterion)}{}
    Let $R$ be a ring and let $M$ be an $R$-module. A subset $N$ of $M$ is a submodule of $M$ if and only if \begin{enumerate}
        \item $N \neq \emptyset$, and 
        \item $x+r.y \in N$ for all $r \in R$ and for all $x,y \in N$
    \end{enumerate}
\end{prop}
\begin{proof*}{}{}
    If $N$ is a submodule, then $0 \in N$ so $N \neq \emptyset$. Also $N$ is closed under addition and is stable under the action of elements of $R$, so $x+r.y \in N$ for all $r \in R$ and $x,y \in N$.

    Conversely, suppose the two points hold. Let $m \in N$ since $N$ is non-empty. Then $m+(-1).m = (1+(-1)).m = 0.m = 0$, so $0 \in N$. Moreover, for all $m,n \in N$ we have that $m+(-n) = m+(-1).n \in N$ by hypothesis, so $N$ is a subgroup of $M$. Now, we can take $x = 0$ and observe that for all $y \in N$ and all $r \in R$, $r.y = 0+r.y \in R$, so $N$ is stable under the action. Thus $N$ is indeed a submodule of $M$.
\end{proof*}


\begin{defn}{}{}
    Let $R$ be a commutative ring with identity. An \Emph{$R$-algebra} is a ring $A$ with identity together with a ring homomorphism $f:R\rightarrow A$ mapping $1_R$ to $1_A$ such that the subring $f(R)$ of $A$ is contained in the center of $A$.
\end{defn}

Observe that if $A$ is an $R$-algebra, then $A$ has a natural left and right (unital) $R$-module structure defined by $r\cdot a = a\cdot r = f(r)a$ where $f(r)a$ is just the multiplication in the ring $A$. Other $R$-module structures are possible on $A$, but this is the standard one.


\begin{defn}{}{}
    If $A$ and $B$ are two $R$-algebras, an \Emph{$R$-algebra homomorphism (or isomorphism)} is a ring homomorphism (isomorphism, respectively) $\phi:A\rightarrow B$ mapping $1_A$ to $1_B$ such that $\phi(r.a) = r.\phi(a)$ for all $r \in R$ and $a \in A$.
\end{defn}


\begin{eg}{}{}
    Let $R$ be a commutative ring with $1$. \begin{enumerate}
        \item Any ring with identity is a $\Z$-algebra.
        \item For any ring $A$ with identity, if $R$ is a subring of the center of $A$ containing the identity of $A$ then $A$ is an $R$-algebra. 
        \item If $A$ is an $R$-algebra then the $R$-module structure of $A$ depends only on the subring $f(R)$ contained in the center of $A$ as in the previous example. If we identify $R$ by its image $f(R)$ we see that ``up to a ring homomorphism" every algebra $A$ arises from a subring of the center of $A$ that contains $1_A$.
    \end{enumerate}
\end{eg*}

If $A$ is an $R$-algebra, then $A$ is a ring with identity that is a (unital) left $R$-module satisfying $r\cdot (ab) = (r\cdot a)b = a(r\cdot b)$ for all $r \in R$ and $a,b \in A$.



\section{Module Homomorphisms}

\begin{defn}{}{}
    Let $M_1$ and $M_2$ be $R$-modules. An $R$-module homomorphism from $M_1$ to $M_2$ is a map of sets $\phi:M_1\rightarrow M_2$ such that \begin{enumerate}
        \item $\phi$ is a group homomorphism
        \item For every $r \in R$ and $m_1 \in M_1$, we have $\phi(r\cdot m_1) = r\cdot \phi(m_1)$.
    \end{enumerate}
    The last condition can be rewritten in terms of the following commutative diagram:
        \begin{center}
            \begin{tikzpicture}[baseline = (a).base]
            \node[scale = 1] (a) at (0,0){
                \begin{tikzcd}
                    R\times M_1  \ar[d, "act_{M_1}", swap] \ar[r, "\id_R\times \phi"] & R\times M_2 \ar[d,"act_{M_2}"] \\
                    M_1 \ar[r, "\phi"] & M_2
                \end{tikzcd}
            };
            \end{tikzpicture}
        \end{center}
    We denote the set of $R$-module homomorphisms $M_1\rightarrow M_2$ by $\Hom_{\Rmod}(M_1,M_2)$.
\end{defn}

\begin{rmk}{}{}
    Give $R$-modules $M_1,M_2,M_3$, and $R$-module homomorphisms $\phi:M_1\rightarrow M_2$, $\psi:M_2\rightarrow M_3$, the composed map \begin{equation}
        \psi \circ \phi:M_1\rightarrow M_3
    \end{equation}
    is an $R$-module homomorphism. We can regard the operation of composition as a map of sets \begin{equation}
        \Hom_{\Rmod}(M_2,M_3)\times\Hom_{\Rmod}(M_1,M_2) \rightarrow \Hom_{\Rmod}(M_1,M_3)
    \end{equation}
\end{rmk}

\subsection{Evaluation Bijections}

\begin{defn}{}{}
    Let $M$ be an arbitrary $R$-module. Consider the set $\Hom_{\Rmod}(R,M)$, where $R$ is considered as an $R$-module. We define the map of sets \begin{equation}
        \map{\ev:\Hom_{\Rmod}(R,M) \rightarrow M}{\phi\mapsto \phi(1) \in M}
    \end{equation}
    $\ev$ as defined is a bijection of sets. That is, to give a map of modules $R\rightarrow M$ is the same as to give an element of $M$.
\end{defn}

\begin{defn}{}{}
    Generalizing the previous definition we obtain the bijection \begin{equation}
        \map{\ev:\Hom_{\Rmod}(R^{1\times n},M) \rightarrow M^{1\times n}}{\phi\mapsto (\phi(1,0,...,0),\phi(0,1,...,0),...,\phi(0,0,...,1)) \in M^{1\times n}}
    \end{equation}
\end{defn}

\begin{rmk}{}{}
    In particular, taking $M = R^{1\times m}$, we obtain a bijection \begin{equation}
        \Hom_{\Rmod}(R^{1\times n},R^{1\times m}) \overset{\ev}{\cong} (R^{1\times m})^{1\times n} \cong Mat_{m\times n}(R)
    \end{equation}
    In particular, for the composition map \begin{equation}
        \Hom_{\Rmod}(R^{1\times n_2},R^{1\times n_3}) \times \Hom_{\Rmod}(R^{1\times n_1},R^{1\times n_2}) \xrightarrow{comp} \Hom_{\Rmod}(R^{1\times n_1},R^{1\times n_3})
    \end{equation}
    we obtain the commutative diagram 
        \begin{center}
            \begin{tikzpicture}[baseline = (a).base]
            \node[scale = 1] (a) at (0,0){
                \begin{tikzcd}
                    \Hom_{\Rmod}(R^{1\times n_2},R^{1\times n_3})\times \Hom_{\Rmod}(R^{1\times n_1},R^{1\times n_2}) \ar[d, "\ev\times \ev", swap] \ar[r, "comp"] & \Hom_{\Rmod}(R^{1\times n_1},R^{1\times n_3}) \ar[d,"\ev"] \\
                    Mat_{n_3\times n_2}(R)\times Mat_{n_2\times n_1}(R) \ar[r, "mult_{mat}"] & Mat_{n_3\times n_1}(R)
                \end{tikzcd}
            };
            \end{tikzpicture}
        \end{center}
\end{rmk}


\section{Submodules}


%%%%%%%%%%%%%%%%%%%%%% - P4.Chapter 2
\chapter{\textsection Free Modules}



%%%%%%%%%%%%%%%%%%%%%% - P4.Chapter 3
\chapter{\textsection Linear Transformations}


%%%%%%%%%%%%%%%%%%%%%% - P4.Chapter 4
\chapter{\textsection Matrix Theory for Free Modules}



%%%%%%%%%%%%%%%%%%%%%% - P4.Chapter 5
\chapter{\textsection Modules over PIDs}


%%%%%%%%%%%%%%%%%%%%%% - P4.Chapter 6
\chapter{\textsection Tensor Products}

%%%%%%%%%%%%%%%%%%%%%% - Appendix
\begin{appendices}
    \section{Semi-Groups and Monoids}
    
    \begin{defn}{}{}
        A \Emph{semi-group} is a set $A$ equipped with a binary operation \begin{equation}
            A\times A \xrightarrow{mult} A,\;\;\;(a_1,a_2)\mapsto a_1\cdot a_2
        \end{equation}
        which satisfies the associativity axiom:\begin{equation}
            \forall a_1,a_2,a_3 \in A,\;\;\;a_1\cdot (a_2\cdot a_3) = (a_1\cdot a_2)\cdot a_3
        \end{equation}
        We can write this associativity axiom as the following commuting diagram:
        \begin{center}
            \begin{tikzpicture}[baseline = (a).base]
            \node[scale = 1] (a) at (0,0){
                \begin{tikzcd}
                    A\times A \times A \ar[d, "mult \times \id_A", swap] \ar[r, "\id_A\times mult"] & A\times A \ar[d,"mult"] \\
                    A\times A \ar[r, "mult"] & A
                \end{tikzcd}
            };
            \end{tikzpicture}
        \end{center}
    \end{defn}
    
    \begin{defn}{}{}
        A semi-group is said to be a \Emph{monoid} if there exists an element $1 \in A$ that satisfies \begin{equation}
            \forall a \in A,\;\;\;1\cdot a = a = a \cdot 1
        \end{equation}
        An element $1 \in A$ is called the \Emph{unity} or \Emph{identity} in $A$.
    \end{defn}
    
    \begin{lem}{}{}
        A monoid contains a unique identity element.
        \begin{proof*}{}{}
            (Left to the reader)
        \end{proof*}
    \end{lem}
    
    \begin{defn}{}{}
        An inverse of $a \in A$, a monoid, is an element $a^{-1} \in A$ such that \begin{equation}
            a\cdot a^{-1} = 1 = a^{-1} \cdot a
        \end{equation}
    \end{defn}
    
    \begin{lem}{}{}
        If $a \in A$ admits an inverse, then this inverse is unique.
        \begin{proof*}{}{}
            (Left to the reader)
        \end{proof*}
    \end{lem}
    
    \begin{defn}{}{}
        A monoid is said to be a \Emph{group} if every element admits an inverse.
    \end{defn}
    
    \begin{eg}{}{}
        \leavevmode
        \begin{enumerate}
            \item $(\Z,+)$ is a group
            \item $(\Z,\cdot)$ is a monoid but not a group
            \item $(\R,+)$ is a group
            \item $(\R,\cdot)$ is a monoid but not a group ($0$ doesn't have an inverse)
            \item $(\R-\{0\},\cdot)$ is a group
            \item $\{\pm 1\} \subset \R$ with the operation $\cdot$ is a group.
            \item $(\C-\{0\},\cdot)$ is a group
            \item $(\{z\in \C-\{0\}:|z| = 1\},\cdot$ is a group, often denoted $S^1$ (the \Emph{circle group})
        \end{enumerate}
    \end{eg}
    
    \begin{defn}{}{}
        A semi-group/monoid/group $A$ is said to be \Emph{commutative} if \begin{equation}
            \forall a_1,a_2 \in A,\;\;\;a_1\cdot a_2 = a_2\cdot a_1
        \end{equation}
        We call such a structure \Emph{abelian}. We may rewrite the commutativity condition as the commutative diagram:
        \begin{center}
            \begin{tikzpicture}[baseline = (a).base]
            \node[scale = 1] (a) at (0,0){
                \begin{tikzcd}
                    A\times A  \ar[d, "swap_A", swap] \ar[r, "mult"] & A \ar[d,"\id_A"] \\
                    A\times A \ar[r, "mult"] & A
                \end{tikzcd}
            };
            \end{tikzpicture}
        \end{center}
        where for any set $X$, $\map{swap_X:X\times X\rightarrow X\times X}{(x_1,x_2)\mapsto (x_2,x_1)}$
    \end{defn}
\end{appendices}


\end{document}


%%%%%% END %%%%%%%%%%%%%
