%%%%%%%%%%%%%%%%%%%%% chapter.tex %%%%%%%%%%%%%%%%%%%%%%%%%%%%%%%%%
%
% sample chapter
%
% Use this file as a template for your own input.
%
%%%%%%%%%%%%%%%%%%%%%%%% Springer-Verlag %%%%%%%%%%%%%%%%%%%%%%%%%%
%\motto{Use the template \emph{chapter.tex} to style the various elements of your chapter content.}
\chapter{General Concepts}
\label{GenCon} % Always give a unique label
% use \chaptermark{}
% to alter or adjust the chapter heading in the running head


\section{ Introduction to Modal Logic}

Modal logic (ML) is the logic of \Emph{modalities}, ways in which a statement can be true.

\begin{example}
    \Emph{Necessity} and \Emph{possibility} are two such modalities: a statement can be true, but it can also be necessarily true (true no matter how the world might have been). A possible statement may not actually be true, but it might have been true. We use $\square$ to express necessity, and $\diamond$ to express possibility. So $\square\metav{A}$ can be read as ``\emph{it is necessarily the case that $\metav{A}$}," and $\diamond\metav{A}$ as ``\emph{it is possibly the case that $\metav{A}$}."
\end{example}

There are also many kinds of necessity. ML is able to deal with any of these kinds of necessary truth. We start with a basic set of rules that govern $\square$ and $\diamond$, and then we add more rules as needed to fit the kind of modality we are interested in. 


We actually do not nead to read $\square$ and $\diamond$ as necessity and possibility. All we need is choose the right rules for different readings of $\square$ and $\diamond$.

\begin{definition}
    A \Emph{modal formula} is one that includes modal operators such as $\square$ and $\diamond$. Depending on the interpretation we assign to $\square$ and $\diamond$, different modal formulas will be provable or valid.
\end{definition}


\section{ The Language of Modal Logic}

The language of ML is an extension of the language of TFL. If we build off of FOL, we would have \Emph{Quantified Modal Logic (QML)}. 

\begin{definition}
    ML has an infinite stock of \Emph{atoms} (just like in TFL). These are written as capital letters, with or without numerical subscripts: $A,B,...,A_1,B_1,...$.
\end{definition}

\begin{definition}
    For the rules of ML we take all of the rules of TFL, and add two more for $\square$ and $\diamond$: \begin{enumerate}
        \item Every atom of ML is a sentence of ML
        \item If $\metav{A}$ is a sentence of ML, then $\enot \metav{A}$ is a sentence of ML
        \item If $\metav{A}$ and $\metav{B}$ are sentences of ML, then $(\metav{A}\eand \metav{B})$ is a sentence of modal logic
        \item If $\metav{A}$ and $\metav{B}$ are sentences of ML, then $(\metav{A}\eor \metav{B})$ is a sentence of ML
        \item If $\metav{A}$ and $\metav{B}$ are sentences of ML, then $(\metav{A}\eif \metav{B})$ is a sentence of ML
        \item If $\metav{A}$ and $\metav{B}$ are sentences of ML, then $(\metav{A}\eiff \metav{B})$ is a sentence of ML
        \item If $\metav{A}$ is a sentence of ML, then $\square\metav{A}$ is a sentence of ML
        \item If $\metav{A}$ is a sentence of ML, then $\diamond\metav{A}$ is a sentence of ML
        \item Nothing else is a sentence of ML
    \end{enumerate}
\end{definition}

\begin{example}
    \begin{align*}
        &A, (P\eor Q), \square A, (C\eor \square D),\\
        &\square\square(A\eif R), \square\diamond(S\eand(Z\eiff (\square W\eor \diamond Q)))
    \end{align*}
\end{example}


\section{ Natural Deduction for ML}

As before, we will write $\metav{A}_1,...,\metav{A}_n \proves \metav{C}$ to express that $\metav{C}$ can be proven from $\metav{A}_1,...,\metav{A}_n$. However, since we will be looking at many different systems of ML, we will add a subscript to `$\proves$' to indicate which system we are working with.

\subsection{ System \textbf{K}}

First, the system \textbf{K} includes all of the natural deduction rules of TFL, including the derived rules as well as the basic ones. We then add two additional rules for $\square$ and a special kind of subproof:

\begin{definition}
    A \Emph{strict subproof} is one of the form:
    \begin{fitchproof}
        \open
            \hypo[m]{a1}{\square} 
            \have[n]{a2}{\metav{A}}
        \close
        \have[\ ]{a3}{\square\metav{A}} \by{$\square$I}{a1-a2}
    \end{fitchproof}
    No line above $m$ may be cited by any rule within the strict subproof begun at line $m$, unless the rule explicitly allows it.
\end{definition}

These allow us to reason and prove things about alternate possibilities. What we can prove inside a strict subproof holds in any alternate possibility, in particular, in alternate possibilities where the assumptions in force in our proof may not hold. 

Due to this formulation, the idea is that if $\metav{A}$ is a theorem, then $\square\metav{A}$ should be a theorem too.

\begin{definition}
    The $\square$ elimination rule is given by: 
    \begin{fitchproof}
        \have[m]{a1}{\square\metav{A}}
        \open
            \hypo[\ ]{a2}{\square}
            \have[n]{a3}{\metav{A}} \by{$\square$E}{a1}
        \close
    \end{fitchproof}
    $\square E$ can only be applied if the line $m$ (containing $\square \metav{A}$) lies \emph{outside} of the strict subproof in which line $n$ falls, and this strict subproof is not itself part of a strict subproof not containing $m$ (i.e. you can't apply it in nested strict subproofs in which $\square\metav{A}$ is not in the one right before where we apply $\square E$)
\end{definition}

\begin{example}
    The following is known as the distribution rule: 
    \begin{fitchproof}
        \hypo{a1}{\square(A\eif B)} 
        \open
            \hypo{a2}{\square A}
            \open
                \hypo{a3}{\square}
                \have{a4}{A} \by{$\square$E}{a2}
                \have{a5}{A\eif B} \by{$\square$E}{a1}
                \have{a6}{B} \ce{a4,a5}
            \close
            \have{a7}{\square B}
        \close
        \have{a8}{\square A\eif \square B} \ci{a2-a7}
    \end{fitchproof}
\end{example}


\subsection{ Possibility}

\begin{definition}
    We can \emph{define} possibility in terms of necessity: \begin{equation*}
        \diamond \metav{A} =_{df} \enot\square\enot\metav{A}
    \end{equation*}
    In other words, to say that $\metav{A}$ is \emph{possibly true}, is to say that $\metav{A}$ is \emph{not necessarily false}.
\end{definition}


\begin{definition}
    TO introduce $\diamond$ into the system \textbf{K}, we introduce the following definitional rules: 
    \begin{fitchproof}
        \have[m]{a1}{\enot\square\enot\metav{A}}
        \have[\ ]{a2}{\diamond \metav{A}} \by{Def$\diamond$}{a1}
    \end{fitchproof}
    \begin{fitchproof}
        \have[m]{a1}{\diamond \metav{A}}
        \have[\ ]{a2}{\enot\square\enot\metav{A}} \by{Def$\diamond$}{a1}
    \end{fitchproof}
\end{definition}

We could leave our rules for \textbf{K} here, but for convenience we shall add some \emph{Modal Conversion} rules: 

\begin{definition}
    \begin{fitchproof}
        \have[m]{a1}{\enot\square\metav{A}}
        \have[\ ]{a2}{\diamond \enot\metav{A}} \by{MC}{a1}
    \end{fitchproof}
    \begin{fitchproof}
        \have[m]{a1}{\diamond \enot\metav{A}}
        \have[\ ]{a2}{\enot\square\metav{A}} \by{MC}{a1}
    \end{fitchproof}
    \begin{fitchproof}
        \have[m]{a1}{\enot\diamond\metav{A}}
        \have[\ ]{a2}{\square \enot\metav{A}} \by{MC}{a1}
    \end{fitchproof}
    \begin{fitchproof}
        \have[m]{a1}{\square \enot\metav{A}}
        \have[\ ]{a2}{\enot\diamond\metav{A}} \by{MC}{a1}
    \end{fitchproof}
\end{definition}

It can be proven that $\diamond A \eiff \enot\square\enot A$. So in particular, we could have started with $\diamond$ and defined $\square$ as $\square \metav{A} =_{df} \enot\diamond\enot \metav{A}$.

In other words, necessity and possibility are exactly as fundamental as each other.


\section{ System \textbf{T}}

The system \textbf{T} is obtained by adding the following rule to \textbf{K}:

\begin{definition}
    \begin{fitchproof}
        \have[m]{a1}{\square\metav{A}}
        \have[n]{a2}{\metav{A}} \by{R\textbf{T}}{a1}
    \end{fitchproof}
    The lien $n$ on which the rule R\textbf{T} is applied must \Emph{not} lie in a strict subproof that begins after line $m$.
\end{definition}

Adding this rule allows us to prove things like $\square A \eif A$ in \textbf{T}, which was previously impossible to prove in \textbf{K}.


\section{System \textbf{S4}}