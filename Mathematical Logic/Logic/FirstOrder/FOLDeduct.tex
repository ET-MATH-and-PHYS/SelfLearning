%%%%%%%%%%%%%%%%%%%%% chapter.tex %%%%%%%%%%%%%%%%%%%%%%%%%%%%%%%%%
%
% sample chapter
%
% Use this file as a template for your own input.
%
%%%%%%%%%%%%%%%%%%%%%%%% Springer-Verlag %%%%%%%%%%%%%%%%%%%%%%%%%%
%\motto{Use the template \emph{chapter.tex} to style the various elements of your chapter content.}
\chapter{Natural Deduction in FOL}
\label{NatDeductFOL} % Always give a unique label
% use \chaptermark{}
% to alter or adjust the chapter heading in the running head

\section{ Basic Rules of FOL}

\begin{definition}
    The universal elimination rule ($\forall E$) is given by:
    \begin{fitchproof}
        \have[m]{a1}{\forall x \metav{A}(...x...x...)} 
        \have[\ ]{a2}{\metav{A}(...\metav{c}...\metav{c}...)} \Ae{a1}
    \end{fitchproof}
\end{definition}

The point of this rule of inference is that you can obtain any \emph{substitution instance} of a universally quantified formula: replace \emph{every} instance of the quantified variable with any name you like.

Recall, that as with any rule of inference, it can only be applied if the universal quantifier is the main logical operator.

\begin{definition}
    Where $\metav{A}$ is a sentence containing the name $\metav{c}$, we can emphasize this by writing `$\metav{A}(...\metav{c}...\metav{c})$'. We write `$\metav{A}(...x...\metav{c}...)$' to indicate any formula obtained by replacing \emph{some or all} of the instances of the name $\metav{c}$ with the variable $x$.
\end{definition}

\begin{definition}
    Using the previously defined notation, we write the existential introduction rule as: 
    \begin{fitchproof}
        \have[m]{a1}{\metav{A}(...\metav{c}...\metav{c}...)} 
        \have[\ ]{a2}{\exists x\metav{A}(...x...\metav{c}...)} \Ei{a1}
    \end{fitchproof}
    where $x$ must \emph{not} occur in $\metav{A}(...\metav{c}...\metav{c}...)$.
\end{definition}

\begin{remark}
    It is clear from these rules of inference that if we want them to always apply and be valid, we must accept that as a matter of logic alone, there exists something rather than nothing. However, this is completely reasonable as we have already stipulated that domains be non-empty.

    In order to consider notions involving empty domains, we would need a more complicated proof system than FOL gives us.
\end{remark}

\begin{definition}
    Note that if we know a statement holds of an \emph{arbitrary} name, we can conclude it holds of everything. This motivates the rule of inference for universal introduction: 
    \begin{fitchproof}
        \have[m]{a1}{\metav{A}(...\metav{c}...\metav{c}...)}
        \have[\ ]{a2}{\forall x\metav{A}(...x...x...)} \Ai{a1}
    \end{fitchproof}
    where $\metav{c}$ must not occur in any undischarged assumption, and $x$ must not occur in $\metav{A}(...\metav{c}...\metav{c}...)$. Note also that we are replacing \emph{all} instances of $\metav{c}$ in $\metav{A}(...\metav{c}...\metav{c}...)$ with $x$ in this rule.
\end{definition}

The name in the above rule of inference may occur in \emph{discharged} assumptions, even though it is not allowed to occur in any \emph{undischarged} assumptions.



\begin{definition}
    Noting that if we know something satisfies $F$, and every $F$ is $G$, we can conclude something satisfies $G$. This reasoning inspires our rule of inference for existential elimination: 
    \begin{fitchproof}
        \have[m]{a1}{\exists x\metav{A}(...x...x...)}
        \open
            \hypo[i]{a2}{\metav{A}(...\metav{c}...\metav{c}...)}
            \have[j]{a3}{\metav{B}}
        \close
        \have[\ ]{a4}{\metav{B}} \Ee{a1,a2-a3}
    \end{fitchproof}
    where $\metav{c}$ must not occur in any assumption undischarged before line $i$, $\metav{c}$ must not occur in $\exists x\metav{A}(...x...x...)$, and $\metav{c}$ must not occur in $\metav{B}$.
\end{definition}

If we want to squeeze information out of an existential quantifier, it is advisable choose a new name for our substitution instance (in general), to operate in the utmost generality.


\begin{definition}
    The identity introduction rule is given by: \begin{fitchproof}
        \have[m]{a1}{\metav{c}=\metav{c}} \by{=I}{}
    \end{fitchproof}
    Notice that this rule does not require referring to any prior lines.
\end{definition}

\begin{definition}
    The identity elimination rule is given by:\begin{fitchproof}
        \have[m]{a1}{\metav{a}=\metav{b}}
        \have[n]{a2}{\metav{A}(...\metav{a}...\metav{a}...)}
        \have[\ ]{a3}{\metav{A}(...\metav{b}...\metav{a}...)} \by{=E}{a1, a2}
    \end{fitchproof}
    Note the notation implies we can replace one or more `$\metav{a}$''s in the sentencewith `$\metav{b}$''s. Symmetrically we have:\begin{fitchproof}
        \have[m]{a1}{\metav{a}=\metav{b}}
        \have[n]{a2}{\metav{A}(...\metav{b}...\metav{b}...)}
        \have[\ ]{a3}{\metav{A}(...\metav{a}...\metav{b}...)} \by{=E}{a1, a2}
    \end{fitchproof}
    which is sometimes called \Emph{Leibniz's Law}.
\end{definition}


\section{ Derived Rules in FOL}

\begin{definition}
    We can pass negations through quantified sentences as follows:
    \begin{fitchproof}
        \have[m]{a1}{\forall x\enot \metav{A}}
        \have[\ ]{a2}{\enot \exists x \metav{A}} \by{CQ}{a1}
    \end{fitchproof}
    and \begin{fitchproof}
        \have[m]{a1}{\enot\exists x \metav{A}}
        \have[\ ]{a2}{\forall x\enot \metav{A}} \by{CQ}{a1}
    \end{fitchproof}
    as well as \begin{fitchproof}
        \have[m]{a1}{\exists x\enot \metav{A}}
        \have[\ ]{a2}{\enot \forall x \metav{A}} \by{CQ}{a1}
    \end{fitchproof}
    and \begin{fitchproof}
        \have[m]{a1}{\enot\forall x \metav{A}}
        \have[\ ]{a2}{\exists x\enot \metav{A}} \by{CQ}{a1}
    \end{fitchproof}
\end{definition}



\section{ Proof Theory Semantics in FOL}

\begin{definition}
    Given sentences $\metav{A}_1,\metav{A}_2,...,\metav{A}_n,\metav{C}$ of FOL, we write \begin{equation*}
        \metav{A}_1,\metav{A}_2,...,\metav{A}_n\proves\metav{C}
    \end{equation*}
    to mean that there \emph{exists} some proof which ends with $\metav{C}$ and whose only undischarged assumptions are among $\metav{A}_1,\metav{A}_2,...,\metav{A}_n$. This is a \Emph{proof-theoretic notion}.
\end{definition}

\begin{definition}
    Conversely to the last definition, the notation \begin{equation*}
        \metav{A}_1,\metav{A}_2,...,\metav{A}_n\entails \metav{C}
    \end{equation*}
    means that no valuation (or interpretation) makes all of $\metav{A}_1,\metav{A}_2,...,\metav{A}_n$ true and $\metav{C}$ false. This is a \Emph{semantic notion} - it concerns the assignments of truth and falsity to sentences.
\end{definition}

\Emph{THESE ARE DIFFERENT NOTIONS}.

\begin{theorem}
    In FOL we have the following result: \begin{equation*}
        \metav{A}_1,\metav{A}_2,...,\metav{A}_n\proves\metav{B}\;\textbf{iff}\;\metav{A}_1,\metav{A}_2,...,\metav{A}_n\entails\metav{B}
    \end{equation*}
    This shows that, whilst provability and entailment are \emph{very different} notions, they are \emph{extensionally} equivalent. As such: \begin{enumerate}
        \item An argument is \emph{valid} iff the conclusion can be proved from the premises (i.e. it is a theorem)
        \item Two sentences are \emph{logically equivalent} iff they are \emph{provably equivalent}
        \item Sentences are \emph{satisfiable} iff they are \emph{not provably inconsistent}
    \end{enumerate}
\end{theorem}


\begin{table}[H]
\tabulinesep=1ex
\begin{tabu}{X[.5,c,m] ||X[1,l,m] |X[1,l,m]}
\textbf{Question} 		&	\textbf{Yes} 	&	\textbf{No} \\ \hline \hline

    Is $\metav{A}$ a \Emph{validity}?  &	give a proof which shows $\proves \metav{A}$ & give an interpretation in which $\metav{A}$ is false	 \\ \hline
 
Is $\metav{A}$ a \Emph{contradiction}?  &	give a proof which shows $\proves\enot \metav{A}$ & give an interpretation in which $\metav{A}$ is true	 \\ \hline

    Are $\metav{A}$ and $\metav{B}$ \Emph{equivalent}? &	give two proofs; one for $\metav{A}\proves \metav{B}$ and one for $\metav{B}\proves\metav{A}$ & give an interpretation in which $\metav{A}$ and $\metav{B}$ have different truth values	 \\ \hline


    Are $\metav{A}_1,\metav{A}_2,...,\metav{A}_n$ \Emph{jointly satisfiable}? &	give an interpretation in which all of $\metav{A}_1,\metav{A}_2,...,\metav{A}_n$ are true & prove a contradiction from the assumptions $\metav{A}_1,\metav{A}_2,...,\metav{A}_n$ (that is give a proof for $\metav{A}_1,\metav{A}_2,...,\metav{A}_n\proves \ered$)	 \\ \hline
    Is $\metav{A}_1,\metav{A}_2,...,\metav{A}_n\therefore\metav{C}$ \Emph{valid}? &	give a proof with assumptions $\metav{A}_1,\metav{A}_2,...,\metav{A}_n$ and concluding with $\metav{C}$ & give an interpretation in which all of $\metav{A}_1,\metav{A}_2,...,\metav{A}_n$ are true and $\metav{C}$ is false \\ \hline
\end{tabu}
\caption{Methods of showing logical concepts}
\label{table:proofs_or_interpretations}
\end{table}


