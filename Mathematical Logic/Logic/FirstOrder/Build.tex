%%%%%%%%%%%%%%%%%%%%% chapter.tex %%%%%%%%%%%%%%%%%%%%%%%%%%%%%%%%%
%
% sample chapter
%
% Use this file as a template for your own input.
%
%%%%%%%%%%%%%%%%%%%%%%%% Springer-Verlag %%%%%%%%%%%%%%%%%%%%%%%%%%
%\motto{Use the template \emph{chapter.tex} to style the various elements of your chapter content.}
\chapter{Building Blocks of FOL}
\label{Block} % Always give a unique label
% use \chaptermark{}
% to alter or adjust the chapter heading in the running head

\section{ Names}

\begin{definition}
    In English, a \Emph{singular term} is a word or phrase that refers to a \emph{specific} person, place, or thing.
\end{definition}

\begin{definition}
    In FOL, our \Emph{names} are lower-case letters `$a$' through to `$r$', possibly with the addition of subscripts. Each name must pick out \Emph{exactly} on thing (like a function).
\end{definition}

\section{ Predicates}

\begin{definition}
    In FOL, \Emph{predicates} are captical letters $A$ through $Z$, with or without subscripts. They can be thought of as representing things which combine with singular terms to make sentences.
\end{definition}


\section{ Quantifiers}

\begin{definition}
    In FOL, the symbol `$\forall$' is called the \Emph{universal quantifier}.
\end{definition}

\begin{remark}
    A quantifier must always be followed by a \Emph{variable}. In FOL, variables are italic lowercase letters `$s$' through `$z$', with or without subscripts.
\end{remark}


\begin{definition}
    In FOL, the symbol `$\exists$' is called the \Emph{existential quantifier}.
\end{definition}

\begin{remark}
    In general, $\forall x\enot \metav{A}$ is logically equivalent to $\enot \exists x\metav{A}$, and $\enot\forall x \metav{A}$ is logically equivalent to $\exists x \enot \metav{A}$.
\end{remark}


\section{ Domains}

\begin{definition}
    In FOL, the \Emph{domain} is the collection of things that we are talking about. The quantifiers in an argument \emph{range over} its domain. A domain must have \emph{at least} one member. Every name must pick out \emph{exactly} one member of the domain, but a member of the domain may be picked out by one name, many names, or none at all.
\end{definition}

\begin{definition}
    A predicate that applies to nothing in the domain is called an \Emph{empty predicate}.
\end{definition}

\begin{remark}
    When $\metav{F}$ is an empty predicate, any sentence $\forall x(\metav{F}\eif...)$ is vacuously true.
\end{remark}

\begin{definition}
    A $k$-place predicate is a predicate $P(x_1,...,x_k)$ which can take in $k$ sentence letters.
\end{definition}

\section{ Identity}

\begin{definition}
    The symbol `$=$' is a two-place predicate of meaning: \begin{equation*}
        x=y: \text{\gap{x} is identical to \gap{y}}
    \end{equation*}
\end{definition}

