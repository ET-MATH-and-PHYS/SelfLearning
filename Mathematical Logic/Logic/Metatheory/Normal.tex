%%%%%%%%%%%%%%%%%%%%% chapter.tex %%%%%%%%%%%%%%%%%%%%%%%%%%%%%%%%%
%
% sample chapter
%
% Use this file as a template for your own input.
%
%%%%%%%%%%%%%%%%%%%%%%%% Springer-Verlag %%%%%%%%%%%%%%%%%%%%%%%%%%
%\motto{Use the template \emph{chapter.tex} to style the various elements of your chapter content.}
\chapter{Normal Forms}
\label{NorForms} % Always give a unique label
% use \chaptermark{}
% to alter or adjust the chapter heading in the running head

\section{ Disjunctive Normal Form}

\begin{definition}
    We will say that a sentence is in \Emph{disjunctive normal form} if and only if it meets all of the following requirements: \begin{itemize}[leftmargin=+1in]
        \item[(DNF1)] No connectives occur in the sentence other than negations, conjunctions, and disjunctions;
        \item[(DNF2)] Every occurence of negation has minimal scope (i.e. any `$\enot$' is immediately followed by an atomic sentence);
        \item[(DNF3)] No disjunction occurs within the scope of any conjunction.
    \end{itemize}
\end{definition}

\begin{example}
    \leavevmode
    \begin{enumerate}
        \item $A$
        \item $(A\eand (\enot B\eand C))$
        \item $(A\eand B)\eor (A\eand \enot B)$
        \item $(A\eand B)\eor(A\eand (B\eand(C\eand(\enot D\eand\enot E))))$
        \item $A\eor (C\eand(\enot P_{234}\eand (P_{233}\eand Q)))\eor \enot B$
    \end{enumerate}
\end{example}

\begin{note}
    We write `$\pm\metav{A}$' to indicate that $\metav{A}$ is an atomic sentence which may or may not be prefaced with an occurence of negation. 
\end{note}

\begin{remark}
    From the previous notation, a sentence in disjunctive normal form has the following shape: \begin{equation*}
        (\pm\metav{A}_1\eand ...\eand\pm\metav{A}_i)\eor(\pm\metav{A}_{i+1}\eand ...\eand\pm\metav{A}_j)\eor...\eor(\pm\metav{A}_{m+1}\eand ...\eand\pm\metav{A}_n)
    \end{equation*}
\end{remark}

\begin{theorem}[Disjunctive Normal Form Theorem]
    For any sentence, there is a logically equivalent sentence in disjunctive normal form.
\end{theorem}
\begin{proof}[Truth Tables Proof]
    Pick any arbitrary sentence, $\metav{S}$, and let $\metav{A}_1,...,\metav{A}_n$ be the atomic sentences that occur in $\metav{S}$. To obtain a sentence in DNF that is logically equivalent to $\metav{S}$, we consider $\metav{S}$'s truth table. There are two cases to consider: \begin{enumerate}
        \item $\metav{S}$ is false on every line of its truth table. Then, $\metav{S}$ is a contradiction. In that case, the contradiction $(\metav{A}_1\eand\enot\metav{A}_1)$ is in DNF and logically equivalent to $\metav{S}$.
        \item $\metav{S}$ is true on at least one line of its truth table. For each line $i$ of the truth table, let $\metav{B}_i$ be a conjunction of the form \begin{equation*}
                (\pm\metav{A}_1\eand...\eand\pm\metav{A}_n)
        \end{equation*}
            where the following rules determine whether or not to include a negation in front of each atomic sentence: \begin{enumerate}
                \item $\metav{A}_m$ is a conjunct of $\metav{B}_i$ if and only if $\metav{A}_m$ is true on line $i$.
                \item $\enot \metav{A}_m$ is a conjunct of $\metav{B}_i$ if and only if $\metav{A}_m$ is false on line $i$.
            \end{enumerate}
            Given these rules, $\metav{B}_i$ is true on and only on line $i$ of the truth table which considers all possible valuations of $\metav{A}_1,...,\metav{A}_n$ (i.e. $\metav{S}'$s truth table).

            Next, let $i_1,...,i_m$ be the numbers of the lines of the truth table where $\metav{S}$ is true. Now let $\metav{D}$ be the sentence: \begin{equation*}
                \metav{B}_{i_1}\eor\metav{B}_{i_2}\eor ...\eor \metav{B}_{i_m}
            \end{equation*}
            Since $\metav{S}$ is true on at least one line of its truth table, $\metav{D}$ is indeed well-defined. 


            By construction, $\metav{D}$ is in DNF. Moreover, by construction, for each line $i$ of the truth table: $\metav{S}$ is true on line $i$ of the truth table if and only if one of $\metav{D}$'s disjuncts (namely, $\metav{B}_i$) is true on, and only on, line $i$. Hence $\metav{S}$ and $\metav{D}$ have the same truth table, and so are logically equivalent.
    \end{enumerate}
\end{proof}

\section{ Conjunctive Normal Form}

\begin{definition}
    A sentence is in \Emph{conjunctive normal form} if and only if it meets all of the following requirements: \begin{itemize}[leftmargin=+1in]
        \item[(CNF1)] No connectives occur in the sentence other than negations, conjunctions and disjunctions;
        \item[(CNF2)] Every occurence of negation has minimal scope;
        \item[(CNF3)] No conjunction occurs within the scope of any disjunction.
    \end{itemize}
\end{definition}

\begin{remark}
    Generally, a sentence in CNF is of the shape \begin{equation*}
        (\pm\metav{A}_1\eor ...\eor\pm\metav{A}_i)\eand(\pm\metav{A}_{i+1}\eor ...\eor\pm\metav{A}_j)\eand...\eand(\pm\metav{A}_{m+1}\eor ...\eor\pm\metav{A}_n)
    \end{equation*}
    where each $\metav{A}_k$ is an atomic sentence.
\end{remark}

\begin{theorem}[Conjunctive Normal Form Theorem]
    For any sentence, there is a logically equivalent sentence in conjunctive normal form.
\end{theorem}
\begin{proof}[Truth Table proof]
    Given a TFL sentence, $\metav{S}$, we first write down the complete truth table for $\metav{S}$. If $\metav{S}$ is true on every line of the truth table, then $\metav{S}$ and $(\metav{A}_1\eor\metav{A}_1)$ are logically equivalent.

    If $\metav{S}$ is false on at least one line of the truth table then, for every line on the truth table where $\metav{S}$ is false, write down a disjunction $(\pm\metav{A}_1\eor...\eor\pm\metav{A}_n)$ which is also false on (and only on) that line. Let $\metav{C}$ be the conjunction of all of these disjuncts; by construction, $\metav{C}$ is in CNF and $\metav{S}$ and $\metav{C}$ are logically equivalent.
\end{proof}
