%%%%%%%%%%%%%%%%%%%%% chapter.tex %%%%%%%%%%%%%%%%%%%%%%%%%%%%%%%%%
%
% sample chapter
%
% Use this file as a template for your own input.
%
%%%%%%%%%%%%%%%%%%%%%%%% Springer-Verlag %%%%%%%%%%%%%%%%%%%%%%%%%%
%\motto{Use the template \emph{chapter.tex} to style the various elements of your chapter content.}
\chapter{Functional Completeness}
\label{FuncComp} % Always give a unique label
% use \chaptermark{}
% to alter or adjust the chapter heading in the running head

\section{ Definitions and Main Theorem}

\begin{definition}
    We say that some set of connectives are \Emph{jointly functionally complete} if and only if, for any possible truth table, there is a sentence containing only those connectives with that truth table.
\end{definition}

\begin{theorem}[Functional Completeness Theorem]
    The connectives of TFL are jointly functionally complete. Indeed, the following pairs of connective are jointly functionally complete: \begin{enumerate}
        \item `$\enot$' and `$\eor$'
        \item `$\enot$' and `$\eand$'
        \item `$\enot$' and `$\eif$'
    \end{enumerate}
\end{theorem}
\begin{proof}
    \emph{Subsidiary Result 1: functional completeness of `$\enot$' and `$\eor$'}. Observe that the scheme that we generate, using the truth table method of proving the DNF Theorem, will only contain the connectives `$\enot$', `$\eand$', and `$\eor$'. So it suffices to show that there is an equivalent scheme which contains only `$\enot$' and `$\eor$'. To show this we simply consider that \begin{equation*}
        (\metav{A}\eand\metav{B}) \;\;\text{ and }\;\;\enot(\enot\metav{A}\eor\enot\metav{B})
    \end{equation*}
    are logically equivalent.

    \emph{Subsidiary Result 2: functional completeness of `$\enot$' and `$\eand$'}. Exactly as in Subsidiary Result 1, making use of the fact that \begin{equation*}
        (\metav{A}\eor\metav{B}) \;\;\text{ and }\;\;\enot(\enot\metav{A}\eand\enot\metav{B})
    \end{equation*}
    are logically equivalent.

    \emph{Subsidiary Result 3: functional completeness of `$\enot$' and `$\eif$'}. Exactly as in Subsidiary Result 1, making use of the equivalences: \begin{align*}
        (\metav{A}\eor\metav{B})\;\;&\text{ and }\;\;(\enot\metav{A}\eif \metav{B}) \\
        (\metav{A}\eand\metav{B})\;\;&\text{ and }\;\;\enot(\metav{A}\eif\enot\metav{B})
    \end{align*}
\end{proof}

\section{ Individually Functionally Complete Connectives}

\begin{definition}
    A connective is \Emph{individually functionally complete} if any truth table can be constructed using only it and valuations of atomic sentences.
\end{definition}

\begin{definition}
    The connective `$\uparrow$' is truth functionally complete, with characteristic truth table: 
        \begin{table}[H]
            \centering
            \caption{\Emph{`$\uparrow$'}}
            \begin{tabular}{cc|c}
                $\metav{A}$ & $\metav{B}$ & $\metav{A} \uparrow \metav{B}$\\ \hline
                \textbf{T} & \textbf{T} & \textbf{F} \\
                \textbf{T} & \textbf{F} & \textbf{T} \\
                \textbf{F} & \textbf{T} & \textbf{T} \\
                \textbf{F} & \textbf{F} & \textbf{T}
            \end{tabular}
        \end{table}
        This is often called `the Sheffer stroke'. It is also commonly called `nand' as its characteristic truth table is the negation of the truth table for `$\eand$'.
\end{definition}
\begin{proof}
    It is sufficicent to show that `$\enot$' and `$\eor$' can be represented by `$\uparrow$', which indeed they can as seen in the the following equivalences: 
    \begin{align*}
        \enot\metav{A}\;\;&\text{ and }\;\;(\metav{A}\uparrow \metav{A}) \\
        (\metav{A}\eor\metav{B})\;\;&\text{ and }\;\;((\metav{A}\uparrow\metav{A})\uparrow(\metav{B}\uparrow\metav{B}))
    \end{align*}
\end{proof}


\begin{definition}
    The connective `$\downarrow$' is truth functionally complete, with characteristic truth table: 
        \begin{table}[H]
            \centering
            \caption{\Emph{`$\downarrow$'}}
            \begin{tabular}{cc|c}
                $\metav{A}$ & $\metav{B}$ & $\metav{A} \downarrow \metav{B}$\\ \hline
                \textbf{T} & \textbf{T} & \textbf{F} \\
                \textbf{T} & \textbf{F} & \textbf{F} \\
                \textbf{F} & \textbf{T} & \textbf{F} \\
                \textbf{F} & \textbf{F} & \textbf{T}
            \end{tabular}
        \end{table}
        This is often called `Peirce arrow'. It is also commonly called `nor' as its characteristic truth table is the negation of the truth table for `$\eor$'.
\end{definition}
\begin{proof}
    It is sufficicent to show that `$\enot$' and `$\eand$' can be represented by `$\downarrow$', which indeed they can as seen in the the following equivalences: 
    \begin{align*}
        \enot\metav{A}\;\;&\text{ and }\;\;(\metav{A}\downarrow \metav{A}) \\
        (\metav{A}\eand\metav{B})\;\;&\text{ and }\;\;((\metav{A}\downarrow\metav{A})\downarrow(\metav{B}\downarrow\metav{B}))
    \end{align*}
\end{proof}


\begin{theorem}
    `$\eor$', `$\eand$', `$\eif$', and `$\eiff$' are not functionally complete by themselves. The \Emph{only} two-place connectives which are individually functionally complete are `$\uparrow$' and `$\downarrow$'.
\end{theorem}

