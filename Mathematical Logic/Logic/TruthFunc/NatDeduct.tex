%%%%%%%%%%%%%%%%%%%%% chapter.tex %%%%%%%%%%%%%%%%%%%%%%%%%%%%%%%%%
%
% sample chapter
%
% Use this file as a template for your own input.
%
%%%%%%%%%%%%%%%%%%%%%%%% Springer-Verlag %%%%%%%%%%%%%%%%%%%%%%%%%%
%\motto{Use the template \emph{chapter.tex} to style the various elements of your chapter content.}
\chapter{Natural Deduction}
\label{NatDeduct} % Always give a unique label
% use \chaptermark{}
% to alter or adjust the chapter heading in the running head


\section{ Initial Definitons and Reasoning}

\begin{definition}
    A \Emph{formal proof} is a sequence of sentences, some of which are marked as being initial assumptions (or premises). The last line of the formal proof is the conclusion.
\end{definition}

\begin{example}
    Our first example will be to prove the statement \begin{equation*}
        \lnot(A\lor B)\therefore \lnot A \wedge \lnot B
    \end{equation*}
    Let us begin with how we start such proofs\begin{fitchproof}
        \hypo{a1}{\enot (A\eor B)}
	\end{fitchproof}
    The horizontal line indicates a separation between the assumptions and main body of the proof. Before we complete this proof we need to determine some laws of inference we can use.
\end{example}

\section{ Rules of Inferences for TFL Natural Deduction}

\begin{definition}[Reiteration Rule]
    If you have already shown something in the course of a proof, the \Emph{reiteration rule} allows you to repeat it on a new line. For example: 
    \begin{fitchproof}
        \have[n]{a1}{\metav{A}}
        \have[$\vdots$]{}{\vdots}
        \have[k]{a2}{\metav{A}} \by{R}{a1}
    \end{fitchproof}
    This indicates that we have written $\metav{A}$ on line $n$. We write `R' to indicate we are using the reiteration rule. Note that both `$\metav{A}$' and `$n$',`$k$' are symbols of our metalanguage
\end{definition}


\begin{definition}[Conjunction Introduction]
    Our conjunction introduction rule states that for any sentences $\metav{A}$ and $\metav{B}$ of TFL: \begin{fitchproof}
        \have[m]{a1}{\metav{A}}
        \have[n]{a2}{\metav{B}}
        \have[\ ]{a3}{\metav{A} \eand \metav{B}} \ai{a1, a2}
    \end{fitchproof}
    When introducing a conjuction we reference the line number of the first conjunct first, and the second conjunct second.
\end{definition}

\begin{definition}[Conjunction Elimination]
    Our conjunction elimination rule states that for any sentences $\metav{A}$ and $\metav{B}$ of TFL: \begin{fitchproof}
        \have[m]{a1}{\metav{A}\eand \metav{B}}
        \have[\ ]{a2}{\metav{A}} \ae{a1}
    \end{fitchproof}
    or \begin{fitchproof}
        \have[m]{a1}{\metav{A}\eand \metav{B}}
        \have[\ ]{a2}{\metav{B}} \ae{a1}
    \end{fitchproof}
\end{definition}


\begin{remark}
    There is no reason that we have to apply conjunction introduction to two separate lines. Indeed: \begin{fitchproof}
        \hypo{a1}{A}
        \have{a2}{A\eand A} \ai{a1, a1}
    \end{fitchproof}
\end{remark}

\begin{definition}[Modus Ponens]
    \Emph{Modus Ponens} is a conditional elimination rule which states that \begin{fitchproof}
        \have[m]{a1}{\metav{A}\eif \metav{B}}
        \have[n]{a2}{\metav{A}}
        \have[\ ]{a3}{\metav{B}} \ce{a1,a2}
    \end{fitchproof}
    In citing we always cite the conditional's line number first.
\end{definition}

\begin{definition}{Conditional Introduction}
    First, we make an additional assumption $\metav{A}$ (in a subproof); from the additional assumption we prove $\metav{B}$. From this, we know that if $\metav{A}$ is true, then $\metav{B}$ is true. This is wrapped up in the following rule: \begin{fitchproof}
        \open
                \hypo[i]{A}{\metav{A}}
                \have[j]{B}{\metav{B}}
                \close
        \have[\ ]{AB}{\metav{A}\eif\metav{B}}\ci{A-B}
    \end{fitchproof}
\end{definition}


\begin{remark}
    Once we have closed a subproof, we cannot use any of the assumptions or results shown in it. In particular, to cite an individual line when applying rule: \begin{enumerate}
        \item the line must come before the line where the rule is applied, but
        \item not occur within a subproof that has been closed before the line where the rule is applied.
    \end{enumerate}
    When we close a subproof we say we have discharged the assumptions of the subproof.
\end{remark}


\begin{remark}
    To cite a subproof when applyiing a rule: \begin{enumerate}
        \item the cited subproof must come entirely before the application of the rule where it is cited,
        \item the cited subproof must not lie within some other closed subproof which is closed at the line it is cited, and 
        \item the last line of the cited subproof must not occur inside a nested subproof.
    \end{enumerate}
\end{remark}

\begin{definition}[Biconditional Introduction]
    The biconditional introduction rule is similar to two conditional introductions: \begin{fitchproof}
        \open
                \hypo[i]{a1}{\metav{A}}
                \have[j]{a2}{\metav{B}}
        \close
        \open 
                \hypo[k]{a3}{\metav{B}}
                \have[l]{a4}{\metav{A}}
        \close
        \have[\ ]{a5}{\metav{A} \eiff\metav{B}} \bi{a1-a2,a3-a4}
    \end{fitchproof}
\end{definition}

\begin{definition}[Biconditional Elimination]
    The biconditional elimination rule works like the conditional elimination rule, but in two ways: \begin{fitchproof}
        \have[m]{a1}{\metav{A}\eiff\metav{B}}
        \have[n]{a2}{\metav{B}}
        \have[\ ]{a3}{\metav{A}} \be{a1,a2}
    \end{fitchproof}
    and \begin{fitchproof}
        \have[m]{a1}{\metav{A}\eiff\metav{B}}
        \have[n]{a2}{\metav{A}}
        \have[\ ]{a3}{\metav{B}} \be{a1,a2}
    \end{fitchproof}
    Note that we always cite the line of the biconditional first when eliminating it.
\end{definition}


\begin{definition}[Disjunction Introduction]
    As disjunctions are inherently weaker than initial statements of truth, we have the following disjunction rules: \begin{fitchproof}
        \have[m]{a1}{\metav{A}}
        \have[\ ]{a2}{\metav{A} \eor \metav{B}}\oi{a1}
    \end{fitchproof}
    and \begin{fitchproof}
        \have[m]{a1}{\metav{A}}
        \have[\ ]{a2}{\metav{B} \eor \metav{A}}\oi{a1}
    \end{fitchproof}
\end{definition}


\begin{definition}[Disjunction Elimination]
    The disjunction elimintation rule requires subproofs, and is as follows: \begin{fitchproof}
        \have[m]{a1}{\metav{A} \eor \metav{B}}
        \open
                \hypo[i]{a2}{\metav{A}}
                \have[j]{a3}{\metav{C}}
        \close
        \open
                \hypo[k]{a4}{\metav{B}}
                \have[l]{a5}{\metav{C}}
        \close
        \have[\ ]{a6}{\metav{C}} \oe{a1, a2-a3, a4-a5}
    \end{fitchproof}
\end{definition}


\begin{definition}[Negation Elimination]
    With negation elimination we introduce The False, $\ered$: \begin{fitchproof}
        \have[m]{a1}{\enot\metav{A}}
        \have[n]{a2}{\metav{A}}
        \have[\ ]{a3}{\ered} \ne{a1,a2}
    \end{fitchproof}
    The order of the negation and statement do not matter, but we always cite the line number of the negation first.
\end{definition}


\begin{definition}[Negation Introduction]
    In introducing a negation we start a subproof with an assumption, and show that it leads to a contradiction (The False): \begin{fitchproof}
        \open
                \hypo[i]{a1}{\metav{A}}
                \have[j]{a2}{\ered}
        \close
        \have[\ ]{a3}{\enot \metav{A}} \ni{a1-a2}
    \end{fitchproof}
\end{definition}


\begin{definition}[Indirect Proof]
    This rule stems from the argument that if the assumption that $\metav{A}$ is false leads to a contradiction, then $\metav{A}$ cannot be false, so it must be true: \begin{fitchproof}
        \open
                \hypo[i]{a1}{\enot \metav{A}}
                \have[j]{a2}{\ered}
        \close
        \have[\ ]{a3}{\metav{A}} \ip{a1-a2}
    \end{fitchproof}
\end{definition}


\begin{definition}[Explosion Rule]
    The explosion rule allows you to derive anything from a contradiction (The False): \begin{fitchproof}
        \have[m]{a1}{\ered}
        \have[\ ]{a2}{\metav{A}} \re{a1}
    \end{fitchproof}
    where $\metav{A}$ is any sentence whatsoever. This rule is also known as \Emph{ex contradictione quod libet}, ``from contradiction, anything." We can rephrase this as a contradiction entails everything: \begin{equation*}
        \ered \entails \metav{A}
    \end{equation*}
\end{definition}



\subsection{ Additional and Derived Rules}

\begin{definition}
    The following is a natural argument form called a \Emph{disjunctive syllogism}: \begin{fitchproof}
        \have[m]{a1}{\metav{A}\eor\metav{B}}
        \have[n]{a2}{\enot\metav{A}}
        \have[\ ]{a3}{\metav{B}} \by{DS}{a1,a2}
    \end{fitchproof}
    and \begin{fitchproof}
        \have[m]{a1}{\metav{A}\eor\metav{B}}
        \have[n]{a2}{\enot\metav{B}}
        \have[\ ]{a3}{\metav{A}} \by{DS}{a1,a2}
    \end{fitchproof}
    When citing we always place the disjunction first.
\end{definition}
\begin{proof}
    Let $\metav{A}$ and $\metav{B}$ be formulas of TFL, and I shall show that $\metav{A}\eor\metav{B},\enot\metav{A}\entails \metav{B}$:
    \begin{fitchproof}
        \have{a1}{\metav{A}\eor\metav{B}}
        \hypo{a2}{\enot\metav{A}}
        \open
            \hypo{a3}{\metav{B}}
            \have{a4}{\metav{B}} \by{R}{a3}
        \close
        \open 
            \hypo{a5}{\metav{A}}
            \have{a6}{\ered} \ne{a2,a5}
            \have{a7}{\metav{B}} \re{a6}
        \close
        \have{a8}{\metav{B}} \oe{a1,a3-a4,a5-a7}
    \end{fitchproof}
\end{proof}

\begin{definition}
    The following is a natural argument form called \Emph{modus tollens}: \begin{fitchproof}
        \have[m]{a1}{\metav{A}\eif \metav{B}}
        \have[n]{a2}{\enot \metav{B}}
        \have[\ ]{a3}{\enot \metav{A}} \by{MT}{a1,a2}
    \end{fitchproof}
    When citing we always place the conditional first.
\end{definition}
\begin{proof}
    Let $\metav{A}$ and $\metav{B}$ be formulas of TFL, and I shall show that $\metav{A}\eif\metav{B},\enot\metav{B}\entails \enot\metav{A}$:
    \begin{fitchproof}
        \have{a1}{\metav{A}\eif\metav{B}}
        \hypo{a2}{\enot\metav{B}}
        \open
            \hypo{a3}{\metav{A}}
            \have{a4}{\metav{B}} \ce{a1,a3}
            \have{a5}{\ered} \ne{a2,a4}
        \close
        \have{a8}{\enot\metav{A}} \ni{a3-a5}
    \end{fitchproof}
\end{proof}


\begin{definition}
    The \Emph{double-negation elimination} rule states that: \begin{fitchproof}
        \have[m]{a1}{\enot\enot\metav{A}}
        \have[\ ]{a2}{\metav{A}} \by{DNE}{a1}
    \end{fitchproof}
\end{definition}
\begin{proof}
    Let $\metav{A}$ be a formula of TFL, and I shall show that $\enot\enot\metav{A}\entails \metav{A}$:
    \begin{fitchproof}
        \hypo{a1}{\enot\enot\metav{A}}
        \open
            \hypo{a2}{\enot\metav{A}}
            \have{a3}{\ered} \ne{a1,a2}
        \close
        \have{a4}{\metav{A}} \by{IP}{a2-a3}
    \end{fitchproof}
\end{proof}


\begin{definition}
    The \Emph{law of excluded middle} states that: \begin{fitchproof}
        \open
            \hypo[i]{a1}{\metav{A}}
            \have[j]{a2}{\metav{B}}
        \close
        \open 
            \hypo[m]{a3}{\enot\metav{A}}
            \have[n]{a4}{\metav{B}}
        \close
        \have[\ ]{a5}{\metav{B}} \by{LEM}{a1-a2,a3-a4}
    \end{fitchproof}
\end{definition}
\begin{proof}
    Let $\metav{A}$ and $\metav{B}$ be formulas of TFL, and I shall show that $\metav{A}\eif\metav{B},\enot\metav{A}\eif\metav{B}\entails \metav{B}$:
    \begin{fitchproof}
        \have{a1}{\metav{A}\eif\metav{B}}
        \hypo{a2}{\enot\metav{A}\eif\metav{B}}
        \open
            \hypo{a3}{\enot\metav{B}}
            \have{a4}{\enot\metav{A}} \by{MT}{a1,a3}
            \have{a5}{\enot\enot\metav{A}} \by{MT}{a2,a3}
            \have{a6}{\ered} \ne{a5,a4}
        \close
        \have{a7}{\metav{B}} \by{IP}{a3-a6}
    \end{fitchproof}
\end{proof}


\begin{definition}
    \Emph{De Morgan's Laws} state that: \begin{fitchproof}
        \have[m]{a1}{\enot(\metav{A}\eor\metav{B})}
        \have[\ ]{a2}{\enot\metav{A}\eand\enot\metav{B}} \by{DeM}{a1}
    \end{fitchproof}
    \begin{fitchproof}
        \have[m]{a1}{\enot(\metav{A}\eand\metav{B})}
        \have[\ ]{a2}{\enot\metav{A}\eor\enot\metav{B}} \by{DeM}{a1}
    \end{fitchproof}
     \begin{fitchproof}
        \have[m]{a1}{\enot\metav{A}\eor\enot\metav{B}}
         \have[\ ]{a2}{\enot(\metav{A}\eand\metav{B})} \by{DeM}{a1}
    \end{fitchproof}
    and
    \begin{fitchproof}
        \have[m]{a1}{\enot\metav{A}\eand\enot\metav{B}}
        \have[\ ]{a2}{\enot(\metav{A}\eor\metav{B})} \by{DeM}{a1}
    \end{fitchproof}
\end{definition}


\begin{remark}
    All of the `proofs' in this section are not proofs of TFL, but rather proof schemes as they use the metavariables $\metav{A}$ and $\metav{B}$. Nonetheless, they show how any of this rules can be derived from our basic rules for specific formulas of TFL.
\end{remark}


