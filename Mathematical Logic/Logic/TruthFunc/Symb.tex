%%%%%%%%%%%%%%%%%%%%% chapter.tex %%%%%%%%%%%%%%%%%%%%%%%%%%%%%%%%%
%
% sample chapter
%
% Use this file as a template for your own input.
%
%%%%%%%%%%%%%%%%%%%%%%%% Springer-Verlag %%%%%%%%%%%%%%%%%%%%%%%%%%
%\motto{Use the template \emph{chapter.tex} to style the various elements of your chapter content.}
\chapter{Symbolization and Ambiguity}
\label{Symb} % Always give a unique label
% use \chaptermark{}
% to alter or adjust the chapter heading in the running head

\section{ Atomic Sentences}

\begin{definition}
    TFL \Emph{symbolizes} basic sentences, or sentence components, as sentence letters such as $A,P_1,P_2,P_{432}$ etc. These sentence letters are \Emph{atomic sentences} of TFL, and when symbolizing a sentence in terms of sentence letters we provide a symbolization key, for instance:\begin{enumerate}
        \item[$\rightarrow$] $A$: I am a cat.
    \end{enumerate}
\end{definition}

\section{ Connectives}

\begin{table}[H]
    \centering
    \caption{Logical Connectives of TFL}
    \begin{tabular}{ccc}
        \textbf{Symbol} & \textbf{Name} & \textbf{Rough meaning} \\
        $\lnot$ & Negation & `It is not the case that...' \\
        $\wedge$ & Conjunction & `Both... and...'\\
        $\lor$ & Disjunction & `Either... or...' \\
        $\rightarrow$ & Conditional & `If ... then ...' \\
        $\leftrightarrow$ & Biconditional & `... if and only if ...'
    \end{tabular}
\end{table}


\begin{definition}
    Consider formulas $\metav{A}$ and $\metav{B}$. In $\lnot \metav{A}$ $\metav{A}$ is said to be the \Emph{negatum}. For $(\metav{A} \wedge \metav{B})$, $\metav{A}$ and $\metav{B}$ are called the \Emph{conjuncts}. For $(\metav{A} \lor \metav{B})$, $\metav{A}$ and $\metav{B}$ are called the \Emph{disjuncts}. For $(\metav{A} \rightarrow \metav{B})$, $\metav{A}$ is called the \Emph{antecedent} and $\metav{B}$ is called the \Emph{consequent}.
\end{definition}


\section{ TFL Sentences}

\begin{definition}
    The symbols of TFL are the atomic sentences ($A,B,...,Z,P_{453},...$), the connectives $\lnot,\wedge,\lor,\rightarrow,\leftrightarrow$, and brackets $(,)$.
\end{definition}

\begin{definition}
    An \Emph{expression of TFL} is any string of symbols of TFL.
\end{definition}


\begin{definition}
    The following are the only sentences of TFL:
    \begin{enumerate}
        \item Every sentence letter is a sentence.
        \item If $\metav{A}$ is a sentence, then $\lnot \metav{A}$ is also a sentence.
        \item If $\metav{A}$ and $\metav{B}$ are sentences, then $(\metav{A}\wedge \metav{B})$ is a sentence.
        \item If $\metav{A}$ and $\metav{B}$ are sentences, then $(\metav{A}\lor \metav{B})$ is a sentence.
        \item If $\metav{A}$ and $\metav{B}$ are sentences, then $(\metav{A}\rightarrow \metav{B})$ is a sentence.
        \item If $\metav{A}$ and $\metav{B}$ are sentences, then $(\metav{A}\leftrightarrow \metav{B})$ is a sentence.
        \item Nothing else is a sentence.
    \end{enumerate}
    The last sentential connective used in constructing a TFL sentence is called the \Emph{main logical operator}.
\end{definition}


\begin{definition}
    The \Emph{scope} of a connective (in a sentence) is the subsentence for which that connective is the main logical operator.
\end{definition}




\section{ Ambiguity}

\begin{definition}
    \Emph{Lexical ambiguity} is when a sentence contains words which have more than one meaning.
\end{definition}

\begin{definition}
    \Emph{Structural ambiguity} occurs when a sentence can be interpreted in different ways, and depending on the interpretation, a different meaning is selected.
\end{definition}


\section{ Object and Meta languages}

\begin{remark}
    When we want to talk about things in the world we just \emph{use} words. When we talk about words we typically have to \emph{mention} the words. Usually, \emph{mentioning} is done using single quotation marks `' (or double quotes if encasing single quotes).
\end{remark}


\begin{definition}
    When we talk about a language the language we are talking about is called the \Emph{object language}. The language that we use to talk about the object language is called the \Emph{metalanguange}.
\end{definition}

\begin{remark}
    In TFL sentence letters are sentences of the object language. When refering to a sentence letter in the metalanguage of English (supplemented with some symbols), we may write something along the lines of: `D' is a sentence letter of TFL.
\end{remark}


\begin{definition}
    We define \Emph{metavariables} for our augmented metalanguage English to talk about any expression of TFL: \begin{equation}
        \metav{A},\metav{B},\metav{C},\metav{D},...
    \end{equation}
    In particular, `$\metav{A}$' is a symbo (called a \Emph{metavariable}) in the augmented English we use to talk about expressions of TFL.
\end{definition}

\begin{definition}
    Suppose we wish to symbolize the premises of an argument by $\metav{A}_1,...,\metav{A}_n$, and the conclusion of the argument by $\metav{C}$. Then we will write:\begin{equation}
        \metav{A}_1,...,\metav{A}_n \therefore \metav{C}
    \end{equation}
    The purpose of the `$\therefore$' symbol is to indicate which sentences are premises and which are conclusions. Strictly speaking `$\therefore$' is a part of our metalanguage, but we shall take the convention to not include quotation marks around the TFL sentences which flank it.
\end{definition}



