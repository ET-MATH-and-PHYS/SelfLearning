%%%%%%%%%%%%%%%%%%%%% chapter.tex %%%%%%%%%%%%%%%%%%%%%%%%%%%%%%%%%
%
% sample chapter
%
% Use this file as a template for your own input.
%
%%%%%%%%%%%%%%%%%%%%%%%% Springer-Verlag %%%%%%%%%%%%%%%%%%%%%%%%%%
%\motto{Use the template \emph{chapter.tex} to style the various elements of your chapter content.}
\chapter{Arguments}
\label{Arg} % Always give a unique label
% use \chaptermark{}
% to alter or adjust the chapter heading in the running head

\section{ Definitions and Examples: Arguments}

\begin{definition}
    An \Emph{argument} (or \Emph{deductive argument}) is any collections of premises, together with a conclusion.

    To be completely general, we can define an argument as a series of sentences. The sentences at the beginning are premises, and the final sentence in the series is the conclusion.
\end{definition}

\begin{definition}
    A \Emph{statement} is a sentence which is either true or false.
\end{definition}


\begin{remark}
    Questions, imperative sentences, and exclamatory sentences are all not statements. Commands are often phrased as imperative sentences.
\end{remark}



\section{ Validity}

\begin{definition}[Informal Validity]{}
    An argument is said to be \Emph{valid} if the conclusion follows from the premises.
\end{definition}

\begin{definition}[Informal Invalidity]{}
    If in an argument the conclusion does not follow from the premises, then the argument is said to be \Emph{invalid}.
\end{definition}


\begin{definition}[Consequence]{}
    A sentence $A$ is a \Emph{consequence} of sentences $B_1,...,B_n$ if and only if there is no case where $B_1,...,B_n$ are all true and $A$ is not true. (We then say that $A$ follows from $B_1,...,B_n$, or that $B_1,...,B_n$ \Emph{entail} $A$)
\end{definition}


\begin{definition}[Informal Validity 2]{}
    An argument is \Emph{valid} if and only if the conclusion is a consequence of the premises.
\end{definition}

\begin{definition}[Informal Invalidity 2]{}
    An argumnt is \Emph{invalid} if and only if it is not valid, i.e., it has a counter-example.
\end{definition}


\begin{remark}
    An argument is \Emph{nomologically valid} if there are no counter-examples which obey the laws of physics. An argument is \Emph{conceptually valid} if there are no counter-examples that don't violate conceptual connections between words.
\end{remark}


\begin{remark}
    An argument is \Emph{formally valid} if we can describe the ``form" of the argument as a logical pattern.
\end{remark}

\section{ Other Notions}

\begin{definition}
    An argument is said to be \Emph{sound} if and only if it is valid and its premises are true.
\end{definition}


\begin{definition}
    An \Emph{inductive argument} is an argument which generalises from observations about many past cases to a conclusion about all future cases. Inductive arguments are not deductively valid.
\end{definition}

\begin{definition}[Jointly Possible]{}
    Sentences are \Emph{jointly possible} if and only if there is a case where they are all true together.
\end{definition}

\begin{definition}[Contingent]{}
    A sentence which is capable of being true in one case and capable of being false in another case is called \Emph{contingent}.
\end{definition}

\begin{definition}
    A sentence is a \Emph{necessary truth} if it is true in all cases.
\end{definition}

\begin{definition}
    A sentence is a \Emph{necessary falsehood} if it is false in all cases.
\end{definition}

\begin{definition}
    If two sentences have the same truth value in every case, we say that they are \Emph{necessarily equivalent}.
\end{definition}
