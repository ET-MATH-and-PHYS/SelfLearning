%%%%%%%%%%%%%%%%%%%%% chapter.tex %%%%%%%%%%%%%%%%%%%%%%%%%%%%%%%%%
%
% sample chapter
%
% Use this file as a template for your own input.
%
%%%%%%%%%%%%%%%%%%%%%%%% Springer-Verlag %%%%%%%%%%%%%%%%%%%%%%%%%%
%\motto{Use the template \emph{chapter.tex} to style the various elements of your chapter content.}
\chapter{Basics}
\label{Basic} % Always give a unique label
% use \chaptermark{}
% to alter or adjust the chapter heading in the running head



\section{Types}

Informally, types are ``things" that ``contain" terms. In dependent type theory, everything is a \textbf{term}, and every term has a \textbf{type}. Notationally, we write $t\;:\;\tau$ to denote the fact that $t$ is a \textbf{term} of \textbf{type} $\tau$.

\begin{eg}
    We have the type of natural numbers, $\N$, as well as the type of rationals, $\Q$, reals, $\R$, etc.
\end{eg}

As all types are terms, what is the type of $\N$? Well it is $\catname{Type}$! (well technically $\catname{Type}\;0$) This means we need a heirarchy of types, so that $\catname{Type}\;0$ can also be a term of the type, $\catname{Type}\;1$, etc. 

We can also put types together to form \textbf{function types}, such as $\N\rightarrow \N, \N\rightarrow \Q$, and $\N\rightarrow \R$. In type theory, the function constructor on types is a \textbf{primitive object}. 

\begin{eg}
    If we want a squaring function, for example, in Lean4 we can specify it by:
    %%
    \begin{equation*}
        \text{fun}\;n\;:\;\N\mapsto n^2
    \end{equation*}
    %%
    The ``fun" keyword in Lean4 can be used as a lambda syntax, but it also has a lambda syntax as well.
\end{eg}


\subsection{Propositions}

In dependent type theory, and Lean4 specifically, propositions are also modeled by types. In order to discuss this we introduce the type hierarchy ``$\text{Sort}\;n\;:\;\text{Sort}\;(n+1)$" where $\text{Type}\;u\;:=\;\text{Sort}\;(u+1)$. In this hierarchy, $\text{Sort}\;0$ is given the special notation $\text{Prop}$, and denotes the type of propositions.

Lean's type theory has a special feature known as \textbf{proof irrelevance}, which is inconsistent with HoTT. Proof irrelevance means that all proofs are equal in Lean4. 

Note that dependent type theory is useful for propositions, and mathematics generally, so that we can perform quantification in our propositions.


\section{Curry-Howard-Correspondence}






