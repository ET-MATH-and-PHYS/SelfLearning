%%%%%%%%%% Grp Actions %%%%%%%%%
\chapter{\textsection\textsection Group Actions}

\section{\textsection Basic Definitions and Examples: Group Actions}

\begin{rmk}[Motivation]
        We want to view elements of a group $G$ as symmetries of a set $X$. In particular, for every $g \in G$ we want to associate $X \xrightarrow{\alpha_g} X$ a bijection, with $\alpha_{gh} = \alpha_g \circ \alpha_h$ and $\alpha_{e_G} = \id_X$.
\end{rmk}

\begin{eg}
        \leavevmode
        \begin{enumerate}
                \item For $X_n$, the regular convex $n$-gon, and $G = D_n$, for all $g \in D_n$ we get a bijection $X_n \xrightarrow{g} X_n$.
                \item For $X = \{1,2,...,n\}$, and $G = S_n$, for all $\sigma \in S_n$ we get a permutation $X \xrightarrow{\sigma} X$.
                \item For $X = \R^n$, and $G = \GL_n(\R)$, for all $A \in \GL_n(\R)$ we get a bijection \begin{equation}
                                \map{\R^n\xrightarrow{L_A}\R^n}{\vec{v} \mapsto A\vec{v}}
                \end{equation}
                \item For $X = G$ a group, we have the action by \Emph{left multiplication}, where for all $g \in G$ we get the bijection \begin{equation}
                                \map{G\xrightarrow{\ell_g}G}{x\mapsto gx}
                \end{equation}
                \item For $X = G$ a group, and $H \leq G$, for all $h \in H$ we have the bijection \begin{equation}
                                \map{G \xrightarrow{\ell_h} G}{g \mapsto hg}
                \end{equation}
                \item For $X = G$ a group, we have the action by \Emph{conjugation}, where for all $g \in G$, we have the bijection \begin{equation}
                                \map{G\xrightarrow{\beta_g}G}{x\mapsto gxg^{-1}}
                \end{equation}
        \end{enumerate}
\end{eg}

\begin{rmk}
        There are two equivalent ways to formalize the notion of group actions.
\end{rmk}

\begin{defn}[Group Action]
        A \Emph{group action} of a group $G$ on a set $X$ is a \begin{enumerate}
                \item group homomorphism \begin{equation}
                                \map{\alpha:G\rightarrow S_X}{g\mapsto \alpha_g}
                \end{equation}
                \item map $a:G\times X \rightarrow X$ such that \begin{enumerate}
                                \item $a(e_G,x) = x$ for all $x \in X$
                                \item $a(gh,x) = a(g,a(h,x))$ for all $g,h \in G$ and all $x \in X$.
                \end{enumerate}
        \end{enumerate}
\end{defn}

\begin{defn}
        Let $G$ be a group acting on a set $X$. The data of 1. (or equivalently 2.) in the previous definitions is called an \Emph{action of $G$ on $X$} and $X$ is called a \Emph{$G$-set}.
\end{defn}

\begin{claim}
        Definitions 1. and 2. of a group action are equivalent. That is, for any group $G$ and any nonempty set $A$ there is a bijection between the actions of $G$ on $A$ and the group homomorphisms of $G$ into $S_A$.
\end{claim}
\begin{proof}
        (Left to the reader)
\end{proof}


\begin{defn}
    Let $G$ be a group acting on a nonempty set $A$. Then the homomorphism $\alpha:G\rightarrow S_A$ associated with the action of $G$ on $A$ is called a \Emph{permutation representation} associated to the given action. We say a given action of $G$ on $A$ \emph{affords} or \emph{induces} the associated permutation representation of $G$.
\end{defn}


\begin{defn}
        The kernel of the group homomorphism associated with a group action is $\ker(\alpha) = \{g \in G:\alpha(g) = \id_X\}$, or equivalently for definition 2. the set $\{g \in G: \forall x \in X, a(g,x) = x\}$. If $\ker(\alpha) = \{e_G\}$, then the action of $G$ on $X$ is said to be \Emph{faithful}.
\end{defn}

\begin{rmk}
    Two group elements in $G$ induce the same permutation of the set $A$ if and only if the exist in the same coset of the kernel of the action (i.e., if and only if they are in the same fiber of the permutation representation $\alpha$).


    Moreover, the inhereted action of the quotient space $G/\ker(\alpha)$ on $A$ is faithful.
\end{rmk}


\begin{eg}
        \leavevmode
        \begin{enumerate}
                \item For the left action of $G$ on $G$, $\ker = \{g \in G: \ell_g = \id_G\}$. We want $\ell_g(h) = h$ for all $h \in G$, where $\ell_g(h) = gh$, so $g = e_G$. Thus, $\ker = \{e_G\}$, so the action is faithful.
                \item For the conjugation action, with the associated group homomorphism \begin{equation}
                                \map{\beta:G\rightarrow S_G}{g\mapsto\map{\beta_g:G\rightarrow G}{x\mapsto gxg^{-1}}}
                        \end{equation}
                        We claim that $\ker(\beta) = Z(G)$, the center of $G$.
        \end{enumerate}
\end{eg}


\begin{defn}
    The permutation representation afforded by the left multiplication action on the elements of the group $G$ is called the \Emph{left regular representation} of $G$.
\end{defn}

\begin{namthm}[Cayley's Theorem]
        Every group $G$ is isomorphic to a subgroup of its group of symmetries $\sym(G)$.
\end{namthm}
\begin{proof}
        (Left to the reader)
\end{proof}


\begin{nota} 
        If $G\times Y\xrightarrow{a}Y$ is a group action, we denote $a(g,y)$ by $g.y$ (or even $gy$ if there is no confusion). Note that we have $(gh).y = g.(h.y)$ and $e_G.y = y$ for all $g,h \in G$ and all $y \in Y$. Moreover, to say that a group $G$ acts on a set $Y$ we write $G \circlearrowright Y$.
\end{nota}


\begin{defn}
        Let $G\times Y \xrightarrow{a} Y$ be an action and let $y \in Y$. \begin{enumerate}
                \item The \Emph{orbit} of $y$ under the action by $G$ is the set $\mathcal{O}_y = \{g.y:g\in G\}\subseteq Y$ (also denoted $G.y$)
                \item The \Emph{stabilizer} of $y$ under the action by $G$ is the set $G_y = \{g\in G:g.y = y\}\subseteq G$.
                \item $y \in Y$ is called a \Emph{fixed point} of the action if $G_y = G$, so for all $g \in G$, $g.y = y$.
        \end{enumerate}
\end{defn}

\begin{defn}
    Let $G$ be a group and $A$ a nonempty set. The action of $G$ on $A$ is said to be \Emph{transitive} if there is only one orbit, i.e., given any two elements $a,b \in A$, there exists $g \in G$ such that $b = g.a$.
\end{defn}



\begin{prop}
        For $G \circlearrowright Y$ and all $y \in Y$, $G_y \leq G$.
\end{prop}
\begin{proof}
        (Left to the reader)
\end{proof}


\begin{eg}
        \leavevmode
        \begin{enumerate}
                \item For the left multiplication action, $G \circlearrowright G$, for all $g \in G$ we have orbits $G.g = \mathcal{O}_g = G$, and stabilizers $G_g = \{e_G\}$.
                \item For the conjugation action, $G \circlearrowright G$, for all $g \in G$ we have orbits \begin{equation}
                                G.g = \mathcal{O}_g = \{a \in G:\exists h \in H, hgh^{-1} = a\}
                \end{equation}
                        These sets are called the \Emph{conjugacy classes} of $G$. The stabilizers of the action are \begin{equation}
                                G_g = \{a \in G: aga^{-1} = g\} = Z(g)
                        \end{equation}
                        the \Emph{centralizer} of $g$ in $G$.
                \item The left multiplication action of a subgroup $H \leq G$ on $G$, $H \circlearrowright G$, for all $g \in G$ the orbit is $H.g = Hg$ the right coset of $H$. Moreover, the stabilizers still are $H_g = \{e_G\}$.
        \end{enumerate}
\end{eg}

\begin{lem}
        Let $a:G\times Y \rightarrow Y$ be a group action. Then \begin{enumerate}
                \item The orbits $G.y = \mathcal{O}_y$ of the action form a partition of $Y$.
                \item For all $y \in Y$, the order of the orbit $|G.y| = |\mathcal{O}_y|$, is the index $|G:G_y|$ of the stabilizer $G_y$ of $y$ in $G$. (\Emph{Orbit Stabilizer Theorem})
        \end{enumerate}
\end{lem}
\begin{proof}
        [1.] First, note that $y = e_G.y \in G.y$ for all $y \in Y$, so \begin{equation}
                Y = \bigcup_{y \in Y}G.y
        \end{equation}
        Next, let $y,y' \in Y$ and suppose $g.y = g'.y' \in G.y \cap G.y'$, for some $g,g' \in G$. Then we have that $y = g^{-1}.(g'.y') = (g^{-1}g').y' \in G.y'$, so for all $h.y \in G.y$, $h.y = (hg^{-1}g').y' \in G.y'$ so $G.y \subseteq G.y'$. Similarly we have that $G.y \supseteq G.y'$, so $G.y = G.y'$. Hence, the orbits indeed partition $Y$.


        [2.] Define a map \begin{equation}
                \map{G/G_y\xrightarrow{f} G.y}{aG_y\mapsto a.y}
        \end{equation}
        where $G/G_y$ denotes the set of left cosets of $G_y$ (not necessarily a group). First, to show the map is well defined suppose $aG_y = bG_y$. Then we have that $a = bg$ for some $g \in G_y$. It follows that $$a.y = (bg).y = b.(g.y) = b.y$$ since $g \in G_y$, so $f(aG_y) = f(bG_y)$ and the map is well defined. Now suppose $aG_y, cG_y \in G/G_y$ such that $a.y = c.y$. Then $(c^{-1}a).y = c^{-1}.(c.y) = e_G.y = y$, which implies $c^{-1}a \in G_y$. This implies by coset equality that $aG_y = cG_y$, so $f$ is an injection. Finally, if $g.y \in G.y$, we have that $f(gG_y) = g.y$, so $f$ is a surjection. Therefore $f$ is a bijection and we conclude that \begin{equation}
                |G:G_y| = |G/G_y| = |G.y|
        \end{equation}
        as claimed.
\end{proof}


\subsection{\textsection Application to Cycle Decompositions}

Using the tools we have developed with group actions, we can provide an alternate proof to the fact that any permutation $\sigma \in S_n$ can be decomposed into disjoint cycles.

\begin{proof}
    Let $A = \{1,2,...,n\}$, let $\sigma \in S_n$, and let $G =\langle \sigma \rangle$. Then consider the action of $\langle \sigma \rangle$ on $A$. Let $\mathcal{O}$ be one of the orbits of this action, and let $x \in \mathcal{O}$. Note that there exists a bijection between the elements of $\mathcal{O}$ and the left cosets of the stabilizer $G_x$ in $G$, given explicitly by \begin{equation*}
        \sigma^ix\mapsto \sigma^iG_x
    \end{equation*}
    Since $G$ is cyclic it is abelian, so $G_x \trianglelefteq G$ and $G/G_x$ is cyclic of order $d$, where $d$ is the order of $\sigma$ in $G/G_x$, in particular it is the smallest positive integer for which $\sigma^d \in G_x$. Also, $d = |G:G_x| = |\mathcal{O}|$. Thus, the distinct cosets of $G_x$ in $G$ are \begin{equation*}
        1G_x,\sigma G_x,\sigma^2G_x,...,\sigma^{d-1}G_x
    \end{equation*}
    This shows that the distinct elements of $\mathcal{O}$ are $x,\sigma(x),...,\sigma^{d-1}(x)$ by our bijection. Orderin the elements of $\mathcal{O}$ in this manner shows that $\sigma$ cycles the elements of $\mathcal{O}$, that is, on an orbit of size $d$, $\sigma$ acts as a $d$-cycle. This proves the existence of a cycle decomposition for each $\sigma \in S_n$.


    The orbits of $\langle \sigma \rangle$ are uniquely determined by $\sigma$. The only choice is in the order the orbits are listed in, which depends on our initial representative from $\mathcal{O}$. It follows that the cycle decomposition above is unique up to a rearrangement of the cycles and up to a cyclic permutation of the integers within each cycle.
\end{proof}


\section{\textsection Counting and Combinatorial Formulas}


\begin{namthm}[Counting Formula]
        Suppose $G \circlearrowright Y$. Then for all $y \in Y$ we have the \Emph{counting formula} \begin{equation}
                |G| = |G_y||G.y|
        \end{equation}
\end{namthm}
\begin{proof}
        (Left to the reader)
\end{proof}


\begin{namthm}[Orbit Decomposition Theorem]
        Let $Y$ be a finite set with $G \circlearrowright Y$. Let $Y_f \subseteq Y$ denote the set of fixed points of $Y$ under the action. Let $G.y_1, ..., G.y_n$ be the distinct non-singular orbits of $Y$ for some integer $n \geq 0$. Then \begin{equation}
                |Y| = |Y_f| + \sum_{i=1}^n|G:G_y|
        \end{equation}
\end{namthm}
\begin{proof}
        (Left to the reader)
\end{proof}

\begin{cor}[Class Equation]
        For a group $G$ and the conjugation action $G \circlearrowright G$, we have the \Emph{class equation} \begin{equation}
                |G| = \sum\limits_{\text{Conjugacy Classes $C$}}|C|
        \end{equation}
\end{cor}

\begin{rmk}
        By the counting formula we have that $|C|\;\vert\;|G|$ for all conjugacy classes $C$.
\end{rmk}


\begin{prop}
        The set of fixed elements of the conjugacy action of $G$ on $G$, $G \circlearrowright G$, is the center of $G$, $Z(G)$.
\end{prop}
\begin{proof}
        (Left to the reader)
\end{proof}


\begin{prop}
        If $H \vartriangleleft G$, then $H$ is a union of conjugacy classes. Indeed, for all $h \in H$ and all $g \in G$, $g.h=ghg^{-1} \in H$, so $G.h \subseteq H$.
\end{prop}
\begin{proof}
        (Left to the reader)
\end{proof}


\begin{eg}
        \leavevmode
        \begin{enumerate}
                \item For an abelian group we get the class equation \begin{equation}
                                |G| = 1+1+...+1
                \end{equation}
                \item For $D_3 (\cong S_3)$, $|D_3| = 6$, and $D_3 = \langle x,y\rangle$. First, note that $|G.x| = [D_3:G_x] = [D_3:Z(x)] = 6/3 = 2$, where $\langle x \rangle \subseteq Z(x)$ while $y \notin Z(x)$ as $yxy = x^2$, so by \ref{thmname:lagrange} $|Z(x)| = 3$ and $Z(x) = \langle x \rangle$. Similarly, $Z(y) = \langle y \rangle$ as $xyx^2 = x^2y \neq y$, so $x,x^2 \notin Z(y)$, so $|G.y| = [D_3:Z(y)] = 6/2 = 3$. Thus, we have that \begin{equation}
                                |D_3| = |D_3.e| + |D_3.x| + |D_3.y| = 1 + 2 + e
                \end{equation}
        \item The class equation of $A_5$ is \begin{equation}
                        |A_5| = 1+ 20 + 12 + 12 + 15
        \end{equation}
                        First note that $|A_5| = \frac{5!}{2} = 60$, and $A_5$ is composed all even permutations in $S_5$. First, $1 = |\{(1)\} = A_5.(1)|$. Then, for the three cycles in $A_5$, there are $\frac{5*4*3}{3} = 20$ three cycles. In $S_5$ all three cycles are in the same conjugacy class. It follows that for any three cycle $\sigma \in A_5$, $|S_5.\sigma| = 20 = |S_5:{S_5}_{\sigma}|$, which implies $|{S_5}_{\sigma}| = 6$. Note $\langle \sigma \rangle \subseteq {S_5}_{\sigma}$, but also for the $1 \leq i,j \leq 5$ not moved by $\sigma$, $(i\;j), \sigma(i\;j), \sigma^{-1}(i\;j) \in {S_5}_{\sigma}$. But, $(i\;j), \sigma(i\;j), \sigma^{-1}(i\;j) \notin A_5$ as they are odd, but $\langle \sigma \rangle \subseteq A_5$ so we have that $|{A_5}_{\sigma}| = 3$. Thus, $|A_5.\sigma| = [A_5:{A_5}_{\sigma}] = 60/3 = 20$, which is the entire conjugacy class. Next, for pairs of disjoint transpositions we have $\frac{5*4*3*2}{2*2*2} = 15$, which are all in the same conjugacy class in $S_5$. Thus, for a transposition pair $\tau$ we have $|S_5.\tau| = 15 = [S_5:{S_5}_{\tau}]$, so $|{S_5}_{\tau}| = 120/15 = 8$. Individual transpositions of the pair are in the centralizer, but they are not in $A_5$ as they are odd permutations, so $|{A_5}_{\tau}| \leq 6$. Moreover, the two four cycles composed of the pair of transpositions adjoined are in the centralizer while not being in $A_5$, so $|{A_5}_{\tau}| \leq 4$. But, note that if $|{A_5}_{\tau}| \leq 3$ then $|A_5.\tau| \geq 20$, but there are only $15$ elements in the conjugacy class of pairs of transpositions in $S_5$. Thus, we have that $|{A_5}_{\tau}| = 4$ so $|A_5.\tau| = |A_5:{A_5}_{\tau}| = 60/4 = 15$. Finally, we have $5!/5 =24$ five cycles in $A_5$. Let $\alpha$ be one such five cycle. Then $|S_5.\alpha| = 24 = [S_5:{S_5}_{\alpha}]$, which implies that $|{S_5}_{\alpha}| = 5$. But $\langle \alpha \rangle \subseteq {S_5}_{\alpha}$ and $\langle \alpha \rangle \subseteq A_5$ with $|\langle \alpha\rangle| = 5$, so ${S_5}_{\alpha} = {A_5}_{\alpha}$. Thus, we have that $|A_5.\alpha| = [A_5:{A_5}_{\alpha}] = 60/5 = 12$. Therefore, the conjugacy class is split in two, and we have arrived at our class equation.
        \end{enumerate}
\end{eg}

\begin{cor} 
        $A_5$ is a simple group.
\end{cor}
\begin{proof}
        Let $N \vartriangleleft A_5$. We must have that $|N|\;\vert\;|A_5| = 60$ and $|N| = \sum|C|$ for conjugacy classes $C$ of $A_5$, and $\{(1)\}$ is one of them. But by the class equation the only possibilities are $|N| = 1$ and $|N| = 60$. Thus, $N \vartriangleleft A_5$ implies either $N = \{(1)\}$ or $N = A_5$, so $A_5$ is a simple group by definition.
\end{proof}

\begin{prop}
        Let $|G| = p^n$ for a prime $p$ (such a group $G$ is called a \Emph{p-group}). The center of $G$ is not the trivial subgroup $\{e_G\}$.
\end{prop}
\begin{proof}
        The class equation of $G$ is \begin{equation}
                p^n = |G| = 1+\sum_i|C_i|
        \end{equation}
        Note $g \in Z(G)$ if and only if $G.g = \{g\}$. So, if $Z(g) = \{e_G\}$ then $|C_i| > 1$ for all $i$. Thus, $p\;\vert\;|C_i|$ for all $i$ since $1 < |C_i|\;\vert\;|G| = p^n$. Hence, \begin{equation}
                p\;\vert\;\left(|G|-\sum_i|C_i|\right) = |C_1| = 1
        \end{equation}
        which is a contradiction as this implies $p = 1$, and $1$ is not prime. So, $|Z(G)| > 1$ as claimed.
\end{proof}


\begin{namthm}[Cauchy's Theorem]\label{thmname:cauchpthm}
        Let $G$ be a finite group. If $p\;\vert\;|G|$ for $p$ a prime, then $G$ has an element of order $p$.
\end{namthm}
\begin{proof}
        We want to define an action on the set $$G^p = \prod\limits_{i=1}^pG$$ by a cyclic group of order $p$ ($\cong \Z/p\Z$). Let $H = \langle \sigma \rangle$ for $o(\sigma) = p$, and we define the action by \begin{equation}
                \sigma.(g_1,g_2,...,g_p):=(g_2,g_3,...,g_p,g_1)
        \end{equation}
        This gives a well-defined group action of $H$ on $G^p$. Moreover, $x \in G^p$ is a fixed point of the action if and only if \begin{equation}
                (g_1,g_2,...,g_p) = (g_2,g_3,...,g_p,g_1)
        \end{equation}
        so $g_1=g_2=...=g_p$. We are interested in a subset $Y \subseteq G^p$ defined by \begin{equation}
                Y:= \{(g_1,g_2,...,g_p)\in G^p: g_1g_2...g_p = e\}
        \end{equation}
        We see that for all $y \in Y$, $H.y \subseteq Y$. Indeed, if $g_1...g_p = e$, then $e = g_1^{-1}g_1 = g_2...g_pg_1$, which is associated to $\sigma.(g_1,...,g_p)$. Hence, if $y \in Y$, then $\sigma.y \in Y$. Thus, we have an action of $H$ on $Y$ taken by corestriction. Next, $|Y| = |G|^{p-1}$. Indeed, choose $g_1,g_2,...,g_{p-1}$ freely, which constitutes $|G|^{p-1}$ choices, then choose $G_p = (g_1g_2...g_{p-1})^{-1}$. Then we have that $(g_1,...,g_p) \in Y$. Note that $y \in Y$ is fixed by $H$, for $y = (g_1,...,g_p)$, if and only if $g=g_1=...=g_p$, and $g_1...g_p = g^p = e$. Then, $o(g) \in \{1,p\}$. Note that $(e,...,e) \in Y$ is a fixed point. We want to show that $|Y_f| > 1$. Applying the Orbit Decomposition Theorem to the action of $H$ on $Y$ we have that \begin{equation}
                |G|^{p-1} = |Y| = |Y_f| + \sum_{i=1}^n|H:H_{y_i}|
        \end{equation}
        where each $|H:H_{y_i}|$ divides $|H| = p$ and is greater than $1$ as they are not fixed points, so in particular $|H:H_{y_i}| = p$ for each $i$. Thus, \begin{equation}
                |Y_f| = |G|^{p-1} - \sum_{i=1}^n|H:H_{y_i}|
        \end{equation}
        which $p$ divides as $p\;\vert\;|G|$ by our initial assumption. Then, $p\;\vert\;|Y_f|$ so in particular $|Y_f| > 1$. Hence, there exists $(g,g,...,g) \in Y_f$ such that $g \neq e$ and $g^p = e$ as desired.
\end{proof}


\section{\textsection Conjugacy Actions and Actions on Subgroups}

In this section we redefine and make precise certain concepts previously mentioned off hand in examples in relation to conjugation and subgroup actions.


\begin{thm}
    Let $G$ be a group, let $H$ be a subgroup of $G$, and let $G$ act by left multiplication on the set $A$ of left cosets of $H$ in $G$. Let $\pi_H$ be the associated permutation representation afforded by this action. Then \begin{enumerate}
        \item $G$ acts transitively on $A$
        \item the stabilizer in $G$ of the point $1H \in A$ is the subgroup $H$
        \item the kernel of the action (i.e., the kernel of $\pi_H$) is $\bigcap_{x\in G}xHx^{-1}$, and $\ker \pi_H$ is the largest normal subgroup of $G$ contained in $H$.
    \end{enumerate}
\end{thm}
\begin{proof}
    (Left to the reader)
\end{proof}

\begin{defn}
    Let $G$ be a group and define $Sub(G)$ as the set of all subgroups of $G$. Then $G$ acts on the set $Sub(G)$ by \begin{equation*}
        a:G\times Sub(G)\rightarrow Sub(G);\;(g,H) \mapsto g\cdot H:= gHg^{-1}
    \end{equation*}
    where \begin{equation*}
        gHg^{-1} := \{ghg^{-1}:h \in H\}
    \end{equation*}
    Observe that the orbits of the action partition the subgroups of $G$ into conjugacy classes.
\end{defn}


\begin{defn}
    Let $g \in G$, where $G$ is a group. The stabilizer of $g$ under the conjugation action of $G$ on itself is equal to \begin{equation*}
        Z(g) = \{h \in G: hgh^{-1} = g\}
    \end{equation*}
    and is called the \Emph{centralizer of $g$}. It follows that the center of $G$ is equal to \begin{equation*}
        Z(G) = \bigcap\limits_{g\in G}Z(g)
    \end{equation*}
\end{defn}


\begin{defn}
    Let $g \in G$. The equivalence class of $g$ with respect to the equivalence relation coming from the conjugation action of $G$ on itself is called the \Emph{conjugacy class of $g$ in $G$}, sometimes denoted $C(g)$; thus \begin{equation*}
        C(g) := \{g' \in G: \exists h \in G; hgh^{-1} = g' \}
    \end{equation*}
\end{defn}


\begin{defn}
    Let $H\leq G$ be a subgroup of $G$. The stabilizer of $H$ under the action of conjugation on $Sub(G)$ is \begin{equation*}
        N_G(H) = \{g \in G:gHg^{-1} = H\}
    \end{equation*}
    and it is called the \Emph{normalizer of $H$ in $G$}.
\end{defn}


\begin{defn}
    Let $G$ be a group and $S \subseteq G$ a subset of $G$. Let $g \in G$ and define $gSg^{-1} :=\{gsg^{-1}:s \in S\}$. Then $G$ acts on its power set $\mathcal{P}(G)$ of all subsets of itself by defining $g\cdot S = gSg^{-1}$ for any $g \in G$ and $S \in \mathcal{P}(G)$. 
\end{defn}

\begin{defn}
    Two subsets $S$ and $T$ of $G$ are said to be \Emph{conjugate} in $G$ if there is some $g \in G$ such that $T = gSg^{-1}$.
\end{defn}

\begin{prop}
    The number of conjugates of a subset $S$ in $G$ is the index of the normalizer of $S$, $|G:N_G(S)|$. Moreover, the number of conjugates of an element $s$ of $G$ is the index of the centralizer of $s$, $|G:Z(s)|$.
\end{prop}

\begin{namthm}[Class Equation (Alternate Form)]
    Let $G$ be a finite group and let $g_1,g_2,...,g_r$ be representatives of the distinct conjugacy classes of $G$ not contained in the center $Z(G)$ of $G$. Then \begin{equation*}
        |G| = |Z(G)| + \sum\limits_{i=1}^r|G:Z(g_i)|
    \end{equation*}
\end{namthm}
\begin{proof}
    Note that for $x \in G$, the conjugacy class of $x$ is the singleton $\{x\}$ if and only if $x \in Z(G)$, since then $gxg^{-1} = x$ for all $g \in G$. Let $Z(G) = \{1,z_2,...,z_m\}$, let $\mathcal{K}_1,...,\mathcal{K}_r$ be the conjugacy classes of $G$ not contained in the center, and let $g_i$ be a representative of $\mathcal{K}_i$ for each $i$. Then the full set of conjugacy classes of $G$ is given by \begin{equation*}
        \{1\},\{z_2\},...,\{z_r\},\mathcal{K}_1,...,\mathcal{K}_r
    \end{equation*}
    Since these partition $G$ we have \begin{align*} 
        |G| &= \sum\limits_{i=1}^m1 + \sum\limits_{i=1}^r|\mathcal{K}_i| \\
        &= |Z(G)| + \sum\limits_{i=1}^r|G:Z(g_i)|
    \end{align*}
    This proves the class equation.
\end{proof}

\begin{thm}
    If $p$ is a prime and $P$ a group of prime power order $p^{\alpha}$ for some $\alpha \geq 1$, then $P$ has a nontrivial center: $Z(P) \neq \{1\}$.
\end{thm}
\begin{proof}
    By the class equation \begin{equation*}
        |P| = |Z(P)| + \sum\limits_{i=1}^r|P:Z(g_i)|
    \end{equation*}
    By definition $Z(g_i) \neq P$ for $i \in \{1,2,...,r\}$, so $p$ divides $|P:Z(g_i)|$. Since $p$ also divides $|P|$ it follows that $p$ must divide $|Z(P)|$, hence the center must be nontrivial.
\end{proof}


\begin{cor}
    If $|P| = p^2$ for some prime $p$, then $P$ is abelian. More precisely, $P$ is isomorphic to either $\Z/p^2\Z$ or $\Z/p\Z\times \Z/p\Z$.
\end{cor}
\begin{proof}
    Since $|Z(P)| \neq 1$ by the previous theorem, it follows that $|Z(P)| \in \{p,p^2\}$. Thus $|P/Z(P)| \in \{1,2\}$, so $P/Z(P)$ is cyclic. Let $P/Z(P) = \langle gZ(P)\rangle$ for some $g \in P$. Let $x,y \in P$, then since $P/Z(P)$ is cyclic there exist $n,m \in \Z$ such that $xZ(P) = g^nZ(P)$ and $yZ(P) = g^mZ(P)$. That is, there exist $z,z' \in Z(P)$ such that $x = g^nz$ and $y = g^mz'$. It follows that \begin{equation*}
        xy = g^nzg^mz' = g^{n+m}zz' = g^mg^nz'z = g^mz'g^nz = yx
    \end{equation*}
    However, $x,y$ were arbitrary elements of $P$ so $P$ must be abelian. Hence $Z(P) = P$. If $P$ has an element of order $p^2$, then $P$ is cyclic and $P \cong \Z/p^2\Z$. Assume therefore that every nonidentity element of $P$ has order $p$. Let $x$ be a non-identity element of $P$ and let $y \in P\backslash\langle x\rangle$. Since $|\langle x,y \rangle| > |\langle x\rangle|=p$, we must have that $P = \langle x,y\rangle$. Both $x$ and $y$ have order $p$ so $\langle x\rangle\times \langle y\rangle \cong \Z/p\Z\times \Z/p\Z$. It now follows directly that the map $(x^a,y^b)\mapsto x^ay^b$ is an isomorphism from $\langle x\rangle\times \langle y\rangle$ onto $P$. This completes the proof.
\end{proof}

\subsection{\textsection Conjugation in Special groups}

\begin{rmk}
    Note that in the matrix group $\GL_n(\F)$, conjugation is equivalent to a change of basis: $A \mapsto PAP^{-1}$. An analogous situation arises in $S_n$.
\end{rmk}

\begin{prop}
    Let $\sigma$ and $\tau$ be elements of the symmetric group $S_n$, and suppose $\sigma$ has cycle decomposition \begin{equation*}
        (a_1\;a_2\;...\;a_{k_1})(b_1\;b_2\;...\;b_{k_2})...
    \end{equation*}
    Then $\tau\sigma\tau^{-1}$ has cycle decompossition: \begin{equation*}
        (\tau(a_1)\;\tau(a_2)\;...\;\tau(a_{k_1}))(\tau(b_1)\;\tau(b_2)\;...\;\tau(b_{k_2}))...
    \end{equation*}
\end{prop}
\begin{proof}
    (Left to the reader)
\end{proof}


\begin{defn}
    \leavevmode
    \begin{enumerate}
        \item If $\sigma \in S_n$ is the product of disjoint cycles of lengths $n_1,n_2,..., n_r$ with $n_1\leq n_2\leq ...\leq n_r$ (including its $1$-cycles) then the integers $n_1,n_2,...,n_r$ are called the \Emph{cycle type} of $\sigma$.
        \item If $n \in \Z^+$, a partition of $n$ is any nondecreasing sequence of positive integers whose sum is $n$.
    \end{enumerate}
\end{defn}


\begin{prop}
    Two elements of $S_n$ are conjugate in $S_n$ if and only if they have the same cycle type. The number of conjugacy classes in $S_n$ equals the number of partitions of $S_n$.
\end{prop}
\begin{proof}
    (Left to the reader - hint: D\& F p.126)
\end{proof}


\subsection{\textsection Right Group Actions}

\begin{defn}
    Let $G$ be a group and define the \Emph{right group action} of $G$ on a nonempty set $A$ as a map from $A\times G$ to $A$, denoted by $a\cdot g$ for $a \in A$ and $g\in G$, that satisfies the axioms: \begin{enumerate}
        \item $(a\cdot g_1)\cdot g_2 = a\cdot(g_1g_2)$ for all $a \in A$, and $g_1,g_2 \in G$, and 
        \item $a\cdot 1 = a$ for all $a \in A$.
    \end{enumerate}
\end{defn}

\begin{rmk}
    Conjugation as a write group action is denoted by $a \cdot g = g^{-1}ag$, and it is sometimes notated with $a\cdot g = a^g$.
\end{rmk}


\section{\textsection P-Groups}

\begin{defn}
    If $p$ is a prime, a group $G$ is called a \Emph{$p$-group} if and only if the order of every element of $G$ is a power of $p$.
\end{defn}

\begin{lem}
    If $G$ is a finite group and $p$ is a prime, then $|G|$ is a power of $p$ if and only if $G$ is a $p$-group.
\end{lem}
\begin{proof}
    Firstly, suppose $|G|$ is a power of $p$. Then for all $g \in G$, $o(g)\vert |G|$ by \ref{thmname:lagrange}, so in particular, $o(g)\vert p^n$ for some $n \in \N$, so $o(g) = p^k$ for $k \in \{0,1,...,n\}$. Thus, $G$ is a $p$-group.

    Conversely, suppose for all $g \in G$ $o(g)$ is a power of $p$. Assume towards a contradiction that $|G|$ is not a power of $p$. Then by prime factorization there exists a prime $q \neq p$ such that $q \vert |G|$. But, by \ref{thmname:cauchpthm} it follows that $G$ must have an element of order $q$, contrary to our assumption. Hence, $|G|$ is a power of $p$.
\end{proof}


\begin{thm}
    Let $K \vartriangleleft G$ be groups and let $p$ be a prime. Then $G$ is a $p$-group if and only if both $K$ and $G/K$ are $p$-groups.
\end{thm}
\begin{proof}
    Firstly suppose $G$ is a $p$-group. Then $K$ is a $p$-group as every element of $K$ is an element of $G$, and hence has order a power of $p$. Consider the canonical projection $\pi:G\rightarrow G/K$. It follows that for all $gK \in G/K$, $(gK)^{o(g)} = \pi(g)^{o(g)} = \pi(e) = K$, where $o(g)$ is a power of $p$ by assumption. It follows that $o(gK)\vert o(g)$, which is a power of $p$, so $o(gK)$ is a power of $p$. Consequently we conclude that $G/K$ is also a $p$-group.

    Converesely, suppose $K$ and $G/K$ are both $p$-groups. Let $g \in G$ and consider $gK \in G/K$. Then $o(gK) = p^l$ for some nonnegative integer $l$. Thus $(gK)^{p^l} = g^{p^l}K = K$, which implies $g^{p^l} \in K$. But $K$ is a $p$-group, so $o(g^{p^l}) = p^r$ for some nonnegative integer $r$. Consequently, we find $$e = (g^{p^l})^{p^r} = g^{p^{l+r}}$$
    Thus, we have that $o(g)\vert p^{l+r}$, so $o(g)$ is a power of $p$. As $g$ was an arbitrary element of $G$, we find that $G$ is indeed itself a $p$-group.
\end{proof}


\begin{thm}
    If $p$ is a prime and $G\neq \{1\}$ is a finite $p$-group, then $Z(G) \neq \{1\}$.
\end{thm}
\begin{proof}
    Let $a_1,...,a_n$ be representatives of the nonsingleton conjugacy classes in $G$. Because $1 \notin N(a_i)$ for each $i$, $N(a_i) \neq G$, and since $|G:N(a_i)|$ divides $|G|$ by \ref{thmname:lagrange}, it follows that $p$ divides $|G:N(a_i)|$ for each $i$. But then $p$ must divide $|Z(G)|$ by the class equation; in particular $Z(G) \neq \{1\}$.
\end{proof}

\begin{thm}
    If $G$ is a group and $|G| = p^2$ where $p$ is a prime, then $G$ is abelian and either $G\cong \Z/p^2\Z$ or $G\cong \Z/p\Z\times \Z/p\Z$.
\end{thm}
\begin{proof}
    By our previous result we know that $|Z(G)| \in \{p,p^2\}$. We aim to show that $|Z(G)| = p$ is impossible. Indeed, if it holds then $G/Z(G)$ is cyclic (being of order $p$), which applies $G$ is abelian and hence $|Z(G)| = p^2$, a contradiction. Thus $|Z(G)| = p^2$, so $Z(G) = G$ and $G$ is abelian. Now, if $G$ is cyclic $G \cong \Z/p^2\Z$ and we're done, so suppose to the contrary. Then for every $g \in G$ we have $g^p = 1$. Choose $1 \neq a \in G$ and write $H = \langle a \rangle$. Then choose $b \notin H$ and write $K = \langle b \rangle$. Because $|K| = p = |H|$, we have $H\cap K = \{1\}$, and consequently $|HK| = |H||K| = p^2$. Thus $|HK| = p^2 = |G|$, so $G = HK \cong H\times K\cong \Z/p\Z\times\Z/p\Z$.
\end{proof}


\begin{thm}
    Let $G$ be a finite $p$-group of order $p^n$. Then there exists a series \begin{equation*}
        G = G_0 \supset G_1\supset ...\supset G_n = \{1\}
    \end{equation*}
    of subgroups of $G$ such that $G_i \vartriangleleft G$, $|G_i| = p^{n-i}$, and $|G_i/G_{i+1}| = p$ for all $i \in \{0,1,...,n-1\}$.
\end{thm}
\begin{proof}
    If $n = 1$ then the statement is immediate, and we now proceed by induction on $n$. Suppose there exists $k \geq 1$ such that if $n = k$, then the hypothesis holds for all groups $G$ of order $|G| = p^k$. Now, consider $|G| = p^{k+1}$. From our previous results $Z(G) \neq \{1\}$. Moreover, by \ref{thmname:cauchpthm}, choose $a \in Z(G)$ such that $o(a) = p$, and write $G_k = \langle a \rangle$. Then $G_k\vartriangleleft G$ and $G/G_k$ has order $p^k$ so, by induction, let $(G/G_k)\supset X_1\supset ... \supset X_k = \{G_k\}$ be a series of subgroups of $G/G_k$ such that $X_i \vartriangleleft G/G_k$ and $|X_i/X_{i+1}| = p$ for each $i$. By the \ref{thmname:corrgroup} we have that each $X_i$ has the form $X_i = G_i/G_k$, where $G_i \vartriangleleft G$ and $|G_i/G_k| = p^{k-i}$. Furthermore, $X_i \supset X_{i+1}$ implies $G_i \supset G_{i+1}$, and $G_i/G_{i+1} \cong X_i/X_{i+1}$ by the third isomorphism theorem. Hence, $G\supset G_1\supset ... \supset G_k \supset \{1\}$ is the required series for $G$. Thus, by Mathematical Induction we conclude that the proposition holds for all $n \geq 1$.
\end{proof}


\section{\textsection Sylow's Theorem}

\begin{defn}
    Let $G$ be a group and let $p$ be a prime.
    \begin{enumerate}
        \item A group of order $p^{\alpha}$ for some $\alpha \geq 1$ is called a \Emph{$p$-group}. Subgroups of $G$ which are $p$-groups are called \Emph{$p$-subgroups}.
        \item If $G$ is a group of order $p^{\alpha}$, where $p\nmid m$, then a subgroup of order $p^{\alpha}$ is called a \Emph{Sylow $p$-subgroup} of $G$.
        \item The set of Sylow $p$-subgroups of $G$ will be denoted by $\operatorname{Syl}_p(G)$ and the numebr of Sylow $p$-subgroups of $G$ will be denoted by $n_p(G)$ (or just $n_p$ when $G$ is clear from context)
    \end{enumerate}
\end{defn}

\begin{namthm}[Sylow's Theorem]\label{thmname:syl}
    Let $G$ be a group of order $p^{\alpha}m$, where $p$ is a prime not dividing $m$.\begin{enumerate}
        \item Sylow $p$-subgroups of $G$ exist, i.e., $\operatorname{Syl}_p(G) \neq \emptyset$
        \item If $P$ is a Sylow $p$-subgroup of $G$ and $Q$ is any $p$-subgroup of $G$, then there exists $g \in G$ such that $Q\leq gPg^{-1}$, i.e., $Q$ is contained in some conjugate of $P$. In particular, any two Sylow $p$-subgroups of $G$ are conjugate in $G$.
        \item The number of Sylow $p$-subgroups of $G$ is of the form $1+kp$,i.e. \begin{equation*}
                n_p \equiv 1\mod p
        \end{equation*}
            Further, $n_p$ is the index in $G$ of the normalizer $N_G(P)$ for any Sylow $p$-subgroup $P$, hence $n_p$ divides $m$.
    \end{enumerate}
\end{namthm}

To establish the proof of this important claim, we first state and prove some preliminary lemmas.

\begin{lem}
    Let $P\in \operatorname{Syl}_p(G)$. If $Q$ is any $p$-subgroup of $G$, then $Q\cap N_G(P) = Q\cap P$.
\end{lem}
\begin{proof}
    Let $H = N_G(P) \cap Q$. Since $P \leq N_G(P)$, it is clear that $P\cap Q \leq H$, so we only have the reverse inclusion to prove. Since by definition $H \leq Q$, this is equivalent to showing $H \leq P$. Consider the subset $PH$ of $G$, containing both $P$ and $H$. Since $H \leq N_G(P)$, for all $h \in H$ and $k \in P$ $hkh^{-1} \in P$, so $hk = (hkh^{-1})h \in PH$, so $HP \subseteq PH$. Similarly, $kh = h(h^{-1}kh) \in HP$, so $PH \subseteq HP$. Then $HKHK = KHHK = KHK = HKK= HK$ and $(HK)^{-1} = K^{-1}H^{-1} = KH = HK$, so $HK$ is closed under the group operation and inversion so it is a subgroup as claimed. From another result we have that \begin{equation*}
        |PH| = \frac{|P||H|}{|P\cap H|}
    \end{equation*}
    where each term on the right is a power of $p$, so $PH$ is a $p$-group. Moreover, $P$ is a subgroup of $PH$ so the order of $PH$ is divisible by $p^{\alpha}$, the largest power of $p$ which divides $|G|$. These two facts force $|PH| = p^{\alpha} = |P|$, so in turn $P = PH$, since $P \leq PH$, and $H \leq P$. This establishes that $N_G(P)\cap Q = H = P\cap Q$.
\end{proof}

We can now prove the first point in Sylow's Theorem:

\begin{proof}[Sylow's Theorem 1.]
    We proceed by induction on $|G|$. If $|G| = 1$, there is nothing to prove. Assume inductively the existence of Sylow $p$-subgroups for all groups of order less than $|G| \geq 2$.

    If $p$ divides $|Z(G)|$, then by \ref{thmname:cauchpthm} $Z(G)$ has a subgroup, $N$, of order $p$. Let $\overline{G} = G/N$, so that $|\overline{G}| = p^{\alpha-1}m$. By induction, $\overline{G}$ has a subgroup $\overline{P}$ of order $p^{\alpha-1}$. If we let $P$ be the subgroup of $G$ containing $N$ such that $P/N =\overline{P}$, then $|P| = |P/N|\cdot|N| = p^{\alpha}$ and $P$ is a Sylow $p$-subgroup of $G$. We now must handle the case when $p$ does not divide $|Z(G)|$.

    Let $g_1,g_2,...,g_r$ be representatives of the distinct non-central conjugacy classes of $G$. The class equation for $G$ is \begin{equation*}
        |G| = |Z(G)| + \sum\limits_{i=1}^r|G:C_G(g_i)|
    \end{equation*}
    If $p\vert |G:C_G(g_i)|$ for all $i$, then since $p\vert |G|$, we would also have $p\vert|Z(G)|$, a contradiction. Thus, for some $i$, $p$ does not divide $|G:C_G(g_i)|$. FOr this $i$ let $H = C_G(g_i)$, so that \begin{equation*}
        |H| = p^{\alpha}, \;\text{ where }p\nmid k
    \end{equation*}
    Since $g_i \notin Z(G)$, $|H| < |G|$. By induction $H$ has a Sylow $p$-subgroup $P$, which of course is also a subgroup of $G$. Since $|P| = p^{\alpha}$, $P$ is a Sylow $p$-subgroup of $G$. THis compeltes the induction and establishes the first bullet of Sylow's Theorem.
\end{proof}

Before proving $2.$ and $3.$ of the Sylow Theorems, we perform some calculations. Note that we now know there exists a Sylow $p$-subgroup, $P$, of $G$. Let \begin{equation*}
    \{P_1,P_2,...,P_r\} = \mathcal{S}
\end{equation*}
be the set of all conjugates of $P$, and let $Q$ be \emph{any} $p$-subgroup of $G$. By definition of $\mathcal{S}$, $G$ and hence $Q$ acts by conjugation on $\mathcal{S}$. Write $\mathcal{S}$ as a disjoint union of orbits under this action by $Q$: \begin{equation*}
    \mathcal{S} = \mathcal{O}_1\coprod\mathcal{O}_2\coprod...\coprod\mathcal{O}_s
\end{equation*}
so $r = \sum_{i=1}^s|\mathcal{O}_i|$. Renumber the elements of $\mathcal{S}$ if necessary so that the first $s$ elements of $\mathcal{S}$ are representatives of the $Q$-orbits: $P_i\in\mathcal{O}_i, 1\leq i \leq s$. By the Orbit Stabilizer Theorem $|\mathcal{O}_i| = |Q:N_Q(P_i)|$. By definition, $N_Q(P_i) = N_G(P_i)\cap Q$, and by our previous Lemma $N_G(P_i)\cap Q = P_i\cap Q$. Combining these fact we obtain \begin{equation*}
    |\mathcal{O}_i| = |Q:P_i\cap Q|,\;\;\; 1 \leq i \leq s
\end{equation*}

We can now prove that $r \equiv 1\mod p$. Since $Q$ was an arbitrary $p$-subgroup, we may take $Q = P_1$ above, so that $|\mathcal{O}_1| =1$. Now, for all $i > 1$, $P_1 \neq P_i$, so $P_1 \cap P_i < P_1$. Then \begin{equation*}
    |\mathcal{O}_i| = |P_1:P_1\cap P_i| > 1,\;\; 2 \leq i \leq s
\end{equation*}
Since $P_1$ is a $p$-group, $|P_1:P_1\cap P_i|$ must be a power of $p$, so that $p\vert |\mathcal{O}_i|$ for all $2 \leq i \leq s$. Thus, \begin{equation*}
    r = \underbrace{|\mathcal{O}_1| + \sum\limits_{i=2}^s|\mathcal{O}_i|}_{1+p\vert thing} \equiv 1\mod p
\end{equation*}

We can now prove parts $2.$ and $3.$ of Sylow's Theorem:

\begin{proof}[Sylow's Theorem 1. and 2.]
    Let $Q$ be any $p$-subgroup of $G$. Suppose $Q$ is not contained in $P_i$ for any $i \in \{1,2,...,r\}$ (i.e. $Q\nleq gPg^{-1}$ for any $g \in G$). In this situation, $Q\cap P_i < Q$ for all $i$, so \begin{equation*}
        |\mathcal{O}_i| = |Q:Q\cap P_i| > 1,\;\;\; 1 \leq i \leq s
    \end{equation*}
    by our previous arguments. Thus $p\vert |\mathcal{O}_i|$ for all $i$, so $p$ divides $|\mathcal{O}_1|+...+|\mathcal{O}_s| = r$. This contradicts the fact that $r \equiv 1\mod p$. This contradiction proves $Q \leq gPg^{-1}$ for some $g \in G$.

    To see that all Sylow $p$-subgroups of $G$ are conjugate, let $Q$ be any Sylow $p$-subgroup of $G$. By the preceding argument, $Q\leq gPg^{-1}$ for some $g \in G$. Since $|gPg^{-1}| = |Q| = p^{\alpha}$, we must have $gPg^{-1} = Q$. This establishes part $2.$ of the theorem. Moreover, this shows that $n_p = r \equiv 1 \mod p$, which is the first part of $3.$

    Finally, since all Sylow $p$-subgroups are conjugate, we have that \begin{equation*}
        n_p = |G:N_G(P)| \;\;\text{ for any } P \in \operatorname{Syl}_p(G)
    \end{equation*}
    so $n_p\vert m$, completing the proof of Sylow's Theorem.
\end{proof}

We now also offer an alternate proof of Sylow's Theorem:

\begin{proof}[Sylow's Theorem Alternate]
    Let $G$ be a finite group such that $|G| = p^{\alpha}m$ for $p$ a prime and $p\nmid m$. 

    $(1)$ First, let $G$ act by translation on the set $\mathcal{J}$ of subset $J \subseteq G$, with $|J| = p^{\alpha}$. The number of such subsets is equal to $\binom{p^{\alpha}m}{p^{\alpha}}$, which is relatively prime to $p$. Thus, some orbit $\mathcal{O}_J$ must have size prime to $p$. Thus, $G_J$ (the stabilizer of $J$ in $G$) has order divisible by $p^{\alpha}$, since $|\mathcal{O}_J| = |G:G_J| = |G|/|G_J|$. But, $J$ is equivalent to the union of the set of right cosets of $G_J$, so $|G_J|\leq |J| = p^{\alpha}$. Therefore, as $p^{\alpha}\vert|G_J|$, $p^{\alpha}\leq |G_J|$, so we obtain that $|G_J| = p^{\alpha}$. Thus $G_J$ is a Sylow $p$-subgroup of $G$, proving existence.


    $(2)$ To be completed 
\end{proof}

Note that since conjugation is an automorphism on $G$, it is an isomorphism between subgroups which implies that every Sylow $p$-subgroup of $G$ is isomorphic.

\begin{cor}
    Let $P$ be a Sylow $p$-subgroup of $G$. Then the following are equivalent: \begin{enumerate}
        \item $P$ is the unique Sylow $p$-subgroup of $G$, i.e., $n_p = 1$
        \item $P$ is normal in $G$
        \item $P$ is characteristic in $G$
        \item All subgroups generated by elements of $p$-power order are $p$-groups, i.e., if $X$ is any subset of $G$ such taht $|x|$ is a power of $p$ for all $x \in X$, then $\langle X \rangle$ is a $p$-group.
    \end{enumerate}
\end{cor}
\begin{proof}
    If $1.$ holds, then $gPg^{-1} = P$ for all $g \in G$ since $gPg^{-1} \in \operatorname{Syl}_p(G)$. Conversely, if $P \vartriangleleft G$ and $Q \in \operatorname{Syl}_p(G)$, then by Sylow's Theorem there exists $g \in G$ such that $Q = gPg^{-1} = P$. Thus $\operatorname{Syl}_p(G) = \{P\}$, so we have $1. \iff 2.$.

    Since characteristic subgroups are normal, $3.$ implies $1.$. Conversely, if $P$ is the unique subgroup of $G$ of order $p^{\alpha}$, then $p$ is chracteristic in $G$ since the image of $P$ under any automorphism on $G$ is a subgroup of order $p^{\alpha}$. Thus we conclude $1.\iff 3.$.

    Finally, assume $1.$ holds and suppose $X$ is a subset of $G$ such that $|x|$ is a power of $p$ for all $x \in X$. By the conjugacy part of Sylow's Theorem, for each $x \in X$ there is some $g \in G$ such that $x \in gPg^{-1} = P$. Thus $X \subseteq P$, and so $\langle X\rangle \leq P$, and hence $\langle X\rangle$ is a $p$-group. Conversely, if $4.$ holds, let $X$ be the union of all Sylow $p$-subgroups of $G$. If $P$ is any Sylow $p$-subgroup, $P$ is any Sylow $p$-subgroup, $P$ is a subgroup of the $p$-group $\langle X\rangle$. Since $P$ is a $p$-subgroup of $G$ of maximal order, we must have $P = \langle X\rangle$, so $1.$ holds.
\end{proof}

\begin{eg}
    Let $G$ be a finite group and let $p$ be a prime.
    \begin{enumerate}
        \item If $p$ does not divide the order of $G$, the Sylow $p$-subgroup of $G$ is the trivial group, and all parts of Sylow's Theorem hold trivially. If $|G| = p^{\alpha}$, $G$ is the unique Sylow $p$-subgroup of $G$.
        \item A finite abelian group has a unique Sylow $p$-subgroup for each prime $p$. This subgroup consists of all elements $x$ whose order is a power of $p$. This is sometimes called the \Emph{$p$-primary component of the abelian group}.
        \item $S_3$ has three Sylow $2$-subgroups: $\langle (1\;2)\rangle,\langle (2\;3)\rangle$ and $\langle (1\;3)\rangle$. It has a unique Sylow $3$-subgroup: $\langle (1\;2\;3)\rangle = A_3$. Note that $3 \equiv 1 \mod 2$.
        \item $A_4$ has a unique Sylow $2$-subgroup: $\langle (1\;2)(3\;4), (1\;3)(2\;4)\rangle\cong V_4$. It has four Sylow $3$-subgroups: $\langle (1\;2\;3)\rangle,\langle (1\;2\;4)\rangle,\langle (1\;3\;4)\rangle$ and $\langle (2\;3\;4)\rangle$. Note that $4 \cong 1 \mod 3$.
    \end{enumerate}
\end{eg}


\subsection{Applications of Sylow's Theorem}

\begin{eg}[Groups of order $pq$, $p$ and $q$ primes with $p < q$]
    Suppose $|G| = pq$ for primes $p$ and $q$ with $p < q$. Let $P \in Syl_p(G)$ and let $Q \in Syl_q(G)$. We show that $Q$ is normal in $G$ and if $P$ is also normal in $G$, then $G$ is cyclic.

    Now, the three conditions: $n_q = 1+kq$ for some $k \geq 0$, $n_q$ divides $p$ and $p < q$, togethor force $k = 0$. Since $n_q = 1$, $Q \trianglelefteq G$.

    Since $n_p$ divides the prime $q$, the only possibility are $n_p = 1$ or $q$. In particular, if $p\nmid q-1$, (that is $q\not\equiv 1 \mod p$), then $n_p$ cannot equal $q$, so $P \trianglelefteq G$.

    Let $P = \langle x \rangle$ and $Q = \langle y \rangle$. If $P \trianglelefteq G$, then since $G/C_G(P)$ is isomorphic to a subgroup of $\aut(\Z/p\Z)$ and the latter group has order $p-1$, Lagrange's Theorem together with the observation that neither $p$ nor $q$ can divide $p-1$ implies that $G = C_G(P)$. In this case $x\in P \leq Z(G)$, so $x$ and $y$ commute. This means $|xy| = pq$, hence in this case $G$ is cyclic: $G\cong \Z/pq\Z$.
\end{eg}


\begin{eg}[Groups of order $30$]
    Let $G$ be a group of order $30$. We show that $G$ has a normal subgroup isomorphic to $\Z/15\Z$. Note that any subgroup of order $15$ is necessarily normal in $G$ since it is of index $2$, and cyclic by the preceding result, so it is only necessary to show there exists a subgroup of order $15$. 

    Let $P \in Syl_5(G)$ and let $Q \in Syl_3(G)$. If either $P$ or $Q$ is normal in $G$, $PQ$ is a subgroup of order $15$. Note also that if either $P$ or $Q$ is normal, then both $P$ and $Q$ are characteristic subgroups of $PQ$, and since $PQ \trianglelefteq G$, both $P$ and $Q$ are normal. Assume therefore that neither Sylow subgroup is normal. The only possibilities are $n_5 = 6$ and $n_3 = 10$. Each element of order $5$ lies in a Sylow $5$-subgroup, each Sylow $5$-subgroup contains $4$ nonidentity elements, and by Lagrange's Theorem, distinct Sylow $5$-subgroups intersect in the identity. Thus the number of elements of order $5$ in $G$ is the number of nonidentity elements in one Sylow $5$-subgroup times the number of Sylow $5$-subgroups. This would be $4\cdot 6 = 24$ elements of order $5$. By similar reasoning, the number of elements of order $3$ would be $2 \cdot 10 = 20$. This is absurd since a group of order $30$ cannot contain $24+20 = 44$ distinct elements. One of $P$ or $Q$ (hence both) must be normal i n$G$.
\end{eg}


\begin{eg}[Groups of order $12$]
    Let $G$ be a group of order $12$. We show that either $G$ has a normal Sylow $3$-subgroup, or $G \cong A_4$.

    Suppose $n_3 \neq 1$ and let $P \in Syl_3(G)$. Since $n_3\vert 4$ and $n_3 \cong 1 \mod 3$, it follows that $n_3 = 4$. Since distinct Sylow $3$-subgroups intersect in the identity and each contains two elements of order $3$, $G$ contains $2\cdot 4 = 8$ elements of order $3$. Since $|G:N_G(P)| = n_3 = 4$, $N_G(P) = P$. Now $G$ acts by conjugation on its four Sylow $3$-subgroups, so this action affords a permutation representation: \begin{equation*}
        \varphi:G\rightarrow S_4
    \end{equation*}
    The kernel $K$ of this action is the subgroup of $G$ which normalizes all Sylow $3$-subgroups of $G$. In particular, $K \leq N_G(P) = P$. Since $P$ is not normal in $G$ by assumption, $K = 1$, so $\varphi$ is injective and \begin{equation*}
        G \cong \varphi(G) \leq S_4
    \end{equation*}
    Since $G$ contains $8$ elements of order $3$ and there are precisely $8$ elements of order $3$ in $S_4$, all contained in $A_4$, it follows that $\varphi(G)$ intersects $A_4$ in a subgroup of order at least $8$. SInce both groups have order $12$ it follows that $\varphi(G) = A_4$, so that $G\cong A_4$.

    Note that $A_4$ has $4$ Sylow $3$-subgroups, so such a group $G$ does indeed exist. Also, letting $V$ be a Sylow $2$-subgroup of $A_4$, $|V| =4$ so it contains all the remaining elements of $A_4$. In particular, there cannot be another Sylow $2$-subgroup. Thus $n_2(A_4) = 1$, so $V\triangleleft A_4$.
\end{eg}

\begin{eg}[Groups of order $p^2q$, $p$ and $q$ distinct primes]
    Let $G$ be a group of order $p^2q$. We show that $G$ has a normal Sylow subgroup (for either $p$ or $q$). Let $P \in Syl_p(G)$ and let $Q \in Syl_q(G)$. 

    Consider first when $p > q$. Since $n_p\vert q$ and $n_p = 1+kp$, we must have $n_p = 1$. Thus $P \trianglelefteq G$.

    Consider now the case $p < q$. If $n_q = 1$, $Q$ is normal in $G$. Assume therefore that $n_q > 1$, i.e., $n_q = 1+tq$, for some $t > 0$. Now $n_q\vert p^2$ so $n_q = p$ or $p^2$. Since $q > p$ we cannot have $n_q = p$, hence $n_q = p^2$. Thus \begin{equation*}
        tq = p^2-1 = (p-1)(p+1)
    \end{equation*}
    Since $q$ is prime, either $q\vert p - 1$ or $q \vert p +1$. THe former is impossible since $q > p$, so the latter holds. Since $q > p$ but $q\vert p+1$, we must  have $q = p+1$. This forces $p = 2$, $q = 3$ and $|G| = 12$. This result now follows from the preceding example.
\end{eg}

\subsubsection{Groups of Order 60}

\begin{prop}
    If $|G| = 60$ and $G$ has more than one Sylow $5$-subgroup, then $G$ is simple.
\end{prop}
\begin{proof}
    Suppose by way of contradiction that $|G| = 60$ and $n_5 > 1$ but that there exists $H$ a normal subgroup of $G$ with $H \neq \{1\}$ or $G$. By Sylow's Theorem the only possibility for $n_5 = 6$. Let $P \in Syl_5(G)$, so that $|N_G(P)| = 10$ since its index is $n_5$.

    If $5\vert |H|$ then $H$ contains a Sylow $5$-subgroup of $G$ and since $H$ is normal, it contains all $6$ conjugates of this subgroup. In particular, $|H| \geq 1 + 6\cdot 4 = 25$, and the only possibility is $|H| = 30$. This leads to a contradiction since a previous example proved that any group of order $30$ has a normal (hence unique) Sylow $5$-subgroup. This argument shows $5$ does not divide $|H|$ for any proper normal subgroup $H$ of $G$.

    If $|H| = 6$ or $12$, $H$ has a normal, hence characteristic, Sylow subgroup, which is therefore also normal in $G$. Replacing $H$ by this subgroup if necessary, we may assume $|H| = 2,3$ or $4$. Let $\overline{G} = G/H$, so $|\overline{G}| = 30, 20$ or $15$. In each case, $\overline{G}$ has a normal subgroup $\overline{P}$ of order $5$ by previous results. If we let $H_1$ be the complete preimage of $\overline{P}$ in $G$, then $H_1 \trianglelefteq G$, $H_1 \neq G$ and $5\vert |H_1|$. This contradicts the preceding paragraph and so completes the proof.
\end{proof}

\begin{cor}
    $A_5$ is simple.
\end{cor}
\begin{proof}
    The subgroups $\langle (1\;2\;3\;4\;5)\rangle$ and $\langle (1\;3\;2\;4\;5)\rangle$ are distinct Sylow $5$-subgroups of $A_5$, so the result follows from the proposition.
\end{proof}

\begin{prop}
    If $G$ is a simple group of order $60$, then $G\cong A_5$.
\end{prop}
\begin{proof}
    Let $G$ be a simple group of order $60$, so $n_2 = 3,5$ or $15$. Let $P \in Syl_2(G)$ and let $N = N_G(P)$, so $|G:N| = n_2$.
    
    First observe that $G$ has no proper subgroup $H$ of index less than $5$, as follows: if $H$ were a subgroup of $G$ of index $4,3$ or $2$, then, $G$ would have a normal subgroup $K$ contained in $H$ with $G/K$ isomorphic to a subgroup of $S_4,S_3$ or $S_2$. Since $K \neq G$, simplicity forces $K = \{1\}$. This is impossible since $|G| = 60$ does not divide $4!$. This argument show, in particular, that $n_2 \neq 3$.

    If $n_2 = 5$, then $N$ has index $5$ in $G$ so the action of $G$ by left multiplication on the set of left cosets of $N$ gives a permutation representation of $G$ into $S_5$. Since the kernel of this representation is a proper normal subgroup and $G$ is simple, the kernel is $\{1\}$ and $G$ is isomorphic to a subgroup of $S_5$. Identify $G$ with this isomorphic copy so that we may assume $G\leq S_5$. If $G$ is not contained in $A_5$, then $S_5 = GA_5$ and, by the Second Isomorphism Theorem, $A_5\cap G$ is of index $2$ in $G$. Since $G$ has no normal subgroup of index $2$, this is a contradiction. This argument proves $G\leq A_5$. Since $|G| = |A_5|$, the isomorphic copy of $G$ in $S_5$ coincides with $A_5$, as desired.

    Finally, assume $n_2 = 15$. If for every pair of distinct Sylow $2$-subgroups $P$ and $Q$ of $G$, $P\cap Q = \{1\}$, then the number of nonidentity elements in Sylow $2-$subgroups of $G$ would be $(4-1)\cdot 15 = 45$. But $n_5 = 6$ so the number of elements of order $5$ in $G$ is $(5-1)\cdot 6 = 24$, accounting for $69$ elements. This contradiction proves that there exist distinct Sylow $2$-subgroups $P$ and $Q$ with $|P\cap Q| = 2$. Let $M = N_G(P\cap Q)$. Since $P$ and $Q$ are abelian (being groups of order $4$), $P$ and $Q$ are subgroups of $M$ and since $G$ is simple, $M \neq G$. Thus $4\vert |M|$ and $|M| > 4$. The only possibility is $|M| = 12$, i.e., $M$ has index $5$ in $G$. But now the argument of the preceding paragraph applied to $M$ in place of $N$ gives $G\cong A_5$. This leads to a contradiction in this case because $n_2(A_5) = 5$. The proof is complete.
\end{proof}
