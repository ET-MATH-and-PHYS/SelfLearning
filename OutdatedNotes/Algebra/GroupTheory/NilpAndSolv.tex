%%%%%%%%%% Nilpotent and Solvable %%%%%%%%%%
\chapter{\textsection\textsection Nilpotent and Solvable Groups}

\section{\textsection $p$-Groups}

\begin{defn}
    A \Emph{maximal subgroup} of a group $G$ is a proper subgroup $M$ of $G$ such that there are no subgroups $H$ of $G$ with $M < H < G$.
\end{defn}

Simply by order considerations we observe that any proper subgroup of a finite group is contained in a maximal subgroup. In contrast, infinite groups may or may not have maximal subgroups. 

\begin{thm}
    Let $p$ be a prime and let $P$ be a group of order $p^a$ for $a \geq 1$. Then \begin{enumerate}
        \item The center of $P$ is nontrivial: $Z(P) \neq \{1\}$,
        \item If $H$ is a nontrivial normal subgroup of $P$ then $H$ intersects the center non-trivially: $H \cap Z(P) \neq \{1\}$. In particular, every normal subgroup of order $p$ is contained in the center.
        \item If $H$ is a normal subgroup of $P$ then $H$ contains a subgroup of order $p^b$ that is normal in $P$ for each divisor $p^b$ of $|H|$. In partciular, $P$ has a normal subgroup of order $p^b$ for every $b \in \{0,1,...,a\}$.
        \item If $H < P$ then $H < N_P(H)$ (i.e.e, every proper subgroup of $P$ is a proper subgroup of its normalizer in $P$).
        \item Every maximal subgroup of $P$ is of index $p$ and is normal in $P$.
    \end{enumerate}
\end{thm}
\begin{proof}
    First note that $1.$ is a result previously proven using the class equation.

    Now, let $H$ be a nontrivial normal subgroup of $P$. Recall that for each conjugacy class $\mathcal{C}$ of $P$, either $\mathcal{C} \subseteq H$ or $\mathcal{C} \cap H = \emptyset$ because $H$ is normal. Pick representatives of the conjugacy classes of $P$, $a_1,...,a_r$, with $a_1,...,a_k \in H$ and $a_{k+1},...,a_r \notin H$. Let $\mathcal{C}_i$ be the conjugacy class of $a_i$ in $P$, for all $i$. Thus \begin{equation*}
        \mathcal{C}_i \subseteq H, 1 \leq i \leq k
    \end{equation*}
    and \begin{equation*}
        \mathcal{C}_i\cap H = \emptyset, k+1\leq i \leq r
    \end{equation*}
    By renumbering $a_1,...,a_k$ if necessary, we may assume $a_1,...,a_s$ represent classes of size $1$ and $a_{s+1},...,a_k$ represent classes of size $> 1$. Since $H$ is the disjoint union of these we have \begin{equation*}
        |H| = |H\cap Z(P)| + \sum\limits_{i=s+1}^k\frac{|P|}{|C_p(a_i)|}
    \end{equation*}
    Now $p$ divides $|H|$ and $p$ divides each term in the sum $\sum_{i=s+1}^k|P:C_p(a_i)|$ so $p$ divides their difference: $|H\cap Z(P)|$. This proves $H\cap Z(P) \neq \{1\}$. If $|H| = p$, since $H\cap Z(P) \neq \{1\}$ we must have $H \leq Z(P)$. 

    Next, we prove $3.$ by induction on $a$. If $a \leq 1$ or $H = \{1\}$, the result is immediate. Assume therefore that $a > 1$ and $H \neq \{1\}$. By part $2$, $H\cap Z(P) \neq 1$ so by Cauchy's Theorem $H\cap Z(P)$ contains a normal subgroup of $Z$ of order $p$. Then the quotient $P/Z$ has order $p^{a-1}$ and $H/Z\trianglelefteq P/Z$. By induction, for every nonnegative inter $b$ such that $p^b$ divides $|H/Z|$ there is a subgroup $K/Z$ of $H/Z$ of order $p^b$ that is normal in $P/Z$. If $K$ is the complete preimage of $K.Z$ in $P$ then $|K| = p^{b+1}$. The set of all subgroups of $H$ obtained by this process together with the identity subgroup provides a subgroup of $H$ that is normal in $P$ for each divisor of $|H|$. This establishes part $3.$.

    We prove $4.$ also by induction on $|P|$. If $P$ is abelian then all subgroups of $P$ are normal in $P$ and the result is immediate. We mau therefore assume $|P| > p$. Let $H$ be a proper subgroup of $P$. Since all elements of $Z(P)$ commute with all elements of $P$, $Z(P)$ normalizes every subgroup of $P$. By part $1.$ we have that $Z(P) \neq \{1\}$. If $Z(P)$ is not contained in $H$, then $H$ is properly contained in $\langle H,Z(P)\rangle$, and the latter subgroup is contained in $N_P(H)$ so $4.$ holds. We may therefore assume $Z(P) \leq H$. Since $P/Z(P)$ has smaller order than $P$, by induction $H/Z(P)$ is properly contained in $N_{P/Z(P)}(H/Z(P))$. It follows directly from the Lattice Isomorphism Theorem that $N_P(H)$ is the complete preimage in $P$ of $N_{P/Z(P)}(H/Z(P))$, hence we obtain proper containment of $H$ in its normalizer in this case as well. This completes the induction.

    To prove $5.$ let $M$ be a maximal subgroup of $P$. By definition, $M < P$ so by part $4.$, $M < N_P(M)$. By definition of maximality we must therefore have $N_P(M) = P$, so $M \triangleleft P$. The Lattice Isomorphism THeorem shows that $P/M$ is a $p$-group with no proper nontrivial subgroups because $M$ is a maximal subgroup. By part $3.$, however, $P/M$ has subgroups of every order dividing $|P/M|$. The only possibility is $|P/M| = p$. This proves $5.$ and completes the proof.
\end{proof}


\section{\textsection Nilpotent Groups}

\begin{defn}
    For any (finite or infinite) group $G$ define the following subgroups inductively: \begin{equation*}
        Z_0(G) = \{1\}, \hspace{15pt} Z_1(G) = Z(G)
    \end{equation*}
    and $Z_{i+1}(G)$ is the subgroup of $G$ containing $Z_i(G)$ such that \begin{equation*}
        Z_{i+1}(G)/Z_i(G) = Z(G/Z_i(G))
    \end{equation*}
    (i.e., $Z_{i+1}(G)$ is the complete preimage in $G$ of the center of $G/Z_i(G)$ under the natural projection). The resulting chain of subgroups \begin{equation*}
        Z_0(G) \leq Z_1(G) \leq Z_2(G) \leq ...
    \end{equation*}
    is called the \Emph{upper central series} of $G$.
\end{defn}


\begin{defn}
    A group $G$ is called \Emph{nilpotent} if $Z_c(G) = G$ for some $c \in \Z$. The smallest such $c$ is called the \Emph{nipotence class} of $G$.
\end{defn}


Each $Z_i(G)$ in this series is in fact characteristic in $G$.

\begin{rmk}
    \leavevmode
    \begin{enumerate}
        \item If $G$ is abelian then $G$ is nilpotent of class $1$ (provided $|G| > 1$), since in this case $G = Z(G) = Z_1(G)$. We can think of the heirarchy of structure as follows: \begin{equation*}
                cyclic\subset abelian \subset nipotent \subset solvable \subset all\;groups
        \end{equation*}
        \item For any finite group there must, by order considerations, be an integer $n$ such that \begin{equation*}
                Z_n(G) = Z_{n+1}(G) = Z_{n+2}(G) = ...
        \end{equation*}
            For example, $Z_n(S_3) = \{1\}$ for all $n \in \Z^+$. Once two terms in the upper central series  are the same, the chain stabilizes at that point. By definition, $Z_n(G)$ is a proper subgroup of $G$ for all $n$ for non-nilpotent groups.
        \item For infinite groups $G$ it may happen that all $Z_i(G)$ are proper subgroups of $G$ (so $G$ is not nilpotent) but \begin{equation*}
                G = \bigcup\limits_{i=1}^{\infty}Z_i(G)
        \end{equation*}
            Groups for which this hold are called \Emph{hypernilpotent}. Results that we shall derive which do not involve the notion of order, Sylow subgroups, etc. also hold for infinite groups.
    \end{enumerate}
\end{rmk}


\begin{prop}
    Let $p$ be a prime and let $P$ be a group of order $p^a$. Then $P$ is nilpotent of nilpotence class at most $a-1$.
\end{prop}
\begin{proof}
    For each $i \geq 0$, $P/Z_i(P)$ is a $p$-group so if $|P/Z_i(P)| > 1$, then $|Z(P/Z_i(P))| > 1$, which implies that $|Z_{i+1}(P)| > |Z_i(P)|$, and in particular if $P \neq Z_i(P)$, then $|Z_{i+1}(P)| \geq p|Z_i(P)|$. This implies inductively that $|Z_{i+1}(P)| \geq p^{i+1}$, so $|Z_a(P)| \geq p^a$. Hence, $P = Z_a(P)$ and $P$ is nilpotent with nilpotence class $\leq a$. The only way for $P$ to be of nilpotence class equal to $a$ is if $|Z_i(P)| = p^i$ for all $1 \leq i \leq a$. In this case, however, $Z_{a-2}(P)$ would have index $p^2$ in $P$, so $P/Z_{a_2}(P)$ would be abelian. But then $P/Z_{a-2}(P)$ would be its own center and $Z_{a-1}(P) = P$, a contradiction. Thus, the nilpotence class of $P$ is at most $a-1$.
\end{proof}


\begin{eg}
    $D_{2^{n-1}}$, the dihedral group of order $2^n$, is nilpotent of nilpotence class $n-1$. This can be proved inductively by showing $|Z(D_{2^{n-1}})| = 2$ and $D_{2^{n-1}}/Z(D_{2^{n-1}}) \cong D_{2^{n-2}}$ for $n \geq 3$. If $n$ is not a power of $2$, $D_n$ is not nilpotent.
\end{eg}


(To Be Continued)


\section{\textsection Composition Series and Solvable Groups}

The following proposition and proof shows how one can use the information on a normal subgroup $N$ and on the quotient $G/N$ to determine information about $G$:

\begin{prop}
    If $G$ is a finite abelian group and $p$ is a prime dividing $|G|$, then $G$ contains an element of order $p$.
\end{prop}
\begin{proof}
    The proof proceeds by induction on $|G|$, namely, we assume the result is valid for every group whose order is strictly smaller than the order of $G$ and then prove the rsult valid for $G$. Since $|G| > 1$, there is an element $x \in G$ with $x \neq 1$. If $|G| = p$ then $x$ has order $p$ by \ref{thmname:lagrange} and we are done. We may therefore assume $|G| > p$.

    Suppose $p$ divides $|x|$ and write $|x| = pn$. Then $|x^n| = p$, and again we have an element of order $p$. We may therefore assume $p$ does not divide $|x|$.

    Let $N = \langle x\rangle$. Since $G$ is abelian, $N \trianglelefteq G$. By \ref{thmname:lagrange}, $|G/N| = |G|/|N|$ and since $N \neq 1$, $|G/N| < |G|$. Since $p$ does not divide $|N|$, we must have $p\vert |G/N|$. We can now apply the induction assumption to the smaller group $G/N$ to conclude it contains an element $yN$ of order $p$. Since $y \notin N$, but $y^p \in N$, we must have $\langle y^p\rangle \neq \langle y\rangle$, that is, $|y^p| < |y|$. Since $|y^p| = \frac{|y|}{\gcd(p,|y|)}$, we must have that $p\vert |y|$. We are now in the situation described in the preceding paragraph, so that the argument again produces an element of order $p$. The induction is complete.
\end{proof}

Note that simple groups, groups without any normal subgroups, are fundamental obstructions to this variety of proof. As simple groups cannot be ``factored" into pieces like $N$ and $G/N$, the role they play is analogous to that of primes in the arithmetic of $\Z$. 

\begin{defn}
    In a group $G$, a sequence of subgroups \begin{equation*}
        1 = N_0 \leq N_1 \leq N_2 \leq ... \leq N_{k-1} \leq N_k = G
    \end{equation*}
    is called a \Emph{composition series} if $N_i \trianglelefteq N_{i+1}$ and $N_{i+1}/N_i$ is a simple group, $0 \leq i \leq k-1$. If the above sequence is a composition series, the quotient groups $N_{i+1}/N_i$ are called \Emph{composition factors} of $G$.
\end{defn}

As an example, two composition series of $D_4$ are \begin{equation*}
    1\trianglelefteq \langle s \rangle \trianglelefteq \langle s,r^2\rangle \trianglelefteq D_4\;\;and\;\;1 \trianglelefteq \langle r^2\rangle \trianglelefteq \langle r \rangle \trianglelefteq D_4
\end{equation*}


\begin{namthm}[Jordan-H\"{o}lder]\label{thmname:JHseries}
    Let $G$ be a finite group with $G \neq \{1\}$. Then \begin{enumerate}
        \item $G$ has a composition series
        \item The composition factors in a composition series are unique, namely, if $1 = N_0 \leq N_1 \leq ... \leq N_r = G$ and $1 = M_0 \leq M_1 \leq ... \leq M_s = G$ are two composition series for $G$, then $r = s$ and there is some permutation $\pi$ of $\{1,2,...,r\}$ such that \begin{equation*}
                M_{\pi(i)}/M_{\pi(i)-1}\cong N_i/N_{i=1}, 1 \leq i \leq r
        \end{equation*}
    \end{enumerate}
\end{namthm}
\begin{proof}
    (To be completed)
\end{proof}


\begin{defn}
    A group $G$ is \Emph{solvable} if there is a chain of subgroups \begin{equation*}
        1 = G_0 \trianglelefteq G_1 \trianglelefteq G_2 \trianglelefteq ... \trianglelefteq G_s = G
    \end{equation*}
    such that $G_{i+1}/G_i$ is abelian for $i \in \{0,1,...,s-1\}$.
\end{defn}

A property of finite solvable groups is the following due to Philip Hall: 

\begin{thm}
    The finite group $G$ is solvable if and only if for every divisor $n$ of $|G|$ such that $gcd\left(n,\frac{|G|}{n}\right) = 1$, $G$ has a subgroup of order $n$.
\end{thm}

Another illustration of how using information on a normal subgroup $N$ and a quotient group $G/N$ is seen in the following result:

\begin{prop}
    If $N$ and $G/N$ are solvable, then $G$ is solvable.
\end{prop}
\begin{proof}
    Let $1 = N_0 \trianglelefteq N_1 \trianglelefteq ... \trianglelefteq N_n = N$ be a chain of subgroups of $N$ such that $N_{i+1}/N_i$ is abelian, $0 \leq i < n$, and $1_{G/N} = G_0/N\trianglelefteq G_1/N\trianglelefteq ... \trianglelefteq G_m/N = G/N$ be achain of subgroups of $G/N$ such that $(G_{i+1}/N)/(G_i/N)$ is abelian, $0 \leq i < m$. Such $G_i$ exist by the Lattice Isomorphism Theorem, with $N \leq G_i$ for each $i$. By the Third Isomorphism Theorem \begin{equation*}
        (G_{i+1}/N)/(G_i/N) \cong G_{i+1}/G_i
    \end{equation*}
    Thus \begin{equation*}
        1 = N_0 \trianglelefteq N_1 \trianglelefteq ... \trianglelefteq N_n = N = G_0 \trianglelefteq G_1 \trianglelefteq ... \trianglelefteq G_m = G
    \end{equation*}
    is a chain of subgroups of $G$ all of whose successive quotient groups are abelian. This proves $G$ is solvable.
\end{proof}





\subsection{The H\"{o}lder Program}

The holder program has two goals: \begin{enumerate}
    \item Classify all finite simple groups
    \item Find all ways of ``putting simple groups together" to form other groups
\end{enumerate}

The classification of finite simple groups was completed in $1980$, resulting in a proof of the following result:

\begin{thm}
    There is a list consisting of $18$ (infinite) families of simple groups and $26$ simple groups not belonging to these families (the \Emph{sporadic} simple groups) such that every finite simple group is isomorphic to one of the groups in this list.
\end{thm}

One such family is $\{\Z/p\Z\vert p\text{ a prime}\}$. The ``extension problem" is one of a much higher difficulty, even for groups of relatively small order.

