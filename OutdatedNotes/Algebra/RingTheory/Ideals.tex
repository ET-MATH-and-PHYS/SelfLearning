%%%%%%%%%% Ideals %%%%%%%%%%
\chapter{\textsection\textsection Ideals and Quotient Rings}

\section{\textsection Basic Definitions and Examples: Ideals}


\begin{rmk}
    Let $f:R\rightarrow R'$ be a ring homomorphism, then $\ker(f)$ is a subgroup of $(R,+)$. Moreover, for all $r_1,r_2 \in R$ and all $k \in \ker(f)$, $r_1kr_2 \in \ker(f)$. Indeed, $f(k) = 0_{R'}$ so $f(r_1kr_2) = f(r_1)f(k)f(r_2) = f(r_1)0_{R'}f(r_2) = 0_{R'}$.
\end{rmk}

\begin{defn}
    Let $R$ be a ring. A subgroup $I$ of $(R,+)$ is called an \Emph{ideal of $R$} if for all $r_1,r_2 \in R$, and all $i \in I$, \begin{equation}
        r_1ir_2 \in I
    \end{equation}
     \begin{enumerate}
         \item[$\drsh$] (This is actually the definition of a two-sided ideal, and the definitions of left and right sided ideals can be derived by relaxing the condition in this definition) 
     \end{enumerate}
\end{defn}

\begin{eg}
    \leavevmode
    \begin{enumerate}
        \item For an arbitrary ring $R$, $\{0\}$ and $R$ are ideals of $R$.
        \item If $R$ is a commutative ring, then \begin{equation}
            aR := \{ar \in R: r \in R\}
        \end{equation}
        is an ideal of $R$ for all $a \in R$. $I$ is called the \Emph{principal ideal generated by a}, and is denoted $(a)$.
        \item $\ker(\ev_2:\R[x]\rightarrow \R)$ is an ideal of $\R[x]$. Indeed, $\ker(\ev_2) = (x-2)$.
        \item For a ring $R$, the ideal generated by a subset $X \subseteq R$ is given by \begin{equation}
            (X) := \left\{\sum\limits_{i=1}^na_ix_ib_i: n \geq 1, a_i,b_i \in R, x_i \in X\right\}
        \end{equation}
        \item Every ideal of $\Z$ is principal. Indeed, every ideal is a subgroup of $(\Z,+)$, and every subgroup of $(\Z,+)$ is cyclic so every ideal is principal. Moreover, every subgroup is an ideal. Indeed, we know that $n\Z$ is an ideal for all $n \in \Z$, but $n\Z$ is precisely the form for subgroups of $\Z$, so all subgroups are principal ideals of $\Z$ and all ideals of $\Z$ are principal.
        \item The ideal $(2,X)$ of $\Z[X]$ is not a principal ideal.
        \begin{proof}
            For the sake of contradiction suppose $(2,X) = (p)$ for some $p \in \Z[X]$. Then $2 = pf$ for some $f \in \Z[X]$. Then $\deg(p) + \deg(f) \leq 0$, so $\deg(p) = 0$. Then $p = n \in \Z$ such that $n\;\vert\;2$. Hence, $p = 2$ or $p = 1$. But, $1 \notin (2,X)$ since $2 \;\cancel{\vert}\;1$ and $X\;\cancel{\vert}\;1$ in $\Z$. Thus, $p = 2$. However, $2\;\cancel{\vert}\;X$ in $\Z[X]$, which contradicts the assumption that $(2,X) = (p)$. Thus, $(2,X)$ can not be principal in $\Z[X].$
        \end{proof}
    \end{enumerate}
\end{eg}


\begin{defn}
    Let $I \subseteq R$ be an ideal of a ring $R$. The quotient group $R/I$, for the additive group $(R,+)$, is a ring for the multiplication \begin{equation}
        (a+I)(b+I) = ab+I \tag{$\star$}\label{eq:quot_ring}
    \end{equation}
\end{defn}

\begin{thm}
    The addition on $R/I$ and the multiplication given by (\ref{eq:quot_ring}) makes $R/I$ into a ring such that the canonical quotient map \begin{equation}
        \map{\pi: R \twoheadrightarrow R/I}{r\mapsto r+I}
    \end{equation}
    is a surjective ring homomorphism of kernel $I$. We call $R/I$ the \Emph{quotient ring} of $R$ by \\$I$.
    \begin{proof}
        (Left to the reader)
    \end{proof}
\end{thm}

\begin{thm}
        Any ideal $I$ is the kernel of a natural ring homomorphism $R \rightarrow R/I$, where $R/I$ is the quotient ring, taking $a \mapsto a+I$. 
\end{thm}

\begin{rmk}
    If $R$ is a commutative ring then so is $R/I$ for all ideals $I$ of $R$.
\end{rmk}

\begin{eg}
        Consider $R = \Z/4\Z$. Then some ideals of $R$ are, $I = (0),(1),(2) = \{0,2\}$. In general, for $R = \Z/n\Z$ we have the ideal $(d)$ for all $d \;\vert\;n$. In particular, if $R = \Z/p^k\Z$, then the distinct ideals are $$(1)\supset(p)\supset(p^2)\supset...\supset(p^k)=(0)$$
\end{eg}


\begin{rmk}
        In general the set of ideals form a lattice for the ring.
\end{rmk}


\begin{eg}
    \leavevmode
    \begin{enumerate}
        \item For any ring $R$, $R/(0) \cong R$ and $R/R \cong \{0\}$.
        \item $\Z/n\Z$, where $n\Z = (n)$, is the ring of integers modulo n.
        \begin{enumerate}
            \item[$\drsh$] By the \ref{thmname:corrring}, we have that ideals of $\Z$ containing $n\Z$ correspond to ideals of $\Z/n\Z$, which is to say, all ideals of the form $m\Z$ for $m \;\vert\;n$.
        \end{enumerate}
        \item Let $R$ be a commutative ring. Then $\ev_a:R[x]\rightarrow R$ for $a \in R$ is a surjective ring homomorphism, so by the First Isomorphism Theorem for rings and the fact that $\ker(\ev_a) = (x-a)$, we have that \begin{equation}
            \overline{\ev_a}: R[x]/(x-a) \xrightarrow{\sim} R
        \end{equation}
        is a ring isomorphism. Note by the \ref{thmname:corrring}, ideals of $R[x]$ containing $(x-a)$ correspond to ideals of $R$. If $R = \F$ a field, then the only ideals of $\F$ are $\{0\}$ and $\F$, which implies $(x-a)$ is \Emph{maximal}. Thus, the only ideals of $\F[x]$ containing $(x-a)$ are $(x-a)$ and $\F[x]$.
        \item For $X$ a set and $z \in X$, $$\ev_z:F(X,R) \rightarrow R$$
        is a surjective ring homomorphism of kernel $$\ker(\ev_z) = \{f:X\rightarrow R\vert f(z) = 0_R\}$$
    \end{enumerate}
\end{eg}

\begin{defn}
    Let $I$ be an ideal of $R$, $a,b \in R$. Then $a + I \in R/I$ is called the \Emph{(congruence) class} of $a$ modulo $I$ (sometimes denoted $a \mod I$). If $a + I = b + I$, so $b - a \in I$, we say that $a$ and $b$ are \Emph{congruent modulo $I$}, written \begin{equation}
        a \equiv b \mod I, a = b \mod I, a = b (I)
    \end{equation}
\end{defn}

\subsection{\textsection Simple Ideals}

\begin{defn}
    A ring $R$ is \Emph{simple} if $R \neq \{0\}$ and if the only ideals of $R$ are $\{0\}$ and $R$. That is, $R$ has exactly two ideals.
\end{defn}

\begin{eg}
    \leavevmode
    \begin{enumerate}
        \item Division rings are simple. If $a \neq 0$ in an ideal $I$ of a division ring $R$, then $1 = a^{-1}a \in I$, so $I = (1) = R$. In particular, fields are simple rings.
        \item $M_n(R)$ for $R$ an arbitrary ring is simple if and only if $R$ is simple.
        \begin{proof}
            Suppose $R$ is simple and let $\mathcal{A}$ be an ideal of $M_n(R)$. Then by the following lemma $\mathcal{A} = M_n(A)$ for some ideal $A$ of $R$. Thus, $A = \{0\}$ or $A = R$, so $\mathcal{A} = M_n(0) = \{0\}$ or $\mathcal{A} = M_n(R)$. Consequently, $M_n(R)$ is simple. Now, suppose $M_n(R)$ is simple and let $A$ be an ideal of $R$. Then $M_n(A)$ is an ideal of $M_n(R)$. Hence, $M_n(A) = \{0\}$ or $M_n(A) = M_n(R)$. Thus, $A = \{0\}$ or $A = R$, so $R$ is a simple ring as claimed.
        \end{proof}
    \end{enumerate}
\end{eg}

\begin{lem}
    For a ring $R$, every ideal of $M_n(R)$ has the form $M_n(A)$ for an ideal $A$ of $R$.
\end{lem}

\begin{prop}
    A commutative ring $R$ is simple if and only if it is a field.
    \begin{proof}
        If $R$ is a field then it is simple by the previous example. Now, suppose $R$ is a commutative simple ring, and let $0 \neq r \in R$. Consider the ideal $(r) = rR$. Since $R$ is simple, $(r) = R$ since $(r) \neq \{0\}$. Hence, $1 \in (r)$, so there exists $r' \in R$ such that $rr' = 1$. Hence, $r$ is a unit of $R$, so in particular $R$ is a field.
    \end{proof}
\end{prop}

\subsection{\textsection Maximal and Prime Ideals}

\begin{defn}
    An ideal $I \subseteq R$ in a ring $R$ is called \Emph{maximal} if \begin{enumerate}
        \item $I \neq R$
        \item For all ideals $J \subseteq R$, if $I \subseteq J$, then either $I = J$ or $J = R$.
    \end{enumerate}
\end{defn}

\begin{prop}
    An ideal $I \subseteq R$ is maximal if and only if the quotient ring $R/I$ is simple.
    \begin{proof}
        Let $R$ be a ring with ideal $I$. 
        
        $\implies$ Suppose $I$ is maximal. Then by the \ref{thmname:corrring} the only ideals of $R/I$ are $\{0_{R/I}\}$ and $R/I$ corresponding to $I$ and $R$ respectively. Moreover, since $I \neq R$ $R/I \neq \{0\}$. Thus, we have that $R/I$ is simple.
        
        $\impliedby$ Suppose $R/I$ is a simple ring and let $I \subseteq J \subseteq R$ be an ideal containing $I$. Then by the \ref{thmname:corrring} we have that $I/I \subseteq J/I \subseteq R/I$ is an ideal of $R/I$. However, as $R/I$ is simple $J/I = I/I$ or $J/I = R/I$. By the bijectivity of the correspondence, $J = I$ or $J = R$. Hence, $I$ is a maximal ideal in $R$ as claimed.
    \end{proof}
\end{prop}

\begin{cor}
    If $R$ is commutative, then an ideal $I \subseteq R$ is maximal if and only if $R/I$ is a field.
\end{cor}

\begin{eg}
    \leavevmode
    \begin{enumerate}
        \item If $F$ is a field, then $\{0\}$ is the only maximal ideal of $F$.
        \item If $p$ is a prime number, then $p\Z \subseteq \Z$ is a maximal ideal. (indeed $\Z/p\Z$ is a field)
        \begin{enumerate}
            \item[$\drsh$] Actually, for $n \geq 0$, $n\Z \subseteq \Z$ is a maximal ideal if and only if $n$ is prime
        \end{enumerate}
        \item In $F[x]$ for $F$ a field, the ideal $(x-a)$ is maximal for all $a \in F$, because $F[x]/(x-a) \cong F$.
    \end{enumerate}
\end{eg}

\begin{defn}
    Let $R$ be a commutative ring. Then an ideal $I$ of $R$ is \Emph{prime} if \begin{enumerate}
        \item $I \subsetneq R$
        \item For all $r_1,r_2 \in R$, if $r_1r_2 \in I$ then either $r_1 \in I$ or $r_2 \in I$.
    \end{enumerate}
\end{defn}

\begin{eg}
    \begin{enumerate}
        \item A commutative ring $R$ is an integral domain if and only if $\{0\}$ is a prime ideal.
        \begin{enumerate}
            \item[$\drsh$] Indeed, $R$ is a commutative integral domain $\iff$ $R \neq \{0\}$ and whenever $ab = 0$, either $a = 0$ or $b = 0$ $\iff$ $R \neq \{0\}$ and whenever $ab \in \{0\}$, $a \in \{0\}$ or $b \in \{0\}$ $\iff$ $\{0\}$ is a prime ideal of $R$. 
        \end{enumerate}
        \item $p\Z \subset \Z$ is a prime ideal for $p$ a prime number. Indeed, $p\Z \neq \Z$ and $p \;\vert\;ab$ implies $p \;\vert \;a$ or $p\;\vert\;b$.
    \end{enumerate}
\end{eg}

\begin{prop}
    Let $R$ be a commutative ring with ideal $I \subseteq R$. Then $I$ is prime if and only if $R/I$ is an integral domain.
    \begin{proof}
        Indeed, $I$ is prime $\iff$ $I \subsetneq R$ and $ab \in I$ implies $a \in I$ or $b \in I$ $\iff$ $R/I \neq \{0\}$ and $ab+I = I$ implies $a + I = I$ or $b + I = I$ $\iff$ $R/I$ is an integral domain.
    \end{proof}
\end{prop}

\begin{cor}
    Every maximal ideal of a commutative ring $R$ is a prime ideal.
\end{cor}

\begin{eg}
    \leavevmode
    \begin{enumerate}
        \item $(x) \subset \Z[x]$ is a prime ideal. Indeed, $\Z[x]/(x) \cong \Z$ is an integral domain (not a field, so $(x)$ is not maximal)
        \item $(p) \subset \Z[x]$ for a prime number $p$ is a prime ideal which is not maximal. Consider \begin{equation}
            \map{\Z[x] \xrightarrow{f} {\Z/p\Z[x]}}{\sum_ia_ix^i\mapsto \sum_i[a_i]_px^i}
        \end{equation}
        Then $f$ is a surjective ring homomorphism and $\ker(f) = (p)$. So, by the first isomorphism theorem $f$ induces a ring isomorphism \begin{equation}
            \overline{f}: \Z[x]/(p) \xrightarrow{\sim} \Z/p\Z[x]
        \end{equation}
        Since $\Z/p\Z[x]$ is an integral domain $(p)$ is a prime ideal, but $\Z/p\Z[x]$ is not a field, so $(p)$ is not maximal.
    \end{enumerate}
\end{eg}


\section{\textsection Ideal Arithmetic and the Chinese Remainder Theorem}

\begin{defn}
    Let $R$ be a ring , and $I,J$ ideals of $R$. We define their sum as \begin{equation}
        I + J := \{i+j:i\in I, j \in J\}
    \end{equation}
    Then $I + J$ is an ideal of $R$. Next, define their product \begin{equation}
        IJ := \langle ij: i \in I, j \in J\rangle = \left\{\sum\limits_{k=1}^ni_kj_k: n \geq 1, i_k \in I, j_k \in J, \forall1 \leq k \leq n\right\}
    \end{equation}
    Then $IJ$ is an ideal of $R$ as well. Recall $I \cap J$ is also an ideal of $R$
\end{defn}

\begin{eg}
    For $R = \Z$ and $m,n > 0$, we have \begin{align}
        m\Z + n\Z &= \gcd(m,n) \\
        m\Z\cdot n\Z &= mn\Z \\
        m\Z \cap n\Z &= \text{lcm}(m,n)\Z
    \end{align}
    So, if $\gcd(m,n) = 1$, then $$m\Z\cdot n\Z = m\Z \cap n\Z$$
    because $mn = \text{lcm}(m,n)\gcd(m,n)$.
\end{eg}

\begin{rmk}
    Let $I,J, K$ be ideals of a ring $R$. Then \begin{enumerate}
        \item $(I+J)K = IK + JK$
        \item $K(I+J) = KI + KJ$
        \item $IR = I = RI$
    \end{enumerate}
    \begin{proof}
        Let $I,J,K$ be ideals of a ring $R$
        
        \textbf{1.} Let $(i+j)k \in (I+J)K$. Then by distributivity $(i+j)k = ik+jk \in IK+JK$. Similarly, for all $ik+jk \in IK+JK$, $ik+jk = (i+j)k \in (I+J)K$. Thus, we have that $(I+J)K = IK+JK$.
        
        \textbf{2.} This statement is identical to 1., replacing right distributivity with left distributivity.
        
        \textbf{3.} Note that for all $i \in I$ and all $r \in R$, $ir,ri \in I$ since $I$ is an ideal, and $i = 1\cdot i \in RI$, $i = i\cdot 1 \in IR$ using the fact that $R$ is unital. Hence, $RI = I  = IR$, completing the proof.
    \end{proof}
    \label{idealProps}
\end{rmk}

\begin{namthm}[Chinese Remainder Theorem (CRT)]
    Let $R$ be a ring, and $I,J$ ideals of $R$. Assume that $I + J = R$ ($I$ and $J$ are said to be \Emph{relatively prime}). Then the ring homomorphism \begin{equation}
        \map{R\xrightarrow{\alpha} R/I \times R/J}{r \mapsto (r+I, r+J)}
    \end{equation}
    is surjective of kernel $I \cap J$. Consequently, by the first isomorphism theorem for rings we have an isomorphism \begin{equation}
        \overline{\alpha}: R/I\cap J \xrightarrow{\sim} R/I \times R/J
    \end{equation}
    Moreover, if $R$ is commutative, then $I+J = R$ implies that $I \cap J = IJ$.
\end{namthm}
\begin{proof}
    Suppose $R$ is a ring with relatively prime ideals $I, J$. Define a map $$\map{R\xrightarrow{\alpha} R/I \times R/J}{r \mapsto (r+I, r+J)}$$
    Let $r,r' \in R$. Then, observe that \begin{align*}
        \alpha(r+r') &= (r+r' + I, r+r' + J) & \alpha(rr') &= (rr' + I, rr' + J) \\
        &= (r+I+r'+I,r+J + r'+J) & &= ((r+I)(r'+I),(r+J)(r'+J)) \\
        &= (r+I,r+J)+(r'+I,r'+J) & &= (r+I,r+J)(r'+I,r'+J) \\
        &= \alpha(r) + \alpha(r') & &= \alpha(r)\alpha(r')
    \end{align*}
    and $$\alpha(1_R) = (1_R + I, 1_R+J) = (1_{R/I},1_{R/J})$$
    so $\alpha$ is a ring homomorphism. To show $\alpha$ is surjective, let $a,b \in R$. We want to find $r \in R$, $i \in I$, and $j \in J$ such that $r+i = a$ and $r+j = b$. But, we then have that $r = a-i = b-j$, so $a-b = i-j$. Note that $I+J = R$ since they are relatively prime, so there exist $i' \in I$ and $j' \in J$ such that $a-b = i' + j'$. Set $i = i'$ and $j = -j'$. Then, observe that $$\alpha(r) = (r+I,r+J) = (r+i + I, r+j + J) = (a+I, b+J)$$
    Therefore $\alpha$ is a surjective ring homomorphism. Observe that $I\cap J \subseteq \ker(\alpha)$ since for all $k \in I \cap J$, $\alpha(k) = (k+I,k+J) = (I,J)$. Then, let $t \in \ker(\alpha)$. Observe that then $(I,J) = \alpha(t) = (t+I,t+J)$, so by definition $t \in I$ and $t \in J$. Thus, $t \in I \cap J$, so $\ker(\alpha) \subseteq I \cap J$. Consequently $\ker(\alpha) = I \cap J$. By the first isomorphism theorem for rings we have our desired result. 
    
    
    Now, suppose $R$ is commutative. Then observe that \begin{align*}
        I \cap J &= (I\cap J)R \\
        &= (I\cap J)(I+J) \\
        &= (I\cap J)I + (I\cap J)J \\
        &\subseteq JI + IJ \\
        &= IJ + IJ \\
        &= IJ
    \end{align*}
    Moreover, $IJ \subseteq I \cap J$ since $IJ$ is generated by $ij$ for $i \in I$ and $j \in J$. However, since $I$ and $J$ are ideals $ij \in I$ and $ij \in J$, so in particular $ij \in I \cap J$. Thus, we conclude that $IJ = I\cap J$.
\end{proof}

\begin{lem}
    If $I_1, I_2,..., I_n$ are pairwise relatively prime ideals of $R$, with $n \geq 2$, then for all $i \in \{1,2,...,n\}$ $\bigcap_{j\neq i} I_j$ and $I_i$ are relatively prime. That is, $\bigcap_{j\neq i} I_j + I_i = R$.
    \begin{proof}
        Let $I_1, I_2,..., I_n$ be as in the statement, for $n \geq 2$. If $n = 2$ we are done by assumption. Then, if $n = 3$ there exist $i_1 \in I_{j_1}$, $i_2 \in I_{j_2}$, $i_3,i_3' \in I_i$ such that $i_1+i_3 = 1$, $i_2+i_3' = 1$, where $i \in \{1,2,3\}$. Then $1 = (i_1+i_3)(i_2+i_3') = i_1i_2 + i_3i_2 + i_1i_3' + i_3i_3'$, where $i_1i_2 \in I_{j_1} \cap I_{j_2}$, and $i_3i_2,i_1i_3',i_3i_3' \in I_i$ since they are ideals. Thus, $1 \in I_{j_1}\cap I_{j_2} + I$, so $I_{j_1} \cap I_{j_2} + I_i = R$, since it is an ideal. Hence, the base cases hold. Now, suppose there exists $k \geq 3$ such that if $n = k$, $I_{j_1}\cap I_{j_2} \cap ... \cap I_{j_{k-1}} + I_i = R$ for $i \in \{1,2,...,k\}$, and $\{j_1,j_2,...,j_{k-1}\} = \{1,2,...,k\}\backslash\{i\}$. Then, choose $i \in \{1,2,...,k+1\}$ and $\{j_1,j_2,...,j_{k}\} = \{1,2,...,k+1\}\backslash\{i\}$. Let $I = I_{j_1}\cap I_{j_2} \cap ... \cap I_{j_{k-1}}$. Then we have by the induction hypothesis and assumption that $I + I_i = R$, $I + I_{j_k} = R$ and $I_i + I_{j_k} = R$. Then, by our argument in the base case for $n = 3$ we have that $I \cap I_{j_k} + I_i = R$. In particular, $I_{j_1}\cap I_{j_2} \cap ... \cap I_{j_{k-1}} \cap I_{j_k} + I_i = R$, as desired. Therefore, by mathematical induction we conclude that for all $n \geq 2$ and all $i \in \{1,2,...,n\}$, $I_{j_1}\cap I_{j_2} \cap ... \cap I_{j_{n-1}} + I_i = R$, completing the proof.
    \end{proof}
    \label{genCRTLem}
\end{lem}


\begin{cor}
    Let $R$ be a ring with ideals $I_1,I_2,...,I_n$ of $R$, for $n \geq 1$. Suppose that $I_i + I_j = R$ for all $i \neq j$. Then \begin{equation}
        \map{R\xrightarrow{\alpha} R/I_1 \times R/I_2 \times ... \times R/I_n = \bigotimes\limits_{i=1}^nR/I_i}{r \mapsto (r+I_1, r+I_2,...,r+I_n)}
    \end{equation}
    is a surjective ring homomorphism of kernel $\bigcap\limits_{i=1}^n I_i =  I_1\cap I_2 \cap ... \cap I_n$. Thus, we have an isomorphism \begin{equation}
        \overline{\alpha}: R/\bigcap\limits_{i=1}^n I_i \xrightarrow{\sim} \bigotimes\limits_{i=1}^nR/I_i
    \end{equation}
    If $R$ is commutative, we have $\bigcap\limits_{i=1}^n I_i = \prod\limits_{i=1}^n I_i$.
\end{cor}
\begin{proof}
    Let $R$ be a ring with pairwise relatively prime ideals $I_1,I_2,...,I_n$ of $R$, for $n \geq 1$. Define a map $$\map{R\xrightarrow{\alpha} R/I_1 \times R/I_2 \times ... \times R/I_n = \bigotimes\limits_{i=1}^nR/I_i}{r \mapsto (r+I_1, r+I_2,...,r+I_n)}$$
    From the proof of the Chinese Remainder Theorem $\alpha$ is a ring homomorphism. Moreover, $\ker(\alpha) = \bigcap\limits_{i=1}^nI_i$. To show surjectivity let $a_1,a_2,...,a_n \in R$. Then, for each $i \in \{1,2,...,n\}$ choose $c_i \in I_{i_1}\cap ... \cap I_{i_{n-1}}$ and $i' \in I_i$ such that $c_i + i' = 1_R$. This choice is possible by the result of Lemma \ref{genCRTLem}. We then define $a = a_1c_1 + ... + a_nc_n$. Note that for each $i \in \{1,2,..,n\}$, $c_i \equiv 0 \mod I_j$ if $j \neq i$, and $c_i \equiv 1 \mod I_i$ since $c_i + i' = 1_R$. It follows that $a \equiv 0_R + ... + a_i + ... + 0_R \equiv a_i \mod I_i$. Thus, we have that $$\alpha(a) = (a+I_1,a+I_2,...,a+I_n) = (a_1+I_1,a_2+I_2,...,a_n+I_n)$$ 
    Therefore, $\alpha$ is a surjective ring homomorphism with kernel $\ker(\alpha) = \bigcap\limits_{i=1}^nI_i$, so by the first isomorphism theorem for rings $$\overline{\alpha}: R/\bigcap\limits_{i=1}^n I_i \xrightarrow{\sim} \bigotimes\limits_{i=1}^nR/I_i$$
    Then, by the existence of this isomorphism $b \cong a_i \mod I_i$ for each $i \in \{1,2,...,n\}$ if and only if $b \cong a \mod \bigcap\limits_{i=1}^n I_i$. 
    
    
    Finally, suppose $R$ is a commutative ring. First, for each basic element of the form $i_1i_2...i_n \in \prod\limits_{i=1}^n I_i$ we have $i_1i_2...i_n \in I_j$ for each $j \in \{1,2,...,n\}$ since they are ideals. Hence, $\prod\limits_{i=1}^n I_i \subseteq \bigcap\limits_{i=1}^n I_i$. Next, write $I_1\cap I_2 \cap ... \cap I_{n-1} = I$. Then by our Lemma \ref{genCRTLem} $I + I_n = R$. It follows that \begin{align*}
        I \cap I_n &= (I \cap I_n)R \tag{Lemma \ref{idealProps}}\\
        &= (I\cap I_n)(I+I_n) \\
        &= (I\cap I_n)I + (I\cap I_n)I_n \tag{Lemma \ref{idealProps}} \\
        &\subseteq I_nI + II_n  \\
        &= II_n + II_n \tag{Commutativity of $R$} \\
        &= II_n \tag{since $0_R \in I_j, \forall j$} 
    \end{align*}
    Thus, we have that $\bigcap\limits_{i=1}^n I_i = I \cap I_n \subseteq II_n = \prod\limits_{i=1}^n I_i$. Therefore, we conclude that if $R$ is commutative, $\prod\limits_{i=1}^n I_i = \bigcap\limits_{i=1}^n I_i$.
\end{proof}

\begin{rmk}[Solving Congruences]
    Suppose we have a system of congruences \begin{equation}
        \left\{\begin{array}{l}
            x \equiv a_1 \mod m_1 \\
            x \equiv a_2 \mod m_2 \\
            \vdots \\
            x \equiv a_n \mod m_n 
            \end{array}\right.
    \end{equation}
    we can find the $c_i$ given in the above proof as follows. Write $M = m_1...m_n$ and \\$M_i = m_1...m_{i-1}m_{i+1}...m_n$. Then, note that $M_i = m_1...m_{i-1}m_{i+1}...m_n\;\vert\;c_i$ by definition of $c_i$, and that since $c_i \equiv 1 \mod m_i$, there exists $b_i \in \Z$ such that $c_i + b_im_i = 1$. Then, write $c_i = y_iM_i$, where we can solve for $y_i$ using the extended Euclidean Algorithm on $(M_i, m_i)$, or noting the inverse of $M_i$ modulo $m_i$, as they are relatively prime. 
\end{rmk}


\section{\textsection Adjunctions}

\begin{defn}[Ring Relations]
        Creating Relations in a ring $R$: Suppose we have an element $a \in R$. If we want a ring $\overline{R}$ which is an image of $R$, where $\overline{a} = 0$, then the largest such quotient is $\overline{R} = R/(a)$. If we want a ring where we have a number of relations $a_1=a_2=...=a_n=0$, we can take $(R/(a_1)/(a_2)/.../(a_n))=\overline{R}=R/(a_1,a_2,...,a_n)$. This is valid because the ideal $(a_1,...,a_n)$ contains $(a_i)$ for all $i$, and then this is successive applications of the Isomorphism Theorem.
\end{defn}

\begin{rmk}
        If $R$ is a ring and $a \in R$, if $a$ is a unit then $R/(a) = \{0\}$ since $(a) = R$. I.e$\rangle$ modding out by a unit mods out all elements of the ring.
\end{rmk}


\section{\textsection Isomorphism Theorems and Correspondence}

\begin{namthm}[First Isomorphism Theorem (for rings)]\label{thmname:isoring}
    Let $f:R\rightarrow R'$ be a ring homomorphism. Then by the First Isomorphism Theorem for groups there exists a unique group isomorphism for $(R,+)$ such that \begin{equation}
        \map{R/\ker(f) \xrightarrow{\overline{f}}f(R)}{r+\ker(f) \mapsto f(r)}
    \end{equation}
    and this is also a ring isomorphism. This theorem can be stated succinctly by the following diagram:
    \begin{center}
            \begin{tikzpicture}[baseline = (a).base]
            \node[scale = 1] (a) at (0,0){
                \begin{tikzcd}
                    R \ar[d, twoheadrightarrow, "\pi", swap] \ar[r, "\forall f"] & \forall R' \\
                    R/\ker(f) \ar[ur, dashed, "\exists!\overline{f}", swap] &
                \end{tikzcd}
            };
            \end{tikzpicture}
        \end{center}
\end{namthm}
\begin{proof}
    (Left to the reader)
\end{proof}


\begin{namthm}[Third Isomorphism Theorem]
    Let $R$ be a ring and suppose $A \subseteq B \subseteq R$ are ideals of $R$. Then $B/A$ is an ideal of $R/A$ and \begin{equation}
        (R/A)/(B/A) \cong R/B
    \end{equation}
    \begin{center}
            \begin{tikzpicture}[baseline = (a).base]
            \node[scale = 1] (a) at (0,0){
                \begin{tikzcd}
                    R \ar[r, twoheadrightarrow, "\pi_B"] \ar[d, twoheadrightarrow, "\pi_A", swap] & R/B \\
                    R/A \ar[ur, dashed, "\exists!\overline{\pi_B}"] \ar[r, twoheadrightarrow, "\pi_{B/A}", swap] & (R/A)/(B/A) \ar[u, dashed, "\exists!\overline{\overline{\pi_B}}", swap]
                \end{tikzcd}
            };
            \end{tikzpicture}
        \end{center}
\end{namthm}
\begin{proof}
    Suppose $R$ is a ring with ideals $A \subseteq B \subseteq R$. By the \ref{thmname:corrring} we know that $B/A$ is an ideal of $R/A$. Then, define a map $$\map{\phi: R/A\rightarrow R/B}{a+A \mapsto a+ B}$$
    First, suppose that $a+A = b+A$. Then we have that $a-b \in A \subseteq B$, so $a - b \in B$. Hence, $a + B = b + B$ by definition of coset equality for quotient groups, so $\phi$ is well defined. Moreover, observe that for all $a+A,c+A \in R/A$, we have $$\phi(a+A+c+A) = \phi(a+c+A) = a+c+B = a+B+c+B = \phi(a+A)+\phi(c+A)$$
    $$\phi((a+A)(c+A)) = \phi(ac+A) = ac+B = (a+B)(c+B) = \phi(a+A)\phi(c+A)$$
    and $$\phi(1_{R/A})=\phi(1_R+A) = 1_R + B = 1_{R/B}$$
    so $\phi$ is a ring homomorphism. Moreover, by construction we have that $\phi$ is surjective. Now, note that $B/A \subseteq \ker(\phi)$. Then, let $k+A \in \ker(\phi)$, so $\phi(k+A) = k+B = B$. Hence, $k \in B$ so $k+A \in B/A$. Therefore, we conclude that $\ker(\phi) = B/A$, so by the First Isomorphism Theorem for rings, \begin{equation}
        (R/A)/(B/A) \cong R/B
    \end{equation}
\end{proof}



\begin{namthm}[Correspondence Theorem (for rings)]\label{thmname:corrring}
    Let $\phi:R \rightarrow R'$ be a surjective ring homomorphism of kernel $K \subseteq R$. Then, the correspondence between subgroups of $(R',+')$ and subgroups of $(R,+)$ containing $K$ induces a bijection between ideals of $R'$ and ideals of $R$ containing $K$: 
    \begin{equation}
        \begin{array}{rcl}
            \left\{I:I\subseteq R'\;\text{an ideal}\right\} &\leftrightarrow& \left\{J:K \subseteq J \subseteq R\;\text{an ideal}\right\} \\
            I &\mapsto& \phi^{-1}(I) \\
            \phi(J) &\mapsfrom& J
        \end{array}
    \end{equation}
    are inverse bijections.
\end{namthm}
\begin{proof}
    (Left to the reader)
\end{proof}

\begin{rmk}[Warning about image of ideals]
        If $f:R\rightarrow R'$ is a ring homomorphism that is not surjective, and $J \subset R$ is an ideal, then $f(J)$ is not necessarily an ideal as well. For example, $i:\Z\hookrightarrow \Q$ is a ring homomorphism and $\Z$ is an ideal in $\Z$, but $i(\Z)$ is not an ideal in $\Q$ since $\Q$ is a field with only trivial ideals.
\end{rmk}