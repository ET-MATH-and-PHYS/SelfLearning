%%%%%%%%%% EDs %%%%%%%%%%
\chapter{\textsection\textsection Euclidean Domains}

\section{\textsection Basic Definitions and Examples: Euclidean Domains}

\begin{defn}[Norm]
    For an integral domain $R$, any function $N:R\rightarrow \Z^+\cup\{0\}$ with $N(0_R) = 0$ is called a \Emph{norm} on the integral domain $R$. If $N(a) > 0$ for $a \neq 0$ define $N$ to be a \Emph{positive norm}.
\end{defn}

\begin{defn}[Euclidean Domain]
    The integral domain $R$ is said to be a \Emph{Euclidean Domain} (or posses a \Emph{Division Algorithm}) if there is a norm $N$ on $R$ such that for any two elements $a,b \in R$, with $b \neq 0_R$, there exist elements $q,r \in R$ with $$a= qb+r$$ where $r = 0_R$ or $N(r) < N(b)$. The element $q$ is called the \Emph{quotient} and the element $r$ the \Emph{remainder} of the division.
\end{defn}

\begin{eg}
    \leavevmode
    \begin{enumerate}
        \item Fields are trivial examples of Euclidean Domains where any norm will satisfy the defining condition. (e.g., $N(a) = 0,\forall a$) This is because for all $a,b$, $b\neq 0$, we have $a = qb+0$, where $q = ab^{-1}$.
        \item The integers are a Euclidean Domain with norm $N(a) = |a|$, the usual absolute value.
        \item If $F$ is a field, the polynomial ring $F[x]$ is a Euclidean Domain with norm $N(p(x)) = \deg(p(x))$
        \item The Gaussian integers, $\Z[i]$, is a Euclidean domain with norm $N(a+bi) = a^2+b^2$
    \end{enumerate}
\end{eg}


\begin{prop}
    Every ideal in a Euclidean Domain is principal. In particular, if $I$ is any nonzero ideal in the Euclidean Domain $R$, then $I = (d)$ for $d$ any nonzero element of $I$ of minimal norm.
\end{prop}
\begin{proof}
    If $I$ is the zero ideal there is nothing to prove. Otherwise, let $d$ be a nonzero element of $I$ of minimum norm. Such a $d$ exists since the set $\{N(a):a \in I\}$ is a nonempty subset of $\Z$, which is bounded below, and hence has a minimum element by the well-ordering of $\Z$. Clearly $(d) \subseteq I$ since $d \in I$. To show the reverse inclusion let $a$ be any element of $I$, and use the division algorithm to write $a = qd + r$ for $q,r \in R$, with $N(r) < N(d)$. Then, since $I$ is an ideal $-qd \in I$, so $r = a-qd \in I$. Thus, as $d$ is an element of minimal norm in $I$, so we must have that $r = 0$. Thus $a = qd \in (d)$, showing $I = (d)$.
\end{proof}

\begin{rmk}
    This proposition can be used to prove that some integral domains $R$ are not Euclidean Domains if they have non-principal ideals.
\end{rmk}

\begin{defn}
    Let $R$ be a commutative ring and let $a,b \in R$ with $b \neq 0$. \begin{enumerate}
        \item $a$ is said to be a \Emph{multiple} of $b$ if there exists an element $x \in R$ with $a = bx$. In this case $b$ is said to \Emph{divide} $a$ or be a \Emph{divisor} of $a$, written $b\;\vert\;a$
        \item A \Emph{greatest common divisor} of $a$ and $b$ is a nonzero element $d \in R$ such that \begin{enumerate}
                \item $d\;\vert\;a$ and $d\;\vert\;b$, and 
                \item if $d'\;\vert\;a$ and $d'\;\vert\;b$ then $d'\;\vert\;d$.
        \end{enumerate}
    \end{enumerate}
    A greatest common divisor of $a$ and $b$ is denoted $\gcd(a,b)$.
\end{defn}


\begin{rmk}
    Translating into the language of ideals, if $I = (a,b) \subseteq R$, then $d$ is the greatest common divisor of $a$ and $b$ if \begin{enumerate}
        \item $I \subseteq (d)$, and
        \item if $I \subseteq (d')$, then $(d) \subseteq (d')$
    \end{enumerate}
    Hence, a greatest common divisor for $a$ and $b$ (if one exists) is a generator for the unique smallest principal ideal containing $a$ and $b$.
\end{rmk}


\begin{prop}
    If $a,b$ are nonzero elements in the commutative ring $R$ such that the ideal generated by $a$ and $b$ is a principal ideal $(d)$, then $d$ is the greatest common divisor of $a$ and $b$.
\end{prop}

\begin{defn}
    An integral domain in which every ideal $(a,b)$ generated by two elements is principal is called a \Emph{Bezout Domain}. Note that Bezout Domain's can have non-principal ideals.
\end{defn}

\begin{prop}
    Let $R$ be an integral domain. If two elements $d$ and $d'$ of $R$ generate the same principal ideal, i.e. $(d) = (d')$, then there exists $u \in R^{\times}$ such tha $d = ud'$. $d$ and $d'$ are called \Emph{associates} in this case.
\end{prop}

\begin{thm}
    Let $R$ be a Euclidean Domain and let $a,b \in R$ be nonzero elements of $R$. Let $d = r_n$ be the last nonzero remainder in the Euclidean Algorithm for $a$ and $b$. Then \begin{enumerate}
        \item $d = \gcd(a,b)$, and 
        \item $(d) = (a,b)$, so there exist $x,y \in R$ such that \begin{equation}
                d = ax + by
        \end{equation}
    \end{enumerate}
\end{thm}
\begin{proof}
    (Left to the reader)
\end{proof}

\begin{defn}
    Let $\widetilde{R} := R^{\times}\cup \{0\}$. An element $u \in R - \widetilde{R}$ is called a \Emph{universal side divisor} if for every $x \in R$ there is some $z \in \widetilde{R}$ such that $u$ divides $x-z \in R$.
\end{defn}


\begin{prop}
    Let $R$ be an integral domain that is not a field. If $R$ is a Euclidean Domain then there are universal side divisors in $R$.
\end{prop}
\begin{proof}
    (Left to the reader)
\end{proof}