%%%%%%%%%% PIDs %%%%%%%%%
\chapter{\textsection\textsection Principal Ideal Domains}

\section{\textsection Basic Definitions and Examples: PIDs}

\begin{defn}
    An integral domain of which every ideal is principal is called a \Emph{principal ideal domain} (\Emph{PID}).
\end{defn}

\begin{rmk}
    $\Z$ is a prototypical example of a PID. Additionally, for a field $\F$, $\F[X]$ is also a PID.
\end{rmk}

\begin{thm}
    Let $\F$ be a field. Then $\F[X]$ is a principal ideal domain. Moreover, every non-zero ideal $I$ of $\F[X]$ is generated by the monic polynomial of lowest degree in $I$.
\end{thm}
\begin{proof}
    Note that since $\F$ is an integral domain so is $\F[X]$. Consider an ideal $I \subseteq \F[X]$. If $I = \{0\} = (0)$ we are done, so assume $I \neq \{0\}$ and that $0 \neq g \in I$ is of minimal degree. We can assume that $g$ is monic; if $a$ is the leading coefficient of $g$, then $a^{-1} \in \F$ so $a^{-1}g$ is monic, and we replace $g$ with $a^{-1}g \in I$ since $I$ is an ideal. We claim that $I = (g)$. Since $g \in I$ and $I$ is an ideal we have $g\F[X] = (g) \subseteq I$. Let $P \in I$. By the division algorithm there exist unique $q,r \in \F[X]$ such that $P = qg + r$ where $r = 0$ or $\deg(r) < \deg(g)$ since the leading coefficient of $g$ is a unit. Then $r = P - qg \in I$ since $qg \in I$ as it is an ideal. Then $r = 0$, since otherwise $\deg(r) < \deg(g)$, contradicting the minimality of $g$'s degree. Thus, $P = qg \in (g)$, so $I \subseteq (g)$. Therefore $I = (g)$ and $\F[X]$ is a PID as claimed.
\end{proof}